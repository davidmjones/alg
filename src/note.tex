%% This work is licensed under a Creative Commons
%% Attribution-NonCommercial 4.0 International License.
%% http://creativecommons.org/licenses/by-nc/4.0/
%% 
%% David M. Jones, July 2016

\Unnumbered{Editor's Note}
\thispagestyle{dropfolio}
\markboth{Editor's Note}{Editor's Note}

\vskip 1em

Hale and Buck's \emph{A Latin Grammar} was first published by Ginn and
Company in 1903.  This edition is a collation of the two different
versions of the original that I am aware of, hereafter referred to as
versions A and~B\@.  Scans of both versions as well as other technical
details are available at \url{https://github.com/davidmjones/alg}.

It seems clear that version~A represents the earlier state of the
text.  Excluding changes to the index, which I have not attempted to
collate exhaustively, version~B differs from version~A in
approximately 60 passages that were evidently modified between
printings with, unfortunately, no corresponding changes to the title
or copyright pages.

Notwithstanding its later date, I don't believe that B always
represents the authors' preferred text.  In many cases, text from
version~A seems to have been deleted merely in order to make room for
necessary emendations elsewhere on the page.  In such cases, I have
retained the more complete version.  A complete list of
\hyperref[variations]{Variations in the Text} is provided for
reference, from which it should be possible to exactly reconstruct the
reading of either version.

Also included is a list of \hyperref[emendations]{Emendations to the
  Text}, documenting a small number of obvious typos or
inconsistencies in punctuation that I have corrected.

My other innovation has been an attempt to provide links to the
\href{http://www.perseus.tufts.edu/}{Perseus Digital Library} for all
quotes from the classical literature.  By my count, \emph{A Latin
  Grammar} cites nearly 1650 distinct passages from 113 works by 21
different classical authors, often only by an abbreviated version of
the title.  I have added two indexes of this material.  The first, the
\hyperref[abbreviations]{Index of Abbreviations}, attempts to identify
every classical source cited by full title.  The second, the
\hyperref[passages]{Index of Passages Cited}, lists each passage and
identifies where in the text it is cited.

Finally, \emph{caveat lector!}  I am neither a classicist nor a
trained scholar of any flavor nor anything but the rankest tyro at the
Latin language.  Although I have tried to ensure that this edition is
free from errors, it is too much to hope that none remain, especially
in the identification of passages cited.  Please report any errors you
find at the website above, and I will attempt to incorporate them into
the text in timely manner.

\noindent
\begin{tabular*}{\textwidth}{@{}l@{\extracolsep{\fill}}r@{}}
& David M. Jones.\\
&\textsc{\small July, 2016.}
\end{tabular*}

\endinput
