%% This work is licensed under a Creative Commons
%% Attribution-NonCommercial 4.0 International License.
%% http://creativecommons.org/licenses/by-nc/4.0/
%% 
%% David M. Jones, July 2016

\Unnumbered{Variations in the Text}
\label{variations}
\markboth{Variations in the Text}{Variations in the Text}
\markthird{}

\thispagestyle{dropfolio}

\contentsentry{B}{Variations in the Text}

All of the variations between versions A and~B of the text are listed
below, but they are described with respect to the wording chosen for
the current edition.  For example, in the first item, it is understood
that the paragraph referred to was added in version~B.

\begin{variations}

\item[p.~\pagelink{vi}, ll.~7–10]

Version~A lacks the paragraph beginning “The views upon.”

\item[p.~\pagelink{vi}, l.~33]

Version~A lacks “proof-reading and suggestions, and also.”

\item[{\xref[3.]{100}, l~2}]

Version~A lacks “Similarly (rarely), \latin{diī} for \latin{diēī}.”

\item[\xref{122}, l.~11]

Version~A has “(\latin{iuvenior} late)” in place of “[\latin{minor
    nātū}]” and “\latin{nātū minimus}” in place of “\latin{minor
  nātū}.”

\item[\xref{122}, l.~12]

Version~A lacks “[\latin{maior nātū}]” and reads “\latin{nātū
  maximus}” in place of “\latin{maximus nātū}.”

\item[{\xref[\emph{a}]{22}, l.~2}]

Version~A lacks “, and the auxiliary \latin{est} (\latin{sunt},
etc.).”

\item[{\xref[\emph{b}]{234}, l.~2}]

Version~A lacks “and \apud{}{1, 308}.”

\item[{\xref[5, \emph{c}]{240}}]

Version~A lacks this paragraph.

\item[{\xref[5, \emph{d}]{240}, ll.~2–3}]

Version~B omits the sentence “Such \emph{intermediate} (or
\emph{semi\hyphenbreak abstract}) nouns are usually classed as Concrete.”

\item[{\xref[3, \emph{a}]{264}}]

Version~A reads
\begin{quote}
Similarly \latin{aliēnus}, \english{belonging to another}, gains the
meaning \english{unfavorable}.  Thus \latin{aliēnō locō}, \english{in
  an unfavorable place}; \apud{B.~G.}{1, 15, 2}.
\end{quote}
Version~B reads
\begin{quote}
Similarly \latin{noster}, \english{our}, may have the meaning
\english{favorable}, and \latin{aliēnus}, \english{belonging to
  another}, the meaning \english{unfavorable}.
\end{quote}

\item[{\xref[3]{274}}]

Version~A reads
\begin{quote}
\latin{Is} or \latin{is quidem}, in combination with various
connectives (\latin{et is}, \latin{atque is}, \latin{isque}, \latin{et
  is quidem}, \latin{nec is}, \latin{neque is}, etc.), is used\ellipsis
\end{quote}

Version~B reads
\begin{quote}
\latin{Is} or \latin{is quidem}, and \latin{ille} or \latin{ille
  quidem}, in combination with various connectives (\latin{et},
\latin{atque}, \latin{nec}, etc.), are used\ellipsis
\end{quote}

\item[{\xref[1]{284}}]

Version~B omits the example beginning “\latin{sunt hūmānissimī}” and
the\linebreak
phrase “(Indefinite Antecedent.)”\ at the end of the second
example.

\item[{\xref[\emph{a}]{284}}]

Version~A lacks this entire subsection.

\item[{\xref[2]{284}}]

Version~B omits the example beginning “\latin{habētis quam}.”

\item[{\xref[3]{284}}]

Version~A reads
\begin{quote}
(in English idiom) \english{the bridge at Geneva};
\end{quote}
in place of Version~B's
\begin{quote}
\english{the bridge \emph{(which was)} at Geneva} (in English idiom,
\english{the bridge at Geneva});
\end{quote}

\item[{\xref[6]{284}}]

Version~B omits “the clause containing the Antecedent.”

\item[{\xref[\emph{c}]{295}}]

Version~A lacks “The poets extend the list.”

\item[{\xref{327}, ll.~1–2}]

Version~A reads
\begin{quote}
The Romans avoided putting an Appositive word directly before a
Relative, preferring to attach it \emph{to the Relative itself}.
\end{quote}

\item[{\xref[\emph{c}]{346}, ll.~4–5}]

Version~B omits the second example (“\latin{quōs omnīs}”);
Version~A lacks the third example (“\latin{reliquīs Gallīs}”).

\item[{\xref[\emph{a}, 2]{352}, l.~2}]

Version~A lacks “(\latin{Miseror} takes the Accusative.)”

\item[{\xref[\emph{c}]{354}, ll.~5, 7}]

Version~B omits the third (“\latin{poenae sēcūrus}”) and fifth
(“\latin{ēreptae virginis īrā}”) examples.

\item[{\xref[\emph{d}]{354}}]

Version~A lacks this entire subsection.

\item[{\xref[\emph{a}]{361}, ll.~1–2}]

Version~A reads
\begin{quote}
\textbf{Later Freer Dative of the Concrete Object for Which.}  The
poets and later writers use the construction of the Concrete Object
more boldly, even attaching it directly to nouns.
\end{quote}

Version~B reads
\begin{quote}
\textbf{Later Freer Dative of the Object for Which.}  The poets and
later writers use the construction of the Object for Which more
boldly, even attaching it directly to nouns.
\end{quote}

\item[{\xref[5]{364}, l.~3}]

Version~A lacks “Similarly \latin{aequō} in poetry.”

\item[{\xref[III]{387}, footnote, l.~2}]

Version~A lacks “The same use appears with \latin{ecquid}, \latin{sī
  quid}, and \latin{nē quid}.”

\item[{\xref[\emph{a}]{388}, note}]

Version~A reads
\begin{quote}
From such combinations arose the free use of \latin{quid} in the sense
of \english{why}, as in \latin{quid tacēs?} \english{why are you
  silent?} \apud{Cat.}{1, 4, 8}.
\end{quote}

Version~B reads
\begin{quote}
Hence arose the use of \latin{quid} in the sense of \english{why}, and
of \latin{quod} in phrases like \latin{quod sī}, \english{but if}
(touching which matter, if).
\end{quote}

\item[{\xref[\emph{a}]{426}, footnote}]

Version~A lacks “\latin{pelagō}.”

Version~B omits “\latin{stagnō}.”

\item[{\xref[\emph{a}]{431}}]

Version~A reads “occasionally” in place of Version~B's “may also.”

\item[{\xref[3, \emph{b}]{438}, l.~1}]

Version~A reads “See~\xref[4, \emph{b}]{406}.”

\item[{\xref[1]{464}, footnote, l.~1}]

Version~A lacks “(in the finite verb).”

\item[{\xref[2]{467}, ll.~1–2}]

Version~A reads
\begin{quote}
An act can be thought of as a whole only if looked at \emph{without}
reference to any particular time.  Hence the aoristic tenses are
\emph{Absolute}.
\end{quote}

Version~B reads
\begin{quote}
An act thought of as a whole (i.e.\ aoristically) may be looked at
either without, or with, reference to a particular time, i.e.\ either
\emph{Absolutely} or \emph{Relatively}.
\end{quote}

In addition, Version~A lacks all of \xref[2, \emph{a}]{467}.

\item[{\xref[4, \emph{a}]{470}, l.~2–3}]

Version~A lacks “(generally).”  Version~B omits “in” three times:
before “Consecutive,” before “Causal-Adversative,” and before
“\latin{quīn}-Clauses.”

\item[{\xref[\emph{b}]{477}}]

Version~A reads
\begin{quote}
The relative tenses of the Indicative all express \emph{situation}.
So do the relative tenses of the Subjunctive, when used with the same
force as the corresponding tenses of the Indicative.  When used with
future force, they may express either the idea of future (or
subsequent) \emph{situation}, or a mere \emph{aoristic} idea for
future (or subsequent) time.
\end{quote}

Version~B reads
\begin{quote}
The relative tenses of the Indicative all express \emph{situation};
the aoristic tenses of the Indicative do not (\xref[2,
  \emph{a}]{467}).

The Subjunctive tenses, when used with relative force, may express
either the idea of situation, or the aoristic idea.  Thus, either a
situation, or an act seen in summary, may be put as relatively future
to a past time.
\end{quote}

\item[{\xref{481}, column~2, l.~2}]

Version~A reads “for the reason that” in place of Version~B's “(I have
written) because.”

\item[{\xref{487}}]

Version~A reads
\begin{quote}
In several verbs the Present Perfect, Past Perfect, and Future Perfect
have come to express a present, past, or future \emph{state}.  Thus
\latin{nōvī}, (\english{have learned}) \english{know},
\latin{nōveram}, \english{knew}, \latin{nōverō}, \english{shall know},
\latin{cognōvī}, \english{know}, \latin{cōnsuēvī}, (\english{have
  formed the habit}) \english{am in the habit}, \latin{meminī},
(\english{have recollected}) \english{remember}, \latin{ōdī},
(\english{have come to dislike}) \english{hate}. Similarly
  \latin{coepī}, \english{begin}.
\end{quote}

Version~B reads
\begin{quote}
In several verbs the Present Perfect, Past Perfect, and Future Perfect
have come to express a present, past, or future \emph{state}.  Thus
\latin{nōvī}, (\english{have learned}) \english{know},
\latin{cōnsuēvī}, \english{am accustomed}, \latin{meminī},
\english{remember}, \latin{ōdī}, \english{hate}, \latin{coepī},
\english{begin}, etc.  Similarly, sometimes, in other verbs.  Thus
\latin{cōnstiterant}, \english{had taken their stand}, = \english{were
  standing}; \apud{B.~G.}{1, 24, 3}.
\end{quote}

\item[{\xref[3, \emph{a})]{502}, footnote~4}]

Version~A reads “these constructions” in place of Version~B's
“substantive clauses” and lacks “(likewise in clauses of purpose).”
Version~B omits “Thus \latin{ut nē sit impūne}, \apud{Mil.}{12, 31}.”

\item[{\xref[1]{507}}]

In the second example (“\latin{nunc est ille diēs}\ellipsis”), Version~B
omits “Similarly, though in indirect discourse, \latin{diem quō
  condant}, \apud{Aen.}{7, 145}.”

Version~B shortens the third example to
\begin{quote}
\latin{nāscētur Troiānus, fāmam quī terminet astrīs},
\english{there will be born a Trojan, who shall \emph{(prophetic, =
    will)} make the stars the boundary of his fame};
\apud{Aen.}{1, 286}.  (A Trojan of what kind?  A Trojan that
shall.\dots\ Cf.\ \latin{quae verteret}, expressing a \emph{past}
Anticipation, \apud{Aen.}{1, 20}.)
\end{quote}

Version~B omits the example beginning “\latin{venient annīs saecula}.”

\item[{\xref[\emph{a}]{507}}]

Version~A has the phrase “almost completely” between “has” and
“driven” instead of after “Subjunctive.”

\item[{\xref[2]{507}, ll.1–2}]

Version~A reads “In \term{Substantive Clauses} with \latin{ut}, after
verbs of \emph{expecting}.”

\item[{\xref[2, \emph{b}]{507}}]

Version~A lacks this subsection.

\item[{\xref[4, \emph{b}]{507}, note}]

Version~A reads
\begin{quote}
Since an event forestalled is one which the main actor tries to make
\emph{impossible}, the Anticipatory Subjunctive of \latin{possum}
(with the Infinitive) is sometimes used in this construction (as in
\apud{B.~G.}{6, 3, 2}, \latin{priusquam convenīre possent}), in place
of the simple verb in the Subjunctive (\latin{priusquam convenīrent}).
\end{quote}

\item[{\xref[4, \emph{d}]{507}, l.~5}]

Version~A lacks “Cf.\ \latin{prius quam ut}, \apud{Lig.}{12, 34}.”

\item[{\xref[1, \emph{c}]{519}, l.~2}]

Version~A reads “Substantive Volitive Clause” in place of Version~B's
“an Infinitive or Volitive Clause.”

\item[{\xref[3, \emph{b}]{521}, l.~2}]

Version~A lacks “, or imply a negative.”

\item[{\xref[\emph{a}]{535}, note~3}]

Version~A lacks note~3 (but see the next item).

\item[{\xref[\emph{b}]{535}}]

Version~A contains the following note here:
\begin{quote}
\begin{note}[Note]

By a natural confusion, \latin{dīcō} is sometimes put in the
Subjunctive in a \latin{quod}-Clause of Reason.
\begin{examples}

\latin{rediit quod sē oblītum nesciō quid dīceret},
\english{he came back, because he said he had forgotten something}
(properly \latin{quod oblītus esset}, \english{because}, as he said,
\english{he had forgotten});
\apud{Off.}{1, 13, 40}.
Similarly \latin{quod exīstimārent}; \apud{B.~G.}{1, 23, 3}.

\end{examples}

\end{note}
\end{quote}

\item[{\xref[\emph{a}]{550}, l.~6}]

Version~A lacks “(See also \xref[\emph{d}]{524}.)”

\item[{\xref{555}, footnote~1, l.~1}]

Version~A lacks “\latin{id maesta est},”.

\item[{\xref{558}, ll.~1–2}]

Version~A lacks “or \latin{simul atque},” but has “(the less common
usage).”

\item[{\xref{577}, ll.~2–3}]

Version~B omits “In what follows, the two kinds will be treated
together.”

\item[{\xref[\emph{a}]{577}}]

Version~A lacks this subsection.

\item[{\xref[\emph{e}]{585}}]

Version~A reads
\begin{quote}
Verbs or phrases of this class suggesting that the action is
\emph{wanted} or \emph{urged} may also take a Volitive Substantive
Clause (\xref[3, \emph{c}]{502}, and lists).
\end{quote}

Version~B reads
\begin{quote}
Verbs or phrases of this class suggesting that the action is
\emph{wanted} or \emph{urged} may also take a Subjunctive Substantive
Clause.
\end{quote}

\item[{\xref[3]{598}}]

Version~A reads
\begin{quote}
The later poets use the Infinitive occasionally as a Substantive with
a Verb, or after certain Prepositions governing the Accusative.
\end{quote}

Version~B reads
\begin{quote}
The later writers, especially the poets, use the Infinitive
occasionally as a mere Substantive depending upon a Verb, or in the
Accusative after certain prepositions.
\end{quote}

\item[{\xref[2]{605}, l.~2}]

Version~A lacks “(and many others in poetry)”.

\item[{\xref[III]{625}}]

Version~A reads “the separation of connected words” in place of
Version~B's “postponement”.

\item[{\xref[11]{623}}]

Version~B omits “till the end of the sentence is reached”.

\item[{\xref[12]{623}}]

Version~A reads
\begin{quote}
\english{if anything} (for \english{any disaster}) \english{should
  happen to the Romans}
\end{quote}

Version~B reads
\begin{quote}
\english{if anything should happen to the Romans} (instead of
\english{if they should be defeated})
\end{quote}

\item[{\xref[14]{623}, ll.~4–5}]

Version~B shortens the parenthetical explanation to
\begin{quote}
(The ship is the state, the billows the civil wars, etc.)
\end{quote}

\item[{\xref[20]{623}}]

Version~B replaces “use of words the sound of which corresponds with
the thing signified” by “matching of sound to sense”.

\item[{\xref[21]{623}}]

Version~A lacks the definition of \term{Figūra Etymologica}.

\item[{\xref[2]{652}}]

Version~A lacks “, especially in the caesura”.

\item[Index, “Absolute tenses”]

Version~A has “\xref[2]{467}, \xref{478}”.

\item[Index, “Deprecated act”]

Version~A has “in subj.\ w.\ \latin{antequam}, etc., \xref[4,
  \emph{d}\)]{507}.”

\item[Index, “\latin{dīcō}”]

Version~A has “\xref[2, \emph{b}, n.]{535}” in place of Version~B's
“\xref[2, \emph{a}, n.~3]{535}”.

\item[Index, “Tenses”]

In the last line, Version~A has “\xref{494}” in place of Version~B's
“\xref[\emph{a}]{577}”.

\end{variations}

\endinput
