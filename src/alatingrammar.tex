%% This work is licensed under a Creative Commons
%% Attribution-NonCommercial 4.0 International License.
%% http://creativecommons.org/licenses/by-nc/4.0/
%% 
%% David M. Jones, July 2016

\documentclass{alg}

\usepackage{algurls}

\hypersetup{%
  pdfinfo={
      Title={A Latin Grammar},
      Author={William Gardner Hale and Carl Darling Buck},
  }
}

\batchmode

%\nofiles

\begin{document}

\frontmatter

% \null\thispagestyle{empty}\clearpage\setcounter{page}{1}

\begin{titlepage}

\begin{center}

{\huge A}

\bigskip

\bigskip

{\Huge LATIN GRAMMAR}

\vfill

\begin{scshape}

by

\bigskip

{\large William Gardner Hale} % 9 Feb 1849 – 23 Jun 1928

\smallskip

{\small Professor of Latin in the University of Chicago}

\bigskip

{\footnotesize and}

\bigskip

{\large Carl Darling Buck} % 2 Oct 1866 – 8 Feb 1955

\smallskip

{\small Professor of Comparative Philology in the\\ University of Chicago}

\end{scshape}

\end{center}

\vfill
\vfill

\noindent
This work is licensed under a
\href{http://creativecommons.org/licenses/by-nc/4.0/}{Creative Commons
  Attribution-NonCommercial 4.0 International License}.

\clearpage

\thispagestyle{empty}

\null

\vfill

\begingroup
\parindent0pt

\textbf{Editor}

\noindent
David M. Jones

\bigskip

\textbf{Revision History}

Version 1.00: July 13, 2016

\bigskip

\emph{For additional materials, see}
\url{https://github.com/davidmjones/alg}.

\endgroup

\vfill

\vfill

\end{titlepage}

%% This work is licensed under a Creative Commons
%% Attribution-NonCommercial 4.0 International License.
%% http://creativecommons.org/licenses/by-nc/4.0/
%% 
%% David M. Jones, July 2016

\Unnumbered{Editor's Note}
\thispagestyle{dropfolio}
\markboth{Editor's Note}{Editor's Note}

\vskip 1em

Hale and Buck's \emph{A Latin Grammar} was first published by Ginn and
Company in 1903.  This edition is a collation of the two different
versions of the original that I am aware of, hereafter referred to as
versions A and~B\@.  Scans of both versions as well as other technical
details are available at \url{https://github.com/davidmjones/alg}.

It seems clear that version~A represents the earlier state of the
text.  Excluding changes to the index, which I have not attempted to
collate exhaustively, version~B differs from version~A in
approximately 60 passages that were evidently modified between
printings with, unfortunately, no corresponding changes to the title
or copyright pages.

Notwithstanding its later date, I don't believe that B always
represents the authors' preferred text.  In many cases, text from
version~A seems to have been deleted merely in order to make room for
necessary emendations elsewhere on the page.  In such cases, I have
retained the more complete version.  A complete list of
\hyperref[variations]{Variations in the Text} is provided for
reference, from which it should be possible to exactly reconstruct the
reading of either version.

Also included is a list of \hyperref[emendations]{Emendations to the
  Text}, documenting a small number of obvious typos or
inconsistencies in punctuation that I have corrected.

My other innovation has been an attempt to provide links to the
\href{http://www.perseus.tufts.edu/}{Perseus Digital Library} for all
quotes from the classical literature.  By my count, \emph{A Latin
  Grammar} cites nearly 1650 distinct passages from 113 works by 21
different classical authors, often only by an abbreviated version of
the title.  I have added two indexes of this material.  The first, the
\hyperref[abbreviations]{Index of Abbreviations}, attempts to identify
every classical source cited by full title.  The second, the
\hyperref[passages]{Index of Passages Cited}, lists each passage and
identifies where in the text it is cited.

Finally, \emph{caveat lector!}  I am neither a classicist nor a
trained scholar of any flavor nor anything but the rankest tyro at the
Latin language.  Although I have tried to ensure that this edition is
free from errors, it is too much to hope that none remain, especially
in the identification of passages cited.  Please report any errors you
find at the website above, and I will attempt to incorporate them into
the text in timely manner.

\noindent
\begin{tabular*}{\textwidth}{@{}l@{\extracolsep{\fill}}r@{}}
& David M. Jones.\\
&\textsc{\small July, 2016.}
\end{tabular*}

\endinput


\Unnumbered{Preface}
\thispagestyle{dropfolio}
\markboth{Preface}{Preface}

\vskip 1em

This grammar aims to be a working text-book, primarily adapted to the
needs of high school students.

The part which deals with Sounds, Inflection, and Word-Formation was
written by the junior author, who is also mainly responsible for
matters of orthography, hidden quantity, etc., throughout the book;
the part which deals with Syntax, Word-Order, Versification, the
Calendar, etc., and the suggestions with regard to Pronunciation in
\xref{35}–\xref{40}, were written by the senior author; but both parts
have been worked over carefully and in detail by both authors.

In the Phonology, Inflection, and Word-Formation, the authors have
been conservative in the introduction of matters of comparative
grammar. In general they have aimed to give only such historical
explanations as are certain and reasonably simple, and deal with the
relations between existing Latin forms, not with the relations between
a Latin form and one of another language. For example, the statement
that original final~\phone{i} became~\phone{e} (\xref[3]{44}) would
not be made, if it were useful only in understanding the relation
between Latin \latin{ante} and Greek ἀντί, but is introduced
because, aside from the existence of the original form in
\latin{anti-cipō}, it explains why the Nominative-Accusative
Singular Neuter of an \phone{i}-stem (e.g. \latin{mare}) ends
in~\latin{e}.

While, then, only a limited amount of historical grammar has been
included, pains have been taken to frame whatever statements are made
as to the relations of forms in the light of our knowledge of the
actual historical development, so that, while not always expressed or
arranged in the way one would adopt in a strictly historical grammar,
they may serve as a sound foundation for possible further study,
instead of fostering wrong conceptions which must be overcome later.

Questions of pronunciation, hidden quantity, orthography, etc., have
received careful and independent study, though space does not permit
the presentation of the arguments in favor of the views
adopted. Departures from the usual practice in such matters may cause
some temporary difficulty to the teacher; but this cannot justify the
authors in perpetuating what they believe to be errors.

No attempt is made to treat early Latin fully, but some of its most
striking peculiarities are mentioned.

In the Syntax, the probable relationships of the constructions treated
are indicated by the arrangement. Where this is not of itself
sufficient, and the origin of the construction is easy to understand,
a brief explanation is added, as of the Subjunctive in Generalizing
Clauses in the Second Person Singular Indefinite
(\xref[2,~\emph{a}]{504}). Where the explanation is more difficult, or would
demand too much space (as of the origin of the Subjunctive of
Actuality, or of the Historical Infinitive), nothing is said. This
last statement applies in general to the constructions of composite
origin (illustrated in \xref[3]{315}).

In the treatment of the verb, subordinate clauses have been put with
the independent constructions to which they stand related; for their
essential nature is thus best understood, while the demand made upon
the memory is reduced. Where contrasting constructions with another
mood exist, cross-references are given.

The constructions dealt with have been treated in as brief and simple
a manner as is consistent with the actual facts of usage; but it has
not been thought that mere omission necessarily makes the student’s
work easier.  Indeed, the \emph{addition} of categories will at a number of
points be found to make for simplicity. Thus the new category
Subjunctive of Obligation or Propriety (\xref{512}) at once illumines
such an example as \latin{quid tē invītem}, \english{why should I
  urge you?} Cic.\ Cat.\ 1, 9, 24, which formerly had to be forced
under the Subjunctive of Deliberation, though there is no shade of
deliberation in it. In the same way, the clear recognition
(\xref{571}) of a use of the Present Indicative with powers
corresponding to those of a number of constructions in other moods or
tenses will lighten the difficulties of any thoughtful teacher or
student. Thus the Present Indicative after \latin{antequam}, which
Cicero uses in the Orations more than twice as frequently as the
Subjunctive, the Present Indicative in a clearly future condition with
\latin{sī}, as in Cic.\ Cat.~2, 5, 11, and the Present Indicative in
questions like \latin{quid agō}, Aen.~4,~534, now become
intelligible; and the student will not have to warp his grammatical
conscience with the old explanation that, in all these cases, the act
is practically “now going on.”

The field covered is the syntax actually found in high school Latin,
with the addition of a comparatively small number of constructions,
which were necessary for the general skeleton of the treatment. Our
Latin grammars, even the shorter ones, have included much that does
not occur at all in high school Latin, and much (as, e.g., the
\latin{id genus} idiom) that is either rare or non-occurrent in the
Latin ordinarily read in colleges. On the other hand, much in the way
of special idiom that does occur in the ordinary college Latin has
been omitted from our grammars.  It has seemed best to the present
authors to reserve all such constructions for a
Supplement,\footnote{The Supplement will also contain explanations of
  the origin of all constructions which need special explanation,
  discussions of the more difficult distinctions, and of certain
  constructions which pre\-sent peculiar difficulty in the high school
  Latin, together with fuller illustrations, both from this Latin and
  from that which is read in colleges.  In addition, it will contain
  further notes on pronunciation, word-order, and versification.} to
follow this book. This Supplement will be at the service of the
teacher, whether teaching in school or in college. The college teacher
may choose to put it into the hands of his students, or may merely use
it as a book of reference. In any case, however, students who are
familiar with the constructions and principles explained in the
present book will have no difficulty in making their way through
college Latin.

Citations are given for all the examples taken from actual Latin, and
no change is made in any of them except the occasional omission of
parts not bearing upon the construction under treatment. The subject
is often omitted where it has nothing to do with the construction to
be illustrated. Wherever the Latin read in the high school affords a
short and satisfactory example, that example has been used; and the
proportion of such examples will be found to be unprecedentedly
large. Other examples have here and there been chosen as simpler, or
as affording parallels in a series (e.g., in \xref{362}), or as
matching better in the exposition of allied or contrasting
constructions (as in \xref{355}, \xref{356}, \xref[3]{582}). But the
works thus necessarily drawn upon outside of the high school Latin are
in many instances represented by only a single example

Latin usage was of course a matter of constant growth and change. The
ordinary division into early, Augustan, and post-Augustan usage is
unserviceable. After Cicero, the most rapid changes take place in
Sallust (who forms an especial turning-point), Virgil, Horace, Ovid,
and Livy,—all belonging to the Augustan period. The division here
made is into early Latin, Ciceronian Latin,\footnote{The statements
  with regard to Ciceronian Latin are based upon the orations and the
  philosophical works, for which alone complete lexicons exist.
  Occasional exceptions or additions will doubtless require to be made
  when the rhetorical works and the letters are taken into account.}
and later Latin (see Table of Authors Cited,
p.~\pageref{authors_cited}); but it must be remembered that Lucretius
and Catullus, who belong to the Ciceronian age, are occasional
innovators.—Where the phrase “poetic Latin” or the word “poetry”
is used, it is intended to cover Plautus and Terence as well as the
later poets.

The authors have allowed themselves the use of certain comparatively
new forms of grammatical terminology, classification, or statement,
which they believe to be helpful, as well as scientifically sound.
Among these are the following: the subdivisions Volitive Subjunctive,
Anticipatory Subjunctive, Subjunctive of Obligation or Propriety,
Subjunctive of Ideal Certainty; the solution of the Subjunctive with
\latin{dum}, \latin{dōnec}, \latin{quoad}, \latin{antequam}, and
\latin{priusquam} as Anticipatory; the distinctions and phrases Act
Anticipated and Prepared for, Anticipated and Forestalled, Anticipated
and Deprecated; the phrases Determinative Clause, Volitive Substantive
Clause, Optative Substantive Clause, etc., \latin{cum}-Clause of
Situation, Concession for the Sake of Argument, etc.; the statement
that each tense of the Subjunctive has the force of the Indicative
tense of the same name, and, in addition, each has a future force,
etc. Many of these appear in the earlier publications of the senior
author.  Others were devised for purposes of his class-room. All of
them have found acceptance in one or another of various grammars,
grammatical writings, and text-editions of authors, in various
countries.
That they have been taken up so readily into usage is a
matter of much satisfaction, since it seems to show that other workers
also have found them to be both intelligible and needful.

It is hoped that the arrangement and form of exposition found in the
book, together with the division of case-uses and mood-uses into
families, and the accompanying synopses, will lead the student to
conceive of Latin syntax as a living and organic whole, not as a
series of mechanical pigeon-holes.

\versionB*{The views upon the relation of ictus to accent advocated in
  \xref[2]{645} have been tested by the senior editor in many years of
  teaching at Harvard, Cornell, and Chicago, and have been found to
  render the reading of Latin verse both easier and more interesting.}

To specify all the obligations of the authors to the literature upon
the subjects treated would be impossible. Needless to say, they have
availed themselves fully of Brugmann and Delbrück’s Comparative
Grammar, of the Latin Grammars of Lindsay, Sommer, Stolz and Schmalz,
of Neue’s Formenlehre with its unrivalled statistical information, of
the treatises of Madvig, Holtze, Draeger, Kühner, Roby, Antoine,
Riemann, Riemann and Goelzer, of the Schmalz-Landgraf revision of the
Syntax of Reisig, of articles in the various journals, etc., as well
as of the school-grammars most widely used in this country and
elsewhere.

For proof-reading and suggestions, they are much indebted to
Mr.\ E.~M.\  Washburn, of the South Side Academy, Chicago;
Mr.\ C.~E.\ Dixon and Mr.\ W.~F.\ Tibbetts, of the Erasmus Hall High
    School, Brooklyn, N.Y.;
Professors F.~F. Abbott, G.~L. Hendrickson, F.~B. Tarbell, and
    G.~J. Laing, of the University of Chicago;
Professor Willard K.\ Clement, of Evanston, Ill.;
Professor J.~C.\ Rolfe, of the University of Pennsylvania;
and Professor F.~W.\ Shipley, of Washington University, St.\ Louis.
Professor Hempl of Michigan read the sections on Phonology in
manuscript, and made some important suggestions.  In particular,
thanks are due to Mr.\ R.~A.\ von Minckwitz, of the DeWitt Clinton High
School, New York City, for many helpful suggestions; to Professor
G.~E.\ Barber, of the University of Nebraska, for searching and
valuable criticisms; to Professor D.~Thomson, of the University of
Washington, Seattle, for large collections of examples made by him
for the purpose; and to Professor A.~T.\ Walker, of the University of
Kansas, for \versionB*{proof-reading and suggestions, and also}
examples collected
by him when Instructor in the University of Chicago for an Outline of
the Uses of the Latin Moods and Tenses projected by the senior author,
some of which examples have been used in the present grammar. And
finally, the authors wish to record their especial indebtedness to
Mr.\ C.~H.\ Beeson, Fellow in the University of Chicago, formerly of the
Peoria High School, whose assistance has been generously and freely
given at points and in ways too numerous to state in detail.

\noindent
\begin{tabular*}{\textwidth}{@{}l@{\extracolsep{\fill}}r@{}}
& W. G. H.\\
\textsc{\small June, 1903.}
& C. D. B.
\end{tabular*}

\tableofcontents

\Unnumbered{Authors Cited}

\thispagestyle{dropfolio}

\label{authors_cited}

\bigskip

\begin{small}

\begin{tabular}{@{}>{\parindent4em}p{.49\textwidth}@{}|p{.49\textwidth}@{}}
\hline\hline

\multicolumn{1}{c|}{\textbfsc{early latin}} & \multicolumn{1}{c}{\textbfsc{later latin}} \\
\hline

\smallskip

Plautus, 254–184

Ennius, 239–169

Terence, 185?–159

Cato, 234–149

\bigskip
\medskip

\hrule

\medskip

\centerline{\textbfsc{ciceronian latin}}

\medskip

\hrule

\bigskip
\bigskip

Varro, 116–27

Lucretius, 96?–55

Catullus, 87–54?

Caesar, 100–44

Cicero, 106–43
&
\smallskip

Sallust, 86–34\emend{1}{.}{}

Nepos, 99?–24?

Virgil, 70–19

Horace, 65–8

Livy, 59~\bc–17~\ad

Ovid, 43~\bc–18?~\ad

Persius, 34~\ad–62~\ad

Seneca, 4?~\bc–65~\ad

Pliny the Elder, 23~\ad–79~\ad

Quintilian, 35?–95

Martial, 40?–102

Pliny the Younger, 62–113?

Tacitus, 55?–120?

Juvenal, 60?–140?
\\[\medskipamount]
\hline\hline
\end{tabular}

\medskip

For Caesar, Cicero, Virgil, and Horace, and also for Plautus and
Terence, the name of the work alone is given, the name of the author
not being cited.  The works of these authors drawn upon, with the
abbreviations, are as follows:
\begin{abbrevs}
Plautus:
    Amph.~= Amphitruō,
    As.~= Asināria,
    Aul.~= Aululāria,
    Bacch.~= Bacchides,
    Capt.~= Captīvī,
    Cist.~= Cistellāria,
    Epid.~= Epidicus,
    Men.~= Me\-naech\-mī,\break
    Merc.~= Mercātor,
    Mil. Gl.~= Mīles Glōriōsus,
    Pers.~= Persa,
    Poen.~= Poenulus,
    Pseud.~= Pseudolus,
    Rud.~= Rudēns,
    Stich.~= Stichus,
    Trin.~= Trinummus.

Terence:
    Ad.~= Adelphoe,
    And.~= Andria,
    Eun.~= Eunūchus,
    Hec.~= Hecyra,
    Heaut.~= Heautontimoroumenos,
    Ph.~= Phormiō.

Caesar:
    B.~C.~= dē Bellō Cīvīlī,
    B.~G.~= dē Bellō Gallicō.

Virgil:
    Aen.~= Aenēis,
    Ecl.~= Eclogae,
    Georg.~= Geōrgica.

Horace:
    A. P.~= Ars Poētica,
    Carm.~= Carmina,
    Ep.~= Epistolae,
    Epod.~= Epodī,
    Sat.~= Satirae.
\end{abbrevs}

Remaining abbreviations are for the works of Cicero.  The examples are
mostly from the Orations against Catiline (Cat.), for Archias (Arch.),
and for Pompey’s Command (Pomp.).  A few are from the Oration for Milo
(Mil.). The remainder are scattering. The abbreviations for them will
explain themselves, except that Am.~= Laelius dē Amīcitiā, Sen.~=
Catō Maior dē Senectūte, Senat.~= Ōrātiō post Reditum in Senātū
Habita, Fam.~= Epistolae ad Familiārēs, and Att.~= Epistolae ad
Atticum.

\end{small}

\mainmatter

\centerline{\Huge L\,A\,T\,I\,N\enskip G\,R\,A\,M\,M\,A\,R}

\bigskip

\part*{Phonology}

\chapter{The Alphabet}

\contentsentry{B}{The Alphabet, Phonetic Explanations, the Latin
  Sounds}

\section

The Latin alphabet is the same as the English, except that Latin has
no~\grapheme{w} and no~\grapheme{j}.

\subsubsection

\grapheme{K} occurs only in a few words, e.g.\ \latin{Kalendae}, usually
abbreviated to \abbrev{Kal}. \ \grapheme{C}, which comes from a form of
the Greek letter Gamma, retains its original value of~\grapheme{g} in the
abbreviations~\abbrev{C.} for \latin{Gāius} and \abbrev{Cn.} for
\latin{Gnaeus}.  \grapheme{Y} and~\grapheme{z} are used, in Cicero’s time
and later, in the transcription of words borrowed from the Greek.

\begin{note}

The Latin alphabet appears in our English alphabet, with certain
changes that have arisen in the course of time, either in the forms of
letters (our small letters are the results of such changes, for the
Romans regularly used only capitals), or in the evolution of new
characters which did not exist or were not recognized as distinct
letters by the Romans.  Thus \grapheme{V}~was used for both vowel and
consonant, as in~\latin{CVM} and~\latin{VIR}, and similarly
\grapheme{I} in~\latin{IN} and~\latin{IAM}\@.  \grapheme{U} was simply
the rounded form of~\grapheme{V}, while \grapheme{J}~is a late variety
of~\grapheme{I}\@.  The distinction of the letters \grapheme{v}
and~\grapheme{u} is of such convenience as to be commonly retained.
On the other hand, since the consonantal value of~\grapheme{i} is
restricted to an easily defined position~(\xref{11}), there is less
advantage in distinguishing it to the eye, and the use of~\grapheme{j}
may well be discarded.

\end{note}

\chapter{The Latin Sounds}

\headingC{Phonetic Explanations}

\section[Vowels and Consonants]

Vowels, such as \sound{a}, \sound{e}, \sound{o}, etc., furnish the
body of the syllable and bear its stress, while consonants, such as
\sound{t}, \sound{p}, \sound{g}, \sound{n}, etc., are accessory.  Thus
in the word \english{top} the weight of the syllable, as it were, is
in the~\phone{o}.

English \sound{y} and~\sound{w} (as in \english{yet},
\english{wet}), Latin consonantal \grapheme{i} and~\grapheme{v}, are
consonants.  But in their formation they are so closely allied to
the~\sound{i} and~\sound{u} vowels (as in \english{pin},
\english{pull}), differing from them mainly in being uttered more
rapidly, that they are sometimes called \emph{Semivowels}.

\section

Vowels are distinguished in various ways, among others as \emph{open}
and \emph{close}.  The \sound{a} in \english{father} is open, the
tongue lying flat and the breath passing out without any obstruction;
whereas \sound{i} (in \english{pin}) and \sound{u} (in
\english{pull}) are close vowels, the tongue being raised close to the
roof of the mouth, leaving but a narrow space for the breath.
Intermediate are the sounds of \sound{e} in \english{let} and
\sound{o} in \english{hot}.  \emph{Open} and \emph{close} are relative
terms, an infinite number of degrees being possible.  The long
\sound{i} and~\sound{u} in \english{machine}, \english{rule} are still
closer than the short \sound{i} and \sound{u} in \english{pin},
\english{pull}.  So too the long \sound{e} and~\sound{o} of
\english{they}, \english{no} are closer than the short \sound{e}
and~\sound{o} of \english{let}, \english{hot}.

\section

\emph{Nasalized} vowels are such as are heard in the “nasal twang”
which is so common in careless pronunciation.

\section[Diphthongs]

Diphthongs are combinations of two vowels pronounced in the same
breath-impulse, as \sound{ai} in~\english{aisle}, \sound{oi}
in~\english{coin}.  The stress is on the first vowel, the second being
much less distinct.

\section

Consonants are divided, according to the \emph{general nature of the
  sound}, into:

\subsection
\emph{Liquids}, as \sound{l} and~\sound{r}.

\subsection
\emph{Nasals}, as \sound{n}, \sound{m}, and \sound{ng} (in
\english{singing}).

\subsection
\emph{Fricatives} or \emph{Spirants}, as \sound{f}, \sound{s},
\sound{z}, \sound{th} in \english{thin} or \english{then}, etc.  Of
these, \sound{s} and~\sound{z} are also called \emph{Sibilants}.

\subsection
\emph{Mutes} or \emph{Stops}, as \sound{p}, \sound{t}, \sound{b}, etc.

\subsection
\emph{Aspirates} or \emph{Aspirated Mutes}.  These are mutes closely
followed by an additional breath-element, as in compounds like
\english{boat-house}, \english{loop-hole}, etc., except that in these
the mute and aspirate are in different syllables.  The sounds of
English \sound{th}, \sound{ph} in \english{thin}, \english{physic} are
\emph{not} aspirates, but \emph{fricatives}.

\section

Consonants are divided, according to the \emph{position of the organs
  in play}, into:

\subsection

\emph{Labials}, as \sound{p}, \sound{b}, \sound{f}, \sound{m}.

\subsection

\emph{Dentals}, as \sound{t}, \sound{d}, \sound{n}.

\subsection

\emph{Gutturals} or \emph{Palatals}, as \sound{k}, \sound{g},
\sound{ng} (in \english{singing}).

\section

Consonants are divided, according as they are produced \emph{with or
  without vibration of the vocal chords}, into:

\subsection

\emph{Voiced} Consonants or \emph{Sonants}, as \sound{b}, \sound{d},
\sound{g}, \sound{z}, \sound{l}, \sound{r}, \sound{n},~\sound{m}.

\subsection

\emph{Voiceless} Consonants or \emph{Surds}, as \sound{p}, \sound{t},
\sound{k}, \sound{s},~\sound{f}

\headingB{Vowels}

\section

The vowels are pronounced as follows:
\begin{Tabular}{ll}

\grapheme{a} as in the first syllable of \english{aha}.
& \grapheme{ō} about as in \english{no}.
\\

\grapheme{ā} as in \english{father}.
& \grapheme{i} as in \english{pin}.
\\

\grapheme{e} as in \english{let}.
& \grapheme{ī} as in \english{machine}.
\\

\grapheme{ē} about as in \english{they}.
& \grapheme{u} as in \english{pull}.
\\

\grapheme{o} about as in \english{obey}.
& \grapheme{ū} as in \english{rude}.
\end{Tabular}

\noindent \grapheme{y} like French~\grapheme{u} or German~\grapheme{ü}
(with the tongue in position to pronounce \grapheme{i} as in
\english{machine}, and lips in position to pronounce~\grapheme{u} as in
\english{rule}).

\begin{minor}

\subsubsection

True short \sound{a} and short \sound{o} do not exist in English in
accented syllables.  Latin short~\phone{a} was like the
long~\sound{a} in \english{father}, but more quickly uttered.
Short~\phone{o} approached our short~\sound{o} in \english{hot}, but
was made with the lips well rounded and well forward.  In the
pronunciation of many (though not of all) English-speaking people, it
is heard in unaccented positions, as in \english{obey} and
\english{democrat}.  In attempting to reproduce this quality in an
accented syllable one must avoid the natural English tendency to
lengthen the vowel, which would lead us into the serious error of
pronouncing Latin \latin{post} like English \english{post}.

\subsubsection

The English long vowels in such words as \english{they} and
\english{no} are not strictly pure vowels, for they have a slight
“vanishing” sound at the end, giving them the character of
diphthongs, which may roughly be indicated by \sound{\diphthong{ē}}
and~\sound{\diphthong[u]{ō}}.  The Latin \phone{ē} and \phone{ō} were
pure vowels like the corresponding German and French vowels (German
\german{See}, \german{Sohn}; French \french{été}, \french{chose}).

\subsubsection

The Latin long vowels differed from the short not only in the length
of time taken for utterance, but also (except in the case of
\phone{a},~\phone{ā}) in quality, the long vowels being closer
(see~\xref{3}) than the short.  This is also true of the English
vowels.

\end{minor}

\headingB{Diphthongs}

\section

The diphthongs are pronounced as follows:
\begin{Tabular}{l@{\extracolsep{4em}}l}

\grapheme{ae} like \sound{ai} in \english{aisle}.
& \grapheme{eu} as \sound{é\(h\)-oo}, smoothly pronounced
\\

\grapheme{au} like \sound{ou} in \english{out}.
& \quad in the same breath-impulse.
\\

\grapheme{oe} like \sound{oi} in \english{coin}.
& \grapheme{ui} as \sound{oó-ee}, smoothly prounounced
\\

\grapheme{ei} like \sound{ei} in \english{deign}.
& \quad in the same breath-impulse.

\end{Tabular}

\begin{minor}

\subsubsection
The pronunciation of \latin{ae}, \latin{oe}, and~\grapheme{au} as
monophthongs (\grapheme{ae} as open~\sound{ē}, \grapheme{oe} as
close~\sound{ē}, \grapheme{au} as open~\sound{ō}) was current in
vulgar speech from an early date, but in cultivated speech the
diphthongal pronunciation lasted well into imperial times.  An earlier
form of~\grapheme{ae} was~\grapheme{ai}, as was \grapheme{oi}
of~\grapheme{oe}.  Most cases of original \phone{oi} passed through
\phone{oe} to the monophthong~\phone{ū}, as
\latin{oinos},—\latin{oenus},—\latin{ūnus}.

\subsubsection
The original diphthong, \phone{eu}, once very common, was merged in
prehistoric times with~\phone{ou}, and this \phone{ou}, still
existing in early Latin, passed on to~\phone{ū}.  So original
\rec{deucō},\footnote{The asterisk (*) indicates an assumed form,
  that is, one which is not actually found, but is reconstructed,
  either after parallel forms which \emph{are} found, or from our
  knowledge of the related forms in other languages.  Some of the
  assumed forms given in this grammar are reconstructed only as
  regards the particular point under discussion, other matters which
  would only divert the attention being ignored.  So, for example,
  in~\xref[12]{49}, \latin{bīnī} is said to come from
  \rec{bis-nī}, although the fully reconstructed form would be
  \rec{duis-noi}.} early Latin \latin{doucō}, later
\latin{dūcō}.  Hence it is that \phone{eu} is of somewhat rare
occurrence in Latin, being confined to some interjections like
\latin{heu}, some Greek words like \latin{Eurus}, \english{southeast
  wind}, and a few words in which the \phone{eu} was of recent
origin, as \latin{seu}, \latin{neu}, \latin{ceu} (beside the fuller
forms \latin{sīve}, etc.).  \latin{Neuter} was trisyllabic
throughout early and classical Latin.  In \latin{neutiquam} the first
syllable was short, as if the spelling were \latin{n’utiquam}.

\subsubsection

\latin{Ei} is frequent in early inscriptions, representing an original
\phone{ei} (and also \phone{ai} and \phone{oi} in non-initial
syllables; see \xref[3]{42}; \xref[4]{44}), but this \phone{ei}
became~\phone{ī}, e.g.\ early \latin{deicō}, \latin{inceidō},
\latin{servei}, later \latin{dīcō}, \latin{incīdō},
\latin{servī}.  In classical Latin \grapheme{ei} occurs as a
diphthong only in the interjection \latin{hei} and a few words in
which it was of recent origin, e.g.\ \latin{dein}, \latin{deinde} from
\latin{dē-inde}.  In most words \grapheme{ei} forms two distinct syllables,
as in \latin{de-i-ficus}.

\subsubsection

The diphthong~\latin{ui} occurred at first only in the interjection
\latin{hui} (so in German only in the exclamations \german{hui},
\german{pfui}).  But it arose later in the pronominal forms
\latin{huic}, \latin{cui}, and \latin{huius}, \latin{cuius}, coming
from earlier \latin{hoic}, \latin{quoi}, and \latin{hoius},
\latin{quoius}, which were still in use in the time of Cicero. In all
other words \grapheme{ui} forms two distinct syllables, as
\latin{fu-it}, \latin{habu-it}, etc. And even \latin{huic} and
\latin{cui} are dissyllables in post-Augustan poetry.

\end{minor}

\headingB{Consonants}

\section
Most consonants are pronounced as in English, but the following points
are to be noted:
\begin{points}

\grapheme{c} always has the \sound{k} sound as in \english{cat}, never
the \sound{s} sound as in \english{centre}.

\grapheme{g} as in \english{get}, never as in \english{gem}.

\grapheme{t} as in \english{tin}, never as in \english{nation}.

\grapheme{s} as in \english{hiss}, never voiced (\grapheme{z}) as in
\english{his}.

\grapheme{bs} (e.g.\ in \latin{urps}, etc.)\ like \sound{ps} in
\english{cups}, not \sound{bz} as in \english{tubs}.

\grapheme{bt} (e.g.\ in \latin{ob-tineō}) as \sound{pt}.

\grapheme{x} always \sound{ks} as in \english{extra}, never \sound{gz}
as in \english{example}.

\grapheme{n} before \grapheme{c}, \grapheme{g}, \grapheme{qu} has the
sound of \sound{ng} in \english{singing}.  Before~\grapheme{s} it lost
its consonantal value, the preceding vowel being lengthened and
nasalized. So \latin{cōnsul}, pronounced \latin{cōsul} with
  nasalized~\latin{ō}~(\xref{4}).

\grapheme{r} “rolled” or “trilled” as in French.

\grapheme{z} (in words borrowed from the Greek) as in \english{zero}.

\grapheme{i} consonantal as \sound{y} in \sound{yet}.

\begin{indented}

\grapheme{i} is consonantal when standing at the beginning of a word
and followed by a vowel, and also in the interior of a word between
vowels. So, for example, \latin{iungō}, pronounced \sound{yungō},
\latin{biiugis}, pronounced \sound{biyugis}, \latin{maius}, pronounced
\sound{maiyus} (\xref[2,~\emph{a}]{29}), etc.

\begin{indented}

But in a number of words borrowed from the Greek, mostly proper nouns,
an initial \grapheme{i} before vowels represents the vowel,
e.g.\ \latin{iambus}.  In \latin{Gāius} \grapheme{i} is a vowel
(\latin{Gā-i-us}).

\end{indented}

\end{indented}

\grapheme{v} as \sound{w} in \english{wet}.

\begin{indented}

The letter \grapheme{u} has the same value as~\grapheme{v} in the
combinations \grapheme{qu} and \grapheme{ngu} and in the words
\latin{suāvis}, \latin{suādeō}, \latin{suēscō}. Compare
English \english{quarter}, \english{anguish}, \english{persuade}.

\end{indented}

\grapheme{ch}, \grapheme{ph}, \grapheme{th} are pronounced like
\grapheme{k}, \grapheme{p}, \grapheme{t}, but with an added
breath-element,—\emph{not} as in \english{church} or
\english{chagrin}, \english{physic}, \english{thin}. See~\xref[5]{6}.

\begin{indented}[2]

These sounds were introduced in the first century~\bc\ to represent,
in borrowed words, the Greek aspirates, which had previously been
represented by the simple mutes (e.g.\ \latin{teātrum}, later
\latin{theātrum}). They came to be used also in a few genuine Latin
words, as~\latin{pulcher}.

\end{indented}

Double letters represent real double consonants, each being pronounced
with a distinct articulation and in different syllables, as in
\english{book-case}, \english{hoop-pole}, \english{well-laid}, etc. So
\latin{sic-cus}, \latin{ap-pel-lō}.

\end{points}

\begin{note}

Although in general \phone{h} was pronounced by careful speakers as
in English, yet in certain combinations it seems to have been wholly
silent, as it probably was everywhere in the popular speech.  It never
prevents slurring~(\xref{34}), the shortening of vowels before other
vowels~(\xref{21}), or rhotacism (\xref{47}), and often admits
contraction~(\xref{45}).  It is sometimes a mere sign of hiatus, that
is, it is used to make clear to the eye that two vowels are to be
pronounced in two syllables rather than as a diphthong. So, for
example, in \latin{ahēnus}, a not uncommon spelling of
\latin{aēnus}.  Moreover, the Romans were often in doubt as to the
proper spelling, such variations as
\latin{harēna}—\latin{arēna}, \latin{herus}—\latin{erus},
etc., being frequent, and in the case of some words the approved
spelling, which we follow, is not the historically correct one,
for example, in \latin{ānser}, \english{goose}, which, according to
the related forms in other languages, should be~\latin{hānser}.

\end{note}

\headingB{Classification of the Latin Sounds}

\section
The following scheme gives a classification of the simple Latin
sounds. See the phonetic explanations (\xref{2}–\xref{8}). The sounds
borrowed from Greek are inclosed in parentheses.
\begin{center}\small
\begin{tabular}{l@{\qquad}ccccccccc@{\qquad}l}
&
&
&
&
& \grapheme{\u{ā}} \\

&
&
&
& \grapheme{e}
&
& \grapheme{o} \\

Vowels
&
&
& \grapheme{ē}
&
&
&
& \grapheme{ō}
&
&
& Voiced\\

&
& \grapheme{i}
&
&
&
&
&
& \grapheme{u} \\

& \grapheme{ī}
&
&
&
& \clap{(\grapheme{y})}
&
&
&
& \grapheme{ū} \\[\smallskipamount]

Breathing
&
&
&
&
& \grapheme{h}
&
&
&
&
& Voiceless \\[\smallskipamount]

Semivowels
& \rlap{\grapheme{i} consonant}
&
&
&
&
&
&
&
& \grapheme{v}
& Voiced \\[\smallskipamount]

Liquids
&
&
&
&
& \clap{\grapheme{r}, \grapheme{l}}
&
&
&
&
& Voiced\\[\smallskipamount]

Nasals
& \grapheme{n}
&
&
&
& \grapheme{n}
&
&
&
& \grapheme{m}
& Voiced\\[\smallskipamount]

Fricatives
&
&
&
&
& \grapheme{s}
&
&
&
& \grapheme{f}
& Voiceless\\

&
&
&
&
& \clap{(\grapheme{z})}
&
&
&
&
& Voiced\\[\smallskipamount]

Mutes or Stops
& \clap{\grapheme{c}, \grapheme{k}, \grapheme{q}}
&
&
&
& \grapheme{t}
&
&
&
& \grapheme{p}
& Voiceless\\

& \clap{(\grapheme{ch})}
&
&
&
& \clap{(\grapheme{th})}
&
&
&
& \clap{(\grapheme{ph})}
& Voiceless aspirate\\

& \grapheme{g}
&
&
&
& \grapheme{d}
&
&
&
& \grapheme{b}
& Voiced\\

& \clap{Guttural}
&
&
&
& \clap{Dental}
&
&
&
& \clap{Labial}

\end{tabular}
\end{center}

\begin{note}
Since \grapheme{x} represents not a simple sound, but two sounds
(\phone{k} + \phone{s}), it is not included in the classification.
\end{note}

\headingB{Syllables}

\contentsentry{B}{Syllables}

\section
A syllable is a sound or succession of sounds uttered with a single
breath-impulse.

\section
Every Latin word contains as many syllables as it has vowels or
diphthongs. The division of syllables is as follows:

\subsection
A single consonant goes with the following vowel, as in
\latin{bo-nus}, \latin{a-git}, \hbox{\latin{fe-rō}}.

\subsection
In the case of two or more consonants the division falls before the
last consonant, except that the combinations mute + liquid, and
\grapheme{qu} or \grapheme{gu}, go with the following
vowel.\footnote{\label{ftn:6:}It is often stated that such combinations of
  consonants as can be pronounced at the beginning of words (in either
Latin or Greek) were not separated, the pronunciation being, for
example, \latin{fa-ctus}, \latin{ca-stra}, \latin{sān-ctus}, etc.
But the actual division in inscriptions and manuscripts is against
this; nor is the teaching of the Roman grammarians or the evidence of
the Romance languages really in favor of it.} Thus:
\begin{enumerate}

\item
\latin{ter-men}, \latin{in-ter}, \latin{sic-cus}, \latin{fac-tus},
\latin{op-timus}, \latin{prīs-cus}, \latin{magis-ter},
\latin{sānc-tus}, but,

\item
\latin{pa-tris}, \latin{ala-cris}, \latin{tene-brae}, \latin{cas-tra},
\latin{se-quor}, \latin{lin-gua}.

\end{enumerate}

\begin{note}
In a sound-group like \grapheme{tr} (and \grapheme{qu},
\grapheme{gu}), the combination of the two elements is naturally so
close that they were regularly pronounced in the same syllable. But
the poets often made use of a division \latin{pat-ris},
etc.\ (\xref[3,\emph{a}]{89}).

An aspirated mute, though spelled with two letters, is of course a
single consonant (\latin{pul-cher}), while \grapheme{x} has the sound
of two consonants (e.g.\ \latin{axis}, pronounced \sound{ak-sis}, but
best written \latin{a-xis}).

\end{note}

\subsubsection

A syllable ending in a vowel is called \emph{open}, while one ending
in a consonant is called \emph{closed}.  Thus, the first syllable of
\latin{bo-nus} is open, that of \latin{sic-cus} closed.

\begin{minor}

\subsubsection

In the case of a closed syllable, the consonant which ends it may
conveniently be called an \emph{obstructed} consonant, since its clear
and full pronunciation is made more difficult through the fact that it
comes immediately before another consonant.

\end{minor}

\section
\subsection

In the \emph{writing} of compounds it is convenient to divide the syllables
in accordance with the etymology, as \latin{ad-est}, \latin{ad-igō},
etc.; and it is quite possible that they were so pronounced in the
studied utterance of purists.  But in ordinary speech and in verse the
two elements were blended, and so treated in accordance with the
general system of syllabification. For example, \latin{ad-est} and
\latin{ad-hibeō} were pronounced \latin{a-dest},
\latin{a-dhibeō}. But if a mute and a liquid came \emph{through
  composition} to stand together, they were always pronounced in
separate syllables, e.g.\ always \latin{ab-rumpō}, never
\latin{a-brumpō} like \latin{tene-brae}.

\subsection

Between words in connected discourse, at least in ordinary speech and
in verse, the division of syllables is the same as within a single
word. That is, before a word beginning with a vowel or~\phone{h}, a
final consonant goes with the following word, as happens in English in
some common phrases, such as \english{at all}, pronounced
\sound{a-tall}. So, for example, \latin{id est}, \latin{ad haec} were
pronounced \latin{i-dest}, \latin{a-dhaec}.

\pagebreak

\chapter*{Quantity of Vowels}

\contentsentry{B}{Quantity}

\enlargethispage{\baselineskip}

\section

According to the length of time taken in their pronunciation, vowels
are said to be \emph{long} or \emph{short}.  Long vowels are indicated
thus: \grapheme{ā}, \grapheme{ī}, \grapheme{ē}, etc. Vowels without any mark are
short.\footnote{In a few instances a special sign for the short vowel
  has been used; thus \grapheme{ă}, \grapheme{ĕ}.
  Vowels are sometimes marked as common; thus \latin{mih\u{ī}}.
  But this only means that forms belonging to two different periods
  were used by the poets.  In this grammar such words are commonly
  given in the form of the usual prose pronunciation, as
  \latin{mihi}.}

\begin{note}[Note 1]

If we regard the length of time taken in the pronunciation of a single
short vowel as the unit, sometimes called a \emph{mora}, we may assume
that a long vowel contained two of these units or \emph{morae}, that
is, that it took twice as long. But it must be remembered that in a
dead language we cannot know the exact relation in time, and that in
spoken languages there are often more than two variations in
quantity. So in English there are at least three, e.g.\ in
\english{met}, \english{mate}, and \english{made}. In Latin it is
quite possible that where vowels were lengthened before certain groups
of consonants the resulting quantity was not the same as that of the
original long vowels, but something between the usual short and long.
This would account for the fact that the evidence is sometimes
conflicting.  But the matter is beyond our knowledge, and we can take
account only of the two varieties.  The fact that Latin verse is based
on distinctions of quantity shows that the difference between long and
short vowels must have been very marked,—fully as much so as between
the English extremes of \english{met} and~\english{made}.

\end{note}

\begin{note}[Note 2]

In most cases the quantity of a vowel is shown by its value in poetry.
But where the syllable is long without regard to the quantity of the
vowel (as in \latin{dictus}, etc.; see~\xref[3]{29}), that is, in the
cases of what is known as “hidden quantity,” we are dependent on
other kinds of evidence. Such are:
\begin{enumerate}

\item
Statements of the Roman grammarians,

\item
Spelling in inscriptions, in which long vowels are frequently marked
as such.

\item
Greek transcriptions.

\item
Etymology.

\item
Treatment in compounds, long vowels not being subject to the same
changes as short; e.g.\ \latin{adāctus} beside \latin{āctus}, but
\latin{adfectus} beside \latin{factus}. See~\xref{41}, \xref{42}.

\item
The inherited forms of the Romance languages, which have preserved the
differences in quality which went hand in hand with differences in
quantity (\xref[c]{9}); e.g.\ Italian \italian{detto} from
\latin{dictus}, but \italian{scritto} from \latin{scrīptus}.

\end{enumerate}

Often there are several kinds of evidence combined, e.g.\ in
\latin{lēctus}, in which the \phone{ē} is shown by 1), 2), 3),
and 6). But all the evidence must be used with caution, and in a few
cases it is so meagre or conflicting that our designation of the
quantity represents only what is the more probable of the two
possibilities.

\end{note}

\section

The quantity of the vowel in any given word must be learned by
experience in the same way as its quality. From the outset in learning
forms, the student should be as careful to note whether, for example,
the vowel is short~\phone{e} or long~\phone{e}, as to note whether
it is \phone{e} or~\phone{i}. Since the quantity of vowels is
always marked in the grammar and in the texts first read, there is no
difficulty in doing this.

\begin{minor}

At the same time, there are certain general processes of lengthening
or shortening, from which there results a uniform quantity for certain
conditions; and, also, some general groupings of the facts, which,
though purely mechanical, will be of assistance to the memory.
Statements covering these are given in the following sections.

\end{minor}

\headingC{Quantity of Vowels in Syllables not Final}

\section

Vowels are always long before \phone{ns}, \phone{nf},
\phone{nx}, and \phone{nct}, as in \latin{cōnsul},
\latin{īnfrā}, \latin{iūnxī}, \latin{iūnctus}.

\begin{note}

In these combinations the nasal was only faintly sounded, or, in the
case of~\phone{ns}, wholly lost as a consonantal element, the
preceding vowel being itself nasalized.  But the total time taken in
the pronunciation of the syllable remained the same, the nasalization
of the vowel being accompanied by lengthening.

\end{note}

\begin{minor}

\subsubsection

There was a tendency in certain circles to lengthen the vowel before
\phone{r} + consonant.  This pronunciation was regarded in general
as improper, but in some words it became the recognized one. This is
certainly true of \latin{fōrma}, \latin{ōrdō}, \latin{ōrdior},
\latin{ōrnō}, and probably of \latin{Mārcus}, \latin{Mārcius},
\latin{Mārs}, \latin{Lārs}, \latin{quārtus}.

\subsubsection

Somewhat similarly before~\phone{gn}. Aside from \latin{rēgnum},
\latin{stāgnum}, and \latin{sēgnis}, in which the vowel is long by
origin, a pronunciation with lengthened vowel existed in the case of
\latin{dīgnus}, \latin{sīgnum}, \latin{īgnis}, and in words like
\latin{prīvīgnus}; but even in these it did not become
established.  We therefore write \latin{dignus}, \latin{signum}, etc.,
as well as \latin{magnus}, \latin{ignōscō}, etc., for which there
is no evidence whatever of a long vowel.\footnote{A full discussion of
  this matter is impossible here, but a word of justification for the
  departure here made from the previous practice of our grammars and
  lexicons is perhaps desirable. Take, for example, the word
  \latin{signum}. In inscriptions it is written a few times with the
  tall \grapheme{I} or \grapheme{ei}, which point to a pronunciation
  with long vowel.  On the other hand, the inherited forms of the
  Romance languages and the borrowed forms in the Germanic and Celtic
  languages point to a pronunciation with short vowel.  In this and
  some other words, then, both pronunciations existed, probably in
  different strata of society.  But there is no evidence to show that
  the pronunciation with long vowel was considered preferable for any
  of these words.  So, since for the majority of words with
  \phone{gn}, such as \latin{magnus}, \latin{ignōscō}, etc.,
  there is no evidence whatever for the long vowel, the advantage of
  uniformity (barring, of course, the cases of original length, as
  \latin{rēgnum}) may be allowed to tip the scales in favor of
  \latin{signum} with the short vowel.  In writing \latin{signum},
  \latin{magnus}, etc., the authors are in agreement with the most
  recent practice of several other scholars, though many still mark
  the vowel long.  Some, indeed, write “\latin{signum} and
  \latin{sīgnum},” etc., but this is not possible for a school
  grammar.

  It may be added that some scholars question whether the lengthening
  of a vowel before \phone{nx} and \phone{nct} was universal, but
  there is no sufficient reason for doubting this.}

\subsubsection

For the quantity before \phone{x} and \phone{ct}, just as before many
other groups of consonants, there is no uniformity; each case must be
judged by itself.  Just as the vowel is long by origin in \latin{lēx}
(Gen.\ \latin{lēgis}), but short in \latin{nex} (Gen. \latin{necis}),
so some Perfects, as \latin{rēxī}, \latin{tēxī},
etc.\ (\xref[\emph{C},\emph{d}]{173}), have a long vowel parallel to
that in \latin{lēgī}, but others the short vowel, as \latin{spexī},
\latin{coxī}, etc.  Similarly \latin{lēctus}, \latin{rēctus},
\latin{tēctus}, etc., with a long vowel as in \latin{lēgī},
\latin{rēxī}, \latin{tēxī}, but other Participles with a short vowel,
as \latin{dictus}, \latin{factus}.  See~\xref{180}.

\end{minor}

\section

Vowels are long when they result from contraction, or represent
diphthongs. Thus \latin{nīl} from \latin{nihil},
\latin{exīstimō} from~\rec{ex-aestimō}.

\section

Vowels are regularly short (in \emph{all} syllables) before
\phone{nt} and \phone{nd}.  Thus \latin{amantis}, \latin{amandus},
\latin{amant}, beside \latin{amāmus}.

\begin{minor}

\subsubsection

But in certain words, in which the combination of a long vowel with a
following \phone{nt} or \phone{nd} arose after the shortening
process had already taken place, the length is retained.  So
\latin{cōntiō} (from \latin{coventiō}), \latin{prēndō} (from
\latin{pre-hendō}), \latin{vēndō} (\latin{vēnum-dō}),
\latin{nūntius}, \latin{nūndinae}, \latin{quīntus},
\latin{ūndecim}.

\end{minor}

\refstepcounter{subsection}

\subsection

Vowels are short before \phone{ss}, except in the contracted Perfect
forms, like \latin{amāsse} beside \latin{amāvisse}, etc., and in
the short forms of \latin{edō}, \english{eat}, as \latin{ēs},
\latin{ēst}, \latin{ēsse}, etc. So \latin{fissus}, \latin{fossus},
\latin{sessum}, etc.

\begin{note}

This is because an original~\phone{ss}, when preceded by a long
vowel, became~\phone{s}. See \xref[6]{49}.

\end{note}

\section

A vowel is regularly short before another vowel, and also when only
the weak sound \phone{h} intervenes; e.g.\ \latin{pius} (originally
with long~\phone{i}), \latin{de-hīscō} beside \latin{dē-dūcō},
etc.\footnote{\label{ftn:9:}Observe the similar shortening of a
  diphthong, e.g.\ \latin{pre-hendō} for \latin{prae-hendō}; likewise,
  though without change in spelling, \latin{prae-eunte}
  (Aen.~5,~186).} But there are some exceptions, as in:

\subsection

Forms of \latin{fīō}, except when \phone{i} is followed
by~\phone{er}; e.g.\ \latin{fīō}, \latin{fīunt},
\latin{fīēbam}, etc., but \latin{fierī}, \latin{fierem}, etc.

\subsection

Pronominal Genitives like \latin{ūnīus}, \latin{illīus},
\latin{tōtīus}.

\subsection

Genitives and Datives of the Fifth Declension in \ending{-ēī},
when a vowel precedes; e.g. \latin{diēī}, but \latin{fideī}.

\subsection

Old Genitives of the First Declension in \ending{-āī}, as
\latin{aulāī}.

\subsection

Some Greek words, as \latin{āēr}, \latin{Aenēās}, etc.

\subsection

\latin{Dīus} (for \latin{dīvus}), sometimes \latin{Dī\-āna},
\latin{ōhe}, \latin{ēheu}.

\begin{minor}

\subsection

Early Latin \latin{fūit}, \latin{plūit}, etc., but usually
\latin{fuit}, \latin{pluit}.

\end{minor}

\begin{note}

For the Pronominal Genitives the pronunciation \latin{ūnīus},
etc., was the one recognized by the Romans as correct, and we should
follow this.  But there was a tendency in common speech to shorten the
vowel, and forms like \latin{ūnius}, \latin{illius},
\latin{tōtius} are not uncommon in poetry of all periods.  This is
especially frequent in the case of \latin{alterius}, since
\latin{alterīus} could not be used in dactylic poetry; so, always,
\latin{utriusque}.

\end{note}

\section

In the Root-Syllable the quantity of the vowel is generally the same
for all forms derived from the same root; e.g.\ \latin{scrībō},
\latin{scrība}, \latin{scrīptor}, etc.  But some roots appear in
two different forms, which may differ in the quantity of the vowel, as
they do sometimes in its quality. See~\xref{46}.

\begin{note}

For Perfects and Perfect Passive Participles with vowel quantity
different from that of the Present, see~\xref[\emph{C}, \emph{c},
  \emph{d}]{173}; \xref{180}.  Derivatives with variation in vowel
quantity, such as \latin{sēdēs} (\latin{sedeō}), \latin{tēgula}
(\latin{tegō}), etc., are comparatively rare and may be learned in
each individual case.

\end{note}

\section
\subsection

The Stem-Vowel of the First, Second, and Fourth Conjugations is long
(\phone{ā}, \phone{ē}, \phone{ī}), while that of the Third is
short (\phone{e}, \phone{i}, \phone{u}).  Thus, \latin{amāre},
\latin{monēre}, \latin{audīre}, but \latin{tegere},
\latin{tegitur}, \latin{teguntur}.

\begin{minor}

\subsubsection

But in \latin{dō}, \english{give}, the stem-vowel is short~\phone{a}
except in the Second Person of the Present Indicative, \latin{dās},
and Imperative~\latin{dā}; e.g.\ \latin{damus}, \latin{dabam}, etc.

\end{minor}

\subsection

In the formation of Derivatives from Noun or Verb-stems, \phone{a} is
long, representing the stem of Nouns of the First Declension or Verbs
of the First Conjugation; e.g.\ \latin{Rōmānus},
\latin{arātrum}. \phone{E}, \phone{o}, \phone{u} are also usually
long; e.g.\ \latin{fidēlis}, \latin{egēnus}, \latin{patrōnus},
\latin{vīnōsus}, \latin{tribūnus}, \latin{lānūgō} (but
\phone{o} and usually~\phone{u} are short before~\phone{l};
e.g.\ \latin{fīliolus}, \latin{rīvulus}, etc.). \phone{I} is
oftenest short, representing original short~\phone{i}, or a weakened
\phone{e} or~\phone{o} (\xref[2,5]{42}); e.g.\ \latin{cīvitās},
\latin{bonitās}, \latin{dominus}; but long~\phone{i} is also
frequent; e.g.\ \latin{sedīle}, \latin{rēgīna}.

\subsection

In the stem of Nouns of the Third Declension \phone{-on-} is always
long; e.g.\ Gen.\ \latin{sermōnis}; \phone{-or-} is short in
Neuters, e.g.\ \latin{corporis}, but in Masculines and Feminines it is
long except in the Nom.-Voc.\ Sing.; e.g.\ \latin{amor},
\latin{amōris}.  Exceptions are \latin{arbor}, \english{tree},
\gender{f.}, Gen.\ \latin{arboris}; \latin{lepus}, \english{hare},
\gender{m.}, Gen.\ \latin{leporis}.

\subsection

Verbs in \ending{-scō}, except \latin{discō}, \latin{poscō}, and
\latin{compescō} have a long vowel before the suffix;
e.g.\ \latin{crēscō}, \latin{pāsco}, \latin{adolēscō},
etc. See \xref[F,note]{168}; \xref[2]{212}.

\section

In Compounds the quantity of vowels generally remains the same as in
the separate parts.  Thus \latin{cadō}, \latin{incidō};
\latin{cēdō}, \latin{abscēdō}.

But note the following variations in the form of certain prefixes:

\subsection

\latin{Dis} becomes \latin{dī} before a voiced consonant;
e.g.\ \latin{dī-dō}, \latin{dī-moveō},
\latin{dī-iū\-di\-cō}.  In forms like \latin{di-scrībō},
although an ~\phone{s} is dropped, the vowel is not
lengthened. See~\xref[7]{51}.

\subsection

\latin{Prō} has a short vowel before another vowel or \phone{h}, and
before \phone{f} followed by a vowel, except in \latin{prō-ferō}
and \latin{prō-ficiō}.  So \latin{pro-avus}, \latin{pro-inde},
\latin{pro-hibeō}, \latin{pro-fugiō}, \latin{pro-fundō}, etc.
But before vowels \phone{prōd-} is commonly used;
e.g. \latin{prōd-eō}, \latin{prōd-esse}, \latin{prōd-igō}.

\subsubsection

\begin{minor}

The form with the short vowel appears also in \latin{pro-cella},
\latin{pro-nepōs}, \latin{pro-pāgō} (usually), and, in some
other less obvious compounds, as \latin{pro-cul}, \latin{pro-pe},
\latin{pro-bus}.

\end{minor}

\begin{note}

Although before a vowel or~\phone{h} the short vowel may be the result
of the regular shortening (\xref{21}), in the other cases \latin{pro}
represents an inherited variety of \latin{prō} (Greek has regularly
the short vowel).  In early Latin the demarcation in the use of the
two forms was
less fixed than later, and even in classical poetry there are
occasional departures from the normal usage;
e.g.\ \latin{pro-cūrō} beside the usual \latin{prō-cūrō},
and, \emph{vice versa}, \latin{prō-fundō} beside the usual
\latin{pro-fundō}.

\end{note}

Observe that \phone{ā}, \phone{ē}, and \phone{dē} (except in such
forms as \latin{de-hīscō}; see~\xref{21}) always remain long; also
that \phone{re} is always short (for \phone{red} before vowels
and~\phone{h}, see~\xref[15]{51}).

\subsection

\latin{Ne}, not \latin{nē}, is the form of the negative prefix in
\latin{ne-fandus}, \latin{ne-fās}, \latin{ne-queō},
\latin{ne-sciō}, \latin{ne-scius}.

\headingC{Quantity of Vowels in Final Syllables\protect\footnotemark\
  (including Monosyllables)}

\footnotetext{\label{ftn:10:1}These statements do not cover all early
  Latin forms or words borrowed from the Greek, which often retain
  original quantities. Thus \latin{āēr}, \latin{crātēr},
  \latin{Trōes}, \latin{Simoīs}, \latin{Cȳmothoē}.}

\headingD{Words ending in a Consonant}

\section

Unless the final consonant is~\phone{s}, the vowel is short.
Exceptions are:

\subsection

Some (not all) monosyllables in~\phone{-l}, \phone{-r}, \phone{-n},
and~\phone{-c}, namely \latin{sōl}, \latin{sāl}, \latin{nīl},
\latin{pār} (with its compounds), \latin{vēr}, \latin{Lār},
\latin{fūr}, \latin{cūr}, \latin{ēn}, \latin{nōn},
\latin{quīn}, \latin{sīn}, \latin{dīc}, \latin{dūc},
\latin{sīc}; also the Adverbs of Place \latin{hīc}, \latin{hūc},
\latin{illīc}, etc. (For the Nominatives \latin{hoc} and
\latin{hic}, see~\xref[2]{30}.)

\subsection

The contracted forms of the \phone{-īvī} Perfect,
e.g. \latin{audīt}.

\begin{note}

For words ending in more than one consonant no general statement can
be made, except that the vowel is always long before \phone{-ns}
and~\phone{-nx} (\xref{18}), short before~\phone{-nt} (\xref[1]{20}).

\end{note}

\section

This prevalence of the short vowel is mainly due to the fact that
every originally long vowel was regularly shortened before final
\phone{m}, \phone{t}, \phone{nt} (for \phone{nt}, see also~\xref{20}),
and, except in monosyllables, before final \phone{l}
and~\phone{r}.\footnote{Note also that final~\phone{d} cannot occur
  after a long vowel, since in this position it was lost in early
  Latin~(\xref{48}).  Of the other consonants which occur as finals,
  only \latin{n}~is frequent, and this, in large part, in Neuter
  \phone{n}-Stems like \latin{nōmen}, where the short vowel is in
  accordance with the origin of the formation.}  Examples of this
shortening are seen as follows:

\begin{minor}

\subsection

In verb-forms with the personal endings \phone{-m}, \phone{-t}, and
\phone{-nt}, wherever these are added to a tense-stem or mood-stem
ending in a long vowel.  The long vowel shows itself in the Second
Singular and First and Second Plural.  So:
\begin{enuma}

\item
Present Indicative of the First, Second, and Fourth Conjugations;
e.g.\ \latin{amat}, \latin{amant}, beside \latin{amās},
\latin{amāmus}, \latin{amātis};
\latin{monet}, \latin{monent}, beside \latin{monēs}, etc.;
\latin{audit} beside \latin{audīs}, etc.

\item
Imperfect Indicative of all Conjugations; e.g.\ \latin{amābam},
\latin{amābat}, \latin{amābant}, beside \latin{amā\-bās}, etc.

\item
Future Indicative of the Third and Fourth Conjugations;
e.g.\ \latin{tegam}, \latin{teget}, \latin{tegent}, beside
\latin{tegēs}, etc.

\item
Past Perfect Indicative of all Conjugations; e.g.\ \latin{amāveram},
\latin{amā\-ve\-rat}, \latin{amā\-ve\-rant}, beside \latin{amāverās},
etc.

\item
All tenses of the Subjunctive in all Conjugations;
e.g.\ \latin{tegam}, \latin{tegat}, \latin{tegant}, beside
\latin{tegās}, etc.;
\latin{tegerem}, \latin{tegeret}, \latin{tegerent}, beside
\latin{tegerēs}, etc.; \latin{tēxerim}, \latin{tēxe\-rit},
\latin{tēxerint}, beside \latin{tēxerīmus};
\latin{tēxissem}, \latin{tēxisset}, \latin{tēxissent},
beside \latin{tēxissēs}, etc.

\end{enuma}

\subsection

In all Passive forms ending in \phone{-r}; e.g.\ \latin{tegor} from
\rec{tegō-r} (i.e.\ Act.\ \latin{tegō}~+~\phone{r});
Imperf.\ \latin{te\-gē\-bar} beside \latin{tegēbāris};
Fut.\ \latin{tegar} from \rec{tegā-r} (as Act.\ \latin{tegam} from
\rec{tegā-m}); Pres.\ Subj.\ \latin{tegar} beside \latin{tegāris};
Imperf.\ Subj.\ \latin{tegerer} beside \latin{tegerēris};
Imperat.\ \latin{tegitor}, \latin{teguntor} from \rec{tegitō-r},
\rec{teguntō-r}; and so in the other Conjugations.

\subsection

In many Nom.-Voc.\ Sing.\ forms ending in \phone{-r} or \phone{-l}
(also Accusatives in the case of Neuters); e.g.\ \latin{amor} beside
Gen.\ \latin{amōris}, \latin{animal} beside \latin{animālis},
\latin{calcar} beside \latin{calcāris}; likewise \latin{pater},
\latin{māter}, \latin{frāter}, from original \latin{patēr}, etc.

\subsection

In the Accusative Singular of the First and Fifth Declensions, which
ended originally in \phone{-ā-m} and \phone{-ē-m}.

\subsection

In the Genitive Plural of all Declensions, which ended originally in
\phone{-ōm}.  This first became \phone{-om}, then \phone{-um}
(\xref[1]{44}).

\begin{note}

But before \phone{t} and \phone{r} the long vowel was still retained
in early Latin, and is sometimes found even in later poetry;
e.g.\ \latin{arāt}, \latin{vidēt}, \latin{erāt},
\latin{peterēt}, \latin{ferār}, \latin{amōr}, \latin{patēr}.
See under Versification, \xref[2]{652}.

\end{note}

\end{minor}

\section

Before final~\phone{s} the quantity varies.

\subsection

Final \phone{as} is long; e.g.\ \latin{sellās}, \latin{amās}.

Exceptions \latin{anas}, \english{duck}, Gen.\ \latin{anatis};
\latin{as}, \english{copper}, Gen.\ \latin{assis}.

\subsection

Final \phone{es} is usually long; e.g.\ \latin{rēgēs}, \latin{fidēs},
\latin{monēs}.

But final \phone{es} is short:
\begin{enuma}

\item

In the Nom.-Voc.\ Sing.\ of most dental stems which show a short vowel
in the other cases, as \latin{dīves}, Gen.\ \latin{dīvitis};
\latin{mīles}, Gen.\ \latin{mīlitis};
\latin{seges}, Gen.\ \latin{segetis}, etc. But note \latin{pēs},
\latin{abiēs}, \latin{ariēs}, \latin{pariēs}
(Gen.\ \latin{pedis}, \latin{abietis}, etc.).

\item

In \latin{es}, \english{thou art}, or \english{be} (but \latin{ēs},
\english{eat}, from \latin{edō}), and \latin{penes}, \english{with}.

\end{enuma}

\begin{note}

Original short \phone{-es} became \phone{-is} (\xref[2]{44}).  Of the
examples of existing short~\phone{-es} nearly all represent earlier
\phone{-ess}, traces of which are found in early Latin
(\xref[3]{30}).  For example, \latin{es} from \latin{ess},
\latin{mīles} from \latin{mīless} (\rec{mīlet-s}).

\end{note}

\subsection

Final \phone{os} is long; e.g.\ \latin{hortōs}, \latin{flōs}.

Exceptions: \latin{os}, \english{bone} (but \latin{ōs},
\english{mouth}), \latin{compos}, \latin{impos}.

\subsection

Final \phone{is} is oftenest short; e.g. \latin{regis}, \latin{tegis}.

But final \phone{is} is long;
\begin{enuma}

\item

In Plural Case-endings; e.g.\ Dat.-Abl.\ \latin{sellīs},
\latin{hortīs}, \latin{nōbīs},
Acc.\ \latin{fīnīs}.

\item

In the Second Person Singular of verb-forms where the First Plural is
\hbox{\phone{-īmus}}, namely in:
\begin{indented}

Pres.\ Indic.\ Act.\ of the Fourth Conjugation, e.g.\ \latin{audīs}.

Pres.\ Indic.\ Act.\ of some Irregular Verbs, e.g.\ \latin{īs},
\latin{fīs}; also \latin{vīs}, \latin{māvīs}, etc.

Pres.\ Subj.\ Act.\ of some Irregular Verbs, e.g. \latin{sīs},
\latin{velīs}, \latin{nōlīs}

Perf.\ Subj.\ Act., e.g.\ \latin{amāverīs},
\latin{tegerīs} (but sometimes short~\phone{-is}; \emph{vice
  versa} in the Fut.\ Perf.\ Indic.\ sometimes \phone{-īs} beside
the regular~\phone{-is}. See~\xref[6]{164}).

\end{indented}

\item
In \latin{vīs}, \english{force}, \latin{Quirīs},
\latin{Samnīs} (Gen.\ \phone{-ītis}); often
  \latin{sanguīs}, rarely \latin{pulvīs}.

\end{enuma}

\subsection

Final \phone{us} is usually short; e.g. \latin{hortus}, \latin{genus},
\latin{rēgibus}, \latin{tegimus}.

But final \phone{us} is long:
\begin{enuma}

\item

In the Gen.\ Sing.\ and the Nom.\ and Acc.\ Plur.\ of the Fourth
Declension, as \latin{tribūs}.

\item

In the Nom.-Voc.\ Sing.\ of Nouns of the Third Declension which have
long \phone{u} in the other cases, as \latin{virtūs},
\latin{tellūs}, \latin{iūs}, etc.\ (Gen.\ \latin{virtūtis},
\latin{tellūris}, \latin{iūris}).

\end{enuma}

\headingD{Words ending in a Vowel}

\section
\subsection
Final \phone{a} is oftenest short, namely in the Nom.\ Sing.\ of the
First Declension and the Nom.-Acc.\ Plur.\ of all Neuters;
e.g. \latin{sella}, \latin{dōna}, \latin{genera}.

But final \phone{a} is long:
\begin{enuma}

\item
In the Abl.\ Sing.\ of the First Declension, as \latin{sellā}.

\item
In the Imperative of the First Conjugation, as \latin{amā}.

\item
In most uninflected words (except \latin{ita}, \latin{quia}),
e.g.\ \latin{īuxtā}, \latin{trīgintā}, and Adverbs like
\latin{contrā}, \latin{extrā}, \latin{posteā}, which are
Ablatives in origin.

\end{enuma}

\subsection

Final \phone{e} is usually short; e.g.\ \latin{horte}, \latin{tege},
\latin{tegere}.

But final~\phone{e} is long:
\begin{enuma}

\item
In the Abl.\ Sing.\ of the Fifth Declension, e.g.\ \latin{diē}.

\item
In the Imperative of the Second Conjugation, e.g.\ \latin{monē} (but
often short in \latin{ave}, \latin{cave}, \latin{vale}, and, in early
Latin, in many other Imperatives; see note).

\item
In Adverbs derived from Adjectives of the First and Second Declensions
(\xref[1]{126}), e.g.\ \latin{rēctē}, \latin{altē} (but always
short in \latin{bene}, \latin{male}).

\item
In all monosyllables except those used as enclitics
(\enclitic{-que}, \enclitic{-ne}, etc.; see \xref[1]{33}), namely
\latin{ē}, \latin{dē}, \latin{mē}, \latin{tē},
\latin{sē},~\latin{nē}.

\end{enuma}

\subsection

Final \phone{i} is long, except in \latin{nisi}, \latin{quasi}, and,
in the usual prose pronunciations, in \latin{mihi}, \latin{tibi},
\latin{sibi}, \latin{ibi}, \latin{ubi}.  But the older forms
\latin{mihī}, etc., with final long~\phone{i}, are used in poetry
side by side with \latin{mihi}, etc.

\subsection

Final~\phone{o} is long, except in \latin{ego}, \latin{modo},
\latin{cito}, \latin{duo}, \latin{cedo} (\english{give}).  But in
several other words it is sometimes short in poetry,
e.g.\ \latin{homo}, \latin{volo}, \latin{scio}.  (From Ovid on, the
short vowel becomes more and more frequent in the Nom.\ Sing.\ of the
Third Declension, in Verb-forms, and in many other words, such as
\latin{ergo}, \latin{octo}, \latin{immo}, etc.)

\subsection

Final~\phone{u} is long.

\begin{note}

The short final \phone{o} and \phone{i} in all the examples given, and
likewise the short~\phone{e} in \latin{bene} and~\latin{male},
represent originally long vowels or diphthongs, e.g.\ \latin{modo}
from \rec{modō} like \latin{prīmō}, \latin{bene} from \rec{benē} like
\latin{altē} (\xref[1]{126}), \latin{quasi}, \latin{nisi} from
\latin{quasei}, \latin{nisei} (so written on early inscriptions;
cf.\ \latin{sī}, early \latin{sei}).  The change was due to a process
known as iambic shortening.  In words of two syllables the first of
which was short, there was a marked tendency to shorten the final
syllable if long, that is, to change the word-rhythm from~\scan{sl}
to~\scan{ss}. This was not a mere matter of poetic usage, but a
characteristic of common speech. In isolated forms, such as those
mentioned, the tendency had full sway, and the short vowel is
prevalent from the earliest period.  In \latin{mihi}, etc., the form
with the short vowel became established, but the poets continued to
use also the old form \latin{mihī}, etc., at all periods.  For other
classes of words, early poetry, reflecting popular speech, shows many
examples of the same process; e.g.\ Gen.\ Sing.\ \latin{domi},
\latin{viri}; Dat.\ Sing.\ \latin{malo}; Nom.\ Sing.\ \latin{homo};
Imperat.\ \latin{ama}, \latin{puta}, \latin{mone}, \latin{cave},
\latin{abi}, \latin{redi}; First Sing.\ \latin{volo}; Second
Sing.\ \latin{vides}; \latin{viden} (for \latin{vidēn}, from
\latin{vidēs-ne}), \latin{rogan}, etc.  But here the tendency to
uniformity between words of the same class restored the normal type
with the long vowel in the cultivated speech.  Still, the short vowel
remains in \latin{puta}, meaning \english{for instance} (originally an
Imperative of~\latin{putō}), in \latin{viden}, \english{see?}\ and
usually in \latin{ave}, \latin{cave} used as Interjections, sometimes
also in \latin{homo}, \latin{volo}, etc.  Such forms in final
short~\phone{o} gain ground again from Ovid on (see above). The
short~\phone{a} of the Nom.\ Sing.\ of the First Declension and of
Neuter Plurals was once long, but here the short vowel, though
probably arising in iambic forms, extended to all words, and but few
traces of the long~\phone{a} are found even in early Latin.

\end{note}

\chapter*{Quantity of Syllables}

\section
\subsection

Syllables are short or long, according to the length of time taken in
pronouncing them.

\subsection

A syllable is long if it contains a long vowel or a diphthong; for
example, the first syllables of \latin{māter}, \latin{audiō}.

\begin{minor}

\subsubsection

The first syllables of words like \latin{maius}, \latin{cuius},
\latin{eius}, \latin{Troia}, etc., are long because they really
contain diphthongs.  For example, \latin{maius}, sometimes spelled
\latin{maiius}, is pronounced \sound{mai-ius}, much like our
\english{my use}, with the accent on~\english{my}.  These words are
often written \grapheme{māius}, \grapheme{cūius},
\grapheme{ēius}, etc., but should not be, since the vowel itself is
short.

\end{minor}

\subsection

A syllable is also long, even when the vowel is short, provided it
ends in a consonant.  The time taken in pronouncing the consonant,
being added to that taken in pronouncing the vowel, makes the syllable
long.\footnote{\emph{The quantity of the vowel is not affected.}
  Calling the vowel “long by position” often misleads the beginner
  into such an error as pronouncing \latin{est}, \english{is}, with a
  long~\phone{e}.}

\begin{minor}

That is, following the system of syllabification laid down
in~\xref[2]{14}, a syllable is long if its vowel is followed by two or
more consonants, except a mute followed by a liquid
(or~\grapheme{qu}).  So the first syllable is:
\begin{enumerate}

\item
Long in \latin{por-ta}, \latin{sic-cus}, \latin{fac-tus}, \latin{axis}
(\sound{ak-sis}).

\item
Short in \latin{bo-nus}, \latin{pa-tris}, \latin{se-quor},
\latin{a-dhibeō} (\xref[2,note]{14}; \xref[1]{15}).

\end{enumerate}

\subsubsection
In words like \latin{patris} the poets often use a syllabic division
\latin{pat-ris}, \latin{teneb-rae}, etc. (\xref[2,note]{14}), which
makes the first syllable long.\footnotemark[\thefootnote]

\begin{minor}
\subsubsection
The poets, following Greek usage, treat \latin{z} as a double
consonant.

\end{minor}

\end{minor}

\subsection

The same is true of final syllables.  If a word ends in a single
consonant its last syllable is long before a word beginning with a
consonant, but short before a word beginning with a vowel
or~\phone{h}, since in this case the final consonant is carried over
to the next word. See~\xref[2]{15}.

\begin{minor}

\section

In a few words we meet with a long syllable even where a short vowel
is followed by only one consonant in the normal spelling, namely in
compounds of \latin{iaciō} (\latin{adiciō}, \latin{coniciō},
etc.), in \latin{hoc}, and very often in~\latin{hic}.

\begin{minor}

\subsection

In compounds of \latin{iaciō} the regular spelling is \latin{adiciō},
\latin{disiciō}, etc.\ (not \latin{adiiciō}, etc.), and this
represents the usual prose pronunciation.  Yet in poetry the first
syllable is nearly always long.  This is explained by the older forms
\latin{adieciō}, etc., in which the first syllable was, of course,
long.  The poets made use of these earlier forms, which were more
convenient for the metre,—or, at least, retained the old division of
syllables, pronouncing \latin{ad-iciō}, etc.  Similarly, for
\latin{reiciō} the poets made use of the older form \latin{reieciō},
in which the first syllable was long for the same reason as in
\latin{eius}, etc.\ (i.e. \sound{rei-yeciō}, like \sound{ei-yus};
see~\xref[2, \emph{a}]{29}) or at least retained the older form of the
first syllable, pronouncing then \latin{rei-iciō}.  In the same way
the first syllable is long in both \latin{coniciō} and the less common
\latin{coiciō}.

\end{minor}

\subsection

In final syllables which once ended in two consonants, these
consonants were sometimes preserved in pronunciation before vowels,
though not shown by the spelling.  So the
Nom.\ Sing.\ Neut.\ \latin{hoc} stands for \latin{hocc}, that is
\rec{hod} (like \latin{quod}) + \phone{c}(\phone{e}), and was usually
so pronounced before vowels, though rarely so written.  Hence it is
nearly always a long syllable, for example, \latin{hoc dōnum} and
\latin{hoc-c erat}.  The Nom.\ Sing.\ Masc.\ \latin{hic} (earlier
\latin{he-c}) has a short vowel, and in earlier poetry is always a
short syllable before a word beginning with a vowel.  But a form
\latin{hicc} arose under the influence of \latin{hocc}, and in the
classical poets the word is oftener a long syllable than a short one.

\subsection

In early Latin there are other similar cases,
e.g.\ \latin{es}(\latin{s}), \english{thou art},
\latin{mīles}(\latin{s}), \latin{ter}(\latin{r}), etc.

\end{minor}

\chapter*{Accent}

\contentsentry{B}{Accent}

\section

The Latin accent was, like the English, one of stress.
Its position is as follows:

\subsection

In words of two syllables the accent is upon the first;
e.g.\ \latin{mágis}, \latin{tégō}.

\subsection

In words of more than two syllables it is upon the next to the last
(the \emph{penult}) if this is long, otherwise on the next preceding
that (the \emph{antepenult}); e.g. \latin{a-m\'ī-cus},
\latin{ma-gís-ter}, but \latin{bél-li-cus},
\latin{té-ne-brae} (\xref[3]{29}).

\subsection

Compounds are accented in the same way; e.g.\ \latin{ád-ferō} not
\latin{ad-férō}, \latin{c\'ōn\-fi\-cit} not
\latin{cōnfí\-cit}.  But in non-prepositional compounds of
\latin{faciō} such as \latin{calefaciō}, \latin{tepefaciō},
etc., which were written separately in the earliest period, the accent
is always on the verb, e.g.\ \latin{calefácit} not
\latin{caléfacit}.

\begin{note}

The system of accent here described was preceded in the earliest
period of the language, before the beginnings of literature, by a
different system, according to which all words had a stress accent
upon the first syllable.  A relic of this is seen in the early Latin
accentuation of words of four syllables of which the first three are
short, e.g.\ \latin{fácilius}.  Some important phonetic changes are
traced to this \emph{earlier accentual system}.

\end{note}

\section

There are, however, a few exceptions to these statements.

\subsection

When a final syllable following a long penult is lost, the accent
remains on what has now became the final syllable.  So
\latin{ill\'īc} from \latin{ill\'īce}, \latin{tant\'ōn}
from \latin{tant\'ōne}, \latin{add\'ūc} from
\latin{add\'ūce}, Perfect \latin{aud\'īt} from
\latin{aud\'īvit}, etc.; also adjectives in \phone{-ās},
Gen.\ \hbox{\phone{-ātis}}, denoting one’s native place, as
\latin{nostr\'ās} (from \latin{nostr\'ātis}),
\latin{Arpīn\'ās}, \latin{Capēn\'ās}, etc.

\subsection

The Genitive and Vocative in~\phone{-ī} of nouns in~\phone{-ius}
and \phone{-ium} are accented on the penult even when short,
e.g.\ \latin{Vergílī}, \latin{ingénī}.

\begin{note}

According to statements of the grammarians of the fourth and fifth
centuries \ad, words ending with the enclitics \enclitic{-que},
\enclitic{-ve}, \enclitic{-ne}, \enclitic{-ce} were always accented on
the syllable preceding the enclitic, even when this was short,
e.g.\ \latin{bonáque}, \latin{līmináque}, etc.  Since the
vast majority of inflected forms end in a long syllable, so that the
accent would necessarily fall on the syllable preceding the enclitic
(e.g.\ Abl.\ Sing.\ \latin{bon\'āque}, \latin{bon\'ōque},
Acc.\ Sing.\ \latin{bonáque}, etc.), one can readily see how
the few forms ending in a short vowel might come to be accented in the
same position.  But in early Latin such forms were accented in
accordance with the usual system, and there is some reason for
believing that in the Augustan period, too, the accent was still
\latin{bónaque}, \latin{līmínaque}, etc.  It seems best,
therefore, to accent such words in accordance with the general system,
that is, \latin{bónaque}, not \latin{bonáque}; similarly
\latin{ítaque} (in both meanings).

\end{note}

\section

In Latin there existed \english{sentence} accent, as in English, some
words being emphasized by stress, others being pronounced lightly.

\subsection

Certain words which were always unemphatic were pronounced and written
as a part of the preceding word.  Such words are known as
\emph{enclitic particles}, or simply \emph{enclitics}.  The commonest
of these are \enclitic{-que}, \enclitic{-ne}, \enclitic{-ve},
\enclitic{-ce}, \enclitic{-pte}, \enclitic{-met}, \enclitic{-dum}.

\subsection

Besides these, Relative and Indefinite Pronouns, Personal and
Determinative Pronouns when not emphatic, Prepositions, Conjunctions,
and parts of the verb “to be” were pronounced with little or no
stress.

\headingB{Slurring}

\contentsentry{B}{Slurring}

\section
\subsection

When a final vowel is followed by a word beginning with a vowel
or~\phone{h}, it is slurred or \emph{run into} the vowel of the
following word (as in “await alike the~\slur~inevitable hour”),
without, however, changing the quantity of the latter.

\begin{note}

It is frequently said that the final vowel was dropped in such a case.
We know, however, that it was not wholly lost.  One should pronounce
it very lightly, quickly passing to the following word. Thus
\latin{bon\tsup{a} \slur\ et}, \latin{bon\tsup{a}\scan{T}(h)aec}.

\end{note}

\subsection

The same is true of a word ending in~\phone{m}, final~\phone{m} losing
its consonantal value before a word beginning with a vowel
or~\phone{h}.  The preceding vowel was nasalized, and the lips
approached each other in a sort of \sound{w}-sound, which did not
interfere with the slurring of the vowels, e.g.\ \latin{bonum addit},
pronounced \sound{bon\tsup{u}\scan{T}\tsup{w}addit}; \latin{bonum hoc},
pronounced \sound{bon\tsup{u}\scan{T}\tsup{w}\(h\)oc}; \latin{bonam addit},
pronounced \sound{bon\tsup{a}\scan{T}\tsup{w}addit}.

\begin{note}

Such pronunciation need occasion no difficulty in practice.  If ones
tries, in the case of~\phone{-um}, simply to touch lightly upon
the~\phone{u} in passing to the following vowel, the nasalization and
the glide~\sound{w} will be produced unconsciously.  The same habit
can then be easily transferred to combinations with other vowels.

\end{note}

\subsection

Owing to their unemphatic or enclitic use the words \latin{es},
\english{are}, and \latin{est}, \english{is}, \emph{lose their vowel}
when preceded by a word ending in a vowel, just as in English
\english{you’re}, \english{he’s}, \english{she’s}.  And this, in
contrast to the processes mentioned in 1 and~2, is
sometimes indicated in the writing. So \latin{bonas} for \latin{bona
  es}, \latin{bonast} for \latin{bona est}, and also \latin{bonust}
for \latin{bonum est} (\latin{bonum} being pronounced without the
final~\phone{m}; see~2).

\begin{note}

Instead of \latin{bonust}, which is the only contracted spelling for
\latin{bonum est} known on inscriptions, and which is frequent enough
in~\textsc{mss}., our text-books, if they use the contracted spelling
at all, write \latin{bonumst}, which is a later spelling introduced to
distinguish this from another \latin{bonust}, an early Latin form for
\latin{bonus est} (i.e.\ really from \latin{bonu est}, a
final~\phone{s} in early Latin being lost under certain conditions).
The spelling \latin{bonumst} invites a wrong pronunciation and
misleads one as to the way in which the form originated.  If the
\phone{m} had been fully sounded, the vowel of \latin{est} would have
remained, since it is never lost after consonants.  Such a form as
\latin{idst} for \latin{id est}, like English \english{it’s}, is
unknown in Latin.

\end{note}

\chapter*[Suggestions on Pronunciation]{Suggestions with Regard to
  Pronunciation}

\contentsentry{B}{Suggestions with Regard to Pronunciation}

\section

A correct pronunciation is, of course, by no means the most important
thing in the study of Latin, but, if attained, it will lend much
attractiveness to the reading of the literature. The three striking
differences (\xref{36}, \xref{37}, \xref{38}) between Roman
pronunciation and the pronunciation of English should therefore be
constantly kept in mind.

\section

The difference in time between a short vowel and a long vowel was as
great in Roman speech as in the \emph{extremes} of short and long in our
speech (e.g.\ \english{met} and \english{made}), and was
\emph{steadily observed}.  Thus the \phone{ā} in the termination
\phone{-ānus} (e.g.\ \latin{Rōmānus}) took, roughly speaking,
twice as long to pronounce as the short~\phone{a} in \latin{anus},
\english{old-woman} (\phone{-ānus} =
\phone{-ăănus}).\footnote{Instead of trying to remember that,
  in his book, a given vowel in a given word \emph{had a mark over
    it}, or did not, the student should rather, in learning each new
  word, \emph{pronounce} all the long vowels distinctly long, and the
  short vowels distinctly short (or so \emph{think} the pronunciation
  to himself), and thus fix the word in mind \emph{as sounding} so and
  so.  When, later, he has occasion to write the word, he should ask
  himself, not “How did it look in the book?” but “How do I
  pronounce it?”

  A student who possesses the gift of visual memory should of course
  avail himself of it.  But, even in his case, the picture of the
  printed word which he carries in mind should be translated at once
  into a memory of sound.}

\section

The pronunciation of an obstructed consonant (\xref[2, \emph{b}]{14})
was much fuller and clearer in Roman speech than it ordinarily is in
English,\allowbreak—so full and clear, indeed, that it took about
\emph{as much time as a short vowel}.  For example, in
\latin{ă\u{n}-nus}, \latin{pĕ\u{c}-tus}, or \latin{ĭ\u{s}-te}, the
obstructed \phone{n}, \phone{c}, or~\phone{s} at the end of the first
syllable takes as much time to utter as the \phone{a}, the \phone{e},
or the~\phone{i}.  In an English word like \english{protected}, on the
other hand, so little time is spent upon the~\sound{c} in ordinary
speech that the syllable which it ends belongs to the short class
rather than to the long class.

\section

The Romans habitually slurred a vowel (\xref[1 and 2]{34}) at the end
of a word before an initial vowel or~\phone{h}, unless there was some
special reason for pausing.  In English we occasionally do this,
especially with such words as \english{to} or \english{the}
(e.g.\ \english{I desire to \slur\ advance the \slur\ all-important
  interests of}, etc.), but habitually we do not.

\section

There are certain very common combinations of quantities with accent,
which, though they occur in English in \emph{groups} of words, do not
occur in any single word, and are therefore strange to us.  These
accordingly require special practice and care at the beginning. The
most important are as follows:

\begin{enumerate}

\item
The combination \scan{śl}, as in \latin{déae}, \latin{déō},
\latin{ámā}, \latin{mónē}, \latin{ténē},
\latin{iúbēs}, \latin{tórō}.  Compare English \english{át
  home} and \english{tó home} in “I said \emph{át home}, not
\emph{tó home}.” The difficulty here is in making the first
syllable really short, and in keeping \emph{all accent off} the second
syllable, while at the same time tranquilly giving it its full length.
This is the hardest Latin combination for modern speakers.

\item
The combination \scan{śsl}, as in \latin{Latiō},
\latin{rapidī}.  Compare English \english{Mérry Mount} (with
the last word lengthened, but not accented).

\item
The combination \scan{lĺs} or \scan{lĺl}, as in
\latin{rēg\'īna}, \latin{rēg\'īnā}. Compare English
\english{whole paílful}, with full length, but no accent, on
\english{whole}, and full length, \english{with} accent, on
\english{pail}.

\item
The combination (much like the preceding) \scan{slĺs}, or
\scan{slĺl}, as in \latin{amā\-b\'ā\-mus},
\latin{tenēb\'ātur}, \latin{trahēb\'ātur}.  Compare
English \english{a whole paílful}, with the \sound{a} short,
and the rest as above.

\item
The combination \scan{lśss}, or \scan{lśsl}, as in
\latin{dīvídimus}, \latin{iūdícia}, \latin{impériō}, \latin{ō\-cé\-an\-ō}.
This may be reproduced in the English \english{no sílliness},
pronounced with a long \english{no}, not accented, and with a short
and accented first syllable in \english{silliness}.

\end{enumerate}

\section

The student should regard the marking of long vowels in writing Latin
simply as a form of spelling, to represent \emph{differences of
  sound}.  Long~\phone{i} and short~\phone{i}, for example, are as
different in Latin as \sound{i} in \english{fit} and \sound{ee} in
\english{feet} in English.

\chapter{Phonetic Changes\protect\footnotemark}
\footnotetext{Only such changes are mentioned as are fairly obvious,
  and involve the relations of existing Latin forms.  There are many
  other changes, a treatment of which is needful and possible only in
  connection with the forms of other languages.

  Changes in the quantity of vowels have been mentioned already
  (\xref{18}–\xref{21}, \xref{26}, \xref[note]{28}); also some
  changes of original diphthongs (\xref[\emph{a}, \emph{b}, \emph{c},
    \emph{d}]{10}).}

\contentsentry{B}{Phonetic Changes}

\headingC{Weakening of Vowels in Medial Syllables}

\section

The vowels of medial syllables are subject to certain modifications
which do not appear in initial syllables.  This is most apparent in
the variation of the root-syllable, observable between compounds and
the simple words from which they are derived, as \latin{faciō}, but
\latin{per-ficiō}.  But the change is not confined to such cases.

\begin{note}[Note 1]

These changes came about at a time when the older accentual system
(\xref[\break note]{31}) prevailed, according to which all but initial
syllables were unaccented.  The slighting of the vowels of unaccented
syllables is common to languages with a strong stress accent, and
nowhere more so than in English, where the result of the weakening is
usually an obscure vowel much like~\sound{u} in \english{but}.  Note,
for example, the pronunciation of \english{drayman},
\english{ploughman} as compared with that of~\english{man}, or the
identical sound given to the \sound{a}, \sound{e}, and~\sound{o} of
\english{currant}, \english{patient}, \english{patriot} (but
\english{patriótic}).  In Latin the weakening takes the form of
replacing the more open vowel by one less open.  So \phone{a} is
changed to~\phone{e}, and \phone{e} frequently to~\phone{i}; similarly
\phone{ai} (\phone{ae}), through \phone{ei}, to~\phone{ī}.
Sometimes, owing to the character of the surrounding sounds, the
change is to~\phone{u}; similarly \phone{au} (through \phone{ou})
to~\phone{ū}. Long vowels are never affected.  Contrast
\latin{ad-āctus} from \latin{āctus} with \latin{ad-fectus} from
\latin{factus}.

\end{note}

\begin{note}[Note 2]

In many compounds the feeling for the connection with the single word
is so strong that the latter appears without change.  So
\latin{circum-agō}, \english{lead around}, \latin{ad-legō},
\english{elect to}, etc.  Sometimes both weakened and unchanged forms
are found: thus from \latin{necō} the compound \latin{ē-nicō} is
found in early Latin, but the usual form is \latin{ē-necō};
\latin{cōn-secrō}, from \latin{sacrō}, remains the usual form,
but \latin{cōn-sacrō} is also found.  This retention or revival of
the form of the simplex in compounds is known as \emph{Recomposition},
and is seen in our pronunciation of \english{man} in \english{iceman},
as contrasted to that given to it in \english{drayman}, or in the
pronunciation \english{fore-head} beside \english{for’ed},
\english{Saturday} beside \english{Saturd’y} (like \english{Mond’y}),
etc.; also in \english{housewife} beside \english{hussy}, which is in
origin the same word.  In uncompounded words there are other
influences which sometimes prevent the usual changes.

\end{note}

\section

The principal changes are as follows:

\subsection

\phone{a} becomes \phone{i} before a single consonant
except~\phone{r}, and before~\phone{ng}; it becomes \phone{e}
before~\phone{r} and before two consonants, and \phone{u} before
\phone{l} + consonant.\footnote{This statement combines the final
  results of several different changes which took place at successive
  periods.}
\begin{longtable}{>{\bfseries}l
                 l@{\,\,}>{\bfseries}l
                 @{\qquad\qquad}
                 l@{\,\,}>{\bfseries}l
                 l@{\,\,}>{\bfseries}l}

agō
&& ad-igō
&& cadō
& Perf. & cecidī
\\

faciō
&& per-ficiō
&& capiō
&& ac-cipiō
\\

tangō
&& at-tingō
&& frangō
&& cōn-fringō
\\

pariō
& Perf. & peperī
&& fallō
& Perf. & fefellī
\\

factus
&& per-fectus
&& captus
&& ac-ceptus
\\

saltō
&& īn-sultō
&& calcō
&& in-culcō

\end{longtable}

\begin{note}

Recomposition (\xref[note 2]{41}) is seen in \latin{circum-agō},
\latin{com-parō}, etc.  In Noun-Stems ending in \phone{a} +
consonant, the \phone{a} of the Nom.\ Sing.\ remains unchanged in the
other cases; e.g.\ \latin{Caesar}, \english{Caesar},
Gen.\ \latin{Caesaris}.

\end{note}

\pagebreak

\subsection

\phone{e}, unless preceded by~\phone{i}, becomes \phone{i} before a
single consonant except~\phone{r}.

\begin{longtable}{>{\bfseries}l
                 l@{\,\,}>{\bfseries}l
                 @{\qquad\qquad}
                 l@{\,\,}>{\bfseries}l
                 l@{\,\,}>{\bfseries}l}

teneō
&& at-tineō
&& regō
&& cor-rigō
\\

sedeō
&& ad-sideō
&& premō
&& com-primō
\\

mīles
& Gen. & mīlitis
& (but & pariēs
& Gen. & parietis\textnormal)

\end{longtable}

\begin{note}

Recomposition is seen in \latin{ad-legō}, \latin{circum-sedeō},
etc.  In forms like \latin{segetis} (Gen.\ of \latin{seges}) as
compared with \latin{mīlitis}, the retention of the~\phone{e} is
due to the assimilating influence of the~\phone{e} of the first
syllable.

\end{note}

\subsection
\phone{ae} becomes~\phone{ī}, and \phone{au} becomes~\phone{ū}.
\begin{longtable}{>{\bfseries}l
                 l@{\,\,}>{\bfseries}l
                  @{\qquad\qquad}
                 l@{\,\,}>{\bfseries}l
                 l@{\,\,}>{\bfseries}l}

quaerō
&& in-quīrō
&& claudō
&& in-clūdō
\\

caedō
& Perf. & cecīdī
&& causa
&& ac-cūsō

\end{longtable}

\begin{note}

But oftener Recomposition takes place, as \latin{ad-haereō},
\latin{ex-audiō}, etc.

\end{note}

\subsection

\phone{av} and \phone{ov} become~\phone{u}.
\begin{Tabular}{>{\bfseries}l
                  @{\quad}
                 >{\bfseries}l
                  @{\qquad\qquad}
                 >{\bfseries}l
                  @{\quad}
                 >{\bfseries}l}

lavō
& ē-luō
& novus
& dēnuō (\rec{dē-novō})

\end{Tabular}

\subsection
\phone{o} becomes~\phone{i} (or \phone{e} if preceded by~\phone{i})
before a single consonant except~\phone{l}; it becomes \phone{u}
before two consonants and, unless preceded by a vowel,
before~\phone{l}.  Examples:
\begin{Tabular}{>{\bfseries}l@{\,\,}l@{\,\,}>{\bfseries}l
                  @{\qquad\qquad}
                  >{\bfseries}l@{\,\,}l@{\,\,}>{\bfseries}l}

īlicō
& from
& \rec{in(s)locō}
& leguntur
& from
& \rec{legontor}
\\

bonitās
& \ditto[from]
& \rec{bono-tās} (bonus)
& porculus
& \ditto[from]
& \rec{porco-los} (porcus)
\\

societās
& \ditto[from]
& \rec{socio-tās} (socius)
& \cc{3}{but \latin{fīliolus} (\latin{fīlius})}

\end{Tabular}

\begin{note}

But the change to~\phone{i} is rare except before suffixes, as in
\latin{bonitās}. In the root-syllable of compounds \phone{o} nearly
always remains unchanged, e.g.\ \latin{ab-rogō}, \latin{con-locō},
\latin{ad-moneō}, etc.  For the change to~\phone{u}, see
  also~\xref[1]{44}.

\end{note}

\subsection

(Note to 1, 2, and 5.) When the vowel of the medial syllable, whether
\phone{a}, \phone{e}, or~\phone{o}, is followed by a labial
(\phone{p}, \phone{b}, \phone{f}, or \phone{m}), it is sometimes
changed to~\phone{u} instead of to~\phone{i},\allowbreak—but not
always, and the reasons for the difference are not clear, except that
the quality of the vowels of the surrounding syllables was a factor.
In some of these words the \phone{u} remained unchanged, but in most
it was eventually supplanted by~\phone{i}. Examples are:
\latin{oc-cupō} (from the root \latin{cap-} of \latin{capiō}) as
compared with \latin{anti-cipō}; \latin{au-cupis}, Gen.\ of
\latin{auceps}, compared with \latin{prīncipis} from
\latin{prīnceps}; \latin{con-tubernālis} (\latin{taberna});
\latin{possumus}, \latin{volumus} compared with \latin{agimus},
\latin{tegimus}; \latin{mancupium} and \latin{mancipium};
\latin{maxumus} and \latin{maximis}, \latin{proxumus} and
\latin{proximus}, etc.  The same variation is seen when the original
vowel was \phone{u} or~\phone{i}, e.g.\ \latin{cornu-fex} and
\latin{corni-fex} (\latin{cornu-}), \latin{pontu-fex} and
\latin{ponti-fex} (\latin{ponti-}), and in some cases of
original~\phone{u} even in initial syllables, e.g.\ \latin{lubet} and
\latin{libet}, \latin{clupeum} and \latin{clipeum}.

\negbigskip

\headingC{Syncope of Vowels}

\enlargethispage{\baselineskip}

\section
\subsection

Short vowels are sometimes lost in medial and final syllables.  So,
for example, \latin{surgō} beside the older \latin{sur-rigō},
\latin{pergō} from \rec{per-rigō}; \latin{reppulī},
\latin{rettulī}, from the reduplicated Perfects
\rec{re-tetulī}, \rec{re-pepulī}; \latin{valdē} beside
\latin{validus}; \latin{caldus}, \latin{soldus}, beside
\latin{calidus}, \latin{solidus}; in final syllables \latin{nec},
\latin{ac}, beside \latin{neque}, \latin{atque} (cf.\ also words
having enclitic \enclitic{-c}, \enclitic{-n}, beside \enclitic{-ce},
\enclitic{-ne}); Nom.\ Sing.\ of \phone{i}-Stems \latin{pars},
\latin{mōns}, etc., from original Nom.\ \rec{partis}, \rec{montis},
Neut.\ \latin{animal} from \latin{animāle}, \rec{animāli}.

\begin{note}

Like the weakening of vowels, this process began under the old
accentual system (\xref[note]{31}), as shown by \latin{rettulī}
from \rec{ré-tetulī}, etc.  Where double forms like
\latin{calidus} and \latin{caldus} exist, the shorter forms are those
of the rapid utterance of everyday speech, and were often used by the
poets.  A similar relation, as regards use, exists between
\latin{perīculum} and \latin{perīclum}, \latin{saeculum} and
\latin{saeclum}, etc.  But in these the shorter forms represent a
retention of, or in part a reversion to, the original formation; the
vowel before~\phone{l} is a secondary development.

\end{note}

\subsection

Syncope is especially common in syllables containing \phone{ro}
and~\phone{ri}, and, if the \phone{r} is not already preceded by a
vowel, an~\phone{e} is developed before it.  So regularly in the
Nominative Singular of stems in \phone{-ro-} and \latin{-ri-}, as
\latin{puer} from \rec{pueros}, \latin{ager} from \rec{agros},
\latin{imber} from \rec{imbris}, \latin{ācer} from \rec{ācris}.
Similarly \latin{sacerdōs} from \rec{sacri-dōs}, \latin{agellus}
(\rec{ager-los}) from \rec{agro-lo-s}, etc.  The successive stages of
development are, for example, \rec{agros}, \rec{agrs}, \rec{agers},
\rec{agerr} (\xref[11]{49}), \latin{ager} (\xref[13]{49}).

\headingC{Change of Vowels in Final Syllables}

\section
\subsection

Change of~\phone{o} to~\phone{u}.  Before final consonants an
original~\phone{o} became~\phone{u}; e.g.\ \latin{hortus},
\latin{hortum}, \latin{illud}, \latin{legunt}, from \rec{hortos},
\rec{hortom}, \rec{illod}, \rec{legont}, the stem-vowel in all such
cases being~\phone{o}.

\begin{minor}

A similar change took place in medial syllables before two consonants
or~\phone{l} (\xref[5]{42}); and even in initial syllables \phone{o}
became~\phone{u} when followed by \phone{l} + consonant or by
\phone{nc}, \phone{ngu}, \phone{mb}; e.g.\ \latin{multa} from
\latin{molta}, \latin{hunc} from \latin{honc}, etc.  In all three
classes of words this change took place in the third century~\bc, and
examples of the original~\phone{o} are found only in the earliest
inscriptions; e.g. \latin{praifectos}, \latin{opos},
\latin{cōsentiont}, \latin{pōcolom}, \latin{molta}, \latin{honc}.

But if the \phone{o} was preceded by \phone{v} or~\phone{u}, it was
retained for nearly two centuries longer, so that \latin{vivos},
\latin{exiguos}, \latin{servos}, \latin{equos}, \latin{relinquont},
\latin{sequontur}, \latin{volt}, \latin{volgus} are the proper forms
not only for Plautus and Terence, but also for Cicero.  And when the
change to~\phone{u} finally came, the product of \latin{quo} and
\latin{guo} was at first \latin{cu}, \latin{gu}, not \latin{quu},
\latin{guu}, which were introduced later; \latin{cum} for earlier
\latin{quom} remained.

\end{minor}

The forms of the different periods may be illustrated as follows:
\begin{Tabular*}{@{}l
                 @{\extracolsep{\fill}}>{\bfseries}l
                 @{\extracolsep{\fill}}>{\bfseries}l
                 @{\extracolsep{\fill}}>{\bfseries}l
                 @{\extracolsep{\fill}}>{\bfseries}l
                 @{}}

Earliest Inscriptions
& hortos & servos & equos & relinquont
\\

Plautus, Cicero
& hortus & \ditto[servos] & \ditto[equos] & \ditto[relinquunt]
\\

Augustan Period
& \ditto[hortus] & servus & ecus & relincunt
\\

Later Imperial Period
& \ditto[hortus] & \ditto[servos] & equus & relinquunt

\end{Tabular*}

\begin{minor}

\subsection

Before final \phone{s} or~\phone{t} an original~\phone{e}
became~\phone{i}; e.g.\ in Verb forms like \latin{legis},
\latin{legit} from earlier \rec{leges}, \rec{leget} (with the
“thematic vowel”~\phone{e}), or Gen.\ Sing.\ \latin{patris}, etc.,
from \rec{patr-es} (the original Genitive ending of consonant-stems
being \phone{-es} or~\phone{-os}).

\subsection

An original final~\phone{i}, if it was not dropped (\xref[1]{43}),
became~\phone{e}; e.g.\ \latin{ante} from \rec{anti}
(cf.\ \latin{anti-cipō}), or Nom.\ Sing.\ Neut.\ \latin{mare},
\latin{sedīle}, etc., from \rec{mari}, \rec{sedīli}
(\phone{i}-Stems).

\subsection

In final syllables original \phone{oi} (which in initial syllables
became \phone{oe}, \phone{ū}; see~\xref[\break\emph{a}]{10}) and \phone{ai}
(\phone{ae}) became first \phone{ei}, then~\phone{ī}.  So
Nom.\ Plur.\ \latin{hortī}, Dat.-Abl.\ \latin{hortīs}, \latin{sellīs},
from early Latin \latin{hortei}, \latin{hor\-teis}, \latin{selleis},
these from earlier \rec{hortoi}, \rec{hortois}, \rec{sellais}.

\end{minor}

\headingC{Contraction of Vowels}

\section

Two like vowels unite to form the corresponding long vowel, as
\latin{nīl} from \latin{nihil}, \latin{bīmus} from
\rec{bi-himus} (\latin{hiems}), \latin{cōpia} from \rec{co-opia},
\latin{currūm} from \latin{curruum} (Gen.\ Plur.).  For the
contraction of two unlike vowels no brief general statement can be
made; examples are: \latin{cōgō} from \rec{co-agō},
\latin{cōmō} from \rec{co-emō}, \latin{dēgō} from
\rec{dē-agō}, \latin{amō} from \rec{amāō}
(cf.\ \latin{moneō}), Subjunctive \latin{amēs} from
\linebreak
\rec{amāēs}.

\headingC{Vowel Gradation}

\section

There are some vowel variations which are not due to any changes
with\-in the Latin language, but are relics of a system of vowel
interchange inherited from the parent
speech,\footnote{\label{ftn:s46:1}That is, the language from which are
  descended not only Latin (with its own descendants French, Italian,
  etc.)\ and the other dialects of ancient Italy (Oscan, Umbrian,
  etc.), but also Greek, the Germanic languages (German, English,
  etc.), the Celtic languages (Irish, Welsh, etc.), the Slavonic
  languages (Russian, etc.), the languages of India and Persia, and
  others. This parent speech is called Indo-European.} and known as
Vowel Gradation,\allowbreak—such as is seen, for example, in English
\english{sing}, \english{sang}, \english{sung}.  An understanding of
the system as a whole cannot be gained from Latin alone, and is
unnecessary here.

The principal variations are:

\begin{flushleft}
\tabcolsep.5\tabcolsep
\begin{tabular}{@{}llcl@{}}

1. & \phone{e},—\phone{o},
& as
& \latin{tegō},—\latin{toga};
  \latin{sequor},—\latin{socius}.
\\

2. & \phone{e},—\phone{ē},
& \ditto[as]
& \latin{tegō},—\latin{tēxī}, \latin{tēgula};
  \latin{sedeō},—\latin{sēdī}, \latin{sēdēs}.
\\

3. &\phone{ī} (earlier \phone{ei}),—(\phone{oe}),—\phone{i},
& \ditto[as]
& \latin{dīcō},—\latin{dictus}, \latin{abdicō};
  \latin{fīdō},—\latin{foedus},—\latin{fidēs}.
\\

4. & \phone{ū} (earlier \rec{eu}, \phone{ou}),—\phone{u},
& \ditto[as]
& \latin{dūcō},—\latin{ductus}, \latin{dux}, Gen.\ \latin{ducis}.
\\

5. & \latin{a},—\latin{ā},
& \ditto[as]
& \latin{scabō},—\latin{scābī};
  \latin{caveō}, \latin{cāvī}.
\\

6. & \phone{o},—\phone{ō},
& \ditto[as]
& \latin{fodiō},—\latin{fōdī};
  \latin{vocō},—\latin{vōx}.
\\

7. & \phone{a},—\phone{ē},
& \ditto[as]
& \latin{agō},—\latin{ēgī};
  \latin{capiō},—\latin{cēpī}.

\end{tabular}
\end{flushleft}

\headingC{Changes of Single Consonants}

\section[Rhotacism]

An \phone{s} between vowels becomes~\phone{r}, as in \latin{generis}
from \REC{ge\-ne\-sis} (Nom.-Acc.\ \latin{genus}), \latin{gerō} from
\rec{gesō} (Perf.\ \latin{ges-sī},
Perf.\ Pass.\ Partic.\ \latin{ges-tus}), \latin{erō} (\latin{es-t}),
\latin{dir-imō} (cf.\ \latin{dis-pōnō}); also \latin{dir-ibeō}
from \latin{habēo}.

\begin{note}

Compare English \english{were} beside \english{was}.  The intermediate
stage between \phone{s} and \phone{r} was the voiced~\phone{s}, the
sound of~\sound{s} in \english{rose} or \sound{z} in \english{zero},
and this was still preserved in the earliest Latin.  Final~\phone{s}
is not subject to this change, but in some nouns, as \latin{honor}
beside \latin{honōs}, \latin{amor}, etc., the \phone{s} which is
proper in the Nom.\ Sing.\ has yielded to the influence of all the
other cases, in which \phone{s} regularly
became~\phone{r}. See~\xref[4, note]{80}; \xref[note]{86}.

\end{note}

\begin{minor}

\section

A final~\phone{d} is lost after long vowels, though still found in
early inscriptions; e.g.\ Abl.\ Sing. \latin{sententiā}, early
\latin{sententiād}, Imperative \latin{estō},
early~\latin{estōd}.

\end{minor}

\headingC{Changes in Consonant Groups}

\section
\subsection

A voiced mute when followed by a voiceless mute or~\phone{s} becomes
itself voiceless; e.g.\ \latin{scrīp-tus}, \latin{scrīp-sī}
(\latin{scrībō}).

\subsection

Not only \phone{g}, but also \phone{qu}, \phone{gu}, and~\phone{h},
become \phone{c} before~\phone{t} or~\phone{s} (\phone{cs} then
appearing as~\phone{x}), as in the Perf.\ Pass.\ Partic., and the
Perf.\ in~\phone{-sī}; e.g.\ \latin{rēctus},
\latin{rēxī} (\latin{regō}), \latin{coctus},
\latin{coxī} (\latin{coquō}), \latin{ūnctus},
\latin{ūnxī} (\latin{unguō}), \latin{vectus},
\latin{vexī} (\latin{vehō}).  And as \phone{v} between vowels
sometimes stands for original \phone{gu},\footnote{The sound-group
  \phone{gu}, parallel in character and origin with \phone{qu}, was
  retained only after \phone{n}, as in \latin{unguō}, etc.
  Otherwise, when followed by a vowel, it lost the~\phone{g},
  appearing then as~\phone{v}, which, in case the preceding vowel
  was~\phone{u}, was itself lost.  Hence \latin{ninguit}, \latin{nix},
  but \latin{nivis}; \latin{frūctus}, but \latin{fruor} (from
  \rec{frūvor}, \rec{frūguor}), etc.} we find \phone{ct}
and~\phone{x} in interchange with~\phone{v};
e.g.\ Nom.\ Sing.\ \latin{nix}, Gen.\ \latin{nivis} (from
\rec{niguis}; cf.\ \latin{ninguit}), \latin{vīxī},
\latin{vīctus} (\latin{vīvō}); similarly \latin{frūctus}
(\latin{fruor}), \latin{flūxī} (\latin{fluō}), etc.

\subsection

A guttural mute is lost between \phone{l} or~\phone{r} and a following
\phone{t}, \phone{s}, \phone{m}, or~\phone{n}; e.g.\ \latin{fultus},
\latin{fulsī} (\latin{fulciō}), \latin{tortus},
\latin{torsī}, \latin{tormentum} (\latin{torqueō}),
\latin{urna} (\latin{urceus}).

\subsection

A dental mute is assimilated to a following~\phone{s}, and the
resulting~\phone{ss} becomes~\phone{s} if standing after a long
syllable, or before another consonant, or if final;
e.g.\ \latin{messuī} from \rec{met-suī} (\latin{metō}),
\latin{clausī}, earlier \latin{claussī} from
\rec{claud-sī} (\latin{claudō}), \latin{aspiciō}
(\latin{ad-apiciō}), \latin{mīles}, earlier \latin{mīless}
(\xref[3]{30}) from \rec{mīlet-s}.

\subsection

When a final dental of a root comes to stand before a suffix beginning
with a dental, the result is~\phone{ss}, which, after a long syllable,
is reduced to~\phone{s}.  So \latin{sessum} from \rec{sed-tum}
(\latin{sedeō}), \latin{fissus} from \rec{fid-tos}
(\latin{findō}), \latin{clausus}, earlier \latin{claussus} from
\rec{claud-tos} (\latin{claudō}), etc.  But if the second dental is
followed by~\phone{r}, the result is~\phone{str};
e.g.\ \latin{rōstrum} from \rec{rōd-trom} (\latin{rōdō}).

\subsection

Original~\phone{ss}, as well as the \phone{ss} arising under the rules
just given, was reduced to~\phone{s} when preceded by a long syllable.
So \latin{hausī} from \latin{haus-sī} (\latin{hauriō} from
\rec{hau\-siō}, \xref{47}), as \latin{clausī} from \latin{claus-sī}~(4),
\latin{clausus} from \latin{claussus}~(5).  \phone{Ll} sometimes
suffers a similar reduction, as in \latin{mīlia} from \latin{mīllia},
\latin{paulum} from \latin{paul\-lum}.

\begin{minor}

\subsubsection

The \phone{ss} remains in the contracted Perfect forms, like
\latin{amāsse} beside \latin{amā\-vis\-se}, and in the short forms of
\latin{edō}, \english{eat}, as \latin{ēsse}, \latin{ēssētur}.

\end{minor}

\subsection

A \phone{p} is sometimes inserted between \phone{m} and a following
\phone{t} or~\phone{s}; e.g.\ \latin{ēmptus} (\latin{emō}),
\latin{sūmpsī} (\latin{sūmō}), \latin{hiemps} beside
\latin{hiems}.

\subsection

Dental and labial mutes are assimilated to a following guttural, and
dentals to labials.  So \latin{ac-cidō} from \rec{ad-cadō},
\latin{siccus} from \rec{sit-cos} (\latin{sitis}), \latin{oc-cidō}
from \rec{ob-cadō}, \latin{ap-pāreō} from \latin{ad-pāreō},
etc.

\subsection

A nasal is assimilated to the class of the following mute;
e.g.\ \latin{im-putō} (\latin{in-putō}), \latin{eundem}
(\latin{eum-dem}), \latin{prīnceps} with guttural~\phone{n}
(\latin{prīmus}).

\subsection

Labial and dental mutes when followed by a nasal become nasals, and,
if the preceding syllable is long, \phone{mm} becomes~\phone{m}. So:
\begin{phonology}

\latin{summus},
& from
& \rec{sup-mos} (\latin{super})
& \latin{somnus},
& from
& \rec{sop-nos} (\latin{sopor})
\\

\latin{mamma}
& \ditto
& \rec{mad-mā} (\latin{madeō})
& \latin{rāmus}
& \ditto
& \rec{rād-mos} (\latin{rādīx})

\end{phonology}

\subsection

\phone{dl}, \phone{ld}, \phone{nl}, \phone{ln}, \phone{rl}, \phone{ls}
become \phone{ll}, and \phone{rs} becomes~\phone{rr}. So:
\begin{phonology}

\latin{sella},
& from
& \rec{sed-lā} (\latin{sedeō})
& \latin{sallō},
& from
& \rec{saldō} (English \english{salt})
\\

\latin{corōlla}
& \ditto
& \rec{corōn-lā} (\latin{corōna})
& \latin{collis}
& \ditto
& \rec{colnis}
\\

\latin{agellus}
& \ditto
& \rec{ager-los}
& \latin{velle}
& \ditto
& \rec{vel-se} (cf.\ \latin{es-se})
\\

\cc{6}{\latin{ferre} from \rec{fer-se}}

\end{phonology}

\subsection

An~\phone{s}, or group of consonants ending in~\phone{s}, is dropped
before voiced consonants, and the preceding vowel, if short, is
lengthened. So:
\begin{phonology}

\latin{bīnī},
& from
& \rec{bis-nī} (\latin{bis})
& \latin{lūna},
& from
& \rec{louc-snā} (\latin{lūceō})
\\

\latin{prīmus}
& \ditto
& \rec{prīs-mos} (cf.\ \latin{prīs-cus})
& \latin{pīlum}
& \ditto
& \rec{pīns-lom} (\latin{pīnsō})
\\

\multicolumn{3}{@{}l}{\latin{īdem} (Nom.\ Sing.\ Masc.),
    from \rec{is-dem}}
& \latin{sēvirī}
& \ditto
& \latin{secs-virī} (\latin{sex})

\end{phonology}

\subsection[\textbf{Finals}]

Double consonants at the end of a word are simplified.  So
%
\latin{os}, \english{bone}, from \rec{oss} (Gen.\ \latin{ossis});
%
\latin{mīles}, from \latin{mīless}, \rec{mīlets} (4);
%
\latin{mel} from \rec{mell}, \rec{meld} (Gen.\ \latin{mellis}; see~11);
%
\latin{far} from \rec{farr}, \rec{fars} (Gen.\ \latin{farris}; see~11);
%
\latin{ager} from \rec{agerr}, \rec{agers} (11, \xref[2]{43}).
%
Note also \latin{cor} from \latin{cord} (Gen.\ \latin{cordis}) and
\latin{lac} from \latin{lact} (Gen.\ \latin{lactis}).

\begin{minor}

\subsubsection

In Nom.-Acc.\ \latin{hoc} from \latin{hocc}, \rec{hod-c}~(8),
the double consonant was retained, in pronunciation, before a vowel;
in early Latin also \latin{mīless}, etc. See \xref[2,~3]{30}.

\end{minor}

\headingC{Assimilation in Compounds}

\section

When assimilation takes place in compounds, the changes are nearly all
such as have just been mentioned.  But assimilation is often absent,
owing to the influence of the separate form of the word which is the
first member of the compound.  This is the same principle of
Recomposition that often prevents the regular vowel changes in the
second member of compounds (\xref[note 2]{41}).

\begin{minor}

Thus the Nom.-Acc.\ Sing.\ Neut.\ of \latin{quisquam} is regularly
\latin{quicquam} (rarely \latin{quidquam}), but that of
\latin{quisque} is regularly \latin{quidque} (rarely \latin{quicque});
while from \latin{quisquis} both \latin{quidquid} and \latin{quicquid}
were in common use, and from \latin{quis\-pi\-am} both \latin{quippiam}
and \latin{quidpiam}.

The greatest variation is seen in the so-called prepositional
compounds, that is, compounds with adverbial prefixes, most of which
occur separately as Prepositions.  For certain combinations
assimilation predominates from the earliest period; in others only
the unassimilated form is in use until a late period.  So, for
example, spellings like \latin{accipiō}, \latin{attineō} are more
common at all periods than \latin{adcipiō}, \latin{adtineō}, and,
though the latter forms are sometimes found in imperial times, it is
doubtful if the recomposition affected anything but the spelling.  On
the other hand, spellings like \latin{adferō}, \latin{adsignō},
\latin{conlocō}, etc., prevailed to the almost total exclusion of
\latin{afferō}, \latin{assignō}, \latin{collocō} until several
centuries after Christ, so that we must believe that \latin{ad} and
\latin{con} were actually so pronounced in such words.  Yet here again
there are special cases.  For example, the spelling \latin{conlēgium},
exclusively employed down to the time of Augustus, gives way to
\latin{collēgium} in the Augustan period, though \latin{conlocō} and
other similar forms continue to prevail until a much later period.

\end{minor}

\section

The following are the forms of the adverbial prefixes according to the
normal spelling.  For the sake of convenience, the few variations not
coming under the head of assimilation are also mentioned.

\begin{minor}

\subsection

\latin{Ab} remains unchanged before \phone{d}, \phone{g}, \phone{l},
\phone{n}, \phone{r} and~\phone{s}, is replaced by \latin{abs} before
\phone{t} and~\phone{c}, by \phone{as} before~\phone{p}, by \phone{au}
before~\phone{f}, by \phone{ā} before~\phone{m}, and before
\phone{f} in \latin{ā-fuī}.  Examples: \latin{ab-dō},
\latin{ab-gregō}, \latin{ab-luō}, \latin{ab-nuō},
\latin{ab-rumpō}, \latin{ab-solvō}, \latin{abs-tineō},
\latin{abs-condō}, \latin{as-portō}, \latin{au-ferō},
\latin{ā-mittō}.

\subsection

\latin{Ad} is assimilated before \phone{t}, \phone{c}, and~\phone{p},
as \latin{at-tineō}, \latin{ac-cipiō}, \latin{ap-pāreō}.  (But
before \phone{p} in verbs other than \latin{appellō},
\latin{appāreō}, \latin{apparō}, the spelling with~\grapheme{d}
is very frequent, as \latin{ad-probō}, etc.)  The \phone{ad} remains
unchanged before~\phone{b} (\latin{ad-bibō}), \phone{m}
(\latin{ad-mittō}), \phone{q} (\latin{ad-quiēscō}), \phone{g}
(\latin{ad-gredior}, but \latin{ag-gerō} frequently), \phone{f}
(\latin{ad-ferō}), \phone{s} (\latin{ad-signō}), \phone{n}
(\latin{ad-numerō}).  Before~\phone{l} it usually remains unchanged,
as \latin{ad-luō}, \latin{ad-legō}, etc., but in \latin{al-ligō}
(\latin{-āre}) and \latin{al-lātus} the assimilated form is
preferable.  Before~\phone{r} it usually remains unchanged, as
\latin{ad-rogō}, etc., but is assimilated in \latin{ar-ripiō} and
\latin{ar-rigō}.  Before \phone{gn}, \phone{sc}, \phone{sp},
and~\phone{st}, it is assimilated (\latin{ag-gn}, \latin{as-sc},
\latin{as-sp}, \latin{as-st}), and one of the two like consonants is
dropped, as \latin{agnōscō}, \latin{ascrībō},
\latin{aspiciō}, \latin{astō}, etc.  But in many words the
unassimilated form is also frequent, in some even preferable.  So
\latin{agnātus} and \latin{adgnātus}, \latin{agnōscō} and
\latin{adgnōscō}; \latin{ascendō} and \latin{adscendō},
\latin{ascrībō} and \latin{adscrībō},
\latin{ascīscō} and \latin{adscīscō};
\latin{aspīrō} and \latin{adspīrō}, \latin{aspiciō}
and (less commonly) \latin{adspiciō}, but regular \latin{aspergō},
\latin{aspernor}; \latin{astō} and \latin{adstō}, but usually
\latin{adstipulor}, \latin{adstringō} and \latin{adstruō}.

\subsection

\latin{Amb} (older \latin{ambi}), seen in \latin{amb-igō},
\latin{amb-ūrō}, etc., becomes \phone{am} before a consonant, as
\latin{am-plector}, \latin{am-putō}.

\subsection

\latin{Ante} appears as \latin{anti} (its original form) in
\latin{anti-cipō}, \latin{anti-stes}, and sometimes in
\latin{anti-stō}.

\subsection

\latin{Circum} becomes \latin{circu} in \latin{circu-eō} beside
\latin{circum-eō}.

\enlargethispage{\baselineskip}

\subsection

\latin{Cum} appears as \phone{con} before \phone{t}, \phone{d},
\phone{c}, \phone{q}, \phone{g}, \phone{s}, \phone{f}, and~\phone{v};
as \phone{com} before \phone{p}, \phone{b}, and~\phone{m}.
Before~\phone{l} the unassimilated form is preferable except in
\latin{col-ligō} and its compounds, e.g.\ \latin{con-locō},
\latin{con-loquium}, \latin{con-lāpsus}, etc.  But before~\phone{r}
the assimilated form is preferable, as \latin{cor-rumpō},
\latin{cor-ripiō}, etc.  Before vowels, \phone{h}, and \phone{gn} the
form is \phone{co}, as \latin{co-alēscō}, \latin{co-haereō},
\latin{co-gnōscō} (from \latin{gnōscō}, the older form of
\latin{nōscō}). Before \phone{n} the form is \phone{cō}, as
\latin{cō-nīveō}, \latin{cō-nectō}.  \latin{Comb-ūrō}
is probably formed after the analogy of \latin{amb-ūrō}.  Before
consonantal~\phone{i} the proper form is \phone{con}, as
\latin{con-iungō}, \latin{con-iūrō}, etc.; so \latin{con-iciō}
from \latin{con-ieciō}, but also \latin{co-iciō} (\xref[1]{30}),
like \latin{co-alēscō}.

\subsection

\latin{Dis} remains unchanged before \phone{t}, \phone{c}, \phone{q},
\phone{p} and~\phone{s} (but when this is followed by a consonant, one
\phone{s} is dropped), becomes \phone{dif} before~\phone{f},
\phone{dī} before voiced consonants, and \phone{dir} before
vowels.  Examples: \latin{dis-tendō}, \latin{dis-clūdō},
\latin{dis-quīrō}, \latin{dis-pōnō}, \latin{dis-solvō},
\latin{di-scrībō}, \latin{dif-ferō}, \latin{dī-dō},
\latin{dī-gerō}, \latin{dī-moveō},
\latin{dī-numerō}, \latin{dī-luō},
\latin{dī-rigō}, \latin{dī-vulgō},
\latin{dī-iūdicō}, \latin{dir-imō}.  But \latin{dis}
sometimes appears in place of~\phone{dī}, as in
\latin{dis-rumpō} beside \latin{dī-rumpō}, and regularly in
\latin{disiciō}.

\subsection

\latin{Ex} remains unchanged before \phone{t}, \phone{c}, \phone{q},
\phone{p}, and~\phone{s}, but becomes \phone{ē} before voiced
consonants.  Examples: \latin{ex-tendō}, \latin{ex-clūdō},
\latin{ex-quīrō}, \latin{ex-pendō}, \latin{ex-scrībō},
\latin{ē-dīcō}, \latin{ē-gerō}, \latin{ē-bibō},
\latin{ē-mittō}, \latin{ē-ligō}, \latin{ē-numerō},
\latin{ē-rumpō}, \latin{ē-vocō}, \latin{ē-iūrō}.
Before~\phone{f} a form \phone{ec} was used, which became \phone{ef},
as in \latin{ef-ferō}, \latin{ef-ficiō} (earlier
\latin{ec-ferō}, \latin{ec-ficiō}).

\subsection

\latin{In} remains unchanged before \phone{t}, \phone{d}, \phone{c},
\phone{q}, \phone{g}, \phone{n}, \phone{f}, \phone{v}.  Before
\phone{p}, \phone{b}, \phone{m} it becomes \phone{im}, as
\latin{im-perō}, \latin{im-bibō}, \latin{im-mittō} though the
spelling \latin{in-perō}, etc., is also found.  Before \phone{gn}
the \phone{n} is lost, as \latin{ignōscō}.  Before \phone{l}
and~\phone{r}, \latin{in} remains unchanged until a very late period,
as \latin{in-lūstris}, \latin{in-lātus}, \latin{in-rumpō},
\latin{in-rogō}, etc.  A form \phone{ind}, representing an early
\phone{indu} (cf.\ \latin{indu-perator}, \latin{indi-gena}), is seen
in \latin{ind-igeō} (\latin{egeō}), \latin{ind-ipīscor}
(\latin{apīscor}), and \latin{ind-uō} (cf.\ \latin{exuō}).

\subsection

\latin{Inter} remains unchanged except in \latin{intel-legō}.

\subsection

\latin{Ob} is assimilated before \phone{c}, \phone{p}, and \phone{f},
as \latin{oc-cidō}, \latin{op-pōnō}, \latin{of-ferō}.  It
appears as \phone{o} in \latin{o-mittō}, as \phone{om} in
\latin{om-mūtēscō} beside \latin{ob-mūtēscō}, and as
\phone{os} (from \latin{obs}) in \latin{os-tendō}.  Elsewhere it is
retained.

\subsection

\latin{Per} remains unchanged except that it is sometimes assimilated
to a following~\phone{l}.  So \latin{pel-legō} and
\latin{pel-liciō}, preferable to \latin{per-legō},
\latin{per-liciō}, but \latin{per-lūceō} preferable to
\latin{pel-lūceō}, and always \latin{per-luō},
\latin{per-lūstrō}, \latin{per-lātus}.

\subsection

\latin{Por}, a form related to \latin{prō}, and seen in
\latin{por-tendō}, \latin{por-riciō}, \latin{por-rigō}, is
assimilated in \latin{pol-luō}, \latin{pol-liceor},
\latin{pos-sideō}.  For \latin{prō}, \latin{pro}, \latin{prōd},
see~\xref[2]{24}.

\subsection

\latin{Sub} is treated for the most part like \latin{ob}, but before
some words beginning with \phone{t} or~\phone{c} it appears as
\phone{sus} (from \phone{subs}).  So \latin{sus-tineō},
\latin{sus-tulī}, beside \latin{sub-trahō};
\latin{sus-cēnseō}, \latin{sus-cipiō}, beside
\latin{suc-cumbō}, \latin{suc-cidō}.  \latin{Sub-spiciō} becomes
\latin{suspiciō}, but \latin{sub-scrībō} is more usual than
\latin{suscrībō}.  Before~\phone{r}, \latin{sub} remains
unchanged, except in \latin{sur-ripiō} and \latin{sur-rēxī},
Perf.\ of \latin{surgō}; e.g.\ \latin{sub-rogō},
\latin{sub-ruō}, \latin{sub-rīdeō}, etc.
\latin{Sum-mittō}, \latin{sum-moveō} are preferable to
\latin{sub-mittō}, \latin{sub-moveō}, which are examples of late
recomposition.

\subsection

\latin{Re} appears as \phone{red} before vowels and~\phone{h}, as
\latin{red-hibeō}, \latin{redeō}, \latin{red-igō}, etc.; also in
\latin{red-dō}, and in early Latin \latin{red-dūcō} (usually
\latin{re-dūcō}).

\subsection

\latin{Trāns} becomes \phone{trā} before \phone{d}, \phone{n},
and~\phone{v}, as \latin{trādō}, \latin{trā-dūcō},
\latin{trānō}, \latin{trāvehō}, etc.; also
\latin{trā\-i\-ciō}.  But \latin{trāns-dūcō}, etc., are also
found.

\end{minor}

\headingB{Orthography}

\contentsentry{B}{Orthography}

\section

The spelling of many Latin words varied in different periods, or even
in the same period.  Our traditional orthography is that of the first
century~\ad, and we retain this as the normal spelling for school
grammars, and for school editions even of authors like Cicero, the
spelling of whose time was somewhat different.  Some of the more
important classes of variations are as follows:

\subsection

The earlier forms of \latin{servus}, \latin{equus}, \latin{vult},
etc., were \latin{servos}, \latin{equos}, \latin{ecus}, \latin{volt},
etc.  See \xref[1]{41}.

\subsection

For a long time the spelling varied between \phone{u} and~\phone{i} in
\latin{maxumus}, \latin{maximus}, \latin{optumus}, \latin{optimus},
\latin{lubet}, \latin{libet}, etc., but the spelling with \grapheme{i}
finally became the normal one.  See~\xref[6]{42}.

\subsection

The reduction of \phone{ss} to~\phone{s} and \phone{ll} to~\phone{l}
has been mentioned (\xref[6]{49}).  The spelling with one \grapheme{s}
or \grapheme{l} is occasionally found before the Augustan period, and
becomes universal in the first century.  We should write
\latin{causa}, \latin{clausus}, \latin{mīlia},
\latin{paulum},—not \latin{caussa}, \latin{claussus},
\latin{mīllia}, \latin{paullum}.

\subsection

Where \phone{ī} stands for an original diphthong (\xref[\emph{c}]{10};
\xref[3]{42}; \xref[4]{44}) \grapheme{ei} is the common spelling down
through the time of Cicero; e.g.\ \latin{deicō}
(\latin{dīcō}), Nom.\ Plur.\ \latin{servei}
(\latin{servī}), etc,

\subsection

Owing to the reduction of \phone{n} before~\phone{s} (\xref{11}), the
\phone{n} is frequently omitted in inscriptions.  In the Numeral
Adverbs and in the Ordinals like \latin{vīcēnsimus} the
omission is frequent in manuscripts also, and we often find
\latin{totiēs} beside \latin{totiēns}, \latin{deciēs} beside
\latin{deciēns}, \latin{vīcēsimus} beside
\latin{vīcēnsimus}, etc.  But the full forms are to be
preferred.

\subsection

There was much uncertainty at all periods in the use of
initial~\grapheme{h}; for example, \latin{harēna}, \latin{haruspex},
\latin{haedus}, \latin{holus}, beside the incorrect \latin{arēna},
\latin{aruspex}, \latin{aedus}, \latin{olus}, and \latin{erus},
\latin{umerus}, \latin{ūmidus}, beside the incorrect \latin{herus},
\latin{humerus}, \latin{hūmidus}.  See~\xref[note]{11}.

\subsection

For variation in the spelling of compounds, see \xref{50}, \xref{51}.

\part{Inflection}

\section

The \textbf{Parts of Speech} are the same as in English, except that
there is no Article.

\begin{minor}
Definitions of the Parts of Speech are given under Syntax
in~\xref{221}.
\end{minor}

\section

Nouns, Adjectives (including Participles), Pronouns, and Verbs are
capable of \textbf{Inflection}, or change of form expressing the
varied relations of the word to the other parts of the sentence. In
the case of Nouns, Adjectives, and Pronouns such inflection is called
\textbf{Declension}; in the case of Verbs, it is called
\textbf{Conjugation}.

\chapter{Declension}

\section

Declension comprises the variations in Gender, Number, and Case.

\headingB{Gender}

\contentsentry{B}{Gender}

\section

The Genders are three, Masculine, Feminine, and Neuter.

\subsubsection

\term{Natural Gender} is simply the distinction of sex, the names of
males being Masculine, those of females being Feminine, and those of
things without sex being Neuter.

\subsubsection

\term{Grammatical Gender} is a distinction of form as manifested
either by the Noun itself, by an Adjective agreeing with it, or by a
Pronoun agreeing with or referring to it.

\headingC{The Relation of Gender to Signification}

\section

Grammatical gender, which is commonly meant by the term Gender as
applied in grammar, has a marked connection with natural gender, but
is by no means identical with it.\footnote{In English, where almost
  the only surviving sign of grammatical gender is that of the
  Pronouns \english{he}, \english{she}, \english{it}, this agrees with
  natural gender; for the feeling of sex-distinction (or, in the case
  of \english{it}, lack of or indifference to sex-distinction) is
  always associated with these words,—even when used metaphorically
  of inanimate objects (as \english{she} of a ship).

  The view that \emph{all} grammatical gender, for example as seen in
  Latin, is nothing but metaphorical sex-distinction, is losing
  ground.}  In Latin the grammatical gender of names of persons and of
most animals follows the natural gender, but the names of inanimate
objects are as often Masculine or Feminine as Neuter.  For these the
gender is determined simply by the \emph{form},—of the Noun itself, or if,
as is often the case, the form of the Noun is not sufficiently
characteristic of gender, by the form of an Adjective agreeing with
it, or a Pronoun agreeing with or referring to it.  What the forms
characteristic of gender are will be shown under the separate
Declensions, and, moreover, the gender of all Nouns will be marked.

\section

Certain general statements may, however, be made which will help in
remembering the gender of many words.

\subsection

All Names of \emph{Months} and \emph{Winds}, and most names of
\emph{Rivers}, are Masculine. Examples:
\begin{examples}

\latin{Aprīlis}, \english{April};
\latin{Eurus}, \english{the southeast wind};
\latin{Tiberis}, \english{the Tiber}.

\end{examples}

\subsection

Most names of \emph{Trees}, \emph{Plants}, \english{Cities},
\english{Countries}, and \english{Islands} are Feminine. Examples:
\begin{examples}

\latin{fīcus}, \english{fig tree};
\latin{crocus}, \english{crocus};
\latin{Corinthus}, \english{Corinth};
\latin{Aegyptus}, \english{Egypt};
\latin{Cyprus}, \english{Cyprus}.

\end{examples}

\subsection

\emph{Indeclinable Nouns}, \emph{Substantive Clauses},
\emph{Infinitives used substantively}, and \emph{quoted expressions},
are Neuter.
\begin{examples}

\latin{nihil}, \english{nothing};
\latin{tōtum hoc philosophārī}, \english{all this philosophizing};
\latin{istuc taceō}, \english{that “I’ll be still” of yours}.

\end{examples}

\begin{minor}

\subsubsection

With reference to statements 1 and~2, observe the gender of the
corresponding general words: \latin{mēnsis}, \english{month},
\gender{m.}\ (the names of the months are really Adjectives),
\latin{ventus}, \english{wind}, \gender{m.}, \latin{fluvius},
\latin{amnis}, \english{river}, \gender{m.},—but \latin{arbor},
\english{tree}, \gender{f.}, \latin{planta}, \english{plant},
\gender{f.}, \latin{urbs}, \english{city}, \gender{f.}, \latin{terra},
\english{country}, \gender{f.}, \latin{īnsula},
\english{island},~\gender{f.}

\subsubsection

Many words belonging to the classes mentioned under~2 are not
Feminine.  Forms with distinctly Neuter endings, as \latin{Latium},
\latin{Leuctra} (Plur.), \latin{Reāte}, are Neuter; also many names of
plants in~\suffix{-er}, Gen.~\latin{-eris}, as \latin{piper},
\english{pepper}.  Names of cities and countries in~\suffix{-ī}
(Plur.), as \latin{Delphī}, are Masculine.  But Feminines greatly
predominate, since they include not only the forms with distinctly
Feminine endings, but also most of the numerous forms in~\suffix{-us},
Gen.~\suffix{-ī}.

\end{minor}

\section
\subsection

Certain words are of common gender, that is, they are Masculine or
Feminine according to the sex referred to, as \latin{cīvis},
\english{citizen} (male or female), \latin{bōs}, \english{ox} or
\english{cow}.

\subsection

Certain names of animals have a fixed gender without regard to the sex
referred to, as \latin{vulpēs}, \english{fox}, always Feminine,
\latin{ānser}, \english{goose} and \english{gander}, always Masculine.

\negbigskip\negbigskip

\headingB{Number}

\contentsentry{B}{Number}

\section

There are, as in English, two Numbers, the Singular and the Plural.

\headingB{The Cases}

\contentsentry{B}{Case}

\section

There are six Cases:
\begin{Tabular}{>{\bfseries}l@{ }c@{ }l}

Nominative: & the       & case of the subject; \\

Genitive:   & \ditto    & \emph{of} case; \\

Dative:     & \ditto    & \emph{to} or \emph{for} case; \\

Accusative: & \ditto    & case of the direct object, etc.; \\

Vocative:   & \ditto    & case of address; \\

Ablative:   & \ditto    & \emph{from}, \emph{with}, or \emph{in} case.

\end{Tabular}

\pagebreak

\begin{minor}

The meanings given are only for purposes of identification, the uses
of the cases being treated in detail under the head of Syntax.

\end{minor}

\subsubsection

All but the Nominative and Vocative are called \term{Oblique Cases}.

\subsubsection

There were originally two other cases, the \term{Locative} and the
\term{Instrumental} (or Sociative).  They are, for the most part,
merged with the Ablative.  But the Locative is still preserved in many
names of places and adverbial expressions.

\section
\subsection

The Cases are distinguished by different endings, known as
Case-Endings. These are not the same for all Declensions, and in
Pronouns some few endings are used which are unknown in the declension
of Nouns and Adjectives.

\begin{note}

In reality the difference between corresponding case-forms of the
various Declensions is largely one of \emph{Stem}, that is, the base
to which the endings are added.  This is evident, for example, in the
Nominatives~\suffix{-us}, \suffix{-is}, \suffix{-ēs}, in which the
ending proper is the same, namely,~\phone{s}.  Yet sometimes the
ending, too, is different, for example in the Dative and Ablative
Plural, where the~\suffix{-īs} of the First and Second Declensions has
no connection with the~\suffix{-ibus} of the Third, Fourth, and Fifth.
Very often, in the case of stems ending in a vowel, the line between
the stem and the ending proper is not apparent on the surface, owing
to contraction and to other phonetic changes affecting either the stem
or the ending; so that, for practical purposes, we apply the term
Case-Endings to \emph{certain variable parts}, which, in the case of
vowel-stems, include both the final vowel of the stem and the ending
proper.  In the case of consonant-stems, the variable terminations are
also the true case-endings.

\end{note}

\subsection

The Nominative and Accusative are alike in all Neuters, both in the
Singular and in the Plural.\footnote{Hence we speak of the
  Nom.-Acc.\ Sing.\ Neut.\ as a single form; likewise of the
  Nom.-Acc.\ Plur.\ Neut., the Nom.-Voc.\ Sing.\ or Plur., or the
  Dat.-Abl.\ Plur.}

\subsection

The Nominative and Vocative are always alike in the Plural, and also,
except in Masculines and Feminines of the Second Declension, in the
Singular.\footnotemark[\thefootnote]
\subsection

The Dative and Ablative are always alike in the
Plural.\footnotemark[\thefootnote]

\headingB{Nouns}

\contentsentry{B}{Declension of Nouns}

\section

In the declension of Nouns there are five distinct types,
distinguished by different \emph{Stems}.  These are known as the Five
Declensions.  The form of the Genitive Singular is chosen as a
convenient characteristic of each. Thus:
\begin{Tabular}{c@{\enskip}r@{\qquad}l@{\qquad}l}

& & \emph{Stem ends in}: & \emph{Gen.\ Sing.\ ends in}: \\

\textsc{Declension} &   I   & \phone{ā} & \phone{ae} \\

\ditto              &  II   & \phone{o} & \phone{ī} \\

\ditto              & III   & \phone{i} or a consonant & \phone{is} \\

\ditto              &  IV   & \phone{u} & \phone{ūs} \\

\ditto              &  V    & \phone{ē} & \phone{ēī}

\end{Tabular}

\pagebreak

\section

The scheme of the normal endings is as follows:
\begin{Tabular}{>{\itshape}l
                     >{\bfseries}l
                     >{\bfseries}l
                     @{\extracolsep{3em}}
                     >{\bfseries}l
                     @{\extracolsep{1em}}
                     >{\bfseries}l}

& \cc{2}{\textsc{Declension I}}
& \cc{2}{\textsc{Declension II}}
\\

& \textsc{singular}
& \textsc{plural}
& \textsc{singular}
& \textsc{plural}
\\

Nom. & a    & ae    & us, er\tup; \gender{n.} um & ī\tup; \gender{n.} a \\
Gen. & ae   & ārum  & ī                      & ōrum \\
Dat. & ae   & īs    & ō                      & īs \\
Acc. & am   & ās    & um                     & ōs\tup; \gender{n.} a \\
Voc. & a    & ae    & e\tup, er\tup; \gender{n.} um  & ī\tup; \gender{n.} a \\
Abl. & ā    & īs    & ō                      & īs

\end{Tabular}

\begin{Tabular}{>{\itshape}l
                      >{\bfseries}l
                      @{\extracolsep{2em}}
                      >{\bfseries}l
                      >{\bfseries}l}

\cc{4}{\textsc{Declension III}} \\

\cc{4}{\textsc{singular}} \\

     & \cc{1}{\emph{Consonant-Stems}} && \cc{1}{\emph{\phone{i}-Stems}} \\

Nom.
    & s, \na;\footnote{A dash indicates that the ending or, in the
                       case of a vowel-stem, both ending and
                       stem-vowel are lacking.  But the
                       Nom.-Voc.\ Sing.\ in~\suffix{-er} from
                       \phone{o}-Stems or \phone{i}-Stems, though
                       likewise lacking both
                       ending and stem-vowel (\xref{70}, \xref{87}),
                       is, for greater convenience, given
                       as~\suffix{-er}.}
      \gender{n.} \na
    &
    & is, ēs, er; \gender{n.} e, \na\footnotemark[\thefootnote] \\

Gen. &                          & is \\
Dat. &                          & ī  \\
Acc. & em\tup; \gender{n.} \na      &   & em\tup, im\tup; \gender{n.} e\tup, \na \\
Voc. & s\tup, \na\tup; \gender{n.} \na  &   & is\tup, ēs\tup, er\tup; \gender{n.} e\tup, \na \\
Abl. & e                        &   & e\tup, ī
\\[\medskipamount]

\cc{4}{\textsc{plural}} \\

Nom. & \gender{n.} a
     & \gender{m.}, \gender{f.} ēs
     & \gender{n.} ia
\\

Gen. & um                       &       & ium \\
Dat. &                          & ibus  &  \\
Acc. & ēs\tup; \gender{n.} a
     &
     & īs (ēs)\footnote{Here, and in general in examples of
                        inflection, forms inclosed in~(~) are variants
                        which are less common in the best period.};
       \gender{n.} ia \\

Voc. & \gender{n.} a
     & \gender{m.}\tup, \gender{f.} ēs
     & \gender{n.} ia
\\

Abl. &                          & ibus

\end{Tabular}

\begin{note}

The endings given in the middle column are those which are always the
same for both Consonant-Stems and \phone{i}-Stems.

\end{note}

\begin{Tabular}{>{\itshape}l
                     >{\bfseries}l
                     >{\bfseries}l
                     @{\extracolsep{3em}}
                     >{\bfseries}l
                     @{\extracolsep{1em}}
                     >{\bfseries}l}

& \cc{2}{\textsc{Declension IV}}
& \cc{2}{\textsc{Declension V}}
\\

& \textsc{singular}
& \textsc{plural}
& \textsc{singular}
& \textsc{plural}
\\

Nom. & us\tup; \gender{n.} ū    & ūs\tup; \gender{n.} ua
     & ēs                   & ēs \\

Gen. & ūs       & uum
     & ēī\tup, eī   & ērum \\

Dat. & uī\tup, ū\tup; \gender{n.} ū & ibus\tup, ubus
     & ēī\tup, eī               & ēbus \\

Acc. & um\tup; \gender{n.} ū   & ūs\tup; \gender{n.} ua
     & em                  & ēs \\

Voc. & us\tup; \gender{n.} ū    & ūs\tup; \gender{n.} ua
     & ēs                   & ēs \\

Abl. & ū    & ibus\tup, ubus
     & ē    & ēbus

\end{Tabular}

\chapter{First Declension}

\contentsentry{C}{First Declension}

\subtitle{\phone{ā}-\emph{Stems}}

\smallskip

\section

The Nominative Singular ends in short~\phone{a}, which stands for
original~\suffix{-ā}.  Example of Declension:
\begin{Tabular}{>{\itshape}l l @{\extracolsep{2.5em}} l}

\cc{3}{\latin{sella}, \english{seat}, \gender{f.}} \\
\cc{3}{(stem \stem{sellā-})} \\

& \textsc{singular}
& \textsc{plural}
\\

Nom. & \latin{sella}, \english{a \(the\) seat}
     & \latin{sellae}, \english{\(the\) seats} \\

Gen. & \latin{sellae}, \english{of a \(the\) seat}
     & \latin{sellārum}, \english{of \(the\) seats} \\

Dat. & \latin{sellae}, \english{to \emph{or} for a \(the\) seat}
     & \latin{sellīs}, \english{to \emph{or} for \(the\) seats} \\

Acc. & \latin{sellam}, \english{a \(the\) seat}
     & \latin{sellās}, \english{\(the\) seats} \\

Voc. & \latin{sella}, \english{\(O\) seat}
     & \latin{sellae}, \english{\(O\) seats} \\

Abl. & \latin{sellā}, \english{from, with, \emph{or} in a \(the\) seat}
     & \latin{sellīs}, \english{from, with, \emph{or} in \(the\) seats}

\end{Tabular}

\headingC{Remarks on the Case-Forms}

\section
\subsection

An old form of the Genitive Singular in~\ending{-ās} is preserved in
\latin{familiās}, used in such phrases as \latin{pater familiās},
\english{head of the household}, etc.

\subsection

A Genitive form in~\ending{-āī} is found in poetry, as \latin{aulāī},
\english{of the court}.

\subsection

A Genitive Plural in~\ending{-um} beside that in~\ending{-ārum} is
found in compounds of \suffix{-cola}, \english{dwelling in}, and
\suffix{-gena}, \english{descendant of}; also in \latin{amphora},
\english{a liquid measure}, \latin{drach\-ma}, \english{a Greek coin},
and in many proper names of Greek origin, as \latin{Aeneadae},
\latin{Lapithae}, etc.  So \latin{agricolum}, \latin{Troiugenum},
\latin{amphorum}, \latin{Aeneadum}, etc.

\begin{note}

This is not a contraction of~\ending{-ārum}.  The compounds of
\suffix{-cola} and~\suffix{-gena}, being Masculines, fell under the
influence of the Second Declension, in which \ending{-um} is an old
ending.  All the other words mentioned are of Greek origin, and in
these the \ending{-um} corresponds to the Greek ending.

\end{note}

\subsection

For the Dative and Ablative Plural of \latin{fīlia},
\english{daughter}, and \latin{dea}, \english{goddess}, the forms
\latin{fī\-li\-ā\-bus} and \latin{deābus} are frequently used to avoid
confusion with the corresponding cases of \latin{fīlius},
\english{son}, and \latin{deus}, \english{god}.  So in the phrases
\latin{fīliīs et fīliābus}, \latin{diīs deābusque}.  In other words
\ending{-ābus} is rare.

\subsection

There is a Locative Singular form which is identical with the
Genitive, as \latin{Rōmae}, \english{at Rome}.  In the Plural the form
is the same as the Dative and Ablative, as \latin{Athēnīs},
\english{at Athens}.

\begin{minor}

\subsection

The Ablative Singular once ended in~\ending{-ād}, which is preserved
in early inscriptions, e.g. \latin{sententiād}.

\subsection

The Dative and Ablative Plural once ended in~\ending{-ais}, which
first became \ending{-eis} (still preserved in the spelling of the
Ciceronian period, e.g.\ \latin{vieis}), then~\ending{-īs}.

\end{minor}

\headingC{Gender}

\section

Words of the First Declension are Feminine, except a few referring to
male persons, as \latin{nauta}, \english{sailor}, \latin{agricola},
\english{farmer}; also \latin{Hadria}, \english{the Adriatic}.

\headingC{Greek Nouns}

\section

Greek nouns of the First Declension often retain their proper Greek
forms in some cases of the singular.  The resulting mixture of Greek
and Latin declension may be seen in the following examples:
\begin{Tabular}{>{\itshape}l@{\quad}l@{\qquad}l@{\qquad}l}

Nom. & Aenē\ending{ās}
     & Anchīs\ending{ēs}
     & Andromach\ending{ē}, \ending{-a} \\

Gen. & Aenē\ending{ae}
     & Anchīs\ending{ae}
     & Andromach\ending{ēs}, \ending{-ae} \\

Dat. & Aenē\ending{ae}
     & Anchīs\ending{ae}
     & Andromach\ending{ae} \\

Acc. & Aenē\ending{ān},    \ending{-am}
     & Anchīs\ending{ēn},   \ending{-am}
     & Andromach\ending{ēn}, \ending{-am} \\

Voc. & Aenē\ending{ā}
     & Anchīs\ending{ē}, \ending{-ā}, \ending{-a}
     & Andromach\ending{ē}, \ending{-a} \\

Abl. & Aenē\ending{ā}
     & Anchīs\ending{ē}, \ending{-ā}
     & Andromach\ending{ē}, \ending{-ā} \\

\end{Tabular}

\begin{note}

Many proper names of the Greek First Declension are inflected in Latin
according to the Third Declension, as \latin{Aeschinēs},
\latin{Miltiadēs}.  Ablatives like \latin{Anchīsē} are formed
according to the Fifth Declension.

\end{note}

\chapter{Second Declension}

\contentsentry{C}{Second Declension}

\subtitle{\phone{o}-\emph{Stems}}

\smallskip

\section

The Nominative Singular ends in~\ending{-us}, or, in the case of
Neuters, in~\ending{-um}.  These endings were originally \ending{-os},
\ending{-om} (\xref[1]{44}).  Examples of Declension:
\begin{Tabular}{>{\itshape}l
                     l @{\extracolsep{1em}} l
                     @{\extracolsep{3em}}
                     l @{\extracolsep{1em}} l}

& \cc{2}{\latin{hortus}, \english{garden}, \gender{m.}}
& \cc{2}{\latin{dōnum}, \english{gift}, \gender{n.}}
\\

& \cc{2}{(stem \stem{horto-})}
& \cc{2}{(stem \stem{dōno-})}
\\

& \cc{1}{\textsc{singular}} & \cc{1}{\textsc{plural}}
& \cc{1}{\textsc{singular}} & \cc{1}{\textsc{plural}}
\\

Nom.    & hort\ending{us}   & hort\ending{ī}
        & dōn\ending{um}    & dōn\ending{a} \\

Gen.    & hort\ending{ī}    & hort\ending{ōrum}
        & dōn\ending{ī}     & dōn\ending{ōrum} \\

Dat.    & hort\ending{ō}    & hort\ending{īs}
        & dōn\ending{ō}     & dōn\ending{īs} \\

Acc.    & hort\ending{um}   & hort\ending{ōs}
        & dōn\ending{um}    & dōn\ending{a} \\

Voc.    & hort\ending{e}    & hort\ending{ī}
        & dōn\ending{um}    & dōn\ending{a} \\

Abl.    & hort\ending{ō}    & hort\ending{īs}
        & dōn\ending{ō}     & dōn\ending{īs} \\

\end{Tabular}

\section

Most stems in \infix{-ro-}, in the Nominative Singular, drop \phone{o}
and~\phone{s} of the original ending and insert an \phone{e} before
the~\phone{r}, if the latter is not already preceded by a vowel.  See
\xref[2]{43}.  Examples:
\begin{Tabular}{>{\itshape}l
                  l
                  @{\extracolsep{2em}}
                  l
                  @{\extracolsep{2em}}
                  l}

& \cc{1}{\latin{puer}, \english{boy},   \gender{m.}}
& \cc{1}{\latin{ager}, \english{field}, \gender{m.}}
& \cc{1}{\latin{vir},  \english{man},   \gender{m.}}
\\

& \cc{1}{(stem \latin{puero-})}
& \cc{1}{(stem \latin{agro-})}
& \cc{1}{(stem \latin{viro-})}
\\[\jot]

& \cc{3}{\textsc{singular}} \\

Nom.    & puer
        & ager
        & vir
\\

Gen.    & puer\ending{ī}
        & agr\ending{ī}
        & vir\ending{ī}
\\

Dat.    & puer\ending{ō}
        & agr\ending{ō}
        & vir\ending{ō}
\\

Acc.    & puer\ending{um}
        & agr\ending{um}
        & vir\ending{um}
\\

Voc.    & puer
        & ager
        & vir
\\

Abl.    & puer\ending{ō}
        & agr\ending{ō}
        & vir\ending{ō}
\\[\jot]\pagebreak

& \cc{3}{\textsc{plural}} \\

Nom.    & puer\ending{ī}
        & agr\ending{ī}
        & vir\ending{ī}
\\

Gen.    & puer\ending{ōrum}
        & agr\ending{ōrum}
        & vir\ending{ōrum}
\\

Dat.    & puer\ending{īs}
        & agr\ending{īs}
        & vir\ending{īs}
\\

Acc.    & puer\ending{ōs}
        & agr\ending{ōs}
        & vir\ending{ōs}
\\

Voc.    & puer\ending{ī}
        & agr\ending{ī}
        & vir\ending{ī}
\\

Abl.    & puer\ending{īs}
        & agr\ending{īs}
        & vir\ending{īs}

\end{Tabular}

\subsubsection

A few stems in \infix{-ro-} are declined like \latin{hortus}.  So
\latin{erus}, \english{master}, \latin{numerus}, \english{number},
\latin{umerus}, \english{shoulder}, \latin{uterus}, \english{womb},
\latin{hesperus}, \english{evening star}, \latin{taurus},
\english{bull}.

\subsubsection
The words like \latin{puer}, in which the stem is~\infix{-ero},
are:
\latin{gener},     \english{son-in-law},
\latin{socer},     \english{father-in-law},
\latin{adulter},   \english{adulterer},
\latin{Līber},     \english{god of wine},
\latin{līberī},    \english{children},
\latin{vesper},    \english{evening},
\latin{signi-fer}, \english{standard-bearer},
\latin{armi-ger},  \english{armor-bearer},
and other compounds of \suffix{-fer} and~\suffix{-ger}.

\headingC{Remarks on the Case-Forms}

\section
\subsection

Stems in \infix{-vo-}, \infix{-quo-}, \infix{-uo-} retained~\phone{o} in the
Nom.\ and Acc.\ Sing.\ until the end of the Ciceronian period;
e.g.\ Nom.\ \latin{servos}, \latin{equos}, \latin{mortuos}
(Adjective), Acc.\ \latin{servom}, \latin{equom}, \latin{mortuom},
Nom.-Acc.\ Neut.\ \latin{aevom}.  The forms of the Augustan period are
\latin{servus}, \latin{ser\-vum}, etc.,—but \latin{ecus},
\latin{ecum}, the forms \latin{equus}, \latin{equum} being still
later. See~\xref[1]{44}.

\subsection

Proper Names in \suffix{-ius} form their Genitive Singular
in~\suffix{-ī} (not~\latin{-iī}) and also their Vocative Singular
in~\ending{-ī} (not~\latin{-ie}).  The accent is on the penult, even
when it is short.  So \latin{Vergilius}, \latin{Servius},
\latin{Tullius}, \latin{Gāius} have Gen.\ and Voc.\ Sing.\
\latin{Vergílī}, \latin{Servī}, \latin{Tullī}, \latin{Gāī}.

\begin{note}

In such Proper Names, in contrast to the other nouns in \suffix{-ius},
\suffix{-ium} (see~3), the \phone{ī} of the Genitive is \emph{not}
generally replaced by~\phone{-iī}, though the latter is occasionally
found.

\end{note}

\subsection

Other nouns in \suffix{-ius} and \suffix{-ium} also form the Genitive
Singular in~\ending{-ī}, which, however, begins to be replaced
by~\ending{-iī} in the Augustan period.  Such forms as
\latin{imperium}, \latin{ingenium}, \latin{negōtium} have
Gen.\ \latin{impérī}, \latin{ingénī}, \latin{negōtī} in Virgil and
Horace, as well as in the earlier poets, but nearly always
\latin{imperiī}, \latin{ingeniī}, \latin{negōtiī} in Propertius, Ovid,
and later poets.  \latin{Fīlius}, \english{son}, has
Gen.\ Sing.\ \latin{fīlī} and also Voc.\ Sing.~\latin{fīlī}.

\subsection

A Genitive Plural in~\ending{-um} is found:
\begin{enuma}

\item
Usually in words denoting coins or measures, such as \latin{nummus},
\english{coin}, \latin{sestertius}, \english{sesterce},
\latin{modius}, \english{measure}, \latin{iūgerum}, \english{acre};
e.g.\ \latin{nummum}, \english{of coins}, etc.

\item
Frequently in \latin{deus}, \english{god}, \latin{socius},
\english{ally}, \latin{līberī}, \english{children}, and \latin{faber}
in the phrase \latin{praefectus fabrum}, \english{chief engineer}.

\item
Occasionally, in poetry, in \latin{vir}, \english{man}, and some other
words.

\end{enuma}

\subsection

\latin{Deus}, \english{god}, has Voc.\ Sing.\ \latin{deus},
Nom.\ Plur.\ \latin{dī}, Gen.\ Plur.\ \latin{deum} beside
\latin{deōrum}, Dat.-Abl.\ Plur.~\latin{dīs}.

\begin{note}

The forms \latin{dī} and \latin{dīs} were sometimes \emph{spelled}
\latin{diī}, \latin{diīs}, but were regularly \emph{pronounced} as one
syllable.  The forms \latin{deī}, \latin{deīs} represent a dissyllabic
pronunciation, which, however, is rare before Ovid.

\end{note}

\subsection

The Locative Singular form is identical with the Genitive;
e.g.\ \latin{humī}, \english{on the ground}, \latin{domī}, \english{at
  home}, \latin{Corinthī}, \english{at Corinth}.  In the Plural the
form is the same as the Dative and Ablative; e.g.\ \latin{Delphīs},
\english{at Delphi}.

\begin{minor}

\subsection

The Ablative Singular once ended in~\ending{-ōd}, which is preserved
in early inscriptions; e.g.\ \latin{prei\-vā\-tōd}.

\subsection

The Nom.\ Plur.\ and Dat.-Abl.\ Plur.\ ended originally
in~\ending{-oi} and~\ending{-ois}, which first became \suffix{-ei},
\suffix{-eis} (still preserved in the spelling of the Ciceronian
period; e.g.\ \latin{servei}, \latin{serveis}), then
\ending{-ī},~\ending{-īs}.

\end{minor}

\headingC{Gender}

\section

Nouns of the Second Declension ending in ~\ending{-us}, \ending{-er},
\ending{-ir} are mostly Masculine; those in \ending{-um} are Neuter.

\subsubsection

Feminine are:
\begin{enumerate}

\item
Most names of Cities, Countries, and Islands, as \latin{Corinthus},
\latin{Aegyptus}, \latin{Rhodus}, etc.

\item
Most names of Trees and Plants, as \latin{fāgus}, \english{beech},
\latin{fīcus}, \english{fig tree}.

\item
Some Greek Feminines, as \latin{dialectus}, \english{dialect},
\latin{diphthongus}, \english{diphthong}.

\item
Also the following: \latin{alvus}, \english{belly}, \latin{carbasus},
\english{flax}, \latin{colus}, \english{distaff}, \latin{humus},
\linebreak
\english{ground}, \latin{vannus}, \english{winnowing-fan}.

\end{enumerate}

\subsubsection

Neuters are: \latin{vīrus}, \english{poison}, \latin{pelagus},
\english{sea}, \latin{vulgus}, \english{crowd}, \english{rabble}
(sometimes \gender{m.}), in which the Accusative has the same form as
the Nominative.  These words have no Plural, except that for
\latin{pelagus}, which is a Greek word, a
Nom.-Acc.\ Plur.\ \latin{pelagē} is sometimes found.

\begin{note}

In reality these words are Heteroclites (\xref{107}), the
Nom.-Acc.\ form belonging to the Neuters of the Third Declension like
\latin{corpus}, \latin{genus}, etc.~(\xref{85}).

\end{note}

\headingC{Greek Nouns}

\section

Greek Nouns of the Second Declension usually follow the Latin
declension.  But the Nominative and Accusative Singular often end in
\ending{-os}, \ending{-on}, especially in proper names in poetry.
Thus \latin{Tenedos}, Acc.\ \latin{Tenedon} (also \ending{-us},
\ending{-um}), Nom.-Acc.\ \latin{Īlion} (also \latin{Īlium}).

\begin{minor}

\subsubsection

\latin{Androgeōs} has Gen.\ Sing.\ \ending{-eō} and \ending{-eī},
Acc.\ Sing.\ \latin{-eō} and \latin{-eōn}.  \latin{Panthūs} has
Voc.\ Sing.\ \latin{Panthū}.

\end{minor}

\chapter{Third Declension}

\contentsentry{C}{Third Declension}

\section

The Third Declension comprises:
\begin{enumA}

\item
Consonant-Stems, with various subdivisions, according to the nature of
the final consonant.

\pagebreak

\item
\phone{I}-Stems.

\item
Mixed Stems, of which the Singular is declined like that of
Consonant-Stems of the Mute Class, and the Plural like that of
\phone{i}-Stems.\footnote{\label{ftn:36:}There are other, less common,
  forms of mixture between Consonant-Stems and \phone{i}-Stems, which
  are more conveniently treated as individual varieties of one or the
  other of these types.  Words like \latin{mūs},
  Gen.\ Plur.\ \latin{mūrium}, are cited under \phone{s}-Stems.  The
  few forms like \latin{canis}, Gen.\ Plur.\ \latin{canum}, which show
  a combination just the opposite of that seen in the Mixed Stems, are
  mentioned under \phone{i}-Stems.  Nouns in \suffix{-ēs},
  Gen.\ Sing.\ \suffix{-is}, are classed under \phone{i}-Stems,
  although the \suffix{-ēs} itself is probably not formed from an
  \phone{i-Stem} (it perhaps originated in certain \phone{s}-Stems,
  existing beside \phone{i}-Stems formed from the same root, and was
  then extended to other \phone{i}-Stems).  Certain \hbox{\phone{i}-Stems},
  like \latin{imber}, Gen.\ \latin{imbris}, and the Neuters in
  \suffix{-al}, Gen.\ \suffix{-ālis}, \suffix{-ar},
  Gen.\ \suffix{-āris}, which have lost the \phone{i} by regular
  phonetic change, have come to resemble some Consonant-Stems in the
  Nominative Singular; but they are classed where they belong, under
  \phone{i}-Stems.

  Many of the words classed under Mixed Stems are also, in origin,
  \phone{i}-Stems which have lost the \phone{i} in the Nom.\ Sing.;
  e.g.\ \latin{pars} from \rec{parti-s} (cf.\ the Adverb
  \latin{partim}), \latin{gēns} from \rec{genti-s}, and many others
  which contain the once common suffix \suffix{-ti-}.  But it is not
  practicable to separate these from others which are properly
  Mute-Stems that have been drawn into this type, and from still
  others in which the variation between Mute-Stem and \phone{i}-Stem
  is inherited from the parent speech.
  
  Under Mixed Stems, then, are included \emph{not} all forms of
  mixture between Consonant-Stems and \phone{i}-Stems, but only that
  particular and widespread type in which the Singular is like that of
  Mute-Stems.}

\item
Some Irregular Nouns, including Stems in \suffix{-ū} and~\suffix{-ov}.

\end{enumA}

\section

Consonant-Stems and \phone{i}-Stems originally followed two totally
distinct types of declension, which have been partially confused in
Latin, so that many of the forms are identical in both classes.  But
the distinction is wholly or partially preserved in several of the
cases,—most clearly in the Genitive Plural.  See the scheme of
endings given in~\xref{64}, and contrast the declension of \latin{rēx}
(\xref{76}) with that of \latin{turris}~(\xref{87}).

\subsubsection

In Masculine and Feminine \phone{i}-Stems the original endings of the
Acc.\ and Abl.\ Sing., namely \ending{-im} and~\ending{-ī}, were at an
early period supplanted in most words (for exceptions, see below,
under \phone{i}-Stems) by \ending{-em} and~\ending{-e}, the endings of
Con\-so\-nant-Stems; but in the Acc.\ Plur.\ the original \ending{-īs} was
not superseded until after the Augustan period, though \ending{-ēs}
was also used as early as Cicero’s time.  Neuter \phone{i}-Stems
nearly always retain \suffix{-ī} in the Abl.\ Sing.; the
Nom.-Acc.\ Sing.\ ended originally in~\ending{-i}, but this is
regularly changed to~\ending{-e} (\xref[3]{44}), or
dropped~(\xref[1]{43}).

\begin{note}

Most of those forms which in Latin are identical in both types
belonged originally to only one type.  So the ending of the
Gen.\ Sing.\ \ending{-is} (from \ending{-es}) belonged properly only
to Consonant-Stems, but in prehistoric times replaced the ending of
the \phone{i}-Stems, which otherwise would have been \suffix{-īs} in
Latin; and the \ending{-ēs} of the Nom.\ Plur.\ Masc.\ and
Fem.\ belonged only to \phone{i}-Stems, the ending of Consonant-Stems
being properly \suffix{-es}, which would have became \suffix{-is}.
So, too, the \suffix{-ibus} of the Dat.\ and Abl.\ Plur., found in all
Stems, belongs properly only to the \phone{i}-Stems.  The \ending{-ī}
of the Dat.\ Sing.\ may belong to either \phone{i}-Stems or
Consonant-Stems, or both.

\end{note}

\headingF{Consonant-Stems}

\subtitle{\textsc{Mute-Stems}}

\section

Examples:
\begin{Tabular}{>{\itshape}l
                  l
                  @{\extracolsep{2em}}
                  l
                  @{\extracolsep{2em}}
                  l}

& \cc{1}{\latin{rēx},      \english{king},  \gender{m.}}
& \cc{1}{\latin{prīnceps}, \english{chief}, \gender{m.}}
& \cc{1}{\latin{pēs},      \english{foot},  \gender{m.}}
\\

& \cc{3}{\textsc{singular}} \\

Nom.    & rēx
        & prīnceps
        & pēs
\\

Gen.    & rēg\ending{is}
        & prīncip\ending{is}
        & ped\ending{is}
\\

Dat.    & rēg\ending{ī}
        & prīncip\ending{ī}
        & ped\ending{ī}
\\

Acc.    & rēg\ending{em}
        & prīncip\ending{em}
        & ped\ending{em}
\\

Voc.    & rēx
        & prīnceps
        & pēs
\\

Abl.    & rēg\ending{e}
        & prīncip\ending{e}
        & ped\ending{e}
\\

& \cc{3}{\textsc{plural}} \\

Nom.    & rēg\ending{ēs}
        & prīncip\ending{ēs}
        & ped\ending{ēs}
\\

Gen.    & rēg\ending{um}
        & prīncip\ending{um}
        & ped\ending{um}
\\

Dat.    & rēg\ending{ibus}
        & prīncip\ending{ibus}
        & ped\ending{ibus}
\\

Acc.    & rēg\ending{ēs}
        & prīncip\ending{ēs}
        & ped\ending{ēs}
\\

Voc.    & rēg\ending{ēs}
        & prīncip\ending{ēs}
        & ped\ending{ēs}
\\

Abl.    & rēg\ending{ibus}
        & prīncip\ending{ibus}
        & ped\ending{ibus}

\end{Tabular}

\begin{Tabular}{>{\itshape}l
                  l @{\extracolsep{1em}} l
                  @{\extracolsep{3em}}
                  l @{\extracolsep{1em}} l}

& \cc{2}{\latin{mīles}, \english{soldier}, \gender{m.}}
& \cc{2}{\latin{caput}, \english{head},    \gender{n.}}
\\

& \textsc{singular} & \textsc{plural}
& \textsc{singular} & \textsc{plural}
\\

Nom.    & mīles     & mīlit\ending{ēs}
        & caput     & capit\ending{a}
\\

Gen.    & mīlit\ending{is}  & mīlit\ending{um}
        & capit\ending{is}  & capit\ending{um}
\\

Dat.    & mīlit\ending{ī}   & mīlit\ending{ibus}
        & capit\ending{ī}   & capit\ending{ibus}
\\

Acc.    & mīlit\ending{em}   & mīlit\ending{ēs}
        & caput              & capit\ending{a}
\\

Voc.    & mīles              & mīlit\ending{ēs}
        & caput              & capit\ending{a}
\\

Abl.    & mīlit\ending{e}    & mīlit\ending{ibus}
        & capit\ending{e}    & capit\ending{ibus}

\end{Tabular}

\headingC{Changes in the Stems\protect\footnotemark}

\footnotetext{These remarks apply also to nouns of the Mixed Type,
  which are declined as Mute-Stems in the singular~(\xref{90}), and to
  Adjective Mute-Stems~(\xref{117}).}

\section
\subsection

In the Nom.-Voc.\ Sing.\ Masc.\ and Fem., the ending~\ending{s}
combines with a final guttural of a Stem to form~\ending{x}, with a
dental to form~\ending{s}, and with a labial to form \ending{ps}
or~\ending{bs} (\emph{pronounced}~\phone{ps}); e.g.\
\latin{vōx}, \english{voice} (\rec{vōc-s}),
\latin{rēx}, \english{king} (\rec{rēg-s});
\latin{mīles}, \english{soldier} (\rec{mīlet-s}),
\latin{pēs}, \english{foot} (\rec{pēd-s});
\latin{prīnceps}, \english{chief} (\rec{prīncep-s}),
\latin{trabs}, \english{beam} (\rec{trab-s}).
The final consonant has been lost in the Neuters
\latin{cor}, \english{heart} (Gen.\ \latin{cordis}), and
\latin{lac}, \english{milk} (Gen.\ \latin{lactis}).

\subsection

The vowel of the stem generally remains unchanged in all the cases;
e.g.\
\latin{dux}, \english{leader}, Gen.\ \latin{ducis};
\latin{lūx}, \english{light}, Gen.\ \latin{lūcis};
\latin{custōs}, \english{guard}, Gen.\ \latin{custōdis};
\latin{virtūs}, \english{manliness}, Gen.\ \latin{virtūtis};
\latin{lapis}, \english{stone}, Gen.\ \latin{lapidis}, etc.
But:

\pagebreak

\subsection

An interchange of \phone{ē} in the Nom.-Voc.\ Sing.\ with~\phone{e} in
the other cases is seen in \latin{pēs} and its compounds, also in
\latin{abiēs}, \english{fir},
\latin{ariēs}, \english{ram},
\latin{pariēs}, \english{wall};
e.g.\ Gen.\ \latin{pedis}, \latin{abietis}, etc.

\subsection

In words of more than one syllable in which the vowel of the
Nom.-Voc.\ Sing.\ is short~\phone{e}, this is regularly weakened
to~\phone{i} in the other cases (\xref[2]{42}).  So
\latin{auspex}, \english{soothsayer}, Gen.\ \latin{auspicis};
\latin{prīnceps}, \english{chief}, Gen.\ \latin{prīncipis};
\latin{mīles}, \english{soldier}, Gen.\ \latin{mīlitis};
\latin{obses}, \english{hostage}, Gen.\ \latin{obsidis},
etc.
%
Such forms are very numerous, but there are some exceptions, as
\latin{seges}, \english{crop}, Gen.\ \latin{segetis}
(so \latin{teges}, \latin{praepes}, \latin{interpres},
\latin{indiges}).

\subsection

In \latin{anceps}, \english{fowler}, Gen.\ \latin{aucupis}, the
weakening results in~\phone{u} (\xref[6]{42}).  In early Latin also
\latin{manceps}, \english{contractor}, Gen.\ \latin{mancupis}.  An
interchange of \phone{u} and~\phone{i} is seen in \latin{caput},
\english{head}, Gen.\ \latin{capitis}.

\subsection

\latin{Supellēx}, \english{furniture}, has
Gen.\ \latin{supellēctilis}, etc.

\section[Gender]

Neuter are only:
\latin{cor}, \english{heart},
\latin{lac}, \english{milk},
\latin{caput}, \english{head}.
Masculine are:
nouns in \ending{-es}, Gen.\ \ending{-itis};
\ending{-eps}, Gen.\ \suffix{-ipis};
most in \latin{-ex}, Gen. \latin{-icis}.

Feminine are:
nouns in \ending{-tūs}, Gen. \ending{-tūtis};
\ending{-tās}, Gen. \ending{-tātis};
most of the commonest nouns in~\ending{-x} (except those in
\ending{-ex}, \ending{-icis}; see above);
but \latin{grex}, \latin{rēx}, \gender{m.},
\latin{dux}, \latin{coniux} or \latin{coniūnx},
\gender{m.}\ or~\gender{f.}


\begin{note}

Other classes vary too much between Masculine and Feminine to be
brought under any general statement.

\end{note}

\subtitle{\textsc{Liquid Stems}}

\section

Examples:

\subtitle{\emph{Masculines \(and Feminines\)}}

\begin{Tabular}{>{\itshape}l
                  l
                  @{\extracolsep{2em}}
                  l
                  @{\extracolsep{2em}}
                  l}

& \cc{1}{\latin{victor}, \english{victor}, \gender{m.}}
& \cc{1}{\latin{pater},  \english{father}, \gender{m.}}
& \cc{1}{\latin{cōnsul}, \english{consul}, \gender{m.}}
\\[\jot]

& \cc{3}{\textsc{singular}} \\

Nom.    & victor
        & pater
        & cōnsul
\\

Gen.    & victōr\ending{is}
        & patr\ending{is}
        & cōnsul\ending{is}
\\

Dat.    & victōr\ending{ī}
        & patr\ending{ī}
        & cōnsul\ending{ī}
\\

Acc.    & victōr\ending{em}
        & patr\ending{em}
        & cōnsul\ending{em}
\\

Voc.    & victor
        & pater
        & cōnsul
\\

Abl.    & victōr\ending{e}
        & patr\ending{e}
        & cōnsul\ending{e}
\\[\jot]

& \cc{3}{\textsc{plural}} \\

Nom.    & victor\ending{ēs}
        & patr\ending{ēs}
        & cōnsul\ending{ēs}
\\

Gen.    & victōr\ending{um}
        & patr\ending{um}
        & cōnsul\ending{um}
\\

Dat.    & victōr\ending{ibus}
        & patr\ending{ibus}
        & cōnsul\ending{ibus}
\\

Acc.    & victōr\ending{ēs}
        & patr\ending{ēs}
        & cōnsul\ending{ēs}
\\

Voc.    & victor\ending{ēs}
        & patr\ending{ēs}
        & cōnsul\ending{ēs}
\\

Abl.    & victōr\ending{ibus}
        & patr\ending{ibus}
        & cōnsul\ending{ibus}

\end{Tabular}

\pagebreak

\subtitle{\emph{Neuters}}

\begin{Tabular}{>{\itshape}l
                  l @{\extracolsep{1em}} l
                  @{\extracolsep{3em}}
                  l @{\extracolsep{1em}} l}

& \cc{2}{\latin{ebur},  \english{ivory}}
& \cc{2}{\latin{tūber}, \english{swelling}}
\\

& \textsc{singular} & \textsc{plural}
& \textsc{singular} & \textsc{plural}
\\

Nom.    & ebur      & ebor\ending{a}
        & tūber     & tūber\ending{a}
\\

Gen.    & ebor\ending{is}     & ebor\ending{um}
        & tūber\ending{is}    & tūber\ending{um}
\\

Dat.    & ebor\ending{ī}     & ebor\ending{ibus}
        & tūber\ending{ī}    & tūber\ending{ibus}
\\

Acc.    & ebur     & ebor\ending{a}
        & tūber    & tūber\ending{a}
\\

Voc.    & ebur      & ebor\ending{a}
        & tūber     & tūber\ending{a}
\\

Abl.    & ebor\ending{e}     & ebor\ending{ibus}
        & tūber\ending{e}    & tūber\ending{ibus}

\end{Tabular}

\headingC{Remarks}

\section
\subsection

The type represented by \latin{victor} is the most common, comprising
the nouns of agency in \ending{-tor}, and many abstracts in
\ending{-or}, as \latin{amor}, \english{love}.  The stem is
\ending{-tōr} or~\ending{-ōr} throughout, except in the
Nom.-Voc.\ Sing., where the vowel has been shortened before the
final~\phone{r} (\xref[3]{26}).

\subsection

Like \latin{pater} are inflected \latin{māter}, \english{mother},
\latin{frāter}, \english{brother}, \latin{accipiter}, \english{hawk},
and a few proper names.

\subsection

Other Masculines are declined like \latin{cōnsul} in that the stem
remains unchanged throughout.  So, for example,
\latin{vigil}, \english{watchman}, Gen.\ \latin{vigilis};
\latin{sōl}, \english{sun}, Gen.\ \latin{sōlis};
\latin{ānser}, \english{goose}, Gen.\ \latin{ānseris};
\latin{augur}, \english{augur}, Gen.\ \latin{auguris};
\latin{Caesar}, \english{Caesar}, Gen.\ \latin{Caesaris}.

\subsection

\latin{Honor}, \english{honor}, Gen.\ \latin{honōris}, and
\latin{arbor}, \english{tree}, Gen.\ \latin{arboris}, were originally
\phone{s}-Stems, and the old Nominatives \latin{honōs} and
\latin{arbōs} (like \latin{flōs}, \xref{85}) are frequently found.

\begin{note}

Many others of the words classed here as \phone{r}-Stems were
originally \phone{s}-Stems, some of them showing traces of~\phone{s}
in early Latin.  This is true of the whole class of abstracts in
\ending{-or}, \ending{-ōris} mentioned under~1, and of several
Neuters, as \latin{rōbur} (old Latin \latin{rōbus};
cf.\ \latin{rōbustus}), \latin{fulgur}, \latin{aequor}, etc.  So also
\latin{mulier}, \english{woman}, \gender{f.}, and \latin{vōmer},
\english{ploughshare}, \gender{m.}, beside which is found
\latin{vōmis} (like \latin{cinis}, \xref{85}). See~\xref[note]{86}.

\end{note}

\subsection

Other Neuters declined like \latin{ebur} are \latin{rōbur},
\english{oak}, \latin{femur}, \english{thigh}, \latin{iecor},
\english{liver}.  But \latin{femur} has also \latin{feminis},
\latin{feminī}, etc., formed from an \phone{n}-Stem; and \latin{iecur}
(\latin{iocur} in the Augustan period) has Gen.\ \latin{iocineris}
beside \latin{iecoris}.

\subsection

Other Neuters declined like \latin{tūber} are \latin{ūber},
\english{teat}, \latin{cadāver}, \english{dead body}, \latin{cicer},
\english{pea}, \latin{piper}, \english{pepper}, and several names of
plants and trees.  \latin{Iter}, \english{way}, has
Gen.\ \latin{itineris}, etc. (cf.\ \latin{io\-ci\-ne\-ris},~5).

\subsection

There are also Neuters in \ending{-ar}, Gen.\ \ending{-aris};
\ending{-or}, Gen.\ \ending{-oris}; and \ending{-ur},
Gen.\ \ending{-uris}; e.g.\
\latin{nectar}, \english{nectar},
\latin{aequor}, \english{sea},
\latin{fulgur}, \english{lightning},
\latin{Tībur}, \english{Tivoli},
etc.;
also \latin{vēr}, \english{spring}, Gen.\ \latin{vēris};
\latin{far}, \english{spelt}, Gen.\ \latin{farris} (stem
\stem{farr-} from \rec{fars-});
\latin{sāl}, \english{salt}, Gen.\ \latin{salis};
\latin{mel}, \english{honey}, Gen.\ \latin{mellis} (stem \stem{mell-}
from \rec{meld-});
\latin{fel}, \english{gall}, Gen.\ \latin{fellis} (stem \stem{fell-}
from \rec{feld-}).

% \pagebreak

\section[Gender]

Liquid Stems are nearly all Masculine or Neuter.

Masculine are: nouns in \suffix{-tor}, Gen.\ \suffix{-tōris};
\latin{-or}, Gen.\ \latin{-ōris}, except, of course, \latin{soror},
\english{sister}, \gender{f.}, and \latin{uxor}, \english{wife},
\gender{f.}; \suffix{-er}, Gen.\ \suffix{-ris}, except \latin{māter},
\english{mother}, \gender{f.}; \suffix{-l}, except the Neuters
\latin{sāl}, \latin{mel}, \latin{fel}.

Neuter are: nouns in \ending{-ur}, Gen.\ \ending{-oris}; \ending{-or},
Gen.\ \ending{-oris}, except \latin{arbor}, \english{tree},
\gender{f.}

Masculines and Neuters are included in nouns in \ending{-er},
Gen.\ \ending{-eris} (but \latin{mulier}, \english{woman},
\gender{f.}); \ending{-ar}, Gen.\ \ending{-aris}; \ending{-ur},
Gen.~\ending{-uris}.

\subtitle{\textsc{Nasal Stems}}

\addvspace{-\smallskipamount}

\section

Examples:

\begin{Tabular}{>{\itshape}l
                  l @{\extracolsep{1em}}
                  l
                  l}

& \cc{1}{\latin{sermō}, \english{speech}, \gender{m.}}
& \cc{1}{\latin{virgō}, \english{virgin}, \gender{f.}}
& \cc{1}{\latin{nōmen}, \english{name},   \gender{n.}}
\\[\smallskipamount]

& \cc{3}{\textsc{singular}} \\

Nom.    & sermō
        & virgō
        & nōmen
\\

Gen.    & sermōn\ending{is}
        & virgin\ending{is}
        & nōmin\ending{is}
\\

Dat.    & sermōn\ending{ī}
        & virgin\ending{ī}
        & nōmin\ending{ī}
\\

Acc.    & sermōn\ending{em}
        & virgin\ending{em}
        & nōmen
\\

Voc.    & sermō
        & virgō
        & nōmen
\\

Abl.    & sermōn\ending{e}
        & virgin\ending{e}
        & nōmin\ending{e}
\\[\smallskipamount]

& \cc{3}{\textsc{plural}} \\

Nom.    & sermōn\ending{ēs}
        & virgin\ending{ēs}
        & nōmin\ending{a}
\\

Gen.    & sermōn\ending{um}
        & virgin\ending{um}
        & nōmin\ending{um}
\\

Dat.    & sermōn\ending{ibus}
        & virgin\ending{ibus}
        & nōmin\ending{ibus}
\\

Acc.    & sermōn\ending{ēs}
        & virgin\ending{ēs}
        & nōmin\ending{a}
\\

Voc.    & sermōn\ending{ēs}
        & virgin\ending{ēs}
        & nōmin\ending{a}
\\

Abl.    & sermōn\ending{ibus}
        & virgin\ending{ibus}
        & nōmin\ending{ibus}
\\

\end{Tabular}

\headingC{Remarks}

\section
\subsection

Like \latin{sermō} is declined the large class of nouns in
\ending{-iō}, as \latin{regiō}, \english{direction},
Gen.\ \latin{regiōnis}; \latin{āctiō}, \english{action},
Gen.\ \latin{āctiōnis}, etc.

\subsection

Like \latin{virgō} are declined all nouns in \suffix{-gō} or
\suffix{-dō} (except \latin{praedō}, \english{robber},
\latin{har\-pa\-gō}, \english{grappling-hook}, \latin{ligō},
\english{mattock}, which are declined like \latin{sermō}); also
\latin{homō}, \english{man}, \latin{nēmō}, \english{no one},
\latin{turbō}, \english{whirl-wind}, \latin{Apollō}, \english{Apollo}.

\subsection

There are some Masculines in \ending{-en}, Gen.\ \ending{-inis},
Acc.\ \ending{-inem}, as \latin{flāmen}, \english{priest},
\latin{pecten}, \english{comb}, \latin{oscen}, \english{divining bird}
(sometimes~\gender{f.}), and names of players on musical instruments,
as \latin{tībīcen}, \english{flute player}, etc.

\subsection

There is one stem in~\ending{-m}, namely \latin{hiem(p)s},
\english{winter}, \gender{f.}, Gen.~\latin{hiemis}.

\subsection

In \latin{carō}, \english{flesh}, \gender{f.}, the stem appears as
\stem{carn-} (not \stem{carōn-} or \latin{carin-}) in all cases but
the Nom.-Voc.\ Singular; e.g.\ Gen.\ Sing.\ \latin{carnis},
Nom.\ Plur.\ \latin{carnēs}.  Cf.\ \latin{pater},
Gen.\ \latin{patris}, etc.  Another peculiar form is \latin{sanguis},
\english{blood}, \gender{m.}, Gen.\ \latin{sanguinis}, etc.

\begin{note}

Beside \latin{sanguis}, which is properly an \phone{i}-Stem form,
there is also a Nom.\ \latin{sanguīs} (from \rec{sanguin-s}), which is
frequently used by the poets.  The Neuter \latin{sanguen} is an early
Latin form.

\end{note}

\section[Gender]

Masculine are all nouns in~\ending{-ō}, Gen.\ \ending{-ōnis}
(not~\ending{-iō}, Gen.\ \ending{-iōnis}).

Feminine are all nouns in~\ending{-ō}, Gen. \ending{-inis}, except
\latin{cardō}, \latin{margō}, \latin{ōrdō}, \latin{homō},
\latin{nēmō}, \latin{turbō}, \latin{Apollō}, which are Masculine; also
most in \ending{-iō} (abstracts and collectives), though there are
many Masculines, denoting material objects, as \latin{pugiō},
\english{dagger}.

Neuter are all nouns in~\ending{-en}, except those mentioned under
\xref[3]{83}.

\subtitle{\phone{s}-\textsc{Stems}}

\section

Examples:

\begin{Tabular}{>{\itshape}l
                  l @{\extracolsep{1em}} l
                  @{\extracolsep{3em}}
                  l @{\extracolsep{1em}} l}

& \cc{4}{\emph{Masculines \(and Feminines\)}} \\

& \cc{2}{\latin{cinis}, \english{ashes}, \gender{m.}}
& \cc{2}{\latin{flōs},  \english{flower}, \gender{m.}}
\\

& \textsc{singular} & \textsc{plural}
& \textsc{singular} & \textsc{plural}
\\

Nom.    & cinis     & ciner\ending{ēs}
        & flōs      & flōr\ending{ēs}
\\

Gen.    & ciner\ending{is}  & ciner\ending{um}
        & flōr\ending{is}   & flōr\ending{um}
\\

Dat.    & ciner\ending{ī}   & ciner\ending{ibus}
        & flōr\ending{ī}    & flōr\ending{ibus}
\\

Acc.    & ciner\ending{em}  & ciner\ending{ēs}
        & flōr\ending{em}   & flōr\ending{ēs}
\\

Voc.    & cinis             & ciner\ending{ēs}
        & flōs              & flōr\ending{ēs}
\\

Abl.    & ciner\ending{e}    & ciner\ending{ibus}
        & flōr\ending{e}     & flōr\ending{ibus}
\\[\jot]

& \cc{4}{\emph{Neuters}} \\

& \cc{2}{\latin{genus},  \english{race}}
& \cc{2}{\latin{corpus}, \english{body}}
\\

& \cc{1}{\textsc{singular}} & \cc{1}{\textsc{plural}}
& \cc{1}{\textsc{singular}} & \cc{1}{\textsc{plural}}
\\

Nom.    & genus         & gener\ending{a}
        & corpus        & corpor\ending{a}
\\

Gen.    & gener\ending{is}  & gener\ending{um}
        & corpor\ending{is} & corpor\ending{um}
\\

Dat.    & gener\ending{ī}  & gener\ending{ibus}
        & corpor\ending{ī} & corpor\ending{ibus}
\\

Acc.    & genus         & gener\ending{a}
        & corpus        & corpor\ending{a}
\\

Voc.    & genus         & gener\ending{a}
        & corpus        & corpor\ending{a}
\\

Abl.    & gener\ending{e}  & gener\ending{ibus}
        & corpor\ending{e} & corpor\ending{ibus}

\end{Tabular}

\headingC{Remarks}

\section
\subsection

Most \phone{s}-Stems are Neuters, declined like \latin{genus} or
\latin{corpus}. Other Neuters are:
\latin{iūs}, \english{right}, Gen.\ \latin{iūris}
(so \latin{rūs}, \english{country}, \latin{crūs}, \english{leg},
\latin{tūs}, \english{incense}, \latin{pūs}, \english{pus});
\latin{aes}, \english{bronze}, Gen.\ \latin{aeris};
\latin{ōs}, \english{mouth}, Gen.\ \latin{ōris};
\latin{os}, \english{bone}, Gen.\ \latin{ossis} (Nom.\ Plur. \latin{ossa},
Gen.\ Plur. \latin{ossium});
\latin{vās}, \english{vessel}, Gen.\ \latin{vāsis}.

\subsection

Masculines like \latin{cinis} are \latin{pulvis}, \english{dust}, and
\latin{cucumis}, \english{cucumber} (but Acc.\ and
Abl.\ Sing.\ \latin{cucumim}, \latin{cucumī}, after \phone{i}-Stem);
like \latin{flōs} are \latin{rōs}, \english{dew}, \latin{mōs},
\english{custom}, \latin{lepōs}, \english{charm}.  Other Masculines
are: \latin{mās}, \english{male}, Gen.\ \latin{maris}, \latin{mūs},
\english{mouse}, Gen.\ \latin{mūris}, \latin{as}, \english{copper},
Gen.\ \latin{assis}, all with Gen.\ Plur.\ in \ending{-ium};
\latin{lepus}, \english{hare}, Gen.\ \latin{leporis}.

\subsection

Feminines are very rare.  Examples are \latin{Venus}, \english{Venus},
Gen.\ \latin{Veneris}; \latin{tellūs}, \english{earth},
Gen.\ \latin{tellūris}; \latin{Cerēs}, \english{Ceres},
Gen.\ \latin{Cereris}.

\begin{note}

In all cases but the Nom.-Voc.\ Sing.\ (and Acc.\ Sing.\ Neut.)\ the
\phone{s}, as standing between vowels, regularly becomes~\phone{r}
(\xref{47}).  In many original \phone{s}-Stems even this
final~\phone{s} became~\phone{r}, under the influence of the other
cases, so that such Stems became wholly identical with
\phone{r}-Stems, and have been classed as such (e.g.\ \latin{honor},
sometimes \latin{honōs}; see \xref[4]{80}). Of the once numerous forms
in \ending{-ōs}, Gen.\ \ending{-ōris}, only the monosyllables (and
\latin{lepōs}) always retain the~\ending{-s}.

\end{note}

\headingF{\phone{i}-stems}

\section

The Nominative Singular of Masculines and Feminines ends regularly in
\ending{-is}; but there are also many nouns ending in~\ending{-ēs};
and a few in~\ending{-er}, from stems in \infix{-ri-},
e.g.\ \latin{imber} from \latin{imbris}, like \latin{ager} from
\rec{agros} (see~\xref[2]{43}).  The Nominative and Accusative
Singular of Neuters ended originally in~\suffix{-i}, but this has
either been changed to~\ending{-e} (\xref[3]{44}), or, in the case of
most stems in \infix{-āli-} or \infix{-āri-}, dropped (\xref[1]{43}).
Examples:
\begin{Tabular}{>{\itshape}l
                     l @{\extracolsep{1em}}
                     lll}

& \cc{4}{\emph{Masculines and Feminines}} \\

& \cc{1}{\latin{turris},}
& \cc{1}{\latin{fīnis},}
& \cc{1}{\latin{caedēs},}
& \cc{1}{\latin{imber},}
\\

& \cc{1}{\english{tower},     \gender{f.}}
& \cc{1}{\english{end},       \gender{m., f.}}
& \cc{1}{\english{slaughter}, \gender{f.}}
& \cc{1}{\english{shower},    \gender{m.}}
\\

& \cc{4}{\textsc{singular}} \\

Nom.    & turr\ending{is}
        & fīn\ending{is}
        & caed\ending{ēs}
        & imber
\\

Gen.    & turr\ending{is}
        & fīn\ending{is}
        & caed\ending{is}
        & imbr\ending{is}
\\

Dat.    & turr\ending{ī}
        & fīn\ending{ī}
        & caed\ending{ī}
        & imbr\ending{ī}
\\

Acc.    & turr\ending{im} (\ending{-em})
        & fīn\ending{em}
        & caed\ending{em}
        & imbr\ending{em}
\\

Voc.    & turr\ending{is}
        & fīn\ending{is}
        & caed\ending{ēs}
        & imber
\\

Abl.    & turr\ending{ī} or \ending{-e}
        & fīn\ending{e}
        & caed\ending{e}
        & imbr\ending{e} or \ending{-ī}
\\

& \cc{4}{\textsc{plural}} \\

Nom.    & turr\ending{ēs}
        & fīn\ending{ēs}
        & caed\ending{ēs}
        & imbr\ending{ēs}
\\

Gen.    & turr\ending{ium}
        & fīn\ending{ium}
        & caed\ending{ium}
        & imbr\ending{ium}
\\

Dat.    & turr\ending{ibus}
        & fīn\ending{ibus}
        & caed\ending{ibus}
        & imbr\ending{ibus}
\\

Acc.    & turr\ending{īs} (\ending{-ēs})
        & fīn\ending{īs} (\ending{-ēs})
        & caed\ending{īs} (\ending{-ēs})
        & imbr\ending{īs} (\ending{-ēs})
\\

Voc.    & turr\ending{ēs}
        & fīn\ending{ēs}
        & caed\ending{ēs}
        & imbr\ending{ēs}
\\

Abl.    & turr\ending{ibus}
        & fīn\ending{ibus}
        & caed\ending{ibus}
        & imbr\ending{ibus}

\end{Tabular}

\begin{Tabular}{>{\itshape}l
                     l @{\extracolsep{1em}}
                     ll}

& \cc{3}{\emph{Neuters}} \\

& \cc{1}{\latin{sedīle}, \english{seat}}
& \cc{1}{\latin{animal},  \english{animal}}
& \cc{1}{\latin{exemplar}, \english{pattern}}
\\

& \cc{3}{\textsc{singular}} \\

Nom.    & sedīl\ending{e}
        & animal
        & exemplar
\\

Gen.    & sedīl\ending{is}
        & animāl\ending{is}
        & exemplār\ending{is}
\\

Dat.    & sedīl\ending{ī}
        & animāl\ending{ī}
        & exemplār\ending{ī}
\\

Acc.    & sedīl\ending{e}
        & animal
        & exemplar
\\

Voc.    & sedīl\ending{e}
        & animal
        & exemplar
\\

Abl.    & sedīl\ending{ī}
        & animāl\ending{ī}
        & exemplār\ending{ī}
\\

& \cc{3}{\textsc{plural}} \\

Nom.    & sedīl\ending{ia}
        & animāl\ending{ia}
        & exemplār\ending{ia}
\\

Gen.    & sedīl\ending{ium}
        & animāl\ending{ium}
        & exemplār\ending{ium}
\\

Dat.    & sedīl\ending{ibus}
        & animāl\ending{ibus}
        & exemplār\ending{ibus}
\\

Acc.    & sedīl\ending{ia}
        & animāl\ending{ia}
        & exemplār\ending{ia}
\\

Voc.    & sedīl\ending{ia}
        & animāl\ending{ia}
        & exemplār\ending{ia}
\\

Abl.    & sedīl\ending{ibus}
        & animāl\ending{ibus}
        & exemplār\ending{ibus}
\end{Tabular}

\headingC{Remarks}

\section
\subsection

The Accusative Singular always or usually has \suffix{-im} in:
\begin{mexamples}[3]

\latin{būris}, \english{plough-beam}

\latin{febris}, \english{fever}

\latin{pelvis}, \english{basin}

\latin{puppis}, \english{stern}

\latin{restis}, \english{rope}

\latin{secūris}, \english{axe}

\latin{sitis}, \english{thirst}

\latin{turris}, \english{tower}

\latin{tussis}, \english{cough}

\end{mexamples}
\noindent
and names of \emph{rivers} and \emph{cities}, like \latin{Tiberis},
\english{the Tiber}, \latin{Neāpolis}, \emph{Naples}; occasionally in
several others.

\subsection
The Ablative Singular has the form~\ending{-ī}:
\begin{enuma}

\item
In all Neuters except \latin{rēte}, \english{net}, and some names of
places, like \latin{Praeneste}, \english{Praeneste}.  \latin{Mare},
\english{sea}, sometimes has Abl.\ \latin{mare} in poetry.

\item

Always or usually in \latin{secūrus}, \latin{sitis}, \latin{tussis},
\latin{bipennis}, \english{battle-axe}, \latin{canālis},
\english{conduit}, and names of \emph{rivers}, \emph{cities}, and
\emph{months}.

\item

Often in the following, which also have~\ending{e}:
\begin{mexamples}[3]

\latin{amnis}, \english{river}

\latin{avis}, \english{bird}

\latin{cīvis}, \english{citizen}

\latin{classis}, \english{fleet}

\latin{clāvis}, \english{key}

\latin{febris}, \english{fever}

\latin{fūstis}, \english{club}

\latin{ignis}, \english{fire}

\latin{imber}, \english{shower}

\latin{nāvis}, \english{ship}

\latin{pelvis}, \english{basin}

\latin{puppis}, \english{stern}

\latin{sēmentis}, \english{sowing}

\latin{strigilis}, \english{scraper}

\latin{turris}, \english{tower}

\end{mexamples}

\item

Occasionally in \latin{fīnis}, \english{end} (in adverbial phrases;
see~\xref[4]{407}), \latin{collis}, \english{hill}, \latin{orbis},
\english{circle}, \latin{unguis}, \english{nail}, and a few others.

\end{enuma}

\subsection

The Acc.\ Plur.\ Masc.\ and Fem.\ has earlier \ending{-īs}, later
\ending{-ēs}.  See~\xref[\emph{a}]{75}.  The \ending{-īs} also occurs
sometimes in the Nominative, as \latin{aedīs}.

\subsection

The Genitive Plural ends in \ending{-ium}, but \ending{-um} is the
regular form for \latin{canis}, \english{dog}, \latin{iuvenis},
\english{youth}, \latin{volucris}, \english{bird}, and for
\latin{senex}, \english{old man} (Nom.\ Sing.\ formed from a stem
\latin{senec-}\emend{74}{;}{,} Gen.\ Sing.\ \latin{senis}); \ending{-um} is also
found beside \ending{-ium} in \latin{sēdēs}, \english{seat},
\latin{mēnsis}, \english{month}, and, rarely, in \latin{vātēs},
\english{bard}.

\subsection

The Ablative Singular of \latin{famēs}, \english{hunger}, is
\latin{famē}, following the Fifth Declension; \latin{tabē} also occurs
once, from \latin{tabēs}, \english{wasting}.

\section[Gender]

Masculine are nouns in~\ending{-er}, except \latin{linter},
\english{skiff}, \gender{f.}

Feminine are nouns in \ending{-ēs} (but \latin{verrēs},
\english{boar}, \gender{m.}, \latin{vātēs}, \english{bard},
\gender{m.}, \gender{f.}); also the majority of those in \ending{-is}
(but those in \ending{-nis}, and nearly thirty others, are Masculine).

Neuter are nouns in \ending{-e}, \ending{-al}, \ending{-ar}.

\pagebreak

\headingF{Mixed Stems}

\section

The Singular agrees with that of Mute-Stems, the Plural with that of
\phone{i}-Stems. Examples:

\begin{Tabular}{
    >{\itshape}l
    l
    @{\extracolsep{2em}}
    l
    @{\extracolsep{2em}}
    l
}

& \cc{1}{\latin{nox}, \english{night}, \gender{f.}}
& \cc{1}{\latin{urbs}, \english{city}, \gender{f.}}
& \cc{1}{\latin{gēns}, \english{race}, \gender{f.}}
\\

& \cc{3}{\textsc{singular}} \\

Nom.    & no\ending{x}
        & urb\ending{s}
        & gēn\ending{s}
\\

Gen.    & noct\ending{is}
        & urb\ending{is}
        & gent\ending{is}
\\

Dat.    & noct\ending{ī}
        & urb\ending{ī}
        & gent\ending{ī}
\\

Acc.    & noct\ending{em}
        & urb\ending{em}
        & gent\ending{em}
\\

Voc.    & no\ending{x}
        & urb\ending{s}
        & gēn\ending{s}
\\

Abl.    & noct\ending{e}
        & urb\ending{e}
        & gent\ending{e}
\\

& \cc{3}{\textsc{plural}} \\

Nom.    & noct\ending{ēs}
        & urb\ending{ēs}
        & gent\ending{ēs}
\\

Gen.    & noct\ending{ium}
        & urb\ending{ium}
        & gent\ending{ium}
\\

Dat.    & noct\ending{ibus}
        & urb\ending{ibus}
        & gent\ending{ibus}
\\

Acc.    & noct\ending{īs} (\ending{-ēs})
        & urb\ending{īs} (\ending{-ēs})
        & gent\ending{īs} (\ending{-ēs})
\\

Voc.    & noct\ending{ēs}
        & urb\ending{ēs}
        & gent\ending{ēs}
\\

Abl.    & noct\ending{ibus}
        & urb\ending{ibus}
        & gent\ending{ibus}

\end{Tabular}

\headingC{Remarks}

\section
\subsection

To this type belong:
\begin{enuma}

\item

Nouns in \ending{-ns}, \ending{-rs}, \ending{-rx},
\latin{\emend{2}{}{-}lx}, as
\latin{mōns} (Gen.\ Plur.\ \latin{montium}),
\latin{glāns} (\latin{glan\-di\-um}),
\latin{pars} (\latin{par\-ti\-um}),
\latin{arx} (\latin{arcium}),
\latin{falx} (\latin{falcium}),
etc.; also \latin{dōs}, \latin{līs}, \latin{fraus} (also \hbox{\latin{-um}}),
\latin{nox},
\latin{nix} (Gen.\ \latin{nivis}; see~\xref[2]{49}), \latin{faucēs}.
But \latin{cliēns}, \english{client}, and \latin{parēns},
\english{parent}, have Gen.\ Plur.\ \ending{-um} and \ending{-ium}.

\item

Monosyllables in \ending{-ps}, \ending{-bs}, as \latin{stirps}
(\latin{stirpium}), \latin{plēbs} (\latin{plēbium}), etc.  But always
\latin{opum}, \english{of resources}, from \rec{ops},
Gen.\ \latin{opis}.

\item

Nouns in \ending{-ās}, \ending{-īs}, \ending{-tās}, as \latin{Arpīnās}
(\ending{-ium}), \latin{penātēs} (\ending{-ium}), \latin{optimātēs}
(\ending{-ium} and \ending{-um}), \latin{Quirītēs}
(\ending{-ium})\emend{3}{}{,} \latin{Samnītēs} (\ending{-ium}),
\latin{cīvitās} (\ending{-ium} and~\ending{-um}).

\end{enuma}

\begin{note}

\latin{Mās}, \latin{mūs}, and \latin{as}, with Gen.\ Plur.\ in
\ending{-ium}, are classed under \phone{s}-Stems (\xref[2]{86}).

\end{note}

\subsection[\textbf{Gender}]

Nouns of this type are Feminine, except that there are several
Masculines in \ending{-ns}, Gen.\ \ending{-ntis}, as \latin{dēns},
\latin{fōns}, \latin{mōns}, \latin{pōns}.

\pagebreak

\headingF{Irregular Nouns}

\section

The declension of the following nouns differs from any of the usual
types:
\begin{Tabular}{>{\itshape}l
                  l @{\extracolsep{1em}}
                  lll}

& \cc{1}{\latin{vīs},}
& \cc{1}{\latin{sūs},}
& \cc{1}{\latin{bōs},}
& \cc{1}{\latin{Iuppiter},}
\\

& \cc{1}{\english{force},   \gender{f.}}
& \cc{1}{\english{swine},   \gender{m., f.}}
& \cc{1}{\english{ox, cow}, \gender{m., f.}}
& \cc{1}{\english{Jupiter}, \gender{m.}}
\\

& \cc{4}{\textsc{singular}} \\

Nom.    & vīs
        & sūs
        & bōs
        & Iuppiter
\\

Gen.    & (vis)
        & suis
        & bovis
        & Iovis
\\

Dat.    & (vī)
        & suī
        & bovī
        & Iovī
\\

Acc.    & vim
        & suem
        & bovem
        & Iovem
\\

Voc.    & vīs
        & sūs
        & bōs
        & Iuppiter
\\

Abl.    & vī
        & sue
        & bove
        & Iove
\\

& \cc{4}{\textsc{plural}} \\

Nom.    & vīrēs
        & suēs
        & bovēs
\\

Gen.    & vīrium
        & suum
        & boum
\\

Dat.    & vīribus
        & suibus (sūbus, subus)
        & būbus (bōbus)
\\

Acc.    & vīris (-ēs)
        & suēs
        & bovēs
\\

Voc.    & vīrēs
        & suēs
        & bovēs
\\

Abl.    & vīribus
        & suibus (sūbus, subus)
        & būbus (bōbus)
\\

\end{Tabular}

\subsubsection

Like \latin{sūs} is declined \latin{grūs}, \english{crane},
\gender{m.}, \gender{f.}\ (Dat.-Abl.\ Plur.\ \latin{gruibus}).

\subsubsection

Other peculiar forms have been mentioned as varieties of the regular
types, e.g.\ \latin{carō}, Gen.\ \latin{carnis} (\xref[5]{83});
\latin{iter}, Gen.\ \latin{itineris} (\xref[6]{80}); \latin{senex},
Gen.\ \latin{senis} (\xref[4]{88}), etc.

\begin{note}

\latin{Vīs} is an old \phone{s}-Stem (with \latin{vīs},
Nom.\ Plur.\ \latin{vīrēs}, compare \latin{mūs}, \latin{mūrēs}), but
the Dat., Acc., and Abl.\ Sing.\ are formed from a stem \stem{vi-}.
\latin{Sūs} and \latin{grūs} are relics of a \phone{ū}-Declension.
\latin{Bōs} is from a stem \stem{bov-} (\latin{bōs} from
\rec{bō(u)-s}).  \latin{Iuppiter}, earlier \latin{Iūpiter}, comes
from a Vocative form \rec{Iou} (once \rec{dieu}) + \latin{piter} (from
\latin{pater}, \english{father}, by the regular weakening,
\xref[1]{42}).

\end{note}

\headingC{The Locative Singular of the Third Declension}

\section

The Locative Singular is identical with the Ablative Singular in
\ending{-e}, as \latin{Car\-thā\-gi\-ne}, \english{at Carthage}.  But there
are also forms in \ending{-ī}, as \latin{Carthāginī}, \latin{rūrī},
\english{in the country} (beside \latin{rūre}).

\headingC{Gender in the Third Declension}

\section

The following is a summary of such of the important types as are
fairly uniform in gender.  For more detailed statements, with
exceptions, see under the several classes.

\subsection

Masculine: nouns in
\ending{-tor} (Gen.\ \ending{-tōris}),
\ending{-or}  (Gen.\ \ending{-ōris}),
\ending{-er}  (Gen.\ \ending{-ris}),
\ending{-ō}   (Gen.\ \ending{-ōnis}),
\ending{-es}  (Gen.\ \ending{-itis}),
\ending{-eps} (Gen.\ \ending{-ipis}),
\ending{-ex}  (Gen.\ \ending{-icis}).

\begin{minor}

Examples: \latin{dator}, \latin{amor}, \latin{pater}, \latin{sermō},
\latin{mīles}, \latin{prīnceps}, \latin{auspex}.

\end{minor}

% \pagebreak

\subsection

Feminine: nouns in
\ending{-tās} (Gen.\ \ending{-tātis}),
\ending{-tūs} (Gen.\ \ending{-tūtis}),
\ending{-ēs}  (Gen.\ \ending{-is}),
\ending{-gō} or \ending{-dō} (Gen.\ \ending{-inis}),
\ending{-rs}  (Gen.\ -\ending{rtis});
and the
majority of those in \ending{-iō} (Gen.\ \ending{-iōnis}) and
\ending{-is} (Gen.\ \ending{-is}).

\begin{minor}

Examples: \latin{cīvitās}, \latin{virtūs}, \latin{caedēs},
\latin{virgō}, \latin{grandō}, \latin{pars}; \latin{regiō},
\latin{turris}.

\end{minor}

\subsection

Neuter: nouns in \ending{-en}, \ending{-us}, \ending{-e},
\ending{-al} (Gen.\ \ending{-ālis}), \ending{-ar}
(Gen.\ \ending{-āris}), \ending{-ur} (Gen.\ \ending{-oris}),
\ending{-or} (Gen.\ \ending{-oris}).

\begin{minor}

Examples: \latin{nōmen}, \latin{genus}, \latin{sedīle},
\latin{animal}, \latin{exemplar}, \latin{ebur}, \latin{aequor}.

\end{minor}

\headingC{Greek Nouns}

\section

Greek Nouns of the Third Declension often retain their Greek forms in
the Nominative, Accusative, and Vocative Singular, the Nominative and
Accusative Plural, and sometimes in the Genitive Singular. The Latin
endings are nearly always used in the other cases; also, usually, in
the Genitive Singular and frequently in the Accusative Singular.
Examples of Declension:

\begin{Tabular}{>{\itshape}l
                  l @{\extracolsep{1em}}
                  ll}

& \cc{1}{\latin{lampas},   \english{torch}, \gender{f.}}
& \cc{1}{\latin{Sōcratēs}, \english{Socrates}}
& \cc{1}{\latin{hērōs},    \english{hero}, \gender{m.}}
\\

& \cc{3}{\textsc{singular}} \\

Nom.    & lampas
        & Sōcrat\ending{ēs}
        & hērōs
\\

Gen.    & lampad\ending{os}, \ending{-is}
        & Sōcrat\ending{is}, \ending{-ī}
        & hērō\ending{is}
\\

Dat.    & lampad\ending{ī}
        & Sōcrat\ending{ī}
        & hērō\ending{\emend{4}{i}{ī}}
\\

Acc.    & lampad\ending{a}, \ending{-em}
        & Sōcrat\ending{em}, \ending{-ēn}
        & hērō\ending{a}, \ending{-em}
\\

Voc.    & lampas
        & Sōcrat\ending{es}, \ending{-ē}
        & hērōs
\\

Abl.    & lampad\ending{e}
        & Sōcrat\ending{e}
        & hērō\ending{e}
\\

& \cc{3}{\textsc{plural}} \\

Nom.    & lampad\ending{ĕs}
        &
        & hērō\ending{ĕs}
\\

Gen.    & lampad\ending{um}
        &
        & hērō\ending{um}
\\

Dat.    & lampad\ending{ibus}
        &
        & hērō\ending{ibus}
\\

Acc.    & lampad\ending{ăs}
        &
        & hērō\ending{ăs}
\\

Voc.    & lampad\ending{ĕs}
        &
        & hērō\ending{ĕs}
\\

Abl.    & lampad\ending{ibus}
        &
        & hērō\ending{ibus}

\end{Tabular}

\subsection

Proper names in \ending{-eus} usually follow the Latin Second
Declension (often with synizesis; \xref{658}), except in the Vocative,
which ends in \ending{-\t{e}{u}}.

\begin{minor}

But note also Acc.\ \latin{Orphea}, \latin{Īlionēa},
Dat.\ \latin{Orph\t{e}{i}}, etc.  \latin{Perseus} appears also as
\latin{Persēs}, Acc.\ \latin{Persem}, etc.  \latin{Achillēs} sometimes
has forms of \ending{-eus}, as Gen.\ \latin{Achilleī}.

\end{minor}

\subsubsection

Names like \latin{Dīdō} are regularly declined in \ending{-ō},
\ending{-ōnis}, etc.  But there is also a Gen.\ in \ending{-ūs}, as
\latin{Mantūs}, and Acc.\ in \ending{-ō}, as \latin{Dīdō}.

\subsubsection

For names in \ending{-is}, \ending{-idis}, observe
Acc.\ \latin{Paridem}, \latin{Tyndarida}, \latin{Parim},
\latin{Parin}, Voc.\ \latin{Daphni}.  Cf.\ \latin{Darēs},
Acc. \latin{Darēta} and \latin{Darēn}.

\subsubsection

For names in \ending{-ys}, observe Acc.\ \latin{Capyn}, \latin{Halym},
Voc.\ \latin{Tiphy}, Abl.\ \latin{Capye}.

\pagebreak

\chapter{Fourth Declension}

\contentsentry{C}{Fourth Declension}

\section

The Nominative Singular ends in \ending{-us}, or, in the case of
Neuters, in \ending{-ū}.  Examples of Declension:
\begin{Tabular}{>{\itshape}l
                     l @{\extracolsep{1em}}
                     ll}

& \cc{1}{\latin{frūctus}, \english{fruit}, \gender{m.}}
& \cc{1}{\latin{tribus},  \english{tribe}, \gender{f.}}
& \cc{1}{\latin{cornū},   \english{horn},  \gender{n.}}
\\

& \cc{1}{(stem \stem{frūctu-})}
& \cc{1}{(stem \stem{tribu-})}
& \cc{1}{(stem \stem{cornu-})}
\\

& \cc{3}{\textsc{singular}} \\

Nom.    & frūct\ending{us}
        & trib\ending{us}
        & corn\ending{ū}
\\

Gen.    & frūct\ending{ūs}
        & trib\ending{ūs}
        & corn\ending{ūs}
\\

Dat.    & frūct\ending{uī}, \ending{-ū}
        & trib\ending{uī},  \ending{-ū}
        & corn\ending{ū}
\\

Acc.    & frūct\ending{um}
        & trib\ending{um}
        & corn\ending{ū}
\\

Voc.    & frūct\ending{us}
        & trib\ending{us}
        & corn\ending{ū}
\\

Abl.    & frūct\ending{ū}
        & trib\ending{ū}
        & corn\ending{ū}
\\

& \cc{3}{\textsc{plural}} \\

Nom.    & frūct\ending{ūs}
        & trib\ending{ūs}
        & corn\ending{ua}
\\

Gen.    & frūct\ending{uum}
        & trib\ending{uum}
        & corn\ending{uum}
\\

Dat.    & frūct\ending{ibus}
        & trib\ending{ubus}
        & corn\ending{ibus}
\\

Acc.    & frūct\ending{ūs}
        & trib\ending{ūs}
        & corn\ending{ua}
\\

Voc.    & frūct\ending{ūs}
        & trib\ending{ūs}
        & corn\ending{ua}
\\

Abl.    & frūct\ending{ibus}
        & trib\ending{ubus}
        & corn\ending{ibus}
\\

\end{Tabular}

\headingC{Remarks on the Case-Forms}

\section
\subsection

The Dative and Ablative Plural end in \ending{-ubus} as follows:
\begin{enuma}

\item
Always in \latin{arcus}, \latin{tribus}, \latin{quercus}.

\item

Frequently in \latin{artus}, \latin{lacus}, \latin{partus}, \latin{verū}.

\item

Occasionally in \latin{genū}, \latin{tonitrū}, and a few others.

\end{enuma}

\subsection

The Dative Singular in \ending{-ū} is regular in Neuters, and, except
in early Latin, is frequent in Masculines and Feminines.

\subsection

The Genitive Plural sometimes ends in \ending{-um}, as \latin{passum},
formed after \latin{nummum}, etc., of the Second Declension (\xref[4,
  \emph{a}]{71}).

\subsection

In early Latin is found a Genitive Singular in \ending{-uis}, as
\latin{frūctuis}, \latin{quaestuis}; on inscriptions also
\ending{-uos}, as \latin{senātuos}.

\subsection

Some nouns show an intermixture of forms of the Second Declension, as
\latin{senātus}, \english{senate}, Gen.\ \latin{senātī} beside
\latin{senātūs}, and especially \latin{domus}, \english{house}, the
inflection of which is as follows:
\begin{Tabular}{l@{\extracolsep{3em}}l}

domus           & domūs \\
domūs (domī)    & domōrum, domuum \\
domuī (domō)    & domibus \\
domum           & domōs, domūs \\
domus           & domūs \\
domō (domū)     & domibus
\\

\cc{2}{Loc.\ Sing.\ domī, \english{at home}}

\end{Tabular}

\headingC{Gender}

\section

Nouns of the Fourth Declension ending in \ending{-us} are mostly
Masculine, those in \ending{-ū} Neuter.

\subsubsection

But the following in \ending{-us} are Feminine:
\begin{mexamples}

\latin{acus}, \english{needle}

\latin{anus}, \english{old woman}

\latin{domus}, \english{house}

\latin{Īdūs} (Plur.), \english{Ides}

\latin{nurus}, \english{daughter-in-law}

\latin{porticus}, \english{porch}

\latin{Quīnquātrūs} (Plur.), \english{name of a festival}

\latin{socrus}, \english{mother-in-law}

\latin{tribus}, \english{tribe}

\end{mexamples}

\chapter{Fifth Declension}

\contentsentry{C}{Fifth Declension}

\section

The Nominative Singular ends in \ending{-ēs}.  Examples of Declension:

\begin{Tabular}{>{\itshape}l
                  l
                  @{\extracolsep{1em}}
                  l
                  @{\extracolsep{3em}}
                  l
                  @{\extracolsep{1em}}
                  l}

& \cc{2}{\latin{diēs}, \english{day}, \gender{m.} (stem \stem{diē-})}
& \cc{2}{\latin{rēs}, \english{thing} (stem \stem{rē-}),}
\\

& \textsc{singular} & \textsc{plural}
& \textsc{singular} & \textsc{plural}
\\

Nom.    & di\ending{ēs}
        & di\ending{ēs}
        & r\ending{ēs}
        & r\ending{ēs}
\\

Gen.    & di\ending{ēī}
        & di\ending{ērum}
        & r\ending{eī}
        & r\ending{ērum}
\\

Dat.    & di\ending{ēī}
        & di\ending{ēbus}
        & r\ending{eī}
        & r\ending{ēbus}
\\

Acc.    & di\ending{em}
        & di\ending{ēs}
        & r\ending{em}
        & r\ending{ēs}
\\

Voc.    & di\ending{ēs}
        & di\ending{ēs}
        & r\ending{ēs}
        & r\ending{ēs}
\\

Abl.    & di\ending{ē}
        & di\ending{ēbus}
        & r\ending{ē}
        & r\ending{ēbus}
\end{Tabular}

\headingC{Remarks on the Case-Forms}

\section
\subsection

In the Genitive and Dative Singular we find \ending{-ēī} after a
vowel, but \ending{-eī} after a consonant, as \latin{diēī},
\latin{faciēī}, but \latin{reī}, \latin{fideī}.  But this distinction
does not hold in early Latin, where we find, for example, \latin{rēī},
\latin{reī}, and oftener monosyllabic \latin{rei}.

\subsection

A form of the Genitive and Dative Singular in \ending{-ē} is found, as
\latin{diē}, \latin{aciē}.

\subsection

The Genitive Singular of \latin{plēbēs}, \english{people}, is often
\latin{plēbī} in the phrases \latin{tri\-bū\-nus plēbī} and \latin{plēbī
  scītum}.  \versionB*{Similarly (rarely), \latin{diī} for \latin{diēī}.}

\subsection

The only words which have a Complete Plural are \latin{diēs} and
\latin{rēs}, but several others are used in the Nominative and
Accusative Plural.

\headingC{Gender}

\section

Nouns of the Fifth Declension are Feminine, except \latin{diēs},
\english{day}, and \latin{me\-rī\-di\-ēs}, \english{midday}. And
\latin{diēs} is usually Feminine when meaning an appointed time, or
time in general.

\chapter{Defective and Variable Nouns}

\contentsentry{C}{Defective and Variable Nouns}

\section

Nouns may lack one Number or one or more Cases; they may follow partly
one Declension, partly another; or they may vary in Gender.

% \pagebreak

\headingC{Nouns used only in the Singular}

\section

Some words are of such a meaning as to be used commonly only in the
singular.  Such are:
\begin{enumerate*}

\item
Proper Names.

\item
Abstracts, like \latin{cāritās}, \english{affection}.

\item
Collectives, like \latin{vulgus}, \english{the rabble}.

\item
Words denoting Material, as \latin{aes}, \english{bronze}.

\end{enumerate*}

\begin{note}

But some of these are used in the Plural in a peculiar sense, as
\latin{Caesarēs}, \english{the Caesars},
\latin{cāritātēs}, \english{kinds of affection},
\latin{aera}, \english{bronzes}, \english{arms of bronze},
\english{wages}.

\end{note}

\headingC{Nouns used only in the Plural}

\section

Nouns used only in the Plural include:
\begin{enumerate*}

\item
Some names of places, as \latin{Athēnae}, \english{Athens}.

\item
Most names of festivals, as \latin{Bacchānālia}, \english{festival of
  Bacchus}.

\item

Many names of objects naturally Plural in signification, as
\latin{arma}, \english{arms}, \latin{spolia},\footnote{Occasionally
  Singular in poetry.}\savefootnote\ \english{spoils},
\latin{vīscera}, \english{entrails}.

\item

Many others, for some of which English prefers the Singular. The most
important are;
\begin{mexamples}

\latin{angustiae}, \english{defile}, \english{difficulty}
(\english{straits})

\latin{cibāria}, \english{food} (\english{rations})

\latin{dēliciae}, \english{pleasure}

\latin{dīvitiae}, \english{wealth} (\english{riches})

\latin{epulae},\footnote{Also \latin{epulum}, \english{public
    banquet}.} \english{banquet} (\english{viands})

\latin{facētiae},\savedfootnote\ \english{wit} (\english{witticisms})

\latin{forēs},\savedfootnote\ \english{door}

\latin{hīberna}, \english{winter quarters}

\latin{indūtiae}, \english{truce}

\latin{īnsidiae}, \english{ambush}

\latin{līberī}, \english{children}

\latin{minae}, \english{threats}

\latin{moenia}, \english{walls}

\latin{nūndinae}, \english{market-day}

\latin{nūptiae}, \english{wedding} (\english{nuptials})

\latin{reliquiae}, \english{remainder} (\english{remains})

\latin{tenebrae}, \english{darkness}

\latin{verbera},\savedfootnote\ \english{scourging} (\english{lashes})

\end{mexamples}

\end{enumerate*}

\headingC{Different Meaning in Singular and Plural}

\section

Many nouns are used in both the Singular and the Plural, but with a
difference of meaning. The most important instances are:
\begin{longtable}{l@{\extracolsep{2em}}l}

\textsc{singular} & \textsc{plural} \endhead

  \latin{aedēs}, \english{temple}
& \latin{aedēs}, \english{house}
\\

  \latin{auxilium}, \english{help}
& \latin{auxilia}, \english{auxiliaries}
\\

  \latin{carcer}, \english{prison}
& \latin{carcerēs}, \english{barriers}
\\

  \latin{castrum}, \english{fort}
& \latin{castra}, \english{camp}
\\

  \latin{cēra}, \english{wax}
& \latin{cērae}, \english{wax tablets}
\\

  \latin{comitium}, \english{place of assembly}
& \latin{comitia}, \english{assembly}
\\

  \latin{cōpia}, \english{plenty}
& \latin{cōpiae}, \english{troops}
\\

  \latin{facultās}, \english{possibility}
& \latin{facultātēs}, \english{resources}, \english{goods}
\\

  \latin{fīnis}, \english{end}, \english{border}
& \latin{fīnēs}, \english{borders}, \english{territory}
\\

  \latin{fortūna}, \english{fortune}
& \latin{fortūnae}, \english{possessions}
\\

  \latin{impedīmentum}, \english{hindrance}
& \latin{impedīmenta}, \english{baggage}
\\

  \latin{littera}, \english{letter} (of the alphabet)
& \latin{litterae}, \english{letter}, \english{epistle}
\\

  \latin{opera}, \english{work}
& \latin{operae}, \english{workmen}
\\

  \latin{pars}, \english{portion}
& \latin{partēs}, \english{rôle}
\\

  \latin{rōstrum}, \english{beak}
& \latin{rōstra}, \english{platform for speakers}
\\

  \latin{vigilia}, \english{watch}
& \latin{vigiliae}, \english{pickets}

\end{longtable}

\headingC{Nouns Defective in Case-Forms}

\section

Nouns may lack one or more of the Case-Forms.

\subsection

Many \phone{u}-Stems are used only in the Ablative Singular, as
\latin{nātū}, \english{by birth}, \latin{iussū}, \english{by order};
similarly \latin{pondō}, \english{by weight}, \latin{sponte},
\english{of free will} (Gen.\ \latin{spontis} rare).  Of
\latin{forte}, \english{by chance}, the Nom.\ \latin{fors} also
occurs.

\subsection

Several Neuters are used only in the Nom.-Acc.\ Sing., as \latin{fās},
\english{right}, \latin{nihil}, \latin{nīl}, \english{nothing},
\latin{īnstar}, \english{likeness}, \latin{opus}, \english{need}, etc.

\subsection

\latin{Nēmō}, \english{no one}, has a Dat.\ \latin{nēminī} and an
Acc.\ \latin{nēminem}, but the Gen.\ and Abl.\ are supplied by
\latin{nūllīus} and \latin{nūllō}, from \latin{nūllus}.

\subsection

The Nominative Singular is lacking for \latin{dapis}, \english{feast},
\latin{frūgis}, \english{fruit}, \latin{opis}, \english{help} (lacks
also Dat.), \latin{vicis}, \english{change} (lacks also Dat.),
\latin{precī} (Dat.), \english{prayer} (lacks also Gen.), etc.

\subsection

The Genitive Plural is lacking in many nouns, as \latin{pāx},
\latin{lūx}, etc.

\begin{note}

An enumeration of all the examples of Defective Nouns is unnecessary.
It is sometimes a mere accident that a certain case-form is not found.

\end{note}

\headingC{Nouns Variable in Declension}

\section

Some nouns show forms belonging to two different Declensions or to two
classes of the same Declension.  Such are known as Heteroclites
(“differently declined”).

\subsection

Some examples have been given already, as \latin{domus}
(\xref[5]{97}), which varies between the Second and Fourth
Declensions; \latin{vīrus}, etc., of the Second, but having the
Nom.-Acc.\ Sing.\ of the Third (\xref[\emph{b}, note]{72});
\latin{famēs}, of the Third, but having the Abl.\ Sing.\ \latin{famē}
of the Fifth (\xref[5]{88}); \latin{femur}, an \phone{r}-Stem in the
Nom.\ and Acc.\ Sing., but forming its other cases from an
\phone{n}-Stem (\xref[5]{80}).

\begin{note}

From the historical point of view all words of the Third Declension
are Heteroclites, since their case-forms belong partly to
\phone{i}-Stems and partly to Consonant-Stems.

\end{note}

\subsection

Other illustrations are: \latin{vās}, \english{vessel}, with Singular
of the Third Declension (Gen.\ \latin{vāsis}, etc.), and Plural of the
Second (\latin{vāsa}, \latin{vāsōrum}, etc.); \latin{iūgerum},
\english{acre}, with Singular of the Second Declension
(Gen.\ \latin{iūgerī}, etc.), and Plural of the Third (\latin{iūgera},
\latin{iūgerum}, \latin{iūgeribus}); \latin{requiēs}, \english{rest},
of the Third (Gen.\ \latin{requiētis}, etc.), but having also an
Acc.\ Sing.\ \latin{requiem} of the Fifth; \latin{māteria},
\english{material}, of the First, but having also a
Nom.\ Sing.\ \latin{māteriēs} and an Acc.\ Sing.\ \latin{māteriem} of
the Fifth, and similarly many others.

\headingC{Nouns Variable in Gender}

\section

Nouns may have forms of different Genders. Such are known as
Heterogeneous Nouns.

\subsection

Some nouns of the Second Declension have both Masculine and Neuter
forms, as \latin{clipeus}, \gender{m.}, and \latin{clipeum},
\gender{n.}, \english{shield}.

\subsection

Many nouns have different genders in the Singular and Plural, as
\latin{locus}, \gender{m.}, \english{place}, Plur.\ \latin{loca},
\gender{n.}, \english{places} (\latin{locī}, \gender{m.},
\english{passages in authors}); \latin{iocus}, \english{jest},
\gender{m.}, Plur.\ often \latin{ioca}, \gender{n.}; \latin{frēnum},
\english{bit}, \gender{n.}, Plur.\ often \latin{frēnī},~\gender{m.}

\headingB{Adjectives}

\contentsentry{B}{Declension of Adjectives}

\section

There are two types of Adjectival Declension, the one being based on
the First and Second Declensions of Nouns, the other on the Third.

\chapter{Adjectives of the First and Second Declensions}

\contentsentry{C}{Adjectives of the First and Second Declensions}

\section

The Masculine is declined like \latin{hortus}, \latin{puer}, or
\latin{ager} (\xref{69}, \xref{70}), the Feminine like \latin{sella}
(\xref{65}), the Neuter like \latin{dōnum} (\xref{69}). Examples:
\begin{Tabular}{>{\itshape}l
                  lll
                  @{\qquad\qquad}
                  lll}

& \cc{6}{\latin{bonus}, \english{good}} \\

& \cc{3}{\textsc{singular}} & \cc{3}{\textsc{plural}} \\

& \cc{1}{\gender{m.}} & \cc{1}{\gender{f.}} & \cc{1}{\gender{n.}}
& \cc{1}{\gender{m.}} & \cc{1}{\gender{f.}} & \cc{1}{\gender{n.}}
\\

Nom. & bon\ending{us}
     & bon\ending{a}
     & bon\ending{um}
     & bon\ending{ī}
     & bon\ending{ae}
     & bon\ending{a}
\\

Gen. & bon\ending{ī}
     & bon\ending{ae}
     & bon\ending{ī}
     & bon\ending{ōrum}
     & bon\ending{ārum}
     & bon\ending{ōrum}
\\

Dat. & bon\ending{ō}
     & bon\ending{ae}
     & bon\ending{ō}
     & bon\ending{īs}
     & bon\ending{īs}
     & bon\ending{īs}
\\

Acc. & bon\ending{um}
     & bon\ending{am}
     & bon\ending{um}
     & bon\ending{ōs}
     & bon\ending{ās}
     & bon\ending{a}
\\

Voc. & bon\ending{e}
     & bon\ending{a}
     & bon\ending{um}
     & bon\ending{ī}
     & bon\ending{ae}
     & bon\ending{a}
\\

Abl. & bon\ending{ō}
     & bon\ending{ā}
     & bon\ending{ō}
     & bon\ending{īs}
     & bon\ending{īs}
     & bon\ending{īs}

\end{Tabular}

\subsubsection

The Gen.\ and Voc.\ Sing.\ Masc.\ and Neut.\ of adjectives in
\ending{-ius} end in \ending{-iī} and \ending{-ie}, not in
\ending{-ī}, as in Nouns; e.g.\ Gen.\ Sing. \latin{rēgiī},
Voc.\ Sing. \latin{rēgie}, from \latin{rēgius}, \english{royal}.

\section
\subtitle{\latin{līber}, \english{free}\hss\latin{ruber}, \english{red}}

\noindent
\begin{Tabular*}[\small]{@{}>{\itshape}l@{\enskip}
                         lll
                         @{\extracolsep{\fill}}
                         l@{\extracolsep{2\tabcolsep}}l@{\extracolsep{2\tabcolsep}}l@{}}

& \cc{6}{\textsc{singular}} \\[\smallskipamount]

Nom. & līber
     & līber\ending{a}
     & līber\ending{um}
     & ruber
     & rubr\ending{a}
     & rubr\ending{um}
\\

Gen. & līber\ending{ī}
     & līber\ending{ae}
     & līber\ending{ī}
     & rubr\ending{ī}
     & rubr\ending{ae}
     & rubr\ending{ī}
\\

Dat. & līber\ending{ō}
     & līber\ending{ae}
     & līber\ending{ō}
     & rubr\ending{ō}
     & rubr\ending{ae}
     & rubr\ending{ō}
\\

Acc. & līber\ending{um}
     & līber\ending{am}
     & līber\ending{um}
     & rubr\ending{um}
     & rubr\ending{am}
     & rubr\ending{um}
\\

Voc. & līber
     & līber\ending{a}
     & līber\ending{um}
     & ruber
     & rubr\ending{a}
     & rubr\ending{um}
\\

Abl. & līber\ending{ō}
     & līber\ending{ā}
     & līber\ending{ō}
     & rubr\ending{ō}
     & rubr\ending{ā}
     & rubr\ending{ō}
\\[\smallskipamount]\pagebreak

& \cc{6}{\textsc{plural}} \\[\smallskipamount]

Nom. & līber\ending{ī}
     & līber\ending{ae}
     & līber\ending{a}
     & rubr\ending{ī}
     & rubr\ending{ae}
     & rubr\ending{a}
\\

Gen. & līber\ending{ōrum}
     & līber\ending{ārum}
     & līber\ending{ōrum}
     & rubr\ending{ōrum}
     & rubr\ending{ārum}
     & rubr\ending{ōrum}
\\

Dat. & līber\ending{īs}
     & līber\ending{īs}
     & līber\ending{īs}
     & rubr\ending{īs}
     & rubr\ending{īs}
     & rubr\ending{īs}
\\

Acc. & līber\ending{ōs}
     & līber\ending{ās}
     & līber\ending{a}
     & rubr\ending{ōs}
     & rubr\ending{ās}
     & rubr\ending{a}
\\

Voc. & līber\ending{ī}
     & līber\ending{ae}
     & līber\ending{a}
     & rubr\ending{ī}
     & rubr\ending{ae}
     & rubr\ending{a}
\\

Abl. & līber\ending{īs}
     & līber\ending{īs}
     & līber\ending{īs}
     & rubr\ending{īs}
     & rubr\ending{īs}
     & rubr\ending{īs}

\end{Tabular*}

\subsubsection

The adjectives which are declined like \latin{līber} (not like
\latin{ruber}) are: \latin{asper}, \english{rough}; \latin{lacer},
\english{torn}; \latin{prosper}, \english{prosperous}; \latin{tener},
\english{tender}; compounds of \suffix{-fer} and \suffix{-ger}, like
\latin{āliger}, \english{winged}; sometimes \latin{dexter},
\english{right}.

\subsubsection

Some adjective \ending{-ro}-Stems form the Nom.\ Sing.\ Masc.\ in
\ending{-rus} instead of \ending{-er}, as is also the case with some
Nouns (\xref[\emph{a}]{70}).  Such are: \latin{ferus}, \english{wild},
\latin{properus}, \english{quick}, \latin{praeposterus},
\english{absurd}, and usually \latin{īnferus}, \english{under}, and
\latin{superus}, \english{upper}; further, all those in which the
\phone{r} is preceded by a long vowel, as \latin{sincērus},
\english{sincere}, \latin{austērus}, \english{austere}, etc.

\subsubsection

The declension of \latin{satur}, \english{full}, is parallel to that
of \latin{līber}, namely, \latin{satur}, \latin{satura},
\latin{saturum}, etc.

\headingB{Pronominal Adjectives}

\section

Several adjectives show in the Genitive and Dative Singular the
Pronominal endings \ending{-īus} and \ending{-ī}.  In other respects
they are declined like \latin{bonus}, or like \latin{līber} or
\latin{ruber}.  Examples of the Singular:
\begin{Tabular}{>{\itshape}l
                  lll
                  @{\qquad}
                  lll}

& \cc{3}{\latin{tōtus}, \english{whole}}
& \cc{3}{\latin{uter}, \english{which of two}}
\\

& \cc{1}{\gender{m.}} & \cc{1}{\gender{f.}} & \cc{1}{\gender{n.}}
& \cc{1}{\gender{m.}} & \cc{1}{\gender{f.}} & \cc{1}{\gender{n.}}
\\

Nom. & tōt\ending{us}
     & tōt\ending{a}
     & tōt\ending{um}
     & uter
     & utr\ending{a}
     & utr\ending{um}
\\

Gen. & tōt\ending{īus}
     & tōt\ending{īus}
     & tōt\ending{īus}
     & utr\ending{īus}
     & utr\ending{īus}
     & utr\ending{īus}
\\

Dat. & tōt\ending{ī}
     & tōt\ending{ī}
     & tōt\ending{ī}
     & utr\ending{ī}
     & utr\ending{ī}
     & utr\ending{ī}
\\

Acc. & tōt\ending{um}
     & tōt\ending{am}
     & tōt\ending{um}
     & utr\ending{um}
     & utr\ending{am}
     & utr\ending{um}
\\

Abl. & tōt\ending{ō}
     & tōt\ending{ā}
     & tōt\ending{ō}
     & utr\ending{ō}
     & utr\ending{ā}
     & utr\ending{ō}

\end{Tabular}

\begin{note}
In the Genitive ending \ending{-īus} the \phone{ī} is sometimes
shortened in poetry, especially in \latin{alterius} and, always, in
\latin{utriusque}.  See~\xref[note]{21}.
\end{note}

\subsubsection

The adjectives declined in this way are:
\begin{mexamples}[3]

\latin{alius}, \english{other}

\latin{ūllus}, \english{any}

\latin{nūllus}, \english{none}

\latin{sōlus}, \english{alone}

\latin{tōtus}, \english{whole}

\latin{ūnus}, \english{one}

\latin{alter}, \english{the other}

\latin{uter}, \english{which \(of two\)}

\latin{neuter}, \english{neither}

\end{mexamples}

\subsubsection

The Nom.-Acc.\ Sing.\ Neut.\ of \latin{alius} is \latin{aliud}; the
Genitive Singular is usually supplied by \latin{alterīus}.

\begin{note}

Early and rare forms are \latin{alis} and \latin{alid}, for
\latin{alius} and \latin{aliud}; also Dat.\ Sing. \latin{alī} for
\latin{aliī}, and Gen.\ Sing.\ \latin{alīus} and \latin{aliī}.

\end{note}

\subsubsection

The Dat.\ Sing.\ Fem.\ of \latin{alter} is sometimes \latin{alterae}.

\chapter{Adjectives of the Third Declension}

\contentsentry{C}{Adjectives of the Third Declension}

\section

Adjectives of the Third Declension are conveniently classified
according to the number of endings in the Nominative Singular, namely,
\emph{one}, \emph{two}, or \emph{three}.

\headingG{Adjectives of Three Endings}

\section
\subtitle{\latin{ācer}, \english{sharp}}

\begin{Tabular}{>{\itshape}l
                  lll
                  @{\qquad\qquad}
                  lll}

& \cc{3}{\textsc{singular}} & \cc{3}{\textsc{plural}} \\

& \gender{m.} & \gender{f.} & \gender{n.}
& \gender{m.} & \gender{f.} & \gender{n.}
\\

Nom. & ācer
     & ācr\ending{is}
     & ācr\ending{e}
     & ācr\ending{ēs}
     & ācr\ending{ēs}
     & ācr\ending{ia}
\\

Gen. & ācr\ending{is}
     & ācr\ending{is}
     & ācr\ending{is}
     & ācr\ending{ium}
     & ācr\ending{ium}
     & ācr\ending{ium}
\\

Dat. & ācr\ending{ī}
     & ācr\ending{ī}
     & ācr\ending{ī}
     & ācr\ending{ibus}
     & ācr\ending{ibus}
     & ācr\ending{ibus}
\\

Acc. & ācr\ending{em}
     & ācr\ending{em}
     & ācr\ending{e}
     & ācr\ending{īs} (\ending{-ēs})
     & ācr\ending{īs} (\ending{-ēs})
     & ācr\ending{ia}
\\

Voc. & ācer
     & ācr\ending{is}
     & ācr\ending{e}
     & ācr\ending{ēs}
     & ācr\ending{ēs}
     & ācr\ending{ia}
\\

Abl. & ācr\ending{ī}
     & ācr\ending{ī}
     & ācr\ending{ī}
     & ācr\ending{ibus}
     & ācr\ending{ibus}
     & ācr\ending{ibus}

\end{Tabular}

\subsubsection

All adjectives of this type are from stems in~\infix{-ri-}, the
Nom.\ Sing.\ Masc.\ becoming~\ending{-er}, as in Nouns like \latin{imber}
(\xref{87}).  But \emph{some} stems in \infix{-ri-}, as
\latin{fūnebris}, \latin{muliebris}, \latin{inlūstris}, etc., have the
Nom. Sing. Masc. in \ending{-ris}, and so belong to the next class.

\headingG{Adjectives of Two Endings}

\section
\subtitle{\latin{gravis}, \english{heavy}}

\begin{Tabular}{>{\itshape}l
                  ll
                  @{\qquad}
                  ll}

& \cc{2}{\textsc{singular}} & \cc{2}{\textsc{plural}} \\

& \gender{m., f.} & \gender{n.}
& \gender{m., f.} & \gender{n.}
\\

Nom. & grav\ending{is}
     & grav\ending{e}
     & grav\ending{ēs}
     & grav\ending{ia}
\\

Gen. & grav\ending{is}
     & grav\ending{is}
     & grav\ending{ium}
     & grav\ending{ium}
\\

Dat. & grav\ending{ī}
     & grav\ending{ī}
     & grav\ending{ibus}
     & grav\ending{ibus}
\\

Acc. & grav\ending{em}
     & grav\ending{e}
     & grav\ending{īs} (\ending{-ēs})
     & grav\ending{ia}
\\

Voc. & grav\ending{is}
     & grav\ending{e}
     & grav\ending{ēs}
     & grav\ending{ia}
\\

Abl. & grav\ending{ī}
     & grav\ending{ī}
     & grav\ending{ibus}
     & grav\ending{ibus}

\end{Tabular}

\subsubsection

All adjectives of this type are \phone{i}-Stems.

\headingB{Comparatives}

\section
\subtitle{\latin{melior}, \english{better}}

\enlargethispage{\baselineskip}

\begin{Tabular}{>{\itshape}l
                  ll
                  @{\qquad}
                  ll}

& \cc{2}{\textsc{singular}} & \cc{2}{\textsc{plural}} \\

& \gender{m., f.} & \gender{n.}
& \gender{m., f.} & \gender{n.}
\\

Nom. & melior
     & melius
     & meliōr\ending{ēs}
     & meliōr\ending{a}
\\

Gen. & meliōr\ending{is}
     & meliōr\ending{is}
     & meliōr\ending{um}
     & meliōr\ending{um}
\\

Dat. & meliōr\ending{ī}
     & meliōr\ending{ī}
     & meliōr\ending{ibus}
     & meliōr\ending{ibus}
\\

Acc. & meliōr\ending{em}
     & melius
     & meliōr\ending{ēs} (\ending{-īs})
     & meliōr\ending{a}
\\

Voc. & melior
     & melius
     & meliōr\ending{ēs}
     & meliōr\ending{a}
\\

Abl. & meliōr\ending{e}
     & meliōr\ending{e}
     & meliōr\ending{ibus}
     & meliōr\ending{ibus}

\end{Tabular}

\subsubsection

The Comparatives are properly \phone{s}-Stems, the \phone{s} being
preserved only in the Nom.-Acc.\ Sing.\ Neut.  Compare \latin{honor}
(\latin{honōs}), \latin{honōris} (\xref[4]{80}).

\subsubsection

\latin{Plūs}, \english{more}, in the Singular used only as a Neuter,
has Gen.\ Plur.\ \latin{plūrium}, but
Nom.-Acc.\ Plur.\ Neut.\ \latin{plūra} (but \latin{complūria} beside
\latin{complūra}).

\headingG{Adjectives of One Ending}

\subtitle{(\emph{Including Present Participles})}

\section
\subtitle{\latin{duplex}, \english{double}}

\begin{Tabular}{>{\itshape}l
                  ll
                  @{\qquad}
                  ll}

& \cc{2}{\textsc{singular}} & \cc{2}{\textsc{plural}} \\

& \gender{m., f.} & \gender{n.}
& \gender{m., f.} & \gender{n.}
\\

Nom. & duplex
     & duplex
     & duplic\ending{ēs}
     & duplic\ending{ia}
\\

Gen. & duplic\ending{is}
     & duplic\ending{is}
     & duplic\ending{ium}
     & duplic\ending{ium}
\\

Dat. & duplic\ending{ī}
     & duplic\ending{ī}
     & duplic\ending{ibus}
     & duplic\ending{ibus}
\\

Acc. & duplic\ending{em}
     & duplex
     & duplic\ending{īs} (\ending{-ēs})
     & duplic\ending{ia}
\\

Voc. & duplex
     & duplex
     & duplic\ending{ēs}
     & duplic\ending{ia}
\\

Abl. & duplic\ending{ī}
     & duplic\ending{ī}
     & duplic\ending{ibus}
     & duplic\ending{ibus}
\\
% \end{Tabular}
% 
% \begin{Tabular}{>{\itshape}l
%                   ll
%                   @{\qquad}
%                   ll}

& \cc{4}{\latin{amāns}, \english{loving}} \\

& \cc{2}{\textsc{singular}} & \cc{2}{\textsc{plural}} \\

& \gender{m., f.} & \gender{n.}
& \gender{m., f.} & \gender{n.}
\\

Nom. & amāns
     & amāns
     & amant\ending{ēs}
     & amant\ending{ia}
\\

Gen. & amant\ending{is}
     & amant\ending{is}
     & amant\ending{ium}
     & amant\ending{ium}
\\

Dat. & amant\ending{ī}
     & amant\ending{ī}
     & amant\ending{ibus}
     & amant\ending{ibus}
\\

Acc. & amant\ending{em}
     & amāns
     & amant\ending{īs} (\ending{-ēs})
     & amant\ending{ia}
\\

Voc. & amāns
     & amāns
     & amant\ending{ēs}
     & amant\ending{ia}
\\

Abl. & amant\ending{e} (\ending{-ī})
     & amant\ending{e} (\ending{-ī})
     & amant\ending{ibus}
     & amant\ending{ibus}
\\
% \end{Tabular}
% 
% \begin{Tabular}{>{\itshape}l
%                   ll
%                   @{\qquad}
%                   ll}

& \cc{4}{\latin{vetus}, \english{old}} \\

& \cc{2}{\textsc{singular}} & \cc{2}{\textsc{plural}} \\

& \gender{m., f.} & \gender{n.}
& \gender{m., f.} & \gender{n.}
\\

Nom. & vetus
     & vetus
     & veter\ending{ēs}
     & veter\ending{a}
\\

Gen. & veter\ending{is}
     & veter\ending{is}
     & veter\ending{um}
     & veter\ending{um}
\\

Dat. & veter\ending{ī}
     & veter\ending{ī}
     & veter\ending{ibus}
     & veter\ending{ibus}
\\

Acc. & veter\ending{em}
     & vetus
     & veter\ending{ēs}
     & veter\ending{a}
\\

Voc. & vetus
     & vetus
     & veter\ending{ēs}
     & veter\ending{a}
\\

Abl. & veter\ending{e}
     & veter\ending{e}
     & veter\ending{ibus}
     & veter\ending{ibus}

\end{Tabular}

\subsubsection

These Adjectives are Consonant-Stems in origin, but, with the
exception of \latin{vetus} and a few others, they have taken on the
characteristic \phone{i}-Stem forms in the Plural, and for the most
part in the Ablative Singular.  For details, see~\xref{118}.

\begin{minor}

\subsubsection

Of the various classes of Consonant-Stems the Mute-Stems are the most
frequent.  The union of the mute with the \phone{s} of the
Nom.\ Sing., and the changes in the stem between the Nom.\ Sing.\ and
the other cases are in accordance with the statements given above for
Nouns (\xref{77}).  So
\latin{duplex}, \english{double}, Gen.\ \latin{duplicis};
\latin{particeps}, \english{sharing}, Gen.\ \latin{participis};
\latin{dīves}, \english{rich}, Gen.\ \latin{dīvitis}.
Peculiar are the compounds of \latin{caput}, as
\latin{anceps}, \english{two-headed}, Gen.\ \latin{ancipitis};
\latin{praeceps}, \english{headlong}, Gen.\ \latin{praecipitis}.

There are also a few stems in \ending{-l}, \ending{-r}, and
\ending{-s}, as
\latin{vigil}, \english{watchful}, Gen.\ \latin{vigilis};
\latin{memor}, \english{mindful}, Gen.\ \latin{memoris};
\latin{pūbēs}, \english{grown up}, Gen.\ \latin{pūberis};
\latin{vetus}, \english{old}, etc.

\end{minor}

\headingC{Remarks on the Case-Forms}

\section
\subsection

Adjectives of the Third Declension have the \phone{i}-Stem forms of
the Ablative Singular, Genitive Plural, and Nominative and Accusative
Plural Neuter, namely, \suffix{-ī}, \suffix{-ium}, \suffix{-ia}.
But Comparatives have the Consonant-Stem forms, namely, \suffix{-e},
\suffix{-um}, \suffix{-a}. Present Participles have \suffix{-ium} and
\suffix{-ia}, but the Ablative Singular in~\suffix{-e}, unless used in
an Adjective sense, when they usually have \suffix{-ī};
e.g.\ \latin{eō praesente}, \english{in his presence}, but
\latin{praesentī tempore}, \english{at the present time}.

\subsubsection
Exceptions:
\begin{enumerate}

\item
Adjectives of Two or Three Endings.  A Gen.\ Plur.\ in \ending{-um} is
regular in \latin{celer}, \english{swift}, \latin{vo\-lu\-cris},
\english{flying}, occasional in \latin{caelestis}, \english{heavenly},
\latin{agrestis}, \english{rustic}, but rare elsewhere.

\item
Adjectives of One Ending.  An Abl.\ Sing.\ in \ending{-e} and a
Gen.\ Plur.\ in \ending{-um} are regular in the following
(Nom.-Acc.\ Plur.\ Neut.\ wanting in most):
\begin{mexamples*}

\leavevmode\llap{*}\latin{caeles}, \english{heavenly}, Gen.\ \latin{caelitis}

\latin{compos}, \english{master of}

\latin{particeps}, \english{sharing}

\latin{pauper}, \english{poor}

\latin{prīnceps}, \english{chief}

\latin{pūbēs}, \english{grown up}

\latin{impūbēs}, \english{under age}

\latin{sōspes}, \english{safe}

\latin{superstes}, \english{remaining}

\latin{dīves}, \english{rich} (but \latin{dītia})

\latin{vetus}, \english{old} (also \latin{vetera})

\end{mexamples*}

\savecounter{enumi}

\end{enumerate}

A Gen.\ Plur.\ in \ending{-um} is also regular in \latin{inops},
\english{needy}, \latin{memor}, \english{mindful}, \latin{vigil},
\english{ watchful}, and in compounds of \latin{pēs}, \english{foot},
such as \latin{bipēs}, \latin{quadrupēs}.
\begin{enumerate}

\restorecounter{enumi}

\begin{minor}

\item

In other Adjectives of One or Two Endings an Abl.\ Sing.\ in
\ending{-e} is occasionally found, chiefly in poetry;
e.g. \latin{grave}, \latin{duplice}.

\item

Comparatives.  An Ablative in \ending{-ī} is rare.  For \latin{plūs}
see \xref[\emph{b}]{116}.

\item

Present Participles. A Gen.\ Plur.\ in \ending{-um} is found in
poetry.

\end{minor}

\end{enumerate}

\subsection

Adjectives used substantively retain their usual forms, as
Abl.\ \latin{nātālī},
\linebreak
\english{birthday}.  But when they are used as
proper names the Ablative generally ends in~\ending{-e}, as
\latin{Iuvenāle}, \latin{Qui\-rī\-nā\-le}.

\subsection

Participles used substantively retain their usual forms, as in
\latin{ā sapiente}, \english{by a wise man}.

\subsection

The Acc.\ Plur.\ Masc.\ and Fem.\ had the regular \phone{i}-Stem
form~\ending{-īs}, and this was in general more persistently retained
than in Nouns, although forms in~\ending{-ēs} are also found in the
Augustan period.  But the words which had the
Gen.\ Plur.\ in~\ending{-um} had the Consonant-Stem form of the
Acc.\ Plur., namely, \ending{-ēs}, from the outset.

\chapter{Comparison of Adjectives\protect\footnotemark}

\footnotetext{The Comparison of Adjectives is a matter belonging more
  properly to Word-Formation than to Inflection, but is conveniently
  treated in connection with the Declension of Adjectives.}

\contentsentry{B}{Comparison of Adjectives}

\section

There are three Degrees of Comparison, as in English, namely, the
\emph{Positive}, the \emph{Comparative}, and the \emph{Superlative}.

The Comparative is regularly formed by adding \suffix{-ior}, the
Superlative by add\-ing \suffix{-issimus}, to the stem of the Positive
minus its final vowel, if it has one.  The Declension of Comparatives
has been given (\xref{116}).  Superlatives are declined as Adjectives
of the First and Second Declensions.  Participles used as Adjectives
are compared in the same manner.  Examples of Comparison:
\begin{Tabular}{lll}

  \textsc{positive}
& \textsc{comparative}
& \textsc{superlative}
\\

\latin{clārus}, \english{clear}
& \latin{clārior}, \english{clearer}
& \latin{clārissimus}, \english{clearest}
\\

\latin{gravis}, \english{heavy}
& \latin{gravior}, \english{heavier}
& \latin{gravissimus}, \english{heaviest}
\\

\latin{audāx}, \english{bold}
& \latin{audācior}, \english{bolder}
& \latin{audācissimus}, \english{boldest}
\\

\latin{amāns}, \english{loving}
& \latin{amantior}, \english{more loving}
& \latin{amantissimus}, \english{most loving}

\end{Tabular}

\section
\subsection

Adjectives in~\ending{-er} form the Superlative in~\ending{-errimus},
as if by adding \ending{-rimus} to the~\ending{-er}.  Examples;
\begin{Tabular}{lll}

\latin{asper}, \english{rough}
& \latin{asperior}
& \latin{asperrimus}
\\

\latin{celer}, \english{swift}
& \latin{celerior}
& \latin{celerrimus}
\\

\latin{ācer}, \english{sharp}
& \latin{ācrior}
& \latin{ācerrimus}

\end{Tabular}

\subsubsection

So also \latin{vetus}, Superlative \latin{veterrimus}.  The old
Comparative \latin{veterior} is replaced by \latin{vetustior}, formed
from \latin{vetustus}.  \latin{Mātūrus}, \english{ripe}, has once a
Superlative \latin{mātūrrimus}, but usually \latin{mātūrissimus}.

\subsection

Certain adjectives in \ending{-ilis} form the Superlative in
\ending{-illimus}, as if by adding \ending{-limus} to the stem of the
Positive minus its final vowel.  Examples:
\begin{Tabular}{lll}

\latin{facilis}, \english{easy}
& \latin{facilior}
& \latin{facillimus}
\\

\latin{gracilis}, \english{slender}
& \latin{gracilior}
& \latin{gracillimus}
\\

\latin{humilis}, \english{lowly}
& \latin{humilior}
& \latin{humillimus}
\\

\latin{similis}, \english{like}
& \latin{similior}
& \latin{simillimus}

\end{Tabular}

\subsubsection

So also \latin{difficilis}, \latin{dissimilis}.  Other adjectives in
\ending{-ilis} are compared in the usual manner, as \latin{nōbilis},
\latin{nōbilior}, \latin{nōbilissimus}.  But many of them lack the
Superlative.

\begin{note}

The stems of the Superlatives in \ending{-illimus} and
\ending{-errimus} come from \rec{-il-simo-}, \rec{-er-simo-}
(cf.\ \infix{-is-simo-}), the \phone{s} being assimilated to the
preceding \phone{l} or~\phone{r} (\xref[11]{49}).

\end{note}

\subsection

Adjective compounds in \ending{-volus}, \ending{-dicus},
\ending{-ficus} have Comparatives and Superlatives which belong
properly to compounds in \latin{-volēns}, \ending{-dīcēns},
\ending{-ficēns}, of which, except in the case of \ending{-ficēns},
examples occur in early Latin; e.g.\ \latin{be\-ne\-vo\-lēns},
\latin{ma\-le\-dī\-cēns}. Examples:
\begin{center}
%% This should be a Tabular, but somehow that causes an orphaned
%% footnoterule to appear on the following page.
\begin{tabular}{lll}

\latin{benevolus}, \english{benevolent}
& \latin{benevolentior}
& \latin{benevolentissimus}
\\

\latin{maledicus}, \english{slanderous}
& \latin{maledīcentior}
& \latin{maledīcentissimus}
\\

\latin{magnificus}, \english{eminent}
& \latin{magnificentior}
& \latin{magnificentissimus}

\end{tabular}
\end{center}

\begin{minor}

\subsection

There are a few Superlatives in \ending{-mus}, \ending{-imus},
\ending{-timus}, and \ending{-ēmus}, which are cited in~\xref{122},
\xref{123}.  So \latin{sum-mus} (\rec{sup-mos}; see~\xref[10]{49}),
\latin{min-imus}, \latin{op-timus}, \latin{supr-ēmus}.

\end{minor}

\section

Many adjectives form the Comparative and Superlative by prefixing
\latin{magis}, \english{more}, and \latin{maximē}, \english{most}, to
the Positive. This is true of most adjectives in which the vowel of
the stem is itself preceded by another vowel, and of many others.
Examples:
\begin{Tabular}{lll}

\latin{dubius}, \english{doubtful}
& Comp.\ \latin{magis dubius}
& Superl.\ \latin{maximē dubius}
\\

\latin{idōneus}, \english{suitable}
& Comp.\ \latin{magis idōneus}
& Superl. \latin{maximē idōneus}

\end{Tabular}

\begin{note}

Some adjectives are compared by means of \latin{magis} and
\latin{maximē}, as well as by the usual method; e.g.\ \latin{ēlegāns},
\english{select}, Comp.\ \latin{ēlegantior} or \latin{magis ēlegāns},
Superl.\ \latin{ēlegantissimus} or \latin{maximē ēlegāns}.

\end{note}

\headingC{Peculiar or Defective Comparison}

\section

Several adjectives show two or three different stems in the three
Degrees, or different forms of the same stem.  Compare English
\english{good}, \english{better}, \english{best}.  Such are:
\begin{Tabular}{lll}

  \latin{bonus}, \english{good}
& \latin{melior}, \english{better}
& \latin{optimus}, \english{best}
\\

  \latin{malus}, \english{bad}
& \latin{peior}, \english{worse}
& \latin{pessimus}, \english{worst}
\\

  \latin{magnus}, \english{great}
& \latin{maior}, \english{greater}
& \latin{maximus}, \english{greatest}
\\

  \latin{multus}, \english{much}
& \latin{plūs}, \english{more}
& \latin{plūrimus}, \english{most}
\\

  \latin{parvus},  \english{small}
& \latin{minor},   \english{smaller}
& \latin{minimus}, \english{smallest}
\\

  \latin{nēquam} (indecl.), \english{worthless}
& \latin{nēquior}
& \latin{nēquissimus}
\\

  \latin{frūgī} (indecl.), \english{thrifty}
& \latin{frūgālior}
& \latin{frūgālissimus}

\versionA{\\}%
\versionA{\latin{iuvenis}, \english{young} &}%
\versionA{\latin{iūnior} (\latin{iuvenior} late) &}%
\versionA{[\latin{nātū minimus}]}%

\versionB*{\\}%
\versionB*{\latin{iuvenis}, \english{young} &}%
\versionB*{\latin{iūnior} [\latin{minor nātū}]&}%
\versionB*{[\latin{minimus nātū}]}%

\versionA{\\}%
\versionA{\latin{senex}, \english{old}&}%
\versionA{\latin{senior}&}%
\versionA{[\latin{nātū maximus}]}%

\versionB*{\\}%
\versionB*{\latin{senex}, \english{old}&}%
\versionB*{\latin{senior} [\latin{maior nātū}]&}%
\versionB*{[\latin{maximus nātū}]}%

\end{Tabular}

\section

In the case of some adjectives the Positive is wholly lacking, or is
rare except in certain expressions. But the stem of the Positive often
appears in adverbial or prepositional forms. Such are:
\begin{Tabular*}[\small]{l@{\extracolsep{\fill}}ll}

cis, citrā, \english{on this side}
& \latin{citerior}, \english{on this side}
& \latin{citimus}, \english{nearest}
\\

uls, ultrā, \english{beyond}
& \latin{ulterior}, \english{farther}
& \latin{ultimus}, \english{farthest}
\\

in, intrā, \english{within}
& \latin{interior}, \english{inner}
& \latin{intimus}, \english{innermost}
\\

\group{\text{exterus (\latin{nātiōnēs exterae}},\\ \quad\english{foreign nations})}
& \group{\latin{exterior}, \english{outer}\\\strut}
& \hskip-8.5pt\groupLR{\latin{extrēmus},\\ \latin{extimus},} \english{outermost}
\\

prope, \english{near}
& \latin{propior}, \english{nearer}
& \latin{proximus}, \english{nearest}
\\

prae, prō, \english{before}
& \latin{prior}, \english{former}
& \latin{prīmus}, \english{first}
\\

dē, \english{down}
& \latin{dēterior}, \english{worse}
& \latin{dēterrimus}, \english{worst}
\\

\na
& \latin{potior}, \english{preferable}
& \latin{potissimus}, \english{strongest}
\\

\na
& \latin{ōcior}, \english{swifter}
& \latin{ōcissimus}, \english{swiftest}
\\

īnferus, \english{below}
& \latin{īnferior}, \english{lower}
& \hskip-8.5pt\groupLR{\latin{īnfimus},\\ \latin{īmus},} \english{lowest}
\\[10pt]

superus, \english{above}
& \latin{superior}, \english{higher}
& \hskip-8.5pt\groupLR{\latin{suprēmus},\\ \latin{summus},} \english{highest}
\\[10pt]

posterus, \english{following}
& \latin{posterior}, \english{later}
& \hskip-8.5pt\groupL{\latin{postrēmus}, \english{last}\\
           \latin{postumus}, \english{late-born}}

\end{Tabular*}

\subsubsection

The Comparative is wanting for \latin{novus}, \english{new},
\latin{sacer}, \english{sacred}, \latin{pius}, \english{pious}
(Superl.\ \latin{piissimus}), and rare for \latin{fīdus},
\english{faithful}, \latin{falsus}, \english{false}, and others.

\subsubsection

The Superlative is wanting for \latin{iuvenis}, \english{young}, and
\latin{senex}, \english{old} (but see~\xref{122}), and for some
others, including many adjectives in \ending{-ilis}, \ending{-bilis}.

\chapter*{Adverbs}

\subtitle{(\textsc{Prepositions and Conjunctions})}

\contentsentry{B}{Formation of Adverbs, Prepositions, and
  Conjunctions}

\smallskip

\section

Although Adverbs are not themselves capable of inflection, they are
most conveniently treated at this point, because many of them are
regularly formed from Nouns and Adjectives, and with endings which are
identical with the Case-endings.

\begin{note}

It is believed that \emph{all} Adverbs are, in their ultimate origin,
nothing but stereotyped Case-forms.  Some of them show endings which
appear as Case-endings in related languages, but have become obsolete
as such in Latin.  Still others, especially among Adverbs formed from
Pronominal Stems, show endings which even in the parent speech were
used only in Adverbs, not as real Case-endings.

\end{note}

\section

Prepositions and Conjunctions are Adverbs in origin, and some of
\linebreak
them,
which show the common adverbial formations, are cited among the
examples of such formations.  But many of them, including most of the
commonest Prepositions, do not admit of any analysis or classification
as regards \emph{form}.  They are, therefore, treated only as regards
their \emph{uses}, i.e.\ under the head of Syntax.

\section

The common Adverbial endings are:
\begin{enumerate*}

\item
\ending{-ē} (\ending{-e}), as in \latin{altē}, \english{highly}, from
\latin{altus}; \latin{cārē}, \english{dearly}, from \latin{cārus};
\latin{male}, \english{badly}, from \latin{malus}; \latin{bene},
\english{well}, from \latin{bonus}.  This is the usual ending of
Adverbs formed from Adjectives of the First and Second
Declensions. For \latin{ferē} and \latin{fermē}, \english{nearly}, the
Adjective forms are lacking.

\begin{note}

This ending appears on early inscriptions as \ending{-ēd}, which was
once an Ablative ending of \phone{o}-Stems existing beside that
in~\ending{-ōd}, but has become obsolete in Latin, except in Adverbs.
For the short~\phone{e} in \latin{male} and \latin{bene},
see~\xref[note]{28}.

\end{note}

\item
\latin{-ter}, \latin{-iter}, as in \latin{audācter},
\english{boldly}, from \latin{audāx}; \latin{graviter},
\english{heavily}, from \latin{gravis}; \latin{hūmāniter},
\english{humanely}, from \latin{hūmānus}.  This is the usual ending of
Adverbs
\linebreak
formed from Adjectives of the Third Declension, but is not
confined to these.

\begin{note}

This ending is probably the same as that seen in such Adverbs and
Prepositions as \latin{inter}, \latin{subter}, etc., from which it was
extended, but with a loss of its distinctly local force (a transition
which might readily take place in such a word as \latin{circiter},
\english{about}).

\end{note}

\item

\suffix{-ō}, (\latin{-o}), as in \latin{tūtō}, \english{safely}, from
\latin{tūtus}; \latin{prīmō}, \english{at first}, from \latin{prīmus};
\latin{cito}, \english{quickly}, from \latin{citus}; \latin{modo},
\english{only}, from \latin{modus}.  So also the Pronominal Adverbs
\latin{eō}, \latin{quō}, etc.; cf.\ also \latin{retrō}, and, in
composition only, \latin{contrō-}.

\begin{note}

This is the Ablative ending, originally \ending{-ōd}.  For the
short~\phone{o} in \latin{modo} and \latin{cito}, see~\xref[note]{28}.

\end{note}

\item

\ending{-ā}, as in \latin{dextrā}, \english{on the right}, from
\latin{dexter}; \latin{aliā}, \english{otherwise}, from \latin{alius};
\latin{rēctā}, \english{straightway}, from \latin{rēctus}; and other
Adverbs of place.  So also the Pronominal Adverbs \latin{eā},
\latin{quā}, \latin{hāc}, \latin{posteā}, \latin{posthāc}, etc., and
Prepositions like \latin{extrā}.

\begin{note}

This appears on early inscriptions as \ending{-ād}, and is the
Ablative ending of the Feminine (originally, perhaps, \latin{eā viā},
etc.).

\end{note}

\item

\ending{-tim} (\latin{-sim}), as in \latin{fūrtim},
\english{secretly}, from \latin{fūr}; \latin{prīvātim},
\english{privately}, from \latin{prī\-vā\-tus}; \latin{cursim},
\english{quickly}, etc.

\begin{note}

These adverbs originated in forms like \latin{partim},
\english{partly}, from \latin{pars}, in which \ending{-tim} represents
the Acc.\ Sing.\ of a stem in~\infix{-ti-}.

\end{note}

\item

\ending{-um}, as in \latin{multum}, \english{much}, from
\latin{multus}; \latin{pos\-trē\-mum}, \english{finally}, from
\latin{pos\-trē\-mus}; \latin{vērum}, \english{but}, from \latin{vērus};
\latin{cēterum}, \english{for the rest}, from \rec{cēterus}.  So also
the Pronominal Adverbs \latin{tum}, \latin{dum}, \latin{cum}, and the
Preposition \latin{circum}.

\begin{note}

This is the ending of the Acc.\ Sing.\ Neut.\ of \phone{o}-Stems.  The
same Case is seen in the adverbs in \ending{-ius} from Comparatives
(see \xref[note]{128}), and in a few forms in~\ending{-e} from
\phone{i}-Stems, as \latin{facile}, \english{easily}, from
\latin{facilis}; also in the Conjuction \latin{quod}.  The
Acc.\ Plur.\ Neut.\ is seen in the Conjunction \latin{quia}.

\end{note}

\item
\ending{-am}, as in \latin{clam}, \english{secretly}, \latin{palam},
\english{openly}, \latin{cōram}, \english{openly}.  So the Pronominal
Adverbs \latin{tam}, \latin{iam}, \latin{quam}, etc.

\begin{note}

This is the ending of the Acc.\ Sing.\ Fem.  The Acc.\ Plur.\ Fem.\ is
seen in \latin{aliās}, \english{at other times}, and \latin{forās},
\english{out of doors}.

\end{note}

\item
\ending{-tus}, as in \latin{funditus}, \english{from the bottom}, from
\latin{fundus}; \latin{intus}, \english{from within}.

\begin{note}

This is an old suffix~\suffix{-tos}, used also in related languages to
denote source.

\end{note}

\end{enumerate*}

\begin{minor}

\section

Other endings, used chiefly with Pronominal Stems, and mostly of
obscure origin, are:
\begin{enum1}

\item
\ending{-nde}, as in \latin{inde}, \english{thence}, \latin{unde},
\english{whence}.

\item
\ending{-dem}, as in \latin{tandem}, \english{at last},
\latin{prīdem}, \english{long ago}.

\item
\ending{-dam}, as in \latin{quondam}, \english{once}.

\item
\ending{-dum}, as in \latin{dūdum}, \english{a while ago}.

\item
\ending{-dō}, as in \latin{quandō}, \english{when}.

\item
\ending{-im}, \ending{-inc}, as in \latin{illim}, \latin{illinc},
\english{thence}, \latin{hinc}, \english{hence}.

\item
\ending{-īc}, as in \latin{hīc}, \english{here}, \latin{illīc},
\english{there}.  These are Locatives in \ending{-ī-c(e)}.

\item
\ending{-bi} (\ending{-bī}), as in \latin{ibi}, \english{there},
\latin{ubi}, \english{where}.

\item
\ending{-per}, as in \latin{semper}, \english{always}, \latin{nūper},
\english{lately}.

\end{enum1}

\end{minor}

\chapter{Comparison of Adverbs}

\contentsentry{B}{Comparison of Adverbs}

\section

In Comparison the Adverb follows the formation of the Adjective,
except that the Comparative ends in \ending{-ius}, and the Superlative
in \ending{-ē}.  Examples:
\begin{Tabular}{lll}

\latin{altē}, \english{highly}
& \latin{altius}, \english{more highly}
& \latin{altissimē}, \english{most highly}
\\

\latin{audācter}, \english{boldly}
& \latin{audācius}
& \latin{audācissimē}
\\

\latin{ācriter}, \english{fiercely}
& \latin{ācrius}
& \latin{ācerrimē}
\\

\latin{facile}, \english{easily}
& \latin{facilius}
& \latin{facillimē}
\\

\latin{tūtō}, \english{safely}
& \latin{tūtius}
& \latin{tūtissimē}

\end{Tabular}

\begin{note}

The Comparative Adverb is simply the Acc.\ Sing.\ Neut.\ of the
Comparative Adjective, used adverbially; the Superlative is formed
from the Superlative Adjective with the regular adverbial
ending~\ending{-ē}.  Adverbs are also compared by prefixing \latin{magis}
and \latin{maximē}.

\end{note}

\headingC{Special Peculiarities}

\section

The following show two or three different stems in the three Degrees
(like the corresponding Adjectives; see~\xref{122}), or are otherwise
peculiar or defective.
\begin{Tabular}{lll}

\latin{bene}, \english{well}
& \latin{melius}, \english{better}
& \latin{optimē}, \english{best}
\\

\latin{male}, \english{ill}
& \latin{peius}, \english{worse}
& \latin{pessimē}, \english{worst}
\\

\groupR{\latin{magnopere},\\ \latin{multum}}
\group{\english{greatly},\\ \english{much}}
& \latin{magis}, \english{more}
& \latin{maximē}, \english{most}
\\

\latin{multum}, \english{much}
& \latin{plūs}, \english{more}
& \latin{plūrimum}, \english{most}
\\

\latin{parum}, \english{little}
& \latin{minus}, \english{less}
& \latin{minimē}, \english{least}
\\

\latin{satis}, \english{enough}
& \latin{satius}, \english{better}
& \na
\\

\na
& \latin{potius}, \english{rather}
& \latin{potissimum}, \english{especially}
\\

\na
& \latin{prius}, \english{before}
& \latin{prīmum}, \english{first}
\\

\latin{nūper}, \english{recently}
& \na
& \latin{nūperrimē}, \english{most recently}
\\

\latin{saepe}, \english{often}
& \latin{saepius}, \english{oftener}
& \latin{saepissimē}, \english{oftenest}
\\

\latin{diū}, \english{long}
& \latin{diūtius}, \english{longer}
& \latin{diūtissimē}, \english{longest}
\\

\latin{prope}, \english{near}
& \latin{propius}, \english{nearer}
& \latin{proximē}, \english{nearest}, \english{next}

\end{Tabular}

\begin{note}

\latin{Sētius}, \english{less}, is not related to \latin{secus},
\english{otherwise}.

\end{note}

\headingB{Numerals}

\contentsentry{B}{Numerals: Forms and Inflection}

\chapter{Cardinals and Ordinals}

\section

Cardinals answer the question “How many?”  Ordinals, the question
“Which in order?”
\begin{Tabular*}[\small]{@{}r@{\extracolsep{\fill}}llr@{}}

& \textsc{Cardinals}
& \textsc{Ordinals}
& \M{1}{r@{}}{\textsc{Roman Notation}}
\endhead

1.  & ūnus, \english{one}
    & prīmus, \english{first}
    & \textsc{i}
\\

2.  & duo, \english{two}
    & secundus, \english{second}
    & \textsc{ii}
\\

3.  & trēs
    & tertius
    & \textsc{iii}
\\

4.  & quattuor
    & quārtus
    & \textsc{iiii} \emph{or} \textsc{iv}
\\

5.  & quīnque
    & quīntus
    & \textsc{v}
\\

6.  & sex
    & sextus
    & \textsc{vi}
\\

7.  & septem
    & septimus
    & \textsc{vii}
\\

8.  & octō
    & octāvus
    & \textsc{viii}
\\

9.  & novem
    & nōnus
    & \textsc{viiii} \emph{or} \textsc{ix}
\\

10. & decem
    & decimus
    & \textsc{x}
\\

11. & ūndecim
    & ūndecimus
    & \textsc{xi}
\\

12. & duodecim
    & dudecimus
    & \textsc{xii}
\\

13. & tredecim
    & tertius decimus
    & \textsc{xiii}
\\

14. & quattuordecim
    & quārtus decimus
    & \textsc{xiiii} \emph{or} \textsc{xiv}
\\

15. & quīndecim
    & quīntus decimus
    & \textsc{xv}
\\

16. & sēdecim
    & sextus decimus
    & \textsc{xvi}
\\

17. & septendecim
    & septimus decimus
    & \textsc{xvii}
\\

18. & duodēvīgintī
    & duodēvīcēnsimus
    & \textsc{xviii}
\\

19. & ūndēvīgintī
    & ūndēvīcēnsimus
    & \textsc{xviiii} \emph{or} \textsc{xix}
\\

20. & vīgintī
    & vīcēnsimus
    & \textsc{xx}
\\

21. & vīgintī ūnus
    & vīcēnsimus prīmus
    & \textsc{}
\\

    & \quad (ūnus et vīgintī)
    & \quad (ūnus et vīcēnsimus)
    & \textsc{xxi}
\\

22. & vīgintī duo
    & vīcēnsimus secundus
    & \textsc{}
\\

    & \quad (duo et vigintī)
    & \quad (alter et vīcēnsimus)
    & \textsc{xxii}
\\

30. & trīgintā
    & trīcēnsimus
    & \textsc{xxx}
\\

40. & quadrāgintā
    & quadrāgēnsimus
    & \textsc{xxxx} \emph{or} \textsc{xl}
\\

50. & quīnquāgintā
    & quīnquāgēnsimus
    & \textsc{l}
\\

60. & sexāgintā
    & sexāgēnsimus
    & \textsc{lx}
\\

70. & septuāgintā
    & septuāgēnsimus
    & \textsc{lxx}
\\

80. & octōgintā
    & octōgēnsimus
    & \textsc{lxxx}
\\

90. & nōnāgintā
    & nōnāgēnsimus
    & \textsc{lxxxx} \emph{or} \textsc{xc}
\\

100.
    & centum
    & centēnsimus
    & \textsc{c}
\\

101.
    & centum (et) ūnus
    & centēnsimus prīmus
    & \textsc{ci}
\\

120.
    & centum (et) vīgintī
    & centēnsimus vīcēnsimus
    & \textsc{cxx}
\\

121.
    & centum vīgintī ūnus
    & centēnsimus vīcēnsimus\logical{ prīmus}
    & \textsc{cxxi}
\\

\visual{&&\quad prīmus\\}%

200.
    & ducentī
    & ducentēnsimus
    & \textsc{cc}
\\

300.
    & trecentī
    & trecentēnsimus
    & \textsc{ccc}
\\

400.
    & quadringentī
    & quadringentēnsimus
    & \textsc{cccc}
\\

500.
    & quīngentī
    & quīngentēnsimus
    & \textsc{d}
\\

600.
    & sescentī
    & sescentēnsimus
    & \textsc{dc}
\\

700.
    & septingentī
    & septingentēnsimus
    & \textsc{dcc}
\\

800.
    & octingentī
    & octingentēnsimus
    & \textsc{dccc}
\\

900.
    & nōngentī
    & nōngentēnsimus
    & \textsc{dcccc}
\\

1000.
    & mīlle
    & mīllēnsimus
    & (earlier \textsc{ciↄ}) \textsc{m}
\\

1120.
    & mīlle centum\logical{ vīgintī}
    & mīllēnsimus centēnsimus\logical{ vīcēnsimus}
    & \textsc{mcxx}
\\

\visual{&\quad vīgintī&\quad vīcēnsimus\\}%

1900.
    & mīlle nōngentī
    & mīllēnsimus\logical{ nōngentēnsimus}
    & \textsc{mdcccc}
\\

\visual{&&\quad nōngentēnsimus\\}%

2000.
    & duo mīlia
    & bis mīllēnsimus
    & \textsc{mm}
\\

10,000.
    & decem mīlia
    & deciēns mīllēnsimus
    & \textsc{\=x}
\\

100,000.
    & centum mīlia
    & centiēns mīllēnsimus
    & \textsc{\=c}
\\

1,000,000.
    & deciēns centēna\logical{ mīlia}
    & deciēns centiēns\logical{ mīllēnsimus}
    & \boxedX
\\

\visual{&\quad mīlia&\quad mīllēnsimus\\}%

\end{Tabular*}

\begin{note}

For some of the numeral signs, other forms, not resembling Latin
letters, were commonly used in inscriptions, especially in the early
period.  \textsc{m} for 1000 did not replace \textsc{ci\reflectbox{c}}
until the second century~\ad\ For numbers like 4, 9, 14, etc., the
method of notation by adding was commoner than the method by
subtracting; so, for example, \textsc{viiii} is usual, \textsc{ix}
rare.

\end{note}

\headingC{Declension of Cardinals and Ordinals}

\section

Both Cardinals and Ordinals are Adjectives, and the latter are
declined like bonus (\xref{110}).  But of the Cardinals up to~100,
only the first three are declined.

\subsection

\latin{Ūnus} is declined like \latin{tōtus} (\xref{112}).

\subsection

\latin{Duo} and \latin{trēs} are declined as follows:
\begin{Tabular}{>{\itshape}llll@{\qquad\qquad}ll}

Nom.    & duo
        & duae
        & duo
        & trēs
        & tria
\\

Gen.    & duōrum
        & duārum
        & duōrum
        & trium
        & trium
\\

Dat.    & duōbus
        & duābus
        & duōbus
        & tribus
        & tribus
\\

Acc.    & duōs (duo)
        & duās
        & duo
        & trīs (trēs)
        & tria
\\

Abl.    & duōbus
        & duābus
        & duōbus
        & tribus
        & tribus

\end{Tabular}

\begin{note}

Like \latin{duo} is declined \latin{ambō}, \latin{ambae},
\latin{ambō}, \english{both}.

\end{note}

\subsection

The plural of \latin{mīlle} is \latin{mīlia}, declined like
\latin{tria}.  It is not an Adjective like \latin{mīlle}, but a
Substantive, and is followed by the Genitive; for example, \latin{cum
  mīlle mīlitibus}, \english{with a thousand soldiers}, but \latin{cum
  duōbus mīlibus mīlitum}, \english{with two thousand soldiers}.

\subsection

The hundreds, \latin{ducentī}, etc., are declined like the plural of
\latin{bonus}, but the Genitive ends in \ending{-um}, not
in~\ending{-ōrum}.

\begin{note}

The older spelling \latin{mīllia} was supplanted by \latin{mīlia} in
the first century~\ad* The Ordinals like \latin{vīcēnsimus} are also
spelled \latin{vīcēsimus}, etc; but the spelling \latin{-ēnsimus} is
preferable (\xref[5]{52}).  An early spelling of \latin{septimus} and
\latin{decimus} is \latin{septumus} and \latin{decumus}
(\xref[2]{52}).

\end{note}

\headingC{Order of Words in Compound Numerals}

\section
\subsection

The method of making the compound numerals from 20 to~100 is the same
as in English; just as we say either \english{twenty-one} or
\english{one and twenty} (rarely twenty and one), so the
Romans said \latin{vīgintī ūnus} or \latin{ūnus et vīgintī} (rarely
\latin{vi\-gin\-tī et ūnus}).

\subsection

The compound numerals from 100 on regularly begin with the largest
number and descend to the smallest, just as in English.  If there are
only two numbers, \latin{et} is sometimes used, sometimes not.  But if
there are more than two numbers \latin{et} is never used.  So
\latin{trecentī ūnus} or \latin{trecentī et ūnus}, \english{301}, but
\latin{trecentī quadrāgintā ūnus}, \english{341}, and \latin{mīlle
  ducentī trīgintā duo}, \english{1232}.

\subsection

Compound numerals are sometimes used for the numbers 11–19, the large
number usually preceding, as \latin{decem et octō}.

\chapter{Distributives, Multiplicatives, and Numeral Adverbs}

\section

Distributives denote how many apiece, as \latin{singulī}, \english{one
  apiece}, \english{one by one}.  Multiplicatives denote how many
fold, as \latin{duplex}, \english{twofold}, \english{double}.  Numeral
Adverbs denote how many times, as \latin{bis}, \english{twice}.  The
following is a partial list:
\begin{Tabular*}{rl@{\extracolsep{\fill}}ll}
    & \textsc{Distributives}
    & \textsc{Multiplicatives}
    & \textsc{Numeral Adverbs}
\endhead

1.  & singulī, \english{one apiece}
    & simplex, \english{simple}
    & semel, \english{once}
\\

2.  & bīnī, \english{two apiece}
    & duplex, \english{double}
    & bis, \english{twice}
\\

3.  & ternī (trīni)
    & triplex
    & ter
\\

4.  & quaternī
    & quadruplex
    & quater
\\

5.  & quīnī
    & quīncuplex
    & quīnquiēns
\\

6.  & sēnī
    &
    & sexiēns
\\

7.  & septēnī
    & septemplex
    & septiēns
\\

8.  & octōnī
    &
    & octiēns
\\

9.  & novēnī
    &
    & noviēns
\\

10. & dēnī
    & decemplex
    & deciēns
\\

11. & ūndēnī
    &
    & ūndeciēns
\\

12. & duodēnī
    &
    & duodeciēns
\\

13. & ternīdēnī
    &
    & terdeciēns
\\

20. & vīcēnī
    &
    & vīciēns
\\

21. & vīcēnī singulī
    &
    & semel et vīciēns
\\

30. & trīcēnī
    &
    & trīciēns
\\

100.
    & centēnī
    & centuplex
    & centiēns
\\

101.
    & centēnī singulī
\\

200.
    & ducēnī
    &
    & ducentiēns
\\

1000.
    & singula mīlia
    &
    & mīliēns

\end{Tabular*}

\subsubsection

For the use of Distributives in place of Cardinals, see under
Syntax~(\xref{247}).

\begin{note}

The Numeral Adverbs \latin{sexiēns}, etc., are also spelled
\latin{sexiēs}, etc., but the spell\-ing \suffix{-iēns} is preferable
(\xref[5]{52}).

\end{note}

\pagebreak

\headingB{Pronouns}

\contentsentry{B}{Declension of Pronouns}

\chapter{Personal Pronouns}

\section

The Personal Pronouns of the First and Second Persons are declined
as follows:
\begin{Tabular}{>{\itshape}lll@{\qquad}ll}

    & \textsc{singular} & \textsc{plural}
    & \textsc{singular} & \textsc{plural}
\\

Nom.    & ego, \english{I}
        & nōs, \english{we}
        & tū, \english{thou}
        & vōs, \english{you}
\\

Gen.    & meī
        & nostrum, nostrī
        & tuī
        & vestrum, vestrī
\\

Dat.    & mihi (mī)
        & nōbīs
        & tibi
        & vōbīs
\\

Acc.    & mē
        & nōs
        & tē
        & vōs
\\

Voc.    & \na
        & \na
        & tū
        & vōs
\\

Abl.    & mē
        & nōbīs
        & tē
        & vōbīs

\end{Tabular}

\subsection

Beside \latin{mihi} and \latin{tibi}, the old forms with final long
\phone{i} are frequent in poetry (\xref[note]{28}).

\subsection

The Genitive Plural ends in \ending{-um} or \ending{-ī} according to
the meaning.  \latin{Nostrum} and \latin{ves\-trum} are used as
Genitives of the Whole, \latin{nostrī} and \latin{vestrī} as Objective
Genitives.  Early and late forms of \latin{vestrum} and \latin{vestrī}
are \latin{vostrum}, \latin{vostrī}.

\begin{minor}

\subsection

Old forms of the Genitive Singular are \latin{mīs}, \latin{tīs}; of
the Accusative and Ablative Singular \latin{mēd} and \latin{tēd}
(similarly \latin{sēd}).

\subsection

The particles \latin{met} and \latin{te} are added to the pronominal
form for emphasis; \latin{egomet}, \english{I myself}; \latin{tūte},
\english{you yourself} (also \latin{tūtemet}).

\subsection

For the Third Person the Determinative Pronoun \latin{is} (\xref{137})
is used.

\end{minor}

\chapter{Reflexive Pronouns}

\section

For the First and Second Person the ordinary forms of the Personal
Pronoun are used with the reflexive sense, as \latin{laudō mē},
\english{I praise myself}, \latin{laudās tē}, \english{you praise
  yourself}, \latin{laudāmus nōs}, \english{we praise ourselves}.  For
the Third Person there is a distinct Reflexive Pronoun, without
distinction of gender or number, which is declined as follows:
\begin{Tabular}{>{\itshape}l
                @{\enskip}l
                @{\,\,}>{\itshape}l
                *{4}{@{\,\,}>{\itshape}c}}

Gen. & suī,      & of & himself, & herself, & itself, & themselves \\

Dat. & sibi,     & to & \ditto[himself] & \ditto[herself] & \ditto[itself] & \ditto[themselves] \\

Acc. & sē, sēsē, &    & \ditto[himself] & \ditto[herself] & \ditto[itself] & \ditto[themselves] \\

Abl. & sē, sēsē, & by & \ditto[himself] & \ditto[herself] & \ditto[itself] & \ditto[themselves]

\end{Tabular}

\subsubsection

Beside \latin{sibi}, the old form with final long~\phone{i} is
frequent in poetry (\xref[note]{28}).

\chapter{Possessives}

\section

The Adjective forms of the Personal and Reflexive Pronouns are known
as Possessives.  They are:
\begin{Tabular}{l@{\qquad}l}

meus, mea, meum, \english{my};
& noster, nostra, nostrum, \english{our};
\\

tuus, tua, tuum, \english{thy};
& vester, vestra, vestrum, \english{your};
\\

\cc{2}{suus, sua, suum, \english{his}, \english{her}, \english{its},
  \english{their}.}

\end{Tabular}

\subsubsection

They are declined as regular Adjectives of the First and Second
Declensions. But the Vocative Singular of \latin{meus} is \latin{mī}.

\subsubsection

An early and late form of \latin{vester}, \ending{-tra},
\ending{-trum} is \latin{voster}, \ending{-tra}, \ending{-trum}.

\subsubsection

The enclitic \enclitic{-pte} is frequently added to the Ablative
Singular for emphasis, as \latin{meōpte ingeniō}, \english{by my own
  genius}; \latin{suāpte nātūrā}, \english{by its own nature}.

\subsubsection

\latin{Suus} is used only in the reflexive sense, \english{his}
(\english{her}, \english{their}, etc.)\ \english{own}.  For the
Possessive of the Third Person when not reflexive, the Genitive of
\latin{is} is used, as \latin{eius} (\english{of him}, etc.),
\english{his}, \english{her}, \english{its}; \latin{eōrum}, \latin{eārum},
\english{their}.

\chapter{Determinative-Descriptive Pronouns}

\section

The Pronoun \latin{is}, \english{this} (or \english{he}) or
\english{such}, and its compound \latin{īdem}, \english{the same}, are
declined as follows:
\begin{Tabular}[\small]{>{\itshape}llll}

\cc{4}{\latin{is}} \\

\cc{4}{\textsc{singular}} \\

& \gender{m.} & \gender{f.} & \gender{n.} \\

Nom.    & is
        & ea
        & id
\\

Gen.    & eius
        & eius
        & eius
\\

Dat.    & eī
        & eī
        & eī
\\

Acc.    & eum
        & eam
        & id
\\

Abl.    & eō
        & eā
        & eō
\\[\medskipamount]

\cc{4}{\textsc{plural}} \\

Nom.    & iī (ī), eī
        & eae
        & ea
\\

Gen.    & eōrum
        & eārum
        & eōrum
\\

Dat.    & iīs (īs), eīs
        & iīs (īs), eīs
        & iīs (īs), eīs
\\

Acc.    & eōs
        & eās
        & ea
\\

Abl.    & iīs (īs), eīs
        & iīs (īs), eīs
        & iīs (īs), eīs
\\

\\[\bigskipamount]

\cc{4}{\latin{īdem}} \\

\cc{4}{\textsc{singular}} \\

& \gender{m.} & \gender{f.} & \gender{n.} \\

Nom.    & īdem
        & eadem
        & idem
\\

Gen.    & eiusdem
        & eiusdem
        & eiusdem
\\

Dat.    & eīdem
        & eīdem
        & eīdem
\\

Acc.    & eundem
        & eandem
        & idem
\\

Abl.    & eōdem
        & eādem
        & eōdem
\\[\medskipamount]

\cc{4}{\textsc{plural}} \\

Nom.    & īdem (iīdem), eīdem
        & eaedem
        & eadem
\\

Gen.    & eōrundem
        & eārundem
        & eōrundem
\\

Dat.    & īsdem (iīsdem), eīsdem
        & īsdem (iīsdem), eīsdem
        & īsdem (iīsdem), eīsdem
\\

Acc.    & eōsdem
        & eāsdem
        & eāsdem
\\

Abl.    & īsdem (iīsdem), eīsdem
        & īsdem (iīsdem), eīsdem
        & īsdem (iīsdem), eīsdem

\end{Tabular}

\subsubsection

The Gen.\ Sing.\ \latin{eius} was pronounced \sound{ei-yus}, the first
syllable containing a diphthong and being long for this reason
(\xref[2, \emph{a}]{29}).

\begin{minor}

\subsubsection

The Nom.\ Plur.\ Masc.\ and the Dat.-Abl.\ Plur.\ of \latin{is} were
oftenest \emph{written} \latin{iī}, \latin{iīs}, but these were
\emph{pronounced}, and not infrequently written also, \latin{ī},
\latin{īs}.  The forms \latin{eī}, \latin{eīs} are also frequent, but
poetic usage shows that dissyllabic pronunciation was rare.  The same
is true of the corresponding cases of \latin{īdem}, except that
\latin{īdem} and \latin{īsdem}, which represent the actual
pronunciation, are also the commonest spellings.

\subsubsection

The Dative Singular appears in early poetry as \latin{ēī}, \latin{eī},
or monosyllabic~\latin{ei}.

\end{minor}

\section
\subsection

\latin{Hic}, \english{this} or \english{such}, and \latin{ille},
\english{that} or \english{such}, are declined as follows;
\begin{Tabular}{>{\itshape}llll@{\qquad}lll}

& \cc{6}{\textsc{singular}} \\[\smallskipamount]

& \gender{m.} & \gender{f.} & \gender{n.}
& \gender{m.} & \gender{f.} & \gender{n.} \\

Nom.    & hic
        & haec
        & hoc
        & ille
        & illa
        & illud
\\

Gen.    & huius
        & huius
        & huius
        & illīus
        & illīus
        & illīus
\\

Dat.    & huic
        & huic
        & huic
        & illī
        & illī
        & illī
\\

Acc.    & hunc
        & hanc
        & hoc
        & illum
        & illam
        & illud
\\

Abl.    & hōc
        & hāc
        & hōc
        & illō
        & illā
        & illō
\\[\medskipamount]

& \cc{6}{\textsc{plural}} \\

Nom.    & hī
        & hae
        & haec
        & illī
        & illae
        & illa
\\

Gen.    & hōrum
        & hārum
        & hōrum
        & illōrum
        & illārum
        & illōrum
\\

Dat.    & hīs
        & hīs
        & hīs
        & illīs
        & illīs
        & illīs
\\

Acc.    & hōs
        & hās
        & haec
        & illōs
        & illās
        & illa
\\

Abl.    & hīs
        & hīs
        & hīs
        & illīs
        & illīs
        & illīs

\end{Tabular}

\subsection

\latin{Iste}, \english{that} or \english{such}, is declined like
\latin{ille}.

\subsubsection

For \latin{hic} and \latin{hoc} as long syllables, see~\xref[2]{30}.

\subsubsection

The Gen.\ Sing.\ \latin{huius} was prononced \sound{hui-yus}, the
first syllable containing a diphthong and being long for this reason
(\xref[2, \emph{a}]{29}); for the pronunciation of the
Dat.\ Sing.\ \latin{huic}, see~\xref{10}.  The earlier forms
\latin{hoius} and \latin{hoic} were still used in Cicero’s time.

\subsubsection

The particle \particle{-c(e)}, always present in \latin{hic},
\latin{haec}, etc., is often added to other forms.  Thus
\latin{huiusce}, \latin{haec} (Nom.\ Plur.\ Fem.), \latin{hōsce},
\latin{hāsce}, \latin{hīsce}, and, in early Latin, also
\latin{hōrunc}, \latin{hārunc}.  Similarly early Latin \latin{illic}
and \latin{istic}, declined as follows (the Neuter forms \latin{istuc}
and \latin{istaec} also used later);
\begin{Tabular}{>{\itshape}llll@{\qquad}lll}

& \cc{3}{\textsc{singular}}
& \cc{3}{\textsc{plural}}
\\

& \gender{m.} & \gender{f.} & \gender{n.}
& \gender{m.} & \gender{f.} & \gender{n.} \\

Nom.    & illic
        & illaec
        & illuc
        &
        & illaec
        & illaec
\\

Gen.    & illīusce
        & illīusce
        & illīusce
        &
        &
        &
\\

Dat.    & illīc
        & illīc
        & illīc
        & illīsce
        & illīsce
        & illīsce
\\

Acc.    & illunc
        & illanc
        & illuc
        & illōsce
        & illāsce
        & illaec
\\

Abl.    & illōc
        & illāc
        & illōc
        & illīsce
        & illīsce
        & illīsce

\end{Tabular}

\subsubsection

The interrogative particle \enclitic{-ne} is sometimes added to forms
in~\ending{-ce}, the \phone{e} of the latter changing to~\phone{i}
(\xref[2]{42}); e.g.\ \latin{hic(c)ine}, \latin{haecine},
\latin{hoc(c)ine}, etc. (So, too, the adverb \latin{hīcine},
\english{in this place?}  Cf.\ \latin{sīcine}, \english{in this
  way?}\ similarly formed from \latin{sīce}, the old form of
\latin{sīc}.)

\begin{minor}

\subsubsection

Early Latin has a Nom.\ Plur.\ Masc. \latin{hīsce}.

\subsubsection

Some forms of early Latin \latin{olle} or \latin{ollus}, used like
\latin{ille}, occur also in later writers;
e.g.\ Dat.\ Sing.\ \latin{ollī}, Nom.\ Plur.\ Masc.\ \latin{ollī},
Dat.\ Abl.\ Plur.~\latin{ollīs}.

\end{minor}

\subtitle{\textsc{The Intensive Pronoun}}

\smallskip

\section

The Intensive Pronoun \latin{ipse}, \english{self}, is declined as
follows:
\begin{Tabular}{>{\itshape}llll@{\qquad}lll}

& \cc{3}{\textsc{singular}}
& \cc{3}{\textsc{plural}}
\\

& \gender{m.} & \gender{f.} & \gender{n.}
& \gender{m.} & \gender{f.} & \gender{n.}
\\

Nom.    & ipse
        & ipsa
        & ipsum
        & ipsī
        & ipsae
        & ipsa
\\

Gen.    & ipsīus
        & ipsīus
        & ipsīus
        & ipsōrum
        & ipsārum
        & ipsōrum
\\

Dat.    & ipsī
        & ipsī
        & ipsī
        & ipsīs
        & ipsīs
        & ipsīs
\\

Acc.    & ipsum
        & ipsam
        & ipsum
        & ipsōs
        & ipsās
        & ipsa
\\

Abl.    & ipsō
        & ipsā
        & ipsō
        & ipsīs
        & ipsīs
        & ipsīs

\end{Tabular}

\subsubsection

Early Latin has also Nom.\ Sing.\ Masc.\ \latin{ipsus}.  Note
\latin{ea-pse}, \latin{eam-pse}, \latin{eā-pse}, (\latin{reāpse}),
etc.

\chapter{The Relative Pronouns}

\section

The Relative Pronoun \latin{quī}, \english{who}, is declined as follows:
\begin{Tabular}{>{\itshape}llll@{\qquad\qquad}lll}

& \cc{3}{\textsc{singular}}
& \cc{3}{\textsc{plural}}
\\

& \gender{m.} & \gender{f.} & \gender{n.}
& \gender{m.} & \gender{f.} & \gender{n.}
\\

Nom.    & quī
        & quae
        & quod
        & quī
        & quae
        & quae
\\

Gen.    & cuius
        & cuius
        & cuius
        & quōrum
        & quārum
        & quōrum
\\

Dat.    & cui
        & cui
        & cui
        & quibus
        & quibus
        & quibus
\\

Acc.    & quem
        & quam
        & quod
        & quōs
        & quās
        & quae
\\

Abl.    & quō
        & quā
        & quō
        & quibus
        & quibus
        & quibus

\end{Tabular}

\begin{minor}

\subsubsection

The Gen.\ and Dat.\ Sing.\ \latin{cuius} and \latin{cui} were
pronounced in the same manner as \latin{huius} and \latin{huic}.  See
above, \xref[2, \emph{b}]{138}.  The earlier forms \latin{quoius} and
\latin{quoi} were still used in Cicero’s time.

\end{minor}

\subsubsection

An Abl.\ Sing.\ \latin{quī} in place of \latin{quō}, \latin{quā}, is
frequent in the phrase \latin{quīcum}, \english{with whom} or
\english{with which}.  The adverb \latin{quī}, \english{whereby}, also
used interrogatively, is of the same origin.

\subsubsection

A Dat.-Abl.\ Plur.\ \latin{quīs} in place of \latin{quibus} is
frequent.

\subsubsection

Other Relatives are: \latin{quīcumque}, \english{whoever}, with the
\latin{quī} declined as above; \latin{quisquis}, \english{whoever},
with both parts declined like \latin{quis} of the following paragraph
(but only \latin{quisquis}, \latin{quidquid} or \latin{quicquid}
(\xref{50}), and \latin{quōquō} in common use); \latin{uter},
\english{which of two}, the declension of which is given above
(\xref{112}), and \latin{utercumque}, \english{whichever of two}, the
first part of which is declined in the same way.

\chapter{The Interrogative Pronouns}

\section

%%* loose line

The Interrogative Pronoun, when used Substantively, is \latin{quis},
\english{who?}
\linebreak
When used Adjectively, it is \latin{quī},
\english{what?}\ (e.g.\ \latin{quī deus}, \english{what
  god?}). \latin{Quī} is declined like the Relative.  The declension
of \latin{quis}, differing from that of \latin{quī} only in a few
forms, is as follows:
\begin{Tabular}{>{\itshape}lll@{\qquad\qquad}lll}

& \cc{2}{\textsc{singular}}
& \cc{3}{\textsc{plural}}
\\

& \gender{m.}, \gender{f.} & \gender{n.}
& \gender{m.} & \gender{f.} & \gender{n.}
\\

Nom.    & quis
        & quid
        & quī
        & quae
        & quae
\\

Gen.    & cuius
        & cuius
        & quōrum
        & quārum
        & quōrum
\\

Dat.    & cui
        & cui
        & quibus
        & quibus
        & quibus
\\

Acc.    & quem
        & quid
        & quōs
        & quās
        & quae
\\

Abl.    & quō
        & quō
        & quibus
        & quibus
        & quibus

\end{Tabular}

\subsubsection

The distinction between the substantive and adjective forms is not
always maintained; \latin{quis} is sometimes used adjectively, and,
\emph{vice versa}, \latin{quī} is sometimes used substantively (hence
the Fem.\ \latin{quae} also occurs substantively, although the proper
substantive form is \latin{quis} for both Masculine and Feminine).

\subsubsection

Other Interrogatives are: \latin{quisnam}, \english{who, pray?}\ with
the Adjective form \latin{quīnam}; \latin{ecquis}, \english{any one?}\
Adjective \latin{ecquī} (Nom.\ Sing.\ Fem.\ \latin{ecquae} or
\latin{ecqua}); \latin{uter}, \english{which of two?}\ declined
in~\xref{112}.

\begin{note}

The stem is \stem{quo-} in the Relative forms \latin{quī} (earlier
\latin{quoi}) and \latin{quod}, but \stem{qui-} in the Interrogative
forms \latin{quis} and \latin{quid}.  The other forms, which are the
same for both Relative and Interrogative, are from the stem
\stem{quo-}, except \latin{quem} and \latin{quibus}, which are from
the stem \stem{qui-} (\latin{quem} like \latin{fīnem}).  But the
\latin{quī} of \latin{quīcum} (\xref[\emph{b}]{140}) is also from
\stem{qui-}, and, \emph{vice versa}, Dat.-Abl.\ Plur.\ \latin{quīs}
for \latin{quibus} is from \stem{quo-}.  A rare
Nom.\ Plur.\ \latin{quēs} (Interrog.\ and Indef.)\ is also from
\stem{qui-} (like \latin{fīnēs}).  A third stem \stem{quu-}, belonging
properly to adverbial formations, appears in the form \stem{cu-}
(cf.\ \latin{quīncu-plex} from \rec{quīnquu-plex}) in \latin{alicubi},
etc., and, with the loss of the initial consonant, in \latin{ubi},
\latin{unde}, \latin{ut}, and \latin{uter}.

\end{note}

\chapter{Indefinite and Distributive Pronouns}

\section

The principal Indefinite Pronouns are \latin{quis} (\latin{quī}) and
its various compounds.  They are used both substantively and
adjectively.  In Substantive use the Neuter is \latin{quid}, and,
except in a few of the compounds, \latin{quis} is used for both the
Masculine and the Feminine gender; in the Adjective use the Neuter is
\latin{quod}, and \latin{quī} and \latin{quae} (or \latin{qua}) are
used for the Masculine and the Feminine gender.
\begin{distributives}

& \cc{2}{\textsc{Used Substantively}} & \textsc{Used Adjectivally}
\endhead

1.
& quis (quī), \english{any one}
& quid, \english{anything}
& quī (quis), quae \emph{or} qua, quod, \english{any}
\\
& \note{For the Nom.\ Sing.\ Fem.\ and the Nom.-Acc.\ Plur.\ Neut.,
  both \latin{quae} and \latin{qua} are used.}
\\% \pagebreak

2.
& aliquis (aliquī), \english{some one}
& aliquid, \english{something}
& aliquī (aliquis), aliqua, aliquod,\break \english{some}
\\
& \note{The Nom.\ Sing.\ Fem.\ nearly always, and the
Nom.-Acc.\ Plur.\ Neut.\ always, is \latin{aliqua}.}
\\

3.
& quīdam, quaedam, \english{a certain one}
& quiddam, \english{a certain thing}
& quīdam, quaedam, quoddam,\break \english{a certain}
\\
& \note{As in the declension of \latin{īdem}, \phone{m} is changed
to~\phone{n} before~\phone{d}; e.g.\ \latin{quendam} (for
\rec{quemdam}), \latin{quandam}, etc.}
\\

4.
& quispiam, \english{some one}
& quippiam or quidpiam (\xref{50}), \english{something}
& quispiam, quaepiam, quodpiam,\break \english{some}
\\

5.
& quisquam, \english{any one at all}
& quicquam, \english{any thing at all}
& quisquam, quicquam, \english{any} (rare)
\\
& \note{There is no Plural.  The Adjective use is commonly supplied by
\latin{ūllus}.}
\\

6.
& quisque, \english{each one}
& quidque, \english{each thing}
& quisque, quaeque, quodque, \english{each}
\\

7.
& ūnusquisque, \english{each one severally}
& ūnumquidque, \english{each thing severally}
& ūnusquisque, ūnaquaeque, ūnumquodque, \english{each severally}
\\

8.
& quīvīs, quaevīs, \english{any one whatever}
& quidvīs, \english{anything whatever}
& quīvīs, quaevīs, quodvīs, \english{any\break whatever}
\\

9.
& quīlibet, quaelibet, \english{any one}
& quidlibet, \english{anything whatever}
& quīlibet, quaelibet, quodlibet, \english{any whatever}
\end{distributives}

\subsubsection

The following compounds of \latin{uter} have the force of Indefinite or
Distributive Pronouns, in both substantive and adjective use.  For their
declension, see~\xref{112}.
\begin{examples}

\latin{uterque}, \latin{utraque}, \latin{utrumque}, \english{each of
  two}

\latin{utervīs}, \latin{utravīs}, \latin{utrumvīs}, \english{either of two}

\latin{uterlibet}, \latin{utralibet}, \latin{utrumlibet},
\english{either of two}

\latin{alteruter}, \latin{alterutra} or \latin{altera utra},
\latin{alterutrum} or \latin{alterum utrum}, \english{one or the other}

\end{examples}

\begin{note}

In \latin{alteruter} sometimes both parts are declined, sometimes only
the latter.

\end{note}

\subtitle{\textsc{Pronominal Adjectives}}

\contentsentry{B}{Pronomial Adjectives}

\section

Besides the Adjective forms of the Pronouns already given may be
mentioned:
\begin{mexamples}

\latin{tālis}, \latin{tāle}, \english{such}

\latin{quālis}, \latin{quāle}, \english{such as} or \english{of what
  sort?}

\latin{tantus}, \latin{tanta}, \latin{tantum}, \english{so great}

\latin{quantus}, \latin{quanta}, \latin{quantum}, \english{so great
  as} or \english{how great?}

\latin{alius}, \latin{alia}, \latin{aliud}, \english{another}

\latin{alter}, \latin{altera}, \latin{alterum}, \english{the other}

\latin{neuter}, \latin{neutra}, \latin{neutrum}, \english{neither of
  two}

\latin{ūllus}, \latin{ūlla}, \latin{ūllum}, \english{any}

\latin{nūllus}, \latin{nūlla}, \latin{nūllum}, \english{no one}

\latin{nōnnūllus}, \latin{nōnnūlla}, \latin{nōnnūllum},
\english{some}, \english{many a}

\end{mexamples}

\begin{note}

For the declension of the last six forms, see \xref{112}.

\end{note}

% \pagebreak

\headingC{Correlatives}

\contentsentry{B}{Correlative Pronouns, Adjectives, and Adverbs}

\section

Adjectives and Adverbs which stand to each other in the relation of
corresponding Determinative, Interrogative, Relative, and Indefinite
words are called Correlatives.  A partial list is:
\begin{correlatives}

\textsc{Determinative}
& \textsc{Relative}
& \textsc{Interrogative}
& \textsc{Indefinite}
\endhead

\latin{is}, \latin{hic}, etc., \english{this}
& \latin{quī}, \english{who}
& \latin{quis}, \english{who?}
& \latin{aliquis}, \english{any one}
\\

\latin{tālis}, \english{such}
& \latin{quālis}, \english{as}
& \latin{quālis}, \english{of what sort?}
\\

\latin{tantus}, \english{so great}
& \latin{quantus}, \english{as great}
& \latin{quantus}, \english{how great?}
& \latin{aliquantus}, \english{somewhat}
\\

\latin{tot}, \english{so many}
& \latin{quot}, \english{as many}
& \latin{quot}, \english{how many?}
& \latin{aliquot}, \english{several}
\\

\latin{ibi}, \english{there}
& \latin{ubi}, \english{where}
& \latin{ubi}, \english{where?}
& \latin{alicubi}, \english{anywhere}
\\

\latin{inde}, \english{thence}
& \latin{unde}, \english{whence}
& \latin{unde}, \english{whence?}
& \latin{alicunde}, \english{from somewhere}
\\

\latin{eō}, \english{thither}
& \latin{quō}, \english{whither}
& \latin{quō}, \english{whither?}
& \latin{aliquō}, \english{to some place}
\\

\latin{tum}, \english{then}
& \latin{cum}, \english{when}
& \latin{quandō}, \english{when?}
& \latin{aliquandō}, \english{sometime}
\\

\latin{totiēns}, \english{so many\logical{ times}}
& \latin{quotiēns}, \english{as many\logical{ times}}
& \latin{quotiēns}, \english{how many\logical{ times}}
& \latin{aliquotiēns}, \english{several\logical{ times}}
\visual{%
\\
  \quad\english{times}
& \quad\english{times}
& \quad\english{times?}
& \quad\english{times}
}

\end{correlatives}

\chapter*{Verbs}

\section

The Inflection of Verbs, or Conjugation, comprises the variations in
Voice, Mood, Tense, Number, and Person.  There are:
\begin{indented}

Two Voices,—Active and Passive.

\begin{indented}

Some Verbs have only one Voice.  Those which are mostly Passive in
form but Active in meaning are known as Deponents.

\end{indented}

Three Moods,—Indicative, Subjunctive, and Imperative.

Six Tenses,—Present, Imperfect, Future; Perfect, Past
Perfect,\footnote{Commonly, and properly, so named in English
  grammars; commonly called Pluperfect in Latin grammars.} and Future
Perfect.  

\begin{indented}
Only the Indicative has all six Tenses.  The Subjunctive
lacks the Future and the Future Perfect.  The Imperative has only the
Present and the Future.
\end{indented}

Two Numbers,—Singular and Plural.

Three Persons,—First, Second, and Third.

\end{indented}

\section

The Indicative, Subjunctive, and Imperative forms make up what is
known as the Finite Verb.

Besides these, the following Noun and Adjective forms have become a
part of the Verb-System:
\begin{indented}

Verbal Nouns,—Infinitives (Present, Future, and Perfect of both
Voices), the Supine, and the Gerund.

Verbal Adjectives,—Participles (Present and Future Active, Perfect
Passive\footnote{The form commonly known as the Perfect Passive
  Participle is not always Perfect or always Passive.  Similary the
  term Future Passive Participle does not properly describe the
  functions of this form. See the Syntax.}, and Future
Passive\footnotemark[\thefootnote] or Gerundive).

\end{indented}

\headingB{The Three Stems of the Verb}

\contentsentry{B}{Stems of the Verb}

\section

There are three principal Stems about which are grouped the various
forms of the Verb.

\begin{note}[Note 1]

As, in declension, the Stem is the base to which the Case-endings are
added, so, in Conjugation, the Stem of any given Tense is the base to
which the Personal Endings are added.  These stems, the formation of
which, by means of suffixes known as Tense-Signs or Mood-Signs, is
treated below (\xref[ff.]{166}), are conveniently grouped under the
three principal stems, as given above.  Not all tenses of the Present
System, for example, are actually formed directly from the Present
Stem, but most of them are formed from stems which \emph{contain} the
Present Stem with certain fixed additions or substitutions.

\end{note}

\begin{note}[Note 2]

The part which is common to all three stems is known as the Verb-Stem,
that is, the general stem of the verb.  Thus in a verb like
\latin{amō}, \latin{amāre}, \latin{amāvī}, \latin{amātum},
\latin{amā-} is the Verb-Stem, as well as Present Stem.  Often the
only part which is common to all the stems is the monosyllabic element
which is called the Root (see~\xref[footnote]{203}), and in such cases
we speak of the Root or the Root-Syllable rather than of the
Verb-Stem.  The Root occasionally varies in form, owing partly to
regular phonetic change, partly to an original variation.  Thus the
root of \latin{canō} is \root{can}, which has become \root{cin} in the
Perfect \latin{cecinī} (\xref[1]{42}); the root of \latin{tegō} is
\root{teg}, but this had another form \root{tēg}, from which are
formed Perf. \latin{tēxī}, Partic.\ \latin{tēctus}~(\xref{46}).

\end{note}

\begin{enumA}

\item

\emph{The Present Stem}, or stem of the Present System, which consists
of:
\begin{enum1}

\item
The Present, Imperfect, and Future of all Moods and Voices in which
they occur.

\item
The Present Infinitive of both Voices.

\item
The Present Active Participle.

\item
The Future Passive Participle and the Gerund.

\end{enum1}

\item
\emph{The Perfect Stem}, or stem of the Perfect System (Active), which
consists of:
\begin{enum1}

\item
The Perfect, Past Perfect, and Future Perfect,—of the Active Voice.

\item
The Perfect Infinitive of the Active Voice.

\end{enum1}

\item
\emph{The Participial Stem}, or stem of:
\begin{enum1}

\item
The Perfect Passive Participle, from which is formed the Perfect
Passive System, consisting of:

\item
The Perfect, Past Perfect, Future Perfect, and the Perfect
Infinitive,—\allowbreak of the Passive Voice.

From the same stem can also be determined, nearly always:

\item
The Supine.

\item
The Future Active Participle, from which is formed:

\item
The Future Infinitive,—Active and Passive.

\end{enum1}

\end{enumA}

\pagebreak

\headingB{The Conjugations}

\contentsentry{B}{Types of Conjugation}

\section

There are four regular types of Verb Inflection, known as the \emph{Four
Conjugations} and distinguished by the ending of the Present Stem.  The
Present Infinitive is chosen as a convenient characteristic of each
Conjugation.
\begin{Tabular}{>{\scshape}c>{\scshape}r@{\qquad}c@{\qquad}l}

& & \emph{Present Stem ends in:} & \cc{1}{\emph{Infinitive:}} \\

Conjugation & i     & \phone{ā} & \ending{-āre} \\
\ditto      & ii    & \phone{ē} & \ending{-ēre} \\

\ditto      & iii
            & \phone{e}\rlap{\space or
  \phone{o}\footnotemark}\footnotetext{\label{ftn:76:}This variable
  vowel, \phone{e} or \phone{o}, which also occurs in other
  tense-stems, is known as the Thematic Vowel.  This term means really
  nothing more than Stem-Vowel, but has come to be applied to that
  particular stem-vowel which is, or was in the parent speech, the
  commonest in Verb-formation.  It is identical in form with the
  stem-vowel of Nouns of the Second Declension, which is commonly
  \phone{o} (hence the name \phone{o}-Stems) but is sometimes
  \phone{e} (e.g., in Latin, in the Vocative Singular, and in the
  variant form of the Ablative Singular which appears in the Adverbs
  in~\ending{-ē}; see \xref[1, note]{126}).

  Verb-formations which contain this variable vowel are called
  \emph{thematic}, while those in which the endings are added directly
  to the root are known as \emph{unthematic}.  Such are many of the
  forms of the Irregular Verbs~(\xref{170}).}
            & \ending{-ere} \\

\ditto      & iv    & \phone{ī} & \ending{-īre}

\end{Tabular}

\section

There are also some verbs the inflection of which does not conform to
any of the Four Conjugations.  Such are known as \emph{Irregular
  Verbs}.

\headingB{The Principal Parts}

\contentsentry{B}{Principal Parts}

\section

Certain forms of the verbs are known as the Principal Parts, because
they furnish the key to the inflection of any given verb, showing, as
they do, the Present Stem and thereby the Conjugation, and the Perfect
and Participial Stems.  These are:
\begin{enum1}

\item
\emph{The Present Indicative Active}, cited in the First Person
Singular.

\item
\emph{The Present Infinitive Active}.

\item
\emph{The Perfect Indicative Active}, cited in the First Person
Singular.

\item
\emph{The Perfect Passive Participle}, cited in the Nominative
Singular Neuter.\footnote{This is preferred to the Nominative Singular
  Masculine, because of the large number of Verbs in which the Perfect
  Passive Participle occurs only in the Neuter form (i.e.\ is used
  only impersonally), and also because of the advantage of citing a
  form which is identical with that of the Supine.  It is not
  essential for students, in learning the Principal Parts, to distinguish
  between Verbs which have the fully inflected Participle and those
  which have only the Neuter, and, again, those which have only the
  Supine.  The reason for abandoning the older method, of always
  giving the Supine as the fourth of the Principal Parts, is that the
  Perfect Passive Participle is vastly more common than the Supine,
  and that upon it, rather than upon the Supine, is based the Perfect
  Passive System.}

\end{enum1}

So for example:
\begin{Tabular*}{c@{\extracolsep{\fill}}ccc}

\textsc{Pres. Indic.}
& \textsc{Pres. Infin.}
& \textsc{Perf. Indic.}
& \textsc{Perf. Pass. Partic.}
\\

\latin{amō}, \english{love}
& \latin{amāre}
& \latin{amāvī}
& \latin{amātum}

\end{Tabular*}

\subsubsection

For verbs which lack the Perfect Passive Participle, the Supine, if
occurring, is cited; e.g.:
\begin{Tabular*}{c@{\extracolsep{\fill}}c@{\extracolsep{\fill}}c@{\extracolsep{\fill}}c}

\textsc{Pres. Indic.}
& \textsc{Pres. Infin.}
& \textsc{Perf. Indic.}
& \textsc{Supine\emend{5}{.}{}}
\\

\latin{maneō}, \english{remain}
& \latin{manēre}
& \latin{mānsī}
& \latin{mānsum}

\end{Tabular*}

\subsubsection

For verbs which lack both the Perfect Passive Participle and the
Supine, the Future Active Participle, if occurring, is cited; e.g.:
\begin{Tabular*}{c@{\extracolsep{\fill}}c@{\extracolsep{\fill}}c@{\extracolsep{\fill}}c}

\textsc{Pres. Indic.}
& \textsc{Pres. Infin.}
& \textsc{Perf. Indic.}
& \textsc{Fut. Act. Partic.}
\\

\latin{doleō}, \english{grieve}
& \latin{dolēre}
& \latin{doluī}
& \latin{dolitūrus}

\end{Tabular*}

\subsubsection

For verbs which occur only as Passives or Deponents, the form of the
Perfect Indicative answers for both the Perfect and Participial Stems;
e.g.:
\begin{Tabular*}{c@{\extracolsep{2em}}cc}

\textsc{Pres. Indic.}
& \textsc{Pres. Infin.}
& \textsc{Perf. Indic.}
\\

\latin{mīror}, \english{admire}
& \latin{mīrārī}
& \latin{mīrātus sum}

\end{Tabular*}

\headingB{The Personal Endings}

\contentsentry{B}{Personal Endings}

\section

The Personal Endings are:
\begin{Tabular}{r@{ }ll@{\qquad}r@{ }ll}

\cc{3}{\textsc{Active}}
& \cc{3}{\textsc{Passive}}
\\

& \textsc{singular} & \textsc{plural}
&& \textsc{singular} & \textsc{plural}
\\

1.  & \ending{-ō}, \ending{-m}
    & \ending{-mus}
& 1.
    & \ending{-r}
    & \ending{-mur}
\\

2.  & \ending{-s}
    & \ending{-t}
& 2.
    & \ending{-ris} or \ending{-re}
    & \ending{-minī}
\\

3.  & \ending{-t}
    & \ending{-nt}
& 3.
    & \ending{-tur}
    & \ending{-ntur}

\end{Tabular}

\begin{note}

In the Second Singular Passive, \ending{-re} is the usual ending in
early Latin, but yields more and more to \ending{-ris}, which
eventually becomes the normal ending.  In some authors, as Cicero and
Virgil, \ending{-ris} is more common in the Present Indicative, but
\ending{-re} elsewhere.

\end{note}

\subsubsection

The Perfect Indicative Active has different endings, namely:
\begin{Tabular}{r@{\enskip}ll}

\multicolumn{2}{l}{\textsc{singular}} & \textsc{plural} \\

1.  & \ending{-ī}   & \ending{-mus} \\
2.  & \ending{-stī} & \ending{-stis} \\
3.  & \ending{-t}   & \ending{-ērunt} or \ending{-ēre}

\end{Tabular}

\begin{note}

In the Third Plural, \suffix{-ērunt} is the usual ending, but
\suffix{-ēre} is also very common.  In poetry is also found
\suffix{-erunt} with short~\phone{e}.

\end{note}

\subsubsection

The endings of the Imperative are:
\begin{Tabular}{>{\itshape}lr@{ }ll@{\qquad}@{ }ll}

& \cc{3}{\textsc{Active}}
& \cc{2}{\textsc{Passive}}
\\

&
& \textsc{singular} & \textsc{plural}
& \textsc{singular} & \textsc{plural}
\\

Pres. &
2.  & \na
    & \ending{-te}
    & \ending{-re}
    & \ending{-minī}
\\

Fut. &
2.  & \ending{-tō}
    & \ending{-tōte}
    & \ending{-tor}
    & \na
\\

&
3.  & \ending{-tō}
    & \ending{-ntō}
    & \ending{-tor}
    & \ending{-ntor}

\end{Tabular}

\begin{note}

In early Latin there is a rare ending \ending{-minō}, used in place of
\ending{-tor} in a few Deponents; e.g.\ \latin{fruiminō}.

\end{note}

\headingC{The Union of the Endings with the Stem}

\section
\subsection

If the stem to which the endings are added ends in the thematic vowel,
originally \phone{e} or \phone{o} (see p.~\pageref{ftn:76:}, footnote),
this (1)~appears
as \phone{e} before \phone{r}, as in \latin{tege-ris}; (2)~unites with
the ending of the First Person Singular to form~\ending{-ō}, as in
\latin{tegō}; (3)~becomes \phone{u} before~\phone{nt} (\xref[1]{44};
\xref[5]{42}), as in \latin{tegunt}, \latin{teguntur}; (4)~becomes
\phone{i} before all other endings (\xref[2]{44}; \xref[2]{42}), as in
\latin{tegis}, \latin{tegit}, \latin{tegitur}, etc.

\subsection

If the stem to which the endings are added ends in a long vowel, this
vowel is shortened before the endings \ending{-m}, \ending{-t},
\ending{-nt}, and \ending{-r}; e.g.\ \latin{amat}, \latin{amant},
beside \latin{amās}, \latin{amāmus}, \latin{amātis}; \latin{monet},
\latin{monent}, beside \latin{monēs}, etc.; \latin{audit} beside
\latin{audīs}, etc.\ (but not \incorrect{audint}; \latin{audiunt} is
from \rec{audiont}, formed from a stem in \infix{-io-}; see
\xref[note]{169}); Pres.\ Subj.\ \latin{amem} beside \latin{amēs},
Pass.\ \latin{amer} beside \latin{amēris}.  See \xref[1, 2]{26}.
Before the ending \ending{-ō} of the First Singular the \phone{ā} of
the First Conjugation disappears by contraction, as in \latin{amō},
from \rec{amāō}, while in the Second and Fourth Conjugations we find
short \phone{e} and short \phone{i}, as in \latin{moneō},
\latin{audiō} (\xref[note]{167}; \xref[note]{169}).

\begin{note}

But before the ending \ending{-t} the original forms with the long
vowel are found in early Latin and in poetry; e.g.\ \latin{arāt},
\latin{vidēt}, etc. See~\xref[note]{26}.

\end{note}

\subsection

In the Perfect Indicative the endings beginning with a consonant are
preceded by a short~\phone{i}; e.g.\ \latin{amāvistī}, \latin{amāvit},
\latin{amāvimus}.

\begin{note}

In early Latin and in poetry there is also a Third Singular with
long~\phone{i} (probably formed after the analogy of the First
Singular); e.g.\ \latin{subiīt}.  The usual form with the short vowel
is not derived from this (by shortening before~\ending{-t}), but
represents a different formation.

\end{note}

\pagebreak

\headingB{Conjugation of Sum}

\contentsentry{B}{Conjugation of \latin{Sum}}

\section

\latin{Sum}, \english{be}, is one of the Irregular Verbs, but as an
auxiliary it enters into the inflection of the regular verbs, and is
therefore given first.

\begin{Tabular*}{@{\extracolsep{\fill}}*4{c}}

\cc{4}{\textbf{Principal Parts}}\\[\smallskipamount]

\textsc{pres.\ indic.}
& \textsc{infin.}
& \textsc{perf.\ indic.}
& \textsc{fut.\ partic.}
\\

\latin{sum}
& \latin{esse}
& \latin{fuī}
& \latin{futūrus}
\end{Tabular*}

\begin{Tabular}{r@{\enskip}l@{\qquad}l}

& \textbfsc{indicative}
& \textbfsc{subjunctive}

\endhead

& \cc{2}{\emph{Present}} \\
& \cc{2}{\textsc{singular}} \\

1.  & su\ending{m}, \english{I am}
    & si\ending{m}\footnote{Any single translation of the Subjunctive
  is likely to be misleading.  Accordingly none is given.  For the
  different meanings, see the Syntax.}
\\

2.  & es, \english{thou art}
    & s\ending{īs}
\\

3.  & es\ending{t}, \english{he \(she, it\) is}
    & s\ending{it}
\\[\smallskipamount]

& \cc{2}{\textsc{plural}} \\

1.  & su\ending{mus}, \english{we are}
    & s\ending{īmus}
\\

2.  & es\ending{tis}, \english{you are}
    & s\ending{ītis}
\\

3.  & s\ending{unt}, \english{they are}
    & s\ending{int}
\\[\smallskipamount]

& \cc{2}{\emph{Imperfect}} \\
& \cc{2}{\textsc{singular}} \\

1.  & er\ending{am}, \english{I was}
    & es\ending{sem}
\\

2.  & er\ending{ās}, \english{thou wast}
    & es\ending{sēs}
\\

3.  & er\ending{at}, \english{he was}
    & es\ending{set}
\\[\smallskipamount]

& \cc{2}{\textsc{plural}} \\

1.  & er\ending{āmus}, \english{we were}
    & es\ending{sēmus}
\\

2.  & er\ending{ātis}, \english{you were}
    & es\ending{sētis}
\\

3.  & er\ending{ant}, \english{they were}
    & es\ending{sent}
\\

& \cc{2}{\emph{Future}} \\
& \cc{2}{\textsc{singular}} \\

1.  & er\ending{ō}, \english{I shall be} \\
2.  & er\ending{is}, \english{thou wilt be}\\
3.  & er\ending{it}, \english{he will be}\\[\smallskipamount]

& \cc{2}{\textsc{plural}} \\

1.  & er\ending{imus}, \english{we shall be} \\
2.  & er\ending{itis}, \english{you will be} \\
3.  & er\ending{unt}, \english{they will be}\\[\medskipamount]

& \cc{2}{\emph{Perfect}} \\
& \cc{2}{\textsc{singular}} \\

1.  & fu\ending{ī}, \english{I have been, was}
    & fu\ending{erim}
\\

2.  & fu\ending{istī}, \english{thou hast been, wast}
    & fu\ending{erīs}
\\

3.  & fu\ending{it}, \english{he has been, was}
    & fu\ending{erit}
\\

& \cc{2}{\textsc{plural}} \\

1.  & fu\ending{imus}, \english{we have been, were}
    & fu\ending{erīmus}
\\

2.  & fu\ending{istis}, \english{you have been, were}
    & fu\ending{erītis}
\\

3.  & fu\ending{ērunt} or \ending{-ēre}, \english{they have been, were}
    & fu\ending{erint}
\\[\smallskipamount]

& \cc{2}{\emph{Past Perfect}} \\
& \cc{2}{\textsc{singular}} \\

1.  & fu\ending{eram}, \english{I had been}
    & fu\ending{issem}
\\

2.  & fu\ending{erās}, \english{thou hadst been}
    & fu\ending{issēs}
\\

3.  & fu\ending{erat}, \english{he had been}
    & fu\ending{isset}
\\

& \cc{2}{\textsc{plural}} \\

1.  & fu\ending{erāmus}, \english{we had been}
    & fu\ending{issēmus}
\\

2.  & fu\ending{erātis}, \english{you had been}
    & fu\ending{issētis}
\\

3.  & fu\ending{erant}, \english{they had been}
    & fu\ending{issent}
\\[\smallskipamount]

& \cc{2}{\emph{Future Perfect}} \\
& \cc{2}{\textsc{singular}} \\

1.  & fu\ending{erō},  \english{I shall have been} \\
2.  & fu\ending{eris}, \english{thou wilt have been} \\
3.  & fu\ending{erit}, \english{he will have been} \\[\medskipamount]

& \cc{2}{\textsc{plural}} \\

1.  & fu\ending{erimus}, \english{we shall have been} \\
2.  & fu\ending{eritis}, \english{you will have been} \\
3.  & fu\ending{erint},  \english{they will have been}

\end{Tabular}
\begin{Tabular}{>{\itshape}l@{\enskip}r@{\enskip}l@{\qquad}r@{\enskip}l}

& \cc{4}{\textbfsc{imperative}} \\

& \M{2}{l}{\textsc{singular}} & \M{2}{l}{\textsc{plural}} \\

Pres.   & 2.    & es, \english{be thou}
        & 2.    & es\ending{te}, \english{be ye}
\\

Fut.    & 2.    & es\ending{tō}, \english{thou shalt be}
        & 2.    & es\ending{tōte}, \english{you shall be}
\\

        & 3.    & es\ending{tō}, \english{he shall be}
        & 3.    & s\ending{untō}, \english{they shall be}
\\[\medskipamount]

& \M{2}{l}{\textbfsc{infinitive}}
& \M{2}{l}{\textbfsc{participle}} \\

Pres.   & \M{2}{l}{es\ending{se}, \english{to be}}
        & \emph{Fut.} & fu\ending{tūrus}, \english{about to be}\\

Perf.   & \M{2}{l}{fu\ending{isse}, \english{to have been}} \\

Fut.    & \M{2}{l}{fu\ending{tūrus esse}, \english{to be about to be}}

\end{Tabular}

\enlargethispage{\baselineskip}

\section

The following forms are sometimes found in place of those given in the
paradigm:
\begin{enum1}

\item
Imperfect Subjunctive \latin{forem}, \latin{forēs}, \latin{foret},
\latin{forent}.

\item
Future infinitive \latin{fore}.

\begin{minor}

\item
Present Subjunctive (in early Latin) \latin{siem}, \latin{siēs},
\latin{siet}, \latin{sient}; also \latin{fuam}, \latin{fuās},
\latin{fuat}, \latin{fuant}.

\item
For early Latin \latin{es(s)} in the Present Indicative, see
\xref[3]{30}.

\item
For early Latin \latin{fūī} in the Perfect Indicative,
see~\xref[7]{21}.

\begin{note}

The various forms of the verb \latin{sum} are made from two different roots,
one, \latin{es}, related to English \english{is}, the other,
\latin{f\u{ū}}, related to English \english{be}.

\end{note}

\end{minor}

\end{enum1}

\chapter{First Conjugation}

\contentsentry{B}{The Four Regular Conjugations}

\section
\subtitle{\latin{amō}, \english{love}}

\begin{Tabular*}{@{\extracolsep{\fill}}*4{c}}

\cc{4}{\textbf{Principal Parts}}\\[\smallskipamount]

\textsc{pres.\ indic.}
& \textsc{pres.\ infin.}
& \textsc{perf.\ indic.}
& \textsc{perf.\ pass.}
\\

\latin{amō}
& \latin{amāre}
& \latin{amāvī}
& \latin{amātum}
\end{Tabular*}

\begin{Tabular*}{l@{\extracolsep{\fill}}l}

\cc{2}{\textsc{Active Voice}} \\

\textbfsc{indicative} & \textbfsc{subjunctive}

\endhead

\cc{2}{\emph{Present}} \\

am\ending{ō}, \english{I love} & am\ending{em} \\
am\ending{ās} & am\ending{ēs} \\
am\ending{at} & am\ending{et} \\[\smallskipamount]

am\ending{āmus} & am\ending{ēmus} \\
am\ending{ātis} & am\ending{ētis} \\
am\ending{ant}  & am\ending{ent} \\[\medskipamount]

\cc{2}{\emph{Imperfect}} \\

am\ending{ābam}, \english{I was loving} & am\ending{ārem} \\
am\ending{ābās} & am\ending{ārēs} \\
am\ending{ābat} & am\ending{āret} \\[\smallskipamount]

am\ending{ābāmus} & am\ending{ārēmus} \\
am\ending{ābātis} & am\ending{ārētis} \\
am\ending{ābant} & am\ending{ārent} \\[\medskipamount]

\cc{2}{\emph{Future}} \\

am\ending{ābō}, \english{I shall love} \\
am\ending{ābis} \\
am\ending{ābit} \\[\smallskipamount]

am\ending{ābimus} \\
am\ending{ābitis} \\
am\ending{ābunt} \\[\medskipamount]

\cc{2}{\emph{Perfect}} \\

am\ending{āvī}, \english{I have loved, loved} & am\ending{āverim} \\
am\ending{āvistī} & am\ending{āverīs} \\
am\ending{āvit}   & am\ending{āverit} \\[\smallskipamount]

am\ending{āvimus} & am\ending{āverīmus} \\
am\ending{āvistis} & am\ending{āverītis} \\
am\ending{āvērunt} or \ending{-ēre} & am\ending{āverint} \\[\medskipamount]

\pagebreak

\cc{2}{\emph{Past Perfect}} \\

am\ending{āveram}, \english{I had loved} & am\ending{āvissem} \\
am\ending{āverās}   & am\ending{āvissēs} \\
am\ending{āverat}   & am\ending{āvisset} \\[\smallskipamount]

am\ending{āverāmus} & am\ending{āvissēmus} \\
am\ending{āverātis} & am\ending{āvissētis} \\
am\ending{āverant}  & am\ending{āvissent} \\[\medskipamount]

\cc{2}{\emph{Future Perfect}} \\

am\ending{āverō}, \english{I shall have loved} \\
am\ending{āveris}   \\
am\ending{āverit}   \\[\smallskipamount]

am\ending{āverimus}  \\
am\ending{āveritis}  \\
am\ending{āverint}   \\[\medskipamount]

\end{Tabular*}

\negbigskip

\begin{Tabular}{>{\itshape}lll}

& \cc{2}{\textbfsc{imperative}} \\
& \textsc{singular} & \textsc{plural} \\

Pres.   & am\ending{ā}, \english{love thou}
        & am\ending{āte} \\
Fut.    & am\ending{ātō}, \english{thou shalt love}
        & am\ending{ātōte} \\
        & am\ending{ātō}, \english{he shall love}
        & am\ending{antō}

\end{Tabular}

\begin{Tabular}{>{\itshape}l@{\enskip}l@{\qquad}>{\itshape}l@{\enskip}l}

&  \cc{1}{\textbfsc{infinitive}}
&& \cc{1}{\textbfsc{participle}} \\

  Pres.   & am\ending{āre}, \english{to love}
& Pres.   & am\ending{āns}, \english{loving} \\
  Perf.   & am\ending{āvisse}, \english{to have loved}
& Fut.    & am\ending{ātūrus}, \english{about to love} \\
  Fut.    & am\ending{ātūrus esse}, \english{to be about to love}

\end{Tabular}

\begin{Tabular}{>{\itshape}ll@{\qquad}l}

& \textbfsc{gerund} & \textbfsc{supine} \\

Gen.    & am\ending{andī}, \english{of loving} \\
Dat.    & am\ending{andō}, \english{for loving} \\
Acc.    & am\ending{andum}, \english{loving}
        & am\ending{ātum}, \english{to love} \\
Abl.    & am\ending{andō}, \english{by loving}
        & am\ending{ātū}, \english{to love}

\end{Tabular}

\smallskip

\begin{Tabular*}{l@{\extracolsep{\fill}}l}

\cc{2}{\textsc{Passive Voice}} \\

  \textbfsc{indicative}
& \textbfsc{subjunctive}

\endhead

\cc{2}{\emph{Present}} \\

am\ending{or}, \english{I am loved} & am\ending{er} \\
am\ending{āris} or \ending{-re} & am\ending{ēris} or \ending{-re} \\
am\ending{ātur} & am\ending{ētur} \\[\smallskipamount]

am\ending{āmur}  & am\ending{ēmur} \\
am\ending{āminī} & am\ending{ēminī} \\
am\ending{antur} & am\ending{entur} \\[\medskipamount]

% \pagebreak

\cc{2}{\emph{Imperfect}} \\

am\ending{ābar}, \english{I was loved} & am\ending{ārer} \\
am\ending{ābāris} or \ending{-re} & am\ending{ārēris} or \ending{-re} \\
am\ending{ābātur} & am\ending{ārētur} \\[\smallskipamount]

am\ending{ābāmur}  & am\ending{ārēmur} \\
am\ending{ābāminī} & am\ending{ārēminī} \\
am\ending{ābantur} & am\ending{ārentur} \\[\medskipamount]

\cc{2}{\emph{Future}} \\

am\ending{ābor}, \english{I shall be loved}\\
am\ending{āberis} or \ending{-re}  \\
am\ending{ābitur}  \\[\smallskipamount]

am\ending{ābimur}  \\
am\ending{ābiminī} \\
am\ending{ābuntur} \\[\medskipamount]

\cc{2}{\emph{Perfect}} \\

\remember\1{\ending{ātus}}

am\ending{ātus sum}, \english{I have been \(was\) loved}
& am\ending{ātus sim} \\

am\ending{ātus es}
& am\ending{ātus sīs}\\

am\ending{ātus est}
& am\ending{ātus sit} \\[\smallskipamount]

am\ending{\1{ātī} sumus}
& am\ending{\1{ātī} sīmus} \\

am\ending{\1{ātī} estis}
& am\ending{\1{ātī} sītis} \\

am\ending{\1{ātī} sunt}
& am\ending{\1{ātī} sint} \\[\medskipamount]

\cc{2}{\emph{Past Perfect}} \\

am\ending{ātus eram}, \english{I had been loved}
& am\ending{ātus essem} \\

am\ending{ātus erās}
& am\ending{ātus essēs}\\

am\ending{ātus erat}
& am\ending{ātus esset} \\[\smallskipamount]

am\ending{\1{ātī} erāmus}
& am\ending{\1{ātī} essēmus} \\

am\ending{\1{ātī} erātis}
& am\ending{\1{ātī} essētis} \\

am\ending{\1{ātī} erant}
& am\ending{\1{ātī} essent} \\[\medskipamount]

\cc{2}{\emph{Future Perfect}} \\

am\ending{ātus erō}, \english{I shall have been loved} \\

am\ending{ātus eris} \\

am\ending{ātus erit} \\[\smallskipamount]

am\ending{\1{ātī} erimus} \\

am\ending{\1{ātī} eritis} \\

am\ending{\1{ātī} erunt}

\end{Tabular*}

\pagebreak

\begin{Tabular}{>{\itshape}lll}

& \cc{2}{\textbfsc{imperative}} \\
& \textsc{singular} & \textsc{plural} \\

Pres.   & am\ending{āre}, \english{be loved}
        & am\ending{āminī} \\
Fut.    & am\ending{ātor}, \english{thou shalt be loved}
        & am\ending{antor} \\
        & am\ending{ātor}, \english{he shall be loved}

\end{Tabular}

\begin{Tabular}{>{\itshape}ll@{\qquad}>{\itshape}ll}

&  \cc{1}{\textbfsc{infinitive}}
&& \cc{1}{\textbfsc{participle}} \\

  Pres.   & am\ending{ārī}, \english{to be loved}
& Perf.   & am\ending{ātus}, \english{loved} \\

  Perf.   & am\ending{ātus esse}, \english{to have been loved}
& Fut.    & am\ending{andus}, \english{to be loved,\logical{ worthy of love}}\\

\visual{&&         & \quad\english{worthy of love}\\}%

Fut.    & am\ending{ātum īrī}, \english{to be about to be loved}

\end{Tabular}

\chapter{Second Conjugation}

\section
\subtitle{\latin{moneō}, \english{advise}}

\begin{Tabular*}{@{\extracolsep{\fill}}*4{c}}

\cc{4}{\textbf{Principal Parts}}\\[\smallskipamount]

\latin{moneō}
& \latin{monēre}
& \latin{monuī}
& \latin{monitum}
\end{Tabular*}

\begin{paradigm}

  \cc{2}{\textsc{Active}}
& \cc{2}{\textsc{Passive}} \\

  \textbfsc{indicative}
& \textbfsc{subjunctive}
& \textbfsc{indicative}
& \textbfsc{subjunctive}\\[\smallskipamount]

\cc{4}{\emph{Present}} \\[\smallskipamount]

  mon\ending{eō}
& mon\ending{eam}
& mon\ending{eor}
& mon\ending{ear}
\\

  mon\ending{ēs}
& mon\ending{eās}
& mon\ending{ēris}, \ending{-re}
& mon\ending{eāris}, \ending{-re}
\\

  mon\ending{et}
& mon\ending{eat}
& mon\ending{ētur}
& mon\ending{eātur}
\\[\smallskipamount]

  mon\ending{ēmus}
& mon\ending{eāmus}
& mon\ending{ēmur}
& mon\ending{eāmur}
\\

  mon\ending{ētis}
& mon\ending{eātis}
& mon\ending{ēminī}
& mon\ending{eāminī}
\\

  mon\ending{ent}
& mon\ending{eant}
& mon\ending{entur}
& mon\ending{eantur}
\\[\smallskipamount]

\cc{4}{\emph{Imperfect}} \\[\smallskipamount]

  mon\ending{ēbam}
& mon\ending{ērem}
& mon\ending{ēbār}
& mon\ending{ērer}
\\

  mon\ending{ēbās}
& mon\ending{ērēs}
& mon\ending{ēbāris}, \ending{-re}
& mon\ending{ērēris}, \ending{-re}
\\

  mon\ending{ēbat}
& mon\ending{ēret}
& mon\ending{ēbātur}
& mon\ending{ērētur}
\\[\smallskipamount]

  mon\ending{ēbāmus}
& mon\ending{ērēmus}
& mon\ending{ēbāmur}
& mon\ending{ērēmur}
\\

  mon\ending{ēbātis}
& mon\ending{ērētis}
& mon\ending{ēbāminī}
& mon\ending{ērēminī}
\\

  mon\ending{ēbant}
& mon\ending{ērent}
& mon\ending{ēbantur}
& mon\ending{ērentur}
\\[\smallskipamount]

\cc{4}{\emph{Future}} \\[\smallskipamount]

  mon\ending{ēbo}
&
& mon\ending{ēbor}
\\

  mon\ending{ēbis}
&
& mon\ending{ēberis}, \ending{-re}
\\

  mon\ending{ēbit}
&
& mon\ending{ēbitur}
\\[\smallskipamount]

  mon\ending{ēbimus}
&
& mon\ending{ēbimur}
\\

  mon\ending{ēbitis}
&
& mon\ending{ēbiminī}
\\

  mon\ending{ēbunt}
&
& mon\ending{ēbuntur}
\\[\smallskipamount]

\cc{4}{\emph{Perfect}} \\[\smallskipamount]

\remember\1{\ending{itus}}

  mon\ending{uī}
& mon\ending{uerim}
& mon\ending{itus sum}
& mon\ending{itus sim}
\\

  mon\ending{uistī}
& mon\ending{uerīs}
& mon\ending{itus es}
& mon\ending{itus sīs}
\\

  mon\ending{uit}
& mon\ending{uerit}
& mon\ending{itus est}
& mon\ending{itus sit}
\\[\smallskipamount]

  mon\ending{uimus}
& mon\ending{uerīmus}
& mon\ending{\1{itī} sumus}
& mon\ending{\1{itī} sīmus}
\\

  monu\ending{istis}
& monu\ending{erītis}
& mon\ending{\1{itī} estis}
& mon\ending{\1{itī} sītis}
\\

  mon\ending{uērunt}, \ending{-ēre}
& mon\ending{uerint}
& mon\ending{\1{itī} sunt}
& mon\ending{\1{itī} sint}
\\[\smallskipamount]

\cc{4}{\emph{Past Perfect}} \\[\smallskipamount]

  mon\ending{ueram}
& mon\ending{uissem}
& mon\ending{itus eram}
& mon\ending{itus essem}
\\

  mon\ending{uerās}
& mon\ending{uissēs}
& mon\ending{itus erās}
& mon\ending{itus essēs}
\\

  mon\ending{erat}
& mon\ending{uisset}
& mon\ending{itus erat}
& mon\ending{itus esset}
\\[\smallskipamount]

  mon\ending{uerāmus}
& mon\ending{uissēmus}
& mon\ending{\1{itī} erāmus}
& mon\ending{\1{itī} essēmus}
\\

  mon\ending{uerātis}
& mon\ending{uissētis}
& mon\ending{\1{itī} erātis}
& mon\ending{\1{itī} essētis}
\\

  mon\ending{uerant}
& mon\ending{uissent}
& mon\ending{\1{itī} erant}
& mon\ending{\1{itī} essent}
\\[\smallskipamount]

\cc{4}{\emph{Future Perfect}} \\[\smallskipamount]

  mon\ending{uerō}
&
& mon\ending{itus erō}
\\

  mon\ending{ueris}
&
& mon\ending{itus eris}
\\

  mon\ending{uerit}
&
& mon\ending{itus erit}
\\[\smallskipamount]

  mon\ending{uerimus}
&
& mon\ending{\1{itī} erimus}
\\

  mon\ending{ueritis}
&
& mon\ending{\1{itī} eritis}
\\

  mon\ending{uerint}
&
& mon\ending{\1{itī} erunt}

\end{paradigm}

\begin{imperative}

Pres.
& mon\ending{ē}
& mon\ending{ēte}
& mon\ending{ēre}
& mon\ending{ēminī}
\\

Fut.
& mon\ending{ētō}
& mon\ending{ētōte}
& mon\ending{ētor}
\\

& mon\ending{ētō}
& mon\ending{entō}
& mon\ending{ētor}
& mon\ending{entor}

\end{imperative}

\begin{nonfinite}

\cc{4}{\textbfsc{infinitive}} \\

Pres.
& mon\ending{ēre}
&
& mon\ending{ērī}
\\

Perf.
& mon\ending{uisse}
&
& mon\ending{itus esse}
\\

Fut.
& mon\ending{itūrus esse}
&
& mon\ending{itum īrī}
\\[\medskipamount]

\cc{4}{\textbfsc{participle}} \\

Pres.
& mon\ending{ēns}
& Perf.
& mon\ending{itus}
\\

Fut.
& mon\ending{itūrus}
& Fut.
& mon\ending{endus}
\\[\medskipamount]

& \textbfsc{gerund} && \textbfsc{supine} \\

Gen. & mon\ending{endī} \\
Dat. & mon\ending{endō} \\
Acc. & mon\ending{endum} && mon\ending{itum} \\
Abl. & mon\ending{endō}  && mon\ending{itū}

\end{nonfinite}

\chapter{Third Conjugation}

\section
\subtitle{\latin{tegō}, \english{cover}}

\begin{Tabular*}{@{\extracolsep{\fill}}*4{c}}

\cc{4}{\textbf{Principal Parts}}\\[\smallskipamount]

  \latin{tegō}
& \latin{tegere}
& \latin{tēxī}
& \latin{tēctum}
\end{Tabular*}

\begin{paradigm}

  \cc{2}{\textsc{Active}}
& \cc{2}{\textsc{Passive}} \\

  \textbfsc{indicative}
& \textbfsc{subjunctive}
& \textbfsc{indicative}
& \textbfsc{subjunctive}\\[\smallskipamount]

\cc{4}{\emph{Present}} \\[\smallskipamount]

  teg\ending{ō}
& teg\ending{am}
& teg\ending{or}
& teg\ending{ar}
\\

  teg\ending{is}
& teg\ending{ās}
& teg\ending{eris}, \ending{-re}
& teg\ending{āris}, \ending{-re}
\\

  teg\ending{it}
& teg\ending{at}
& teg\ending{itur}
& teg\ending{ātur}
\\[\smallskipamount]

  teg\ending{imus}
& teg\ending{āmus}
& teg\ending{imur}
& teg\ending{āmur}
\\

  teg\ending{itis}
& teg\ending{ātis}
& teg\ending{iminī}
& teg\ending{āminī}
\\

  teg\ending{unt}
& teg\ending{ant}
& teg\ending{untur}
& teg\ending{antur}
\\[\smallskipamount]

\cc{4}{\emph{Imperfect}} \\[\smallskipamount]

  teg\ending{ēbam}
& teg\ending{erem}
& teg\ending{ēbār}
& teg\ending{erer}
\\

  teg\ending{ēbās}
& teg\ending{erēs}
& teg\ending{ēbāris}, \ending{-re}
& teg\ending{erēris}, \ending{-re}
\\

  teg\ending{ēbat}
& teg\ending{eret}
& teg\ending{ēbātur}
& teg\ending{erētur}
\\[\smallskipamount]

  teg\ending{ēbāmus}
& teg\ending{erēmus}
& teg\ending{ēbāmur}
& teg\ending{erēmur}
\\

  teg\ending{ēbātis}
& teg\ending{erētis}
& teg\ending{ēbāminī}
& teg\ending{erēminī}
\\

  teg\ending{ēbant}
& teg\ending{erent}
& teg\ending{ēbantur}
& teg\ending{erentur}
\\[\smallskipamount]

\cc{4}{\emph{Future}} \\[\smallskipamount]

  teg\ending{am}
&
& teg\ending{ar}
\\

  teg\ending{ēs}
&
& teg\ending{ēris}, \ending{-re}
\\

  teg\ending{et}
&
& teg\ending{ētur}
\\[\smallskipamount]

  teg\ending{ēmus}
&
& teg\ending{ēmur}
\\

  teg\ending{ētis}
&
& teg\ending{ēminī}
\\

  teg\ending{ent}
&
& teg\ending{entur}
\\[\smallskipamount]

\cc{4}{\emph{Perfect}} \\[\smallskipamount]

\remember\1{\ending{tus}}

  tēx\ending{ī}
& tēx\ending{erim}
& tēc\ending{tus sum}
& tēc\ending{tus sim}
\\

  tēx\ending{istī}
& tēx\ending{erīs}
& tēc\ending{tus es}
& tēc\ending{tus sīs}
\\

  tēx\ending{it}
& tēx\ending{erit}
& tēc\ending{tus est}
& tēc\ending{tus sit}
\\[\smallskipamount]

  tēx\ending{imus}
& tēx\ending{erīmus}
& tēc\ending{\1{tī} sumus}
& tēc\ending{\1{tī} sīmus}
\\

  tēx\ending{istis}
& tēx\ending{erītis}
& tēc\ending{\1{tī} estis}
& tēc\ending{\1{tī} sītis}
\\

  tēx\ending{ērunt}, \ending{-ēre}
& tēx\ending{erint}
& tēc\ending{\1{tī} sunt}
& tēc\ending{\1{tī} sint}
\\[\smallskipamount]

\pagebreak

\cc{4}{\emph{Past Perfect}} \\[\smallskipamount]

  tēx\ending{eram}
& tēx\ending{issem}
& tēc\ending{tus eram}
& tēc\ending{tus essem}
\\

  tēx\ending{erās}
& tēx\ending{issēs}
& tēc\ending{tus erās}
& tēc\ending{tus essēs}
\\

  tēx\ending{erat}
& tēx\ending{isset}
& tēc\ending{tus erat}
& tēc\ending{tus esset}
\\[\smallskipamount]

  tēx\ending{erāmus}
& tēx\ending{issēmus}
& tēc\ending{\1{tī} erāmus}
& tēc\ending{\1{tī} essēmus}
\\

  tēx\ending{erātis}
& tēx\ending{issētis}
& tēc\ending{\1{tī} erātis}
& tēc\ending{\1{tī} essētis}
\\

  tēx\ending{erant}
& tēx\ending{issent}
& tēc\ending{\1{tī} erant}
& tēc\ending{\1{tī} essent}
\\

\cc{4}{\emph{Future Perfect}} \\[\smallskipamount]

  tēx\ending{erō}
&
& tēc\ending{tus erō}
\\

  tēx\ending{eris}
&
& tēc\ending{tus eris}
\\

  tēx\ending{erit}
&
& tēc\ending{tus erit}
\\[\smallskipamount]

  tēx\ending{erimus}
&
& tēc\ending{\1{tī} erimus}
\\

  tēx\ending{eritis}
&
& tēc\ending{\1{tī} eritis}
\\

  tēx\ending{erint}
&
& tēc\ending{\1{tī} erunt}

\end{paradigm}

\begin{imperative}

Pres.
& teg\ending{e}
& teg\ending{ite}
& teg\ending{ere}
& teg\ending{iminī}
\\

Fut.
& teg\ending{itō}
& teg\ending{itōte}
& teg\ending{itor}
\\

& teg\ending{itō}
& teg\ending{untō}
& teg\ending{itor}
& teg\ending{untor}

\end{imperative}

\begin{nonfinite}

\cc{4}{\textbfsc{infinitive}} \\

Pres.
& teg\ending{ere}
&
& teg\ending{ī}
\\

Perf.
& tēx\ending{isse}
&
& tēc\ending{tus esse}
\\

Fut.
& tēc\ending{tūrus esse}
&
& tēc\ending{tum īrī}
\\[\medskipamount]

\cc{4}{\textbfsc{participle}} \\

Pres.
& teg\ending{ēns}
& Perf.
& tēc\ending{tus}
\\

Fut.
& tēc\ending{tūrus}
& Fut.
& teg\ending{endus}
\\[\medskipamount]

& \textbfsc{gerund} && \textbfsc{supine} \\

Gen. & teg\ending{endī} \\
Dat. & teg\ending{endō} \\
Acc. & teg\ending{endum} && tēc\ending{tum} \\
Abl. & teg\ending{endō}  && tēc\ending{tū}

\end{nonfinite}

\headingC{Verbs in \suffix{-iō} of the Third Conjugation}

\negbigskip

\section

Verbs in \suffix{-iō} of the Third Conjugation have in the Present
System many forms identical with those of the Fourth Conjugation,
namely, all those in which \phone{i} is followed by a vowel.
\pagebreak
\begin{Tabular*}{@{\extracolsep{\fill}}*4{c}}

\cc{4}{\latin{capiō}, \english{take}}\\[\smallskipamount]

\cc{4}{\textbf{Principal Parts}}\\[\smallskipamount]

  \latin{capiō}
& \latin{capere}
& \latin{cēpī}
& \latin{captum}
\end{Tabular*}

\begin{paradigm}

  \cc{2}{\textsc{Active}}
& \cc{2}{\textsc{Passive}} \\

  \textbfsc{indicative}
& \textbfsc{subjunctive}
& \textbfsc{indicative}
& \textbfsc{subjunctive}\\[\smallskipamount]

\cc{4}{\emph{Present}} \\[\smallskipamount]

  cap\ending{iō}
& cap\ending{iam}
& cap\ending{ior}
& cap\ending{iar}
\\

  cap\ending{is}
& cap\ending{iās}
& cap\ending{eris}, \ending{-re}
& cap\ending{iāris}, \ending{-re}
\\

  cap\ending{it}
& cap\ending{iat}
& cap\ending{itur}
& cap\ending{iātur}
\\[\smallskipamount]

  cap\ending{imus}
& cap\ending{iāmus}
& cap\ending{imur}
& cap\ending{iāmur}
\\

  cap\ending{itis}
& cap\ending{iātis}
& cap\ending{iminī}
& cap\ending{iāminī}
\\

  cap\ending{iunt}
& cap\ending{iant}
& cap\ending{iuntur}
& cap\ending{iantur}
\\[\smallskipamount]

\cc{4}{\emph{Imperfect}} \\[\smallskipamount]

  cap\ending{iēbam}\footnote{That is, \latin{capiēbam},
    \latin{capiēbās}, \latin{capiēbat}, etc.  So elsewhere.}
& cap\ending{erem}
& cap\ending{iēbar}
& cap\ending{erer}
\\[\smallskipamount]

\cc{4}{\emph{Future}} \\[\smallskipamount]

  cap\ending{iam}
&
& cap\ending{iar}
\\[\smallskipamount]

\cc{4}{\emph{Perfect}} \\[\smallskipamount]

\remember\1{\ending{tus}}

  cēp\ending{ī}
& cēp\ending{erim}
& cap\ending{tus sum}
& cap\ending{tus sim}
\\[\smallskipamount]

\cc{4}{\emph{Past Perfect}} \\[\smallskipamount]

  cēp\ending{eram}
& cēp\ending{issem}
& cap\ending{tus eram}
& cap\ending{tus essem}
\\[\smallskipamount]

\cc{4}{\emph{Future Perfect}} \\[\smallskipamount]

  cēp\ending{erō}
&
& cap\ending{tus erō}

\end{paradigm}

\begin{imperative}

Pres.
& cap\ending{e}
& cap\ending{ite}
& cap\ending{ere}
& cap\ending{iminī}
\\

Fut.
& cap\ending{itō}
& cap\ending{itōte}
& cap\ending{itor}
\\

& cap\ending{itō}
& cap\ending{iuntō}
& cap\ending{itor}
& cap\ending{iuntor}

\end{imperative}

\begin{nonfinite}

\cc{4}{\textbfsc{infinitive}} \\

Pres.
& cap\ending{ere}
&
& cap\ending{ī}
\\

Perf.
& cēp\ending{isse}
&
& cap\ending{tus esse}
\\

Fut.
& cap\ending{tūrus esse}
&
& cap\ending{tum īrī}
\\[\medskipamount]

\cc{4}{\textbfsc{participle}} \\

Pres.
& cap\ending{iēns}
& Perf.
& cap\ending{tus}
\\

Fut.
& cap\ending{tūrus}
& Fut.
& cap\ending{iendus}
\\[\medskipamount]

\pagebreak

& \textbfsc{gerund} && \textbfsc{supine} \\

Gen. & cap\ending{iendī} \\
Dat. & cap\ending{iendō} \\
Acc. & cap\ending{iendum} && cap\ending{tum} \\
Abl. & cap\ending{iendō}  && cap\ending{tū}

\end{nonfinite}

\chapter{Fourth Conjugation}

\section
\subtitle{\latin{audiō}, \english{hear}}

\begin{Tabular*}{@{\extracolsep{\fill}}*4{c}}

\cc{4}{\textbf{Principal Parts}}\\[\smallskipamount]

  \latin{audiō}
& \latin{audīre}
& \latin{audīvī}
& \latin{audītum}
\end{Tabular*}

\begin{paradigm}

  \cc{2}{\textsc{Active}}
& \cc{2}{\textsc{Passive}} \\

  \textbfsc{indicative}
& \textbfsc{subjunctive}
& \textbfsc{indicative}
& \textbfsc{subjunctive}\\[\smallskipamount]

\cc{4}{\emph{Present}} \\[\smallskipamount]

  aud\ending{iō}
& aud\ending{iam}
& aud\ending{ior}
& aud\ending{iar}
\\

  aud\ending{īs}
& aud\ending{iās}
& aud\ending{īris}, \ending{-re}
& aud\ending{iāris}, \ending{-re}
\\

  aud\ending{it}
& aud\ending{iat}
& aud\ending{ītur}
& aud\ending{iātur}
\\[\smallskipamount]

  aud\ending{īmus}
& aud\ending{iāmus}
& aud\ending{īmur}
& aud\ending{iāmur}
\\

  aud\ending{ītis}
& aud\ending{iātis}
& aud\ending{īminī}
& aud\ending{iāminī}
\\

  aud\ending{iunt}
& aud\ending{iant}
& aud\ending{iuntur}
& aud\ending{iantur}
\\[\smallskipamount]

\cc{4}{\emph{Imperfect}} \\[\smallskipamount]

  aud\ending{iēbam}
& aud\ending{īrem}
& aud\ending{iēbār}
& aud\ending{īrer}
\\

  aud\ending{iēbās}
& aud\ending{īrēs}
& aud\ending{iēbāris}, \ending{-re}
& aud\ending{īrēris}, \ending{-re}
\\

  aud\ending{iēbat}
& aud\ending{īret}
& aud\ending{iēbātur}
& aud\ending{īrētur}
\\[\smallskipamount]

  aud\ending{iēbāmus}
& aud\ending{īrēmus}
& aud\ending{iēbāmur}
& aud\ending{īrēmur}
\\

  aud\ending{iēbātis}
& aud\ending{īrētis}
& aud\ending{iēbāminī}
& aud\ending{īrēminī}
\\

  aud\ending{iēbant}
& aud\ending{īrent}
& aud\ending{iēbantur}
& aud\ending{īrentur}
\\[\smallskipamount]

\cc{4}{\emph{Future}} \\[\smallskipamount]

  aud\ending{iam}
&
& aud\ending{iar}
\\

  aud\ending{iēs}
&
& aud\ending{iēris}, \ending{-re}
\\

  aud\ending{iet}
&
& aud\ending{iētur}
\\[\smallskipamount]

  aud\ending{iēmus}
&
& aud\ending{iēmur}
\\

  aud\ending{iētis}
&
& aud\ending{iēminī}
\\

  aud\ending{ient}
&
& aud\ending{ientur}
\\[\smallskipamount]

\pagebreak

\cc{4}{\emph{Perfect}} \\[\smallskipamount]

\remember\1{\ending{ītus}}

  aud\ending{īvī}
& aud\ending{īverim}
& aud\latin{ītus sum}
& aud\latin{ītus sim}
\\

  aud\ending{īvistī}
& aud\ending{īverīs}
& aud\latin{ītus es}
& aud\latin{ītus sīs}
\\

  aud\ending{īvit}
& aud\ending{īverit}
& aud\latin{ītus est}
& aud\latin{ītus sit}
\\[\smallskipamount]

  aud\ending{īvimus}
& aud\ending{īverīmus}
& aud\ending{\1{ītī} sumus}
& aud\ending{\1{ītī} sīmus}
\\

  aud\ending{īvistis}
& aud\ending{īverītis}
& aud\ending{\1{ītī} estis}
& aud\ending{\1{ītī} sītis}
\\

  aud\ending{īvērunt}, \ending{-ēre}
& aud\ending{īverint}
& aud\ending{\1{ītī} sunt}
& aud\ending{\1{ītī} sint}
\\[\smallskipamount]

\cc{4}{\emph{Past Perfect}} \\[\smallskipamount]

  aud\ending{īveram}
& aud\ending{īvissem}
& aud\latin{ītus eram}
& aud\latin{ītus essem}
\\

  aud\ending{īverās}
& aud\ending{īvissēs}
& aud\latin{ītus erās}
& aud\latin{ītus essēs}
\\

  aud\ending{īverat}
& aud\ending{īvisset}
& aud\latin{ītus erat}
& aud\latin{ītus esset}
\\[\smallskipamount]

  aud\ending{īverāmus}
& aud\ending{īvissēmus}
& aud\ending{\1{ītī} erāmus}
& aud\ending{\1{ītī} essēmus}
\\

  aud\ending{īverātis}
& aud\ending{īvissētis}
& aud\ending{\1{ītī} erātis}
& aud\ending{\1{ītī} essētis}
\\

  aud\ending{īverant}
& aud\ending{īvissent}
& aud\ending{\1{ītī} erant}
& aud\ending{\1{ītī} essent}
\\[\smallskipamount]

\cc{4}{\emph{Future Perfect}} \\[\smallskipamount]

  aud\ending{īverō}
&
& aud\latin{ītus erō}
\\

  aud\ending{īveris}
&
& aud\latin{ītus eris}
\\

  aud\ending{īverit}
&
& aud\latin{ītus erit}
\\[\smallskipamount]

  aud\ending{īverimus}
&
& aud\ending{\1{ītī} erimus}
\\

  aud\ending{īveritis}
&
& aud\ending{\1{ītī} eritis}
\\

  aud\ending{īverint}
&
& aud\ending{\1{ītī} erunt}

\end{paradigm}

\begin{imperative}

Pres.
& aud\ending{ī}
& aud\ending{īte}
& aud\ending{īre}
& aud\ending{īminī}
\\

Fut.
& aud\ending{ītō}
& aud\ending{ītōte}
& aud\ending{ītor}
\\

& aud\ending{ītō}
& aud\ending{iuntō}
& aud\ending{ītor}
& aud\ending{iuntor}

\end{imperative}

\begin{nonfinite}

\cc{4}{\textbfsc{infinitive}} \\

Pres.
& aud\ending{īre}
&
& aud\ending{īrī}
\\

Perf.
& aud\ending{īvisse}
&
& aud\latin{ītus esse}
\\

Fut.
& aud\ending{ītūrus esse}
&
& aud\ending{ītum īrī}
\\[\medskipamount]

\cc{4}{\textbfsc{participle}} \\

Pres.
& aud\ending{iēns}
& Perf.
& aud\latin{ītus}
\\

Fut.
& aud\ending{ītūrus}
& Fut.
& aud\ending{iendus}
\\[\medskipamount]

& \textbfsc{gerund} && \textbfsc{supine} \\

Gen. & aud\ending{iendī} \\
Dat. & aud\ending{iendō} \\
Acc. & aud\ending{iendum} && aud\ending{ītum} \\
Abl. & aud\ending{iendō}  && aud\ending{ītū}

\end{nonfinite}

\chapter{Deponent Verbs}

\contentsentry{B}{Deponents and Semi-Deponents}

\section

Deponent Verbs\footnote{For many verbs ordinarily Deponent, early
  Latin shows Active forms.} are mostly Passive in form but Active in
meaning.  In addition to the Passive forms, they have also the Present
and Future Participles Active, the Future Infinitive Active, and the
Supine.  The Future Passive Participle, and occasionally the Perfect
Participle, are Passive in meaning.  The inflection follows that of
the regular Conjugations.  Examples:
\begin{Tabular}{>{\scshape}c>{\scshape}rlll}

Conjugation & i
& \latin{mīror}, \english{admire}
& \latin{mīrārī}
& \latin{mīrātus sum}
\\

\ditto & ii
& \latin{vereor}, \english{fear}
& \latin{verērī}
& \latin{veritus sum}
\\

\ditto & iii
& \latin{sequor}, \english{follow}
& \latin{sequī}
& \latin{secūtus sum}
\\

\ditto & iv
& \latin{partior}, \english{share}
& \latin{partīrī}
& \latin{partītus sum}

\end{Tabular}

\begin{Tabular}{>{\itshape}lllll}

\cc{5}{\textbfsc{indicative}} \\

& \textsc{i} & \textsc{ii} & \textsc{iii} & \textsc{iv} \\

Pres.
& mīror
& vereor
& sequor
& partior
\\

& mīrāris, -re
& verēris, -re
& sequeris, -re
& partīris, -re
\\

& mīrātur
& verētur
& sequitur
& partītur
\\[\smallskipamount]

& mīrāmur
& verēmur
& sequimur
& partīmur
\\

& mīrāminī
& verēminī
& sequiminī
& partīminī
\\

& mīrantur
& verentur
& sequuntur
& partiuntur
\\[\smallskipamount]

Imperf.
& mīrābar
& verēbar
& sequēbar
& partiēbar
\\[\smallskipamount]

Fut.
& mīrābor
& verēbor
& sequar
& partiar
\\[\smallskipamount]

Perf.
& mīrātus sum
& veritus sum
& secūtus sum
& partītus sum
\\[\smallskipamount]

Past Perf.
& mīrātus eram
& veritus eram
& secūtus eram
& partītus eram
\\[\smallskipamount]

Fut. Perf.
& mīrātus erō
& veritus erō
& secūtus erō
& partītus erō
\\[\medskipamount]

\cc{5}{\textbfsc{subjunctive}} \\

Pres.
& mīrer
& verear
& sequar
& partiar
\\

Imperf.
& mīrārer
& verērer
& sequerer
& partīrer
\\

Perf.
& mīrātus sim
& veritus sim
& secūtus sim
& partītus sim
\\

Past Perf.
& mīrātus essem
& veritus essem
& secūtus essem
& partītus essem
\\[\medskipamount]

\cc{5}{\textbfsc{imperative}} \\

Pres.
& mīrāre
& verēre
& sequere
& partīre
\\

Fut.
& mīrātor
& verētor
& sequitor
& partītor
\\[\medskipamount]

\cc{5}{\textbfsc{infinitive}} \\

Pres.
& mīrārī
& verērī
& sequī
& partīrī
\\

Perf.
& mīrātus esse
& veritus esse
& secūtus esse
& partītus esse
\\

Fut.
& mīrātūrus esse
& veritūrus esse
& secūtūrus esse
& partītūrus esse
\\[\medskipamount]

\pagebreak

\cc{5}{\textbfsc{participle}} \\

Pres.\ Act.
& mīrāns
& verēns
& sequēns
& partiēns
\\

Fut.\ Act.
& mīrātūrus
& veritūrus
& secūtūrus
& partītūrus
\\

Perf.\ Pass.
& mīrātus
& veritus
& secūtus
& partītus
\\

Fut.\ Pass.
& mīrandus
& verendus
& sequendus
& partiendus
\\

\end{Tabular}

\begin{Tabular}{l@{\extracolsep{\fill}}lll}

\cc{4}{\textbfsc{gerund}} \\

mīrandī, etc.
& verendī, etc.
& sequendī, etc.
& partiendī, etc.
\\[\smallskipamount]

\cc{4}{\textbfsc{supine}} \\

mīrātum, -tū
& veritum, -tū
& secūtum, -tū
& partītum, -tū

\end{Tabular}

\smallskip

\subtitle{\textsc{Semi-Deponents}}

\smallskip

\section

Semi-Deponents are verbs of which the Perfect System is Passive in
form but Active in meaning, such as:
\begin{Tabular}{lll}

\latin{audeō}, \english{dare},
& \latin{audēre}
& \latin{ausus sum}
\\

\latin{gaudeō}, \english{rejoice},
& \latin{gaudēre}
& \latin{gāvīsus sum}
\\

\latin{soleō}, \english{am wont},
& \latin{solēre}
& \latin{solitus sum}
\\

\latin{fīdō}, \english{trust},
& \latin{fīdere}
& \latin{fīsus sum}

\end{Tabular}

\begin{note}

Some verbs, otherwise regular, have a Perfect Passive Participle with
active meaning.  So \latin{cēnātus}, \english{having dined}, from
\latin{cēnō}, \english{dine}; \latin{iūrātus}, \english{having sworn},
from \latin{iūrō}, \english{swear}; \latin{prānsus}, \english{having
  breakfasted}, from \latin{prandeō}, \english{breakfast};
\latin{pōtus}, \english{having drunk}, from \latin{pōtō},
\english{drink}.

\end{note}

\chapter{Periphrastic Conjugation}

\contentsentry{B}{Periphrastic Conjugation}

\section

The Periphrastic Conjugation\footnote{That is, a Conjugation in which
  all the parts are expressed by a phrase rather than by a single form.
  \emph{Some} parts of the Regular Conjugations are also periphrastic,
  as \latin{amātus sum}.} is a combination of the Future Active or
Future Passive Participle with the verb \latin{sum}.
\begin{Tabular}{>{\itshape}lll}

\cc{3}{\textsc{Active}} \\

\cc{3}{\latin{Amātūrus sum}, \english{I am about to love}} \\

& \textbfsc{indicative} & \textbfsc{subjunctive} \\

Pres.
& amātūrus sum, \english{I am about to love}
& amātūrus sim
\\

Imper.
& amātūrus eram, \english{I was about to love}
& amātūrus essem
\\

Fut.
& amātūrus erō, \english{I shall be about to love}
\\

Perf.
& amātūrus fuī, \english{I have been, was, about to love}
& amātūrus fuerim
\\

Past Perf.
& amātūrus fueram, \english{I had been about to love}
& amātūrus fuissem
\\

Fut.\ Perf.
& amātūrus fuerō, \english{I shall have been about to love}
\\[\medskipamount]

\cc{3}{\textbfsc{infinitive}} \\

Pres.
& \M{2}{l}{amātūrus esse, \english{to be about to love}}\\

Perf.
& \M{2}{l}{amātūrus fuisse, \english{to have been about to love}}
\\[\bigskipamount]

\pagebreak

\cc{3}{\textsc{Passive}} \\

\cc{3}{\latin{Amandus sum}, \english{I \(am to be loved\) have to be loved}} \\

& \textbfsc{indicative} & \textbfsc{subjunctive} \\

Pres.
& amandus sum, \english{I have to be loved}
& amandus sim
\\

Imper.
& amandus eram, \english{I had to be loved}
& amandus essem
\\

Fut.
& amandus erō, \english{I shall have to be loved}
\\

Perf.
& amandus fuī, \english{I have had to be loved}
& amandus fuerim
\\

Past Perf.
& amandus fueram, \english{I had had to be loved}
& amandus fuissem
\\

Fut.\ Perf.
& amandus fuerō, \english{I shall have had to be loved}
\\[\medskipamount]

\cc{3}{\textbfsc{infinitive}} \\

Pres.
& \M{2}{l}{amandus esse, \english{to have to be loved}}\\

Perf.
& \M{2}{l}{amandus fuisse, \english{to have had to be loved}}

\end{Tabular}

\chapter{Peculiarities in Conjugation}

\contentsentry{B}{Peculiarities in Conjugation}

\headingC{Short Forms of the Perfect System}

\section
\subsection

Perfects in \ending{-āvī} and \ending{-ēvī}, as well as the other
tenses formed from the same stem, have a series of shortened forms in
which the \phone{v}, together with the following vowel, is lost before
\phone{s} and~\phone{r}.  Perfects in \ending{-īvī} also have forms
without the~\phone{v}, but the vowel is lost only before~\phone{s},
not before~\phone{r}.  The two sets of forms may be seen in the
following\footnote{\label{ftn:96:1}The student should observe that in
  the shortened forms the vowel before~\phone{s} is always long, and
  also that before~\phone{r}, except in forms like \latin{audieram},
  etc., in which both \phone{i} and \phone{e} are short.}:
\begin{Tabular*}{l@{\,\,}l@{\,\,}l
                 @{\extracolsep{\fill}}
                 >{\bfseries}l@{\extracolsep{0pt}\,\,}>{\bfseries}l@{\,\,}>{\bfseries}l}

\cc{3}{\textsc{Full Forms}} & \cc{3}{\textsc{Shortened Forms}} \\

\cc{6}{\emph{Perfect Indicative}} \\

   amāvistī,
& dēlēvistī,
& audīvistī
&  amāstī,
& dēlēstī,
& audīstī
\\

   amāvistis,
& dēlēvistis,
& audīvistis
&  amāstis,
& dēlēstis,
& audīstis
\\

   amāvērunt,
& dēlēvērunt,
& audīvērunt
&  amārunt,
& dēlērunt,
& audiērunt
\\[\medskipamount]

\cc{6}{\emph{Past Perfect Indicative}} \\

   amāveram,\footnote{That is, \latin{amāveram}, \latin{amāverās},
     \latin{amāverat}, etc.  Similarly elsewhere.}
& dēlēveram,
& audīveram
&  amāram,
& dēlēram,
& audieram
\\[\medskipamount]

\cc{6}{\emph{Future Perfect Indicative}} \\

   amāverō,
& dēlēverō,
& audīverō
&  amārō,
& dēlērō,
& audierō
\\[\medskipamount]

\cc{6}{\emph{Perfect Subjunctive}} \\

   amāverim,
& dēlēverim,
& audīverim
&  amārim,
& dēlērim,
& audierim
\\[\medskipamount]

\cc{6}{\emph{Past Perfect Subjunctive}} \\

   amāvissem,
& dēlēvissem,
& audīvissem
&  amāssem,
& dēlēssem,
& audīssem
\\[\medskipamount]

\cc{6}{\emph{Perfect Infinitive}} \\

   amāvisse,
& dēlēvisse,
& audīvisse
&  amāsse,
& dēlēsse,
& audīsse

\end{Tabular*}

\subsection

Similarly from
\latin{nōvī}: nōvistī,—\latin{nōstī};
nōvērunt,—\latin{nōrunt};
nōveram,—\latin{nō\-ram};
nōvisse,—\latin{nōsse}, etc.
(but Fut.\ Perf.\ \latin{nōro} only in compounds).

\subsection

Beside Perfects in \ending{-ivī} are sometimes found, in the First and
Third Singular, forms in \ending{-iī}, \ending{-iit}, as
\latin{audiī}, \latin{audiit}; and, rarely, a similar First Plural
form, such as \latin{audiimus}.  A contracted form \latin{audīt} from
\latin{audīvit} also occurs.

\begin{note}

It is probable that neither the forms like \latin{audiī}, nor those
like \latin{audieram}, mentioned above, really come from the forms
with~\phone{v}.  They seem, rather, to have started from Perfects
which were regularly formed without~\phone{v}, especially the Perfect
of \latin{eō}, \english{go}, and its compounds, e.g.\ \latin{iī},
\latin{abiī}, etc.\ (\xref[\emph{a}]{194}).  On the other hand, forms
like \latin{audīstī}, \latin{audīssem}, etc., and all the short forms
of Perfects in \ending{-āvī} and \ending{-ēvī}, are the result of
contraction.

\end{note}

\begin{minor}

\subsection

Perfects in \ending{-sī} and the other tenses formed from the same
stem sometimes have contracted forms, beside the full forms, wherever
the \phone{s} is itself followed by~\infix{-is-} in the ending;
e.g.\ \latin{dīxtī} beside \latin{dīxistī}; \latin{dīxem} beside
\latin{dīxissem}; \latin{dīxe} beside \latin{dīxisse}.  Such forms are
more frequent in early Latin, but are also found in later authors.

\subsection

In the Future Perfect Indicative and the Perfect Subjunctive early
Latin has forms in \ending{-sō} and \ending{-sim} (\ending{-ssō} and
\ending{-ssim}); e.g.\ \latin{faxō} and \latin{faxim} from
\latin{faciō}; \latin{ausim} from \latin{audeō}; \latin{capsō} from
\latin{capiō}; \latin{axim} from \latin{agō};
Perf.\ Subj.\ \latin{sīrīs}, \latin{sīrit}, etc., from \latin{sinō}
(\latin{sīrīs} from \rec{sī-sīs}; see \xref{47}); \latin{amāssō},
  \latin{negāssim} (also Infin., as \latin{re\-con\-ci\-li\-ās\-sere}).

\latin{Faxō}, \latin{faxim}, \latin{ausim}, and, rarely, \latin{sīrīs}
occur also in later authors.

\end{minor}

\headingC{Other Peculiarities}

\section
\subsection

The Imperatives of \latin{dīcō}, \latin{dūcō}, and \latin{faciō} are
\latin{dīc}, \latin{dūc}, and \latin{fac}, instead of \latin{dīce},
\latin{dūce}, and \latin{face}, though in early Latin the latter are
more frequent.  The same is true of compounds of \latin{dīcō} and
\latin{dūcō}; e.g.\ \latin{maled\'īc}, \latin{ēd\'ūc} (for the accent,
see \xref[1]{32}); but in compounds of \latin{faciō} only the full
form is known; e.g.\ \latin{calfáce}.  Cf.\ also \latin{fer},
\latin{c\'ōnfer} from \latin{ferō} (\xref{193}).

\subsection

The Future Passive Participle and the Gerund of the Third and Fourth
Conjugations often end in \suffix{-undus} and \suffix{-undī};
e.g.\ \latin{ferundus}, \latin{faciundus}, \latin{potiundī}.

\begin{minor}

\subsection

The Present Infinitive Passive has an early and poetical form in
\ending{-ier}; e.g.\ \latin{laudārier}, \latin{vidērier},
\latin{dīcier}.

\subsection

The Imperfect Indicative of the Fourth Conjugation has an old form in
\ending{-ībam}, found in poetry of all periods; e.g.\ \latin{lēnībat},
\latin{polībant}.

\subsection

The Future of the Fourth Conjugation has an early form in
\ending{-ībō}; e.g.\ \latin{audībō}, \latin{dormībō}.

\subsection

There is some confusion in the Second Singular and First and Second
Plural between the Future Perfect Indicative, which regularly has
short~\phone{i} in these forms, and the Perfect Subjunctive, which
regularly has long~\phone{i}.  In the former \ending{-īs} is nearly as
common as the normal \ending{-is}, and \ending{-īmus}, \ending{-ītis}
are also found; while in the latter \ending{-is} is frequent beside
\ending{-īs}, and \ending{-imus} occurs once for \latin{-īmus}. See
\xref[2]{174}, \xref[\emph{b}]{175}.

\end{minor}

\subsection

In the Future Active Infinitive and the Perfect Passive Infinitive,
\latin{esse} is often omitted; e.g.\ \latin{amātūrus} for
\latin{amātūrus esse}; \latin{amātus} for \latin{amātus esse}.

\subsection

In the Perfect Passive System the auxiliary is sometimes a form of the
Perfect System of \latin{sum}, instead of the usual form of the
Present System; e.g.\ \latin{amātus fuit} for \latin{amātus est},
\latin{amātus fuerat} for \latin{amātus erat}, etc.  The form
\latin{amātus fuerō}, etc., also occurs, but more rarely.

\headingC{Variation between Conjugations}

\section
\subsection

Some verbs in \ending{-ior} which in general follow the Third
Conjugation have also forms with~\phone{ī}.  So \latin{orior} has
usually \latin{oritur} (but \latin{adorītur}), but \latin{orīrētur}
beside \latin{orerētur}, and always Infinitive \latin{orīri};
\latin{potior} has nearly always \latin{potitur}, but
\latin{potīrētur} beside \latin{poterētur}, and nearly always
\latin{potīri}; \latin{morior} has \latin{moritur}, but sometimes
Infinitive \latin{morīri} (chiefly in early Latin) beside
\latin{morī}.  Other examples are rare.

\begin{minor}

\subsection

Beside \latin{lavō}, \english{wash}, \latin{lavāre}, there are also
forms of the Third Conjugation, as \latin{lavit}, \latin{lavimus},
etc.  Other examples of variation between the First and Third
Conjugations, and also between the Second and Third, are mostly
confined to early Latin.

\end{minor}

\chapter{Formation of the Stems}

\contentsentry{B}{Formation of the Stems}

\headingC{The Present Stem}

\section

Conjugation I\@.—Present Stem in \infix{-ā-}.

\subsection

Most verbs of the First Conjugation are
Denominatives,\footnote{\label{ftn:98:1}The term Denominative (from
  \latin{dē} and \latin{nōmen}) is used of Verbs which are derived
  from Nouns or Adjectives.  In contrast to these are the Primary
  Verbs, which are formed directly from Roots.}  as \latin{cūrō},
\english{care for}, \latin{cūrāre}, from the Noun \latin{cūra},
\english{care}.  See~\xref[1]{211}.  The Present Stem is also the
Verb-Stem, the \phone{ā} appearing in the Perfect and Participial
Stems; e.g.\ \latin{cūrāvī}, \latin{cūrātum}.

\subsection

The Frequentatives, like \latin{rogitō}, \english{keep asking},
\latin{rogitāre}, are also Denominative in origin, and form a large
class. See~\xref[1]{212}.

\subsection

There are also a few Primary Verbs\footnotemark[\thefootnote] from
roots ending in~\phone{ā}, as \latin{nō}, \english{swim}, \latin{nāre}
(Perf.\ \latin{nāvī}), \latin{stō}, \english{stand}, \latin{stāre}
(\latin{stetī}).

\subsection

There are a few Primary Verbs in which the \phone{ā} belongs only to
the Present Stem, as \latin{iuvō}, \english{aid}, \latin{iuvāre} (but
\latin{iūvī}, \latin{iūtum}); \latin{secō}, \english{cut},
\latin{secāre} (but \latin{secuī}, \latin{sectum}).

\begin{note}

The Present Stem of the Denominatives was originally \sound{-ā-yo-}
and \sound{-ā-ye-} (\xref[\allowbreak note]{211}), which became \sound{-ā-},
partly by loss of \sound{y} and contraction, partly through the
influence of Primary Verbs having the stem in original~\sound{-ā-}.

\end{note}

\section

Conjugation II\@.—Present Stem in \infix{-ē-}.

\subsection

In most verbs the \phone{ē} is confined to the Present System.  We may
further distinguish:
\begin{enuma}

\item
Primary Verbs, as \latin{videō}, \english{see}, \latin{vidēre}
(\latin{vīdī}, \latin{vīsum}).

\item
Causatives, as \latin{moneō}, \english{cause to think},
\english{advise}, \latin{monēre} (\latin{monuī}, \latin{monitum}),
from the root \root{men} seen in \latin{me-min-ī}, \english{remember}.

\item
Denominatives, as \latin{albeō}, \english{be white}, \latin{albēre},
from \latin{albus}, \english{white}.

\end{enuma}

\subsection

There are some Primary Verbs from roots ending in~\phone{ē}.  In these
the \phone{ē} belongs to the whole verb-system.  So \latin{fleō}, \english{weep}, \latin{flēre},
\latin{flēvī}, \latin{flētum}.  Similarly \latin{neō}, \english{spin},
\latin{pleō}, \english{fill}, etc.

\begin{note}

The Present Stem of the Denominatives was originally \sound{-e-yo-}
and \sound{-e-ye-} (\xref[\allowbreak note]{211}; the Causatives also had
\sound{-eyo} and \sound{-eye-}).  The latter became regularly
\infix{-ē-}, and, aided also by the influence of Primary Verbs having
the stem in original \infix{-ē-}, this became established as the
Present Stem for most forms.  But the First Sing.\ Pres.\ Indic.\ in
\ending{-eō} and the Present Subj.\ in \ending{-eam}, etc., are based
upon the stem \sound{-e\(y\)o-}, the \sound{y} being lost, but without
contraction of the vowels.

\end{note}

\section

Conjugation III\@.—Present Stem in \infix{-e-} and \infix{-o-}, the
thematic vowel.\footnote{See \ftn*{76}{}, footnote.}  Various types
are to be distinguished:
\begin{enumA}

\item

Simple Thematic Present, as \latin{dūcō}, \english{lead}
(\latin{dūxī}, \latin{ductum}).  The Present Stem consists simply of
the root with the thematic vowel.  This is by far the commonest type.

\item

Reduplicated Present, as \latin{si-stō}, \english{set} (\latin{stitī},
\latin{statum}).  The reduplication consists of the initial consonant
of the root and the vowel~\phone{i}.

\subsubsection

A less obvious example is \latin{serō}, \english{sow} (\latin{sēvī},
\latin{satum}), which comes from \rec{si-sō}.

\item

Present with Inserted Nasal, as \latin{rumpō}, \english{break}
(\latin{rūpī}, \latin{ruptum}).  Sometimes the nasal is extended to
the Perfect and Participial Stems; e.g.\ \latin{iungō}, \english{join},
\latin{iūnxī}, \latin{iūnctum}.

\begin{note}
Those verbs in which the nasal is extended throughout the verb-system
outwardly resemble verbs like \latin{pendō}, in which the \phone{n} is
a part of the root.  But the related forms (as \latin{iugum} beside
\latin{iungō}, but \latin{pondus} beside \latin{pendō}, as well as
some other less obvious factors, make it possible to distinguish the
two classes.  In verbs in \ending{-endō}, as \latin{tendō},
\latin{pendō}, \latin{fendō}, the \phone{n} belongs to the root.

\end{note}

\item
Present in \ending{-nō}, as \latin{ster-nō}, \english{strew}
(\latin{strāvi}, \latin{strātum}).

\setcounter{subsubsection}{0}

\subsubsection
This type properly includes some Presents in \ending{-llō}, coming
from \ending{-lnō} by the regular assimilation of \phone{ln}
(\xref[11]{49}); e.g.\ \latin{pellō}, \english{strike}
(\latin{pepulī}), \latin{tollō}, \english{raise} (\latin{sus-tulī}),
\latin{fallō}, \english{deceive} (Perf.\ \latin{fefellī}, with
extension of \phone{ll} from the Present).

\item
Present in \ending{-tō}, as \latin{flec-tō}, \english{turn}
(\latin{flexī}, \latin{flexum}).

\item
Present in \ending{-scō}, as \latin{crē-scō}, \english{increase}
(\latin{crēvī}, \latin{crētum}).

\begin{note}

The root to which the \ending{-scō} is added ends in a long vowel in
all examples of this formation, except \latin{discō}, \english{learn},
\latin{poscō}, \english{ask}, and \latin{compescō},
\english{restrain}, in which a consonant has been lost before the
\ending{-scō}.  \latin{Discō} comes from \rec{dic-scō}, earlier
\rec{di-dc-scō} with reduplication (cf.\ Perf.\ \latin{didicī});
\latin{poscō} from \rec{porc-scō} (cf.\ \latin{precor});
\latin{compescō} from \rec{comperc-scō}.  In \latin{poscō} the
\phone{sc} has spread from the Present to the whole verb-system
(Perf.\ \latin{po-poscī}).

\end{note}

\setcounter{subsubsection}{0}

\subsubsection

The extension of this suffix in the specific meaning of
\english{becoming} or \english{beginning to} has given rise to the
numerous class of Inchoatives in \ending{-ēscō}, \ending{-īscō},
\ending{-āscō} formed from other Verbs and from Nouns.
See~\xref[2]{212}.

\item

Verbs in \ending{-essō}, as \latin{capessō}, \english{seize eagerly}
(\latin{capessīvī}, \latin{capessītum}).  See~\xref[4]{212}.

\item

Verbs in \ending{-uō}, as \latin{statuō}, \english{set}
(\latin{statuī}, \latin{statūtum}).  They include:
\begin{enum1}

\item
Denominatives from \phone{u}-Stems, as \latin{statuō} from
\latin{status}.

\item
Primary Verbs from roots in \phone{u}, as \latin{suō}, \english{sew}.

\item
Primary Verbs with a suffix \suffix{-nuō}, as \latin{mi-nuō},
\english{lessen}.

\end{enum1}

\setcounter{subsubsection}{0}

\subsubsection
This type includes also \latin{solvō}, \english{loose}
(\latin{solvī}, \latin{solūtus}), and \latin{volvō}, \english{roll}
(\latin{volvī}, \latin{volūtus}), in which \phone{u} has become
consonantal.

\item
Present in \ending{-iō}, as \latin{capiō}, \english{take},
\latin{capere} (\latin{cēpī}, \latin{captum}).

\begin{note}

Verbs like \latin{capiō} are in origin closely connected with the
Primary Verbs of the Fourth Conjugation. The short~\phone{i}, partly
inherited, but partly, also, due to “iambic shortening”
(\xref[note]{28}; observe that the first syllable is short in all
verbs of this type), caused a resemblance to the forms of the Third
Conjugation, in which \phone{i} comes from~\phone{e}.  For some
examples of confusion with the Fourth Conjugation, see~\xref[1]{165}.

\end{note}

\end{enumA}

\section

Conjugation IV\@.—Present Stem in \infix{-ī-}.  Two
types are to be distinguished:

\subsection

Denominatives, as \latin{fīniō}, \english{finish}, \latin{fīnīre},
from \latin{fīnis}, \english{end}.  The Present Stem is also the
Verb-Stem (\latin{fīnīvī}, \latin{fīnītum}).

\subsection

Primary Verbs, as \latin{veniō}, \english{come}, \latin{venīre}.  The
\phone{ī} is usually confined to the Present System (\latin{vēnī},
\latin{ventum}),

\begin{note}

The Present Stem of Denominatives was originally \sound{-i-yo-} and
\sound{-i-ye}, or, when formed from consonant-stems, \sound{-yo-} and
\sound{-ye-} (\xref[note]{211}); that of Primary Verbs was
\sound{-iyo-} or \sound{-yo-} and \sound{-ī-}.  This last became
established as the Present Stem for most forms.  But the First
Sing.\ and Third Plur.\ Pres.\ Indic.\ in \ending{-iō} and
\ending{-iunt}, as well as the Imperf.\ Indic.\ in \ending{-iēbam},
etc., the Fut.\ in \ending{-iam}, etc., and the Pres.\ Subj.\ in
\ending{-iam}, etc., are based upon the stem~\sound{-i\(y\)o-}.

\end{note}

\section

The Irregular Verbs.

These are relics of a once extensive system of conjugation known as
unthematic,\footnote{See above, \ftn*{76}{}, footnote.} in which the endings
were added directly to the root, as in \latin{es-t}, \english{is},
\latin{fer-t}, \english{brings}, etc.  But only part of the forms of
the Irregular Verbs are of this nature; the rest differ in no way from
those of the Third Conjugation, e.g.\ \latin{ferō}, \latin{ferimus},
\latin{ferunt}.

\headingC{The Imperfect Indicative}

\section

The Tense-Sign of the Imperfect Indicative is \infix{-bā-} (shortened
to \infix{-ba-} before the endings \ending{-m}, \ending{-t},
\ending{-nt}, \ending{-r}; see~\xref[1,2]{26}), added to forms ending
in a long vowel, namely, \ending{-ā} for the First Conjugation,
\ending{-ē} for the Second and Third, and \ending{-iē}
(sometimes~\ending{-ī}) for the Fourth.

\begin{note}

This formation originated in the combination of a past tense of the
verb \english{to be} (cf.\ \latin{fuī}) with certain case-forms
(probably old Instrumentals), which, in this combination, became
associated with the verb-system.  The case-form in \ending{-ē}
belonged to a noun-stem in~\ending{-o} (cf.\ Adverbs in~\ending{-ē},
originally Ablatives of \phone{o}-Stems; see~\xref[1]{126}); and,
since in verbs the corresponding stem-vowel is the thematic vowel of
the Third Conjugation (\emend{6}{\ftn*{80}{}}{\ftn*{76}{}}, footnote), the form in
\ending{-ē-bam} came to be used in the Third Conjugation as well as in
the Second, in which the Present Stem ends in~\infix{-ē-}.  In the
Fourth Conjugation, \ending{-ī-bam} represents an earlier type than
\ending{-iē-bam} (cf.\ early Latin Fut.\ \latin{audībō}, not
\rec{audiēbō}).

\end{note}

\headingC{The Future Indicative}

\section

In the First and Second Conjugations, and in early Latin sometimes in
the Fourth, the Future is formed like the Imperfect, except that the
tense-sign is \phone{b} + the thematic vowel, instead of \infix{-bā-}.

In the Third and Fourth Conjugations the tense-sign is \infix{-ā-},
which occurs in the First Person (shortened, because before the
endings \ending{-m} or \ending{-r}; see~\xref[1, 2]{26}), or
\infix{-ē-}, which occurs in all other Persons (shortened before the
endings \ending{-t}, \ending{-nt}; see~\xref[1]{26}).

\begin{note}

The Future in \ending{-bō} originated in the combination of a Future
of the verb \english{to be} (formed like \latin{erō}, but from the
root seen in \latin{fuī}) with the same case-forms which appear in the
formation of the Imperfect.

Both \phone{ā} and \phone{ē} were Mood-Signs of the Subjunctive
(\xref{175}), so that the Future of the Third and Fourth Conjugation
are in origin Present Subjunctives,—only, except in the First
Person, of a different type from that which is seen in the regular
Present Subjunctive.

The Future of \latin{sum} (\latin{erō}, \latin{eris}, etc.)\ is also a
Subjunctive in origin, but of still another type, the mood-sign being
simply the thematic vowel.  This was originally the regular
Subjunctive formation for all \emph{unthematic} Indicatives, and so
would have been the normal formation for the Future of the Irregular
Verbs so far as they are truly unthematic (\xref{170}).  But in all of
these except \latin{sum}, the Present contains so many thematic forms
identical with those of the Third Conjugation that the Future also has
the same formation as in the Third Conjugation;
e.g.\ Fut.\ \latin{feram} beside Pres.\ \latin{ferō}, like
\latin{tegam} beside \latin{tegō}.

\end{note}

\headingC{The Perfect Indicative}

\section

Various types are to be distinguished:
\psection
Perfect in \ending{-vī}.  In the form \ending{-āvī} this type is
common to most verbs of the First Conjugation, and in the form
\ending{-īvī} to a large proportion of the verbs of the Fourth.
Several verbs of the Second and Third Conjugations have Perfects in
which \ending{-vī} is added to the root, or to a variant form of the
root, ending in a long vowel, giving rise to forms in \ending{-ēvī},
\ending{-ōvī}, as well as \ending{-āvī}, \ending{-īvī}.  Examples (the
prevailing types in black):
\begin{Tabular}{l@{\,\,}c@{\,\,}l
                @{\quad}
                l@{\,\,}c@{\,\,}l}

\latin{amā-vī}, & from & \latin{amō}, \english{love}, \latin{amāre}
& \latin{fīnī-vī}, & from & \latin{fīniō}, \english{finish}, \latin{fīnīre}
\\

flē-vī\footnote{Observe that \ending{-ēvī} is not the common
  type for verbs of the Second Conjugation, as \ending{-āvī} is for
  those of the First.  Only those verbs in which \phone{ē} belongs to
  the root-syllable have Perfects in~\ending{-ēvī}.  An apparent
  exception is \latin{dēleō}, \english{destroy},
  Perf.\ \latin{dēlēvī}, but this is really a compound \latin{dē-leō}
  (cf.\ \latin{linō}, \english{smear}, Perf.\ \latin{lēvī}).}
& \ditto & fleō, \english{weep}, flēre
& nō-vī & \ditto & nōscō, \english{know}, nōscere \\

  crē-vī  & \ditto & crēscō, \english{grow}, crēscere
& strā-vī & \ditto & sternō, \english{strew}, sternere \\

  sprē-vī & \ditto & spernō, \english{spurn}, spernere
& trī-vī  & \ditto & terō, \english{rub}, terere

\end{Tabular}

\begin{minor}

\subsubsection
For shortened forms of \ending{-vī}-Perfects, see above,~\xref{163}.

\end{minor}

\psection
Perfect in \ending{-uī}.  This is common to a large proportion of the
verbs of the Second Conjugation, and to many of the Third, mostly
those in which the root-syllable ends in \phone{l}, \phone{r},
\phone{m}, or~\phone{n}.  It is also found in some Primary Verbs of
the First and a very few of the Fourth Conjugation.  Examples:
\begin{Tabular}{l@{\,\,}c@{\,\,}l
                @{\quad}
                l@{\,\,}c@{\,\,}l}

  \latin{monuī}, & from   & \latin{moneō}, \english{advise}, \latin{monēre}
& \latin{moluī}, & from   & \latin{molō}, \english{grind}, \latin{molere}\\

  secuī          & \ditto & secō, \english{cut}, secāre
& saluī          & \ditto & saliō, \english{leap}, salīre

\end{Tabular}

\begin{note}

This is obviously related to the preceding type.  The apparent
difference is that \ending{-vī} is used after vowels and \ending{-uī}
after consonants.  But \ending{-u\emend{75}{i}{ī}} probably comes through
\rec{-o-vī}, from \rec{-ĕ-vī}, just as \latin{denuō} comes from
\rec{dē-novō} (\xref[4]{42}) and this \latin{novo-} from an earlier
\rec{nevo-}.  With this assumed \rec{-e-vī} compare the Participle in
\ending{-itus}, from \rec{-e-tos}, which nearly always accompanies the
Perfect in \ending{-uī} (\xref[3]{179}).

\end{note}

\setcounter{subsubsection}{0}

\subsubsection

A combination of this with the following type is seen in
\latin{messuī}, from \latin{metō}, \english{mow}, and \latin{nexuī}
from \latin{nectō}, \english{bind}.

\psection

Perfect in \ending{-sī}.  This is most common in the Third
Conjugation, but is not infrequent in the Second, and is occasionally
found in the Fourth. Examples;
\begin{Tabular}{l@{\,\,}c@{\,\,}l
                @{\quad}
                l@{\,\,}c@{\,\,}l}

\latin{dīxī},   & from  & \latin{dīcō}, \english{say}, \latin{dīcere}
& \latin{serpsī},& from  & \latin{serpō}, \english{crawl}, \latin{serpere}\\

\latin{auxī}    & \ditto & \latin{augeō}, \english{increase}, \latin{augēre}
& sēnsī         & \ditto & sentiō, \english{feel}, sentīre

\end{Tabular}

\setcounter{subsubsection}{0}

\begin{minor}

\subsubsection

Consonant changes.  The changes resulting from the combination of the
final consonant of the root with the \phone{s} are in accordance with
the statements aheady given (\xref[1–4, 7]{49}).  \latin{Ius-sī}
belongs under \xref[4]{49}, since the \phone{b} of \latin{iubeō}
stands for an original dental (\phone{dh}).  In \latin{ges-sī},
\latin{us-sī}, the root itself ends in~\phone{s}, which has
become~\phone{r} in the Presents \latin{gerō}, \latin{ūrō}
(\xref{47}).  Similarly \latin{pres-sī} from \root{pres-}, although
the Present \latin{premō} is from \root{prem-}.  For \latin{vīxī},
\latin{strūxī}, \latin{flūxī} (\latin{vīvō}, \latin{struō},
\latin{fluō}), see~\xref[2]{49}.

\subsubsection

In general, barring the regular lengthening before \phone{ns} and
\phone{nx} (\xref{18}), the quantity of the vowel in the root-syllable
of this Perfect is the same as in the Present.  But there are some
examples of an inherited variation (\xref{46}), as follows:

\subsubsection

A short vowel, as against a long vowel in the Present, is seen in
\latin{ussī}, from \latin{ūrō}; \latin{cessī}, from \latin{cēdō}.

\subsubsection

A long vowel, as against a short vowel in the Present, is seen in
\latin{mīsī}, from \latin{mittō}; \latin{dīvīsī}, from \latin{dīvidō};
\latin{rēxī}, from \latin{regō}; \latin{tēxī}, from \latin{tegō};
\latin{flūxī}, from \latin{fluō}; \latin{strūxī}, from \latin{struō};
and (probably) \latin{trāxī}, from \latin{trahō}.  Compare
\latin{lēgī}, from \latin{legō}, of type~E.

\end{minor}

\psection

Reduplicated Perfect.  This is confined to the Third Conjugation,
except for four examples from the Second (\latin{mordeō},
\latin{pendō}, \latin{spondeō}, \latin{tondeō}), and the verbs \latin{dō} and
\latin{stō}.  The vowel of the reduplication is regularly~\phone{e};
but this is replaced by the vowel of the root-syllable wherever the
latter, in both the Present and the Perfect, is \phone{i}, \phone{u},
or~\phone{o}.  For the changes in the vowel of the root-syllable, see
\xref{42}.  Examples:
\begin{Tabular}{l@{\,\,}c@{\,\,}l
                @{\quad}
                l@{\,\,}c@{\,\,}l}

  \latin{cecinī}, & from & \latin{canō}, \english{sing}
& \latin{didicī}, & from & \latin{discō}, \english{learn} \\

  \latin{cecīdī}    & \ditto & \latin{caedō}, \english{cut}
& \latin{cucurrī}   & \ditto & \latin{currō}, \english{run} \\

  \latin{pepulī}    & \ditto & \latin{pellō}, \english{strike}
& \latin{momordī}   & \ditto & \latin{mordeō}, \english{bite} \\

  \latin{tetendī}   & \ditto & \latin{tendō}, \english{stretch}
& \latin{stetī}     & \ditto & \latin{stō}, \english{stand}

\end{Tabular}

\setcounter{subsubsection}{0}

\subsubsection

In compounds, except those of \latin{dō}, \latin{stō}, \latin{sistō},
\latin{discō}, \latin{poscō}, the reduplication is usually lost.  So
\latin{oc-cīdī}, \latin{at-tendī}, etc.; but often \ending{-cucurrī}
beside \ending{-currī} in compounds of \latin{currō}.  Compounds of
\latin{canō} and \latin{pungō} (Perf.\ \latin{pupugī}) substitute
other formations; e.g.\ \latin{oc-cinuī} and \latin{ex-pūnxī}.

\begin{minor}

\subsubsection

In verbs beginning with \phone{sp} or \phone{st}, both consonants
appear in the reduplication, but \phone{s}~is lost in the
root-syllable; e.g.\ \latin{spopondī} (for \rec{spo-spondī}) from
\latin{spondeō}, \latin{stetī} from \latin{stō}.

\end{minor}

\psection

Perfect in \ending{-ī} with lengthened vowel in the root-syllable.
This type is found mostly in the Second and Third
Conjugations.  Examples:
\begin{Tabular}{l@{\,\,}c@{\,\,}l
                @{\quad}
                l@{\,\,}c@{\,\,}l}

  \latin{sēdī}, & from  & \latin{sedeō}, \english{sit}
& \latin{ēdī},  & from  & \latin{edō}, \english{eat}, \latin{edere} \\

  \latin{mōvī}  & \ditto & \latin{moveō}, \english{move}
& \latin{fēcī}  & \ditto & \latin{faciō}, \english{do}, \latin{facere} \\

  \latin{cāvī}  & \ditto & \latin{caveō}, \english{beware}
& \latin{fōdī}  & \ditto & \latin{fodiō}, \english{dig}, \latin{fodere} \\

  iūvī & \ditto & iuvō, \english{aid}, iuvāre
& vēnī & \ditto & veniō, \english{come}, venīre

\end{Tabular}

\psection

Perfect in \ending{-ī} without change of the vowel of the
root-syllable.  This type is found in many Verbs of the Third
Conjugation, including nearly all in \ending{-uō}.  Examples:
\begin{Tabular}{l@{\,\,}c@{\,\,}l
                @{\qquad\qquad}
                l@{\,\,}c@{\,\,}l}

  \latin{vertī},    & from & \latin{vertō}, \english{turn}
& \latin{luī},\footnote{Such Perfects, though ending in \ending{-uī},
    are not to be classified under the \ending{-uī} type, since the
    \phone{u} belongs to the Verb-Stem.}
    & from & \latin{luō}, \english{atone for} \\

  \latin{fidī} & \ditto & \latin{findō}, \english{split}
& \latin{minuī} & \ditto & \latin{minuō}, \english{lessen}

\end{Tabular}

\headingC{The Past Perfect Indicative and the Future Perfect}

\section
\subsection

The Past Perfect Indicative is formed from the Perfect Stem +
\infix{-erā-} (originally \infix{-esā-}), with the regular shortening
of \phone{ā} before the endings \ending{-m}, \ending{-t}, \ending{-nt}
(\xref[1]{26}).

\subsection

The Future Perfect is formed from the Perfect Stem + \infix{-er-}
(originally \infix{-es-}), followed by the thematic vowel.  The Third
Plural in \ending{-int} (not \ending{-unt}) and the Second Singular
and First and Second Plural forms in \ending{-īs}, \ending{-īmus},
\ending{-ītis} beside the regular \ending{-is}, \ending{-imus},
\ending{-itis}, are due to confusion with the Perfect Subjunctive, in
which \phone{ī} was original. See~\xref[6]{164}; \xref[\emph{b}]{175}.

\headingC{The Subjunctive}

\section

The Subjunctive has three Mood-Signs, namely, \infix{-ā-},
\infix{-ē-}, and \infix{-ī-} (shortened before the endings
\ending{-m}, \ending{-t}, \ending{-nt}, and \ending{-r}; see~\xref[1,
  2]{26}).

The \infix{-ā-} occurs in the Present Subjunctive of the Second,
Third, and Fourth Conjugations.

The \infix{-ē-} occurs in the Present Subjunctive of the First, and in
the Imperfect and Past Perfect Subjunctive of all Conjugations.

The \infix{-ī-} occurs in the Present Subjunctive of many Irregular
Verbs and in the Perfect Subjunctive of all Conjugations.

\begin{note}

The Latin Subjunctive represents in its formation, as well as in its
functions (see the Syntax), a mixture of two originally distinct
moods, namely, the Subjunctive proper and the Optative.  The
\infix{-ī-} is the mood-sign of the old Optative.  Another form of
this was \infix{-iē-}, seen in early Latin \latin{siem}, \latin{siēs},
etc., beside \latin{sim}, \latin{sīs}, etc.  The \infix{-ā-} and
\infix{-ē-} belong to the Subjunctive proper, and, besides the forms
enumerated, they are seen in the Future of the Third and Fourth
Conjugations (\xref[note]{172}).  Still another old Subjunctive
formation, with the simple thematic vowel, is seen in the Future
\latin{erō} (\xref[note]{172}), and in the Future Perfect
\ending{-erō}, \latin{-eris}, etc.

\end{note}

\subsubsection

In the Imperfect Subjunctive the mood-sign \infix{-ē-} is added to the
Present Stem +~\phone{s}, the \phone{s} becoming \phone{r} regularly
after a vowel (\xref{47}).  So \latin{es-s-ē-s} (also \latin{vellēs},
\latin{ferrēs}, from \rec{vel-s-ē-s}, \rec{fer-s-ē-s}; \xref[11]{49}),
but \latin{amā-r-ē-s}, \latin{tege-r-ē-s}, etc.

\subsubsection

In the Perfect Subjunctive the mood-sign \infix{-ī-} is added to the
Perfect Stem~+ \infix{-er-} (originally \infix{-es-}).  The frequently
occurring Second Singular in \ending{-is} beside the normal
\ending{-īs}, and the rare \ending{-imus} for \ending{-īmus}, are due
to confusion with the Future Perfect.  See~\xref[6]{164};
\xref[2]{174}.

\subsubsection

In the Past Perfect Subjunctive the mood-sign \infix{-ē-} is added to
\infix{-is-s-}; e.g.\ \latin{tēx-is-s-ē-s}.

\section[The Imperative]

The Imperative has no special mood-sign, and is characterized only by
its peculiar endings.

\section[The Passive]

The formation of the Moods and Tenses is the same as in the Active,
except in the Perfect System, which is periphrastic.

\headingC{The Infinitive}

\section
\subsection

The Suffix of the Present Infinitive Active is \suffix{-se}, which is
preserved in \latin{es-se} (also \latin{velle}, \latin{ferre}, from
\rec{vel-se}, \rec{fer-se}; \xref[11]{49}), but which became
\suffix{-re} after a vowel (\xref{47}).  So \latin{amā-re},
\latin{tege-re}, etc.  The Perfect Infinitive Active also has
\suffix{-se}, which in this case is added to the Perfect
Stem~+~\infix{-is-}; e.g.\ \latin{amāv-is-se}, \latin{tēx-is-se}, etc.

\subsection

The Present Infinitive Passive has \suffix{-rī} in all Conjugations
except the Third, where the ending is simply~\suffix{-ī}.  So
\latin{amā-rī}, \latin{monē-rī}, \latin{audī-rī}, but \latin{teg-ī}.
With the addition of an \suffix{-er} (of doubtful origin), and the
regular shortening of the long vowel before another vowel (\xref{21}),
arose the variant forms \latin{laudārier}, \latin{dīcier},
etc. (\xref[3]{164}).

\subsection

The other Infinitives are periphrastic, the Perfect Passive Infinitive
being formed from the Perfect Passive Participle with \latin{esse},
the Future Active Infinitive from the Future Participle with
\latin{esse}, and the Future Passive Infinitive from the Supine with
\latin{īrī} (Pres.\ Infin.\ Pass.\ of \latin{eō}, used impersonally
like \latin{ītur}, etc., but not occurring separately).

\begin{note}

Infinitives are, in origin, case-forms which have become associated
with the verb-system.

\end{note}

\headingC{The Perfect Passive Participle}

\section

The Perfect Passive Participle is formed with the suffix
\suffix{-to-}, and is declined like an Adjective of the First and
Second Declensions.  As regards the form of the stem to which the
suffix is added, there is a certain relationship between the formation
of this Participle and that of the Perfect Indicative, as follows:

\subsection

\suffix{-ātus}, \suffix{-ītus}.  Such are the forms for nearly all
verbs which have Perfects in \suffix{-āvī}, \suffix{-īvī}, as:
\begin{Tabular}{l@{\,\,}c@{\,\,}l@{\,\,}c@{\,\,}l}

\latin{amātus}  & beside & \latin{amāvī},
& from & \latin{amō}, \english{love}, \latin{amāre} \\

\latin{audītus} & \ditto & \latin{audīvī}
& \ditto & \latin{audiō}, \english{hear}, \latin{audīre}

\end{Tabular}

\begin{minor}

\subsubsection

Exceptions are: \latin{pōtus} (but also \latin{pōtātus}) beside
\latin{pōtāvī}, from \latin{pōtō}, \english{drink}, \latin{pōtāre};
\latin{sepultus} beside \latin{sepelīvī}, from \latin{sepeliō},
\english{bury}, \latin{sepelīre}.

\end{minor}

\subsection

\suffix{-ūtus}. This is the regular formation for Verbs in
\suffix{-uō} (\suffix{-vō}), as
\begin{Tabular}{l@{\,\,}c@{\,\,}l@{\,\,}c@{\,\,}l}

\latin{minūtus} & from & \latin{minuō}, \english{lessen},
\latin{minuere}, \latin{minuī}.

\end{Tabular}

\begin{minor}

\subsubsection

So also \latin{secūtus}, \latin{locūtus}, from the two Deponents in
\suffix{-quor}, \latin{sequor}, \english{follow}, and \latin{loquor},
\english{speak}.  Compounds of \latin{ruō}, \english{fall}, have
\suffix{-rutus}.

\end{minor}

\subsection

\latin{-itus}. This is the usual formation for Verbs having Perfects
of the \suffix{-uī} type, as:
\begin{Tabular}{l@{\,\,}c@{\,\,}l@{\,\,}c@{\,\,}l}

\latin{monitus} & beside & \latin{monuī},
& from   & \latin{moneō}, \english{advise}, \latin{monēre} \\

\latin{molitus} & \ditto & \latin{moluī}
& \ditto & \latin{molō}, \english{grind}, \latin{molere} \\

\latin{domitus} & \ditto & \latin{domuī}
& \ditto & \latin{domō}, \english{tame}, \latin{domāre}

\end{Tabular}

\subsubsection

But there are a few exceptions, e.g.:
\begin{Tabular}{l@{\,\,}c@{\,\,}l@{\,\,}c@{\,\,}l}

doctus & beside & docuī, & from   & doceō, \english{teach}, \latin*{docēre}\\
cultus & \ditto & coluī  & \ditto & colō, \english{cultivate}, \latin*{colere}\\
sectus & \ditto & secuī  & \ditto & secō, \english{cut}, \latin*{secāre}

\end{Tabular}

\pagebreak

\subsection

\suffix{-tus} (\suffix{-sus}) added directly to the root-syllable.
This formation is common to all Verbs with Perfects of other types
than those already mentioned.  The consonant changes follow the
statements given in~\xref{49}.  Examples:
\begin{Tabular}{l@{\,\,}c@{\,\,}l@{\,\,}c@{\,\,}l}

\latin{flētus}   & beside & \latin{flē-vī},
& from   & \latin{fleō}, \english{weep}, \latin{flēre}\\

\latin{scrīptus} & \ditto & \latin{scrīpsī}
& \ditto & \latin{scrībō}, \english{write}, \latin{scrībere}\\

\latin{morsus}   & \ditto & \latin{momordī}
& \ditto & \latin{mordeō}, \english{bite}, \latin{mordēre}\\

\latin{factus}   & \ditto & \latin{fēcī}
& \ditto & \latin{faciō}, \english{make}, \latin{facere}\\

\latin{fissus}   & \ditto & \latin{fidī}
& \ditto & \latin{findō}, \english{split}, \latin{findere}

\end{Tabular}

\begin{note}

The form in \suffix{-sus} is regular for all roots ending in dentals,
as \latin{fissus}, etc.\ (\xref[5]{49}); but, after the analogy of such
forms, \suffix{-sus} came to be used also in a number of verbs with roots not
ending in a dental; e.g.:
\begin{Tabular}{l@{\extracolsep{.3333em}}c@{\extracolsep{.3333em}}l
                      @{\extracolsep{4em}}
                      l@{\extracolsep{.3333em}}c@{\extracolsep{.3333em}}l}

  \latin{lāpsus}, & from & \latin{lābor}, \english{slip}
& \latin{pulsus}, & from & \latin{pellō}, \english{drive} \\

  \latin{mulsus}  & \ditto & \latin{mulgeō}, \english{milk}
& \latin{cēnsus}  & \ditto & \latin{cēnseō}, \english{think} \\

  \latin{fīxus}   & \ditto & \latin{fīgō}, \english{fix}
& \latin{amplexus}& \ditto & \latin{amplector}, \english{embrace}

\end{Tabular}

\end{note}

\section

Vowel Changes.  In general, barring the regular lengthening before
\phone{ns} and \phone{nct} (\xref{18}), the quantity of the vowel of
the root-syllable is the same in the Perfect Participle as in the
Present.  But there are some examples of an inherited variation
(\xref{46}).  Besides such cases as \latin{satus} (\latin{serō},
Perf.\ \latin{sē-vī}), \latin{strātus} (\latin{sternō},
Perf.\ \latin{strā-vī}), \latin{mōtus} (\latin{moveō},
Perf.\ \latin{mōvī}), etc., there are also differences among verbs
coming from roots ending in a mute, as follows:

\subsection

A short vowel, as against a long vowel in the Present, is seen in
\latin{cessus} (cf.\ Perf.\ \latin{cessī}), from \latin{cēdō};
\latin{ustus} (cf.\ Perf.\ \latin{ussī}), from \latin{ūrō}; and \latin{dictus},
\latin{ductus}, from \latin{dīcō}, \latin{dūcō} (Perf.\ also \latin{dīxī}, \latin{dūxī}).

\subsection

A long vowel, as against a short vowel in the Present, is seen in
\latin{cāsūrus}, \latin{dīvīsus}, \latin{fūsus}, \latin{ēsus},
\latin{ēmptus}, \latin{āctus}, \latin{lēctus}, \latin{rēctus},
\latin{tēctus}, \latin{flūxus} (Adj.), \latin{strūctus},
\latin{frāctus}, \latin{pāctus}, \latin{tāctus}.

\begin{note}

With the exception of \latin{cāsūrus} and \latin{tāctus} (with which
compare \latin{con-tāg-iō}), these long-vowel forms are accompanied by
Perfects with a long vowel, as \latin{fūdī}, \latin{lēgī},
\latin{rēxī}, etc.\ (though in some it is a different long vowel,
e.g.\ \latin{ēgī}, beside \latin{āctus}, etc.).  But it does not
follow that a long-vowel Perfect is always accompanied by a long-vowel
Participle.  Cf.\ \latin{mīsī}, but \latin{missus}; \latin{sēdī}, but
\latin{sessum} (Supine); \latin{fōdī}, but
\latin{fossus}; \latin{rūpī}, but \latin{ruptus}; \latin{fēcī}, but
\latin{factus}, etc.

It is also true that the Participles mentioned come from roots ending
in a voiced mute, namely, \phone{d} or~\phone{g} (for \latin{flūxus},
\latin{strūctus}, see~\xref[2]{49}).  But among verbs from roots in a
voiced mute there are also some that have a short vowel;
e.g. \latin{fossus} from \latin{fodiō}; \latin{sessum} from
\latin{sedeō}; \latin{strictus} from \latin{stringō}.

\end{note}

\section[The Supine]

This is formed in the same way as the Perfect Passive Participle, but
with the suffix \suffix{-tu-}, not \suffix{-to-}.  It is, then, a Verbal
Noun of the Fourth Declension, of which the Accusative and Ablative
only are in common use.

\begin{note}
The Dative form in \suffix{-tuī} is rare.
\end{note}

\section[The Future Active Participle]

This is formed with the suffix \suffix{-tūro-}, and is declined like an
Adjective of the First and Second Declensions.  As regards the form of
the stem to which the suffix is added, it usually follows the
formation of the Perfect Passive Participle; e.g.\ \latin{amātūrus},
like \latin{amātus}.  But there are occasional variations, as:
\begin{Tabular}{l@{\,\,}c@{\,\,}l@{\,\,}c@{\,\,}l}

\latin{moritūrus} & beside & \latin{mortuus},
& from & \latin{morior}, \english{die}, \latin{morī} \\

\latin{oritūrus}  & \ditto & \latin{ortus}
& \ditto & \latin{orior}, \english{arise}, \latin{orīrī} \\

\latin{paritūrus} & \ditto & \latin{partus}
& \ditto & \latin{pariō}, \english{bring forth}, \latin{parere} \\

\latin{ruitūrus} & \ditto & \latin{-rutus}
& \ditto & \latin{ruō}, \english{tumble down}, \latin{ruere}

\end{Tabular}

\section[The Present Active Participle]

This is formed with the suffix \suffix{-nt-}, add\-ed to the Present
Stem, and is declined as an adjective of one ending (\xref{117}).  The
long vowels of the First, Second, and Fourth Conjugations are
regularly shortened (\xref[1]{20}).  The thematic vowel of the Third
Conjugation appears as \phone{e}, not \phone{o} (\phone{u});
e.g.\ Gen.\ \latin{legentis}, contrasted with \latin{legunt}, from
\rec{legont}, of the Present Indicative. But \latin{iēns}, from
\latin{eō}, \english{go}, has the stem \stem{eunt-} in all other
forms; e.g.\ Gen.\ \latin{euntis}.

\section[The Future Passive Participle and the Gerund]

These are formed with the suffix \suffix{-ndo-}, added to the Present
Stem, which appears in the same form as in the Present Participle.
The Future Passive Participle is declined as an Adjective of the First
and Second Declensions.  The Gerund is the Neuter Singular of this,
lacking the Nominative and Vocative.

\begin{note}

The forms in \suffix{-undus}, \suffix{-iundus} of the Third and Fourth
Conjugations (\xref[2]{164}) represent what was probably the original
type in these conjugations, the forms in \suffix{-endus},
\suffix{-iendus} having arisen under the influence of the Present
Participles, which have \suffix{-ent-}.  From \latin{eō},
\english{go}, the Gerund is always \latin{eundī}, corresponding to the
Participle \latin{euntis}.

\end{note}

\chapter*{Illustrations of the Various Types of Verbs}

\contentsentry{B}{Illustrations of the Various Types of Verbs}

\begin{minor}

\section

The Principal Parts of any given Verb are found most conveniently by
reference to a single list arranged alphabetically, such as the
Catalogue of Verbs given at the end of this grammar, or to a lexicon.
The following list is merely illustrative, showing what different
combinations of Present, Perfect, and Participial Stems occur, and
which are the most common.

The types which are most common are given in \emph{black letters}, the
others in \emph{Roman}.  Where \emph{small letters} are used, it is to
be understood that \emph{all} examples of the type are given.  A
prefixed hyphen indicates that the form occurs only in compounds.  For
all details, such as variant forms, etc., see the \emph{Catalogue of
  Verbs}.

\end{minor}

\section
\subtitle{\textsc{First Conjugation}}

\subsection
Perfect in \suffix{-āvī}.
\begin{vexamples}

\latin{amō}, \english{love}
& \latin{amāre}
& \latin{amāvī}
& \latin{amātum}
\setrowsize{\footnotesize}\\

pōtō, \english{drink}
& pōtāre
& pōtāvī
& pōtum (pōtātum)

\end{vexamples}

\subsection

Perfect in \suffix{-uī}.
\begin{vexamples}
domō, \english{tame} & domāre & domuī & domitum
\end{vexamples}

\begin{minor}
Here also \latin{crepō}, \latin{cubō}, \latin{fricō}, \latin{micō},
\latin{-plicō}, \latin{secō}, \latin{sonō}, \latin{tonō},
\latin{vetō}.\footnote{But Perf.\ Pass.\ Partic.\ \latin{frictum},
  \latin{sectum}.  In this and the similar statements below, it is not
  meant that the verbs cited \emph{as belonging to the type specified
    in the heading} follow in every detail the example under which
  they stand.}
\end{minor}

\subsection

Perfect in \suffix{-ī} with lengthened vowel.
\begin{vexamples*}

iuvō, \english{help}
& iuvāre
& iūvī
& iūtum
\\

lavō, \english{wash}
& lavāre
& lāvī
& lautum, lōtum

\end{vexamples*}

\subsection

Reduplicated Perfect.
\begin{vexamples*}

stō, \english{stand}
& stāre
& stetī
& stātūrus

\end{vexamples*}

\subsection

Deponents.
\begin{Tabular*}{l@{\extracolsep{\fill}}ll}

\latin{mīror}, \english{wonder}
& \latin{mīrārī}
& \latin{mīrātus sum}

\end{Tabular*}

\section
\subtitle{\textsc{Second Conjugation}}

\subsection

Perfect in \suffix{-uī}.
\begin{vexamples}

\latin{moneō}, \english{advise}
& \latin{monēre}
& \latin{monuī}
& \latin{monitum}
\\

doceō, \english{teach}
& docēre
& docuī
& doctum
\\

egeō, \english{need}
& egēre
& eguī
& \na

\end{vexamples}

\subsection

Perfect in \suffix{-sī}.
\begin{vexamples}

augeō, \english{increase}
& augēre
& auxī
& auctum

\end{vexamples}

\subsection

Perfect in \suffix{-vī}.
\begin{vexamples}

fleō, \english{weep}
& flēre
& flēvī
& flētum

\end{vexamples}

\begin{minor}

Here also are \latin{neō}, \latin{-pleō}, \latin{dēleō},
\latin{aboleō}, \latin{cieō}.

\end{minor}

\subsection

Reduplicated Perfect.
\begin{vexamples}

mordeō, \english{bite}
& mordēre
& momordī
& morsum

\end{vexamples}

\begin{minor}

Here also \latin{pendeō}, \latin{spondeō}, \latin{tondeō}.

\end{minor}

\subsection

Perfect in \suffix{-ī} with lengthened vowel.
\begin{vexamples}

caveō, \english{take care}
& cavēre
& cāvī
& cautum

\end{vexamples}

\begin{minor}

Here also \latin{faveō}, \latin{foveō}, \latin{moveō}, \latin{paveō},
\latin{voveō}, \latin{sedeō}, \latin{videō}, of which all but the last
two end in \suffix{-veō}.

\end{minor}

\subsection

Perfect in \suffix{-ī} without lengthening.
\begin{vexamples*}

ferveō, \english{boil}
& fervēre
& fervī
& \na
\\

prandeō, \english{lunch}
& prandēre
& prandī
& prānsum
\\

strīdeō, \english{grate}
& strīdēre
& strīdī
& \na

\end{vexamples*}

\subsection
Deponents.
\begin{Tabular*}{l@{\extracolsep{\fill}}ll}

vereor, \english{fear}
& verērī
& veritus sum
\\

fateor, \english{confess}
& fatērī
& fassus sum

\end{Tabular*}

\section
\subtitle{\textsc{Third Conjugation}}

\psection

Simple Thematic Present.

\subsection

Perfect in \suffix{-sī}.
\begin{vexamples}

\latin{scrībō}, \english{write}
& \latin{scrībere}
& \latin{scrīpsī}
& \latin{scrīptum}
\\

\latin{dīcō}, \english{say}
& \latin{dīcere}
& \latin{dīxī}
& \latin{dictum}
\\

\latin{claudō}, \english{shut}
& \latin{claudere}
& \latin{clausī}
& \latin{clausum}

\end{vexamples}

\enlargethispage{\baselineskip}

\begin{minor}

With few exceptions, from roots ending in a mute.

\end{minor}

\subsection

Perfect in \suffix{-uī}.
\begin{vexamples}

molō, \english{grind}
& molere
& moluī
& molitum

\end{vexamples}

\begin{minor}

With few exceptions, from roots ending in a liquid or nasal.

\end{minor}

\begin{vexamples*}

metō, \english{mow}
& metere
& messuī
& messum

\end{vexamples*}

\subsection

Reduplicated Perfect.
\begin{vexamples}

cadō, \english{fail}
& cadere
& cecidī
& cāsūrus

\end{vexamples}

\begin{minor}

Here also \latin{caedō}, \latin{canō}, \latin{currō}, \latin{parcō},
\latin{pendō}, \latin{tendō}.

\end{minor}

\subsection

Perfect in \suffix{-ī} without lengthening.
\begin{vexamples}

vertō, \english{turn}
& vertere
& vertī
& versum

\end{vexamples}

\subsection

Perfect in \suffix{-ī} with lengthened vowel.
\begin{vexamples}

agō, \english{drive}
& agere
& ēgī
& āctus

\end{vexamples}

\begin{minor}

Here also \latin{edō}, \latin{emō}, \latin{legō}.

\end{minor}

\subsection

Perfect in \suffix{-īvī}.
\begin{vexamples*}

terō, \english{rub}
& terere
& trīvī
& trītum
\\

petō, \english{seek}
& petere
& petīvī (-iī)
& petītum
\\

quaerō, \english{seek}
& quaerere
& quesīvī
& quaesītum

\end{vexamples*}

\psection

Present with reduplication.
\begin{vexamples*}

sistō, \english{set}
& sistere
& stitī
& statum
\\

gignō, \english{beget}
& gignere
& genuī
& genitum
\\

serō, \english{sow}
& serere
& sēvī
& satum
\\

bibō, \english{drink}
& bibere
& bibī
& \na

\end{vexamples*}

\psection

Present with inserted nasal.

\subsection

Perfect in \suffix{-sī}.
\begin{vexamples}

iungō, \english{join}
& iungere
& iūnxī
& iūnctum

\end{vexamples}

\begin{minor}

So \latin{cingō}, \latin{lingō}, \latin{-mungō}, \latin{ninguit},
\latin{pangō}, \latin{plangō}, \latin{stinguō}, \latin{tinguō},
\latin{unguō}.

\end{minor}

\begin{vexamples}

fingō, \english{mould}
& fingere
& fīnxī
& fictum

\end{vexamples}

\begin{minor}

So \latin{mingō}, \latin{pingō}, \latin{stringō}.

\end{minor}

\subsection

Reduplicated Perfect.
\begin{vexamples}

tango, \english{touch}
& tangere
& tetigī
& tāctum
\end{vexamples}

\begin{minor}

Here also \latin{pangō}, \latin{pungō}, \latin{tundō}.

\end{minor}

\subsection

Perfect in \suffix{-ī} with lengthened vowel.
\begin{vexamples}

fundō, \english{pour}
& fundere
& fūdī
& fūsum

\end{vexamples}

\begin{minor}

Here also \latin{rumpō}, \latin{vincō}, \latin{linquō},
\latin{frangō}.

\end{minor}

\enlargethispage{\baselineskip}

\subsection

Perfect in \suffix{-ī} without lengthening.
\begin{vexamples*}

findō, \english{split}
& findere
& fidī
& fissum
\\

scindō, \english{rend}
& scindere
& scidī
& scissum
\\

pandō, \english{open}
& pandere
& pandī
& passum

\end{vexamples*}

\subsection

Perfect in \suffix{-uī}.
\begin{vexamples}

-cumbō, \english{recline}
& -cumbere
& -cubuī
& -cubitum

\end{vexamples}

\psection

Present in \suffix{-nō} (and \suffix{-llō} from \suffix{-lnō}).

\subsection

Perfect in \suffix{-vī}.
\begin{vexamples}
sternō, \english{strew}
& sternere
& strāvī
& strātum
\end{vexamples}
Here also \latin{spernō}, \latin{cernō}, \latin{linō}, \latin{sinō}.

\subsection

Reduplicated Perfect.
\begin{vexamples*}

pellō, \english{drive}
& pellere
& pepulī
& pulsum
\\

fallō, \english{deceive}
& fallere
& fefellī
& (falsus, Adj.)
\\

tollō, \english{raise}
& tollere
& (sus-tulī)
& (sub-lātum)

\end{vexamples*}

\subsection
Perfect in \suffix{-sī}.
\begin{vexamples*}

temnō, \english{scorn}
& temnere
& con-tempsī
& con-temptum

\end{vexamples*}

\psection

Present in \suffix{-tō}.
\begin{vexamples}

flectō, \english{bend}
& flectere
& flexī
& flexum

\end{vexamples}

\begin{minor}

So \latin{pectō}, \latin{plectō}, \latin{nectō} (but
Perf.\ \latin{nexuī} commoner than \latin{nexī}).

\end{minor}

\psection
Present in \suffix{-scō}.

\subsection
Primary Verbs.
\begin{vexamples}

discō, \english{learn}
& discere
& didicī
& \na
\\

crēscō, \english{grow}
& crēscere
& crēvī
& crētum
\\

nōscō, \english{get to know}
& nōscere
& nōvī
& (nōtus, Adj.)

\end{vexamples}

\subsection

Inchoatives in \suffix{-ēscō}.
\begin{vexamples}

\latin{calēscō}, \english{become hot}
& \latin{calēscere}

& \latin{caluī}\footnote{The Perfect of Inchoatives which are derived
  from Verbs is the same as that of the simple verbs; in the case of
  Inchoatives derived from Nouns or Adjectives, it follows the type
  which is commonest in those derived from verbs, namely,
  \suffix{-uī} for Presents in \suffix{-ēscō},
  \suffix{-īvī} for Presents in \suffix{-īscō},
  \suffix{-āvī} for Presents in \suffix{-āscō}.
  The Participial Stem is nearly always lacking.}

& \na\footnotemark[\thefootnote] (from \latin{caleō})
\\

\latin{dūrēscō}, \english{become hard}
& \latin{dūrēscere}
& \latin{dūruī}\footnotemark[\thefootnote]
& \na\footnotemark[\thefootnote] (from \latin{dūrus})
\\

algēscō, \english{catch cold}
& algēscere
& alsī\footnotemark[\thefootnote]
& \na\footnotemark[\thefootnote] (from \latin{algeō})

\end{vexamples}

\subsection

Inchoatives in \suffix{-īscō}.
\begin{vexamples}

-dormīscō, \english{fall asleep}
& -dormīscere
& -dormīvī\footnotemark[\thefootnote]
& \na\footnotemark[\thefootnote] (from dormiō)

\end{vexamples}

\subsection

Inchoatives in \suffix{-āscō}.
\begin{vexamples}

vesperāscō, \english{become evening}\quad
& vesperāscere
& vesperāvī\footnotemark[\thefootnote]
& \na\footnotemark[\thefootnote] (from vesper)

\end{vexamples}

\psection

Present in \suffix{-essō}.
\begin{vexamples}

capessō, \english{seize}
& capessere
& capessīvī
& capessītum

\end{vexamples}

\psection

Present in \suffix{-uō}.

\subsection

Perfect in \suffix{-uī}.
\begin{vexamples*}

statuō, \english{set up}
& statuere
& statuī
& statūtum
\\

ruō, \english{fall}
& ruere
& ruī
& ruitūrus

\end{vexamples*}

\subsection

Perfect in \suffix{-xī}.
\begin{vexamples*}

struō, \english{build}
& struere
& strūxī
& strūctum
\\

fluō, \english{flow}
& fluere
& flūxī
& (flūxus, Adj.)

\end{vexamples*}

\psection

Present in \suffix{-iō}.

\subsection
Perfect in \suffix{-ī} with lengthened vowel.
\begin{vexamples}

capiō, \english{take}
& capere
& cēpī
& captum

\end{vexamples}

\begin{minor}

Here also \latin{faciō}, \latin{iaciō}, \latin{fodiō},
\latin{fugiō}.

\end{minor}

\subsection
Perfect in \suffix{-sī}.
\begin{vexamples*}

-spiciō, \english{see}
& -spicere
& -spexī
& -spectum
\\

-liciō, \english{allure}
& -licere
& -lexī
& -lectum
\\

quatiō, \english{shake}
& quatere
& -cussī
& quassum

\end{vexamples*}

\subsection
Perfect in \suffix{-īvī}.
\begin{vexamples*}

cupiō, \english{wish}
& cupere
& cupīvī
& cupītum
\\

sapiō, \english{be wise}
& sapere
& sapīvī
& \na

\end{vexamples*}

\subsection
Perfect in \suffix{-uī}.
\begin{vexamples*}

rapiō, \english{seize}
& rapere
& rapuī
& raptum

\end{vexamples*}

\subsection
Reduplicated Perfect.
\begin{vexamples*}

pariō, \english{bring forth}
& parere
& peperī
& partum

\end{vexamples*}

\psection

Deponents.
\begin{Tabular*}{l@{\extracolsep{\fill}}ll}

ūtor, \english{use}
& ūtī
& ūsus sum (cf.\ \emph{A})
\\

fungor, \english{perform}
& fungī
& fūnctus sum (cf.\ \emph{C})
\\

amplector, \english{embrace}
& amplectī
& amplexus sum (cf.\ \emph{E})
\\

vēscor, \english{feed}
& vēscī
& \na\ (cf.\ \emph{F}, 1)
\\

oblīvīscor, \english{forget}
& oblīvīscī
& oblītus sum (cf.\ \emph{F}, 3)
\\

sequor, \english{follow}
& sequī
& secūtus sum (cf.\ \emph{H})
\\

gradior, \english{step}
& gradī
& gressus sum (cf.\ \emph{I})

\end{Tabular*}

\section
\subtitle{\textsc{Fourth Conjugation}}

\subsection

Perfect in \suffix{-īvī}.
\begin{vexamples}

\latin{audiō}, \english{hear}
& \latin{audīre}
& \latin{audīvī}
& \latin{audītum}
\setrowsize{\footnotesize}\\

sepeliō, \emph{bury}
& sepelīre
& sepelīvī
& sepultum

\end{vexamples}

\subsection
Perfect in \suffix{-sī}.
\begin{vexamples}
vinciō, \english{bind}
& vincīre
& vīnxī
& vīnctum
\end{vexamples}

\subsection
Perfect in \suffix{-uī}.
\begin{vexamples*}

aperiō, \english{open}
& aperīre
& aperuī
& apertum
\\

operiō, \english{cover}
& operīre
& operuī
& opertum
\\

saliō, \english{leap}
& salīre
& saluī
& \na

\end{vexamples*}

\subsection

Reduplicated Perfect (originally).
\begin{vexamples*}
reperiō, \english{find}
& reperīre
& repperī
& repertum
\\

comperiō, \english{learn}
& comperīre
& comperī
& compertum

\end{vexamples*}

\subsection

Perfect in \suffix{-ī} with lengthened vowel.
\begin{vexamples*}

veniō, \english{come}
& venīre
& vēnī
& ventum

\end{vexamples*}

\subsection
Deponents.
\begin{Tabular*}{l@{\extracolsep{\fill}}ll}

partior, \english{divide}
& partīrī
& partītus sum
\\

ōrdior, \english{begin}
& ōrdīrī
& ōrsus sum

\end{Tabular*}

\chapter{Irregular Verbs}

\contentsentry{B}{Irregular Verbs}

For the character of Irregular Verbs in general, see~\xref{170}.

\headingC{Compounds of Sum}

\section

\latin{Adsum}, etc.  Most compounds of \latin{sum} follow the
conjugation of the simple verb, which has been given above
(\xref{153}).  So \latin{ad-sum}, \latin{ad-esse}, \latin{ad-fuī},
etc.  The Present Participle, which is wanting in the simple verb,
appears in the adjectival \latin{absēns}, \english{absent}, from
\latin{ab-sum}, and \latin{praesēns}, \english{present}, from
\latin{prae-sum}.  In \latin{prō-sum} the preposition appears as
\latin{prōd} before a vowel (\xref[2]{24});
e.g.\ Pres.\ Indic.\ \latin{prōsum}, \latin{prōdes}, \latin{prōdest},
\latin{prōsumus}, \latin{prōdestis}, \latin{prōsunt};
Imperf.\ Indic.\ \latin{prōderam}; Imperf.\ Subj.\ \latin{prōdessem};
Fut.\ \latin{prōderō}; Pres.\ Infin.\ \latin{prōdesse};
Imperat.\ \latin{prōdes}, \latin{prōdestō}, etc.

\section

\latin{Possum}, \english{be able}.  The Present System is based on a
union of \latin{potis} or \latin{pote}, \english{able}, with
\latin{sum}; the Perfect System does not contain \latin{sum},
but belongs to a Present \rec{poteō}, \rec{potēre}, of which only the
participial form \latin{potēns} is in use.

\begin{Tabular}{>{\itshape}ll@{\qquad}c@{\qquad}l}

\cc{4}{\textbf{Principal Parts}} \\

& \latin{possum} & \latin{posse} & \latin{potuī} \\

& \textbfsc{indicative} & & \textbfsc{subjunctive} \\

Pres.   & possum    && possim \\
        & potes     && possīs \\
        & potest    && possit \\[\smallskipamount]
        & possumus  && possīmus \\
        & potestis  && possītis \\
        & possunt   && possint \\

Imperf. & poteram\footnote{That is, \latin{poteram}, \latin{poterās},
  \latin{poterat}, etc.  Similarly elsewhere.}
       && possem \\

Fut.    & poterō \\

Perf.   & potuī     && potuerim \\

Past. Perf.
        & potueram  && potuissem \\

Fut. Perf. & potuerō \\[\medskipamount]

& \textbfsc{infinitive} & & \textbfsc{participle} \\

Pres.   & posse     && potēns (Adj.) \\

Perf.   & potuisse

\end{Tabular}

\begin{minor}

\subsubsection

Early Latin shows the uncompounded forms \latin{potis sum} or
\latin{pote sum}, \latin{potis est} or \latin{pote est}, etc.; also
Infin.\ \latin{potesse}.

\end{minor}

\pagebreak

\subtitle{\latin{Volō}, \english{wish}, \textbf{and its Compounds}}

\section
\subtitle{\textbf{Principal Parts}}

\begin{Tabular}{lll}

\latin{volō}, \english{wish}
& \latin{velle}
& \latin{voluī}
\\

\latin{nolō}, \english{be unwilling}
& \latin{nōlle}
& \latin{nōluī}
\\

\latin{malō}, \english{prefer}
& \latin{mālle}
& \latin{māluī}

\end{Tabular}

\begin{Tabular*}{l@{\extracolsep{\fill}}lllll}

\cc{6}{\emph{Present}}\\

  \textbfsc{indic.}
& \textbfsc{subj.}
& \textbfsc{indic.}
& \textbfsc{subj.}
& \textbfsc{indic.}
& \textbfsc{subj.}
\\

volō    & velim   & nōlō       & nōlim   & mālō     & mālim \\
vīs     & velīs   & nōn vīs    & nōlīs   & māvīs    & mālīs \\
vult    & velit   & nōn vult   & nōlit   & māvult   & mālit \\[\smallskipamount]
volumus & velīmus & nōlumus    & nōlīmus & mālumus  & mālīmus \\
vultis  & velītis & nōn vultis & nōlītis & māvultis & mālītis \\
volunt  & velint  & nōlunt     & nōlint  & mālunt   & mālint \\[\medskipamount]

\cc{6}{\emph{Imperfect}} \\

  volēbam & vellem
& nōlēbam & nōllem
& mālēbam & māllem \\[\medskipamount]

\cc{6}{\emph{Future}} \\

  volam &
& nōlam &
& mālam \\[\medskipamount]

\cc{6}{\emph{Perfect}} \\

  voluī & voluerim
& nōluī & nōluerim
& māluī & māluerim \\[\medskipamount]

\cc{6}{\emph{Past Perfect}} \\

  volueram & voluissem
& nōlueram & nōluissem
& mālueram & māluissem \\[\medskipamount]

\cc{6}{\emph{Future Perfect}} \\

  voluerō &
& nōluerō &
& māluerō

\end{Tabular*}

\begin{Tabular}{>{\itshape}lll}

\cc{3}{\textbfsc{imperative}} \\

Pres. & nōlī   & nōlīte \\
Fut.  & nōlītō & nōlītōtē

\end{Tabular}

\begin{Tabular}{>{\itshape}llll}

\cc{4}{\textbfsc{infinitive}} \\

Pres.   & velle     & nōlle     & mālle \\
Perf.   & voluisse  & nōluisse  & māluisse\\[\medskipamount]

\cc{4}{\textbfsc{participle}} \\

Pres.   & volēns    & nōlēns

\end{Tabular}

\subsubsection

For \latin{vult}, \latin{vultis}, the forms in use down to the
Augustan period were \latin{volt}, \latin{voltis} (\xref[1]{44}).  For
\latin{sī vīs} a contracted form \latin{sīs} is common, and, in early
Latin, \latin{sultis} is used for \latin{sī voltis} (\latin{sultis} is
from \rec{soltis}, which arose beside \latin{sī voltis}, not by
contraction, but after the analogy of the relation of \latin{vīs} to
\latin{sī vīs}.

% \pagebreak

\subsubsection

For \latin{nōn vīs} and \latin{nōn vult}, early Latin has
\latin{nevīs}, \latin{nevolt}.

\subsubsection

For \latin{mālō}, \latin{mālim}, etc., the early forms are
\latin{māvolō}, \latin{māvelim}, and these are compounds of
\latin{magis} and \latin{volō} (\latin{māvolō} probably from
\rec{mag(i)s-volō}; cf.\ \latin{sēvirī}, \xref[12]{49}).

\section
\subtitle{\latin{Ferō}, \english{bear}}

\begin{pparts}

\latin{ferō} & \latin{ferre} & \latin{tulī} & \latin{lātum}

\end{pparts}

\enlargethispage{\baselineskip}

\begin{paradigm}

  \cc{2}{\textsc{Active}}
& \cc{2}{\textsc{Passive}} \\

  \textbfsc{indic.}
& \textbfsc{subj.}
& \textbfsc{indic.}
& \textbfsc{subj.}\\[\smallskipamount]

\cc{4}{\emph{Present}} \\[\smallskipamount]

ferō & feram & feror & ferar \\

fers & ferās & ferris & ferāris, -re \\

fert & ferat & fertur & ferātur \\[\smallskipamount]

ferimus & ferāmus & ferimur  & ferāmur  \\
fertis  & ferātis & feriminī & ferāminī \\
ferunt  & ferant  & feruntur & ferantur \\[\medskipamount]

\cc{4}{\emph{Imperfect}} \\[\smallskipamount]

ferēbam & ferrem & ferēbar & ferrer \\[\medskipamount]

\cc{4}{\emph{Future}} \\[\smallskipamount]

feram   &        & ferar \\[\medskipamount]

\cc{4}{\emph{Perfect}} \\[\smallskipamount]

tulī    & tulerim   & lātus sum & lātus sim \\[\medskipamount]

\cc{4}{\emph{Past Perfect}} \\[\smallskipamount]

tuleram & tulissem  & lātus eram    & lātus essem \\[\medskipamount]

\cc{4}{\emph{Future Perfect}} \\[\smallskipamount]

tulerō  &           & lātus erō

\end{paradigm}

\begin{imperative*}

\cc{5}{\textbfsc{imperative}} \\

Pres.   & fer   & ferte     & ferre     & feriminī \\
Fut.    & fertō & fertōte   & fertor \\
        & fertō & feruntō   & fertor    & feruntor

\end{imperative*}

\begin{Tabular}{>{\itshape}ll>{\itshape}ll}

\cc{4}{\textbfsc{infinitive}} \\

Pres.   & ferre         & & ferrī \\
Perf.   & tulisse       & & lātus esse \\
Fut.    & lātūrus esse  & & lātum īrī \\[\medskipamount]

\pagebreak

\cc{4}{\textbfsc{participle}} \\

Pres.   & ferēns    & Perf. & lātus \\
Fut.    & lātūrus   & Fut.  & ferendus\\[\medskipamount]

\cc{4}{\textbfsc{gerund}} \\
\cc{4}{ferendī, etc.}

\end{Tabular}

\begin{minor}

\subsubsection

The earlier form of the Perfect is \latin{tetulī} (cf.\ also
\latin{rettulī}, see~\xref[1]{43}); the root is the same as in
\latin{tollō}; \latin{lātus} is for \rec{tlātus}, \latin{tlā} being
another form of the same root.

\end{minor}

\section
\subtitle{\latin{Eō}, \english{go}}

\begin{pparts}
\latin{eō} & \latin{īre} & \latin{iī} & \latin{itum}
\end{pparts}

\begin{Tabular}{>{\itshape}lll@{\qquad}ll}

& \M{2}{l}{\textbfsc{indicative}}
& \M{2}{l}{\textbfsc{subjunctive}} \\

Pres.   & eō    & īmus      & eam   & eāmus \\
        & īs    & ītis      & eās   & eātis \\
        & it    & eunt      & eat   & eant \\[\smallskipamount]

Imperf. & ībam  &           & īrem \\

Fut.    & ībō   \\

Perf.   & iī    & iimus     & ierim \\
        & īstī (iistī) & īstis (iistis) \\
        & iit, īt   & iērunt, -ēre \\[\smallskipamount]

Past Perf.
        & ieram &           & īssem \\

Fut.\ Perf.
        & ierō

\end{Tabular}

\begin{Tabular}{>{\itshape}l@{\enskip}ll
                     @{\quad}
                 >{\itshape}l@{\enskip}l
                     @{\quad}
                 >{\itshape}l@{\enskip}l}

\M{3}{l}{\textbfsc{imperative}}
& \multicolumn{2}{l}{\textbfsc{infinitive}}
&& \textbfsc{participle}
\\

  Pres. & ī     & īte
& Pres. & īre
& Pres. & iēns, Gen. euntis, etc.\\

  Fut.  & ītō   & ītōte
& Perf. & īsse (iisse)
& Fut.  & itūrus \\

        & ītō   & euntō
& Fut.  & itūrus esse
& Fut. Pass.
        & eundum (est) \\[\smallskipamount]

\cc{7}{\textbfsc{gerund}} \\

\cc{7}{eundī, etc.} \\

\end{Tabular}

\begin{minor}

\subsubsection

In the Perfect System, forms with \phone{v}, as \latin{īvī},
\latin{īveram}, etc., are rare, except in late writers.  Compounds
often have \latin{iistī}, \latin{iistis}, etc., for \latin{īstī},
\latin{īstis}, etc.

\subsubsection

The Passive is used only impersonally; e.g.\ \latin{ītur},
\latin{ībātur}, \latin{itum est}, etc.  But compounds with transitive
meaning have the full Passive; e.g.\ \latin{adeor}, \latin{adīris},
etc., from \latin{adeō}, \english{approach}.

\end{minor}

\subsubsection

\latin{Queō}, \english{can}, follows the conjugation of \latin{eō},
except that the Perfect is \latin{quīvī}.

\section

The verb \latin{fīō}, \english{become}, serves as the Passive of
\latin{faciō}, \english{make}, in the Present System.  The Perfect
System and the Future Passive Participle are formed regularly from
\latin{faciō}.

\pagebreak

\begin{pparts}

\latin{fīō}\footnote{The student should remember that the \phone{i} is
long throughout, except in the Third Singular and before short~\phone{e}.}
& \latin{fierī}
& \latin{factus sum}

\end{pparts}

\enlargethispage{\baselineskip}

\begin{Tabular}{>{\itshape}lll@{\qquad}ll}

& \M{2}{l}{\textbfsc{indicative}}
& \M{2}{l}{\textbfsc{subjunctive}} \\

Pres.   & fīō   & \na       & fīam   & fīāmus \\
        & fīs   & \na       & fīās   & fīātis \\
        & fit   & fīunt     & fīat   & fīant \\

Imperf. & fīēbam &          & fierem \\

Fut.    & fīam \\

Perf.   & factus sum &      & factus sim \\

Past Perf.
        & factus eram &     & factus essem \\

Fut.\ Perf.
        & factus erō

\end{Tabular}

\begin{Tabular}{>{\itshape}l@{\enskip}ll
                 @{\qquad}>{\itshape}l@{\enskip}l
                 @{\qquad}>{\itshape}l@{\enskip}l}

\M{3}{l}{\textbfsc{imperative}}
& \M{2}{@{}l}{\textbfsc{infinitive}}
& \M{2}{@{}l}{\textbfsc{participle}}
\\

  Pres. & fī    & fīte
& Pres. & fierī \\

        & &
& Perf. & factus esse
& Perf. & factus \\

        & &
& Fut.  & factum īrī
& Fut.  & faciendus

\end{Tabular}

\subsubsection

Prepositional compounds of \latin{faciō} usually have the regular
Passive; e.g.\ \latin{cōnficō}, Pass.\ \latin{cōnficior}.  But
compounds of \latin{fīō} also occur in some forms;
e.g.\ \latin{cōnfit}, \english{happens}, \latin{dēfit},
\english{lacks}, \latin{īnfit}, \english{begins}.  For the Passive of
compounds like \latin{benefaciō}, \latin{calefaciō}, etc.,
\latin{benefīō}, \latin{calefīō}, etc., are used.

\headingC{Present System of \latin{Edō}, \english{eat}}

\section

The Present System of \latin{edō} contains several forms in which the
endings are added directly to the root in the form \root{ēd-}.  The
Perfect System is regular.

\begin{pparts}

\latin{edō} & \latin{ēsse} & \latin{ēdī} & \latin{ēsum} \\

\end{pparts}

\enlargethispage{\baselineskip}

\begin{Tabular}{>{\itshape}l@{\enskip}ll@{\qquad}ll}

\cc{5}{\textsc{Active}} \\

        & \M{2}{@{}l}{\textbfsc{indicative}}
        & \M{2}{@{}l}{\textbfsc{subjunctive}} \\

Pres.   & edō       & edimus    & edim, edam    & edīmus, edāmus \\
        & ēs        & ēstis     & edīs, edās    & edītis, edātis \\
        & ēst       & edunt     & edit, edat    & edint, edant \\%[\smallskipamount]

Imperf. & \M{2}{@{}l}{edēbam}   & ēssem \\
Fut.    & edam \\%[\bigskipamount]

& \M{2}{@{}l}{\textbfsc{imperative}}
& \textbfsc{infinitive}
& \textbfsc{participle} \\

Pres.   & ēs   & ēste       & ēsse          & edēns \\
Fut.    & ēstō & ēstōte     & ēsūrus esse   & ēsūrus \\
        & ēstō & edunto

\end{Tabular}

\begin{Tabular}{>{\itshape}l@{\quad}>{\itshape}l@{\enskip}l}

Pres.\ Indic.\  & 3.\ Sing. & ēstur \\
Imperf.\ Subj.\ & 3.\ Sing. & ēssētur

\end{Tabular}

\begin{minor}

\subsubsection

The Subjunctive forms \latin{edim}, etc., which go with the Indicative
forms \latin{ēs}, \latin{ēst}, just as \latin{sim}, etc., with
\latin{es}, \latin{est} (\xref{175}), were almost exclusively employed
until well into the Augustan period.  Horace has only \latin{edim},
but Ovid \latin{edam}.

\subsubsection

Indicative forms \latin{edis}, \latin{edit}, \latin{editis}, following
the regular type, are not found until late times.

\subsubsection

\latin{Comedō} has a Perfect Passive Participle \latin{comēstus}
beside \latin{comēsus}.

\end{minor}

\headingC{Present System of \emend{7}{\latin{Do}}{\latin{Dō}},
  \english{give}}

\section

The Present System of \latin{dō} differs from that of verbs of the
First Conjugation only in having the \phone{a} short everywhere except
in the Second Singular of the Present Indicative and Present
Imperative, and, of course, the Nom.\ Sing.\ of the Present
Participle.  The Perfect System is regular.

\begin{pparts}

\latin{dō} & \latin{dare} & \latin{dedī} & \latin{datum}

\end{pparts}

\begin{Tabular}{>{\itshape}lll@{\qquad\qquad}ll}

\cc{5}{\textsc{Active}} \\

        & \M{2}{l}{\textbfsc{indicative}} & \M{2}{l}{\textbfsc{subjunctive}} \\

Pres.   & dō    & damus     & dem   & dēmus \\
        & dās   & datis     & dēs   & dētis \\
        & dat   & dant      & det   & dent \\[\smallskipamount]

Imperf. & dabam &           & darem \\

Fut.    & dabō \\[\smallskipamount]

& \M{2}{l}{\textbfsc{imperative}}
& \textbfsc{infinitive}
& \textbfsc{participle} \\

Pres.   & dā    & date      & dare          & dāns \\
Fut.    & datō  & datōte    & datūrus esse  & datūrus \\
        & datō  & dantō \\[\bigskipamount]

& \M{2}{l}{\textbfsc{gerund}} & \M{2}{l}{\textbfsc{supine}} \\

& \M{2}{l}{dandī, etc.} & \M{2}{l}{datum, datū} \\[\bigskipamount]

\cc{5}{\textsc{Passive}} \\

\cc{5}{datur, damur, etc.; dabar, dabor, darer, datus, dandus, etc.}

\end{Tabular}

\begin{minor}

\subsubsection

Early Latin often uses a Present Subjunctive \latin{duim},
\latin{duīs}, \latin{duit}, etc., and sometimes \latin{duam},
\latin{duās}, \latin{duat}, etc.  These are from a related root
\root{du-}.  Such Subjunctives are also formed from compounds like
\latin{crēdō}, \english{believe}, \latin{perdō},
\english{lose},\footnote{These compounds are really from a root
  meaning “put,” which was originally distinct from \latin{dō},
  \english{give}.} etc., which otherwise follow the Third Conjugation.
So \latin{crēduit}, \latin{perduit}, etc.

\end{minor}

\chapter{Defective Verbs}

\contentsentry{B}{Defective Verbs}

\headingC{Aiō, Inquam, Fārī}

\section
\subsection

\latin{Aiō}, \english{say}, \english{assent}, has the following forms.
It is pronounced and often spelled \latin{ai-iō} (\xref[2,
  \emph{a}]{29}).
\begin{Tabular}{>{\itshape}l@{\enskip}l@{\qquad\qquad}>{\itshape}l@{\enskip}l}

Pres.\ Indic.\ & aiō, ais, ait, aiunt
& Imperat. & aī \\

Pres.\ Subj.    & aiās, aiat
& Partic.   & aiēns \\

Imperf.\ Indic.
    & \M{3}{l}{aiēbam, aiēbās, aiēbat, aiēbāmus, aiēbātis, aiēbant.}

\end{Tabular}

\begin{minor}

\subsubsection

Early Latin has Imperf.\ \latin{aibam}, etc., from \rec{aībam} like
\latin{lēnībam} (\xref[4]{164}), but with \latin{ai} pronounced as one
syllable.

\end{minor}

\subsection

\latin{Inquam}, \english{say} (\english{said I}, \english{said he},
etc.), has the following forms, of which only \latin{inqam},
\latin{inquis}, and \latin{inquit} are in common use.
\begin{Tabular}{>{\itshape}l@{\enskip}l
                @{\qquad}
                >{\itshape}l@{\enskip}l}

Pres.\ Indic.
    & inquam, inquis, inquit\\
    & inquimus, inquitis, inquiunt
    & Perf.\ Indic. & inquiī, inquīstī \\

Imperf.\ Indic.
    & inquiēbat
& Imperat.
    & inque, inquitō \\

Fut.\ Indic.
    & inquiēs, inquiet

\end{Tabular}

\subsection

\latin{Fārī}, \english{speak}, has the following forms, of which some
occur only in compounds.
\begin{Tabular}{>{\itshape}l@{\enskip}l
                @{\qquad}
                >{\itshape}l@{\enskip}l}

Pres.\ Indic.
    & fātur, -fāmur
& Infin.
    & fārī \\

Imperf.\ Indic.
    & -fābar, -fābantur
& Pres.\ Act.\ Partic.
    & fāns \\

Fut.
    & fābor, fābitur, -fābimur
& Perf.\ Pass.\ Partic.
    & fātus \\

Perf.\ Indic.
    & fātus sum, etc.
& Fut.\ Pass.\ Partic.
    & fandus \\

Past Perf.\ Indic.
    & fātus eram, etc.
& Gerund
    & fandī, fandō \\

Imperat.
    & fāre
& Supine
    & fātū

\end{Tabular}

\headingC{Meminī, Ōdī, Coepī}

\section
\subsection

\latin{Meminī}, \english{remember}, and \latin{ōdī}, \english{hate},
are Present in meaning, but in form belong to the Perfect System.  But
\latin{meminī} has an Imperative, and \latin{ōdī} a Future Participle.
\begin{Tabular}{lll}

\cc{2}{\textbfsc{indicative}} & \cc{1}{\textbfsc{subjunctive}} \\

meminī, \english{I remember}
& ōdī, \english{I hate}
& meminerim, ōderim \\

memineram, \english{I remembered}
& ōderam, \english{I hated}
& menimissem, ōdissem \\

meminerō, \english{I shall remember}
& ōderō, \english{I shall hate}

\end{Tabular}
\begin{Tabular}{>{\itshape}l@{\,}l}

Imperat.        & mementō, mementōte \\
Infin.          & meminisse, ōdisse \\
Fut.\ Partic.   & \na, ōsūrus

\end{Tabular}

\subsection

\latin{Coepī}, \english{have begun}, \english{began}, is also confined
mainly to the Perfect System, the Present being supplied by
\latin{incipiō}.  When a Passive Infinitive follows, \latin{coeptus sum} takes
the place of \latin{coepī}; e.g.\ \latin{quae rēs agī coeptae sunt},
\english{which began to be done}.

\begin{minor}

\subsubsection

Some forms of the Present System are found in early Latin, as
\latin{coepiō}, \latin{coepere}.

\end{minor}

\section

Other isolated forms are:

\subsection

Imperat.\ \latin{salvē}, \english{hail}, \latin{salvēte},
\latin{salvētō}; Infin.\ \latin{salvēre}.

\subsection

Imperat.\ \latin{(h)avē}, \english{hail}, \latin{(h)avēte},
\latin{(h)avētō}; Infin.\ \latin{(h)avēre}.

\subsection

Imperat.\ \latin{cedo}, \english{give}, \latin{cette}.

\subsection

Pres.\ Indic.\ \latin{quaesō}, \english{beg}, \latin{quaesumus}.

\headingC{Impersonal Verbs}

\contentsentry{B}{Impersonal Verbs}

\section

A special class of Defectives consists of verbs used only
impersonally, the forms occurring being those of the Third Person
Singular, the Infinitive, and sometimes the Gerund. So, for example,
\latin{licet}, \english{it is allowed}.

\begin{longtable}{>{\itshape}lll}

& \textbfsc{indicative}
& \textbfsc{subjunctive}
\\

Pres.
& \latin*{licet}
& \latin*{liceat}
\\

Imperf.
& \latin*{licēbat}
& \latin*{licēret}
\\

Fut.
& \latin*{licēbit}
& \na
\\

Perf.
& \latin*{licuit}   \emph{or} \latin*{licitum est}
& \latin*{licuerit} \emph{or} \latin*{licitum sit}
\\

Past Perf.
& \latin*{licuerat}  \emph{or} \latin*{licitum erat}
& \latin*{licuisset} \emph{or} \latin*{licitum esset}
\\

Fut. Perf.
& \latin*{licuerit} \emph{or} \latin*{licitum erit}
& \na
\\

\multicolumn{3}{c}{\emph{Infin.}\enskip \latin*{licēre},
  \latin*{licuisse}, \latin*{licitūrum esse}}

\end{longtable}

\subsubsection
The following verbs are always, or usually, Impersonal:
\begin{mexamples}

\latin{decet}, \english{it is fitting}

\latin{libet}, \english{it is pleasing}

\latin{licet}, \english{it is permitted}

\latin{miseret}, \english{it excites pity}

\latin{ninguit}, \english{it snows}

\latin{oportet}, \english{it is necessary}

\latin{piget}, \english{it grieves}

\latin{pluit}, \english{it rains}

\latin{pudet}, \english{it shames}

\latin{taedet}, \english{it disgusts}

\latin{tonat}, \english{it thunders}

\end{mexamples}

\subsubsection
Many verbs are used impersonally only in certain senses;
e.g.\ \latin{placet}, \english{it pleases}, \english{is decided},
\latin{accidit}, \english{it happens}.

\subsubsection
The Passive of intransitive verbs can be used only impersonally;
e.g.\
\latin{ītur}, \english{there is a going}, \english{some one goes},
\latin{curritur}, \english{there is a running}, \english{some one
  runs}.

\part{Word-Formation}

\begin{minor}

\section[\textsc{\small Introductory}]

In the broadest sense, Word-Formation includes the subject of
Inflection; but the term as used here refers to the formation of the
word as a whole, i.e.\ the unit of which the inflectional forms are
variations.

In the case of declensional forms the true unit is the Stem, to which
the case-endings are added, so that Stem-Formation would be a more
precise term to use in this connection. But in the case of Verb-forms,
although the unit of any given tense is the stem, the verb as a whole
includes several different stems (tense and mood-stems), and their
formation is properly treated as a part of Inflection.  As regards
Verbs, then, there is left for treatment here only what is known as
Secondary Verbal Derivation, that is, the formation of Verbs from
Nouns, Adjectives, or other Verbs.

The derivation of most uninflected words is too obscure to be
discussed except in connection with the related forms of other
languages.  The formation of Adverbs, however, which in part stands in
close relation with case-formation, has been touched upon under
Inflection.

\end{minor}

Under Word-Formation, then, will be included:
\begin{enumerate}

\item
Derivation of Nouns and Adjectives by means of suffixes.

\item
Secondary Verbal Derivation.

\item
Composition.

\end{enumerate}

\chapter{Derivation of Nouns and Adjectives}

\contentsentry{B}{Derivation of Nouns, Adjectives, and Verbs}

\enlargethispage{\baselineskip}

\section

The stems to which the case-endings are added are sometimes identical
with the Root,\footnote{\label{ftn:s203:1}By a Root is meant the
  simplest element common to a group of related words and conceived as
  containing the essential meaning common to all.  It is what is left
  after the separation of all recognizable formative parts, such as
  prefixes, suffixes, endings.  But for any given language it is only
  a convenient grammatical abstraction, not necessarily an absolutely
  primitive element.  For example, in English the root of
  \english{preacher} and \english{preaching} is \english{preach}, but
  this, when taken back to \latin{prae-dicō}, is seen to be a
  compound.  The roots assumed in Latin are not necessarily the same
  as those assumed for the parent speech, much less are they to be
  thought of as ultimate roots.} as in \latin{dux},
Gen.\ \latin{ducis}, of which \stem{duc-} is both stem and root. But
usually they are formed by means of
Suffixes\footnote{\label{ftn:s203:2}Suffixes are doubtless independent
  words in origin, just as the English suffix \english{-ly} (man-ly,
  etc.)\ meaning \english{like}, \english{having the form of}, has
  arisen in historical times from a word meaning \english{body},
  \english{form} (from which comes also \english{like},
  i.e.\ \english{having the same form as}).  But most suffixes had
  already, in a remote period, become mere formative elements, which
  it is useless to try to connect with any known independent words.

 The ultimate origin of the inflectional endings is of the same
 nature, but in detail will always be obscure. There is, then, in the
 general principles of formation no hard and fast line between
 composition, derivation by suffixes, and inflection.} added either to
the root or to other stems.

If the suffix is added to a Root or a Verb-Stem, the form is known as
a primary Derivative; if the suffix is added to a Noun-Stem,
Adjective-Stem, or Adverb, the form is known as a Secondary
Derivative. Thus \latin{vic-tor}, \english{victor}, in which the
suffix is added to the root of \latin{vincō}, \english{conquer}, is
a Primary Derivative; while \latin{victōr-ia}, \english{victory}, in
which the suffix is added to the Noun-Stem \stem{victōr-}, is a
Secondary Derivative. Many suffixes were used primarily for only one
of these two kinds of derivation, but their use was often extended,
the same suffix appearing in both Primary and Secondary Derivation.

\section

Some suffixes are seen in words inherited from the parent speech, but
are no longer used freely to form new words. They are called
\emph{non-productive} suffixes. Thus \suffix{-ti-}, seen in \latin{par-ti-}
(\latin{pars}), \latin{mor-ti-} (\latin{mors}), etc., was once one of
the commonest suffixes for the formation of feminine abstracts, but in
Latin it is not productive, its place having been usurped by an
extension of it, namely, \suffix{-tiōn-}, seen in \latin{āctiōn-}
(\latin{āctiō}), etc.

\section

Regarding the combination of the root or stem with the suffix,
observe:

\subsection

If the root or stem ends in a consonant and the suffix begins with
one, the usual consonant changes take place; e.g.\ \latin{scrīptor}
from \latin{scrībō}; \latin{dēfēn-sor} from \latin{dēfendō},
etc. See~\xref{49}.

\subsection

The final vowel of a stem is lost before the initial vowel of a suffix;
e.g.\ \latin{aur-eus} from \latin{auro-} (\latin{aurum}).

\subsection

The final short vowel of a Stem suffers the regular weakening before a
suffix beginning with a consonant; e.g.\ \latin{boni-tās} from
\latin{bono-} (\latin{bonus}); \latin{porcu-lus} from \latin{porco-}
(\latin{porcus}). See~\xref[5]{42}.

\begin{note}

The final vowel of a stem (or sometimes a consonant, or even a whole
syllable belonging properly to the stem) often becomes so closely
associated with the suffix that it is felt to be a part of it, and not
a part of the stem.  In this way a new suffix arises. So from words
like \latin{Rōmānus} (really \latin{Rōmā-nus}) the suffix
\suffix{-ānus} arises, and is then applied freely to stems not ending
in~\suffix{-ā}, as \latin{urb-ānus} (\latin{urbs}), \latin{pāg-ānus}
(\latin{pāgus}), etc. Similarly from words like \latin{porcu-lus} (see
above, 3) arose words like \latin{rēg-ulus} (\latin{rēx}), etc. This
process was constantly going on.\footnote{For this reason, the division
  between the stem and the suffix is often somewhat arbitrary. For
  example, one may divide \latin{Rōmā-nus} in accordance with its
  origin, but, since \suffix{-ānus} has itself come to be a common
  suffix, one may properly divide \latin{Rōm-ānus} like
  \latin{urb-ānus}, etc.  Indeed, in some cases, the only practicable
  division is one which includes in the suffix an element which a
  scientific analysis shows to have belonged properly to the stem; for
  any other division would be not only confusing to the student, but
  contrary to the feeling which the Romans themselves had in using the
  suffix.}

\end{note}

\pagebreak

\headingG{Nouns—Primary Derivatives}

\section
\subsection

\suffix{-tor}\footnote{Only some of the commoner and more productive
  suffixes will be mentioned.  For the sake of convenience the form
  of the Nominative Singular is cited, rather than the stem.}
(Fem.\ \suffix{-trīx}) forms nouns denoting the \emph{agent} or
\emph{doer} of the action expressed by the verb
(cf.\ English~\english{-er}).
\begin{mexamples}

\latin{vic-tor}, \english{victor}, \gender{f}. \latin{vic-trīx}
(\latin{vincō})

\latin{can-tor}, \english{singer} (\latin{canō})

\latin{scrīp-tor}, \english{writer} (\latin{scrībō})

\latin{amā-tor}, \english{lover} (\latin{amō})

\latin{dēfēn-sor} (for \rec{dē-fend-tor}\footnote{Before suffixes
  beginning with~\phone{t}, the same consonant-changes take place as in
  Perfect Passive Participles.  Hence \suffix{-sor}, \suffix{-siō},
  \suffix{-sus}, \suffix{-sūra}, beside \suffix{-tor}, \suffix{-tiō},
  \suffix{-tus}, \suffix{-tūra}.  See \xref[4]{49}.}),
\english{defender} (\latin{dēfendō}).

\end{mexamples}

\begin{minor}

\subsection

By an extension of use, this suffix is sometimes added to Nouns to form
other Nouns, as \latin{iāni-tor}, \english{janitor}, \english{one who
  keeps the door} (\latin{iānua}), \latin{gladi-ātor} (\suffix{-ātor}
from \latin{amā-tor}, etc.; see \xref[note]{205}), \english{one who
  uses the sword} (\latin{gladius}).

\end{minor}

\subsection

\suffix{-iō}, \suffix{-tiō}, \suffix{-tus} (Gen.\ \suffix{-tūs}),
\suffix{-tūra}, and often \suffix{-ium}, form \emph{abstracts}
denoting the \emph{action} expressed by the verb, or, by a frequent
transfer from the abstract to a concrete meaning, the \emph{result of
  the action}.  Compare English \english{clipping}, the act of
clipping, and \emph{a \textup{(newspaper)} clipping}, the thing
clipped.
\begin{mexamples}

\latin{leg-iō}, \english{legion}, originally \english{the selecting},
\english{levying} (\latin{legō})

\latin{āc-tiō}, \english{the doing, act} (\latin{agō})

\latin{mis-siō},\footnotemark[\thefootnote] \english{dismissal}
(\latin{mittō})

\latin{can-tus}, \english{singing} (\latin{canō})

\latin{adven-tus}, \english{arrival} (\latin{adveniō})

\latin{vī-sus},\footnotemark[\thefootnote] \english{sight} (\latin{videō})

\latin{scrīp-tūra}, \english{writing} (\latin{scrībō})

\latin{tōn-sūra}, \english{shearing} (\latin{tondeō})

\latin{gaud-ium}, \english{joy} (\latin{gaudeō})

\latin{stud-ium}, \english{zeal} (\latin{studeō})

\latin{imper-ium}, \english{command} (\latin{imperō})

\latin{iūdic-ium}, \english{judgment} (\latin{iudicō})

\end{mexamples}

\begin{minor}

\subsection

Many words with the suffix \suffix{-tūra} are closely associated with
agent-nouns in~\suffix{-tor}, and denote \emph{office}.
\begin{mexamples}

\latin{quaes-tūra}, \english{quaestorship} (\latin{quaestor})

\latin{cēn-sūra}, \english{censorship} (\latin{cēnsor})

\end{mexamples}

\end{minor}

\subsection

\suffix{-men} and \suffix{-mentum} \emend{8}{from}{form} nouns denoting
\emph{action}, or, oftener, the \emph{result of an action}.
\begin{mexamples}

\latin{flū-men}, \english{stream} (\latin{fluō})

\latin{sē-men}, \english{seed} (\latin{serō}, Perf.\ \latin{se-vī})

\latin{frag-men}, \latin{frag-mentum}, \english{fragment}
(\latin{frangō})

\latin{ōrnā-mentum}, \english{ornament} (\latin{ōrnō})

\end{mexamples}

\begin{minor}

\subsection

So sometimes \suffix{-mōnium}, as \latin{ali-mōnium},
  \english{nourishment} (\latin{alō}); but this suffix is more
  frequent in secondary derivatives, as \latin{patri-mōnium},
  \english{patrimony} (\latin{pa\-ter}).

\end{minor}

\pagebreak

\subsection

\suffix{-or} (Gen.\ \suffix{-ōris}) forms abstracts which usually
indicate a \emph{physical} or \emph{mental state}.
\begin{mexamples}

\latin{trem-or}, \english{trembling} (\latin{tremō})

\latin{cal-or}, \english{warmth} (\latin{caleō})

\latin{cand-or}, \english{brightness} (\latin{candeō})

\latin{am-or}, \english{love} (\latin{amō})

\end{mexamples}

\subsection

\suffix{-dō} and \suffix{-gō} form nouns of various meanings.
\begin{mexamples}

\latin{cupī-dō}, \english{desire} (\latin{cupiō})

\latin{orī-gō}, \english{source} (\latin{orior})

\end{mexamples}

\subsection

\suffix{-ulum}, \suffix{-bulum}, \suffix{-culum}, \suffix{-brum},
\suffix{-crum}, and \suffix{-trum} (also \suffix{-ula},
\suffix{-bula}, \suffix{-bra}, etc.), form nouns denoting
\english{instrument} or \english{means}.  The idea sometimes passes
over into that of \english{place} or \english{result}.
\begin{mexamples}

\latin{vinc-ulum}, \english{chain} (\latin{vincō})

\latin{pā-bulum}, \english{fodder} (\latin{pāscō})

\latin{vehi-culum}, \english{wagon} (\latin{vehō})

\latin{fā-bula}, \english{tale} (\latin{fārī})

\latin{dēlū-brum}, \english{shrine} (\latin{dēluō})

\latin{simulā-crum}, \english{image} (\latin{simulō})

\latin{arā-trum}, \english{plough} (\latin{arō})

\latin{dolā-bra}, \english{axe} (\latin{dolō})

\end{mexamples}

\headingG{Nouns—Secondary Derivatives}

\section
\subsection

\suffix{-lus} (Fem.~\suffix{-la}, Neut.~\suffix{-lum}) and its various
combinations \suffix{-ulus}, \suffix{-olus}, \suffix{-ellus},
\suffix{-illus}, \suffix{-ullus}, and \suffix{-culus}, form
\emph{Diminutives}.  These usually follow the gender of the word from
which they are derived.
\begin{mexamples}

\latin{porcu-lus}, \english{little pig} (\latin{porcus})

\latin{fīlio-lus}, \english{young son} (\latin{fīlius})

\latin{agel-lus}, \english{small field} (\latin{ager}; see~\xref[11]{49})

\latin{lapil-lus}, \english{small stone} (\latin{lapis};
\rec{lapid-lo-}; see~\xref[11]{49})

\latin{ōs-culum}, \english{little mouth}, \english{kiss} (\latin{ōs})

\latin{rēg-ulus}, \english{chieftain} (\latin{rēx})

\latin{fīli-ola}, \english{young daughter} (\latin{fīlia})

\latin{tabel-la}, \english{tablet} (\latin{tabula})

\latin{homul-lus}, \english{manikin} (\latin{homō}; \rec{homon-lo-};
see~\xref[11]{49})

\latin{arti-culus}, \english{joint} (\latin{artus})

\end{mexamples}

\subsection

\suffix{-ia}, \suffix{-tia}, \suffix{-tiēs}, \suffix{-tās},
\suffix{-tūdō}, \latin{-tūs}, and sometimes \suffix{-ium} and
\suffix{-tium} form abstracts denoting \emph{quality} or
\emph{condition}.
\begin{mexamples}

\latin{miser-ia}, \english{misery} (\latin{miser})

\latin{audāc-ia}, \english{boldness} (\latin{audāx})

\latin{dūri-tia}, \latin{dūri-tiēs}, \latin{dūri-tās},
\english{hardness} (\latin{dūrus})

\latin{boni-tās}, \english{goodness} (\latin{bonus})

\latin{magni-tūdō}, \english{greatness} (\latin{magnus})

\latin{cīvi-tās}, \english{citizenship}, \english{state} (\latin{cīvis})

\latin{vir-tūs}, \english{manliness} (\latin{vir})

\latin{sacerdōt-ium}, \english{priesthood} (\latin{sacerdōs})

\latin{servi-tium}, \english{servitude} (\latin{servus})

\end{mexamples}

\subsection

\suffix{-adēs}, \suffix{-iadēs}, \suffix{-idēs}, \suffix{-īdēs}
(Masc.)\ and \suffix{-ias}, \suffix{-is}, \suffix{-ēis} (Fem.)\ occur
in Greek Patronymics, denoting \emph{descent}.
\begin{mexamples}

\latin{Aene-adēs}, \english{son of Aeneas}

\latin{Anchīs-iadēs}, \english{son of Anchises}

\latin{Tantal-idēs}, \english{descendant of Tantalus}

\latin{Pēl-īdēs}, \english{son of Peleus}

\latin{Thest-ias}, \english{daughter of Thestius}

\latin{Tyndar-is}, \english{daughter of Tyndarus}

\latin{Nēr-ēis}, \english{daughter of Nereus}

\end{mexamples}

\pagebreak

\subsection

\suffix{-īna} often forms nouns denoting an \emph{art}
or~\emph{craft}, or the place where a craft is practiced.
\begin{mexamples}

\latin{medic-īna}, \english{healing} (\latin{medicus})

\latin{discipl-īna}, \english{instruction} (\latin{discipulus})

\latin{doctr-īna}, \english{teaching} (\latin{doctor})

\latin{tōnstr-īna}, \english{barber’s shop} (\latin{tōnsor})

\end{mexamples}

\begin{minor}

\subsubsection
This type originated in Adjectives used substantively, \latin{ars} or
\latin{officīna} being understood.  But the suffix~\suffix{-īna} is
used in other ways, e.g.\ in simple Feminines like \latin{rēg-īna},
\english{queen} (\latin{rēx}) or in Primary Derivatives, as
\latin{rap-īna}, \english{robbery} (\latin{rapiō}).

\end{minor}

\subsection

Other significant suffixes are:
\suffix{-ātus} (Gen.~\suffix{-ātūs}), denoting \english{office} or
    \english{official body};
\suffix{-ārius}, \english{a dealer} or \english{artisan};
\suffix{-ārium}, \emph{a place where things are kept};
\suffix{-īle}, \emph{a place for animals}.
\begin{mexamples}

\latin{cōnsul-ātus}, \english{consulship} (\latin{cōnsul})

\latin{argent-ārius}, \english{money changer} (\latin{argentum})

\latin{aer-ārium}, \english{treasury} (\latin{aes})

\latin{ov-īle}, \english{sheepfold} (\latin{ovis})

\end{mexamples}

\headingG{Adjectives—Primary Derivatives}

\section
\subsection

\suffix{-āx} and sometimes \suffix{-ulus} form adjectives denoting
\emph{tendencies} or \emph{qualities}.
\begin{mexamples}

\latin{aud-āx}, \english{bold} (\latin{audeō})

\latin{ten-āx}, \english{tenacious} (\latin{teneō})

\latin{bib-ulus}, \english{fond of drink} (\latin{bibō})

\latin{crēd-ulus}, \english{credulous} (\latin{crēdō})

\latin{vor-āx}, \english{voracious} (\latin{vorō})

\end{mexamples}

\subsection

\suffix{-ilis} and \suffix{-bilis} form adjectives denoting \emph{passive
qualities}.
\begin{mexamples}

\latin{frag-ilis}, \english{breakable}, \english{frail} (\latin{frangō})

\latin{fac-ilis}, \english{easy} (\latin{faciō})

\latin{bib-ilis}, \english{drinkable} (\latin{bibō})

\latin{mō-bilis}, \english{movable} (\latin{moveō})

\latin{amā-bilis}, \english{lovable} (\latin{amō})

\latin{crēdi-bilis}, \english{worthy of belief} (\latin{crēdō})

\end{mexamples}

\subsection

\suffix{-bundus} forms adjectives having about the force of a Present
Participle, but is more intensive; \suffix{-cundus} denotes a
\emph{characteristic}.
\begin{mexamples}

\latin{verberā-bundus}, \english{flogging} (\latin{verberō})

\latin{mori-bundus}, \english{dying} (\latin{morior})

\latin{īrā-cundus}, \english{wrathful} (\latin{īrāscor})

\latin{fā-cundus}, \english{eloquent} (\latin{fārī})

\end{mexamples}

\headingG{Adjectives—Secondary Derivatives}

\section
\subsection

\latin{-eus}, \latin{-āceus}, and sometimes \latin{-nus},
\latin{-neus}, \latin{-inus}, form adjectives of \emph{material}.
\begin{mexamples}

\latin{aur-eus}, \english{golden} (\latin{aurum})

\latin{ferr-eus}, \english{of iron} (\latin{ferrum})

\latin{ros-āceus}, \english{of roses} (\latin{rosa})

\latin{acer-nus}, \english{of maple} (\latin{acer})

\latin{ebur-neus}, \english{of ivory} (\latin{ebur})

\latin{fāg-inus}, \english{of beech} (\latin{fāgus})

\end{mexamples}

\subsection

\suffix{-ōsus} and \suffix{-lentus} form adjectives denoting
\emph{fullness}.
\begin{mexamples}

\latin{vīn-ōsus}, \english{drunken} (\latin{vīnum})

\latin{verbō-sus}, \english{verbose} (\latin{verbum})

\latin{vīno-lentus}, \english{drunken} (\latin{vīnum})

\latin{opu-lentus}, \english{wealthy} (\rec{ops}, \latin{opis})

\latin{bellic-ōsus}, \english{warlike} (\latin{bellicus})

\end{mexamples}

\pagebreak

\subsection

\suffix{-tus}, identical with the suffix of the Perfect Passive
Participle, is also added to Noun-Stems, forming adjectives meaning
\emph{provided with} (cf.\ English~\english{-ed}).
\begin{mexamples}

\latin{barbā-tus}, \english{bearded} (\latin{barba})

\latin{dent-ātus}, \english{toothed} (\latin{dēns})

\latin{aurī-tus} \english{\(long-\)eared} (\latin{auris})

\latin{cornū-tus}, \english{horned} (\latin{cornū})

\latin{onus-tus}, \english{laden} (\latin{onus})

\end{mexamples}

\subsection

\suffix{-idus} forms adjectives denoting a \emph{condition}.
\begin{mexamples}

\latin{lūc-idus}, \english{light} (\latin{lūx})

\latin{fūm-idus}, \english{smoky} (\latin{fūmus})

\end{mexamples}

\begin{minor}

\subsubsection
This suffix, though originating in Secondary Derivatives (properly
compounds; e.g.\ \latin{lūci-dus}, \english{light-giving};
cf.\ \latin{dō}, \english{give}, or~\suffix{-dō}, \english{put}), is
also used to form Primary Derivatives.
\begin{mexamples}

\latin{cup-idus}, \english{eager} (\latin{cupiō})

\latin{langu-idus}, \english{weak} (\latin{langueō})

\end{mexamples}

\end{minor}

\subsection

\suffix{-ernus}, \suffix{-ternus}, \suffix{-urnus}, \suffix{-turnus},
and \suffix{-tinus}, form adjectives denoting
\emph{time}, mostly from Adverbs.
\begin{mexamples}

\latin{hodi-ernus}, \english{of today} (\latin{hodiē})

\latin{hes-ternus}, \english{of yesterday} (\latin{herī})

\latin{di-urnus}, \english{daily} (\latin{diēs})

\latin{diu-turnus},\footnote{In spite of the connection with
  \latin{diū}, the \phone{u} in the second syllable is short in all
  the passages thus far noted in poetry.}
\english{long-continued} (\latin{diū})

\latin{diū-tinus}, \english{long-continued} (\latin{diū})

\latin{crās-tinus}, \english{of to-morrow} (\latin{crās})

\latin{annō-tinus}, \english{last year’s} (\latin{annus})

\end{mexamples}

\section
\subsection

\suffix{-ius}, \suffix{-cus}, \suffix{-icus}, \suffix{-icius},
\suffix{-īcius}, \suffix{-nus}, \suffix{-ānus}, \suffix{-īnus},
\suffix{-ālis}, \suffix{-īlis}, \suffix{-ēlis}, \suffix{-āris},
\suffix{-ārius}, form adjectives meaning \emph{belonging to},
\emph{connected with}, \emph{derived from}, etc.
\begin{mexamples}

\latin{patr-ius}, \english{paternal} (\latin{pater})

\latin{senātōr-ius}, \english{senatorial} (\latin{senātor})

\latin{hosti-cus}, \english{hostile} (\latin{hostis})

\latin{bell-icus}, \english{of war} (\latin{bellum})

\latin{patr-icius}, \english{patrician} (\latin{pater})

\latin{nov-īcius}, \english{new} (\latin{novus})

\latin{pater-nus}, \english{paternal} (\latin{pater})

\latin{urb-ānus}, \english{of the city} (\latin{urbs})

\latin{can-īnus}, \english{canine} (\latin{canis})

\latin{rēg-ālis}, \english{royal} (\latin{rēx})

\latin{cīv-īlis}, \english{of a citizen} (\latin{cīvis})

\latin{crūd-ēlis}, \english{cruel} (\latin{crūdus})

\latin{popul-āris}, \english{of the people} (\latin{populus})

\latin{legiōn-ārius}, \english{of a legion} (\latin{legiō})

\end{mexamples}

\subsection

\latin{-īvus}, seen in \latin{aest-īvus}, \english{of summer}
(\latin{aestus}), was often added to the stem of the Perfect Passive
Participle, giving rise to a suffix~\suffix{-tīvus}.
\begin{mexamples}

\latin{cap-tīvus}, \english{captive} (\latin{capiō}, \latin{captus})

\latin{fugi-tīvus}, \english{fugitive} (\latin{fugiō})

\end{mexamples}

\begin{minor}

\subsubsection
Observe also the names for the Cases and Moods;
e.g.\ \latin{nōminā-tīvus}, \latin{gene-tīvus}, \latin{indicā-tīvus},
etc.\ (used substantively, \latin{cāsus} or \latin{modus} being
understood).

\end{minor}

\pagebreak

\subsection

\suffix{-ēnsis} and \suffix{-iēnsis} form adjectives from words
denoting place, mostly
names of towns.\footnote{Many such adjectives
  are also used substantively, especially in the Plural;
  e.g.\ \latin{Athēniēnsēs}, \english{Athenians}, \latin{Arpīnātēs},
  \english{inhabitants of Arpinum}, \latin{Rōmānī}, \english{Romans},
  etc.}
\begin{mexamples}

\latin{castr-ēnsis}, \english{of the camp} (\latin{castra})

\latin{Carthāgin-iēnsis}, \english{of Carthage}

\latin{Cann-ēnsis}, \english{of Cannae}

\end{mexamples}

\subsection

Other suffixes frequently added to names of towns and countries are
\suffix{-ās}, \suffix{-ānus}, \suffix{-īnus},
and~\suffix{-icus}.\footnotemark[\thefootnote]
\begin{mexamples}

\latin{Arpīn-ās}, \english{of Arpinum}

\latin{Rōm-ānus}, \english{of Rome}, \english{Roman}

\latin{Lat-īnus}, \english{of Latium}, \english{Latin}

\latin{Ital-icus}, \english{of Italy}, \english{Italian}

\end{mexamples}

\begin{minor}

\subsubsection

\suffix{-ās} is used only with names of Italian towns.  Adjectives
denoting \emph{nationality} usually, though not always, end
in~\suffix{-icus}; e.g.\ \latin{Gall-icus}, \english{Gallic},
\latin{Germān-icus}, \english{Germanic}.

\end{minor}

\subsection

Adjectives derived from names of persons commonly end in
\suffix{-ānus} or \suffix{-iānus}.
\begin{mexamples}

\latin{Sull-ānus}, \english{of Sulla}

\latin{Cicerōn-iānus}, \english{of Cicero}

\end{mexamples}

\chapter{Secondary Verbal Derivatives}

\headingG{Verbs \emend{76}{d}{D}erived from Nouns and Adjectives
(Denominatives)}

\section

The great mass of Denominatives follow the First Conjugation, but
there are also many of the Fourth, some of the Second, and a few (from
\phone{u}-Stems) of the Third.

\subsection

First Conjugation.
\begin{mexamples}

\latin{cūrō}, \english{care for} (\latin{cūra})

\latin{dōnō}, \english{give} (\latin{dōnum})

\latin{levō}, \english{lift} (\latin{levis})

\latin{sinuō}, \english{bend} (\latin{sinus}, Gen.\ \latin{sinūs})

\latin{honōrō}, \english{honor} (\latin{honor}, Gen.\ \latin{honōris})

\latin{laudō}, \english{praise} (\latin{laus}, Gen.\ \latin{laudis})

\end{mexamples}

\subsection

Fourth Conjugation.
\begin{mexamples}

\latin{fīniō}, \english{end} (\latin{fīnis})

\latin{partior}, \english{divide} (\latin{pars}, Gen.\ \latin{partis})

\latin{custōdiō}, \english{guard} (\latin{custōs}, Gen.\ \latin{custōdis})

\latin{serviō}, \english{be a slave} (\latin{servus})

\end{mexamples}

\subsection

Second Conjugation.
\begin{mexamples}

\latin{albeō}, \english{be white} (\latin{albus})

\latin{flōreō}, \english{blossom} (\latin{flōs}, Gen.\ \latin{flōris})

\end{mexamples}

\begin{minor}

\subsubsection

These are mostly intransitive, denoting a condition.  Contrast
\latin{clāreō}, \english{be bright} (\latin{clārus}), with
\latin{clārō}, \english{make bright}, \latin{clārāre}.

\end{minor}

\pagebreak

\subsection

Third Conjugation,
\begin{mexamples}

\latin{statuō}, \english{set up} (\latin{status})

\latin{tribuō}, \english{assign} (\latin{tribus})

\end{mexamples}

\begin{note}

Denominatives were formed from Noun-Stems by means of a
suffix~\suffix*{-yo-} and \suffix*{-ye-}. The \phone{y} disappeared
between vowels, and, in most forms, the vowels then contracted.  Thus
\latin{cūrō} from \rec*{cūrā-yō}; \latin{albeō} from
\rec*{albe-yō}. See notes to \xref{166}, \xref{167}, \xref{169}.

The type in~\suffix{-ō}, \suffix{-āre} originated in \phone{ā}-Stems,
that in~\suffix{-iō} in \phone{i}-Stems and Consonant-Stems, that
in~\suffix{-eō} in \phone{o}-Stems (but with the \phone{e}-form of the
stem, which appears in the Vocative).  But the different formations
came finally to be used without reference to the form of the
Noun-Stem, and especially the type~\suffix{-ō}, \suffix{-āre} was used
to form Denominatives from all kinds of stems.

\end{note}

\headingG{Verbs \emend{77}{d}{D}erived from Other Verbs}

\section
\subsection[\emph{Frequentatives}]

These end in \suffix{-tō} (\suffix{-sō}), \suffix{-itō}, and sometimes
\suffix{-titō}, and denote \emph{repeated}, or sometimes merely
\emph{intensive}, action.
\begin{mexamples}

\latin{dictō}, \english{dictate} (\latin{dīcō})

\latin{versō}, \english{keep turning} (\latin{vertō})

\latin{habitō}, \english{dwell} (\latin{habeō})

\latin{rogitō}, \english{keep asking} (\latin{rogō})

\latin{dictitō}, \english{keep saying}, \english{declare}
(\latin{dīcō})

\end{mexamples}

\begin{note}

The Frequentatives are Denominative in origin, being formed from the
stem of the Perfect Passive Participle.  But, owing to their
distinctive meaning, they came to be contrasted with the simple Verbs
and were felt to be derived from them.  In general they follow the
formation of the Participle, but many are formed directly from the
Present Stem, as \latin{agitō}, \english{move violently} (\latin{agō},
Partic.\ \latin{āctus}).  The Frequentatives from verbs of the First
Conjugation always end in~\suffix{-itō}, not \suffix{-ātō}, as
\latin{rogitō}.  The forms in~\suffix{-titō} are double
Frequentatives, being based on the forms in~\suffix{-tō}.

\end{note}

\subsection[\emph{Inchoatives}]

These end in \suffix{-ēscō}, \suffix{-āscō}, \latin{-īscō}, and denote
\emph{beginning} or \emph{becoming}.
\begin{mexamples}

\latin{calēscō}, \english{become hot} (\latin{caleō})

\latin{obdormīscō}, \english{fall asleep} (\latin{dormiō})

\end{mexamples}

\subsubsection

Some Inchoatives are derived from Nouns or Adjectives, and so,
properly, form a special class of Denominatives.
\begin{mexamples}

\latin{dūrēscō}, \english{become hard} (\latin{dūrus})

\latin{vesperāscō}, \english{become evening} (\latin{vesper})

\end{mexamples}

\begin{note}

This formation has its origin in the Primary Verbs in \suffix{-scō},
as \latin{crē-scō}, \latin{nō-scō}, etc.  Gaining the specific meaning
of \emph{beginning to} or \emph{becoming} (through verbs like
\latin{crēscō}, \english{grow}, that is, \english{begin to be large}),
its use was then extended so as to form Verbs from other Verbs, and
also to form Denominatives.

\end{note}

\subsection[\emph{Desideratives}]

These end in \suffix{-turiō} (\suffix{-suriō}), and denote
\emph{desire}; e.g.\ \latin{par-turiō}, \english{desire to bring
  forth}, \english{be in travail} (\latin{pariō});
\latin{ēsuriō},\footnote{From \rec{-ēd-turiō}.  See \xref[5]{49}.}
\english{desire to eat}, \english{be hungry} (\latin{edō}).

\begin{minor}

\subsection[\emph{Intensives} \textup{(also called Meditatives)}]

These end in \suffix{-essō} and denote ear\-nest action;
e.g.\ \latin{petessō}, \english{seek eagerly} (\latin{petō});
\latin{capessō}, \english{seek eagerly} (\latin{capiō}).

\end{minor}

\chapter{Composition}

\contentsentry{B}{Composition}

\section

Composition is the union of two or more words in one.

\headingG{Nouns and Adjectives}

\headingC{Form}

\section

According to the \emph{form} of the first part, compounds may be
classified as follows:

\subsection

The first part is the Stem of a Noun or Adjective. The final vowel of
the stem appears as \phone{i} before consonants, and is dropped before
vowels (rarely before consonants).  Consonant-Stems usually take the
form of \phone{i}-Stems.
\begin{mexamples}

\latin{armi-ger}, \english{armor-bearer} (\latin{arma})

\latin{agri-cola}, \english{farmer} (\latin{ager})

\latin{tubi-cen}, \english{trumpeter} (\latin{tuba})

\latin{parti-ceps}, \english{sharing}

\latin{corni-ger}, \english{horned} (\latin{cornū})

\latin{ūn-oculus}, \english{one-eyed} (\latin{ūnus})

\latin{prīn-ceps}, \english{chief} (\latin{prīmus})

\latin{frātri-cīda}, \english{fratricide} (\latin{frāter})

\latin{bi-dēns}, \english{two-pronged} (\prefix{bi-}, found only in
compounds)

\end{mexamples}

\begin{note}

The final \phone{i} of the first part may represent
original~\phone{i}, or, by the regular weakening (\xref[5,6]{42}),
\phone{o} or~\phone{u}; and, by the analogy of such cases, it is also
used for~\phone{ā}.

\end{note}

\subsection

The first part is an Adverbial Prefix.  Such prefixes, with the
exception of the negatives \prefix{in-} and \prefix{vē-}, are also
common in the composition of Verbs, and most of them occur separately
as Prepositions. See~\xref[1]{218}.
\begin{mexamples}

\latin{in-grātus}, \english{unpleasant}

\latin{vē-cors}, \english{senseless}

\latin{per-facilis}, \english{very easy}

\latin{sub-rūsticus}, \english{somewhat rustic}

\end{mexamples}

\begin{note}

Some compounds outwardly resembling those mentioned are of essentially
different origin, being derived from phrases consisting of a
Preposition with its proper case.  So
\latin{prō-cōnsul}, \english{one who is in the place of a consul}
    (\latin{prō cōnsule});
\latin{ē-gregius}, \english{distinguished}, \english{out of the common
  run} (\latin{ē grege});
\latin{ob-vius}, \english{in the way} (\latin{ob-viam}).

\end{note}

\subsection

The first part is a Case-form or Adverb.  Since this is merely the
union of forms which can be used separately, it is sometimes called
Improper Composition, or Juxtaposition.
\begin{mexamples}

\latin{senātūs-cōnsultum}, \english{decree of the senate}

\latin{aquae-ductus}, \english{aqueduct}

\latin{bene-volēns}, \english{well-wishing}

\end{mexamples}

\section
\subsection

The \emph{second} part of a compound is always the Stem of a Noun or
Adjective.  But sometimes it is one which appears only in composition;
e.g.\ \suffix{-fer}, \suffix{-ger}, \suffix{-ficus}, \suffix{-ceps},
\suffix{-cen}, \suffix{-cīda} (related to the verbs \latin{ferō},
\latin{gerō}, \latin{faciō}, \latin{capiō}, \latin{canō},
\latin{caedō}), \suffix{-duum} (\latin{bī-duum}, \english{two days};
related to \latin{diēs}), etc.

\subsection

Adjective compounds, of which the second part represents a noun of the
First or Second Declension, are commonly declined like \latin{bonus}
(\xref{110}), but many of them are made into \phone{i}-Stems,
e.g.\ \latin{bi-fōrmis}, \english{double}
(\latin{fōrma}).\footnote{Similarly \latin{in-ermis} (\latin{arma}),
  \latin{bi-iugis} (\latin{iugum}), but also \latin{in-ermus},
  \latin{bi-iugus}.  Most adjectives of varying declension are
  compounds. But cf.\ also \latin{hilarus} beside \latin{hilaris},
  etc.}  In some compounds a suffix is added, especially
\suffix{-ius}, \suffix{-ium}; e.g.\ \latin{in-iūrius},
\english{unlawful} (\latin{iūs}); \latin{bi-ennium}, \english{period
  of two years} (\latin{annus}).  For the vowel-weakening in the
second part of compounds, see~\xref{42}.

\headingC{Meaning}

\section

According to their \emph{meaning}, compounds may again be classified as
follows:

\subsection[\emph{Copulative Compounds}]

The parts are coördinate, as in \latin{suove-taurīlia},
\english{sacrifice of a swine, a sheep, and a bull},
\latin{quattuor-decim}, \english{fourteen}.

\subsection[\emph{Descriptive Compounds}]

The first part stands to the second in the relation of an adjectival
or an adverbial modifier, as in \latin{lāti-fundium}, \english{large
  estate}, \latin{per-facilis}, \english{very easy}.

\subsection[\emph{Dependent Compounds}]

The first part stands in a logical (not formal) case-relation to the
second, as in \latin{armi-ger}, \english{armor-bearer}.

\subsection[\emph{Possessive Compounds}]

Compounds of which the second part is a Noun may become Adjectival
with the force of \emph{possessed of}.  So \latin{ūn-oculus} means not
\english{one eye}, but \english{possessed of one eye},
\english{one-eyed}.

\headingG{Adverbs}

\section
\subsection

Most Adverbs that are apparently compounds are simply Adverbs formed
from Nouns or Adjectives already compounded, as \latin{perfacile},
\english{very easily}, from \latin{perfacilis}, \english{very easy}.
But:

\begin{minor}

\subsection

Some of the compounds with the prefixes \prefix{in-}, \english{not},
and \latin{per}, \english{very}, are formed directly from the simple Adverbs;
e.g.\ \latin{in-grātiīs}, \english{without thanks}, from
\latin{grātiīs}, \english{with thanks}; \latin{in-iussū},
\english{without command}, from \latin{iussū}, \english{by command};
\latin{per-saepe}, \english{very often}, from \latin{saepe},
\english{often}.

\subsection

The juxtaposition (\xref[3]{214}) of Prepositions and Adverbs of Time
or Place is frequent; e.g.\ \latin{ab-hinc}, \english{from this time},
\english{since}, \latin{dē-super}, \english{from above},
\latin{ad-hūc}, \english{hitherto}, \latin{inter-ibi},
\english{meanwhile}.

\subsection

The juxtaposition of a Preposition and its case gives rise to some
compound Adverbs; e.g.\ \latin{ob-viam}, \english{in the way},
\latin{ad-modum}, \english{to a degree}, \english{very},
\latin{dē-nuō}, \english{anew} (from \rec{dē-novō}; see \xref[4]{42}),
\latin{dē-subitō}, \english{of a sudden}, \english{suddenly}.

\end{minor}

\headingG{Verbs}

\section
\subsection

The only genuine and widely extended type of Verbal Composition is
that in which the first part is an Adverbial Prefix, as \latin{ab-eō},
\english{go away}, \latin{dir-imō}, \english{take apart}.

These prefixes, many of which are also used separately as Prepositions
or Adverbs, are as follows (for change in form, see~\xref{51}):

\emph{a}) Also used separately.
\begin{mexamples}[3]

\prefix{ā-}, \prefix{ab-}, \prefix{abs-}, \english{away}

\prefix{ad-}, \english{to}

\prefix{ante-}, \english{before}

\prefix{circum-}, \english{about}

\prefix{con-}, \english{with} (\prefix{cum-})

\prefix{dē-}, \english{without}

\prefix{ē-}, \prefix{ex-}, \english{out}

\prefix{in-}, \english{in}

\prefix{inter-}, \english{between}

\prefix{ob-}, \prefix{obs-}, \english{before}, \english{against}

\prefix{per-}, \english{through}, \english{thoroughly}

\prefix{post-}, \english{after}

\prefix{prae-}, \english{before}

\prefix{praeter-}, \english{beside}

\prefix{prō-}, \prefix{pro-}, \prefix{prōd-}, \english{forth}

\prefix{retrō-}, \english{back}

\prefix{sub-}, \prefix{subs-}, \english{under}

\prefix{subter-}, \english{beneath}

\prefix{super-}, \english{over}

\prefix{suprā-}, \english{over}

\prefix{trāns-}, \english{across}

\end{mexamples}

\emph{b}) Not used separately.
\begin{mexamples}[3]

\prefix{amb-}, \prefix{am-}, \english{about}

\prefix{an-}, \english{in} (rare)

\prefix{dis-}, \english{part}

\prefix{intrō-}, \english{within}

\prefix{por-}, \english{forth}

\latin{re-}, \latin{red-}, \english{back}

\latin{sē-}, \latin{sēd-}, \english{apart}

\end{mexamples}

\subsection

Juxtaposition is seen in forms like \latin{bene-dīcō},
\english{bless}, \latin{manūmittō}, \english{set free},
\latin{animadvertō}, \english{attend to}, from \latin{animum advertō}.

\subsection

Forms like \latin{cale-faciō}, \english{make hot},
\latin{cande-faciō}, \english{make white}, originated in simple
juxtaposition (\latin{cale faciō} written separately in early Latin),
but came to be felt as derived from Verbs in~\suffix{-eō}.

\begin{minor}

\subsection

Forms like \latin{aedi-ficō}, \english{build}, are apparently
compounds of a Noun-Stem with a Verb, but this type really originated
in Denominatives from Nouns already compounded;
e.g.\ \latin{aedificō}, from \rec{aedi-fex} or \rec{aedi-ficus},
\english{house builder}.

\end{minor}

\part{Syntax}

\numchapter{Introductory}

\contentsentry{A}{Introductory}

\contentsentry{C}{The Parts of Speech; The Sentence; Clauses and
  Phrases}

\section

Syntax treats of the use of words in the expression of thought or
feeling.

\section

A Sentence is a complete expression of thought or
feeling through the use of words.

\section

The Latin Sentence is made up of some or all of the following
\emph{kinds} of words, called \term{Parts of Speech}:
\begin{indented}

The Noun, which expresses a person or thing.

The Adjective, which expresses a quality, condition, etc.

The Pronoun, which stands instead of a Noun.

The Verb, which expresses an act or state.

The Adverb, which expresses manner, degree, etc.

The Preposition, which expresses relations between words.

The Conjunction, which expresses connection.

The Interjection, which expresses feeling, etc.

\end{indented}

\subsubsection

Nouns are called \term{Substantives}; e.g.\ \latin{arbor},
\latin{tree}; \latin{mūrus}, \english{wall}; \latin{amātor},
\english{lover}; \latin{vīta}, \english{life}.

\subsubsection

Pronouns, Adjectives, and Participles, when \emph{taking the place of
  Nouns}, are, like Nouns, called Substantives; e.g.\ \latin{hic},
\english{this man}; \latin{bonī}, \english{the good}; \latin{amāns},
\english{a lover}.

\subsubsection

The Verb-forms called Participles often express \emph{condition},
\emph{quality}, etc., and so have much in common with Adjectives.
Compare, e.g., \latin{fatīgātus}, \english{wearied}, with
\latin{fessus}, \english{weary}; and \latin{vir laudandus}, \english{a
  man to be praised}, with \latin{vir laudābilis}, \english{a
  praise-worthy man}.  In what follows, statements that are true both
of the Adjective and of the Participle will be given in the treatment
of the former.

\subsubsection

The last four Parts of Speech, the Adverb, Preposition, Conjunction,
and Interjection, are often called \term{Particles}.

\begin{minor}

\subsubsection

Latin has no article.

\end{minor}

\headingB{Simple, Compound, and Complex Sentences}

\section

Generally, one or more Verbs are either expressed or clearly
understood in every sentence.

\begin{minor}

\subsubsection

Certain verbs which can easily be supplied are often omitted.  Thus
\latin{dīcō}, \latin{loquor}, \latin{agō}\versionB*{, and the auxiliary
  \latin{est} (\latin{sunt}, etc.)}.
\begin{examples}

\latin{sīc Venus}, \english{thus \emph{\(spoke\)} Venus};
 \apud{Aen.}{1, 325}.

\end{examples}

\subsubsection

Occasionally a sentence does not admit of a verb.

\begin{examples}

\latin{ō tempora, ō mōrēs!}
\english{O the times, O the ways of men!}
\apud{Cat.}{1, 1, 2}.

\latin{ō fortūnātam rem pūblicam!}
\english{O happy Commonwealth!}
\apud{Cat.}{2, 4, 7}.

\end{examples}

\end{minor}

\section
\subsection

A \term{Simple Sentence} is one that contains not more than a single
Finite Verb.
\begin{examples}

\latin{dīcit līberius},
\english{he speaks with more freedom};
\apud{B.~G.}{1, 18, 2}.

\end{examples}

\subsection

A \term{Compound Sentence} is one that consists of two or more Simple
Sentences \emph{of the same rank}, called \term{Coördinate}.

\begin{examples}

\latin{D.\ Brūtum classī praeficit, et in Venetōs proficīscī iubet},
\english{he appoints Decimus Brutus to the command of the fleet, and
  orders him to proceed to the country of the Veneti};
\apud{B.~G.}{3, 11, 5}.
(\latin{Praeficit} and \latin{iubet} are Coördinate.)

\end{examples}

\subsection

A \term{Complex Sentence} is one in which, in addition to one or more
simple sentences, there are one or more sentences \emph{of inferior
  rank}, called \term{Subordinate} or \term{Dependent}.

\begin{examples}

\latin{quod iussī sunt, faciunt},
\english{they do what they have been told \(to do\)};
\apud{B.~G.}{3, 6, 1}.
(\latin{Quod iussī sunt} is a \emph{Dependent} Sentence, while
\latin{faciunt} is the \emph{Main}, or \emph{Principal}, Sentence.)

\end{examples}

\headingB{Clauses and Phrases}

\section

In a Complex Sentence,

\subsection

The Independent Sentences are called \term{Main}, or \term{Principal},
\term{Sentences}; while the Dependent Sentences are generally
distinguished by being called \term{Dependent}, or \term{Subordinate},
\term{Clauses}.  Thus, in \latin{quod iussī sunt, faciunt},
\english{they do what they have been told \(to do\)}, \latin{faciunt}
is called a Principal Sentence, and \latin{quod iussī sunt} a
Dependent Clause.

\subsection

But, for convenience, the word Clause is sometimes used of the main
sentence also, so that one speaks of \term{Principal Clauses} as well
as of \term{Dependent Clauses}.

\subsubsection

The word Clause is confined to members of a sentence that contain a
Finite Verb (\xref{146}) or an Infinitive (cf.\ \xref[\emph{a}]{238}).

\subsubsection

A Phrase is a group of associated words not containing a Finite Verb
or an Infinitive.
\begin{examples}

\latin{hominēs magnae virtūtis},
\english{men of great courage};
\apud{B.~G.}{2, \emend{134}{15}{14}, 5}.
(\latin{Magnae virtūtis} is a Phrase.)

\latin{ūnā ex parte},
\english{on one side};
\apud{B.~G.}{1, 2, 3}.

\end{examples}

\section

Clauses, like sentences, may be Coördinate; \emph{or} one may be Dependent
upon another.
\begin{examples}

\latin{huic mandat Belgās adeat atque in officiō contineat},
\english{\emph{\(Caesar\)} instructs him to go to the Belgians and
  hold them to their allegiance};
\apud{B.~G.}{3, 11, 2}.
(\latin{Adeat} and \latin{contineat} are Coördinate.)

\latin{equitātum praemittit, quī videant quās in partīs hostēs iter
  faciant},
\english{he sends the calvary ahead, to find out in what direction the
  enemy are moving};
\apud{B.~G.}{1, 15, 1}.
(\latin{Quās\ellipsis faciant} is Subordinate to \latin{quī
  videant},—which itself is Subordinate to \latin{praemittit}.)

\end{examples}

\headingB{Dependence and Semi-Dependence (or Parataxis)}

\contentsentry{C}{Dependence and Semi-Dependence}

\section

The term Dependence, or Subordination, as used in grammar, means
\emph{dependence both in thought and in form}.

Thus in \latin{quod advēnit, gaudeō}, \english{I am glad because he
  has come}, not only the obvious thought, but the form of the clause,
show the dependence of \latin{advēnit}, \english{he has come}, upon
\latin{gaudeō}, \emph{I am glad}.

\section

Semi-Dependence, or Parataxis,\footnote{A Greek word meaning
  \emph{setting side by side}.} is \emph{dependence in thought, with
  independence in form}.

Thus in \latin{advēnit: gaudeō}, \english{he has come: I am glad},
\latin{advēnit} is really dependent upon \latin{gaudeō} (I am glad
\emph{because} he has come), though there is nothing in the form to
show this.

\begin{minor}

\subsubsection

Almost all dependent clauses have passed through the middle stage of
Parataxis.  Thu \latin{eās\footnote{\latin{Eās} is dependent, not
    paratactic.  Cf.\ \xref[3, \emph{a}, 2)]{501}.} necesse est},
\english{it is necessary that you go}, must have come down from a
paratactic stage, \latin{eās: necesse est}, \english{go: it is
  necessary}.

\subsubsection

In passing into the dependent form a sentence often shifts its meaning
somewhat, to fit the closer relationship in which it stands in the new
form.  Thus the (original) paratactic combination \latin{mē ēripiam:
  nē causam dīcam} must have meant \english{I will save myself: I will
  not plead my cause}; while \latin{mē ēripiam nē causam dīcam} means
\english{I will save myself from pleading my cause}.  Cf.\ \latin{nē
  causam dīceret sē ēripuit}, \apud{B.~G.}{1, 4, 2}.

\end{minor}

\headingB{Classification of Sentences and Clauses}

\section

Every Sentence or Clause \emph{declares}, \emph{assumes},
\emph{inquires}, or \emph{exclaims}.

\subsection

It declares (tells) something (\emph{Declarative} Sentence or Clause).
\begin{examples}

\latin{veniat}, \english{let him come};
\latin{utinam veniat},
\english{I wish he would come};
\latin{venit}, \english{he is coming}.

\end{examples}

\begin{note}

To declare is to \emph{make known}.  Thus in the above, the various
verbs declare respectively the speaker’s \emph{will}, his
\emph{desire}, and his \emph{perception of a fact}.  A \emph{Dependent
  Clause} may likewise declare.  Thus in \latin{dīc ut veniat},
\english{tell him that he is to come}, \latin{veniat} \emph{declares}
the speaker’s \emph{will} (he \emph{is to come}).

\end{note}

\subsection

It assumes something as a condition for something else
(\emph{Conditional}, or \emph{Assumptive},\footnote{The word
  “conditional” is convenient, as being in common use.  The word
  \emph{assumptive}, as corresponding to the verb \emph{assume}, would
  be more exact.  Cf.~\xref{573}.} Sentence or Clause).
\begin{examples}

\latin{sī venit},
\english{if he is coming};
\latin{sī veniat},
\english{if he should come};
\latin{quisquis vēnerit, occīdētur},
\english{whoever comes will be killed}
(i.e.\ \textsc{if} any man comes, he will be killed).

\end{examples}

\subsection

It inquires or exclaims about something (\emph{Interrogative} or
\emph{Exclamatory} Sentence or Clause).
\begin{examples}

\latin{venit?}
\english{is he coming?}
\latin{fortis est?}
\english{is he brave?}
\latin{quam fortis est!}
\english{how brave he is!}

\end{examples}

\begin{minor}

\subsubsection

Interrogative and Exclamatory sentences, if the latter contain a verb,
have the same form in Latin.  It is therefore best to treat them
together.

\subsubsection

All true Dependent Clauses introduced by a Relative (\latin{quī},
etc.), or by any Conjunction implying a Relative idea
(\latin{quotiēns}, \latin{cum}, \latin{dum}, \latin{antequam},
\latin{postquam}, etc.), are necessarily confined to the first two
uses, i.e.\ they are either \emph{Declarative} or \emph{Conditional};
for it is impossible to inquire or exclaim in a really dependent
Relative Clause.

\end{minor}

\headingB{Subject and Predicate}

\contentsentry{C}{Subject and Predicate; Predicate Noun, Adjective, or
  Pronoun}

\section

The \term{Subject} is that about which something is declared, assumed,
or asked.  That which is declared, assumed, or asked, is called the
\term{Predicate}.\footnote{The word Predicate is derived from
  \latin{praedicō}, \english{predicate}, \english{assert}.}
\begin{examples}

\latin{Caesar respondit},
\english{Caesar answered};
\apud{B.~G.}{1, 14, 1}.
(\latin{Caesar} is the Subject, and \latin{respondit} the Predicate.)

\end{examples}

\subsubsection

The Predicate is often omitted, especially if formed from the verb
\latin{sum}.
\begin{examples}

\latin{quot hominēs, tot sententiae},
\english{as many men, so many minds};
\apud{Ph.}{454}.

\end{examples}

\headingB{Predicate Noun, Adjective, or Pronoun}

\section

A Noun, Adjective, or Pronoun forming a part of that which is
predicated is called a \term{Predicate} Noun, Adjective, or Pronoun.
\begin{examples}

\latin{hōrum omnium fortissimī sunt Belgae},
\english{of all these, the Belgians are the bravest};
\apud{B.~G.}{1, 1, 3}.
(The idea “bravest” is as much predicated as is the idea “are.”)

\end{examples}

\begin{minor}

\subsubsection

The verb \latin{sum}, when thus joining a predicate word with its
subject, is called a \term{Copula} (i.e. “joiner”).

\subsubsection

Participles employed as Adjectives (\xref{248}) are often used
predicatively, true Participles very rarely.

\end{minor}

\headingB{Forms of Interrogative Sentences}

\contentsentry{C}{Questions and Answers; Alternative Questions;
  Rhetorical Questions}

\section

Questions are of two main kinds:

\subsection

Questions of the whole sentence (\emph{“yes” or “no” questions}).
Of these there are four possible forms:
\begin{enuma}

\item
Without introductory word, as in English:
\begin{examples}

\latin{vīs pugnāre?}
\english{do you want to fight?}
\apud{Rud.}{1011}.

\latin{nōn sentīs?}
\english{do you not see?}
\apud{Cat.}{1, 1, 1}.

\end{examples}

\item
With the neutral enclitic \enclitic{-ne} (implying nothing about the
answer).  The enclitic is attached to the emphatic word:
\begin{examples}

\latin{voltisne eāmus vīsere?}
\english{do you wish that we should go and call upon her?}
\apud{Ph.}{102}.

\end{examples}

\begin{note}[Note 1]

The neutral enclitic \enclitic{-ne} is occasionally used where the
context makes it clear what the answer \emph{must be}.  Thus
\latin{vidētisne ut apud Homērum?}  \apud{Sen.}{10, 31} (answer
“yes”); \latin{potestne tibi huius caelī spīritus esse iūcundus?}
\apud{Cat.}{1, \emend{135}{16}{6}, 15} (answer “no”).

\end{note}

\begin{note}[Note 2]

In poetry, \enclitic{-ne} is sometimes attached to interrogative
words.  Thus \latin{quōne malō?} \english{by what curse?}
\apud{Sat.}{2, 3, 295}.

\end{note}

\begin{note}[Note 3]

\enclitic{-ne} sometimes loses its \phone{e}, especially in early
Latin.  Thus
\latin{ain?}\ (for \latin{aisne?} for loss of~\phone{s}, see~\xref[12]{49}),
\latin{audīn?}\ (for \latin{audīsne?}),
\latin{itan?}\ (for \latin{itane?}),
\latin{satin?}\ (for \latin{satisne?}),
\latin{scīn?}\ (for \latin{scīsne?}),
\latin{viden?}\ (for \latin{vidēsne?}; for the quantity, see
    \xref[note]{28}),
\latin{vīn?}\ (for \latin{vīsne?}),
\latin{utin} (for \latin{uti-ne}, from \latin{uti}, a by-form of \latin{utī},
\latin{ut}, as in \latin{uti-nam}, \latin{uti-que}).
Similarly \latin{Pyrrhīn} (=~\latin{Pyrrhīne}), \apud{Aen.}{3, 319}.

\end{note}

\item

With \latin{nōnne}, implying the answer “yes”:
\begin{examples}

\latin{Mithridātēs nōnne ad Cn.\ Pompeium lēgātum mīsit?}
\english{did not Mithridates send an ambassador to Gnaeus Pompey?}
\apud{Pomp.}{16, 46}.

\end{examples}

\item

With \latin{num}, implying the answer “no”:
\begin{examples}

\latin{num negāre audēs?}
\english{you dare not deny, do you?}
\apud{Cat.}{1, 4, 8}.

\end{examples}

\end{enuma}

\subsection

Questions \emph{of detail}.
\begin{examples}

\latin{quid exspectās?}
\english{what are you looking for?}
\apud{Cat.}{2, 8, 18}.

\latin{cūr tam diū loquimur?}
\english{why do we talk so long?}
\apud{Cat.}{2, 8, 17}.

\end{examples}

\begin{note}

\latin{Tandem}, or the interrogative enclitic \enclitic{-nam}, may be
added to the simple interrogative to strengthen it; thus
\latin{quousque tandem?}
\english{how long, pray?}
\apud{Cat.}{1, 1, 1};
\latin{quibusnam manibus?}
\english{with what hands, pray?}
\apud{B.~G.}{2, \emend{136}{30}{29}, 4}.

\end{note}

\chapter{Forms of Answers to “Yes” or “No” Questions}

\section
\subsection

“Yes” may be expressed by repeating the Verb; or, less formally, by
\latin{ita}, \latin{sīc}, \latin{etiam}, \latin{vērō}, \latin{certō},
\latin{sānē}, etc.
\begin{examples}

\latin{“fuistīn līber?” “Fuī,”}
\english{“were you a free man?” “I was”};
\apud{Capt.}{628}.

\latin{“illa maneat?” “Sīc,”}
\english{“is she to remain?” “Yes”};
\apud{Ph.}{813}.

\end{examples}

\subsection

“No” may be expressed by repeating the Verb and adding a negative;
or, less formally, by \latin{nōn}, \latin{minimē}, etc.
\begin{examples}

\latin{“nōn ego illī argentum redderem?” “Nōn redderēs,”}
\english{“should I not have paid him the money?”  “You should not
  have paid him”};
\apud{Trin.}{133}.

\latin{“ea praeteriit?” “Nōn,”}
\english{“has that \emph{(day)} passed?” “No”};
\apud{Ph.}{525}.

\end{examples}

\section

An answer correcting or heightening the force of a preceding question
is introduced by \latin{immō}, \english{on the contrary}, \english{why
  even}.
\begin{examples}

\latin{vīvit?  Immō vērō etiam in senātum venit!}
\english{lives, do I say?  Why!\ he even comes into the senate!}
\apud{Cat.}{1, 1, 2}.

\end{examples}

\pagebreak

\chapter{Alternative Questions}

\section

\latin{Alternative Questions}, or questions that offer the hearer or
reader two or more things to choose among, are expressed as follows:
\begin{Tabular}{r@{\enskip}l@{ }l}

I.   & With \latin{utrum}\dots, & \latin{an}\dots \\
II.  & With \latin{-ne}\dots,   &  \latin{an}\dots \\
III. & With \na\dots,           &  \latin{an}\dots

\end{Tabular}

\pushright\1{III}

\begin{examples}

\1{I}.
\latin{haec utrum lēx est, an lēgum omnium dissolūtiō?}
\english{\emph{\(whether\)} is this a law, or an undoing of all laws?}
\apud{Phil.}{1, 9, 21}.

\1{II}.
\latin{Rōmamne veniō, an hīc maneō, an Arpīnum fugiam?}
\english{do I come to Rome, or stay here, or shall I flee to Arpinum?}
\apud{Att.}{16, 8, 2}.

III.
\latin{prīvātim servitūtem servit, an pūblicam?}
\english{is he a slave to a private person, or to the state?}
\apud{Capt.}{334}.

\end{examples}

\subsubsection

If the second part of the question is \emph{negatived}, \latin{nōn},
\english{not}, is added to \latin{an}, making \latin{an nōn}
(\latin{annōn}), \english{or not}.  The verb is regularly omitted.  In
an Indirect Question (\xref{537}), \latin{necne} may also be used
instead of \latin{an nōn} (rarely in a Direct one).

\begin{examples}

\latin{pater eius rediit an nōn?}
\english{has his father returned or not?}
\apud{Ph.}{147}.

\latin{quaesīvī in conventū fuisset necne},
\english{I asked whether he had been at the meeting or not};
\apud{Cat.}{2, 6, 13}.

\end{examples}

\subsubsection

In the Indirect Question, the forms \latin{utrum}\dots, \latin{-ne} and
\latin{\na, -ne} sometimes occur; also, in poetry, \latin{-ne}\ellipsis,
  \latin{-ne} (as in \apud{Aen.}{5, 702}\versionB*{ and \apud{}{1, 308}}).

\chapter{Rhetorical Questions, E\lowercase{\textsc{tc}}.}

\section

Questions that do not really ask for information, but are only
stronger ways of \emph{declaring} something, are called
\term{Rhetorical Questions}.
\begin{examples}

\latin{quis dubitat?}
\english{who doubts?}
(= nobody doubts).

\latin{quis dubitet?}
\english{who would doubt?}
(= nobody would doubt).

\latin{cūr dubitem?}
\english{why should I doubt?}
(= I ought not to doubt).

\latin{quid prōdest?}
\english{what is the use?}
(= there is no use).

\end{examples}

\section

An \term{Absurd Question} is often introduced by \latin{an} alone.
\begin{examples}

\latin{an vērō Catilīnam perferēmus?}
\english{are we really going to tolerate Catiline?}
\apud{Cat.}{1, 1, 3}.

\end{examples}

\section

A \term{Question Suggesting the Probable Answer} may be introduced by
\latin{an}.
\begin{examples}

\latin{cuium pecus?  An Meliboeī?}
\english{whose flock?  That \emph{\(perhaps\)} of Meliboeus?}
\apud{Ecl.}{3, 1}.

\end{examples}

\chapter{Substantive Clauses}

\contentsentry{C}{Substantive Clauses; Adverbial Clauses}

\section

Indicative, Subjunctive, and Infinitive Clauses are often used
\term{Substantively} (i.e.\ in some \emph{case}-relation in the
sentence).
\begin{examples}

\latin{ut nē addam quod ingenuam nactus es},
\english{not to add that you have now a freeborn wife};
\apud{Ph.}{168}.
(\latin{Quod nactus es} is the Object of \latin{addam}.)

\latin{placuit eī, ut ad Ariovistum lēgātōs mitteret},
\english{it seemed best to him that he should send ambassadors to
  Ariovistus};
\apud{B.~G.}{1, 34, 1}.
(\latin{Ut\ellipsis mitteret} is the Subject of \latin{placuit}.)

\latin{lēgātōs mittī placet?}
\english{does it seem best that ambassadors be sent?}
\apud{Phil.}{5, 9, 25}.

\latin{placuit experīrī},
\english{it seemed best to try};
\apud{Caecin.}{7, 20}.

\end{examples}

\begin{minor}

\subsubsection

The Infinitive in such relations, even when standing alone, is the
equivalent of a clause.  No line can be drawn in the above between the
three subjects of \latin{placuit} or \latin{placet}.

\end{minor}

\chapter{Adverbial Clauses}

\section

Clauses modifying Verbs are called \term{Adverbial}.
\begin{examples}

\latin{nec enim, dum eram vōbīscum, animum meum vidēbātis},
\english{for while I was with you, you did not see my soul};
\apud{Sen.}{22, 79}.
(\latin{Dum eram vōbīscum} is attached, like an \emph{Adverb of
  \emend{80}{t}{T}ime}, to \latin{vidēbātis}.)

\end{examples}

\numchapter{The Parts of Speech in Detail}

\contentsentry{B}{The Parts of Speech in Detail: Nouns, Adjectives,
  Pronouns, Verbs, Adverbs, Prepositions, Conjunctions, Interjections}

\chapter*{Nouns}

\section

Nouns are divided into the following kinds:

\subsection

\term{Proper Nouns} denote \emph{particular} persons, places, or
things, as \latin{Cicerō}, \english{Cicero}; \latin{Rōma},
\english{Rome}; \latin{Mausōlēum}, \english{the tomb of Mausolus}.

\subsection

\term{Common Nouns} denote \emph{any} person or thing of a given
\emph{class}, as \latin{senātor}, \english{senator}; \latin{servus},
\english{slave}; \latin{mīles}, \english{soldier}; \latin{urbs},
\english{city}; \latin{sepulcrum}, \english{tomb}.

\subsubsection

Proper Nouns are sometimes used like Common Nouns, as \latin{Catōnēs},
\english{men like Cato}; \apud{Am.}{6, 21}.

\subsubsection
Common Nouns are sometimes used like Adjectives, as \latin{victōrem
  exercitum}, \english{victorious army}; \apud{B.~G.}{7, 20, 12}.

\subsection

\term{Collective Nouns} denote a \emph{group} or \emph{class} of
persons or things, as \latin{senātus}, \english{senate} (collection of
senators); \latin{exercitus}, \english{army} (collection of soldiers).

\subsection

\term{Concrete Nouns} denote things that can be perceived by the
senses (sight, touch, hearing, etc.), as \latin{mūrus},
\english{wall}; \latin{aurum}, \english{gold}; \latin{sonus},
\english{sound}.

\subsection

\term{Abstract Nouns} denote things that cannot be perceived by the
senses, namely, qualities, states of mind, conditions, activities, and
the like, as \latin{virtūs}, \english{virtue}; \latin{sapientia},
\english{wisdom}; \latin{servitium}, \english{serfdom},
\english{slavery}.

\subsubsection

Abstract Nouns are occasionally used with concrete meaning.  Thus
\latin{servitia concitat}, \english{he is stirring up the slaves};
\apud{Cat.}{4, 6, 13}.

\subsubsection

The Plural of Abstracts is often used to express \emph{acts},
\emph{instances}, or \emph{kinds}.  Thus \latin{au\-dā\-ci\-ae},
\english{acts of insolence}; \apud{Cat.}{2, 5, 10}.

\versionB*{\subsubsection

In poetry, the Plural of either Abstract or Concrete Nouns is
sometimes used for the singular to produce a more striking effect.}

\subsubsection

The line between Concrete and Abstract Nouns is impossible to draw
sharply.  Thus \latin{animus}, \english{mind}, lies between the two.
\versionA*{Such \emph{intermediate} (or \emph{semi-abstract}) nouns are usually
classed as Concrete.}

\chapter*[Adjectives]{Adjectives (\lowercase{and}
  P\lowercase{articiples}, \lowercase{in} C\lowercase{ertain}
  P\lowercase{arallel} U\lowercase{ses})}

\headingA{Comparison}

\section
\subsection

The three degrees of Comparison have the same 
meanings as in English.

\subsection

But the Comparative is also used merely to indicate a \emph{higher}
degree of the quality or condition \emph{than is usual} (English
\emph{rather} or \emph{too}), as \latin{loquācior}, \english{rather
  talkative}; \latin{audācior}, \english{too bold}.

\subsection

The Superlative is used, more freely than in English, to indicate a
\emph{very high} degree of the quality or condition, as
\latin{loquācissimus}, \english{most talkative}, \english{very
  talkative}; \latin{ērudītissimus}, \english{very learned}.

\subsubsection

In this sense, the Superlative is often strengthened by the addition
of \latin{vel}, \english{even}; or \latin{ūnus}, \english{the one}.
Thus \latin{vel summa paupertās}, \english{even the greatest poverty};
\apud{Tusc.}{5, 39, 113}.

\subsection

To indicate the \emph{highest degree possible}, the Superlative is
accompanied by \latin{quam} with some form of \latin{possum}, or by
\latin{quam} alone.
\begin{examples}

\latin{nāvīs quam plūrimās possunt cōgunt},
\english{they collect as many ships as they can} (as many as
possible);
\apud{B.~G.}{3, 9, \emend{137}{9}{9–10}}.

\latin{quam plūrimās cīvitātīs},
\english{as many states as possible};
\apud{B.~G.}{1, 9, 3}.

\end{examples}

\section[Two Comparatives]

When an object is said to possess a quality in a higher degree than
some other quality (English \english{rather\ellipsis than}), both
Adjectives regularly take the same form.
\begin{examples}

\latin{pestilentia minācior quam perniciōsior},
\english{a plague that was alarming rather than destructive} (more
alarming than destructive);
\apud{Liv.}{4, 52, 3}.

\latin{magis invidiōsō crīmine quam vērō},
\english{on an accusation that was invidious rather than true} (more
invidious than true);
\apud{Verr.}{2, 46, 113}.

\end{examples}

\subsubsection

The uses of the Comparative Adverb correspond, as also for~\xref{241}.
Compare \xref{241} with \xref{300}, and \xref{242} with~\xref{301}.

\headingB{Special Uses of Certain Adjectives and Participles}

\section

The Romans used the Adjectives \latin{prior}, \latin{prīnceps},
\latin{prīmus}, \latin{postrēmus}, and \latin{ultimus} to express the
idea of \english{first}, or \english{last}, \english{to do a thing}.
\begin{examples}

\latin{ea prīnceps poenās persolvit},
\english{this was the first to pay the penalty};
\apud{B.~G.}{1, 12, 6}.

\end{examples}

\pagebreak

\section

The Romans used certain Adjectives to denote a \english{part}.  Thus:
\begin{mexamples}[3]

\latin{prīmus}, \english{first}

\latin{postrēmus}, \english{last}

\latin{extrēmus}, \english{outermost}

\latin{summus}, \english{topmost}

\latin{īnfimus} (\latin{īmus}), \english{lowest}

\latin{intimus}, \english{innermost}

\latin{medius}, \english{middle}

\latin{sērus}, \english{late}

\latin{multus}, \english{much}

\end{mexamples}

\begin{examples}

\latin{summus mōns},
\english{the top of the mountain};
\apud{B.~G.}{1, 22, 1}.

\latin{multō diē},
\english{late in the day} (in the late part of the day);
\apud{B.~G.}{1, 22, 4}.

\latin{prīmō impetū},
\english{at the beginning of the attack};
\apud{B.~G.}{2, \emend{101}{24}{23}, 1}.

\end{examples}

\subsubsection

This use must be carefully distinguished from the ordinary one, as in
\latin{ante prīmam vigiliam}, \english{before the first watch};
\apud{B.~G.}{7, 3, 3}.

\section

The Romans generally used certain Adjectives and Participles where we
use Adverbs.  The most common of these are:
\begin{mexamples}[3]

\latin{sciēns}, \english{witting\(ly\)}

\latin{īnsciēns}, \english{unwitting\(ly\)}

\latin{libēns}, \english{willing\(ly\)}

\latin{invītus}, \english{unwilling\(ly\)}

\latin{laetus}, \english{glad\(ly\)}

\latin{maestus}, \english{sorrowful\(ly\)}

\latin{assiduus}, \english{constant\(ly\)}

\latin{praeceps}, \english{headlong}

\latin{frequēns}, \english{in great numbers}

\end{mexamples}

\begin{examples}

\latin{laetī pergunt},
\english{proceed joyfully};
\apud{B.~G.}{3, \emend{125}{18}{16}, 8}.

\latin{frequentēs vēnērunt},
\english{came in great numbers};
\apud{B.~G.}{4, 13, 4}.

\end{examples}

\section

When \latin{multus} is used with an Adjective or Participle expressing
quality, the two are generally connected by a word meaning “and.”
\begin{examples}

\latin{multīs gravibusque vulneribus},
\english{with many dangerous wounds};
\apud{B.~G.}{2, \emend{100}{25}{24}, 1}.

\latin{multīs ac summīs virīs},
\english{to many influential men};
\apud{Cat.}{1, 4, 10}.

\end{examples}

\section

A Distributive Numeral is used instead of a Cardinal:

\subsection

If its Noun is Singular in meaning, though Plural in form
(\xref[4]{104}; \xref{105}).  Thus \latin{duās epistulās} or
\latin{bīnās litterās}, \english{two letters} (of correspondence.
\latin{Duās litterās} would mean \english{two letters of the
  alphabet}).

\subsubsection
For \english{three}, \latin{trīnī}, not \latin{ternī}, is used with
such a noun.

\subsubsection
For \english{one}, \latin{ūnī} is used (not \latin{singulī}), as
\latin{ūnās litterās}, \english{one letter}.

\subsection

Usually in multiplication, as \latin{bis bīnī}, \english{twice two}.

\subsection

Occasionally in poetry with the meaning of the corresponding Cardinal,
as in \latin{centēnās manūs}, \english{a hundred hands};
\apud{Aen.}{10, 566}.

\chapter{The Participle as Adjective}

\section

Participles are often used as Adjectives.  But in Ciceronian Latin the
only \english{Future Active} Participles thus used are \latin{futūrus}
and \latin{ventūrus}.
\begin{examples}

\latin{acūtus et prōvidēns},
\english{intelligent and farsighted};
\apud{Fam.}{6, 6, 9}.

\latin{opīniō ventūrī bonī},
\english{an impression of coming good};
\apud{Tusc.}{4, 7, 14}.

\end{examples}

\pagebreak

\chapter{Adjective or Participle as Substantive}

\section

Certain Adjectives and Participles are used as Substantives.

\subsection

In the Singular Number, the \emph{Masculine} denotes a class of
persons, the \emph{Neuter} a quality, or a corresponding abstract
idea.
\begin{mexamples}

\latin{iūstus}, \english{the just man}

\latin{timidus}, \english{the timid man}

\latin{iūstum}, \english{that which is just}, \english{justice}

\latin{timēns}, \english{the man that fears}

\end{mexamples}

\subsection

In the Plural, the \emph{Masculine} denotes a class of persons, the
\emph{Neuter} either a class of things or a number of instances of a
quality.
\begin{mexamples}

\latin{iūstī}, \english{the just}

\latin{bonī}, \english{the good}

\latin{doctī}, \english{the learned}, \english{scholars}

\latin{iūsta}, \english{due ceremonies} (just things)

\latin{bona}, \english{good things}, \english{goods}, \english{blessings}

\latin{praeterita}, \english{past things}, \english{the past}

\end{mexamples}

\section

In prose, the Substantive uses of the Adjective and Participle are
confined within certain limits, as follows:

\subsection

In the Singular:
\begin{enuma}

\item

The \emph{Masculine}\footnote{The uses of the Feminine correspond for all the
  constructions of this section, but examples are rare.} of the
\emph{Participle} is freely used in any Case except the Nominative and
Vocative, and in any construction.

\item

The \emph{Masculine} of the \emph{Adjective} is freely used in the
Predicate Genitive only (\xref{340}).  If it is of the Third
Declension, this construction is \emph{preferred} to that of the
Predicate Nominative.
\begin{examples}

\latin{dēmentis est},
\english{it is the part of a madman} (not \latin{dēmēns est});
\apud{Off.}{1, 24, 83}.

\end{examples}

\item

The \emph{Masculine Nominative} of either the \emph{Adjective} or the
\emph{Participle} is rare in prose, unless modified by a Pronoun
(\latin{hic}, \latin{quisque}, etc.); but it is freely employed by the
poets.
\begin{examples}

\latin{semper avārus eget},
\english{the miserly is always poor};
\apud{Ep.}{1, 2, 56}.

\end{examples}

\begin{note}[Note 1]

In place of using the Masculine Nominative Adjective alone, the prose
writers generally couple it with \latin{vir} or \latin{homō}, as in
\latin{vir bonus}, \english{the good man}; \apud{Tusc.}{5, 16, 48}.

\end{note}

\begin{note}[Note 2]

In place of using the Masculine Nominative Participle alone, the prose
writers generally use \latin{is quī}, e.g.\ \latin{is quī timent},
\english{the man who fears}; \apud{Leg.}{1, 14, 41}.

\end{note}

\item

The \emph{Neuter} of the \emph{Adjective} is freely used in any Case,
but is especially frequent with Prepositions and in the Genitive of
the Whole (\xref{346}).
\begin{examples}

\latin{in angustō}, \english{in straits};
\apud{B.~G.}{2, \emend{102}{25}{24}, 1}.

\latin{sine dubiō}, \english{without doubt}; \apud{Cat.}{2, 1, 1}.

\latin{nihil solidī}, \english{nothing solid}; \apud{N.~D.}{1, 27, 75}.

\end{examples}

\begin{note}

This Genitive is rare with Adjectives of the \emph{Third} Declension
(\xref[\emph{a}]{346}).

\end{note}

\end{enuma}

\subsection

In the Plural:

\begin{enuma}

\item

The \emph{Masculine} of either the Adjective or the Participle is
freely used in any Case and in any construction.
\begin{examples}

\latin{cognōvit montem ā suīs tenērī},
\english{learned that the mountain was held by his men};
\apud{B.~G.}{1, 22, 4}.

\latin{quī leviter aegrōtantīs lēniter cūrant},
\english{who cure the moderately sick by mild methods};
\apud{Off.}{1, 24, 83}.

\end{examples}

\item

The \emph{Neuter} is generally avoided except in the Nominative and
Accusative, in which the form makes the Gender clear.  In other Cases,
the Noun \latin{rēs}, with the Adjective in agreement, is generally
preferred.
\begin{examples}

\latin{omnia}, \english{all things}, \apud{Cat.}{1, 13, 32};
but \latin{omnium rērum}, \english{of all things}, \apud{Pomp.}{9, 22}.

\end{examples}

\end{enuma}

\begin{note}

Perfect Passive Participles used as Substantives may retain the
verb-feeling sufficiently to be modified by an Adverb, or they may
completely become Nouns, and so have an Adjective agreeing with them.
Thus
\latin{facta rēctē}, \english{deeds well done},
\apud{Cat.}{3, \emend{138}{13}{12}, 27}; 
but
\latin{improbīs factīs}, \english{evil deeds}, \apud{Fin.}{1, 16, 51}.
Similarly in the Singular.

\end{note}

\section

Many words which came to be used as simple Nouns were originally
Adjectives or Participles.  Thus:
\begin{mexamples}

\latin{amīcus}, \english{friend}

\latin{inimīcus}, \english{enemy}

\latin{propinquus}, \english{relation}

\latin{pār} (an even thing), \english{a pair}

\latin{dextra}, \english{the right hand}

\latin{sinistra}, \english{the left hand}

\latin{hīberna}, \english{winter quarters}

\latin{īnstitūtum}, \english{institution}

\end{mexamples}

\section

Rarely, a Perfect Passive Participle is used \emph{impersonally}
(\xref{287}) as a Noun.
\begin{examples}

\latin{nōtumque furēns quid fēmina possit}, \english{and the knowledge
  to what lengths a woman in wrath may go}; \apud{Aen.}{5, 6}.
(\latin{Nōtum} = \latin{nōtitia}.)

\end{examples}

\headingB{Pronouns and Corresponding Adjectives}

\pushright\1{VIII}

\section

Pronouns and corresponding Adjectives are divided into the following
classes:
\begin{multicols}{2}

\parindent0pt

\1{I}. Personal, and Personal Possessive

\1{II}. Reflexive, and Reflexive Possessive

\1{III}. Reciprocal

\1{IV}. Intensive

\1{V}. Identifying

\1{VI}. Determinative-Descriptive

\1{VII}. Interrogative

\1{VIII}. Indefinite

\1{IX}. Collective

\1{X}. Distributive

\1{XI}. Pronominal Adjectives

\1{XII}. Relative

\end{multicols}

\pagebreak

\chapter{I. The Personal Pronouns and the Corresponding Possessive
  Pronouns}

\section

The Personal Pronouns denote \emph{persons}, with no further idea
(\english{I}, \english{you}, etc.).  The Possessive Pronouns denote
persons as \emph{possessors} (\english{mine}, \english{your}, etc.).

\subsubsection

The Personal Genitives in~\suffix{-ī} (\latin{meī}, \latin{tuī},
\latin{suī}, \latin{nostrī}, and \latin{vestrī}) are generally
Objective (\xref{354}); while \latin{nostrum} and \latin{vestrum} are
Genitives of the Whole (\xref{346}).
\begin{examples}

\latin{memoriam nostrī},
\english{the recollection of us};
\apud{Sen.}{22, 81}.

\latin{ūnumquemque nostrum},
\english{every one of us};
\apud{Cat.}{1, 1, 2}.

\end{examples}

\begin{minor}

\subsubsection

But the form in \suffix{-um} is \emph{occasionally} used Objectively, and is
\emph{regularly} used with \latin{omnium}, whatever the construction.
Thus \latin{omnium nostrum salūtem}, \english{the safety of us all};
\apud{Cat.}{1, 6, 14}.

\end{minor}

\section

Latin has no true \emph{Personal Pronoun of the Third Person}
(\english{he}, \english{she}, etc.).  The place of this is supplied by
one of the Determinative Pronouns (\xref{271}),—most frequently
by~\latin{is}.
\begin{examples}

\latin{Helvētiī lēgātōs ad eum mīsērunt},
\english{the Helvetians sent ambassadors to him};
\apud{B.~G.}{1, 27, 1}.

\end{examples}

\section

The place of the \emph{Possessive Pronoun of the Third Person} is
supplied by the Genitive of one of the Determinative Pronouns
(\xref{271}),—most frequently of~\latin{is}.
\begin{examples}

\latin{cōnsiliō eius},
\english{by his plan} (the plan of him);
\apud{B.~G.}{4, 21, 5}.

\end{examples}

\section

Since the form of the Verb shows its person, the Personal Pronouns are
generally not expressed as Subjects.  But they are necessarily
expressed when \emph{emphasis} or \emph{contrast} is intended.
\begin{examples}

\emph{Not expressed}:
\latin{polliceor},
\english{I promise};
\apud{Cat.}{1, 13, 32}.

\emph{Expressed for emphasis}:
\latin{ego cūrābō},
\english{\emph{I} will attend to it};
\apud{Ph.}{713}.

\emph{Expressed for contrast}:
\latin{tuos (= tuus) est damnātus gnātus, nōn tū};
\english{it was \textsc{your son} that was condemned, not \textsc{you}};
\apud{Ph.}{422}.

\end{examples}

\subsubsection

\latin{Quidem} is often added to \latin{tū} for still further
emphasis.  \latin{Equidem} is mostly confined to the first person, and
the pronoun is not expressed.  Thus \latin{haud equidem
  adsentior\dots; persevērās tū quidem}, \english{I do not
  assent\dots; \textsc{you} keep up}; \apud{Leg.}{3, 11, 26}.

\section

The Possessive Pronouns are generally not expressed if the meaning is
clear without them.  But they are necessarily expressed where
\emph{clearness} requires, or where \emph{emphasis} or \emph{contrast}
is intended.

When expressed for clearness, they, like Adjectives, follow their
Nouns.
\linebreak
When expressed for emphasis or contrast, they, like Adjectives
under the same circumstances, precede their Nouns.
\begin{examples}

\emph{Not expressed}:
\latin{fīlium nārrās mihi?}
\english{do you talk to me of \(my\) son?}
\apud{Ph.}{401}.

\emph{Expressed for clearness}:
\latin{operā tuā ad restim mihi rēs redit},
\english{through \textsc{your} doing it has come to be a hanging
  matter for me};
\apud{Ph.}{\emend{139}{685}{685–686}}.

\emph{Expressed for emphasis}:
\latin{mī patrue!}
\english{\textsc{my dear} uncle!}
\apud{Ph.}{254}.

%%* loose line

\emph{Expressed for contrast}:
\latin{nostran culpa ea est an iūdicum?}
\emph{is it \textsc{our} fault or the \textsc{judges’}?}
\apud{Ph.}{275}.

\end{examples}

% \pagebreak

\section

\emph{Plural of Dignity}.  The Personal or Possessive Pronoun of the
First Person is often used in the Plural instead of the Singular, for
greater dignity.
\begin{examples}

\latin{ad senātum referēmus},
\english{we \emph{\(= I\)} shall refer \emph{\(other matters\)} to the
  senate};
\apud{Cat.}{2, 12, 26}.

\end{examples}

\chapter{II. The Reflexive Pronouns and\\ the Corresponding Possessive
  Pronouns}

\section

The Reflexive Pronouns and corresponding Possessives denote persons
who are also the Subject of the Verb (as in \emph{\textsc{I} love
  \textsc{myself}}, \emph{\textsc{you} love \textsc{your} son}), or of
an activity expressed by an Adjective or Noun.
\begin{examples}

\latin{sē alunt},
\english{they support themselves};
\apud{B.~G.}{4, 1, \emend{140}{5}{4}}.

\latin{cōnservātiō suī},
\english{the saving of himself};
\apud{Fin.}{5, 13, 37}.

\end{examples}

\subsubsection

In the \emph{First} and \emph{Second Persons}, the Reflexive Pronouns and
Possessives are identical with the Personal Pronouns and Possessives
(\latin{ego}, \latin{meus}, \latin{tū}, \latin{tuus}, etc.).  Thus
\latin{mē amat}, \english{he loves me}, and \latin{mē amō}, \english{I
  love myself} (I love me).

\subsubsection

In the \emph{Third Person}, the Reflexive Pronoun is \latin{sē} (or
\latin{sēsē}), and the Reflexive Possessive is \latin{suus}.  (For
\latin{ipse} as Reflexive, see~\xref{263}.)

\section

The Latin Reflexive Pronouns and corresponding Possessives are
generally not accompanied by any word corresponding to our English
“self.”
\begin{examples}

\latin{mē dēfendī},
\english{I have defended myself};
\apud{Cat.}{1, 5, 11}.

\latin{sē ex nāvī prōiēcit},
\english{he flung himself from the ship};
\apud{B.~G.}{4, 25, 4}.

\latin{suīs fīnibus eōs prohibent},
\english{they keep them from their territory};
\apud{B.~G.}{1, 1, 4}.

\end{examples}

\begin{minor}

\subsubsection

Yet \latin{ipse}, \english{self}, is sometimes added.  See~\xref{268}.

\end{minor}

\section

\latin{Sē} and \latin{suus} are used mainly in two ways:

\subsection

To refer to the Subject of the clause in which they stand.  (“Direct
Reflexive.”)
\begin{examples}

\latin{sē suaque omnia dēdidērunt},
\english{they surrendered themselves and all their possessions};
\apud{B.~G.}{2, \emend{141}{15}{14}, 2}.

\end{examples}

\subsection

To refer to the Subject of the \emph{main clause}, though themselves
standing in a subordinate clause. (“Indirect Reflexive.”)

This is possible only where the subordinate clause expresses the
thought of the Subject of the main clause.
\begin{examples}

\latin{hīs mandāvit ut quae dīceret Ariovistus ad sē referrent},
\english{he charged them to report to him what Ariovistus should say};
\apud{B.~G.}{1, 47, 5}.

\end{examples}

\begin{minor}

\subsubsection

Where the subordinate clause expresses the thought, not of the subject
of the main clause, but of the \emph{writer} or \emph{speaker},
\latin{is} is used, not \latin{sē}, and \latin{eius}, etc., not
\latin{suus}, to refer to that subject; for the \emph{idea} in this
case is not reflexive.
\begin{examples}

\latin{cum propter multās eius virtūtēs magnā cum dignitāte vīveret},
\english{since on account of his many virtues he was enjoying great
  authority};
\apud{Nep.\ Them.}{8, 2}.

\end{examples}

\end{minor}

\section

In a subordinate clause \latin{ipse} \emph{may} be used as a Reflexive
referring to the Subject of the \emph{main} clause, as follows:

\subsection

Where \emph{two} Reflexives are needed, referring to \emph{the same}
person or thing.
\begin{examples}

\latin{Ariovistus respondit: sī quid ipsī ā Caesare opus esset, sēsē ad
  eum ventūrum fuisse},
\english{Ariovistus replied that, if he himself had wanted anything
  from Caesar, he would have come to him};
\apud{B.~G.}{1, 34, 2}.

\end{examples}

\subsection

Where \emph{two} Reflexives are needed, referring to \emph{different}
persons or things.
\begin{examples}

\latin{cūr dē suā virtūte aut dē ipsīus dīligentiā dēspērārent?}
\english{\emph{\(C\emend{9}{æ}{ae}sar asked\)} why they should despair of their own
  valor or his vigilance};
\apud{B.~G.}{1, 40, 4}.

\end{examples}

\subsection

Where \latin{sē} or \latin{suus} would be ambiguous:
\begin{examples}

\latin{erat eī praeceptum ā Caesare nē proelium committeret, nisi
  ipsīus cōpiae prope hostium castra vīsae essent},
\english{he had been instructed by Caesar not to join battle, unless
  his \emph{\(Caesar’s\)} forces should be seen near the enemy’s
  camp};
\apud{B.~G.}{1, 22, 3}.

\end{examples}

\section

\latin{Sē}, \latin{suus}, and \latin{ipse} gain the following
\emph{extensions} of usage:

\subsection

\latin{Suus} is often used to refer to the subject of an act involved
in the thought, though not distinctly stated.
\begin{examples}

\latin{Caesar Fabium cum suā legiōne remittit in hīberna},
\english{Caesar sends Fabius back with his \emph{\(Fabius’s\)} legion
  to his winter quarters} (Fabius went back with his legion);
\apud{B.~G.}{5, 53, 3}.

\end{examples}

\subsection

\latin{Suus} is occasionally used to refer to the person most prominent in
the thought of the sentence, even though that person is neither the
grammatical nor the real (“logical”) subject.
\begin{examples}

\latin{dēsinant īnsidiārī domī suae cōnsulī},
\english{let them cease to set an ambuscade for the consul in his own
  house};
\apud{Cat.}{1, 13, 32}.

\end{examples}

\subsubsection

So especially with \latin{quisque}, as in \latin{suus cuique erat
  locus attribūtus}, \english{to each had been assigned his proper
  place} (his own place to each); \apud{B.~G.}{7, 81, 4}.

\subsection

Out of the meaning \emph{own} arise the meanings \emph{proper},
\emph{favorable}, etc.
\begin{examples}

\latin{dē ōrdine praecipiēmus suō tempore},
\english{on the matter of rank we will give instructions at the proper
  time} (\emph{its} time);
\apud{Quintil.}{2, 4, 21}.

\latin{sī hostīs in suum locum ēlicere posset},
\english{if he could draw the enemy into a favorable place} (\emph{his
  own} place);
\apud{B.~G.}{5, 50, 3}.

\end{examples}

\subsubsection

\versionA{Similarly \latin{aliēnus}, \english{belonging to another},
  gains the meaning \english{unfavorable}.}
\versionB*{Similarly \latin{noster}, \english{our}, may have the
  meaning \english{favorable}, and \latin{aliēnus}, \english{belonging
  to another}, the meaning \english{unfavorable}.}
\versionA*{Thus \latin{aliēnō locō},
  \english{in an unfavorable place}; \apud{B.~G.}{1, 15, \emend{142}{2}{1}}.}

\subsection

\latin{Sē}, \latin{suus}, and \latin{ipse} are often used of an
\emph{indefinite} self.
\begin{examples}

\latin{dēfōrme est dē sē ipsum praedicāre},
\english{it is bad form to brag about one’s self}:
\apud{Off.}{1, 38, 137}.

\end{examples}

\chapter{III. Pronouns Used with Reciprocal Force}

\section

The Pronouns used with Reciprocal Force denote two or more persons as
affecting \emph{each other} or \emph{one another}.

The reciprocal idea may be expressed, for \emph{two} persons or
things, by using \latin{alter} or \latin{uterque} twice, in different
cases; for \english{more than two} persons or things, by using
\latin{alius} twice, in different cases.
\begin{examples}

\latin{ut alter alterī auxīliō esset},
\english{so that each was of assistance to the other};
\apud{B.~G.}{5, 44, 14}.

\latin{uterque utrīque est cordī},
\english{they are dear to each other};
\apud{Ph.}{800}.

\latin{accēdēbat ut aliōs aliī deinceps exciperent};
\english{then besides, they relieved one another successively};
\apud{B.~G.}{5, 16, 4}.

\end{examples}

\subsubsection

The reciprocal idea is sometimes expressed by \latin{ipsī}, with
\latin{suī}, \latin{sibi}, or \latin{sē}.
\begin{examples}

\latin{ubi mīlitēs sibi ipsōs esse impedīmentō vīdit},
\english{when he saw that the soldiers were hindering one another};
\apud{B.~G.}{2, \emend{103}{25}{24}, 1}.

\end{examples}

\section

The phrase \latin{inter nōs} (or \latin{vōs}, or \latin{sē}), is used
with reciprocal force, in \emph{any} case-relation.
\begin{examples}

\latin{cohortātī inter sē},
\english{encouraging one another};
\apud{B.~G.}{4, 25, 5}.

\latin{quō differant inter sēsē},
\english{wherein they differ from one another};
\apud{B.~G.}{6, 11, 1}.

\end{examples}

\chapter{IV. The Intensive Pronoun}

\section

The Intensive Pronoun \latin{ipse}, \english{self}, expresses emphasis
or distinction.
\begin{examples}

\latin{Catilīna ipse profūgit; hī quid exspectant?}
\english{Catiline himself has fled; what, then, are these men waiting for?}
\apud{Cat.}{2, 3, 6}.

\end{examples}

\subsubsection

\latin{Ipse} is sometimes used alone, to denote a person prominent in
the minds of the speaker and the hearer.
\begin{examples}

\latin{respondēre solitōs: “ipse dīxit”; “ipse” autem erat
  Pȳthagorās},
\english{they used to answer “he said so himself”; now “himself”
  was Pythagoras};
\apud{N.~D.}{1, 5, 10}.

\latin{sēdēs in Galliā, ab ipsīs concessās},
\english{a home in Gaul, granted by \emph{(the Gauls)} themselves};
\apud{B.~G.}{1, 44, 2}.

\end{examples}

\section

When \latin{ipse} is used at the same time with the Reflexive Pronoun
(see \xref{261}) it agrees with the Subject or with the Reflexive,
according as the idea of the one or the other is to be emphasized.
\begin{examples}

\latin{mē ipse condemnō},
\english{I condemn myself};
\apud{Cat.}{1, 2, 4}.  (Self as \emph{actor}.)

\latin{nec agrum sed ipsum mē excolō},
\english{and I cultivate, not my field, but myself};
\apud{Plin.\ Ep.}{4, 6, 2}.  (Self as \emph{object}.)

\end{examples}

\section

\latin{Ipse} is much more freely used than English “self,” being
employed to express ideas conveyed by our “very,” “mere,”
“precisely,” “exactly,” “positively,” “in person,” “of his
own motion,” etc.
\begin{examples}

\latin{in ipsīs flūminis rīpīs},
\english{on the very banks of the river};
\apud{B.~G.}{2, \emend{143}{23}{22}, 3}.

\latin{Catilīnam ipsum ēgredientem verbīs prōsecūtī sumus},
\english{we have presented Catiline with our compliments as he went
  out of his own accord};
\apud{Cat.}{2, 1, 1}.

\end{examples}

\begin{minor}

\subsubsection

For \latin{ipse} as a Reflexive, see~\xref{263}; for \latin{ipsīus}
with a Possessive Pronoun, \xref[\emph{b}]{339}.

\end{minor}

\chapter{V. The Identifying Pronoun}

\section

\latin{Īdem}, \english{same}, identifies a person or thing with one
that has just been mentioned or is immediately to be mentioned.
\begin{examples}

\latin{eādem dē causā},
\english{for the same reason} (mentioned before);
\apud{B.~G.}{2, 7, 2}.

\end{examples}

\begin{minor}

\subsubsection

\latin{Īdem} often corresponds to English “also,” “likewise,” or
“yet.”
\begin{examples}

\latin{dīxī ego īdem in senātū},
\english{I also said in the senate} (I, the same man);
\apud{Cat.}{1, 3, 7}.

\end{examples}

\end{minor}

\subsubsection

“The same as” is expressed by \latin{īdem quī} or \latin{īdem atque}
or \latin{ac} (\xref[2, \emph{a}]{307}).

\chapter{VI. The Determinative-Descriptive Pronouns and\\ Corresponding
  Adjectives}

\subtitle{\latin{hic}, \latin{iste}, \latin{ille}, \latin{is},
  \latin{tālis}, \latin{tantus}, \latin{tot}}

\section

These Pronouns and Pronominal Adjectives have the power of telling
either (I)~\emph{what} person or thing is meant (\emph{determinative} power),
or (II)~\emph{what kind of} person or thing is meant
(\emph{descriptive} power).
\begin{examples}

I.\enskip \emph{Determinative Use}:
\latin{Q.\ Maximum, eum quī Tarentum recēpit},
\english{Quintus Maximus, \emph{\(I mean\)} the man who recovered
  Tarentum};
\apud{Sen.}{4, 10}.

\latin{id quod cōnstituerant facere cōnantur},
\english{they endeavor to do \emph{\(what?\)} that which they had
  determined upon};
\apud{B.~G.}{1, 5, 1}.

II.\enskip \emph{Descriptive Use}: \latin{habētis eum cōnsulem quī nōn
  dubitet},
\english{you have \emph{\(what kind of consul?\)} a consul that does
  not hesitate};
\apud{Cat.}{4, 11, 24}.

\end{examples}

\subsubsection

The distinctive meanings of these words are as follows:
\begin{vocab}

\latin{Hic}, \english{this}, or \english{of this kind}, refers to
something near the \emph{speaker}, in space, time, or thought.

\latin{Iste}, \english{that} (of yours), or \english{of that kind},
refers to something near the \emph{person addressed}, in space, time,
or thought.

\latin{Ille}, \english{that}, or \english{of that kind}, refers to
something more remote from both the \emph{speaker} and the
\emph{person addressed}, in space, time, or thought.

\latin{Is}, \english{this}, \english{that}, or \english{of this kind},
\english{of that kind}, is less specific than any of these, and may be
used in place of any of them.

\latin{T\/ālis}, \english{such}, expresses a quality just indicated or
to be indicated immediately.

\latin{Tantus}, \english{so great}, expresses a size just indicated or
to be indicated immediately.

\latin{Tot}, \english{so many}, expresses a number just indicated or
to be indicated immediately.

\end{vocab}

\begin{note}

\latin{Hic} is often called the Pronoun of the First Person
(\emph{this} \textsc{by me}), \latin{iste} of the Second (\emph{that}
\textsc{by you}), and \latin{ille} of the Third (\emph{that}
\textsc{by him}).

\end{note}

\section

The Determinative Pronouns are often used substantively, thus
supplying the place of the Third Personal Pronoun.  See~\xref{255},
\xref{256}.

\subsubsection
In the Neuter, the substantive use is very common.

\begin{minor}

\subsubsection

\latin{Ea rēs}, etc., is often preferred to \latin{id}, \latin{hoc},
etc., especially where there might be a doubt about the gender
(cf.~\xref[2, \emph{b}]{250}).

\end{minor}

\section

From their meanings, the Determinative Pronouns and Pronominal
Adjectives are adapted to point to something at hand, either in bodily
presence or in the speaker’s thought.
\begin{examples}

\latin{hic tamen vīvit},
\english{yet this man is allowed to live} (Catiline, who sits before
the speaker, and at whom he points);
\apud{Cat.}{1, 1, 2}.

\latin{hīs paucīs diēbus},
\english{within these few days} (i.e.\ the last few);
\apud{B.~G.}{3, \emend{144}{17, 3}{15, 7}}.

\end{examples}

\subsubsection

A neuter pronoun is often used to point backward or forward to a
substantive clause.  So especially \latin{id}, \latin{eō},
\latin{hoc}, \latin{hōc}, and \latin{illud}.
\begin{examples}

\latin{eō quod memoriā tenēret},
\english{for the reason that he remembered} (for this reason, namely
that);
\apud{B.~G.}{1, 14, 1}.

\latin{cum id nūntiātum esset, eōs cōnārī},
\english{when it was announced that they were endeavoring};
\apud{B.~G.}{1, 7, 1}.  (\latin{Id} is a mere “expletive,” like
English “it.”)

\end{examples}

\subsubsection

In Latin, a Noun-idea \emph{repeated}, with a change only in a
\emph{dependent} word, is generally left unexpressed.  In English, we
use a Pronoun.
\begin{examples}

\latin{carīnae aliquantō plāniōrēs quam nostrātum nāvium},
\english{the hulls were somewhat flatter than \emph{\(those\)} of our
  ships};
\apud{B.~G.}{3, 13, 1}.

\end{examples}

\section

Certain Determinative Pronouns gain special uses:

\subsection

\latin{Ille} is often used of a person or thing familiar to
everybody,—\emph{that} (well-known), \emph{that} (famous)
\emph{person} or \emph{thing}.
\begin{examples}

\latin{M.\ Catōnem, illum senem},
\english{Marcus Cato, that \emph{\(famous\)} old man};
\apud{Arch.}{7, 16}.

\end{examples}

\subsection

\latin{Hic} and \latin{ille} are often used to \emph{distinguish between}
persons or things just mentioned, \latin{hic} meaning the one last
mentioned (“the latter”), and \latin{ille} the one mentioned farther
back (“the former”).
\begin{examples}

\latin{sī haec nōn dīcō maiōra fuērunt in Clōdiō quam in Milōne, sed
  in illō maxima, nūlla in hōc},
\english{if these \emph{\(bad qualities\)} were, I will not say
  greater in Clodius than in Milo, but immensely great in the former,
  and non-existent in the latter};
\apud{Mil.}{13, 35}.

\end{examples}

\subsubsection

But sometimes \latin{hic} refers to the person or thing more prominent
in the speaker’s thought, and \latin{ille} to the one less prominent,
although the order in which they have been mentioned is the opposite.

\subsubsection

\latin{Hic} and \latin{ille} are often weakened into mere \emph{Indefinite
Pronouns}.
\begin{examples}

\latin{modo hoc modo illud},
\english{now one thing, now another};
\apud{N.~D.}{1, 18, 47}.
Similarly \latin{iam hōs iam illōs},
\apud{Aen.}{4, 157}.

\end{examples}

\subsection

\latin{Is} or \latin{is quidem}, \versionB*{and \latin{ille} or
  \latin{ille quidem},} in combination with various connectives%
%
\versionA{ (\latin{et is}, \latin{atque is}, \latin{isque}, \latin{et
    is quidem}, \latin{nec is}, \latin{neque is}, etc.), is }%
%
\versionB*{ (\latin{et}, \latin{atque}, \latin{nec}, etc.), are }%
%
used when a second and still more striking quality or action is to
be added to one already attributed to a person or thing (English “and
that,” “and that too”).
\begin{examples}

\latin{vincula, et ea sempiterna},
\english{imprisonment, and that too for life};
\apud{Cat.}{4, 4, 7}.

\end{examples}

\subsubsection

\latin{Id}, in combination with various connectives (\latin{et},
\enclitic{-que}, etc.), is used when a similar addition is to be made
to an idea expressed by a verb.
\begin{examples}

\latin{doctum hominem cognōvī, idque ā puerō},
\english{I know him to be a person of learning, and that too from
  boyhood};
\apud{Fam.}{13, 16, 4}.

\end{examples}

\subsection

\latin{Iste} is often used to express contempt.
\begin{examples}

\latin{dē istīs, quī sē populārīs habērī volunt},
\english{of these fellows who want themselves to be thought friends of
  the people};
\apud{Cat.}{4, 5, 10}.

\end{examples}

\chapter{VII. The Interrogative Pronouns and Corresponding Adjectives}

\section

The Interrogative Pronouns and corresponding Adjectives are those
which ask a question, namely:

\subsection

\latin{Uter}, \english{which?} used in speaking of two persons or
things, \latin{quis}, \english{who?} \english{which?}\ in speaking of
any larger number.
\begin{examples}

\latin{in utrō haec fuit, in Milōne, an in Clōdiō?}
\english{in which of the two did this exist, in Milo or in Clodius?}
\apud{Mil.}{16, 43}.

\latin{quis est mē mītior?}
\english{who is gentler than I?}
\apud{Cat.}{4, 6, 11}.

\end{examples}

\subsection

\latin{Cuius} (\latin{-a}, \latin{-um}), \english{whose?}\ (rare).
\begin{examples}

\latin{cuium pecus?}
\english{whose flock is this?}
\apud{Ecl.}{3, 1}.

\end{examples}

\subsection

\latin{Quot}, \english{how many?}\ \latin{quotus}, \english{which in
  order?} (e.g.\ \english{second}, \english{third}, etc.)
\begin{examples}

\latin{“quot sunt?” “Totidem quot ego et tū sumus,”}
\english{“how many are there of them?”  “As many as of you and
  me”};
\apud{Rud.}{564}.

\latin{hōra quota est?}
\english{what o’clock is it?} (what in the order of hours?);
\apud{Sat.}{2, 6, 44}.

\end{examples}

\subsection

\latin{Quī}, \english{what?}\ \english{of what kind?}
(=~\latin{quālis}; see under~5).
\begin{examples}

\latin{at quod erat tempus?}
\english{but what kind of situation was it?}
\apud{Mil.}{15, 39}.

\end{examples}

\begin{minor}

\subsubsection

The poets sometimes use \latin{quī} for \latin{quis} in independent
questions.  In dependent questions, the distinction stated is not
always observed, even in Ciceronian Latin.  Cf.~\xref[\emph{a}]{141}.

\end{minor}

\subsection

\latin{Quālis}, \english{of what kind?}\ \latin{quantus}, \english{how
  great?}
\begin{examples}

\latin{ubi tua (mēns) aut quālis?}
\english{where is your mind, or of what nature?}
\apud{Tusc.}{1, 27, 67}.

\latin{“quantī (ēmptae)?” “Octussibus,”}
\english{“\(bought\) at what price?”  “Eight cents”};
\apud{Sat.}{2, 3, 156}.

\end{examples}

\subsection

\latin{Ecquis}, \english{any?}\ (without implication), and \latin{num
  quis}, \english{any?}\ (implying “none”), are \emph{indefinite}
interrogatives.
\begin{examples}

\latin{ecquid adferēbat festīnātiōnis?}
\english{did it occasion any delay?}
\apud{Mil.}{19, 49}.

\end{examples}

\chapter{VIII. The Indefinite Pronouns and Corresponding Adjectives}

\section

The Indefinite Pronouns and corresponding Adjectives present the idea
of \emph{some} person, thing, quality, or quantity, without further
explanation.
\begin{center}

\latin{quis}, \latin{quī};
\latin{aliquis}, \latin{aliquī};
\latin{quispiam};

\latin{nesciō}\footnote{\latin{Nesciŏ quis} with iambic shortening as
  in \latin{volō}, etc.\ (\xref[note]{28}), in all poetical
  occurrences, in the hexameter necessarily so.} \latin{quis};
\latin{quīdam}, \latin{nōn nūllus};
\latin{quisquam}, \latin{ūllus};

\latin{utervīs}, \latin{uterlibet};
\latin{quīvīs}, \latin{quīlibet};
\latin{neuter}, \latin{nūllus};

\latin{quantusvīs}, \latin{quantuslibet}

\end{center}

\subsection

\latin{Quis} (or the corresponding Adjective \latin{quī}), the vaguest
of the indefinites, means \english{any one}, \english{some one}, and
is used chiefly with \latin{sī}, \latin{nisi}, \latin{nē}, and with
Interrogative\footnote{E.g.\ \latin{num?}\ \latin{ubi?}} or
Relative\footnote{E.g.\ \latin{cum}, \latin{ubi}, \latin{quō},
  \latin{quantō}.} words.  It always stands \emph{after} one or more
words of its clause.
\begin{examples}

\latin{roget quis},
\english{some one may ask};
\apud{Eun.}{\emend{10}{511}{510}}.

\latin{sī quid hīs accidat},
\english{if anything should happen to them};
\apud{B.~G.}{3, \emend{145}{22}{20}, 2}.

\end{examples}

\subsection

\latin{Aliquis} (or the corresponding Adjective \latin{aliquī}) means
\english{somebody}, \english{some one}, as opposed to
\english{nobody}.
\begin{examples}

\latin{sī vīs esse aliquid},
\english{if you want to be somebody} (something);
\apud{Iuv.}{1, 74}.

\end{examples}

\subsection

\latin{Quispiam}, \english{some one}, approaches \latin{aliquis} in
force.
\begin{examples}

\latin{cum quaepiam cohors ex orbe excesserat},
\english{when some cohort had gone out of the circle};
\apud{B.~G.}{5, 35, 1}.

\end{examples}

\subsection

\latin{Nesciō quis} (originally \emph{I don’t know who}) means
\english{somebody or other} (it doesn’t matter who).  It often is
contemptuous.
\begin{examples}

\latin{nesciō quō pactō},
\english{in some way or other};
\apud{Cat.}{1, 13, 31}.

\end{examples}

\subsection

\latin{Quīdam} means \english{a certain one} (who might be named or
more definitely made known or described, if necessary).
\begin{examples}

\latin{videō esse hīc quōsdam, quī tēcum ūnā fuērunt},
\english{I see that there are certain men here present who were in
  your company};
\apud{Cat.}{1, 4, 8}.

\end{examples}

\begin{minor}

\subsubsection

Like English “a certain,” \latin{quīdam} is sometimes employed to
\emph{soften} an adjective or noun.  In this use it is frequently
accompanied by \latin{quasi}, \english{as it were}, \english{so to
  speak}.
\begin{examples}

\latin{omnēs artēs quasi cognātiōne quādam inter sē continentur},
\english{all the arts are bound together by a certain relationship, as
  it were};
\apud{Arch.}{1, 2}.

\end{examples}

\end{minor}

\subsection

\latin{Nōn nūllus} (\english{not none}) means \english{some}, or, in
the Plural, \english{several}, \english{a number of}.  It differs from
\latin{quīdam} in \emph{not} suggesting that a more definite statement
might be made.
\begin{examples}

\latin{nōn nūllī inter carrōs matarās subiciēbant},
\english{some of them were throwing javelins from below among the
  carts};
\apud{B.~G.}{1, 26, 3}.

\end{examples}

\begin{minor}

\subsubsection
\latin{Nōn nēmō} may be used in the same way.  See
example,~\xref[2]{298}.

\end{minor}

\subsection

\latin{Quisquam}, \english{any at all}, and the corresponding
Adjective \latin{ūllus} are used only in negative sentences or
phrases, in questions implying a negative, in clauses following a
Comparative or Superlative, in Relative Clauses, and in Conditions.
\begin{examples}

\latin{neque quisquam est tam āversus ā Mūsīs},
\english{nor is any one so hostile to the Muses};
\apud{Arch.}{9, 20}.

\latin{cūr quisquam iūdicāret?}
\english{why should any one judge?} (= no one should);
\apud{B.~G.}{1, 40, 2}.

\latin{sine ūllō perīculō},
\english{without any danger};
\apud{B.~G.}{2, 11, 6}.

\latin{taetrior quam quisquam superiōrum},
\english{more hideous than any of his predecessors};
\apud{Verr.}{4, 55, 123}.

\latin{quam diū quisquam erit quī tē dēfendere audeat},
\english{as long as there shall be any one who will dare to defend
  you};
\apud{Cat.}{1, 2, 6}.

\latin{sī quicquam spērent},
\english{if they have any hope};
\apud{B.~G.}{5, 41, 5}.

\end{examples}

\subsection

\latin{Utervīs} and \latin{uterlibet} mean \english{either of two
  indifferently} (“whichever you wish”), and \latin{quīvīs} and
\latin{quīlibet}, \english{any one whatever} (“any you wish”) \emph{of
three or more}.  \latin{Quantusvīs} and \latin{quantuslibet} mean
\english{of any size whatever}.
\begin{examples}

\latin{minus habeō vīrium quam vestrum utervīs},
\english{I have less strength than either of you};
\apud{Sen.}{10, 33}.

\latin{ad quemvīs numerum},
\english{up to any number whatever};
\apud{B.~G.}{4, 2, 5}.

\latin{quantāsvīs cōpiās},
\english{forces of any size whatever};
\apud{B.~G.}{5, 28, \emend{130}{4}{3}}.

\end{examples}

\subsection

\latin{Neuter} means \english{neither of the two}, and \latin{nūllus},
\english{no one out of a larger number}.  They are thus the negative
words corresponding respectively to \latin{utervīs} and
\latin{quīvīs}.
\begin{examples}

\latin{neutrī trānseundī initium faciunt},
\english{neither party begins the crossing};
\apud{B.~G.}{2, 9, 2}.

\latin{nūllō hoste prohibente},
\english{with no enemy to prevent};
\apud{B.~G.}{3, 6, 5}.

\end{examples}

\begin{minor}

\subsubsection

The Plural forms of \latin{neuter} have regularly the meaning of
\english{neither of the two parties}, as in the first example just
above.

\subsubsection

\latin{Nūllus} is sometimes used for \latin{nēmō} (i.e.\ as a
Substantive), but rarely in Cicero.

\subsubsection

\latin{Nēmō} is occasionally used for \latin{nūllus} (i.e.\ as an
Adjective), as in \latin{servus est nēmō}, \english{there is no
  slave}; \apud{Cat.}{4, 8, 16}; \latin{nēmō homō}, \english{no man};
\apud{Pers.}{211}.

\subsubsection

\latin{Nēmō} is regularly used instead of \latin{nūllus}, to agree
with a Proper Name or an Adjective, Participle, or Pronoun used
substantively.
\begin{examples}

\latin{nēmō Cornēlius},
\english{no Cornelius};
\apud{Att.}{6, 1, 18}.

\latin{nēmō alius},
\english{no other};
\apud{Brut.}{88, 302}.

\end{examples}

\end{minor}

\subsection

\latin{Quīcumque}, \english{whosoever}, and \latin{quāliscumque},
\english{of what kind soever} (properly Generalizing; \xref[II]{282}),
are sometimes used as Indefinite Pronouns or Adjectives even in
Cicero’s time, and very frequently later.
\begin{examples}

\latin{quae sānārī poterunt, quācumque ratiōne sānābō}, \english{what
  can be healed, I’ll heal in any way soever}; \apud{Cat.}{2, 5, 11}.

\end{examples}

\pagebreak

\chapter{IX. The Collective Pronoun}

\section

\latin{Ambō} means \english{both}, i.e.\ \english{two taken together}.
\begin{examples}

\latin{ambō incolumēs sēsē recipiunt},
\english{both return unharmed};
\apud{B.~G.}{5, 44, 13}.

\end{examples}

\subsubsection

For a larger number, Latin use the Adjective \latin{omnēs}, \english{all}.

\chapter{X. The Distributive Pronouns}

\section
\subsection

\latin{Uterque} (\latin{uter}, \english{either of two}, plus the
indefinite enclitic \enclitic{-que}, \english{soever}) means
\english{either soever of two}, \english{each of two}, taken
separately.  (Compare \latin{ambō}, \english{both of two}, taken
\emph{together}.)
\begin{examples}

\latin{uterque cum equitātū venīret},
\english{\emph{\(demanded\)} that each of the two should come with
    cavalry};
\apud{B.~G.}{1, 42, \emend{146}{4}{5}}.

\end{examples}

\begin{minor}

\subsubsection

The Plural forms of \latin{uterque} have the sense of \english{each of
  the two sides}, \english{each of the two parties}, etc.
\begin{examples}

\latin{pugnātum est ab utrīsque ācriter},
\english{each of the two sides fought valiantly};
\apud{B.~G.}{4, \hbox{26, 1}}.

\end{examples}

\subsubsection

But with a Noun Singular in meaning though Plural in form
(\xref{105}), the Plural of \latin{uterque} is Singular in meaning.
\begin{examples}

\latin{utrīsque castrīs},
\english{for each camp};
\apud{B.~G.}{1, 51, 1}.

\end{examples}

\subsubsection

For \latin{uterque} with reciprocal force, see~\xref{265}.

\end{minor}

\subsection

\latin{Quisque} (\latin{quis}, \english{any}, plus the indefinite
enclitic \enclitic{-que}, \english{soever}) means \english{any one
  soever}, \english{each}, \english{all}, etc., taken \emph{individually}.
(Compare \latin{omnēs}, \english{all}, taken \english{together}.)  It
is used with the following words, and immediately after them:
\begin{enuma}

\item

With \emph{Reflexive}, \emph{Relative}, or \emph{Interrogative} words.
\begin{examples}

\latin{prō sē quisque},
\english{each to the best of his power};
\apud{B.~G.}{2, \emend{147}{25}{24}, 3}.

\latin{quam quisque in partem dēvēnit},
\english{to whatever place each came};
\apud{B.~G.}{2, \emend{117}{21}{20}, 6}.

\latin{quid quōque locō faciendum esset},
\english{what needed to be done in each place};
\apud{B.~G.}{5, \hbox{33, 3}}.

\end{examples}

\item

With \emph{Superlatives}, to indicate a class.
\begin{examples}

\latin{optimus quisque},
\english{all the best men} (each best man);
\apud{Arch.}{11, 26}.

\end{examples}

\item

With \emph{Ordinal Numerals}.
\begin{examples}

\latin{decimum quemque},
\english{one man in ten} (every tenth man);
\apud{B.~G.}{5, 52, 2}.

\latin{quotus quisque fōrmōsus est!}
\english{how few are handsome!}\ (one of how many is each handsome
man?);
\apud{N.~D.}{1, 28, 79}.

\end{examples}

\end{enuma}

\pagebreak

\chapter{XI. Pronominal Adjectives}

\headingC{alter, alius}

\section
\subsection

When used singly, \latin{alter} means \english{the other} or
\english{one}, where \english{two} are
\linebreak
thought of; and \latin{alius}
means \english{other} or \english{another}, where \emph{more than two}
are thought of.
\begin{examples}

\latin{itinera duo, ūnum per Sēquanōs, alterum per prōvinciam
  nostram},
\english{two ways, one through the country of the Sequani, the other
  through the province};
\apud{B.~G.}{1, 6, 1}.

\latin{alterō oculō capitur},
\english{is blinded in one eye};
\apud{Liv.}{22, 2, 11}.

\latin{fīlius Domitī aliīque complūrēs adulēscentēs},
\english{the son of Domitius and several other young men};
\apud{B.~C.}{1, 23, 2}.

\end{examples}

\subsubsection

\latin{Cēterī} differs from \latin{aliī} in meaning
\english{\textsc{all} the others}, \english{the \textsc{rest}}.
\begin{examples}

\latin{hōsce ego hominēs excipiō; cēterī vērō quā virtūte
  cōnsentiunt!}
\english{these men I except; but how nobly all the rest agree!}
\apud{Cat.}{4, 7, 15}.

\end{examples}

\subsubsection

\latin{Reliquī}, \english{those remaining}, approaches \latin{cēterī}
in force, but does not so insist upon completeness.
\begin{examples}

\latin{oppida sua, vīcōs, reliqua prīvāta aedificia incendunt},
\english{they set fire to their towns, their villages, and the private
  buildings that remained};
\apud{B.~G.}{1, 5, 2}.

\end{examples}

\subsection

\latin{Alter} or \latin{alius} is often used twice, with correlative
meaning, \english{one\ellipsis the other}, \english{one\ellipsis another}.
\begin{examples}

\latin{hārum altera occīsa, altera capta est},
\english{of these, one was killed, the other taken prisoner};
\apud{B.~G.}{1, 53, 4}.

\end{examples}

\subsection

\latin{Alius} is often used twice in the same clause or phrase, with
the meaning \english{one\ellipsis one\dots}, \english{another\ellipsis
  another}.

\begin{examples}

\latin{alius aliā ex nāvī sē adgregābat},
\english{they were gathering, one from one ship, another from
  another};
\apud{B.~G.}{4, 26, 1}.

\end{examples}

\subsection

For \latin{alter} and \latin{alius} with reciprocal force,
see~\xref{265}.

\begin{minor}

\subsubsection

The Adverbs \latin{aliter}, \latin{aliās}, and \latin{alibi} are used
with forces corresponding in all respects to those of \latin{alius},
as given in 3 and~4.

\end{minor}

\chapter{XII. Relative Pronouns and Corresponding Adjectives}

\begin{minor}

\section[\textsc{Introductory}]

The Latin Relative Pronoun is probably derived from two sources (which
were doubtless originally one), the Interrogative Pronoun and the
Indefinite Pronoun, as follows:

In sentences like \latin{quis volet, vindex estō} (Twelve Tables, II),
the \latin{quis} could be either Interrogative or Indefinite.  “Who
shall wish?  He shall be protector” would lead to the \emph{relative}
feeling, \english{who shall wish, he shall be protector},
i.e.\ \english{he who shall wish shall be
  protector.}\footnote{Similarly, the English Relative “who” has
  arisen from the Interrogative “who.”} But so, also, could “any
man shall wish: he shall be protector,” i.e.\ \english{whoever shall
  wish, he shall be protector},

\end{minor}

\section

The Relative Pronouns and Adjectives are \emph{connecting} Pronouns
and Adjectives referring to something that precedes or follows.

\subsubsection

The word to which a Relative refers is called its
\term{Antecedent}.\footnote{Because the word referred generally
  \emph{comes before} the Relative.}
\begin{examples}

\latin{rēgnum quod pater habuerat}, \english{the royal power which his
  father had had}; \apud{B.~G.}{1, 3, \emend{148}{4}{3}}.
(\latin{Rēg\-num} is the Antecedent.)

\end{examples}

\section

The meanings of the Relatives are as follows:

\smallskip

\subtitle{I. \textbf{Individual or Generalizing}}
\begin{mexamples}[2]

\latin{quī}, \english{who}, or \english{whoever}

\latin{quālis}, \english{of which kind}, or \english{of what kind
  soever}

\latin{quantus}, \english{of what size}, or \english{of what size
  soever}

\latin{quot}, \english{of what number}, or \english{of what number
  soever}

\end{mexamples}

\subtitle{II. \textbf{Generalizing Only}}

\begin{mexamples}[2]

\latin{quīcumque}, \english{whoever}

\latin{quisquis}, \english{whoever}

\latin{quāliscumque}, \english{of what kind soever}

\latin{quantuscumque}, \english{of what size soever}

\latin{quotcumque}, \english{of what number soever}

\latin{quotquot}, \english{of what number soever}

\end{mexamples}

\begin{minor}

\subsubsection

Note that the uncompounded forms are either Individual or Generalizing
in meaning, while the compounded forms are always Generalizing.

\end{minor}

\headingB{Generalizing Forms with Merely Indefinite Meaning}

\section

The same Pronouns, Pronominal Adjectives, or Adverbs which may be used
in a Generalizing sense can also be employed of \emph{individual} persons or
things \emph{not definitely known} to the speaker.
\begin{examples}

\latin{tibi hercle deōs īrātōs esse oportet, quisquis es},
\english{the gods must surely be angry at you, whoever you are};
\apud{Rud.}{1146}.  (The “you” is of course a particular person, but
the speaker does\emend{11}{ }{}n’t know \emph{who}.)  Similarly
\latin{quaecumque}, \apud{Aen.}{1, 330}.

\end{examples}

\headingG{Peculiarities in the Use of the Latin Relative}

\section
\subsection

The Antecedent is often omitted, especially if \emph{indefinite}.
\begin{examples}

\versionA*{\latin{sunt hūmānissimī quī Cantium incolunt},
\english{the most civilized are \emph{\(those\)} who live in Kent};
\apud{B.~G.}{5, 14, 1}.  (Definite Antecedent.)}

\latin{ut quae bellō cēperint quibus vēndant habeant},
\english{that they may have \emph{\(people\)} to whom to sell what
  they take in war};
\apud{B.~G.}{4, 2, 1}.
\versionA*{(Indefinite Antecedent.)}

\end{examples}

\versionB*{%
\subsubsection

The antecedent is often incorporated into the relative clause,
appearing only here.
\begin{examples}

\latin{habētis quam petīstis facultātem},
\english{you have the opportunity which you have been waiting for};
\apud{B.~G.}{6, 8, 3}.

\end{examples}}

\subsection

\emph{The Relative is never omitted in Latin.}
\begin{examples}

\versionA*{\latin{habētis quam petīstis facultātem},
(in English idiom) \english{you have the opportunity you have been
  waiting for};
\apud{B.~G.}{6, 8, 3}.}

\end{examples}

\subsection

The Relative Clause is frequent in Latin, where English would use a
shorter expression (Noun, Participle, Appositive, etc.).
\begin{examples}

\latin{pontem quī erat ad Genāvam},
\versionA{(in English idiom) \english{the bridge at Geneva};}
\versionB*{\english{the bridge \emph{(which was)} at Geneva} (in
  English idiom, \english{the bridge at Geneva});}
\apud{B.~G.}{1, 7, 2}.

\latin{quī decimae legiōnis aquilam ferēbat},
\english{the man who bore the standard of the tenth legion} (=
\latin{aquilifer});
\apud{B.~G.}{4, 25, 3}.

\end{examples}

\begin{minor}

\subsubsection

Yet occasionally the same condensation is found in Latin as in
English.
\begin{examples}

\latin{sēdēs habēre in Galliā ab ipsīs concessās},
(said) \english{that he had a home in Gaul \emph{\(which had been\)}
  granted him by the Gauls themselves};
\apud{B.~G.}{1, 44, 2}.

\end{examples}

\end{minor}

\subsection

The Antecedent Noun is sometimes repeated, for greater distinctness,
in the Relative Clause.
\begin{examples}

\latin{ultrā eum locum, quō in locō Germānī cōnsēderant},
\english{beyond the place in which \emph{\(place\)} the Germans had
  encamped};
\apud{B.~G.}{1, 49, 1}.

\end{examples}

\subsection

The Relative Clause often precedes its Antecedent.  So especially the
Rhetorical Determinative Clause (\xref[\emph{a}, n.~3]{550}).
\begin{examples}

\latin{quōs ferrō trucīdārī oportēbat, eōs nōndum vōce vulnerō},
\english{I do not yet wound with a word the men who ought to be slain
  with the sword} (what men\dots, those\dots);
\apud{Cat.}{1, 4, 9}.

\end{examples}

\begin{minor}

\subsubsection

English idiom does not tolerate this order in prose.

\end{minor}

\subsection

When the Relative Clause precedes \versionA*{the clause containing the
  Antecedent}, the principal Noun is generally attached to the
Relative and takes its case.
\begin{examples}

\latin{implōrāre dēbētis ut quam urbem pulcherrimam esse voluērunt,
  hanc dēfendant},
\english{it is your duty to implore \emph{\(the gods\)} that, since
  they have chosen to make this city the fairest in the world, they
  will defend it};
\apud{Cat.}{2, 13, 29}.  (For the translation, see~\emph{a}, just
above.)

\end{examples}

\subsection

The Relative Clause frequently attracts into itself an Adjective
belonging to the Antecedent, especially if that Adjective is a
Superlative.
\begin{examples}

\latin{cōnsiliīs pārē, quae nunc pulcherrima Nautēs dat},
\english{follow the admirable plans which Nautes now proposes}
(follow the plans which,—admirable they are,—Nautes proposes);
\apud{Aen.}{5, 728}.

\end{examples}

\subsection

Latin often uses a Relative Pronoun where English would use a
Determinative or Personal Pronoun introduced by \english{and},
\english{but}, etc.
\begin{examples}

\latin{quae cum ita sint},
\english{and since this is so};
\apud{Cat.}{1, 5, 10}.

\end{examples}

\subsection

More frequently than in English, the relative belongs in government to
a clause \emph{Subordinate} to that which it really introduces.
\begin{examples}

\latin{nōn polītus iīs artibus quās quī tenent ērudītī appellantur},
\english{not finished in those accomplishments the possessors of which
  are called learned};
\apud{Fin.}{1, 7, 26} (those who possess which; similarly \latin{cui
  quī pāreat}, \apud{Sen.}{1, 2}).

\end{examples}

\subsection

More frequently than in English, a Relative Adverb of place is used,
instead of a Relative Pronoun, to refer to a Personal Antecedent.
\begin{examples}

\latin{is unde tē audīsse dīcis},
\english{the man from whom you say you heard it}
(the man whence);
\apud{De~Or.}{2, 70, 285}.

\end{examples}

\headingB{Verbs}

\headingG{Expression (or Omission) of the Subject}

\section

Since the termination of the Finite Verb shows its Person and Number
(e.g.\ \latin{amō}, \english{I love}; \latin{amās}, \english{you
  love}; \latin{amant}, \english{they love}), the Subject does not
need to be expressed, except for emphasis or contrast, or to prevent
ambiguity (cf.~\xref{257}).
\begin{examples}

\emph{Subject omitted}:
\latin{abiit},
\english{he has gone away};
\apud{Cat.}{2, 1, 1}.

\emph{Subject expressed for emphasis or contrast}:
\latin{tam ille apud nōs servit qam eqo nunc apud tē serviō},
\english{\textsc{he} is a slave in our country just as \textsc{I} am
  now a slave in yours};
\apud{Capt.}{312}.

\emph{Subject expressed to avoid ambiguity}:
\latin{Q.\ Laberius Dūrus, tribūnus mīlitum, interficitur.  Illī
  plūribus submissīs cohortibus repelluntur},
\english{Quintus Laberius Durus, a military tribune, is killed.  They
  \emph{(i.e.\ the enemy)} are driven off by the sending of a number of
  cohorts to the rescue};
\apud{B.~G.}{5, 15, 5}.

\end{examples}

\headingG{Indefinite Subject}

\section

The First and Third Persons Plural, and the Second Person Singular
Indefinite are used, as in English, to express an \emph{Indefinite
  Subject}; (“we,” “they,” or “you” in the sense of “any
one”).
\begin{examples}

\latin{fortūnātōrum memorant īnsulās},
\english{they tell of the islands of the blessed} (men tell);
\apud{Trin.}{549}.

\latin{datur ignis, tametsī ab inimīcō petās},
\english{fire is given you, even if you ask it of an enemy}
(“you” is \emph{anybody});
\apud{Trin.}{679}.

\end{examples}

\headingG{Impersonal Verbs}

\section

Some Verbs are used in the Third Singular without a Subject, either
expressed or understood, and are accordingly called \emph{Impersonal}.

These Verbs express \emph{operations of nature}, or \emph{mental
  distress}, or \emph{acts considered merely as such}, without
reference to the performer.
\begin{examples}

\latin{iam advesperāscit},
\english{it is getting dark now};
\apud{And.}{\emend{149}{581}{582}}.

\latin{eius mē miseret},
\english{I pity him}
(it makes me pitiful of him);
\apud{Ph.}{188}.

\latin{pugnātum est ācriter},
\english{there was a fierce fight};
\apud{B.~G.}{3, \emend{150}{21}{19}, 1}.

\end{examples}

\begin{minor}

\subsubsection

The name Impersonal is also conveniently applied to verbs that have
an Infinitive or a Clause for Subject, as in \latin{īnsānīre iuvat},
\english{’t is a pleasure to play the madman}; \apud{Carm.}{3, 19, 18}.

\end{minor}

\pagebreak

\chapter{Voice}

\section
\subsection

The Active Voice represents the Subject of the Verb as \emph{acting}
or \emph{being}.
\begin{examples}

\latin{Helvētiī lēgātōs mittunt},
\english{the Helvetians send ambassadors};
\apud{B.~G.}{1, 7, 3}.

\latin{erant omnīnō itinera duo},
\english{there were in all but two ways};
\apud{B.~G.}{1, 6, 1}.

\end{examples}

\subsection

The Passive Voice represents the Subject as \emph{acted upon}.
\begin{examples}

\latin{mittitur C.\ Arpīneius},
\english{Gaius Arpineius is sent};
\apud{B.~G.}{5, 27, 1}.

\end{examples}

\subsection

\textbf{Reflexive Use of the Passive.}\footnote{\label{ftn:158:}Often
  called “Middle Voice,” as in Greek.}  The Passive Voice is
sometimes used, especially in poetry, in a \emph{reflexive} sense, to
express an act as done by the actor to or for \emph{himself}.
\begin{examples}

\latin{ad spectāculum omnēs effunduntur},
\english{all pour out to see the show};
\apud{Liv.}{39, 49, 8}.
(Cf.\ \latin{sēsē multitūdō effūdit},
\english{the crowd poured itself out};
\apud{B.~C.}{2, 7, 3}.)

\latin{umerōs īnsternor pelle},
\emph{I cover my shoulders with a skin};
\apud{Aen.}{2, 721}.

\end{examples}

\begin{minor}

\subsubsection

An Active verb that can be used reflexively in a Passive Finite form
can also be used reflexively in the Present Active Participle.
Compare \latin{exercentur}, \english{exercise} (\emph{themselves}),
\apud{Tusc.}{2, 23, 56}, with \latin{exercentibus},
\english{exercising}, \apud{De~Or.}{2, 71, 287}.

\subsubsection

The Deponent Verbs (\xref{160}) were originally Reflexive.  Thus
\latin{vēscor}, \english{eat} (originally, \english{feed myself}).

\end{minor}

\headingB{Transitive and Intransitive Verbs}

\section

A \latin{Transitive Verb} is one that expresses an action immediately
directed upon some person or thing (“transitive” = \emph{passing
  over upon}).  That upon which the action is immediately directed is
called the \emph{Direct Object} (\xref{390}).
\begin{examples}

\latin{Caesar eius dextram prēndit},
\english{Caesar took his hand};
\apud{B.~G.}{1, 20, 5}.

\end{examples}

\begin{minor}

\subsubsection

\textbf{Absolute Use.}  A Transitive Verb may be used \emph{without}
an Object, to represent the mere action, without reference to that
upon which it is directed.  Thus \latin{arāre māvelim}, \emph{I should
  prefer to plough}; \apud{Merc.}{356}.

\subsubsection

Similarly, verbs governing \emph{other} cases than the Accusative may be used
\term{Absolutely}.  Thus \latin{sus\-cēn\-sen\-dī tempus erit},
\english{there will be a time for being angry}; \apud{Liv.}{22, 29,
  2}; \latin{vēscendī causā}, \english{for the purpose of eating};
\apud{Sall.\ Cat.}{13, 3}.

\end{minor}

\section

 An \term{Intransitive Verb} is one that expresses an act or state
\emph{not} immediately directed upon any person or thing.
\begin{examples}

\latin{vīvō et rēgnō},
\english{I live and reign};
\apud{Ep.}{1, 10, 8}.

\end{examples}

\subsubsection

Intransitive Verbs, generally speaking, have no Passive.  But
\begin{enumerate}

\item

An Intransitive Verb may be used \emph{impersonally} in the
Passive.
\begin{examples}

\latin{diū pugnātum est},
\english{there was a long fight}
(it was fought long);
\apud{B.~G.}{1, 26, 1}.

\end{examples}

\item

A few Intransitive Verbs may be used with a Subject of Kindred
Meaning.
\begin{examples}

\latin{illa (pugna) quae cum rēge est pugnāta},
\english{the battle which was fought with the king};
\apud{Mur.}{16, 34}.

\end{examples}

\item

Verbs generally Intransitive are occasionally used in the Future
Passive Participle with true Passive meaning.
\begin{examples}

\latin{laetandum magis quam dolendum putō cāsum tuum},
\english{I think your fate is rather to be rejoiced at than grieved
  over};
\apud{Sall.\ Iug.}{14, 22}.

\end{examples}

\item

A few Perfect Passive Participles from Intransitve Verbs may be used
with Active meaning; thus \latin{iūrātus}, \english{having sworn},
\latin{cēnātus}, \english{having dined}, \latin{prānsus},
\english{having breakfasted}, \latin{pōtus}, \english{having drunk}.
\begin{examples}

\latin{Lūcullus iūrātus dīxit},
\english{Lucullus, having taken the oath, said};
\apud{Mil.}{27, 73}.

\end{examples}

\item

\latin{Coepī} and \latin{dēsinō} with Infinitives of \emph{true
  Passive meaning} are generally themselves made Passive in form.
\begin{examples}

\latin{Milōnis cōnsulātus temptārī coeptus est},
\english{Milo’s candidature for the consulship began to be assailed};
\apud{Mil.}{13, 34}.
(But \latin{vidērī coepit}, \english{began to \textsc{seem}}, in
\apud{Verr.}{1, 50, 132}, since \latin{vidērī} has not true passive
meaning here.)

\end{examples}

\begin{note}[Note 1]

A verb may of course be Active, yet not be Transitive.  Thus
\latin{rēgnat}, \english{reigns}, is Active, because it expresses
activity; but it is not Transitive, because the activity is not
represented as immediately directed upon a person or thing.  We cannot
say, for example, “the king reigns his subjects.”

\end{note}

\begin{note}[Note 2]

Yet the poets sometimes \emph{force the meanings} of Intransitive
verbs, and use them in the Passive.
\begin{examples}

\latin{terra rēgnāta Lycurgō},
\english{a land reigned over by Lycurgus};
\apud{Aen.}{3, 13}.

\end{examples}

\end{note}

\end{enumerate}

\headingG{Voice-Meanings of Deponent and Semi-Deponent Verbs}

\section

Deponent and Semi-Deponent Verbs (\xref{160}, \xref{161}) are active
in meaning, \emph{except in the Future Passive Participle}.

\subsubsection

Accordingly, Transitive Deponents and Semi-Deponents have three
Participles of active meaning, and one of passive. Thus:
\begin{mexamples}[2]

\latin{admīrāns}, \english{admiring}

\latin{admīrātūrus}, \english{about to admire}

\latin{admīrātus}, \english{having admired}

\latin{admīrandus}, \english{to be admired}

\end{mexamples}

\subsubsection

\emph{Intransitive} Deponents and Semi-Deponents of course lack a true
Future Passive Participle. Thus \latin{proficīscēns},
\latin{profectus}, \latin{profectūrus}, \na.  But such Verbs may have
a Gerund, and they may also have an \emph{impersonal} Future Passive
Participle.  Thus \latin{ad proficīscendum}, \english{for departing};
\apud{B.~G.}{1, 3, 1}; \latin{eī proficīscendum est}, \english{he
  must depart}; \apud{Fin.}{3, 22, 73}.

\begin{minor}

\subsubsection

For Future Passive Participles like \latin{laetandus}, \english{to be
  rejoiced at}, see \xref[\emph{a}, 3)]{290}.

\subsubsection

The Perfect Passive Participle of Deponents and Semi-Deponents is
sometimes used with a true passive force.\footnote{Especially of such
  Deponents as had also an active form in occasional use
  (e.g.\ \latin{pacīscor}, occasionally \latin{pacīscō};
  \latin{adipīscor}, occasionally \latin{adipīscō}).}

\begin{examples}

\latin{pactam diem},
\english{a date agreed upon};
\apud{Cat.}{1, 9, 24}.

\latin{adeptā lībertāte},
\english{after freedom had been won};
\apud{Sall.\ Cat.}{7, 3}.

\end{examples}

\end{minor}

\headingG{Subject of the Passive Voice}

\section

The Subject of the Passive Voice corresponds to the Direct Object
(\xref{390}) of the Active.  Thus \emph{Dick struck Tom} (Active
Voice) becomes in the Passive \emph{Tom was struck by Dick}.

\begin{minor}

\subsubsection

Verbs that do not take an Accusative Object (\xref{390}) in the Active
Voice are regularly used only \emph{impersonally} (\xref{287}) in the
Passive, with the same cases as in the Active.
\begin{examples}

\latin{ut hostibus nocērētur},
\english{that harm might be done to the enemy};
\apud{B.~G.}{5, 19, 3}.
Compare \latin{nocēre alterī}, under~\xref[I]{362}.

\latin{num argūmentīs ūtendum?}
\english{must one make use of arguments?}
\apud{Verr.}{4, 6, 11}.  Compare~\xref{429}.

\end{examples}

\subsubsection

Yet Passives are sometimes formed from such verbs. Thus
\latin{crēdita}, \english{believed}, \apud{Aen.}{2, 247};
\latin{persuāsus est}, \english{is persuaded},
\apud{Caecin.\ ap.\ Fam.}{6, 712}; \latin{invideor}, \english{I am
  envied}, \apud{A.~P.}{56}.

\end{minor}

\headingB{Adverbs}

\begin{minor}

\section[\textsc{\small Introductory}]

As explained in \xref{124}, \xref{126}, many Adverbs are simply
stereotyped case-forms, e.g.\ \latin{partim} (\english{as regards a
  part}), \english{partly} (old Acc.\ of Respect, \xref{388}),
\latin{hāc}, \english{by this way} (Abl.\ of Route, \xref{426}),
\latin{vērō}, \english{in truth} (Abl.\ of Respect, \xref{441}),
\latin{modo} (with a measure, exactly), \english{just} (Abl.\ of
Manner, \xref{445}), \latin{miserē}, \english{in a wretched manner}
(old Ablative, \xref[1]{126}).  A few are made up of Prepositions with
a case, as \latin{admodum} (to a degree), \english{very}.
Cf.~\xref[4]{217}.

\end{minor}

\section

Adverbs express ideas of manner, degree, place, time, etc.

\begin{minor}

Thus \latin{ita}, \english{so} or \english{so much}, \latin{ibi},
\english{there}, \latin{tum}, \english{then}.

\end{minor}

\section

Adverbs modify Verbs, Adjectives, and other Adverbs (or Adverbial
Phrases).
\begin{examples}

\latin{ita exercitum trādūcit},
\english{in this way he takes the army across};
\apud{B.~G.}{1, 13, 1}.

\latin{quārtam ferē partem},
\english{about a fourth part};
\apud{B.~G.}{1, 12, 2}.

\latin{minus facile},
\english{less easily};
\apud{B.~G.}{1, 2, 4}.

\latin{paene in cōnspectū},
\english{almost within sight};
\apud{B.~G.}{1, 11, 3}.

\end{examples}

\begin{minor}

\subsubsection

Adverbs of number or degree may also, through brevity of expression,
seem to modify Nouns.
\begin{examples}

\latin{bis ūnā cōnsulēs},
\english{twice consuls together}
(= who had twice been consuls together);
\apud{Am.}{1, \emend{12}{139}{39}}.

\end{examples}

\subsubsection

In poetry and later prose, other Adverbs sometimes modify Nouns
\emph{implying action}.
\begin{examples}

\latin{populum lātē rēgem},
\english{a people monarch \emph{\(= ruling\)} far and wide};
\apud{Aen.}{1, 21}.

\latin{haud dubiē victor},
\english{beyond doubt a victor}
(= victorious);
\apud{Sall.\ Iug.}{102, 1}.

\end{examples}

\end{minor}

\subsubsection

A few Adverbs are freely used in the sense of Adjectives, especially
\latin{ita}, \latin{sīc}, \latin{satis}, \latin{bene}, \latin{male}.
\versionB*{The poets extend the list.}
\begin{examples}

\latin{quod satis esse arbitrābātur},
\english{which he thought to be sufficient};
\apud{B.~G.}{4, 22, 6}.

\latin{sīc sum},
\english{that’s the way I am}
(that’s the kind of man);
\apud{Ph.}{527}.

\end{examples}

\headingB{Negative Adverbs}

\section
\subsection

The Sentence-Negative for the ideas of \emph{Command}, \emph{Will}, or
\emph{Wish} is \latin{nē}, \english{not}; or, if the negative is also
a connective, \latin{nēve} or \latin{neu}, \english{and not},
\english{nor}.

\subsubsection

\latin{Nē} and \latin{nēve} (\latin{neu}) also become
Conjunctions. See, e.g.,~\xref[2, 3]{502}.

\subsection

The Sentence-Negative for \emph{Statements} or corresponding Questions is
\latin{nōn}, \english{not}; or, if also a connective, \latin{neque},
\english{and not}, \english{nor}.

\begin{minor}

\subsubsection

For further details with regard to the negatives, see~\xref{464}.

\end{minor}

\section

\latin{Haud} (\latin{haut}, \latin{hau}) negatives a single word.  In
Ciceronian use, it is employed sparingly,—mostly to modify
Adjectives and Adverbs expressing Quantity, Kind, or Manner.
\begin{examples}

\latin{haud mediocris vir},
\english{no ordinary man};
\apud{Rep.}{2, 31, 55}.

\latin{haud facile},
\english{not easily};
\apud{Rep.}{1, 3, 6}.

\end{examples}

\begin{minor}

\subsubsection

\latin{Haud} is also used with a few Verbs, as \latin{sciō}
(\apud{B.~G.}{5, 54, 5}), \latin{dubitō} (\apud{Rep.}{1, 15, 23}).

\end{minor}

\section
\subsection

Instead of \latin{dīcō nōn}, \english{I say that\ellipsis not},
\latin{negō} is preferred.
\begin{examples}

\latin{negāvī mē esse factūrum},
\english{I said I would not \(so\) act};
\apud{Cat.}{3, 3, 7}.

\end{examples}

\subsection

In general, two negatives make an affirmative.
\begin{examples}

\latin{videō abesse nōn nēmimen},
\english{I see that some one is absent};
\apud{Cat.}{4, 5, 10}.

\end{examples}

\begin{minor}

\subsubsection

But after a sweeping negative, the negatives \latin{nē\ellipsis quidem},
\latin{neque\ellipsis neque}, or \latin{nēve\ellipsis nēve} simply add
emphasis.
\begin{examples}

\latin{numquam illum nē minimā quidem rē offendī},
\english{I never offended him, not even in the smallest thing};
\apud{Am.}{27, 103}.

\end{examples}

\end{minor}

\section

When the phrase \latin{nōn modo} (or \latin{nōn sōlum})\latin{\ellipsis
  sed nē\ellipsis quidem} is used in a sentence containing but a single
verb, the second negative is felt throughout the whole (\english{not
  only not\ellipsis but not even}).
\begin{examples}

\latin{tālis vir nōn modo facere, sed nē cōgitāre quidem quicquam
  audēbit, quod nōn audeat praedicāre},
\english{such a man will not only \textsc{not} venture to do a thing he
  dare not speak of, but will not even dare to think of it};
\apud{Off.}{3, 19, 77}.

\end{examples}

\pagebreak

\headingB{Comparison of Adverbs}

\section

The Comparative and Superlative degrees of Adverbs correspond in
meaning to those of Adjectives (\xref{241}).  Thus \latin{facile},
\english{easily}; \latin{facilius}, \english{more easily} or
\english{rather easily}; \latin{facillimē}, \english{most easily} or
\english{very easily}; \latin{vel facillimē}, \english{very easily
  indeed}; \latin{quam facillimē}, \english{as easily as possible}.

\section[Two Comparatives]

When an act is said to be done in one way rather than in another
(English \english{with more\ellipsis than\dots}, \english{rather\ellipsis than
  \dots}), both Adverbs regularly take the same form (cf.~\xref{242}).
\begin{examples}

\latin{libentius quam vērius},
\english{with more readiness than truth};
\apud{Mil.}{29, 78}.

\latin{magis honestē quam vērē},
\english{rather in compliment than truthfully};
\apud{Planc.}{15, 37}.

\end{examples}

\headingG{Forces of Certain Important Adverbs}

\section
\subsection

\latin{Quidem}, \english{to be sure}, \english{indeed}, \english{at
  any rate} (postpositive\footnote{I.e.\ put immediately after the
  word on which the particle bears.}), is a particle of
\emph{emphasis}, generally expressing either a moderate concession or
a moderate claim. It is often followed by \latin{sed}, \latin{autem},
etc.
\begin{examples}

\latin{dīcitur quidem ā Cottā; sed\dots},
\english{Cotta does say so, to be sure; but\dots};
\apud{Div.}{1, 5, 8}.
(Moderate Concession.)

\latin{mihi quidem illa certissima vīsa sunt argūmenta},
\english{to me, at any rate, these things seemed indubitable proofs};
\apud{Cat.}{3, 5, 13}.
(Moderate Claim.)

\end{examples}

\begin{minor}

\subsubsection

For \latin{quidem} (and \latin{equidem}) with pronouns,
see~\xref[\emph{a}]{257}.

\end{minor}

\subsection

\latin{Etiam} and \latin{et},\footnote{The same words as the
  Conjunctions \latin{etiam} and \latin{et}, but used Adverbially.}
\english{even}, \english{also} (regularly
prepositive\footnote{I.e.\ put immediately before the word on which
  the particle bears.}), are used as strengthening particles.

\latin{Quoque}, \english{also}, \english{too} (postpositive),
expresses mere addition.
\begin{examples}

\latin{etiam in extrēmā spē},
\english{even at the last ebb of hope};
\apud{B.~G.}{2, \emend{151}{27}{26}, 3}.

\latin{vērum et aliī multī},
\english{but also many others};
\apud{Rosc.\ Am.}{33, 94}.

\latin{haec quoque ratiō (eōs dēdūxit)},
\english{this reason, too, \(impelled them\)};
\apud{B.~G.}{2, 10, \emend{106}{15}{5}}.

\end{examples}

\begin{minor}

\subsubsection

\latin{Etiam} modifying a phrase containing no preposition is
generally placed \emph{inside} that phrase.  Thus \latin{nostrā etiam
  memoriā}, \english{even within our memory}; \apud{B.~G.}{2, 4, 7}.

\subsubsection

\latin{Et} in the sense of \latin{etiam} is not used by Caesar.

\subsubsection

The later writers use \latin{etiam} (or \latin{et}) and \latin{quoque}
with less careful distinction.

\end{minor}

\subsection

\latin{Prīmō} and \latin{prīmum} should be carefully distinguished.
With \latin{prīmō}, \english{at first}, the idea of \emph{time} is
more important; with \latin{prīmum}, \english{firstly}, the idea of
\emph{logical order}.

These Adverbs often begin a series (more or less complete).  Thus:
\begin{mexamples}[2]

\latin{prīmō} (= \latin{prīncipiō}), \english{at first}, \english{at
  the beginning},
\latin{deinde} (\latin{inde}) or \latin{posteā}, \english{later},
\latin{tum}, \english{then}, etc.,
\latin{postrēmō} or \latin{dēnique}, \english{finally}.

\latin{prīmum}, \english{firstly}, \english{in the first place},
\latin{deinde} (\latin{inde}) or \latin{posteā}, \english{secondly},
\latin{tum}, \english{then}, etc.,
\latin{postrēmō} or \latin{dēnique}, \english{lastly}.

\end{mexamples}

\begin{examples}

\latin{ille prīmō negāvit; post autem aliquantō surrēxit,
  quaesīvit\dots},
\english{at first he denied; a little later, however, he rose and
  asked};
\apud{Cat.}{3, 5, 11}.

\latin{id aliquot dē causīs acciderat, prīmum, quod\dots, tum etiam
  quod\dots; accēdēbat quod\dots},
\english{this had come about through several reasons; first,
  because\dots; then also because\dots; further because\dots};
\apud{B.~G.}{3, 2, 2}.

\end{examples}

\begin{minor}

\subsubsection

The feeling of logical order sometimes prevails, even where the idea
of order in time is also present.  Thus \latin{prīmum Antiochīae, nam
  ibi nātus est, \dots; post in cēterīs Asiae partibus\dots},
\english{first at Antioch, for this was his birthplace\dots; then in
  the rest of Asia\dots}; \apud{Arch.}{3, 4}.

\end{minor}

\subsection

\latin{Nunc}, \english{now}, deals with a single point of time,
without reference to any other.  Thus \latin{nunc adest}, \english{he
  is now present}.

\begin{minor}

\subsubsection

After a Condition Contrary to Fact (\xref{581}), \latin{nunc} means
\english{as it is}.

\end{minor}

\subsection

\latin{Iam}, \english{by this time}, \english{already}, contrasts a
time with a preceding one.  Thus \latin{iam aderat}, \english{he was
  by this time present} (had not been before); \latin{iam adest},
\english{he is by this time present} (has not been before); \latin{iam
  aderit} (\apud{Aen.}{2, 662}), \english{he will soon be present}
(is not now).

With negatives, \latin{iam} means \english{no longer} (by this time,
\emph{not}).

With the Imperfect, \latin{iam} may suggest the \emph{beginning} of an
act or state.  Thus \latin{quod iam incrēdibile vidēbātur},
\english{which was beginning to seem incredible}; \apud{Pomp.}{14,
  41}.

\subsection

\latin{Potius}, \english{preferably}, \english{rather}, and
\latin{potissimum}, \english{in preference to all other} persons or
things, express the idea of \emph{selection}.
\begin{examples}

\latin{iīs potissimum ostendam, quī\dots},
\english{I shall display it to those before all others, who\dots};
\apud{Pomp.}{1, 2}.

\end{examples}

\subsection

\latin{Adeō}, \latin{eō}, and \latin{tam} express \emph{degree},
\latin{ita} and \latin{sīc} \emph{manner}, occasionally \emph{degree}.
(For other Correlatives, see~\xref{144}.)

\subsection

\latin{Nē}, \english{surely}, should be carefully distinguished from
\latin{nē}, \english{not}, \english{lest}.
\begin{examples}

\latin{nē illī vehementer errant},
\english{surely they are grievously in error};
\apud{Cat.}{2, 3, 6}.

\end{examples}

\headingB{Prepositions}

\section

Prepositions define the relation of a Substantive to
another word.
\begin{examples}

\latin{iter per prōvinciam},
\english{a journey through the province};
\apud{B.~G.}{1, 14, 3}.

\end{examples}

\subsubsection

Prepositions were originally Adverbs, modifying, not the Noun, which
at a later time they seemed to govern, but a Verb or Adjective.  At
this period, all case-relations were expressed by the bare Case alone.
Thus a sentence like \latin{portā ab iit} would have been used to
express the idea \english{from the gate, he went away}.  But such a
combination suggested a \emph{relation} between the Noun and the Verb
(\english{he went away from the gate}).  In consequence, the Adverb
came to be placed \emph{before the Noun}, whence the name Preposition
(“placed in front”).

\begin{minor}

\subsubsection

In certain combinations, the Adverb remained permanently attached to
the Verb, as in \latin{īnferō}, \english{bring-in}.  In others, it
remained with the Verb, even when repeated (as Preposition) with the
Noun, as in \latin{ā portā abiit}, \english{he went-away from the
  gate}.  It is customary and convenient to call such Verbs
\emph{prepositional compounds}.

\subsubsection

Certain words can be used either as Prepositions or as Adverbs.  So
especially \latin{ante}, \latin{adversus}, \latin{circā},
\latin{circum}, \latin{circiter}, \latin{contrā}, \latin{post},
\latin{prope}, \latin{super}.
\begin{examples}

\latin{annō post},
\english{a year after}
(= afterward by a year);
\apud{B.~G.}{4, 1, 5}.

\end{examples}

\end{minor}

\headingB{Conjunctions}

\section

Conjunctions connect words, phrases, sentences, or clauses.  They are
of two main kinds:

\section

%%* loose line

I.\enskip \term{Coördinating Conjunctions} join words, phrases, sentences, or
\linebreak
clauses of equal rank and essentially similar nature.
\begin{examples}

\latin{nōbilissimus et dītissimus},
\english{the noblest and the richest man};
\apud{B.~G.}{1, 2, 1}.

\latin{cōnsulem interfēcerat et eius exercitum sub iugum mīserat},
\english{had killed the consul and sent his army under the yoke};
\apud{B.~G.}{1, 12, 5}.

\end{examples}

\subsubsection

\term{Asyndeton}, or “want of connective.”  The same effect of
joining is often produced still more sharply by using no connective at
all.
\begin{examples}

\latin{frīgus, sitim, famem ferre poterat},
\english{he could bear cold, thirst, hunger};
\apud{Cat.}{3, 7, 16}.

\latin{senātus haec intellegit, cōnsul videt},
\english{the senate knows all this, the consul sees it};
\apud{Cat.}{1, 1, 2}.

\end{examples}

\begin{note}

In certain common phrases the conjunction is habitually omitted.  Thus
\latin{Iuppiter Optimus Maximus}, cf.~\apud{Cat.}{3, 9, 21};
\latin{volēns propitius}, \apud{Liv.}{1, 16, 3}; \latin{vultis
  iubētisne}, cf.~\apud{Liv.}{1, 46, 1}.  So generally with the names
of colleagues, unless a single name only is given for each.  Thus
\latin{L.\ Pīsōne A.\ Gabīniō cōnsulibus}, \apud{B.~G.}{1, 6, 4}; but
\latin{Lepidō et Tullō cōnsulibus}, \apud{Cat.}{1, 6, 15}.

\end{note}

II.\enskip \term{Subordinating Conjunctions} join a dependent clause to the
sentence or clause upon which it depends.
\begin{examples}

\latin{cum quaeret, sīc reperiēbat},
\english{when he inquired, he learned the following};
\apud{B.~G.}{2, 4, 1}.

\end{examples}

\chapter{Coördinating Conjunctions in Detail}

\section

Coördinating Conjunctions fall under four classes, according as they
express Union (Copulative Conjunctions), Separation (Disjunctive
Con\-junc\hyphenbreak tions), Opposition (Adversative
Conjunctions), or Inference (Inferential Conjunctions).

\headingE{Copulative Conjunctions: \lowercase{\latin{et},
    \protect\enclitic{-que}, \latin{atque}, \latin{ac}, \latin{neque},
    \latin{nēve}}}

\section
\subsection

\latin{Et} expresses simple connection (examples in \xref[I.]{305});
while \enclitic{-que} expresses closer connection,—often one which
exists in the nature of things.
\begin{examples}

\latin{multitūdō perditōrum hominum latrōnumque},
\english{a multitude of desperadoes and brigands};
\apud{B.~G.}{3, \emend{152}{17, 4}{15, 8}}.

\latin{eī legiōnī castrīsque},
\english{this legion and camp};
\apud{B.~G.}{6, 32, 6}.

\end{examples}

\subsubsection

But a natural connection is often left \emph{unexpressed}, as in
\latin{impedītōs et inopīnantīs}, \english{encumbered and off their
  guard}; \apud{B.~G.}{1, 12, 3}.

\begin{minor}

\subsubsection

When \enclitic{-que} introduces a word, it is attached to it. Thus
\latin{oppida vīcōsque}, \english{towns and villages};
\apud{B.~G.}{1, 28, 3}.

When it introduces a phrase, it is generally attached to the first
word of that phrase; but if that first word is a preposition, the
\enclitic{-que} is generally attached to the second word of the
phrase. Thus \latin{ob eāsque rēs}, \english{and on account of these
  achievements}; \apud{B.~G.}{2, \emend{119}{35}{34}, 4}.

When it introduces a clause, it is generally attached to the first
word of that clause, and this word is generally \emph{not} the verb.
Thus, \latin{duāsque ibi legiōnēs cōnscrībit}, \english{and there
  enrolls two legions}; \apud{B.~G.}{1, 10, 3}.

\subsubsection

When several members are put together in a series, Latin ordinarily
uses the connective throughout, or not at all.
\begin{examples}

\latin{turpem et īnfirmam et abiectam},
\english{base and weak and downcast};
\apud{Cat.}{4, 10, 20}.

\latin{ferōx, vehemēns, prōmptus},
\english{rough, ardent, quick};
\apud{Sall.\ Cat.}{43, 4}.

\end{examples}

\subsubsection

Sometimes, however, in Latin as in English, the last two members only
are connected (generally by \enclitic{-que}, rarely by \latin{et}).
\begin{examples}

\latin{pācem, tranquillitātem, ōtium, concordiamque},
\english{peace, tranquility, repose, and concord};
\apud{Mur.}{1, 1}.

\end{examples}

\end{minor}

\subsection

\latin{Atque} or \latin{ac}, \english{and also}, \english{and indeed},
\english{and}, likewise expresses close connec\-tion,—sometimes with
stress upon the word which it introduces.
\begin{examples}

\latin{ā cultū atque hūmānitāte prōvinciae},
\english{from the civilization and refinement of the Province};
\apud{B.~G.}{1, 1, 3}.

\latin{habetī ingeniō atque nūllō},
\english{of a dull mind, and indeed of none at all};
\apud{Tusc.}{5, 15, 45}.

\end{examples}

\begin{minor}

\subsubsection

After words of likeness or difference, \latin{atque} or \latin{ac} has
the force of \english{as} or \english{than}.  Thus after \latin{īdem},
\latin{is}, \latin{aequus} or \latin{aequē}, \latin{alius} or
\latin{aliter}, \latin{contrā}, \latin{pār} or \latin{pariter},
\latin{similis} or \latin{similiter}, \latin{simul}.
\begin{examples}

\latin{Gallōrum eadem atque Belgārum oppugnātiō est haec},
\english{the Gallic way of storming is the same as that of Belgians,
  as follows};
\apud{B.~G.}{2, 6, 2}.

\latin{prō eō ac mereor},
\english{according as I deserve}
(in proportion to that, as);
\apud{Cat.}{4, 2, 3}.

\end{examples}

\subsubsection

\latin{Alius} and \latin{aliter} may also be followed by \latin{nisi},
\english{except}, or \latin{quam}, \english{than}.

\subsubsection

For the choice between the forms \latin{atque} and \latin{ac},
see~3,~\emph{c}, below.

\end{minor}

\subsection

\latin{Neque} (\latin{nec}), and \latin{nēve} (\latin{neu}),
\english{and not}, \english{nor}, are at the same time negatives and
connectives.  (For the difference between them, see~\xref{464}.)
\begin{examples}

\latin{Orgetorīx mortuus est; neque abest suspīciō\dots},
\english{Orgetorix died; and a suspicion is not lacking\dots};
\apud{B.~G.}{1, 4, 3\emend{153}{}{–4}}.

\end{examples}

\begin{minor}

\subsubsection

The idea “and not” is regularly expressed in Latin (as in the above
examples) by \latin{neque} or \latin{nēve}, not by \latin{et nōn} or
\latin{et nē}.  Similarly “and none” is expressed by \latin{nec
  ūllus}, “and never” by \latin{nec umquam}; etc., etc.
\begin{examples}

\latin{resistere neque dēprecārī},
\english{to resist and not beg off};
\apud{B.~G.}{4, 7, 3}.

\end{examples}

\subsubsection

But \latin{et nōn} may be used to express \emph{contrast} or
\emph{emphasis}.
\begin{examples}

\latin{manēre et nōn discēdere},
\english{to remain and \textsc{not} give way};
\apud{Caecil.}{2, 5}.

\latin{perinīquum et nōn ferundum},
\english{very unjust, and \textsc{not} to be endured};
\apud{Pomp.}{22, 63}.

\end{examples}

\subsubsection

The forms \latin{atque} and \latin{neque} are used before either vowels or
(less frequently) consonants, \latin{ac} and \latin{nec} only before
consonants (rarely before a guttural, as in \latin{ac contrā},
\apud{B.~G.}{1, 44, 3}).  But the poets allow themselves more freedom.
\begin{examples}

\latin{atque ea}, \apud{B.~G.}{1, 1, 3};
\latin{atque pecore}, \apud{}{4, 1, 8};
\latin{neque eam}, \apud{}{3, 2, 3};
\latin{neque pedibus}, \apud{}{3, 12, 1};
\latin{ac lassitūdine}, \apud{}{2, \emend{154}{23}{22}, 1};
\latin{nec locō}, \apud{}{7, 48, 4}.
(But \latin{nec exanimēs}, \apud{Aen.}{5, 669}.)

\end{examples}

\end{minor}

\headingE{Disjunctive Conjunctions: \lowercase{\latin{aut}, \latin{vel},
  \protect\enclitic{-ve}, \latin{sīve} (\latin{seu})}}

\section
\subsection

\latin{Aut}, \english{or}, is used to connect alternatives.  These may
both be possible, or they may be mutually exclusive.
\begin{examples}

\latin{cūr dē suā virtūte aut dē ipsīus dīligentiā dēspērārent?}
\english{why \emph{\(\emend{13}{Cæsar}{Caesar} asked\)} should they
  despair of their own valor or of his vigilance?}
\apud{B.~G.}{1, 40, 4}. (They \emph{might} do both.)

\latin{hōrae mōmentō cita mors venit aut victōria laeta},
\english{in the brief space of an hour comes swift death or joyful
  victory};
\apud{Sat.}{1, 1, 7}.
(Only one \emph{could} come in a given case.)

\end{examples}

\subsection

\latin{Vel}\footnote{An old Imperative of \latin{volō}, meaning
  \english{choose}.} or \enclitic{-ve} (enclitic) is used to connect
alternatives between which there may be a \emph{choice}.
\begin{examples}

\latin{Catilīnam vel ēiēcimus vel ēmīsimus vel ipsum ēgredientem
  verbīs prōsecūtī su\-mus},
\english{we have turned Catiline out, or, if you choose, have sent him
  out, or, if you choose, have presented him our compliments as he
  went out of his own accord}; \apud{Cat.}{2, 1, 1}.

\end{examples}

\subsection

\latin{Sīve} or \latin{seu}, \english{or} (originally \english{or if})
is used to connect alternatives between which there is \emph{doubt}.
\begin{examples}

\latin{ēiectō sīve ēmissō ex urbe Catilīnā},
\english{when Catiline had been turned out of the city, or sent out};
\apud{Sull.}{5, 17}.

\end{examples}

\begin{minor}

\subsubsection
\latin{Aut}, \latin{vel}, or \latin{sīve} may introduce a
\emph{correction} (“or rather,” “or perhaps”).

\end{minor}

\headingG{Copulative or Disjunctive Conjunctions in Pairs}

\section

The following pairs of Conjunctions are in frequent use.

\begin{examples}

\latin{et\ellipsis et\dots},
\english{both\ellipsis and\dots};
\apud{Arch.}{1, 1}.

\latin{neque (nec)\ellipsis neque (nec)\dots},
\english{neither\ellipsis nor\dots};
\apud{B.~G.}{2, \emend{155}{22}{21}, 1}.

\latin{et\ellipsis neque (nec)\dots},
\english{both\ellipsis and at the same time not\dots};
\apud{Cat.}{3, 8, 20}.

\latin{neque (nec)\ellipsis et\dots},
\english{not\ellipsis and at the same time\dots};
\apud{B.~G.}{2, \emend{104}{25}{24}, 1}.

\latin{aut\ellipsis aut\dots},
\english{either\ellipsis or\dots};
\apud{B.~G.}{1, 39, 4}.

\latin{vel\ellipsis vel\dots}\emend{14}{}{,}
\english{either\ellipsis or\ellipsis};
\apud{B.~G.}{1, 19, 5}.

\latin{sīve (seu)\ellipsis sīve (seu)\dots},
\english{whether\ellipsis or\dots};
\apud{B.~G.}{1, 12, 6}.

\end{examples}

\begin{minor}

\subsubsection

\latin{-que\ellipsis que\dots} and \latin{-que\ellipsis atque (ac)} are
found in later Latin.
\begin{examples}

\latin{sēque remque pūblicam},
\english{both themselves and the Commonwealth};
\apud{Sall.\ Cat.}{9, 3}.

\latin{sēque ac līberōs},
\english{themselves and their children};
\apud{Tac.\ Hist.}{3, 63}.

\end{examples}

\end{minor}

\headingE{Adversative Conjunctions: \lowercase{\latin{at},
    \latin{autem}, \latin{sed}, \latin{tamen}, \latin{vērō}, etc.}}

\section
\subsection

\latin{At}, \english{but}, \english{yet} (regularly first in its
clause), expresses contrast or objection.
\begin{examples}

\latin{quid tē impedit?  Mōsne maiōrum?  At persaepe etiam prīvātī
  perniciōsōs cīvīs morte multārunt},
\english{what hinders you?  The traditions of our ancestors?  But even
  men in private life have often punished mischief-making citizens
  with death};
\apud{Cat.}{1, 11, 28}.

\end{examples}

\begin{minor}

\subsubsection

\latin{At}, \english{but}, or \latin{at enim}, \english{but indeed},
may introduce the supposed objection of an adversary.
\begin{examples}

\latin{at rēs populāris},
\english{but, you will say, it is a popular movement};
\apud{Phil.}{1, 9, 21}.

\end{examples}

\subsubsection

\latin{At} often merely shifts the scene to another person or place.
\begin{examples}

\latin{pāret Amor dictīs cārae genetrīcis.  At Venus\dots},
\english{Cupid obeys his beloved parent’s words.  But Venus\dots};
\apud{Aen.}{1, 689}.

\end{examples}

\subsubsection

The form \latin{ast} is sometimes used in legal Latin and in poetry.

\end{minor}

\subsection

\latin{Autem}, \english{however}, \english{on the other hand}
(postpositive), expresses continuation and contrast.
\begin{examples}

\latin{hanc sī nostrī trānsīrent, hostēs exspectābant; nostrī autem,
  sī ab illīs initium trānseundī fieret, parātī erant},
\english{the enemy were waiting, in case our men should cross this
  \emph{\(swamp\)}; our men, on the other hand, were ready, in case
  the enemy should start to cross};
\apud{B.~G.}{2, 9, 1}.

\end{examples}

\subsubsection

Continuative \latin{autem} must sometimes be translated by
\english{now}, and sometimes must be left untranslated;
e.g.\ \latin{Rhēnus autem}, \apud{B.~G.}{4, 10, 3}.

\subsubsection

\latin{Autem} \emph{only rarely expresses addition} (“moreover”).

\subsection

\latin{Atquī}, \english{but at any rate}, \english{but yet},
\english{and yet}, is an emphatic \latin{at}.
\begin{examples}

\latin{atquī nihil interest},
\english{and yet there is no difference};
\apud{Balb.}{10, 26}.

\end{examples}

\subsection

\latin{Sed}, \english{but}, and the less common \latin{vērum},
\english{but in truth}, \english{but}, are used to modify or
contradict a previous statement.  They are often accompanied by
\latin{tamen}.
\begin{examples}

\latin{aetāte iam adfectum, sed tamen exercitātiōne rōbustum},
\english{feeling the effects of old age already, but nevertheless kept
  vigorous by exercise};
\apud{Cat.}{2, 9, 20}.  (Modification.)

\latin{reliquōs nōn ex bellō, sed ex tuō scelere},
\english{the survivors, not of war, but of your wickedness};
\apud{Verr.}{3, 54, 126}.  (Contradiction.)

\end{examples}

\begin{minor}

\subsubsection

\latin{Cēterum}, \english{but}, resembles \latin{sed} in meaning (not
in Cicero or Caesar as a true conjunction).

\subsubsection

\latin{Sed} and \latin{vērum} often follow \latin{nōn}, in pairs of
phrases.  Thus
\begin{examples}

\latin{nōn sōlum (modo)\ellipsis sed (vērum)},
\english{not only\ellipsis but\dots};
\apud{Cat.}{3, 10, 24}.

\end{examples}

\latin{Etiam} or \latin{quoque}, \english{also}, is often added to the
\latin{sed} or \latin{vērum}.  Thus
\begin{examples}

\latin{nōn sōlum mīlitāris virtūs, sed aliae quoque virtūtēs};
\apud{Pomp.}{22, 64}.

\end{examples}

\end{minor}

\subsection

\latin{Vērō}, \english{in fact}, \english{indeed}, \english{but},
\english{however} (postpositive), is used to express strong contrast
or emphasis.
\begin{examples}

\latin{mihi vērō ferreus},
\english{to me, indeed, he \emph{\(would seem\)} hard of heart};
\apud{Cat.}{4, 6, 12}.

\end{examples}

\begin{minor}

\subsubsection

\latin{Autem} and \latin{vērō} are interchangeable, but \latin{vērō}
is stronger.

\subsubsection

\latin{Vērō} is often on the doubtful line between Conjunction and
Adverb.

\end{minor}

\subsection

\latin{Tamen}, \english{yet}, \english{nevertheless}, expresses
something as true in spite of a previous concession, objection, or
difficulty.  It may be placed either at the beginning of a clause or
after the emphatic word.
\begin{examples}

\latin{vehementissimē pertubātus, tamen signum cognōvit},
\english{though greatly disturbed, still he recognized the seal};
\apud{Cat.}{3, 5, 12}.

\end{examples}

\subsection

\latin{Quamquam}, \latin{etsī}, and \latin{tametsī}, \english{and
  yet}, \english{however}, are sometimes used to introduce a
modification or objection made by the speaker (\emph{Corrective}
\latin{quam\-quam}, \latin{etsī}, \latin{tametsī}).
\begin{examples}

\latin{quamquam quid loquor!}
\english{and yet why am I talking!}
\apud{Cat.}{1, 9, 22}.

\end{examples}

\headingE{Inferential Conjunctions}

\headingC{\latin{ergō}, \latin{igitur}, \latin{itaque},
  \latin{quārē}, \latin{proinde}, \latin{nam}, \latin{enim}}

\section
\subsection

\latin{Ergo}, \english{therefore}, expresses either natural result or
logical inference.

\subsection

\latin{Igitur}, \english{accordingly}, \english{therefore},
\english{then} (usually postpositive), expresses natural result or
logical inference, or the resumption of an interrupted thought.

\subsection

\latin{Itaque} (\english{and so}), \english{accordingly}, introduces
an action naturally following from a preceding one, or an example of
something stated just before.

\subsection

\latin{Quārē}, \english{wherefore}, introduces a consequence.

\subsection

\latin{Proinde} (\english{forth from that}), \english{therefore}, and
sometimes \latin{igitur} and \latin{quārē}, introduce an inference
which is also a command or exhortation.
\begin{examples}

\latin{proinde exeant},
\english{let them therefore depart};
\apud{Cat.}{2, 5, 11}.

\end{examples}

\subsection

\latin{Nam} and \latin{enim},\footnote{Originally \english{indeed}.}
\english{for}, introduce an explanation of what has preceded, a
justification of it, or a fuller statement.  \latin{Enim} is
postpositive.

\begin{minor}

\subsubsection

%%* Override righthyphenmin

\latin{Namque}, \english{for indeed}, is stronger than \latin{nam},
and \latin{etenim}, \english{for indeed},\footnote{Originally
  \english{and indeed}.} strong\-er than \latin{enim}.  (Note that
\latin{etenim} begins the clause, since in it the postpositive
\latin{enim} has an \latin{et} to which to attach itself.)

\subsubsection

In \latin{nec enim} and \latin{sed enim}, \latin{enim} has its
original sense of \english{indeed}.
\begin{examples}

\latin{nec requiēvit enim},
\english{nor indeed did he rest};
\apud{Aen.}{2, 100}.

\latin{sed enim audierat},
\english{but she had heard indeed};
\apud{Aen.}{1, 19}.

\end{examples}

\end{minor}

\chapter{Subordinating Conjunctions}

\section

These can be understood only in connection with the constructions in
which they are found, and accordingly will be treated under the Uses
of the Moods.

\headingB{Interjections}

\section

Interjections are exclamatory words (1)~expressing feeling,
(2)~calling attention to some one or something, or (3)~calling the
attention of a person addressed to the speaker.

Thus \latin{ā} or \latin{āh!} \english{alas!} \
\latin{ecce!} \english{behold!} \
\latin{o}, \english{O}.

\numchapter{The Expression of Ideas through Cases, Moods, and Tenses}

\contentsentry{A}{The Expression of Ideas through Cases, Moods, and
  Tenses}

\contentsentry{C}{General Principles}

\headingB{Principles of Grammatical Expression}

\section
\subsection

The varying forms of Nouns, Pronouns, and Adjectives make, beside
other things, what are called \term{Cases}; the varying forms of Verbs
make, beside other things, what are called \term{Moods} and
\term{Tenses}.

\begin{minor}

\subsection

The study of Latin Syntax is in large degree the study of \emph{the
  way in which the Romans expressed ideas by Cases, Moods, and
  Tenses}.

\subsection

A given way of expressing an idea by a Case, a Mood, a Tense, etc., is
called a \term{Construction}.

\end{minor}

\section

Each Case, each Mood, and each Tense probably had at one time a single
meaning of a simple kind, or a limited sphere of closely related
meanings.\footnote{But see, for a probable or possible exception,
  footnote, p.~\pageref{ftn:303:}.}

There took place, however, partly in the parent speech, partly in
Latin itself, a large growth and change of these meanings; and in
Latin literature we find \emph{many} meanings of the Cases, and
\emph{many} meanings of the Moods and Tenses.

These growths came about mainly in four ways:

\subsection

Through the \term{Figurative Use} of a Case, a Mood, or a Tense.
\begin{minor}

Thus \latin{prō castrīs}, \english{before the camp} (literal
place-idea), but also \latin{prō patriā}, \english{in defense of
  country} (figurative idea).

\end{minor}

\subsection

Through the \term{Association} of a new idea with an existing
construction.
\begin{minor}

Thus the idea of \emph{Definition} or \emph{Explanation} (\xref{341})
grows up through association with the Genitive in combinations like
\latin{nōmen poētae}, \english{the name of poet} (originally merely
\english{the name which belongs to a poet}).

\end{minor}

\subsection

Through the \term{Fusion} of two or more constructions into one.
(Constructions arising in this way may be called \emph{Constructions
  of Composite Origin}.)

\begin{minor}

Thus three different Kinds of Ablative may express \emph{Cause}
(\xref{444}): the Separative, as in our “ill from anxiety”
(cf.~\xref[\emph{b}]{444}), the Sociative, as in “ill with anxiety,”
and the Locative, as in “you take pleasure in my anxiety.”  There is
evidence that Latin originally expressed Cause in all three of these
ways.  But since the form in the developed language was the \emph{same} for
all three, there must to the Roman feeling have seemed to be merely a
\emph{single} construction of Cause.

\end{minor}

\subsection

Through \term{Analogy}, i.e.\ the influence of one or more
constructions upon another resembling them in meaning.
\begin{minor}

Thus, since the Ablative was used with \latin{vēscor}, \english{feed},
\english{eat}, it might occur to some one to use the same case with
\latin{epulor}, \english{feast},—as it did to Virgil in
\apud{Aen.}{3, 224} (see~\xref[\emph{d}]{429}).  This particular use
is exceptional; but many \emph{fixed uses} grew up in just such a way.

\end{minor}

\headingB{Agreement}

\contentsentry{C}{Agreement}

\section

By \term{Agreement} a word is put in the same case, number, etc., as a
Noun or Pronoun, to show that it \emph{belongs with} that Noun or
Pronoun.

\section

There are \emph{three ways} in which an agreeing word may be attached
to its Noun or Pronoun:

\subsection

A word may be \emph{closely united} with its Noun or Pronoun.  Words
so used are called \term{Attributive}.\footnote{The word
  \emph{adherent} would more exactly describe the relation.}
\begin{examples}

\latin{hic vīlicus},
\english{\textsc{this} steward}.
(\latin{Hic} is Attributive.)

\latin{vēlicus meus bonus},
\english{\textsc{my good} steward}.
(\latin{Meus} and \latin{bonus} are Attributive.)

\end{examples}

\subsection

A word may be \emph{loosely added} to its Noun or Pronoun.  Words so
used are called \term{Appositive}\footnote{Nouns so attached are
  regularly called Appositive, as here.  \emph{Adjectives} similarly
  attached have regularly been called Attributive.  But there is no
  difference of relation, and it is better to use the same word in
  both cases.} (i.e.\ \emph{put beside}).
\begin{examples}

\latin{vīlicus meus, adiūtor rērum meārum},
\english{my steward, \textsc{the aid} of my fortunes}.
(\latin{Adi\-ū\-tor} is Appositive.)

\latin{vīlicus meus, bonus et impiger},
\english{my steward, \textsc{good} and \textsc{energetic}}.
(\latin{Bonus} and \latin{impiger} are Appositive.)

\latin{vīlicus meus, rēs meās adiūtāns},
\english{my steward, \textsc{aiding} my fortunes}.
(\latin{Adiūtāns} is Appositive.)

\end{examples}

\subsubsection

An Appositive may be defined as a word loosely attached to another to
exhibit it \emph{under some special aspect}.  Thus \latin{Caesar
  cōnsul} means \english{Caesar \textsc{in the capacity of} consul},
\english{Caesar \textsc{as} consul}.

\begin{minor}

\subsubsection

Apposition is, in reality, a sort of \emph{shortened} Predication.
Thus \latin{Caesar cōnsul} means \emph{Caesar—he was at the time
  consul—}, etc.

\subsubsection

An attributive or appositive word may express Condition, Cause, or
Opposition.  Thus \latin{prīvātus}, \english{although in private
  life}; \apud{Cat.}{1, 1, 3}.  See also~\xref[6]{578}.

\end{minor}

\subsection

A word may be \emph{predicated} of its Noun or Pronoun
(see~\xref{229}).  Words so used are called \term{Predicates}, or
\term{Predicative}.
\begin{examples}

\latin{vīlicus meus bonus et impiger est},
\english{my steward \textsc{is good} and \textsc{energetic}}.
(\latin{Est} is a Predicate Verb, and \latin{bonus} and
  \latin{impiger} are Predicate Adjectives.)

\latin{vīlicus meus adiūtor rērum meārum est},
\english{my steward \textsc{is} the \textsc{aid} of my fortunes}.
(\latin{Est} is a Predicate Verb, and \latin{adiūtor} a Predicate
Noun.)

\latin{vīlicus meus mē adiūtat},
\english{my steward \textsc{aids} me}.
(\latin{Adiūtat} is a Predicate.)

\end{examples}

\begin{minor}

\subsubsection
A \emph{Verb} can be attached to a Substantive in this way only.

\end{minor}

\chapter{General Statement of Agreement}

\section

\emph{So far as forms exist to make it possible, an Attributive,
  Appositive, or Predicative word agrees in Gender, Number, Person,
  and Case with the word to which it belongs.}

\chapter{Details of Agreement for Nouns, Adjectives, Participles,\\
 and Pronouns}

\headingE{Agreement with a Single Word}

\headingC{Agreement of Nouns}

\section

Nouns agree in \emph{Case} with the substantives to which they belong,
and, if possible, in \emph{Gender} and \emph{Number} also.

To these substantives they may be either \emph{appositive} or
\emph{predicative}.

I.\enskip \emph{Appositive Noun}:
\begin{examples}

\latin{C.\ Volusēnus, tribūnus},
\english{Gaius Volusenus, a tribune};
\apud{B.~G.}{3, 5, 2}.

\latin{Volsiniī, oppidum Tuscōrum},
\english{Volsinii, a city of the Etruscans};
\apud{Plin.\ N.~H.}{2, \emend{15}{139}{57}}.
(Agreement in \emph{gender} and \emph{number} impossible.)

\end{examples}

\begin{minor}

\subsubsection

\term{Partitive Apposition}.  A noun denoting a whole may be followed
by a distributive pronoun in apposition, or by two or more words in
apposition, each denoting a part.
\begin{examples}

\latin{quisque suōs patimur mānīs},
\english{we suffer, each his own spirit};
\apud{Aen.}{6, 743}.

\latin{duo rēgēs, ille bellō hic pāce, cīvitātem auxērunt},
\english{two kings built up the state, one by war, the other by
  peace};
\apud{Liv.}{1, 21, 6}.

\end{examples}

\end{minor}

\setcounter{subsubsection}{0}

II.\enskip \emph{Predicative Noun}:
\begin{examples}

\latin{stilus optimus dīcendī effector (est)},
\english{the pen is the best producer of eloquence};
\apud{De~Or.}{1, 33, 150}.
(Notice the Gender of \latin{effector}.)

\latin{pecūnia est effectrīx multārum voluptātum},
\english{money is the producer of many pleasures};
\apud{Fin.}{2, 17, 55}.
(Notice the Gender of \latin{effectrīx}.)

\end{examples}

\subsubsection

On the other hand, a noun may also be made to agree in Gender and
Number with an Appositive which is \emph{going to be} used.
\begin{examples}

\latin{illās omnium doctrīnārum inventrīcēs Athēnās},
\english{that inventor of all learning, Athens};
\apud{De~Or.}{1, 4, 13}.

\latin{rēgīna Pecūnia},
\english{the almighty Dollar}
(our lady Money);
\apud{Ep.}{1, 6, 37}.

\end{examples}

\subsubsection

Most nouns exist in but a single gender-form, and agreement with another
noun in Gender is therefore often impossible.

\subsubsection

A substantive clause (indicative, subjunctive, or infinitive) may be
used as an appositive or predicate.  See especially \xref{238}
and~\xref[1, \emph{a}), \emph{b})]{597}.

\headingC{Agreement of Adjectives and Participles}

\section

Adjectives and Participles agree in \emph{Case}, \emph{Gender}, and
\emph{Number} with the substantives to which they belong.

To these substantives they may be \emph{attributive},
\emph{appositive}, or \emph{predicative}.

I.\enskip \emph{Attributive Adjective or Participle}:
\begin{examples}

\latin{magnam partem},
\english{a large part};
\apud{B.~G.}{2, \emend{156}{20, 2}{19, 1}}.

\latin{ācta vīta},
\english{my past life};
\apud{Sen.}{11, 38}.

\end{examples}

II.\enskip \emph{Appositive Adjective or Participle}:
\begin{examples}

\latin{Lūcīlī rītū, nostrum meliōris utrōque},
\english{in the manner of Lucilius, a better man than either of us};
\apud{Sat.}{2, 1, 29}.

\latin{Dīviciācus, Caesarem complexus, obsecrāre coepit},
\english{Diviciacus, embracing Caesar, began to implore him};
\apud{B.~G.}{1, 20, 1}.

\end{examples}

III.\enskip \emph{Predicative Adjective or Participle}:
\begin{examples}

\latin{Caesar fit ab Ubiīs certior},
\english{Caesar is informed by the Ubii}
(made more certain);
\apud{B.~G.}{6, 10, 1}.

\latin{Gallia est omnis dīvīsa in partīs trēs},
\english{Gaul as a whole is divided into three parts};
\apud{B.~G.}{1, 1, 1}.

\end{examples}

\headingC{Agreement of Determinative Pronouns}

\section

Determinative pronouns agree in \emph{Case}, in \emph{Gender}, and in
\emph{Number} with the substantives to which they belong.

To these substantives they may be \emph{attributive} or
\emph{predicative}.

I.\enskip \emph{Attributive Pronoun}:
\begin{examples}

\latin{is diēs},
\english{this day};
\apud{B.~G.}{5, 39, 4}.

\latin{eās rēs},
\english{these facts};
\apud{B.~G.}{1, 14, 1}.

\end{examples}

II.\enskip \emph{Predicative Pronoun}:
\begin{examples}

\latin{haec fuit ōrātiō},
\english{their address was as follows}
(was this);
\apud{B.~G.}{4, 7, 2}.

\end{examples}

\headingC{Agreement of Relative Pronouns}

\section

Relative Pronouns agree with their Antecedents (\xref[\emph{a}]{281})
in \emph{Gender} and \emph{Number}, but their \emph{Case} depends upon
their relations in the Clauses to which they belong.
\begin{examples}

\latin{ad eam partem pervēnit quae nōndum flūmen trānsierat},
\english{came to the part which had not yet crossed the river};
\apud{B.~G.}{1, 12, 2}.
(Feminine Singular, because referring to \latin{eam partem};
Nominative, because the Subject of \latin{trānsierat}.)

\latin{omnīs clientīs, quōrum magnum numerum habēbat},
\english{all his clients, of whom he had a great number};
\apud{B.~G.}{1, 4, 2}.
(Masculine Plural, because referring to \latin{clientīs}; Genitive,
because depending upon \latin{numerum}.)

\end{examples}

\headingE{Agreement with Two or More Words\footnotemark}

\footnotetext{The uses of the Relative, which in no wise differ, are
  included in the statements of \xref{323}–\xref{326}.}

\section
\subsection

An Adjective, Participle, or Pronoun belonging or referring to two or
more substantives \emph{of the same} Gender and Number must agree with
them in Gender, and may be either of the Number of the nearest, or
Plural, even if the nearest is Singular.

\emph{Of the Number of the individual substantives}:
\begin{examples}

\latin{ventum et aestum nactus secundum},
\english{getting a favorable wind and tide};
\apud{B.~G.}{4, 23, 6}.

(Relative)
\latin{prō suā clēmentia ac mānsuētūdine, quam ipsī ab aliīs
  audīrent},
\english{in accordance with his clemency and gentleness, of which they
  themselves heard from others};
\apud{B.~G.}{2, \emend{16}{31, 4}{30, 4}}.

\end{examples}

\emph{Of the Plural Number}:

\begin{examples}

\latin{angēbant ingentis spīritūs virum Sicilia Sardiniaque āmissae},
\english{the lost \(\emph{i.e.}\ the loss of\) Sicily and Sardinia
  troubled the high-spirited man};
\apud{Liv.}{21, 1, 5}.

(Relative)
\latin{Cottae et Titurī calamitātem, quī occiderint},
\english{the fate of Cotta and Titurius, who fell};
\apud{B.~G.}{6, 37, 8}.

\end{examples}

\subsection

An Adjective, Participle, or Pronoun belonging or referring to two or
more substantives \emph{of different} Gender or Number, or both, may
agree with the nearest of them; otherwise it must be in the Masculine
Plural if \emph{one} of the substantives denotes a man, in the
Feminine Plural if one of them denotes a woman and \emph{none} of them
a man, or in the Neuter Plural if \emph{all} of them denote things.

\emph{Agreeing with the nearest substantive}:
\begin{examples}

\latin{signum et manum suam cognōvit},
\english{acknowledged his seal and hand};
\apud{Cat.}{3, 5, 12}.

(Relative)
\latin{nostrī nōn eādem alacritāte ac studiō quō ūtī cōnsuērant
  ūtēbantur},
\english{our men were not showing the same eagerness and zeal that
  they were in the habit of showing};
\apud{B.~G.}{4, 24, 4}.

\end{examples}

\emph{In the Masculine Plural where one substantive denotes a man}:
\begin{examples}

\latin{rēx rēgiaque classis profectī (sunt)},
\english{the king and the royal fleet set out};
\apud{Liv.}{21, 50, 11}.

\end{examples}

\pagebreak

\emph{In the Neuter Plural where all the substantives denote things}:
\begin{examples}

\latin{ubi īra et aegritūdō permixta sunt},
\english{when anger and grief are united};
\apud{Sall.\ Iug.}{68, 1}.

(Relative)
\latin{ūsus ac disciplīna, quae ā nōbīs accēpissent},
\english{the experience and discipline which they had gained from us};
\apud{B.~G.}{1, 40, 5}.

\end{examples}

\begin{minor}

\subsubsection

The Neuter Plural may be used even if the substantives are \emph{all}
Masculine or \emph{all} Feminine, \emph{provided they all denote
  things}.

\end{minor}

\headingG{Agreement by Form, by Sense, and by Attraction}

\section

In \term{Agreement by Form},\footnote{Also called Grammatical
  Agreement.} a word takes its Gender and Number from the \emph{form}
of the word or phrase to which it belongs.
\begin{examples}

\latin{sex mīlia hostium caesa},
\english{six thousand of the enemy were killed};
\apud{Liv.}{21, 60, 7}.

\end{examples}

\section

In \term{Agreement by Sense}, a word takes its Gender and Number from
the \emph{real meaning} of the word or phrase to which it belongs.  So
from a Collective Noun or Adverb, the name of a Country or Town, a
Possessive Pronoun or Adjective, or a Noun connected with another by
\latin{cum}.  Thus:
\begin{examples}

\latin{magna pars occīsī (sunt)},
\english{a large part were killed};
\apud{Sall.\ Iug.}{58, 2}.

\latin{cum partim ē nōbīs timidī sint, partim ā rē pūblicā āversī},
\english{since some of us are timid, and others hostile to the
  commonwealth};
\apud{Phil.}{8, 11, 32}.

\latin{Latium Capuaque multātī},
\english{Latium and Capua were punished};
\apud{Liv.}{8, 11, 12}.

\latin{nostrā, quī remānsissēmus, caede contentum},
\english{satisfied with killing us who had\linebreak stayed behind};
\apud{Cat.}{1, 3, 7}.

\latin{fīliam cum minōre fīliō, accītōs Amphipolim},
\english{the daughter with the younger son, being summoned to
  Amphipolis};
\apud{Liv.}{45, 28, 11}.

\end{examples}

\subsubsection

A Pronoun referring to the \emph{general thought} of what precedes, or
follows, is in the Neuter Gender.
\begin{examples}

\latin{diērum quīndecim supplicātiō dēcrēta est, quod ante id tempus
  accidit nūllī},
\english{a thanksgiving of fifteen days was voted, which up to this
  time had happened to no one};
\apud{B.~G.}{2, \emend{121}{35}{34}, 4}.
Similarly with \latin{id quod}, \apud{B.~G.}{4, 29, 3}.

\latin{quod bonum, faustum, fēlīxque sit, Quirītēs, rēgem creāte},
\english{citizens,—may it be attended with good, with fortune, and
  with blessing,—appoint a king};
\apud{Liv.}{1, 17, 10}.
(The Relative refers to what is to follow.)

\end{examples}

\begin{note}[Note 1]

The word \latin{rēs} (\kern1.5pt\english{fact}, \english{circumstance}, etc.)\
may be used, in which case the pronoun must agree with it. So
\latin{quae rēs}, \apud{B.~G.}{3, \emend{157}{15, 4}{14, 13}}.

\end{note}

\begin{note}[Note 2]

There are thus three possible forms in such a case,—\latin{quod},
\latin{id quod}, and \latin{quae rēs}.

\end{note}

\subsubsection

Substantive clauses, infinitives used substantively, and quoted
expressions, are neuter.  Examples in~\xref[3]{58}.

\subsubsection

A Neuter Adjective used substantively may be a predicate to a subject
of any Gender.
\begin{examples}

\latin{mūtābile semper fēmina},
\english{a woman is always a fickle thing};
\apud{Aen.}{4, 569}.

\end{examples}

\subsubsection

With similar feeling, the Romans liked to use the neuter \emph{in
  general expressions}, in place of the masculine or feminine.  Thus
\latin{mihi tē cārius nihil esse}, \english{\emph{\(be sure\)} that
  nothing \emph{\(= no one\)} is dearer to me than yourself};
\apud{Fam.}{14, 3, 5}; \latin{quicquid invalidum est},
\english{whatever \emph{\(= whoever\)} is weak}; \apud{Aen.}{5, 716}.

\section

In \term{Agreement by Attraction}, a word takes its Gender and Number
from some word closely connected with the one to which it really
belongs.  Thus:

\subsection

An Adjective, Participle, or Pronoun may be attracted into the Gender
and Number of an Appositive or Predicate.
\begin{examples}

\latin{Corinthum patrēs vestrī, tōtīus Graeciae lūmen, exstīnctum esse
  voluērunt},
\english{your ancestors chose that Corinth, the light of the whole
  Greek world, should be extinguished};
\apud{Pomp.}{5, 11}.
(\latin{Exstīnctum} is attracted by \latin{lūmen}.)

\latin{idem velle atque nōlle, ea amīcitia est},
\english{to have the same desire and the same aversion, that is
  friendship};
\apud{Sall.\ Cat.}{20, 4}.
(\latin{Ea} is attracted by \latin{amīcitia}.)

(Relative)
\latin{omnīs Belgās, quam tertiam esse Galliae partem dīxerāmus,
  coniūrāre},
\english{that all the Belgians, who \(which\) we have said are a third
  part of Gaul, were conspiring};
\apud{B.~G.}{2, 1, 1}.

\end{examples}

\subsection

For Attraction of a Predicate into the Dative after \latin{licet
  esse}, etc., \english{it is permitted \(to a man\) to be\dots}, see
\xref[\emph{c}]{585}.

\subsection

A word denoting a Name \emph{may} be attracted by a Dative depending
upon \latin{nōmen est} (\xref{374}), \latin{nōmen dō} (\xref{365}),
etc.
\begin{examples}

\latin{nōmen Arctūrō est mihi},
\english{my name is Arcturus};
\apud{Rud.}{5}.

\end{examples}

\begin{minor}

\subsubsection

Otherwise the Appositive construction is regularly used with
\latin{nōmen est} (\emph{not} the Explanatory Genitive); thus
\latin{Troia huic locō nōmen est}, \apud{Liv.}{1, 1, 5}.

\end{minor}

\subsection

Rarely, the Relative is attracted into the \emph{Case} of its
Antecedent.
\begin{examples}

\latin{quibus quisque poterat ēlātīs},
\english{picking up what each could}
(= \latin{iīs ēlātīs quibus quis\-que po\-te\-rat},
in place of \latin{iīs ēlātīs quae}, etc.);
\apud{Liv.}{1, 29, 4}.

\end{examples}

\subsection

In poetry, the Noun is sometimes put before the Relative and attracted
into its \emph{Case}.
\begin{examples}

\latin{urbem quam statuō vestra est},
\english{the city which I build, ’t is yours};
\apud{Aen.}{1, 573}.

\end{examples}

\section

\versionA{The Romans avoided putting an Appositive word directly
  before a Relative, preferring to attach it \emph{to the Relative
    itself}.}

\versionB*{The Romans avoided making a Relative refer to an Appositive
  Noun, preferring to attach the latter \emph{to the Relative
    itself}.}

\begin{examples}

\latin{tanta tranquillitās exstitit, ut sē ex locō movēre nōn possent;
  quae quidem rēs maximē fuit opportūna},
\english{so great a calm arose that they could not stir from the
  place; a circumstance which \emph{\(which circumstance\)} was most
  fortunate};
\apud{B.~G.}{3, \emend{158}{\hbox{15, 3}}{\hbox{14, 12}}}.

\end{examples}

\pagebreak

\chapter{Details of Agreement for Verbs}

\headingE{Agreement with a Single Subject}

\section
\subsection

A Finite Verb (\xref{146}) agrees with its Subject in \emph{Number}
and \emph{Person}.
\begin{examples}

\latin{relinquēbātur ūna via},
\english{one road remained};
\apud{B.~G.}{1, 9, 1}.

\latin{erant itinera duo},
\english{there were two ways};
\apud{B.~G.}{1, 6, 1}.

\end{examples}

\begin{minor}

\subsubsection

When the subject is a Relative, the verb follows the Person of the
Antecedent.
\begin{examples}

\latin{adsum quī fēcī},
\english{here am I, who did it};
\apud{Aen.}{9, 427}.

\end{examples}

\end{minor}

\subsection

If a verb-form contains a Participle, this Participle must agree with
the Subject in \emph{Case}, in \emph{Gender}, and in \emph{Number}.
\begin{examples}

\latin{ea rēs est ēnūntiāta},
\english{the affair was made known};
\apud{B.~G.}{1, 4, 1}.

\latin{ita Helvētiōs īnstitūtōs esse},
\english{\emph{\(answered\)} that the Helvetians has been so trained};
\apud{B.~G.}{1, 14, 7}.

\end{examples}

\headingE{Agreement with Two or More Subjects}

\section

A Verb may have two or more words for its Subject, and these may be of
different Persons, Genders, or Numbers.  The usage in such cases is as
follows:

\subsection

Where the Subjects are of different persons, the First Person is
preferred to the other two, and the Second Person to the Third.
\begin{examples}

\latin{sī tū et Tullia valētis, ego et suāvissimus Cicerō valēmus},
\english{if you and Tullia are well, my dear boy and I are well};
\apud{Fam.}{14, 5, 1}.

\end{examples}

\subsection

When a Verb belongs to two or more words, it may either agree with the
nearest of them, or be put in the Plural.
\begin{examples}

\latin{Orgetorīgis fīlia atque ūnus ē fīliīs captus est},
\english{the daughter of Orgetorix and one of his sons were taken
  prisoners};
\apud{B.~G.}{1, 26, 4}.

\latin{ubi Titurius atque Aurunculeius cōnsēderant},
\english{where Titurius and Aurunculeius had established themselves};
\apud{B.~G.}{6, 32, 4}.

\end{examples}

\subsection

When a Verb belongs to several Subjects connected by \latin{aut},
\latin{aut\ellipsis aut\dots}, or \latin{nec\ellipsis nec\dots}, it may be in
either the Singular or the Plural.
\begin{examples}

\latin{neque pēs neque mēns suom officium facit},
\english{neither foot nor mind does its duty};
\apud{Eun.}{\emend{17}{729}{728}}.

\latin{haec sī neque ego neque tū fēcimus},
\english{if neither you nor I did it};
\apud{Ad.}{\emend{18}{103}{104}}.

\end{examples}

\pagebreak

\headingG{Agreement of Verbs by Form, by Sense, and by Attraction}

\section

In \term{Agreement by Form}, a Verb takes its Number from the
\emph{form} of the word to which it belongs.
\begin{examples}

\latin{pars stupet dōnum},
\english{a part \emph{\(is\)} are amazed at the gift};
\apud{Aen.}{2, 31}.

\end{examples}

% \pagebreak

\section

In \term{Agreement by Sense}, a Verb takes its Number from the
\emph{real meaning}, not the \emph{form}, of its Subject.  This takes
place as follows:

\subsection

A Verb agreeing with a \emph{Collective Noun} may be in the Plural.
\begin{examples}

\latin{pars mōlem mīrantur},
\english{a part admire the mighty bulk};
\apud{Aen.}{2, 31}.

\latin{Cīvitātī persuāsit ut exīrent},
\english{persuaded the state to go out};
\apud{B.~G.}{1, 2, 1}.

\end{examples}

\subsection

A Verb agreeing with \latin{quisque}, \latin{uterque}, etc., may be in
the Plural.
\begin{examples}

\latin{uterque eōrum exercitum ēdūcunt},
\english{each of them leads out his army};
\apud{B.~C.}{3, 30, 3}.

\end{examples}

\begin{minor}

\subsubsection

For the more common Partitive Apposition, see~\xref[I, \emph{a}]{391}.

\end{minor}

\subsection

A Verb agreeing with two or more Subjects which make \emph{one
  compound idea} may be in the Singular.
\begin{examples}

\latin{ratiō ōrdōque agminis aliter sē habēbat},
\english{the plan and arrangement of the line of march was different};
\apud{B.~G.}{2, \emend{159}{19}{18}, 1}.

\end{examples}

\subsection

A Verb agreeing with a Subject attached to another word by \latin{cum}
may be in the Plural.
\begin{examples}

\latin{Lentulus, cum cēterīs quī prīncipēs coniūrātiōnis erant,
  cōnstituerant\dots},
\english{Lentulus, with the other leaders of the conspiracy, had
  determined\dots};
\apud{Sall.\ Cat.}{43, 1}.

\end{examples}

\section

In \term{Agreement by Attraction}, a Verb may take its Number, not
from the Subject, but from an Appositive or Predicate which stands
\emph{between} it and the Subject.
\begin{examples}

\latin{pictōrēs suum quisque opus ā vulgō cōnsīderārī vult},
\english{painters want each his own work to be examined by the
  public};
\apud{Off.}{1, 41, 147}.

\latin{amantium īrae amōris integrātiō est},
\english{lovers’ tiffs are love’s renewal};
\apud{And.}{555}.

\end{examples}

\headingB{Leading Idea Not in the Principal Noun}

\section

The leading idea of a phrase may be carried, not by the grammatically
leading Noun, but by an Adjective, Participle, Pronoun, or Noun
\emph{in agreement with} it.  (See also~\xref{608}.)
\begin{examples}

\latin{post urbem conditam},
\english{after the founding of the city};
\apud{Cat.}{4, 7, 14}.

\latin{ante Verrem praetōrem},
\english{before the praetorship of Verres};
\apud{Verr.}{3, 6, 15}.

\latin{duce laetus Achātē},
\english{rejoicing in the guidance of Achates};
\apud{Aen.}{1, 696}.

\end{examples}

\subsubsection

The usage is common in Cicero, but still more frequent later.

\pagebreak

\headingB{Remaining Uses of the Cases}

\contentsentry{A}{Remaining Uses of the Cases}

\begin{minor}

\section[\textsc{\small General Introduction}]
\subsection

The earliest ideas expressed by the cases (as these are represented in
Latin) were probably as follows:
\begin{Tabular}{c@{ }c@{ }l@{ }>{\itshape}l}

By& the &
Nominitive, & the Name. \\

\ditto&\ditto &
Genitive, & that which Possesses; \emph{or} a Whole, of which a Part
only is affected.\footnote{The idea of Possession was perhaps the
  older; for the Part \emph{belongs to} the Whole.  Thus \latin{multī
    Rōmānōrum}, \english{many belonging to \emph{\(= of\)} the
    Romans}.}\\

\ditto&\ditto &
Dative, & Direction. \\

\ditto&\ditto &
Accusative,   & Contact \emph{or} Nearness. \\

\ditto&\ditto &
Vocative,     & Address. \\

\ditto&\ditto &
Ablative,     & \kern -10pt\groupL{Separation. \\
                        Assocation.\\
                        Location.}
\end{Tabular}

\subsection
The Ablative is made up (\xref[\emph{b}]{61}) of remains of three
cases possessed by the parent speech: I.~the true Ablative, expressing
Separation, II.~the Sociative (generally called, from a derived use,
the Instrumental), expressing Association (i.e.\ Accompaniment), and
III.~the Locative, expressing the Place Where.

\enlargethispage{\baselineskip}

\subsection
It is obvious that these three cases of the parent speech originally
expressed, or involved, \emph{space}-ideas: the Ablative that of
motion \emph{from} some place, the Locative that of being \emph{in}
some place, the Sociative that of being \emph{with} something
(necessarily \emph{in} some place).  The two other common and striking
space-ideas, namely that of Direction toward something, and that of
Contact or Nearness, must have been expressed by two out of the
remaining cases; and the actual uses of the Dative and the Accusative
make it probable that these were respectively the two.

\subsection

All space-ideas were originally expressed by bare cases; for
Prepositions were of comparatively late origin (see \xref{125};
\xref[\emph{a}]{303}).

\subsection

From expressions of space-relations arose a variety of figurative
expressions.  Compare English \english{\textsc{from} the camp} and
\english{\textsc{from} affection}, \english{\textsc{in} the camp} and
\english{\textsc{in} haste}.

\end{minor}

% \negbigskip

\chapter{The Nominative}

\contentsentry{C}{Uses of the Nominative}

% \negsmallskip

\headingC{Subject of a Finite Verb}

\section

The \emph{Subject of a Finite Verb} is put in the Nominative.
\begin{examples}

\latin{hic tamen vīvit},
\english{still this mans lives};
\apud{Cat.}{1, 1, 2}.

\latin{interfectus est C.\ Gracchus},
\english{Gaius Gracchus was killed};
\apud{Cat.}{1, 2, 4}.

\end{examples}

\subsubsection

The Subject is sometimes a Substantive Clause or an Infinitive
(\xref{238}, \xref[1, \emph{a}]{597}).

\subsubsection

A Nominative is frequently used without a Verb, to present a person or
thing simply as doing, suffering, or being, without telling
\emph{what} he or it does, suffers, or is.
\begin{examples}

\latin{ēn Priamus},
\english{lo and behold, Priam};
\apud{Aen.}{1, 461}.

\latin{clāmor inde concursusque populī},
\english{then a shouting and a rushing together of the people};
\apud{Liv.}{1, 41, 1}.

\end{examples}

\subsubsection

The \emph{Subject of the Historical Infinitive} is likewise put in the
Nominative.  (Examples under~\xref{595}.)

\section

The Nominative is also used:
\begin{enum1}

\item

As an Appositive.  See \xref[2]{317}, and \xref{319}.

\item

As a Predicate.  See \xref[3]{317}, and \xref{319}.

\item

In Exclamations.  See \xref[\emph{a}]{399}.

\item

In place of the Vocative.  See \xref{401}.

\end{enum1}

% \negmedskip

\chapter{The Genitive}

\contentsentry{C}{Uses of the Genitive}

\section

The Latin Genitive expresses three general classes of ideas:
\begin{enumI}[III]

\item
\emph{Possession.}

\item
\emph{The Whole, of Which a Part is affected.}

\item
\emph{Various ideas, in constructions of Composite Origin \(Fusion\).}

\end{enumI}

\section
\subtitle{\textsc{Synopsis of the Principal Uses of the Genitive}}

\begin{synopsis}

\a{I}{Possesive Genitive}

\b{\xref{339}}{Genitive of Possession or Connection, directly attached}

\b{}{Possessive Genitive in Predicate (Genitive of
  Possession, Duty, Mark, etc.; \xref{340})}

\b{}{Derivatives from Genitive of Possession, directly attached:}

\c{\xref{341}}{Explanatory Genitive\footnotemark}

\d{\xref{342}}{Genitive of the Charge\footnotemark[\thefootnote]}

\e{\xref{343}}{Genitive of the Penalty or Fine\footnotemark[\thefootnote]}

\d{\xref{344}}{Subjective Genitive}

\d{\xref{345}}{Genitive with \latin{rēfert} and \latin{interest}}

\medskip

\a{II}{Genitive of the Whole}

\b{\xref{346}}{True Genitive of the Whole}

\c{\xref{347}}{Genitive of Plenty or Want}

\c{\xref{348}}{Poetic Genitive of Separation}

\b{\xref{349}}{Genitive of Material or Composition}

\b{}{Genitive of the Object, with Verbs:}

\c{\xref{350}}{with \latin{oblīvīscor}, \latin{meminī}, \latin{reminīscor}}

\c{\xref{351}}{\ditto[with]
\latin{admoneō}, \latin{commoneō}, \latin{commonefaciō}}

\c{\xref{352}}{\ditto[with]
\latin{miseret}, \latin{paenitet}, \latin{piget}, \latin{pudet},
\latin{taedet}; \latin{misereor}, \latin{miserēscō}}

\c{\xref{353}}{\ditto[with] \latin{potior}}

\medskip

\a{III}{Of Composite Origin}

\b{\xref{354}}{Objective Genitive and Genitive of Application}

\b{\xref{355}}{Descriptive Genitive}

\c{\xref{356}}{Genitive of Value or Price}

\b{\xref{357}}{Genitive with Neuter Plural Adjectives}

\end{synopsis}
\footnotetext{In this table and those that follow, the setting back of
  a construction from the line means that it is derived from the
  \emph{first} construction above standing\,\emph{farther to the left}.
  Thus (under~I) from the Possessive Genitive is derived the
  Explanatory Genitive; from the latter, the Genitive of the Charge;
  and from the last, the Genitive of the Penalty.}

% \vskip-\bigskipamount

\headingE{The Possessive Genitive and its Derivatives}

\vskip-\bigskipamount

\headingC{Possesive Genitive in Direct Attachment}

\vskip-\bigskipamount

\section

\emph{Possession} or \emph{Connection} may be expressed by a Genitive
attached to a Noun.
\begin{examples}

\latin{servō accūsātōris},
\english{a slave belonging to \emph{(of)} the accuser};
\apud{Mil.}{22, 59}.

\latin{difficultātēs bellī},
\english{difficulties connected with the war};
\apud{Leg.\ Agr.}{2, 30, 83}.

\end{examples}

\subsubsection

As in English, the possessive pronoun of the first or second person or
of the reflexive is regularly preferred to the Genitive of the
personal pronoun; similarly, \latin{aliēnus} to the Genitive of
\latin{alius}.
\begin{examples}

\latin{meum fīlium},
\english{my son};
\apud{Cat.}{4, 11, 23}.

\latin{aliēnīs praecēptīs},
\english{the teachings of others};
\apud{Pomp.}{10, 28}.

\end{examples}

\subsubsection

When used with a possessive pronoun, \latin{ipse}, \latin{sōlus},
\latin{ūnus}, \latin{omnis}, and sometimes other words, agree with the
implied Genitive.
\begin{examples}

\latin{nostrō omnium flētū},
\english{the tears of us all};
\apud{Mil.}{34, 92}.

\latin{vestrae ipsōrum virtūti},
\english{your own valor};
\apud{Liv.}{1, 28, 4}.

\latin{tuum studium adulēscentis},
\english{your zeal as a youth};
\apud{Fam.}{15, 13, 1}.

\latin{aedem Nymphārum},
\english{the temple of the Nymphs};
\apud{Mil.}{27, 73}.

\latin{cuius pater},
\english{whose father} (the father of whom);
\apud{B.~G.}{1, 3, \emend{160}{4}{3}}.

\latin{amīcōs populī Rōmānī},
\english{friends of the Roman people};
\apud{B.~G.}{1, 35, 4}.

\end{examples}

\subsubsection

Certain adjectives meaning \emph{like}, \emph{common},
\emph{connected}, or the opposite, may take either the Dative of
Relation (\xref{362}) or the Genitive of Possession or
Connection:\footnote{So especially \latin{similis}, \latin{pār},
  \latin{commūnis}, \latin{adfīnis}, and their opposites
  \latin{dissimilis}, \latin{contrārius}, \latin{aliēnus},
  \latin{proprius}.  Also \latin{superstes}, \english{surviving}
  (\english{left over with relation to}, or \emph{the survivor of}).}
\begin{examples}

\latin{tuī smilis},
\english{like you} (the like of you);
\apud{Cat.}{1, 2, 5}.

\latin{superstes omnium meōrum},
\english{the survivor of all my people};
\apud{Quintil.}{6, Pr.~4}.

\latin{aliēnum dignitātis},
\english{inconsistent with dignity};
\apud{Fin.}{1, 4, 11}.

\end{examples}

\begin{note}

With words denoting persons, \latin{similis} more frequently takes the
Genitive.

\end{note}

\begin{minor}

\subsubsection

The idea of Possession or Connection may be lost, though the Genitive
remains.  Thus with \latin{īnstar}, \latin{causā}, \latin{grātiā}, and
\latin{ergō} (the last three post-positive).
\begin{examples}

\latin{īnstar montis equum},
\english{a horse \emph{(the like of)} like a mountain};
\apud{Aen.}{2, 15}.

\latin{amīcitiae causā},
\english{by reason of their friendship};
\apud{B.~G.}{1, 39, 2}.

\latin{illius ergō},
\english{on his account} (on account of him);
\apud{Aen.}{6, 670}.

\end{examples}

\subsubsection

In a few expressions, the noun on which the Genitive depends may be
omitted (so regularly with the master’s name).  Thus \latin{ad
  Castoris}, \emph{to \(the temple\) of Castor}; \apud{Mil.}{33, 91};
\latin{Hectoris Andromachē}, \english{Hector’s \(wife\) Andromache};
\apud{Aen.}{3, 319}.

\subsubsection
For the Genitive with \latin{prīdiē} and \latin{postrīdiē}, see
\xref[\emph{c}]{380}.

\subsubsection

For the occasional Genitive with \latin{tenus}, see \xref[3]{407}.

\end{minor}

\headingC{Possessive Genitive in the Predicate}

\section

The Possessive Genitive may be used \emph{in the Predicate} with
\latin{sum} or \latin{faciō} to express the idea of \emph{belonging
  to}, or various ideas naturally suggested by this (\english{is the
  business of}, \emph{the part of}, \emph{the duty of}, etc.).
\begin{examples}

\latin{neque Galliam potius esse Ariovistī quam populī Rōmānī},
\english{and that Gaul did not belong to Ariovistus any more than to
  the Roman people};
\apud{B.~G.}{1, 45, 1}.

\latin{virī fortis (est) nē suppliciīs quidem movērī},
\english{it is the duty of a brave man not to be stirred even by
  tortures};
\apud{Mil.}{30, 82}.

\end{examples}

\begin{minor}

\subsubsection

In certain phrases, the Idea of Possession is faint or wholly lost.
\begin{examples}

\latin{nihil reliquī fēcērunt},
\english{they left nothing undone}
(made nothing to belong to the left undone);
\apud{B.~G.}{2, \emend{161}{26}{25}, 5}.

\end{examples}

\subsubsection

For the Dative of Possession with the verb \latin{sum}, see
\xref{374}.

\end{minor}

\headingC{Explanatory Genitive}

\section

The Genitive may be attached to a Noun to \emph{define} or
\emph{explain} its meaning.
\begin{examples}

\latin{hoc poētae nōmen},
\english{this name of “poet”};
\apud{Arch.}{8, 19}.

\latin{Troiae urbem},
\english{the city of Troy};
\apud{Aen.}{1, 565}.

\end{examples}

\headingC{Genitive of the Charge}

\section

Verbs of \emph{accusing}, \emph{condemning}, or
\emph{acquitting}\footnote{So especially \latin{accūsō} and
  \latin{incūsō}, \latin{arcessō}, \latin{arguō}, \latin{dēferō},
  \latin{postulō}, \latin{damnō}, \latin{condemnō}, \latin{convincō},
  \latin{absolvō}, \latin{līberō}.  Similarly, in poetry or later
  prose, \latin{interrogō} and the adjectives or participles
  \latin{innocēns}, \latin{īnsōns}, \latin{manifestus},
  \latin{noxius}, \latin{innoxius}, \latin{suspectus}, etc.} may take
a \emph{Genitive of the Thing Charged}.
\begin{examples}

\latin{eum accūsās avāritiae?}
\english{do you accuse him of avarice?}
\apud{Flacc.}{33, 83}.

\latin{mē inertiae condemnō},
\english{I condemn myself for negligence};
\apud{Cat.}{1, 2, 4}.

\end{examples}

\subsubsection

Similarly \latin{reus}, \english{defendant} (i.e.\ person accused),
may take the Genitive.  Thus \latin{pe\-cū\-ni\-ā\-rum repetundārum
  reus}, \english{charged with extortion} (money to be recovered);
\apud{Sall.\ Cat.}{18, 3}.

\begin{minor}

\subsubsection

The Thing Charged may also be expressed by \latin{dē} with the
Ablative.  Thus \latin{dē vī postulāvit}, \english{arraigned him on a
  charge of violence}; \apud{Senat.}{8, 19}.

\subsubsection

By a different turn of the thought, \latin{inter} may be used to
denote the class in which the accused is placed.  Thus \latin{inter
  sīcāriōs accūsābant}, \english{accused him of belonging among
  cutthroats} (i.e.\ of murder); \apud{Rosc.\ Am.}{32, 90}.

\subsubsection

The Thing Charged may become the Direct Object (\xref{390}), the
Person being left unmentioned.  Thus \latin{ambitum accūsās?}
\english{do you charge bribery?} \apud{Mur.}{32, 67}.

\end{minor}

\headingC{Genitive of the Penalty or Fine}

\section

Verbs of \emph{accusing}, \emph{condemning}, or \emph{acquitting} may
take a \emph{Genitive of the Penalty} or \emph{Fine}.
\begin{examples}

\latin{octuplī damnāre},
\english{to condemn \(to pay\) eightfold};
\apud{Verr.}{3, 11, 28}.

\latin{capitis condemnārī},
\english{to be condemned to death};
\apud{Rab.\ Perd.}{4, 12}.

\latin{damnātum vōtī},
\english{successful in his vow} (condemned to pay it);
\apud{Nep.\ Timol.}{5, 3}.
With similar meaning \latin{vōtī reus}, \apud{Aen.}{5, 237}.

\end{examples}

\begin{minor}

\subsubsection

The construction is confined in prose to indefinite words like
\latin{pecūniae}, \emph{money}, and \latin{quantī}, \english{how
  much}, multiples like \latin{octuplī}, \english{eightfold}, and the
word \latin{capitis}, \english{death}.

\subsubsection

For the Ablative of the Penalty with verbs of \emph{punishing} or
\emph{fining}, see \xref{428}.

\end{minor}

\headingC{Subjective Genitive}

\section

The Genitive may be used to express the \emph{Subject of an Activity
  denoted by a Noun}.
\begin{examples}

\latin{ab iniūriā Cassivellaunī},
\english{from wrong at the hands of Cassivellaunus};
\apud{B.~G.}{5, 20, \emend{162}{3}{2}}.
(He \emph{committed} the wrong.)

\latin{Caesaris adventus},
\english{Caesar’s coming};
\apud{B.~G.}{6, 41, 4}.
(Caesar \emph{came}.)

\end{examples}

\headingC{Genitive of the Person or Thing Concerned, with
  \latin{rēfert} and \latin{interest}}

\section

\latin{Rēfert} and \latin{interest}, \english{it concerns},
\english{is for the interest of}, take the \emph{Genitive of the
  Person} or \emph{Thing Concerned}, if expressed by a Noun, the Feminine
Ablative Singular of the Possessive if expressed by a Pronoun
(\latin{meā}, \latin{tuā}, etc.).
\begin{examples}

\latin{quantum interesset P.\ Clōdī sē perīre cōgitābat},
\english{he always kept in mind how much his death concerned Publius
  Clodius};
\apud{Mil.}{21, 56}.

\latin{nihil meā refert},
\english{it does not concern me};
\apud{Pis.}{17, 39}.

\latin{meā videō quid intersit},
\english{I see what is to my interest};
\apud{Cat.}{4, 5, 9}.

\end{examples}

\begin{minor}

\subsubsection

With the Genitive of the Person Cicero prefers \latin{interest}.

\subsubsection

The \emph{degree} of the concern or interest may be expressed by an
Accusative of Degree (\xref{387}), a Genitive of Value (\xref{356}),
or an Adverb.  Thus \latin{meā interest plūrimum}, \latin{plūrimī}, or
\latin{maximē}, \english{it is greatly to my interest}.

\end{minor}

\pagebreak

\headingE{The Genitive of the Whole\footnotemark, and its
  Derivatives} \footnotetext{\label{ftn:183:}The name \emph{Partitive
    Genitive}, which is often used, is convenient because of its
  shortness.  But the student should remember that what is expressed
  by the Genitive word itself is the \emph{Whole}, not the Part.}

\headingC{Genitive of the Whole in the Strict Sense}

\section

The \emph{Whole to which a Part Belongs} may be expressed by the
Genitive.

The construction may be used with any Noun, Adjective, Pronoun, or
Adverb that can imply a \emph{part} of a whole.
\begin{examples}

\latin{eōrum ūna pars},
\english{one part of them};
\apud{B.~G.}{1, 1, 5}.

\latin{prīmōs cīvitātis},
\english{the first men of the state};
\apud{B.~G.}{2, 3, 1}.

\latin{ubinam gentium sumus?}
\english{where in the world are we?}
\apud{Cat.}{1, 4, 9}.

\latin{sceleris nihil},
\english{no crime} (nothing of crime);
\apud{Mil.}{12, 32}.

\latin{quid suī cōnsilī sit},
\english{what his plan is}; \apud{B.~G.}{1, 21, 2}.
(For \latin{quid sibi cōnsilī sit}, \english{what of plan he has}.)

\end{examples}

\subsubsection

With words like \latin{nihil} and \latin{aliquid}, adjectives of the
Second Declension may be put either in the neuter Genitive of the
Whole, or in direct agreement; while adjectives of the Third
Declension are almost always in direct agreement.
\begin{examples}

\latin{nihil certī} (\apud{Ac.}{1, 12, 46}) and \latin{nihil certum}
(\apud{Tull.}{15, 35}), \english{nothing certain}.

\latin{nihil maius},
\english{nothing greater};
\apud{Lig.}{12, \emend{163}{38}{37}}.

\end{examples}

\begin{minor}

\subsubsection

\latin{Uterque}, \english{each of two}, and \latin{quisque},
\english{each of a larger number}, regularly agree with a noun, but
take the Genitive of the Whole if a pronoun is used.
\begin{examples}

\latin{uterque dux},
\english{each general}, \english{both generals};
\apud{Marc.}{8, 24}.

\latin{quōrum utrīque},
\english{to each of whom};
\apud{Mil.}{27, 75}.

\end{examples}

\subsubsection

English often uses the word “of” where there is no partitive
relation, as in “all of us,” meaning “we all.” Latin is
\emph{generally} exact in this respect.
\begin{examples}

\latin{hī omnēs},
\english{all \(of\) these};
\apud{B.~G.}{1, 1, 2}.

\versionA*{\latin{quōs omnīs},
\english{all \(of\) whom};
\apud{Pomp.}{19, 58}.}

\versionB*{\latin{reliquīs Gallīs},
\english{the rest of the Gauls}
(the remaining Gauls);
\apud{B.~G.}{2, 2, 3}.}

\end{examples}

\subsubsection

In poetry and later prose the Genitive of the Whole is sometimes used
with words \emph{not} implying a part.
\begin{examples}

\latin{tē, sāncte deōrum},
\english{thee, O holy one of the gods};
\apud{Aen.}{4, 576}.

\latin{fīēs nōbilium tū quoque fontium},
\english{thou too shalt be of the world’s great fountains};
\apud{Carm.}{3, 13, 13}.  (In Predicate.)

\end{examples}

\subsubsection

After many words, the Whole to which a Part belongs \emph{may} be
expressed by \latin{dē} or \latin{ex} with the Ablative (\xref{405}).
So regularly with \latin{quīdam} and with cardinal numbers
(\xref{130}).  Thus \latin{ūnus ex istīs}, \english{the only one of
  these}; \apud{Cat.}{3, 7, 16}.

\end{minor}

\headingC{Genitive of Plenty or Want}

\section

Certain Adjectives and Verbs of \emph{plenty} or \emph{want} may
take the Genitive.

\begin{examples}

\latin{plēna exemplōrum vetustās},
\english{the past is full of examples};
\apud{Arch.}{6, 14}.

\latin{implentur Bacchī},
\english{they take their fill of wine};
\apud{Aen.}{1, 215}.

\latin{inopēs amīcōrum},
\english{poor in friends};
\apud{Am.}{15, 53}.

\latin{nē quis auxilī egēret},
\english{that none might be in need of aid};
\apud{B.~G.}{6, 11, 4}.

\end{examples}

\subsubsection

So, in Ciceronian Latin, the adjectives \latin{plēnus},
\latin{refertus}, \latin{expers}, \latin{inops}, \latin{inānis}, and
the verbs \latin{indigeō}, \latin{egeō}, \latin{compleō},
\latin{impleō} (the last three rarely).\footnote{Also, in later Latin
  (often with forced meaning), \latin{dīves}, \latin{egēnus},
  \latin{laetus}, and many others; and the verbs \latin{repleō},
  \latin{careō}, and others.}

\subsubsection

The words of this list also take the Ablative (\xref{425}) freely in
Ciceronian Latin, except \latin{plēnus}, \latin{inops},
\latin{indigeō} (these three rarely), and \latin{expers} (never).

\begin{minor}

\subsubsection

Other words of Plenty or Want take the Ablative in Ciceronian Latin
(\xref{425}).

\end{minor}

\headingC{Poetic Genitive of Seperation}

\section

In poetry the Genitive is sometimes used to express \emph{Separation}.
\begin{examples}

\latin{ut mē labōrum levās!}
\english{how you relieve me of toil!}
\apud{Rud.}{247}.

\latin{līber labōrum},
\english{free from toil};
\apud{A.~P.}{212}.
(Cf.\ \latin{līberī ā dēliciīs}; \apud{Leg.\ Agr.}{1, 9, 27}.)

\latin{dēsine querellārum},
\english{cease from complaints};
\apud{Carm.}{2, 9, 17}.

\latin{neque ciceris invīdit},
\english{nor grudged his chick-pea};
\apud{Sat.}{2, 6, 83}.

\end{examples}

\begin{note}[Remark]

This construction is an extension of the Genitive of Want; but the
extension was doubtless \emph{helped} by the influence of the Greek
Genitive of Separation.

\end{note}

\headingC{Genitive of Material or Composition}

\section

\emph{Material} or \emph{Composition} may be expressed by a Genitive attached
to a Noun.
\begin{examples}

\latin{obtortī circulus aurī},
\english{a chain of twisted gold};
\apud{Aen.}{5, 559}.

\latin{ancillārum gregēs},
\english{crowds \emph{(composed)} of maidservants};
\apud{Mil.}{21, 55}.

\end{examples}

\begin{minor}

\subsubsection

The same idea \emph{may} be expressed by the Ablative with \latin{ex}
(in poetry with \latin{dē} also, or without preposition), and
\emph{must} be so expressed if a verb is used (\xref[4]{406}).
\begin{examples}

\latin{factae ex rōbore},
\english{made of oak};
\apud{B.~G.}{3, 13, 3}.

\end{examples}

\end{minor}

\headingG{Genitive of the Object, with Verbs}

\headingC{Genitive of the Object of Mental Action}

\section

\latin{Oblīvīscor}, \latin{meminī}, and \latin{reminīscor},
\emph{forget}, \emph{remember}, and \emph{recall}, may take a
\emph{Genitive Object}.

If the Object is a \emph{person}, \latin{oblīvīscor} takes the
Genitive only, \latin{meminī} \emph{either} the Genitive or the
Accusative, \latin{reminīscor} the Accusative only.

If the Object is a \emph{thing}, all three verbs take \emph{either}
the Genitive \emph{or} the Accusative of a Noun, and (regularly) only
the Accusative of a Neuter Pronoun or Adjective.
\begin{examples}

\latin{vīvōrum meminī, nec tamen Epicūrī licet oblīvīscī},
\english{I remember the living, and, at the same time, it isn’t
  possible for me to forget Epicurus};
\apud{Fin.}{5, 1, 3}.

\latin{nec umquam oblīvīscar noctis illīus},
\english{nor shall I ever forget that night};
\apud{Planc.}{42, 101}.
Cf.\ \latin{reminīscerētur virtūtis},
\apud{B.~G.}{1, 13, 4}

\latin{an vērō oblītī estis sermōnes et opīniōnēs?}
\english{have you forgotten the expressions of opinion?}
\apud{Mil.}{23, 62}.

\latin{sī id meminerītis, quod oblīvīscī nōn potestis},
\english{if you bear in mind this fact, which you cannot forget};
\apud{Mil.}{4, 11}.

\end{examples}

\begin{minor}

\subsubsection

\latin{Meminī} may also take \latin{dē} of \emph{person}
(\emph{remember about}).

\subsubsection

\latin{Recordor}, \english{recollect}, takes \latin{dē} of a
\emph{person}, and either \latin{dē} or the Accusative of a
\emph{thing}.

\end{minor}

\section

\latin{Admoneō} and \latin{commoneō}, \english{remind}, and
\latin{commonefaciō}, \english{remind} or \english{inform}, may take,
besides an Accusative of the Person, a Genitive of \emph{the Thing of
  Which} he is reminded or informed.
\begin{examples}

\latin{admonēbat alium egestātis, alium cupiditātis suae},
\english{he would remind one man of his poverty, another of his
  covetousness};
\apud{Sall.\ Cat.}{21, 4}.

\latin{grammaticōs officiī suī commonēmus},
\english{we remind the professors of languages of their duty};
\apud{Quintil.}{1, 5, 7}.

\end{examples}

\begin{minor}

\subsubsection

The Thing of Which one is reminded or informed, if expressed by a
neuter pronoun or a neuter adjective, is regularly in the Accusative.
(See \xref{397}.)

\subsubsection

These verbs of Reminding and Informing \emph{may} take \latin{dē} with
the Ablative.

\end{minor}

\headingC{Genitive of the Object of Feeling}

\section
\subsection

Impersonal Verbs of Feeling may take, besides the Accusative of the
Person Feeling, a Genitive of that \emph{toward which the feeling is
  directed}.

These Verbs are \latin{miseret}, \latin{paenitet}, \latin{piget},
\latin{pudet}, and \latin{taedet}, \english{it makes one pitiful,
  repentant, disgusted, ashamed, \emph{or} bored}.
\begin{examples}

\latin{mē meōrum factōrum numquam paenitēbit},
\english{I shall never repent of what I have done};
\apud{Cat.}{4, 10, 20}.
(Cf.\ “It repenteth me,” \apud{\emph{Genesis}}{, VI, 7}.)

\latin{eōrum nōs miseret},
\english{we feel pity for them};
\apud{Mil.}{34, 92}.

\end{examples}

\subsubsection

\latin{Miseret} never has a Subject. The other verbs of the list
sometimes have for a Subject a Neuter Pronoun in the Singular, an
Infinitive, or a \latin{quod}-Clause (\xref{552}).
\begin{examples}

\latin{taedet caelī convexa tuērī},
\english{it wearies her to gaze upon the vault of Heaven};
\apud{Aen.}{4, 451}.

\end{examples}

\subsection

The \emph{personal} Verbs of Feeling \latin{misereor} and the poetic
\latin{miserēscō}, \english{I pity}, take their Object in the
Genitive.
\versionB*{(\latin{Miseror} takes the Accusative.)}
\begin{examples}

\latin{miserēre animī nōn digna ferentis},
\english{pity a soul that bears ill undeserved};
\apud{Aen.}{2, 144}.

\end{examples}

\begin{minor}

\subsection

The old Genitive of the Object of Feeling is also found in poetry with
the personal verbs \latin{cupiō}, \latin{fastīdiō}, \latin{mīror},
\latin{studeō}, and \latin{vereor}.
\begin{examples}

\latin{cupiunt tuī},
\english{long for you};
\apud{Mil.\ Gl.}{963}.

\latin{iūstitiaene mīrer?}
\english{should I admire your justice?}
\apud{Aen.}{11, 126}.

\end{examples}

\end{minor}

\headingC{Genitive with \latin{potior}}

\section

The Genitive is sometimes used with \latin{potior}, \english{become
  master of}, \english{gain}.
\begin{examples}

\latin{tōtīus Galliae sēsē potīrī posse spērant},
\english{they hope to be able to master the whole of Gaul};
\apud{B.~G.}{1, 3, \emend{93}{8}{7}}.

\latin{urbis potīrī},
\english{to gain possession of the city};
\apud{Sall.\ Cat.}{47, 2}.

\end{examples}

\begin{examples}

\subsubsection

For the regular Ablative, see \xref{429}; for the occasional
Accusative, \xref[\emph{b}]{429}.

\end{examples}

\headingE{Genitive Constructions of Composite Origin (Fusion)}

\headingC{Objective Genitive and Genitive of Application}

\section

The Genitive may be used to express the \emph{Object} or the
\emph{Application} of a Noun, an Adjective, or a Participle used
adjectively.

The list of nouns is very large.  The adjectives are especially those
denoting \emph{desire}, \emph{knowledge}, \emph{skill}, \emph{memory},
or \emph{participation}.\footnote{So especially \latin{avidus},
  \latin{cōnscius}, \latin{cōnsors}, \latin{cupidus}, \latin{exhērēs},
  \latin{ignārus}, \latin{immūnis}, \latin{īnscius}, \latin{īnsolēns},
  \latin{īnsuētus}, \latin{memor}, \latin{immemor}, \latin{particeps},
  \latin{perītus}, \latin{imperītus}, \latin{rudis},
  \latin{studiōsus}.  Also \latin{expers}, when meaning \english{not
    sharing}, \english{without knowledge of}, and \latin{cōnsultus}
  in \latin{iūris cōnsultus}.

  \latin{Rudis}, \latin{īnsolēns}, and \latin{īnsuētus} differ but
  little in meaning from \latin{īnscius} and \latin{imperītus},
  \emph{and therefore followed them in taking the Genitive}; similarly
  \latin{cōnsultus} followed \latin{studiōsus} and \latin{perītus}.
  But the \emph{feeling} of the Genitive necessarily changed somewhat
  \emph{to fit the meanings of the new group}, becoming that of
  \emph{Application}.}
\begin{examples}

\latin{rēgnī cupiditāte},
\english{by desire of sovereignty};
\apud{B.~G.}{1, 2, 1}.

\latin{cupidum rērum novārum},
\english{desirous of a revolution};
\apud{B.~G.}{1, 18, 3}.

\latin{cōnscius iniūriae},
\english{conscious of wrong-doing};
\apud{B.~G.}{1, 14, 2}.

\latin{amantissimōs reī pūblicae virōs},
\english{firm friends of the state};
\apud{Cat.}{3, 2, 5}.

\latin{reī pūblicae iniūriam},
\english{the wrong done to the state};
\apud{B.~G.}{1, 20, 5}.

\latin{excessū vītae},
\english{by departure from life};
\apud{Tusc.}{1, 12, \emend{19}{27}{26}}.

\latin{cui summam omnium rērum fidem habēbat},
\english{in whom he had the greatest confidence in all matters};
\apud{B.~G.}{1, 19, 3}.

\latin{praestantiam virtūtis},
\english{preëminence in virtue};
\apud{Am.}{19, 70}.

\end{examples}

\subsubsection

Instead of the Objective Genitive depending on a noun, prepositions
with the Accusative are often employed, especially \latin{ergā},
\latin{in}, and \latin{adversus}, \english{toward}, \english{against}.
\begin{examples}

\latin{in hominēs iniūriam},
\english{wrong to men};
\apud{N.~D.}{3, 34, 84}.
(Cf.\ \latin{reī pūblicae iniūriam}, above.)

\latin{deōrum summō ergā vōs amōre},
\english{by Heaven’s great love toward you};
\apud{Cat.}{3, 1, 1}.

\end{examples}

\subsubsection

In Ciceronian Latin, only a moderate number of adjectives, mostly
expressing or suggesting \emph{Activity}, take this Genitive.  With
nouns it is more freely used.

\begin{minor}

\subsubsection

\textbf{Freer poetic and later Genitive of the Object or of
  Application.}  In poetry and later Latin this Genitive is used with
greater freedom.\footnote{Thus, with \emph{Objective} feeling, with
  \latin{certus}, \latin{exsors}, \latin{līberālis}, \latin{potēns},
  \latin{praescius}, \latin{profūsus}, \latin{sēcūrus}, \latin{tenāx}.
  The list with the feeling of \emph{Application} is very large.}
\begin{examples}

\latin{fessī rērum},
\english{weary of trouble};
\apud{Aen.}{1, 178}.

\latin{integer vītae},
\english{upright of life};
\apud{Carm.}{1, 22, 1}.

\versionA*{\latin{poenae sēcūrus},
\english{safe from punishment};
\apud{Ep.}{2, 2, 17}.}

\latin{indignus avōrum},
\english{unworthy of my ancestors};
\apud{Aen.}{12, 649}.

\versionA*{\latin{ēreptae virginis īrā},
\english{wrath at the loss of the maiden};
\apud{Aen.}{2, 413}.}

\end{examples}

\versionB*{%
\subsubsection

Adjectives and possessive pronouns are sometimes used with objective
force.
\begin{examples}

\latin{metus hostīlis},
\english{fear of the enemy};
\apud{Sall.\ Iug.}{41, 2}.

\end{examples}}

\end{minor}

\headingC{Descriptive Genitive}

\section

\emph{Kind} or \emph{Measure} may be expressed by the Genitive of a
Noun accompanied by a modifier.

The construction may be either appositive or predicative.
\begin{examples}

\latin{Catō, adulēscēns nūllīus cōnsilī},
\english{Cato, a young man of no judgment};
\apud{Q.~Fr.}{1, 2, 5, 15}.

\latin{Quīntus Lūcānius, eiusdem ōrdinis},
\english{Quintus Lucanius, of the same rank};
\apud{B.~G.}{5, 35, 7}.

\latin{hominēs magnae virtūtis},
\english{men of great courage};
\apud{B.~G.}{2, \emend{164}{15}{14}, 5}.

\latin{eius modī tempestātēs},
\english{storms of such a kind};
\apud{B.~G.}{3, \emend{86}{29}{27}, 2}.

\latin{māteria cuiusque generis},
\english{timber of every kind};
\apud{B.~G.}{5, 12, 5}.

\latin{diērum vīgintī supplicātiō},
\english{a thanksgiving of twenty days};
\apud{B.~G.}{4, 38, 5}.

\latin{meam erus esse operam dēputat parvī pretī},
\english{my master considers my services to be of small value};
\apud{Hec.}{\emend{20}{799}{801}}.

\end{examples}

\subsubsection

Compounds equivalent to a noun \emph{plus} an adjective, and nouns not
used with serious meaning (e.g.\ \latin{nihilī}, \english{zero},
\english{naught}, \latin{naucī}, \english{a peascod}), take no
modifier.
\begin{examples}

\latin{trīduī (= trium diērum) mora},
\english{a delay of three days};
\apud{B.~G.}{4, 11, 4}.

\latin{homō nihilī},
\english{man of naught};
\apud{Trin.}{1017}
(= \latin{vir minumī pretī}, \apud{Trin.}{925}).

\end{examples}

\begin{minor}

\subsubsection

In Ciceronian Latin this Genitive is generally attached to a
\emph{class}-name in apposition with the name of the person (as in the
first example above).  In later Latin it is more freely attached to
the name of the person (as in the second example above).

\subsubsection

For the Descriptive Ablative, see \xref{443}.

\end{minor}

\headingC{Genitive of Value or Price}

\section

\emph{Indefinite Value} or \emph{Price}\footnote{The principal verbs
  with which the construction is used are \latin{est}, \latin{aestimō}
  and \latin{exīstimō}, \latin{putō}, \latin{habeō}, \latin{dūcō},
  \latin{faciō}, \latin{pendō}, \latin{emō}, \latin{redimō},
  \latin{vēndō}, \latin{vēneō}.

  \latin{Aestimō} with this construction is rare before Cicero;
  \latin{exīstimō} is always rare with it.} may be expressed by the
Genitive of:

\subsection

Certain Adjectives, especially \latin{tantī}, \latin{quantī},
\latin{magnī}, \latin{parvī}; \latin{plūris}, \latin{minōris};
\latin{plūrimī}, \latin{maximī}, \latin{minimī}.

\subsection

Certain Substantives \emph{not used with serious meaning}, especially
\latin{nihilī}, \english{zero}, \latin{naucī}, \emph{a peascod},
\latin{assis}, \english{a copper}, \latin{floccī}, \english{a straw},
\latin{pilī}, \english{a hair}, \latin{huius}, \english{that much}
(with a snap of the finger).
\begin{examples}

\latin{haec nōlī putāre parvī},
\english{don’t reckon these things of small account};
\apud{Catull.}{23, 25}.
(Cf.\ \latin{esse dēputat parvī pretī} in \xref{335}.)

\latin{nōlī spectāre quantī homō sit; parvī enim pretī est quī tam
  nihilī est},
\english{don’t consider how much the fellow is worth, for he is of
  little value who is so worthless};
\apud{Q.~Fr.}{1, 2, 4, 14}.
(Note the parallel expressions \latin{parvī pretī}, \latin{quantī},
and \latin{nihilī}.)

\latin{nōn habeō naucī Marsum argurem},
\english{I don’t care a peascod for a Marsian augur};
\apud{Div.}{1, 58, 132}.

\end{examples}

\begin{minor}

\subsubsection

For the Ablative of Price or Value, see \xref{427}.

\end{minor}

\headingC{Genitive with Neuter Plural Adjectives}

\section

In the later writers a Genitive Noun is often attached to the Neuter
Plural of an Adjective, where in Ciceronian Latin the Adjective would
agree with the Noun.
\begin{examples}

\latin{strāta viārum (= strātās viās)},
\english{the paved streets};
\apud{Aen.}{1, 422}.

\latin{angusta viārum (= angustās viās)},
\english{the narrow streets};
\apud{Aen.}{2, 332}.

\end{examples}

\chapter{The Dative}

\contentsentry{C}{Uses of the Dative}

\section

The Latin Dative expresses three general classes of
ideas:
\begin{enumI}[III]

\item
\emph{Figurative Direction \(to- \emph{or} for-Dative\).}

\item
\emph{\(Rarely\) Literal Direction \(to-Dative\).}

\item
\emph{Person or Thing after Verbs compounded with certain
  Prepositions} (Construction of Composite Origin).

\end{enumI}

\section
\subtitle{\textsc{Synopsis of the Principal Uses of the Dative}}

\begin{synopsis}

\a{I}{Dative of Figurative Direction}

\b{\xref{360}}{Dative of Tendency or Purpose}
\b{\xref{361}}{Dative of the Concrete Object for Which}
\b{\xref{362}}{Dative of Direction or Relation, with Adjectives, Verbs, Adverbs, etc.}
\b{\xref{365}}{Dative of the Indirect Object}

\b{}{Dative of Reference or Concern:}
\c{\xref{366}}{With any Verb}
\c{\xref{367}}{\emph{Versus} the Accusative}
\c{\xref{368}}{\emph{In place of} the Genitive}
\c{\xref{369}}{Freer Poetic Dative of Reference or Concern}
\c{\xref{370}}{Dative of the Person Judging}
\d{\xref[\emph{a}]{370}}{Dative of the Local Point of View}

\b{\xref{371}}{Dative with Verbs of Taking Away}
\b{\xref{372}}{Ethical Dative}
\b{\xref{373}}{Dative of the Agent}
\b{\xref{374}}{Dative of Possession}

\medskip

\a{II}{Dative of Direction in Space}

\b{\xref{375}}{Poetic Dative of Direction in Space}

\medskip

\a{III}{Of Composite Origin}

\b{\xref{376}}{Dative after Verbs compounded with certain Prepositions}

\end{synopsis}

\vskip-\bigskipamount

\headingE{Dative of Figurative Direction}

\vskip-\bigskipamount

\headingC{Dative of Tendency or Purpose\protect\footnotemark}

\vskip-\bigskipamount

\footnotetext{Compare English “it is \emph{for} men’s health to be
  temperate,” “give a thing \emph{for} a present,” “he is not
  \emph{to} my satisfaction,” etc.}

\section

The Dative of many Nouns may be used to express \emph{Tendency} or
\emph{Purpose}.\footnote{\label{ftn:190:2}The verbs mostly commonly
  used with this construction are \latin{sum}, \latin{fīō},
  \latin{dō}, \latin{dōnō}, \latin{relinquō}, \latin{mittō},
  \latin{eō}, \latin{veniō}, \latin{habeō}, \latin{dūcō},
  \latin{tribuō}, \latin{vertō}.  The nouns most commonly used are
  \latin{auxiliō} and \latin{subsidiō}, \latin{praesidiō},
  \latin{salūtī}, \latin{exitiō}, \latin{bonō}, \latin{malō} and
  \latin{dētrīmentō}, \latin{impedimentō}, \latin{onerī},
  \latin{cūrae}, \latin{dolōrī}, \latin{ōrnāmentō}, \latin{honōrī},
  \latin{probrō}, \latin{ūsuī}, \latin{cordī}, \latin{odiō},
  \latin{dōnō} and \latin{mūnerī}, \latin{crīminī}, \latin{vitiō}.
  \latin{Frūgī} (for profit), \english{useful}, as in \latin{est frūgī
    bonae}, \apud{Trin.}{321}, comes also to be used as an
  indeclinable adjective.  In early and later writers, many other
  verbs and nouns appear in this construction.}
\begin{examples}

\latin{sibi eam rem cūrae futūram},
\english{that he would take care of this matter}
(this matter would be to him for a care);
\apud{B.~G.}{1, 33, 1}.

\latin{sī haec vōx nōn nūllīs salūti fuit},
\english{if this voice of mine has been \emph{(for)} the salvation of
  a number of men}
(has tended toward);
\apud{Arch.}{1, 1}.

\latin{mūnerī mīsit},
\english{sent as a present}
(for a present);
\apud{Nep.\ Att.}{8, 6}.

\latin{auxiliō Nerviīs venīrent},
\english{were coming to assist the Nervii};
\apud{B.~G.}{2, \emend{165}{29}{28}, 1}.

\end{examples}

\subsubsection

These Datives are mostly Abstracts, and all are Singular.

\subsubsection

The Dative of Tendency or Purpose is often accompanied by a Dative of
the Person (Dative of Reference, \xref{366}), as in \latin{auxiliō
  Nerviīs}.  Hence the common name “Two Datives.”

\headingC{Dative of the Concrete Object for Which}

\section

The Dative of the \emph{Concrete Object for Which} something is
intended may be used with Phrases containing Verbs of \emph{choosing}
or \emph{appointing}, and a few others.\footnote{Thus \latin{diem dīcō
    (cōnstituō) conciliō}, \latin{conloquiō}, \latin{operī},
  \latin{pugnae}, \latin{huic reī}, etc.; \latin{locum dēligō (capiō)
    castrīs}, \latin{oppidō}, \latin{domiciliō}, etc.; also
  \latin{receptuī canō}, \english{sound for retreat}, and even
  \latin{receptuī signum}, \english{signal for retreat}; sometimes
  \latin{fundāmenta iaciō (fodiō) urbī}, \latin{dēlūbrō}, etc.\ (but
  the Genitive is more common).

  The later writers extend the list of phrases.}
\begin{examples}

\latin{castrīs locum dēlēgit},
\english{chose a place for a camp};
\apud{B.~G.}{1, 49, 1}.

\latin{diēs conloquiō dictus est},
\english{a day was set for a conference};
\apud{B.~G.}{1, 42, \emend{166}{3}{4}}.

\end{examples}

\pagebreak

\begin{minor}

\subsubsection

\textbf{Later Freer Dative of the\versionA{ Concrete} Object for Which.}
The poets and later writers use the construction of the\versionA{
  Concrete} Object\versionB*{ for Which} more boldly, even attaching it
directly to nouns.
\begin{examples}

\latin{aggeritur tumulō tellūs},
\english{earth is heaped together for a mound};
\apud{Aen.}{3, 63}.

\latin{causam lacrimīs},
\english{a cause for tears}
(tending toward tears);
\apud{Aen.}{3, 305}.
Similarly \latin{causās bellō}, \apud{Tac.\ Ann.}{2, 64}.  (In
Ciceronian Latin the Genitive, as in \latin{bellī causa},
\apud{B.~G.}{3, 7, 2}.)

\end{examples}

\end{minor}

%%* FOOTNOTE NUMBER WEIRDNESS

\headingC{Dative of Direction or Relation\footnote{The line between
    these meanings is often not sharp.}}

\section

The Dative is used to express that \emph{toward which a Quality},
\emph{Attitude}, or \emph{Relation} is directed (English “to,”
“toward,” “for”).

The construction occurs after many Adjectives, Verbs, and Adverbs, and
after certain Nouns in combination with Verbs\footnote{(\emph{a})~The total
  list, especially of adjectives and verbs, is very large.  The
  commonest meanings shared by two or more of the parts of speech are:
  \emph{Pleasing}, \emph{helpful}, \emph{advantageous};
  \emph{friendly}, \emph{favoring}, \emph{obedient}; \emph{indulgent},
  \emph{forgiving}, \emph{trustful}, \emph{yielding};
  \emph{persuasive}, \emph{commanding}, \emph{angry},
  \emph{threatening}; \emph{flattering}, \emph{envious},
  \emph{jealous}; \emph{good}, \emph{sufficient}, \emph{necessary},
  \emph{permissible}, \emph{suitable}; \emph{near}, \emph{similar},
  \emph{related}; or the opposites of any of these.

  (\emph{b}) The principal verbs or phrases with verbs\emend{21}{,}{}
  occurring with this
  construction in B.~G., Cat., Arch., Pomp., and Mil.\ are:
  \latin{appropinquō}, \latin{audiēns sum}, \latin{auxilior},
  \latin{cēdō}, \latin{cōnfīdō}, \latin{dēsum}, \latin{diffīdō},
  \latin{fidem faciō} and \latin{habeō}, \latin{crēdō},
  \emend{22}{\latin{dēsum}, }{}%
  \latin{faveō}, \latin{grātiam habeō}, \latin{ignōscō},
  \latin{imperō}, \latin{indulgeō}, \latin{īnsidior}, \latin{invideō},
  \latin{īrāscor}, \latin{libet}, \latin{licet}, \latin{medeor},
  \latin{minor}, \latin{noceō}, \latin{oboediō}, \latin{obsistō},
  \latin{obstō}, \latin{officiō}, \latin{obsum}, \latin{obtemperō},
  \latin{obtrectō}, \latin{obvius est}, \latin{obviam fīō},
  \latin{veniō}, etc., \latin{opitulor}, \latin{parcō}, \latin{pāreō},
  \latin{placeō}, \latin{praestō} (\english{am superior}),
  \latin{praestō sum}, \latin{praestōlor},
  \latin{prōsum}, \latin{resistō}, \latin{repugnō}, \latin{satis faciō},
  \latin{serviō}, \latin{studeō}, \latin{suādeō} and \latin{persuādeō},
  \latin{succēnseō}, \latin{temperō}.

  (\emph{c}) The more important remaining verbs or phrases with verbs are:
  \latin{adsentior}, \latin{adversor}, \latin{aemulor}, \latin{appāreō},
  \latin{auscultō}, \latin{bene} or \latin{male} with
  \latin{dīcō}, \latin{loquor}, or \latin{faciō}, \latin{blandior},
  \latin{convenit}, \latin{convenienter} with a verb,
  \latin{condūcit}, \latin{dictō audiēns sum},
  \latin{expedit}, \latin{fidēs est} (poetical), \latin{fīdō},
  \latin{grātificor}, \latin{grātulor}, \latin{liquet} and
  \latin{lūcet}, \latin{moderor}, \latin{mōrem gerō},
  \latin{obsequor}, \latin{sufficiō}, \latin{supplicō}, \latin{vacō}.
  To these may be added \latin{nūbō} (put on the veil for),
  \english{marry}.}:

I.\enskip After words expressing or implying the \emph{Quality} (Character,
Nature) of a Person, Thing, or Act.
\begin{examples}

\latin{mihi perniciōsius},
\english{more injurious to me};
\apud{Sat.}{2, 7, 104}.

\latin{nocēre alterī},
\english{to injure one’s neighbor} (be injurious to);
\apud{Off.}{3, 5, 23}.

\latin{mihi suāvissimum},
\english{very acceptable to me};
\apud{Fam.}{8, 1, 1}.

\latin{cīvitātī persuāsit},
\english{persuaded \emph{(made acceptable to)} the state};
\apud{B.~G.}{1, 2, 1}

\latin{sibi satis esse dūxērunt},
\english{thought it was enough for them};
\apud{B.~G.}{1, 3, 2}.

\latin{satis facere reī pūblicae},
\english{satisfy the state}
(do enough for);
\apud{Cat.}{1, 1, 2}.

\latin{neque eī fās erat},
\english{nor was it proper for him}
(to speak);
\apud{Off.}{3, 7, 34}.

\latin{sibi idem licēre},
\english{\emph{(thought)} the same was proper for them};
\apud{B.~G.}{3, 10, 2}.

\end{examples}

\smallskip

II.\enskip After words and phrases expressing or implying \emph{Attitude}.
\begin{examples}

\latin{blandus est pauperī},
\english{is flattering to the poor};
\apud{Aul.}{196}.

\latin{mātrī blandītur},
\english{flatters the mother}
(is flattering to);
\apud{Flacc.}{37, 92}.

\latin{adversus nēminī},
\english{opposed to no man};
\apud{And.}{64}.

\latin{quī vōbīs adversantur},
\english{who oppose you}
(are opposed to you);
\apud{Phil.}{1, 15, 36}.

\latin{dictō audientēs},
\english{obedient}
(listening to the word);
\apud{B.~G.}{1, 39, 7}.

\latin{Serviō dictō audientem},
\english{obedient to Servius};
\apud{Liv.}{1, 41, 5}.

\latin{mihi crēde},
\english{trust me}
(be trustful toward);
\apud{Cat.}{1, 3, 6}.

\latin{habēbat studiīs honōrem},
\english{he had respect for literary pursuits};
\apud{Plin.\ Ep.}{6, 2, 2}.

\end{examples}

\smallskip

III.\enskip After words and phrases expressing or implying \emph{Relation}.
\begin{examples}

\latin{servīre meae laudī},
\english{to serve my glory}
(be serviceable to);
\apud{Cat.}{1, 9, 23}.

\latin{vectīgālīs sibi fēcērunt},
\english{made them tributary to themselves};
\apud{B.~G.}{4, 3, 4}.

\latin{proximī sunt Germānīs},
\english{they are next to the Germans};
\apud{B.~G.}{1, 1, 4}.

\latin{cīvitātēs propinquae iīs locīs},
\english{states near \(to\) these places};
\apud{B.~G.}{2, \emend{118}{35}{34}, 3}.

\latin{fīnibus appropinquāre},
\english{to be drawing near the boundaries};
\apud{B.~G.}{2, 10, 5}.

\latin{fit obviam Clōdiō},
\english{meets Clodius}
(becomes in-the-way to);
\apud{Mil.}{10, 29}.

\latin{virtūs hominem iungit deō},
\english{virtue joins men to the gods};
\apud{Ac.}{2, 45, 139}.

\end{examples}

\headingC{Details of the Dative of Direction or Relation}

\section
\subsection

In general, the Dative of Direction is not used with a \emph{noun
  alone}, though it may be with a noun plus a verb.  Compare
\latin{cui fidem habēbat}, \english{in whom he had confidence} (= \latin{cui
  cōnfīdēbat}), \apud{B.~G.}{1, \emend{167}{19, 3}{42, 6}},
with \latin{testimōnī fidem},
\english{confidence in the testimony}, \apud{Flacc.}{15, 36}, and
\latin{fidēs ergā plēbem}, \english{confidence in the people},
\apud{Leg.\ Agr.}{2, 8, 20}.
\begin{enumerate}

\item
But abstract and semi-abstract nouns strongly suggesting action
sometimes take the Dative of Direction.  Thus \latin{obtemperātiō
  lēgibus},  \english{obedience to the laws}, \apud{Leg.}{1, 15, 42};
\latin{īnsidiae cōnsulī}, \english{the plotting against the consul},
\apud{Sall.\ Cat.}{32, 1}.

\item

A few personal nouns, mostly official, \emph{may} take the Dative of
Direction (rarely without a verb) instead of the ordinary Genitive.
So especially \latin{adiūtor}, \latin{comes}, \latin{custōs},
\latin{dux}, \latin{hērēs}, \latin{lēgātus}, \latin{patrōnus},
\latin{quaestor}, \latin{socius}, \latin{tūtor}.  Similarly the
adjective \latin{cōnscius}.
\begin{examples}

\latin{tibi vēnit adiūtor},
\english{came as assistant to you};
\apud{N.~D.}{1, 7, 17}.

\latin{suīs bonīs hērēdem esse},
\english{to be heir to his goods};
\apud{Caecin.}{4, 12}.

\latin{nūllus est portīs custōs},
\english{there is no guard for the gates};
\apud{Cat.}{2, 12, 27}.

\latin{huic ego mē bellō ducem profiteor},
\english{I offer myself as leader for this war};
\apud{Cat.}{2, 5, 11}.

\end{examples}

\end{enumerate}

\subsection

\textbf{Poetic and later Dative of Direction or Relation}.  The poets
and later writers extend the construction, using it
\begin{enuma}

\item

With many personal nouns of \emph{attitude} or \emph{relation}, with
or without a verb.  So (beside the list above) with \latin{acceptor},
\latin{auctor}, \latin{caput}, \latin{cognātus}, \latin{con\-iūnx},
\latin{fīlius}, \latin{frāter}, \latin{hospes}, \latin{māter},
\latin{nūtrīx}, \latin{parēns}, \latin{pater}, \latin{patruus},
\latin{prōmus}, \latin{rēctor}, \latin{rēgnātor}, \latin{rēx} (also
\latin{rēgnum}), \latin{sacerdōs}, \latin{sodālis}, \latin{servus},
\latin{tes\-tis}.
\begin{examples}

\latin{Faunō Pīcus pater},
\english{to Faunus, Picus was father};
\apud{Aen.}{7, 48}.

\end{examples}

\pagebreak

\item

With verbs resembling those of \xref{362} in meaning.
\begin{examples}

\latin{propinquābam portīs},
\english{was approaching the gates};
\apud{Aen.}{2, 730}
(with \latin{propinquō} as with \latin{appropinquō}).

\latin{aequāta caelō},
\english{made level with \emph{(equal to)} the sky};
\apud{Aen.}{4, 89}
(with \latin{aequō} as with \latin{aequus}).

\latin{dubiīs nē dēfice rēbus},
\english{fail not our doubtful fortunes};
\apud{Aen.}{6, 196}.
Cf.\ \xref[\emph{a}]{364}.

\end{examples}

\item

With verbs of \emph{union}, \emph{contention}, or
\emph{difference}.\footnote{\label{ftn:193:}So with \latin{misceō} (in
  prose regularly with Abl.\ or \latin{cum}; \xref{431});
  \latin{sociō}, \latin{cōnsociō} (in prose regularly with
  \latin{cum}; \xref[1]{419}); \latin{haereō} (in prose with
  \latin{ad} or \latin{in}; in the Dative with \emph{personal} nouns
  only); \latin{nectō} (in prose with \latin{ex}); \latin{altercor},
  \latin{certō}, \latin{contendō}, \latin{luctor}, \latin{obluctor},
  \latin{pugnō} (in prose regularly with \latin{cum}; \xref[4]{419});
  \latin{differō}, \latin{discordō}, \latin{discrepō},
  \latin{dissentior}, \latin{distō}, \english{differ} (in prose
  regularly with \latin{ab}; \xref{412}).  Similarly with certain
  participles of other verbs.  Thus \latin{āversa hostī},
  \english{turned away from the enemy}; \apud{Tac.\ Ann.}{1, 66}.

  \latin{Haereō} also occurs with a locative ablative (\xref{436})
  without a preposition (rarely in prose, oftener in poetry).  Thus
  \latin{haeret pede pēs}, \apud{Aen.}{10, 361}.}
\begin{examples}

\latin{sē miscet virīs},
\english{mingles with the men};
\apud{Aen.}{1, 439}.

\latin{haeret laterī lētālis harundō},
\english{the deadly shaft sticks in the side};
\apud{Aen.}{4, 73}.

\latin{pugnābis amōri?}
\english{shall you struggle against love?}
\apud{Aen.}{4, 38}.

\end{examples}

\item

With \latin{adsuēfaciō}, \latin{adsuēscō}, and \latin{suēscō} (the
last poetic only).  Thus \latin{mēnsae adsuētus erīlī},
\english{accustomed to the table of his mistress}, \apud{Aen.}{7, 490}
(Ablative in Ciceronian prose; \xref[\relax and \emph{a}]{431}); \latin{hīs
  suētus}, \english{accustomed to these}; \apud{Aen.}{5, 414}.

\item

With \latin{īdem}, \english{the same} (cf.\ the Dative with
\latin{similis}).
\begin{examples}

\latin{idem facit occīdentī},
\english{does the same thing as a murderer};
\apud{A.~P.}{467}.

\end{examples}

\item

With verbs corresponding to adjectives that take the Dative, and
adjectives corresponding to verbs that take the Dative.
\begin{examples}

\latin{mihi saevit},
\english{is savage to me};
\apud{Rud.}{825}.
Cf.\ \latin{saevam ambōbus}, \apud{Aen.}{1, 458}.

\latin{simulāta magnīs Pergama},
\english{a Trojan citadel made like the great one};
\apud{Aen.}{3, 349}.

\latin{crēdula posterō},
\english{trusting to the future};
\apud{Carm.}{1, 11, 8}.

\end{examples}

\end{enuma}

\subsection

Several adjectives which ordinarily take the Dative \emph{may} take
the Genitive.  Compare English “neighbor to” and “neighbor of”;
and see \xref[\emph{c}]{339}.

\subsection

\latin{Propior} and \latin{proximus} may take the Accusative of
Space-Relation.  See \xref[\emph{b}]{380}.

\subsection

\latin{Fīdō} and \latin{cōnfīdō} may take the Ablative.  See~\xref{439}.

\headingC{Remarks on the Dative of Direction or Relation}

\section
\subsection

Verbs of Quality, Attitude, or Relation are with few exceptions
\emph{intransitive} in Latin, while in English we more frequently
employ \emph{transitive} verbs.  Compare \latin{noceō}, \english{am
  injurious to}, with the English “injure.”

\begin{minor}

\subsubsection

But Latin also possesses several \emph{transitive} verbs of similar
meaings, e.g.\ \latin{iubeō}, \english{order}, \latin{iuvō},
\english{help}, \english{assist}, \latin{laedō}, \english{harm},
\latin{dēficiō}, \english{fail}, \latin{dēlectō}, \english{please}.
These of course take the Accusative of the Direct Object (\xref{390}).

\end{minor}

\subsection

In the passive, verbs of this class are regularly used only
\emph{impersonally}.  The Dative remains.
\begin{examples}

\latin{hīs persuādērī nōn poterat},
\english{they could not be persuaded}
(it could not be made agreeable to them);
\apud{B.~G.}{2, 10, 5}.
\end{examples}

\begin{minor}

\subsubsection

For rare exceptions, see~\xref[\emph{b}]{292}.

\end{minor}

\subsection

\latin{Crēdō} takes a Direct Object of the \emph{thing} believed.
\begin{examples}

\latin{id quod volunt crēdunt},
\english{believe what they want to believe};
\apud{B.~G.}{3, \emend{122}{18}{16}, 6}.

\end{examples}

\subsection

A few Verbs that take a Dative may take a Direct Object in addition.
Thus \vrb{imperō}, \english{levy},
\vrb{indulgeō}, \english{indulge}, \vrb{minor}, \english{threaten},
\vrb{persuadeō}, \english{persuade}.
\begin{examples}

\latin{id iīs persuāsit},
\english{he persuaded them \emph{(to)} this}
(made this agreeable);
\apud{B.~G.}{1, 2, 3}.

\end{examples}

\subsection

Several Verbs take either the Dative of Direction or the Accusative of
the Direct Object, with somewhat different meanings, or at different
periods.  Thus \vrb{aemulor}, \vrb{medeor},
\vrb{praestōlor}, \vrb{temperō}.
\versionB*{Similarly \latin{aequō} in poetry.}

\subsection

The \emph{End for which a Quality is Adapted} is generally expressed
by \latin{ad} (occasionally \latin{in}) with the Accusative.  Thus
\latin{ad pugnam inūtilēs},
\english{useless for fighting},
\apud{B.~G.}{2, \emend{168}{16}{15}, 5};
\latin{ad bellum apta},
\english{in shape for war},
\apud{B.~C.}{1, 30, 5};
\latin{ad hanc rem idōneō},
\english{suited for this thing},
\apud{Verr.}{1, 33, 83}.

\subsection

Instead of the Dative, many Adjectives of \emph{Attitude} may take
\latin{ergā}, \latin{in}, or \latin{adversus} with the Accusative.  Thus
\latin{in Teucrōs benignam},
\english{kindly disposed toward the Trojans};
\apud{Aen.}{1, 304}.
Cf.\ \latin{aliī benigna},
\english{kindly disposed toward another};
\apud{Carm.}{3, 29, 52}.

\subsection

The feeling of Direction sometimes leads to the use of \latin{in} with
Adjectives of \emph{Quality}.  Thus
\latin{grātae in vulgus},
\english{agreeable to the populace};
\apud{Liv.}{2, 8, 3}.

\headingC{Dative of the Indirect Object}

\section

The \emph{Indirect Object} of a Transitive Verb is put in the
Dative.\footnote{So especially with verbs like
  \vrb{dō}, \vrb{reddō}, \vrb{trādō}, \vrb{tribuō},
  \vrb{tendō}, \vrb{praebeō}, \latin{praestō} (\english{exhibit},
  \english{furnish}), \latin{sūmō}; \latin{ferō}, \latin{mittō};
  \latin{dēbeō}, \latin{polliceor}, \latin{prōmittō}, \latin{spondeō},
  \latin{negō}; \latin{mandō}, \latin{praecipiō}; \latin{mōnstrō},
  \latin{nārrō}, \latin{dīcō}, \latin{nūntiō}, \latin{respondeō};
  \latin{faciō} (\english{do}), \latin{agō} (\english{render},
  \english{give}).

  With \latin{ferō} and \latin{mittō}, the force of the Dative is on
  the line between the original one of Direction in Space and the
  derived one of the Indirect Object.}
\begin{examples}

\latin{dat negōtium Senonibus},
\english{assigns the task to the Senones};
\apud{B.~G.}{2, 2, 3}.

\latin{rēgī haec dīcite},
\english{tell your king this}
(tell this \emph{to} him);
\apud{Aen.}{1, 137}.

\end{examples}

\subsubsection

Since a Transitive Verb ordinarily takes a Direct Object, an Indirect
and a Direct Object often appear together, as above.

\begin{minor}

\subsection

With some verbs, e.g.\ \vrb{dōnō} and \vrb{aspergō}, different
conceptions are possible, and different constructions may accordingly
be used.
\begin{examples}

\latin{praedam mīlitibus dōnat},
\english{presents the booty \textsc{to} the soldiers};
\apud{B.~G.}{7, 11, 9}.

\latin{cīvitāte multōs dōnāvit},
\english{presented many \textsc{with} citizenship};
\apud{Arch.}{10, 16}.

\end{examples}

\end{minor}

\headingC{Dative of Reference or Concern, after any Verb}

\section

Almost any Verb may be followed by a Dative of the Person to whom the
act or state \emph{refers}, or whom it \emph{concerns}.

A Dative of the Thing is less frequent.
\begin{examples}

\latin{mī ēsuriō, nōn tibi},
\english{’t is for myself I’m hungry, not for you};
\apud{Capt.}{866}.

\latin{praeterita sē frātrī condōnāre dīcit},
\english{tells \emph{(Dumnorix)} that he forgives the past for the
  sake of \emph{(having reference to)} his brother};
\apud{B.~G.}{1, 20, \emend{91}{6}{5}}.

\end{examples}

\subsubsection

The Dative of Reference is especially frequent with
\vrb{est} combined with a noun or adverbial phrase.
\begin{examples}

\latin{nūllus est iam lēnitātī},
\english{there is no longer any room for gentleness};
\apud{Cat.}{2, 4, 6}.

\latin{tibi in cōnsiliō sunt},
\english{advise \emph{(are in council for)} you};
\apud{Quinct.}{1, 4}.

\end{examples}

\subsubsection

\textbf{Poetic and later Dative of Reference with Nouns.}  The poets and
later prose writers often attach the Dative of Reference to nouns.
\begin{examples}

\latin{collō monīle},
\english{a collar for the neck}
(necklace);
\apud{Aen.}{1, 654}.

\latin{pectorī tegimen},
\english{a covering for the breast};
\apud{Liv.}{1, 20, 4},

\end{examples}

\subsubsection

The Dative of Reference \emph{may} be used, with words denoting
persons, after \vrb{interdīcō}, \english{forbid},
\vrb{interclūdō}, \english{cut off}, and
\vrb{dēpellō}, \english{turn away}; also, in poetry, after
\vrb{arceō}, \english{keep off}, and \vrb{dēfendō}, \english{ward
  off}.
\begin{examples}

\latin{quibus cum aquā ignī interdīxisset},
\english{after forbidding them \emph{(from)} the use of fire and
  water};
\apud{B.~G.}{6, 44, 3}.
(So regularly in this phrase.)

\latin{dēfendit aestātem capellīs},
\english{wards off the heat from my goats};
\apud{Carm.}{1, 17, 3}.

\end{examples}

\begin{note}

These verbs commonly take an Accusative of the Person and an Ablative
of the Thing (\xref{408}).  \vrb{Interdīcō} may also combine a Dative
of the Person (as above) with an Ablative of the Thing.
\begin{examples}

\latin{utī frūmentō Caesarem interclūderet},
\english{in order to cut Caesar off from supplies};
\apud{B.~G.}{1, 48, 2}.
(\latin{Frūmentō} is Ablative.)

\latin{quā adrogantiā Galliā Rōmānīs interdīxisset},
\english{with what arrogance he had excluded the Romans from Gaul}
(interdicted them from);
\apud{B.~G.}{1, 46, 4}.
(\latin{Rōmānīs} is Dative.)

\end{examples}

\end{note}

\begin{minor}

\subsubsection

“For,” meaning “in defence of,” must be expressed by \latin{prō}
(\xref[1]{407}).

\end{minor}

\headingC{Dative of Reference \emph{versus} the Accusative}

\section

Several Verbs of Feeling or Thought take either the Accusative or the
Dative, according as the word which they govern is thought of as the
\emph{Direct Object}, or as that \english{in behalf of which},
\english{for which}, the feeling or thought is entertained.

So \latin{cōnsulō}, \latin{cupiō}, \latin{dēspērō}, \latin{metuō},
\latin{prōspiciō}, \latin{prōvideō}, \latin{timeō}.
\begin{examples}

\latin{suīs rēbus timēre},
\english{to feel fears for their own position};
\apud{B.~G.}{4, 16, 1}.
(But
\latin{magnitūdenem silvārum timēre},
\english{feared the great stretch of forest};
\apud{B.~G.}{1, 39, 6}.)

\latin{cōnsulite vōbīs, prōspicite patriae},
\english{look out for yourselves, take thought for your country};
\apud{Cat.}{4, 2, 3}.
(But
\latin{sī mē cōnsulis},
\english{if you ask my advice};
\apud{Cat.}{1, \emend{169}{5}{6}, 13}.)

\end{examples}

\headingC{Dative of Reference \emph{in place of} the Genitive}

\section

The Dative of the Person \emph{concerned by an act or state as a
  whole} is often used in place of a Possessive Genitive.
\begin{examples}

\latin{sēsē Caesarī ad pedēs prōiēcērunt},
\english{cast themselves at Caesar’s feet};
\apud{B.~G.}{1, 31, 2}.
Cf.\ \latin{cum sē ad Caesaris pedēs abiēcisset}, \apud{Fam.}{4, 4, 3}.

\latin{quotiēns tibi iam extorta est ista sīca dē manibus!}
\english{how often has that dagger of yours been twisted \emph{(out of
    the hands for you, i.e.)}\ out of your hands!}
\apud{Cat.}{1, 6, \emend{170}{16}{15}}.
Similarly \apud{Aen.}{1, 477}.

\end{examples}

\subsubsection

This construction gives a touch of \emph{feeling}, of \emph{concern},
to the expression.  English has no corresponding idiom.

\headingC{Freer Poetic Dative of Reference or Concern}

\section

The later poets freely use the Dative of Reference in loose attachment
to the rest of the sentence.
\begin{examples}

\latin{tālia iactantī procella vēlum adversa ferit},
\english{as he utters these words a blast strikes the sail athwart}
(for him, uttering these words);
\apud{Aen.}{1, 102}.

\latin{vīvitur parvō bene, cui paternum splendet in mēnsā tenuī salīnum},
\english{he lives well upon a little, for whom there shines, upon a
  frugal board, the saltcellar which his father had before him};
\apud{Carm.}{2, 16, 13}.

\end{examples}

\subsubsection

The warmth and feeling of this construction gave it great vogue in
later poetry.  It is used with pronouns with especial frequency.

\headingC{Dative of the Person Judging}

\section

The Dative is used to denote the person \emph{in whose eyes} or
\emph{for whom} the statement of the sentence holds good.
\begin{examples}

\latin{Quīntia fōrmōsa est multīs},
\english{in the eyes of many \emph{(to many)} Quintia is beautiful};
\apud{Catull.}{86, 1}.

\latin{levāta mihi vidētur},
\english{\emph{(the state)} seems to me relieved};
\apud{Cat.}{2, 4, 7}.

\end{examples}

\subsubsection

Out of this grew the \term{Dative of the Local Point of View} (with
the Participle, first in Caesar).
\begin{examples}

\latin{quod est oppidum prīmum Thessaliae venientibus ab Ēpīrō},
\english{which is the first town of Thessaly as one comes \emph{(to
    people coming)} from Epirus};
\apud{B.~C.}{3, 80, 1}.

\end{examples}

\headingC{Dative with Verbs of Taking Away}

\section

Verbs of \emph{taking away}\footnote{Various compounds of \latin{ab},
  \latin{dē}, and \latin{ex}, together with \latin{adimō},
  \latin{subripiō}, \latin{tollō}, etc.} are regularly followed by the
Dative of words denoting Persons.
\begin{examples}

\latin{hunc mihi timōrem ēripe},
\english{remove this fear from me};
\apud{Cat.}{1, 7, 18}.

\latin{scūtō mīlitī dētractō},
\english{snatching a shield from a soldier};
\apud{B.~G.}{2, \emend{171}{25}{24}, 2}.

\latin{omnia sociīs adimere},
\english{took everything from the allies};
\apud{Sall.\ Cat.}{12, 5}.

\end{examples}

\subsubsection

The original conception was that of the Person as \emph{concerned} by
the act.  Thus “remove \emph{for me} this fear.”

\subsubsection

The poets use the construction more boldly, employing it with names of
things as well, and also after verbs of \emph{stealing}, \emph{going
  away}, etc.
\begin{examples}

\latin{silicī scintillam excūdit},
\english{struck out a spark from the flint};
\apud{Aen.}{1, 174}.

\latin{fessōs oculōs fūrāre labōrī},
\english{steal your weary eyes from toil};
\apud{Aen.}{5, 845}.

\latin{ēvādere pugnae},
\english{to escape from the battle};
\apud{Aen.}{11, 702}.

\end{examples}

\headingC{Ethical\footnote{“Ethical” means “of feeling,” and so
    might be used of many Datives.  But its use is confined in grammar
    to the \emph{Personal Pronouns}, in this construction.} Dative}

\section

A Personal Pronoun in the Dative may be loosely attached to a sentence
to suggest \emph{Concern} or \emph{Interest} on the part of the person
denoted.

The effect is generally whimsical or ironical.
\begin{examples}

\latin{quī mihi accubantēs in convīviīs ēructant caedem},
\english{and these men—bless me!—as they recline at their
  banquets, belch forth talk about blood and murder};
\apud{Cat.}{2, 5, 10}.
Cf.\ \apud{Cat.}{2, 2, 4}.
(Cf.\ “they drank me two bottles,” Fielding, \emph{Tom Jones}.)

\latin{ecce tibi tellūs},
\english{there lies the land you wish to reach};
\apud{Aen.}{3, 477}.

\end{examples}

\headingC{Dative of the Agent}

\section

The Dative is used to express the \emph{Agent}:

\subsection

Regularly with the \emph{Future Passive Participle}.
\begin{examples}

\latin{Caesarī omnia ūnō tempore erant agenda},
\english{everything had to be attended to by Caesar at one and the same time};
\apud{B.~G.}{2, \emend{172}{20}{19}, 1}.

\latin{vōbīs erit videndum},
\english{you will have to see to it};
\apud{Cat.}{3, 12, 28}.

\end{examples}

\begin{minor}

\subsubsection

But the construction of the Agent with \latin{ab} (\xref[1]{406}) is
occasionally used, either for sharper contrast, or to avoid confusion
with the Dative of the Person Concerned, etc.
\begin{examples}

\latin{aguntur bona multōrum cīvium, quibus est ā vōbīs cōnsulendum},
\english{the property of many citizens is at stake, and for this
precautions must be taken by you};
\apud{Pomp.}{2, 6}.

\end{examples}

\end{minor}

\subsection

Somewhat freely with the \emph{Perfect Passive Participle}, and forms
com\-pound\-ed with it.
\begin{examples}

\latin{meīs cīvibus suspectum},
\english{suspected by my fellow-citizens};
\apud{Cat.}{1, 7, 17}.

\latin{quī tibi ad caedem cōnstitūtī fuērunt},
\english{who have been set apart for death by you};
\apud{Cat.}{1, 7, 16}.

\end{examples}

\subsection

Occasionally, in the later writers, with \emph{any} passive form.
\begin{examples}

\latin{neque cernitur ūllī},
\english{and is not seen by any one};
\apud{Aen.}{1, 440}.

\end{examples}

\begin{minor}

\subsubsection

The later writers sometimes used the construction with
an \emph{adjective of passive meaning}.
\begin{examples}

\latin{multīs bonīs flēbilis},
\english{by many a good man to be mourned};
\apud{Carm.}{1, 24, 9}.

\latin{tolerābilīs vōbīs eās fore crēditis?}
\english{do you think they will be endurable to you \emph{(possible to
be endured by you)}?}
\apud{Liv.}{34, 3, 2}.

\end{examples}

\end{minor}

\headingC{Dative of Possession}

\section

Possession may be expressed by the \emph{Dative with the
  Verb \latin{sum}}.
\begin{examples}

\latin{erat eī cōnsilium ad facinus aptum},
\english{he possessed an understanding specially adapted for crime};
\apud{Cat.}{3, 7, 16}.

\latin{sunt mihi bis septem Nymphae},
\english{I have twice seven Nymphs};
\apud{Aen.}{1, 71}.

\end{examples}

\begin{examples}

\subsubsection

The Dative with \latin{sum} asserts the fact of Possession.  The
Possessive Genitive (\xref{339}) \emph{involves} the fact of
possession, but this idea is only a subordinate one in the sentence.

\subsubsection

For the attraction of the Name into the case of the Possessor, see
\xref[3]{326}.

\end{examples}

\headingE{Poetic Dative of Direction in Space}

\section

The poets use the Dative freely to express \emph{Direction in Space}.
\begin{examples}

\latin{it clāmor caelō},
\english{the shout rises to the heavens};
\apud{Aen.}{5, 451}.

\latin{caelō capita ferentēs},
\english{raising their heads toward heaven};
\apud{Aen.}{3, 678}.

\latin{pelagō dōna praecipitāre},
\english{hurl the gifts into the sea};
\apud{Aen.}{2, 36}.

\end{examples}

\begin{examples}

\subsubsection

The construction is sometimes used with great boldness of phrase.
\begin{examples}

\latin{stīpat carīnīs argentum},
\english{packs silver into the ships}
(for \emph{packs the ships with silver});
\apud{Aen.}{3, 465}.
Similarly \apud{Aen.}{1, 195}.
The feeling is as in \latin{laterī abdidit ēnsem} (for \latin{in latus
abdidit}), \apud{Aen.}{2, 553}.

\end{examples}

\subsubsection

The prose construction is the Accusative with \latin{ad} or \latin{in}
(\xref{385}).  Thus \latin{it ad aethera clāmor}, \english{the shout
rises to the sky}; \apud{Aen.}{12, 409}.

\end{examples}

\headingE{Dative in a Construction of Composite Origin (Fusion)}

\headingC{Dative after Verbs compounded with certain Prepositions}

\section

The Dative of the Person or Thing Concerned may be used after
\emph{many Verbs compounded with the Prepositions} \latin{ad},
\latin{ante}, \latin{circum}, \latin{con}, \latin{in}, \latin{inter},
\latin{ob}, \latin{post}, \latin{prae}, \latin{sub}, or
\latin{super}.
\footnote{\latin{Adsentior}, \latin{cōnsentior},
  \latin{adversor}, \latin{convenit}, \latin{obsequor}, \latin{officiō},
  \latin{obsistō}, \latin{obstō}, \latin{obsum}, \latin{prōsum}, are
  generally placed here, but belong more properly under~\xref{362}.
  Cf.\ the Dative with the corresponding (or opposite) words
  \latin{adversus}, \latin{cōnsentāneus}, \latin{oboediō},
  \latin{pāreō}, \latin{repugnō}, \latin{resistō}, \latin{dēsum},
  \latin{expedit}.  Yet \latin{oppōnō} shows the impossibility of
  drawing fixed lines.  \latin{Excellō}, \english{excel}, follows the
  analogy of \latin{praestō}, \english{surpass}.}
\begin{examples}

\latin{adportō vōbīs Plautum},
\english{I bring \emph{(to)} you Plautus};
\apud{Men.}{3}.

\latin{fīnitimīs bellum īnferre},
\english{to make war upon their neighbors};
\apud{B.~G.}{1, 2, 4}.

\latin{virtūte omnibus praestārent},
\english{were above all in valor};
\apud{B.~G.}{1, 2, 2}.

\end{examples}

\subsubsection

If the verb of the compound is Transitive, it may of course take a
Direct Object (\xref{390}), in addition to the Dative taken by the
compound as a whole.  See \latin{fīnitimīs bellum īnferre}, above.

\begin{minor}

\subsubsection

Several compounds may take either this construction or an Accusative
of the \emend{81}{object}{Direct Object} and an Ablative of
\emend{23}{m}{M}eans (\xref{423}).  Thus \latin{circumdō},
\latin{circumfundō}, \latin{aspergō}, \latin{induō} (in later Latin,
\latin{accingō}, \latin{implicō}, etc.).
\begin{examples}

\latin{arma circumdat umerīs},
\english{puts his armor about his shoulders};
\apud{Aen.}{2, 509}.

\latin{reliquōs equitātū circumdederant},
\english{had surrounded the rest with cavalry};
\apud{B.~G.}{4, 32, 5}.

\end{examples}

\subsubsection

Several compounds may take either the Dative or the Accusative
(\xref[2, \emph{a}]{391}).  Thus \latin{inlūdō}, \english{jeer
at}, \english{mock}.

\subsubsection

Several compounds expressing \emph{comparison}, \english{union},
or \english{agreement} may take either the Dative, or the Ablative
with \latin{cum} (\xref[1, 3]{419}).
Thus \latin{comparō}, \latin{cōnferō} (cf.\ English “compare to” and
“compare with”).

\end{minor}

\headingC{Remarks on the Dative after Compound Verbs}

\section
\subsection

Compounds expressing \emph{literal motion only} are regularly followed
by the Accusative with \latin{ad} or \latin{in}.  Thus
\latin{ad eum adcurrit}, \english{runs up to him}, \apud{B.~G.}{1, 22, 2};
\latin{in gladium incubuerat}, \english{had fallen upon his sword},
\apud{Inv.}{2, 51, 154}.

\subsection

For compounds capable of expressing \emph{both literal motion and a
figurative idea} (like most under~\xref{376}), no fixed rule can be
laid down.

\subsubsection

With some compounds both constructions are in use.  Thus \latin{in mē
incidit}, \english{he fell in with me}, \apud{Planc.}{41, 99}; and
\latin{hominī incidī}, \english{I fell in with the
man}, \apud{Verr.}{2, 74, 182}.

\subsubsection

In general, it may be said that the preposition is regularly
used \emph{if the literal side of the meaning is to be brought out
more strongly than usual}.  Thus \latin{bellum intulit prōvinciae
Galliae}, \english{has made war upon the province of
Gaul}, \apud{Phil.}{5, 9, 24}; but \latin{dē bellō ā Parthīs in
prōvinciam Syriam inlātō}, \english{with regard to the war which has
been carried by the Parthians into the province of
Syria}, \apud{Fam.}{15, 2, 1}.

\subsubsection

Yet many compounds with purely \emph{figurative} meanings regularly
take a preposition.  Thus \latin{incumbite ad salūtem reī
pūblicae}, \english{bend your energies to the welfare of the
state}; \apud{Cat.}{4, 2, 4}.

\subsection

The poets and later prose writers love to vary the older construction,
whatever it may be, \emph{for the mere sake of variety}.  Thus Virgil,
\apud{Aen.}{5, 15}, says \latin{incumbere rēmīs}, \english{to bend to
the oars} (compare Cicero, under 2, \emph{c} above);
and \apud{Livy}{, 9, 22, 4}, says \latin{adequitāre
vāllō}, \english{rode up to the rampart}, where Caesar would have used
\latin{ad} (cf.\ \latin{ad nostrōs adequitāre}, \english{were riding
up to our men}, \apud{B.~G.}{1, 46, 1}.

\subsection

The poets and later writers likewise use the Dative with compounds not
employed at all in Ciceronian Latin.  Thus with \latin{ingeminō}
(\apud{Aen.}{5, 434}), \latin{invergō} (\apud{Aen.}{6, 244}).

\subsection

The poets sometimes use the Dative with verbs resembling those
of \xref{376} in meaning, but differently formed.
\begin{examples}

\latin{captae superāvimus urbī},
\english{have survived the capture of the city};
\apud{Aen.}{2, 643}.
(\latin{Superō} like \latin{supersum}.)

\end{examples}

\chapter{The Accusative}

\contentsentry{C}{Uses of the Accusative}

\section

The Latin Accusative expresses three general classes
of ideas:
\begin{enumI}

\item
\emph{Space-Relation} (\textsc{not} \emph{Separative or Locative}).

\item
\emph{Respect.}

\item
\emph{The Direct Object.}

\end{enumI}

\section
\subtitle{\textsc{Synopsis of the Principal Uses of the Accusative}}

\begin{synopsis}

\a{I}{Accusative of Space-Relations (not Separative or Locative)}
\b{\xref{380}–\xref{384}}{Accusative with Prepositions}
\c{\xref{386}}{With Verbs compounded with \latin{trāns} or \latin{circum}}
\c{\xref{385}}{Regular expression of the Place Whither}
\b{\xref[\emph{b}]{385}, \xref{450}}{Accusative of Names of Towns, etc., Whither, \emph{without} a preposition}
\b{\xref{387}}{Accusative of Extent, Duration, or Degree}

\medskip

\a{II}{Accusative of Respect}
\b{}{Accusative of Respect:}
\c{\xref{388}}{In Ciceronian prose in a few phrases only}
\c{\xref{389}}{In freer use in later Latin}

\medskip

\a{III}{Accusative of the Direct Object}
\b{\xref{390}}{Accusative of the Direct Object}
\c{\xref[1]{391}}{With Verbs ordinarily Intransitive}
\c{\xref[2]{391}}{With Compounds acquiring Transitive Force}
\b{}{Two Objects with Verbs:}
\c{\xref{392}}{of \emph{making}, \emph{choosing}, \emph{having}, \emph{regarding}, \emph{calling}, or \emph{showing}}
\c{\xref{393}}{of \emph{inquiring}, \emph{requesting}, \emph{teaching}, or \emph{concealing}}
\b{\xref{394}}{Accusative of the Result Produced}
\c{\xref{395}}{Accuative in Apposition to a sentence}
\b{\xref[1]{396}}{Accusative of Kindred Meaning}
\c{\xref[2]{396}}{Extended use of the Accusative of Kindred Meaning}
\c{\xref{397}}{Freer Neuter Accusative Modifiers}
\b{\xref{398}}{Subject of an Infinitive}
\b{\xref{399}}{Accusative of Exclamation}

\end{synopsis}

% \pagebreak

\headingE{Accusative of Space-Relations (Not Separative or
  Locative) and of Corresponding Figurative Relations}

\vskip-1.5\bigskipamount

\headingC{Accusative with Prepositions\protect\footnotemark}

\vskip-\bigskipamount

\footnotetext{For summarized statements for all Prepositions,
see \xref{455}–\xref{458}.}

\section

The Accusative is always used with the Prepositions \latin{ad},
\latin{adversus} or \latin{adversum}, \latin{ante}, \latin{apud},
\latin{circā}, \latin{circiter}, and \latin{circum}, \latin{cis} and
\latin{citrā}, \latin{contrā}, \latin{ergā}, \latin{extrā},
\latin{īnfrā}, \latin{inter}, \latin{intrā}, \latin{iūxtā},
\latin{ob}, \latin{penes}, \latin{per}, \latin{pōne} and \latin{post},
\latin{praeter}, \latin{prope}, \latin{propter}, \latin{secundum},
\latin{suprā}, \latin{trāns}, \latin{ultrā} (and \latin{uls}),
\latin{versus}.
\begin{examples}

\latin{iūxtā mūrum},
\english{close to the wall};
\apud{B.~C.}{1, 16, 4}.

\latin{ante oppidum},
\english{in front of the town};
\apud{B.~G.}{2, \emend{173}{32}{31}, 4}.

\latin{Hannibal erat ad portās},
\english{Hannibal was at the gates};
\apud{Phil.}{1, 5, 11}.

\latin{ad omnīs nātiōnēs sānctum},
\english{sacred among all peoples};
\apud{B.~G.}{3, 9, 3}.

\latin{ad castra contendērunt},
\english{hastened to the camp};
\apud{B.~G.}{2, 7, 3}.

\latin{iter per prōvinciam},
\english{a passage through the province};
\apud{B.~G.}{1, 8, 3}.

\latin{vestra ergā mē voluntās},
\english{your good will toward me};
\apud{Cat.}{4, 1, 1}.

\end{examples}

\subsubsection

\latin{Versus} follows its noun.  Thus
\latin{orientem versus}, \english{toward the east};
\apud{Plin.\ N.~H.}{5, \emend{24}{43}{14}}.
But this is generally preceded by a preposition, unless it denotes a
Town or Small Island (\xref{450}).  Thus
\latin{ad merīdiem versus}, \english{toward the south};
\apud{Plin.\ N.~H.}{5, \emend{25}{43}{14}}.

\begin{minor}

\subsubsection

The adverbs \latin{propius} and \latin{proximē} commonly, and the
adjectives \latin{propior} and \latin{proximus} occasionally, take the
Accusative of Space-Relation.  (For the Dative with these adjectives,
see \xref{362}; for \latin{ab} and the Ablative, \xref[2]{406}.)
\begin{examples}

\latin{proximē deōs accessit},
\english{has come very near the gods};
\apud{Mil.}{22, 59}.

\latin{quī proximī Rhēnum incolunt},
\english{who live next the Rhine};
\apud{B.~G.}{1, 54, 1}.

\end{examples}

\subsubsection

\latin{Prīdiē} and \latin{postrīdiē}, \emph{the day before} and
\english{the day after}, generally take the Accusative (of
Time-Relation), but sometimes the Genitive (of Connection;
\xref{339}).
\begin{examples}

\latin{prīdiē Kalendās},
\english{the day before the Calends};
\apud{Cat.}{1, 6, 15}.

\latin{postrīdiē eius diēī},
\english{the day after that day}
(on the after-day of that day);
\apud{B.~G.}{1, 23,~1}.

\end{examples}

\subsubsection

\latin{Per} may be used to represent persons as the \emph{Means
  through Which}, in contrast to the Ablative with \latin{ab}, which
represents them as \emph{Agents} (\xref[1]{406}).  Compare \latin{rē
  per speculātōrēs cognitā}, \english{the fact having been learned
  \textsc{through} spies}, \apud{B.~G.}{2, 11, 2}, with
\latin{cōnfirmātā rē ab explōrātibus}, \english{the report having been
  confirmed \textsc{by} scouts}, \apud{B.~G.}{2, 11, 3}.

\end{minor}

\section

The Accusative is used with \latin{in} and \latin{sub} to express the
Place Whither something \emph{moves}.
\begin{examples}

\latin{cum in castra contenderent},
\english{when hurrying into camp};
\apud{B.~G.}{4, 37, 1}.

\latin{sub nostram aciem successērunt},
\english{came up under our line};
\apud{B.~G.}{1, 24, \emend{97}{5}{4}}.

\end{examples}

\subsubsection

The Ablative is used to express the Place \emph{Where} something
\emph{is} or \emph{is done} (\xref{433}).

\begin{minor}

\subsubsection

\latin{Sub} regularly takes the Accusative when meaning \english{just
  before}, \english{just after}, or \english{about}.
\begin{examples}

\latin{sub occāsum sōlis},
\english{just before sunset};
\apud{B.~G.}{2, 11, 6}.

\latin{sub vesperum},
\english{about evening};
\apud{B.~G.}{7, 60, 1}.

\end{examples}

\end{minor}

\section

The Accusative is regularly used with \latin{subter},
\english{beneath}.
\begin{examples}

\latin{īram in pectore, cupiditātem subter praecordia locāvit},
\english{placed the seat of anger in the breast, the seat of desire
  below the diaphragm};
\apud{Tusc.}{1, 10, 20}.

\end{examples}

\subsubsection

The Ablative \emph{may} be used with \latin{subter} in poetry to
express the Place beneath which something \emph{is} or \emph{is done}.
\begin{examples}

\latin{subter dēnsā testūdine},
\english{under the close-packed roof of shields};
\apud{Aen.}{9, 514},

\end{examples}

\section

The Accusative is regularly used with \latin{super} in the sense of
\english{upon}, \english{at}, or \english{in addition to} (the
Ablative in the sense of \emph{concerning}; see \xref{435}).
\begin{examples}

\latin{saeva sedēns super arma},
\english{sitting upon a pile of cruel arms};
\apud{Aen.}{1, 295}.

\end{examples}

\begin{minor}

\subsubsection

For the poetical Ablative with other senses than \emph{concerning},
see \xref[\emph{a}]{435}.

\end{minor}

\section

The Accusative with a Preposition is used to express a great variety
of figurative ideas.  Notice especially:

\subsection

\emph{The Condition} or \emph{Situation into Which}, with \latin{in}:
\latin{fīliam suam in mā\-tri\-mō\-nium dat}, \english{gives his daughter in
  marriage} (into that condition); \apud{B.~G.}{1, 3, \emend{174}{5}{4}}.
Cf.\ \xref[1]{434}; \xref[3]{406}.

\subsection

\emph{Figurative Direction}, with \latin{ad}, \latin{in},
\latin{ergā}, etc.: \latin{locō ad aciem īnstruendam opportūnō},
\english{in a place suitable for drawing up a line of battle},
\apud{B.~G.}{2, 8, 3}; \latin{intentī ad pācem}, \english{eager for
  peace}, \apud{B.~C.}{3, 19, \emend{175}{4}{5}}; \latin{grātae in vulgus},
\emph{agreeable to the populace}, \apud{Liv.}{2, 8, 2}; \latin{summō
  ergā vōs amōre}, \english{with the greatest love \(toward\) for
  you}; \apud{Cat.}{3, 1, 1}.

\begin{minor}

\subsubsection

The construction is thus often an alternative for the Dative of
Direction after Adjectives and Participles signifying \emph{useful},
\emph{suitable}, or \emph{prepared} (\xref[\relax and 6, 7, 8]{364}).  Also
  for the Objective Genitive depending upon nouns
  (\xref[\emph{b}]{354}).

\subsubsection

\latin{Parātus} takes the Dative also (\xref{362}) in later Latin.
Thus \latin{parāta necī}, \apud{Aen.}{2, 334}; \latin{pācī parātum},
\apud{Liv.}{1, 1, 8}.

\end{minor}

\subsection

\emph{Purpose or Aim}, with \latin{ad} or \latin{in}:
\latin{eō ad conloquium vēnērunt}, \english{came there for a conference},
\apud{B.~G.}{1, 43, 1};
\latin{convīvium in honōrem victōriae}, 
\english{a banquet to celebrate the victory},
\apud{Quintil.}{11, 2, 12}.

\begin{minor}

\subsubsection

Hence the use of \latin{ad} with the Gerundive or Gerund to express
Purpose (\xref[III]{612}).

\end{minor}

\headingC{Regular Expression of the Place Whither}

\section

In accordance with \xref{380} and~\xref{381},

\emph{Place Whither} is regularly expressed by \latin{ad}, \latin{in},
or \latin{sub}, with the Accusative.  The meaning may be either
literal or figurative.
\begin{examples}

\latin{ut in Galliam venīrent},
\english{to come into Gaul};
\apud{B.~G.}{4, 16, 1}.

\latin{ad illa veniō quae\dots},
\english{I come to the things which\dots};
\apud{Cat.}{1, 6, 14}.

\latin{sub populī Rōmānī imperium cecidērunt},
\english{fell under the dominion of the Roman people};
\apud{Font.}{5, 12}.

\end{examples}

\begin{minor}

\subsubsection

With names of Countries, \latin{in} means \english{into}, \latin{ad},
\english{to the borders of}.

\subsubsection

With names of Towns or Small Islands, and with \latin{domus} and
\latin{rūs}, the Place Whither is expressed by the Accusative
\emph{without} a Preposition (\xref{450}).

\subsubsection

The poets freely omit the Preposition with nouns of any kind.
\begin{examples}

\latin{Ītaliam vēnit},
\english{came to Italy};
\apud{Aen.}{1, 2}.

\latin{spēluncam dēveniunt},
\english{came to the cave};
\apud{Aen.}{4, 165}.

\end{examples}

\end{minor}

\headingC{Two Accusatives, after Verbs compounded with
  \latin{trāns} and \latin{circum}}

\section

Transitive Verbs compounded with \latin{trāns} or \latin{circum} may
take an \emph{Accusative depending upon the Preposition}, as well as a
Direct Object (\xref{390}) depending upon the Verb.\footnote{So
  especially \latin{trādūcō}, \latin{trāiciō}, \latin{trānsportō},
  \latin{circumdūcō}.  The later writers extend the list.}
\begin{examples}

\latin{exercitum Ligerim trādūcit},
\english{he leads his army across the Loire}
(= \latin{exercitum trāns Ligerim dūcit});
\apud{B.~G.}{7, 11, 9}.

\latin{quōs Pompeius sua praesidia circumdūxit},
\english{these men Pompey led around his intrenchments};
\apud{B.~C.}{3, 61, 1}.

\end{examples}

\subsubsection

The Accusative is also found with the passive of these verbs, and
with \latin{prae\-ter\-ve\-hor}.
\begin{examples}

\latin{Rhēnum trāductōs},
\english{brought across the Rhine};
\apud{B.~G.}{2, 4, 1}.

\latin{praetervehor ōstia},
\english{I am carried past the mouth};
\apud{Aen.}{3, 688}.

\end{examples}

\subsubsection

But the Preposition \latin{trāns} is often repeated.
\begin{examples}

\latin{nē quam multitūdinem hominum amplius trāns Rhēnum trādūceret},
\english{that he\linebreak
 should lead no more crowds of men across the Rhine};
\apud{B.~G.}{1, 35, \emend{98}{5}{3}}.

\end{examples}

\headingC{Accusative of Extent, Duration, or Degree}

\section

\emph{Extent of Space}, \emph{Duration of Time}, and \emph{Degree} are
expressed by the Accusative.

I.\enskip \emph{Extent of Space}.
\begin{examples}

\latin{oppidum aberat mīlia passuum octō},
\english{the town was eight miles distant};
\apud{B.~G.}{2, 6, 1}.

\latin{multa mīlia passuum prōsecūtī},
\english{after pursuing for many miles};
\apud{B.~G.}{2, 11, 4}.

\end{examples}

II.\enskip \emph{Duration of Time}.
\begin{examples}

\latin{tot annōs bella gerō},
\english{so many years have I been waging war};
\apud{Aen.}{1, 47}.

\latin{haec magnam partem aestātis faciēbant},
\english{this they were engaged in doing during a large part of the summer};
\apud{B.~G.}{3, 12, 5}.\footnote{This construction of \latin{partem}
  should be distinguished from that of~\xref{388}.}

\latin{quīnque et vīgintī nātus annōs},
\english{twenty-five years old}
(having been born twenty-five years);
\apud{Tusc.}{5, 20, 57}.

\end{examples}

\begin{minor}

\subsubsection
But \latin{per} is sometimes used of Duration of Time, as in
\latin{per hōsce annōs}, \english{through \emph{(during)} all these
  years}; \apud{Cat.}{2, 4, 7}.

\subsubsection

With \latin{abhinc}, \english{ago}, either the Accusative of Duration
of Time or the Ablative of the Degree of Difference (\xref{424}) may
be used.  Thus \latin{abhinc triennium} and \latin{abhinc annīs XV}
are used almost side by side in \apud{Rosc.\ Com.}{13, 37}
(\english{ago \textsc{to the extent of} three years}, and \english{ago
  \textsc{by the amount of} fifteen years}).

\subsubsection

For the occasional Ablative of Duration of Time, see \xref{440}.

\end{minor}

\pagebreak

III. \emph{Degree}.\footnote{So especially \latin{quid},
  \latin{aliquid}, \latin{aliquantum}, \latin{quicquam},
  \latin{multum}, \latin{plūs}, \latin{plūrimum}, \latin{tantum},
  \latin{quantum}, \latin{nihil}.\versionB*{ The same use appears with
    \latin{ecquid}, \latin{sī quid}, and \latin{nē quid}.}}
\begin{examples}

\latin{quid in bellō possent},
\english{how strong they were in war}
(to what extent they were powerful);
\apud{B.~G.}{2, 4, 1}.

\latin{multum sunt in vēnātiōnibus},
\english{they are occupied to a large extent in hunting};
\apud{B.~G.}{4, 1, 8}.

\end{examples}

\headingE{Accusative of Respect}

\section

In Ciceronian prose the Accusative of Respect is confined to
\latin{vicem} and \latin{partem} with modifiers, and \latin{quid},
\english{in what respect}.
\begin{examples}

\latin{et meam et aliōrum vicem pertimēscere},
\english{to fear both for myself and for others}
(as touching my part and that of others);
\apud{Dom.}{S.~4, 8}.

\latin{et meam partem tacēre, quom (= cum) aliēnast ōrātiō},
\english{and to keep silent on my side, when it is another man’s turn to talk};
\apud{Mil.\ Gl.}{646}.

\latin{quid hoc differt?}
\english{in what respect does this differ?}
\apud{Caecin.}{14, 39}.

\end{examples}

\begin{minor}

\subsubsection

In early Latin, the Neuter Accusative of several Pronouns (\latin{id},
\latin{istuc}, \latin{aliud}, \latin{quod}, etc.)\ is still freely
used as an Accusative of Respect.
\begin{examples}

\latin{id maesta est},
\english{that’s what she’s sad about}
(she is sad with regard to that);
\apud{Rud.}{397}.

\latin{id nōs ad tē vēnimus},
\english{that’s why we came to you}
(we came about this);
\apud{Mil.\ Gl.}{1158}.

\latin{quid vēnistī?}
\english{why did you come?}
(with reference to what?);
\apud{Amph.}{377}.

\end{examples}

\begin{note}

\versionA{From such combinations }\versionB*{Hence }%
arose the\versionA{ free} use of \latin{quid} in
  the sense of \english{why}%
\versionA*{, as in \latin{quid tacēs?} \english{why
    are you silent?} \apud{Cat.}{1, 4, 8}}%
\versionB*{, and of \latin{quod} in phrases like \latin{quod sī},
  \english{but if} (touching which matter, if)}.

\end{note}

\subsubsection

The indeclinable modifiers \latin{id temporis}, \english{at that
  time}, and \latin{id} (\latin{hoc}, etc.) \latin{aetātis},
\english{of that age}, are used like adverbs and adjectives
respectively (originally Accusatives of Respect).
\begin{examples}

\latin{quōs id temporis ventūrōs esse praedīxeram},
\english{who I had said would come at that time};
\apud{Cat.}{1, 4, 10}.

\latin{cum id aetātis fīliō},
\english{with a son of that age};
\apud{Clu.}{51, 141}.

\end{examples}

\end{minor}

\section

Under the influence of Greek literature, in which the Accusative of
Respect always remained common, the later Roman writers revived its
use in some degree, employing it especially with words expressing
\emph{birth}, \emph{mind}, or \emph{parts of the body}.
\begin{examples}

\latin{Crēssa genus},
\english{a Cretan in respect of birth};
\apud{Aen.}{5, 285}.

\latin{clārī genus},
\english{men illustrious of race};
\apud{Tac.\ Ann.}{6, 9}.

\latin{mentem pressus},
\english{o’erwhelmed in mind};
\apud{Aen.}{3, 47}.

\latin{nūda genū},
\english{with bared knee}
(bare as to the knee);
\apud{Aen.}{1, 320}.

\latin{adversum femur ictus},
\english{hit in the front of the thigh};
\apud{Liv.}{21, 7, 10}.

\end{examples}

\pagebreak

\begin{minor}

\subsubsection

The later writers use the construction also with \latin{cūncta},
\latin{omnia}, \latin{alia}, \latin{rēliqua}, \latin{cētera},
\latin{plēraque}, and with \latin{frontem}, \latin{terga},
\latin{latus} (\emph{front}, \emph{rear}, and \emph{flank}).

\begin{examples}

\latin{cētera Graius},
\english{in other respects a Greek};
\apud{Aen.}{3, 594}.

\latin{iuvenem alia clārum},
\english{a youth famous in other respects};
\apud{Tac.\ Ann.}{12, 3}.

\end{examples}

\end{minor}

\headingE{Accusative of the Direct Object, and its Derivatives}

\headingC{Accusative of the Direct Object}

\section

The \emph{Direct Object of a Transitive Verb} is put in the
Accusative.
\begin{examples}

\latin{duās legiōnēs cōnscrīpsit},
\english{enrolled two legions};
\apud{B.~G.}{2, 2, 1}.

\latin{Rēmōs cohortātus},
\english{after encouraging the Remi};
\apud{B.~G.}{2, 5, 1}.

\end{examples}

\subsubsection

Impersonal Verbs, if Transitive, take the Accusative of the Direct
Object, like any other Transitive Verb.  Thus \latin{decet},
\english{it becomes}, \latin{iuvat} and \latin{dēlectat}, \english{it
  pleases}, \latin{fallit}, \latin{fugit}, and \latin{praeterit},
\english{it escapes}.

Similarly \latin{miseret}, \latin{paenitet}, \latin{piget},
\latin{pudet}, \latin{taedet}.
\begin{examples}

\latin{sī vōs paenitet}
(if it repenteth you),
\english{if you repent};
\apud{B.~C.}{2, 32, 14}.

\latin{nisi mē fallit},
\english{unless I am deceived};
\apud{Sest.}{50, 106}.

\end{examples}

\begin{minor}

\subsubsection

The poets often attach an Object to a \emph{passive form used reflexively}
(\xref[3]{288}), and even to a \emph{true passive}.\footnote{The
  Accusative with the true passive is very close \emph{in feeling} to
  the Accusative of Respect (cf.~\xref{388}).}
\begin{examples}

\latin{galeam induitur},
\english{puts on the helmet};
\apud{Aen.}{2, 392}.
Cf.\ \latin{galeam induit}, \apud{Aen.}{9, 366}.

\latin{tūnsae pectora},
\english{beating their breasts};
\apud{Aen.}{1, 481}.

\latin{manūs post terga revīnctum},
\english{with his hands bound behind his back};
\apud{Aen.}{2, 57}.
(True passive.)

\end{examples}

\subsubsection

The Subject of a dependent clause is sometimes \emph{attracted into
  the main clause}, becoming the Object of its Verb.
\begin{examples}

\latin{nōstī Mārcellum, quam tardus sit},
\english{you know Marcellus, how slow he is};
\apud{Cael., Fam.}{8, 10, 3}.

\end{examples}

\begin{note}

Corresponding passive constructions also occur, and various other
turns of expression.
\begin{examples}

\latin{quīdam perspiciuntur quam sint levēs}
(some are found how inconstant they are),
\english{we find how inconstant some are};
\apud{Am.}{17, 63}.

\end{examples}

\end{note}

\end{minor}

\section
\subsection

Several Verbs which also have an Intransitive use may be used
\emph{Transitively}, with an Accusative:

So especially \latin{taceō}, \latin{maneō}, and the Verbs of Feeling
\latin{dēspērō}, \latin{doleō}, \latin{fleō}, \latin{gemō},
\latin{queror}, \latin{horreō}, \latin{lūgeō}, \latin{maereō},
\latin{rīdeō}, and \latin{sitiō}.
\begin{examples}

\latin{multa tacuī},
\english{many things I have passed by in silence};
\apud{Cat.}{4, 1, 2}.

\latin{honōrēs quōs dēspērant},
\english{the honors of which they despair};
\apud{Cat.}{2, 9, 19}.

\end{examples}

\begin{minor}

\subsubsection

So also, rarely, \latin{iurō}, \english{swear by}.  Thus \latin{maria
  aspera iūrō}, \apud{Aen.}{6, 351}.

\subsubsection

The poets and later prose writers extend the list.  Thus
\latin{ārdeō}, \english{love passionately}, \latin{pereō}, \english{be
  dead in love with}, \latin{paveō}, \english{shudder at},
\latin{lateō}, \english{escape the knowledge of}, \latin{cēnō},
\english{dine upon}.
\begin{examples}

\latin{ārdēbat Alexim},
\english{passionately loved Alexis};
\apud{Ecl.}{2, 1}.

\latin{eārum alteram perit},
\english{he is dead in love with one of them};
\apud{Poen.}{1095}.

\latin{nec latuēre dolī frātrem Iūnōnis},
\english{nor did Juno’s wiles escape her brother};
\apud{Aen.}{1, 130}.

\end{examples}

\end{minor}

\subsection

A compound made up of an Intransitive Verb and a Preposition may,
\emph{as a whole}, have Transitive force, and so take an
Accusative.\footnote{So especially (out of a large list) \latin{adeō},
  \latin{adscendō}, \latin{adfor}, \latin{adorior} and
  \latin{adgredior}, \latin{circumveniō}, \latin{circumsistō} and
  \latin{circumstō}, \latin{circumeō}, \latin{conveniō}
  (\english{visit}), \latin{increpō} and \latin{increpitō},
  \latin{ineō}, \latin{inrumpō}, \latin{inveniō}, \latin{obeō},
  \latin{obsideō}, \latin{oppugnō}, \latin{peragrō}, \latin{praestō}
  (\english{show}, \english{perform}), \latin{praetereō},
  \latin{subeō}, \latin{subterfugiō}, \latin{trāiciō}
  (\english{pierce}), \latin{trānseō}, \latin{trānsgredior}.  Passives
  also occur, e.g.\ \latin{circumvenīrētur},
  \apud{B.~G.}{1, 42, \emend{176}{4}{5}},
  \latin{obsessīs}, \apud{B.~G.}{3, \emend{177}{24}{22}, 2}.
  Other compounds, not so
  used in Ciceronian prose, are found with an Accusative in poets and
  later prose writers.  Thus \latin{accēdō}, \latin{ērumpō},
  \latin{ēvādō}, \latin{innō}, \latin{interluō}, \latin{invādō},
  \latin{praenatō}, \latin{praevertor}, \latin{superēmineō}.}

These Prepositions are \latin{ad}, \latin{ante}, \latin{circum},
\latin{con}, \latin{in}, \latin{ob}, \latin{per}, \latin{prae},
\latin{praeter}, \latin{sub}, \latin{subter}, \latin{super},
\latin{trāns}.
\begin{examples}

\latin{omnia obīre},
\english{to accomplish everything};
\apud{B.~G.}{5, 33, 3}.

\latin{officium praestiterō},
\english{I shall perform my duty};
\apud{B.~G.}{4, 25, 3}.

\latin{eōs adgressus},
\english{attacking them};
\apud{B.~G.}{1, 12, 3}.

\latin{flūmen trānsgressī},
\english{having crossed the river};
\apud{B.~G.}{2, \emend{108}{19}{18}, 4}.

\end{examples}

\begin{minor}

\subsubsection

Several compounds similarly formed\footnote{Especially
  \latin{antecēdō}, \latin{anteeō}, \latin{invādō}, \latin{praecurrō}.
  Similarly, in later Latin, \latin{incēdō}, \latin{interfluō},
  \latin{interiaceō}, \latin{interveniō}, \latin{praestō}
  (\english{surpass}), \latin{succēdō}, \english{approach}, and many
  others.} take either the Accusative or the Dative (\xref{376}).
Thus \latin{antecēdō} (go before), \english{surpass}, governs the
Accusative in \latin{cēterōs antecēdunt}, \apud{B.~G.}{3, 8, 1}, and
the Dative in \latin{pecudibus antecēdat}, \apud{Off.}{1, 30, 105}.

\end{minor}

\subsection

A few phrases made up of a \emph{Noun and a Verb} may as a whole have
Transitive force, and so take an Accusative.  Thus \latin{animum
  advertō} (turn the mind upon), \english{notice}.  (In the Passive
the Accusative \latin{animum} remains.)
\begin{examples}

\latin{postquam id animum advertit},
\english{upon noticing this};
\apud{B.~G.}{1, 24, 1}.

\latin{quā rē animum adversā},
\english{when this fact was noticed};
\apud{B.~C.}{1, 80, 4}.

\end{examples}

\subsection

Intransitive Verbs of Motion are sometimes used with Transitive force.
So \latin{ambulō}, \latin{nāvigō}, and, in poetry, \latin{currō},
\latin{eō}, \latin{errō}, \latin{fugiō} (rarely also in prose), and
even passives like \latin{vehor}.
\begin{examples}

\latin{ventīs maria omnia vectī},
\english{swept by the winds o’er every sea};
\apud{Aen.}{1, 524}.

\end{examples}

\headingC{Two Objects}

\section

Verbs of \emph{making}, \emph{choosing}, \emph{having},
\emph{regarding}, \emph{calling}, or \emph{showing} may take two
Objects.\footnote{Thus (\emph{making}) \latin{faciō}, \latin{creō},
  \latin{reddō}, \latin{redigō}; (\emph{choosing} or \emph{deputing})
  \latin{ēligō}, \latin{lēgō}; (\emph{having}) \latin{habeō};
  (\emph{regarding}) \latin{habeō}, \latin{dūcō}, \latin{putō},
  \latin{exīstimō}, \latin{iūdicō}, \latin{cēnseō}; (\emph{calling})
  \latin{apellō}, \latin{nōminō}, \latin{dīcō}, \latin{vocō};
  (\emph{showing}) \latin{praebeō}, \latin{praestō}; similarly verbs
  like \latin{profiteor}, \latin{adscīscō}, \latin{sūmō}, etc., which
  \emph{involve} one of the meanings given above.} The
Second\footnote{“First Object” means \emph{principal} object, and
  “Second Object” means \emph{secondary} object, without regard to
  their order in the sentence.} may be either a Noun or an Adjective.
\begin{examples}

\latin{cōnsulēs creat L.\ Papīrium L.\ Semprōnium},
\english{appointed Lucius Papirius and Lucius Sempronius consuls};
\apud{Liv.}{4, 7, 10}.

\latin{illī mē comitem mīsit},
\english{sent me as companion for him};
\apud{Aen.}{2, 86}.

\latin{mē sevērum praebeō},
\english{I show myself unrelenting};
\apud{Cat.}{4, 6, 12}.

\end{examples}

\subsubsection

The Second Object is really in a kind of \emph{predicative} relation
(“makes \emph{to be}”), and may therefore be called a Predicate
Accusative.

\subsubsection

In the Passive construction, the First Object of the Active Voice
becomes the Subject, and the Second Object becomes the Predicate.
\begin{examples}

\latin{cōnsulēs creantur Iūlius Caesar et P.\ Servīlius},
\english{Julius Caesar and Publius Servilius are appointed consuls};
\apud{B.~C.}{3, 1, 1}.

\end{examples}

\section

Many Verbs of \emph{inquiring}, \emph{requesting}, \emph{teaching}, or
\emph{concealing}\footnote{Thus (\emph{inquiring}) \latin{interrogō},
  \latin{rogō}; (\emph{requesting}) \latin{rogō}, \latin{poscō},
  \latin{reposcō}, \latin{ōrō}, \latin{postulō}, \latin{flāgitō};
  (\emph{teaching}) \latin{doceō}; (\emph{concealing}) \latin{cēlō}.
  Also, in poetry and Later Latin, \latin{percontor}, \english{inquire
    strictly}.} may take two Objects, one of the Person, the other of
the Thing,
\begin{examples}

\latin{hōs sententiam rogō},
\english{I ask them their opinion};
\apud{Cat.}{1, 4, 9}.

\latin{iter omnīs cēlat},
\english{he conceals his route from everybody};
\apud{Nep.\ Eum.}{8, 7}.

\end{examples}

\subsubsection

In the Passive construction, the Person becomes the Subject, but the
Accusative of the Thing remains.
\begin{examples}

\latin{sententiam rogātus},
\english{having been asked his opinion};
\apud{Sall.\ Cat.}{50, 4}.

\latin{nōsne hoc cēlātōs tam diū!}
\english{the idea of our having been kept so long in the dark about this!}
\apud{Hec.}{\emend{26}{645}{646}}.

\end{examples}

\subsubsection

Other turns of expression also occur. Thus:
\begin{enumerate}

\item
\latin{Interrogō}, \latin{doceō}, and \latin{cēlō} may take \latin{dē}
of the Thing (“about,” “concerning”).
\begin{examples}

\latin{tē dē causā rogābō},
\english{I shall ask you about the case};
\apud{Vat.}{16, 40}.

\end{examples}

\item
\latin{Flāgitō}, \latin{poscō}, and \latin{postulō} may take
\latin{ab} of the Person asked (English “of”). \latin{Postulō}
\emph{generally} does so.
\begin{examples}

\latin{quod ā mē optimī cīvēs flāgitābant},
\english{which the best citizens were demanding of me};
\apud{Sest.}{17, 39}.

\end{examples}

\end{enumerate}

\subsubsection

\latin{Petō} takes \emph{only} \latin{ab} of the Person asked.
\latin{Quaerō} takes \emph{only} \latin{ab}, \latin{dē}, or \latin{ex}
of the Person asked, or the Accusative or \latin{dē} of the Thing
asked about.
\begin{examples}

\latin{causam quaerō},
\english{I ask the reason};
\apud{Leg.\ Agr.}{3, 3, 12}.

\latin{sīn dē causā quaeritis},
\english{but if you ask about the case};
\apud{Caecin.}{36, 104}.

\latin{haec cum ā Caesare peteret},
\english{when he asked this of Caesar};
\apud{B.~G.}{1, 20, 5}.

\latin{quōrum dē mōribus cum quaereret},
\english{on asking about their customs};
\apud{B.~G.}{2, \emend{178}{15}{14}, 3}.

\end{examples}

\headingC{Accusative of the Result Produced}

\section

The \emph{Result Produced} by the action of the Verb may be
expressed by the Accusative.
\begin{examples}

\latin{scrībere versūs},
\english{to write verses};
\apud{Sat.}{1, 9, 23}.

\latin{rumpit vōcem},
\english{breaks into utterance};
\apud{Aen.}{2, 129}.

\end{examples}

\headingC{Accusative in Apposition to a Sentence}

\section

An Accusative may stand in Apposition to a \emph{sentence as a whole}.
\begin{examples}

\latin{audītā mūtātiōne prīncipis immittere latrōnum globōs,
  exscindere castella, cau\-sās bellō},
\english{upon hearing of the change of emperor he sent in bands of
  brigands, and razed forts,—grounds for declaring war};
\apud{Tac.\ Ann.}{2, 64}.

\end{examples}

\begin{minor}

\subsubsection

The construction is probably an extension of that of~\xref{394}.

\end{minor}

\headingC{Accusative of Kindred Meaning\protect\footnotemark}

\footnotetext{\label{ftn:209:1}Also called the Cognate Accusative.}

\section
\subsection

An Intransitive Verb may take an Accusative Noun with a \emph{meaning
  kindred to its own}.
\begin{examples}

\latin{longam īre viam},
\english{be going a long journey};
\apud{Aen.}{4, 467}.

\latin{vīvere eam vītam},
\english{to live that life};
\apud{Sen.}{21, 77}.

\end{examples}

\subsection

\textbf{Extended Use of the Accusative of Kindred Meaning.}  An
Intransitive Verb may take an Accusative which, though not of a
meaning kindred to its own, \emph{modifies the idea of such a
  meaning}.

This Accusative may be a Noun, a Pronoun, or an Adjective.
\begin{examples}

\latin{quī Bacchānālia vīvunt},
\english{who live Bacchanalian lives};
\apud{Iuv.}{2, 3}.

\latin{pauca querar},
\english{I shall make a few complaints};
\apud{Phil.}{1, 4, 11}.
Cf.\ \apud{Aen.}{1, 385}.

\latin{poētīs pingue quiddam sonantibus atque peregrīnum},
\english{to poets having a certain heavy and foreign style};
\apud{Arch.}{10, 26}.

\latin{quae hominēs arant},
\english{men’s ploughing}
(the ploughing that men do);
\apud{Sall.\ Cat.}{2, 7}.

\end{examples}

\begin{minor}

\subsubsection
The poets like to make bold combinations of phrase.
\begin{examples}

\latin{nec mortāle sonāns},
\english{not sounding like a mortal};
\apud{Aen.}{6, 50}.

\latin{vōx hominem sonat},
\english{the voice sounds human};
\apud{Aen.}{1, 328}.

\latin{acerba tuēns},
\english{with savage looks}
(looking savage looks);
\apud{Aen.}{9, 794}.

\latin{dulce rīdentem},
\english{sweetly smiling};
\apud{Carm.}{1, 22, 23}.\footnote{\label{ftn:209:2}In such examples
  with neuter adjectives, the Accusative is in effect
  \emph{adverbial}.}

\end{examples}

\subsubsection

The construction may be used in poetry with the true Passive and with
a Passive form used Reflexively (\xref[2 and 3]{288}).
\begin{examples}

\latin{corōnārī Olympia},
\english{be crowned with the Olympic crown};
\apud{Ep.}{1, 1, 50}.

\latin{Satyrum movētur},
\english{dances the Satyr dances};
\apud{Ep.}{2, 2, 125}.

\end{examples}

\end{minor}

\headingC{Freer Neuter Accusative Modifiers}

\section

Neuter Accusatives of Pronouns and of several Adjectives may be used
to modify Verbs which do not take the Accusative of a
Noun.\footnotemark[\thefootnote] So especially with:

\subsection

Several Verbs of \emph{advising}, \emph{urging}, \emph{compelling}, or
\emph{accusing}.  Thus with \latin{mo\-ne\-ō} and its compounds,
\latin{hortor}, \latin{iubeō}, \latin{volō}, \latin{arguō},
\latin{accūsō} and \latin{incūsō}, \latin{cogō}, and \latin{addūcō}.
\begin{examples}

\latin{quod tē iam dūdum hortor},
\english{which I have long been urging \(upon\) you};
\apud{Cat.}{1, 5, 12}.

\latin{sī quid ille sē velit},
\english{if Caesar wanted anything of him};
\apud{B.~G.}{1, 34, 2}.

\latin{eōs hoc moneō},
\english{I give them this advice}
(advise them this);
\apud{Cat.}{2, 9, 20}.

\latin{id cōgit omnīs},
\english{forces everybody to this};
\apud{Rep.}{1, 2, 3}.

\end{examples}

\begin{minor}

\subsubsection

In the passive voice, the Accusative of the Thing remains.
\begin{examples}

\latin{ego hoc cōgor},
\english{I am forced to this};
\apud{Rab.\ Post.}{7, 17}.

\latin{illud addūcī vix possum, ut\dots},
\english{I can hardly be forced to the conclusion that};
\apud{Fin.}{1, 5,~14}.

\end{examples}

\end{minor}

\subsection

Several Verbs of \emph{assenting}, \emph{boasting}, \emph{contending},
\emph{striving}, or \emph{rejoicing}.\linebreak
Thus with \latin{adsentior},
\latin{gaudeō}, \latin{glōrior}, \latin{laetor}, \latin{pugnō},
\latin{studeō}.
\begin{examples}

\latin{ūnum studētis},
\english{you have one common aim};
\apud{Phil.}{6, 7, 18}.

\latin{illud nōn adsentior tibi},
\english{I do not agree with you in this};
\apud{Rep.}{3, 35, 47}.

\latin{id pugnat},
\english{contends for this};
\apud{Phil.}{8, 3, 8}.

\end{examples}

\headingC{Accusative as Subject of an Infinitive}

\section

The \emph{Subject of an Infinitive} is put in the Accusative.
\begin{examples}

\latin{līberōs ad sē addūcī iussit},
\english{ordered the children to be brought to him};
\apud{B.~G.}{2, 5, 1}.

\latin{nūntiāvērunt manūs cōgī},
\english{brought word that bands of men were gathering};
\apud{B.~G.}{2, 2,~4}.

\end{examples}

\begin{minor}

\subsubsection

The \emph{Historical} Infinitive has a Nominative Subject
(\xref{595}).

\end{minor}

\headingC{Accusative of Exclamation}

\section

The Accusative is often used in \emph{Exclamations}, to express the
Object of Feeling.
\begin{examples}

\latin{ō tempora, ō mōrēs!}
\english{O the times! O the ways of men!}
\apud{Cat.}{1, 1, 2}.

\latin{mē miseram!}
\english{wretched woman that I am!}
\apud{Eun.}{197}.

\latin{quō mihi fortūnam!}
\english{what’s the use of fortune to me!}
\apud{Ep.}{1, 5, 12}.

\end{examples}

\begin{minor}

\subsubsection

The Nominative is occasionally used in Exclamations.
\begin{examples}

\latin{ō fēstus diēs!}
\english{O joyful day!}
\apud{Eun.}{\emend{27}{560}{559}}.

\latin{ō frūstrā susceptī labōrēs!}
\english{O toils performed in vain!}
\apud{Mil.}{34, 94}.

\end{examples}

\end{minor}

\chapter{The Vocative}

\contentsentry{C}{Uses of the Vocative}

\vskip-\smallskipamount

\headingC{Vocative of Address}

\section

The \emph{Person} or \emph{Thing Addressed} is put in the Vocative.
\begin{examples}

\latin{quō usque abūtēre, Catilīna, patientiā nostrā?}
\english{how long, Catiline, shall you abuse our patience?}
\apud{Cat.}{1, 1, 1}.

\end{examples}

\begin{minor}

\section

In poetry and ceremonious prose, the Nominative is sometimes used
instead of the Vocative, or as an Appositive or Predicate to a
Vocative.
\begin{examples}

\latin{audī tū, populus Albānus},
\english{hear, people of Alba};
\apud{Liv.}{1, 24, 7}.

\latin{nāte, meae vīrēs, mea magna potentia sōlus},
\english{O son, my strength, my great power, thou alone};
\apud{Aen.}{1, 664}.

\latin{salvē, prīmus omnium parēns patriae appellāte},
\english{hail thou, named first of all the father of thy country};
\apud{Plin.\ N.~H.}{7, \emend{28}{117}{51}}.

\end{examples}

\end{minor}

\vskip-\smallskipamount

\chapter{The Ablative}

\contentsentry{C}{Uses of the Ablative}

\begin{minor}

\section[\textsc{\small Introductory}]
\subsection

The Latin Ablative inherited (\xref[2]{334}) three forces from the
parent speech, those of (1)~Separation (Separative Ablative, or
\emph{from}-case), (2)~Association (Sociative Ablative, or
\emph{with}-case), (3)~Location (Locative Ablative, or
\emph{in}-case).

\subsection
These three forces gave rise to a number of constructions, most of
which correspond fairly closely to our constructions with
\english{from}, \english{with}, or \english{in}. In addition, several
constructions arose through Fusion (\xref[3]{315}).

\end{minor}

\section

The Latin Ablative expresses four general classes of ideas:
\begin{enumI}

\item
\emph{Separation} (\emph{Separative Ablative}).

\item
\emph{Association} (\emph{Sociative Ablative}).

\item
\emph{Location} (\emph{Locative Ablative}).

\item
\emph{Various ideas, in constructions of Composite
Origin} (\emph{Fusion}).

\end{enumI}

\section
\subtitle{\textsc{Synopsis of the Principal Uses of the Ablative}}

\begin{synopsis}

\a{I}{Separative Ablative}

\b{}{Ablative with the Separative Prepositions \latin{ab}, \latin{dē},
\latin{ex}, \latin{sine} (\xref{405}).  Note especially:}

\c{\xref[1]{406}}{Agent of the Passive voice, with \latin{ab}}

\c{\xref[2]{406}}{Point of View from Which, with \latin{ab} or \latin{ex}}

\c{\xref[4]{406}}{Material of Which a thing is made, with \latin{ex}}

\c{\xref{409}}{Regular expression of the Place Whence}

\b{\xref{407}}{Ablative with the Prepositions \latin{cōram}, \latin{palam},
\latin{prae}, \latin{prō}}

\b{\xref{408}}{Ablative with Verbs of Separation}

\b{\xref{411}}{Ablative with Adjectives of Separation}

\b{\xref{412}}{Ablative with Verbs and Adjectives of Difference or Aversion}

\b{\xref{413}}{Ablative of Parentage or Origin}

\b{\xref{414}}{Ablative of Accordance}

\b{\xref{415}}{Ablative of the Standard}

\b{\xref{416}, \xref{417}}{Ablative with a Comparative}

\medskip

\a{II}{Sociative Ablative}

\b{\xref{418}, \xref{419}}{Ablative with the Sociative Preposition \latin{cum}}

\b{\xref{420}}{Ablative of Accompaniment, with or without \latin{cum}}

\b{\xref{421}}{Ablative Absolute}

\b{\xref{422}}{Ablative of Attendant Circumstances}

\b{\xref{423}}{Ablative of Means or Instrument}

\c{\xref{424}}{Ablative of the Degree of Difference}

\c{\xref{425}}{Ablative of Plenty or Want}

\c{\xref{426}}{Ablative of the Route}

\c{\xref{427}}{Ablative of Price or Value}

\d{\xref{428}}{Ablative of the Penalty or Fine}

\c{\xref{429}}{Ablative of the Object, with \latin{ūtor}, \latin{fruor},
    \latin{fungor}, \latin{potior}, \latin{vēscor}}

\d{\xref{430}}{Ablative with \latin{opus est} and \latin{ūsus est}}

\b{\xref{431}}{Ablative with Verbs of \english{exchanging}, \english{mixing},
\english{accustoming}, or \english{joining}}
\c{\xref{432}}{with
\latin{frētus}, \latin{contineor}, \latin{comitātus},
\latin{stīpātus}}

\medskip

\a{III}{Locative Ablative}

\b{\xref{433}}{Ablative with \latin{in}, \latin{sub}, etc. (Regular expression of the
Place Where)}

\b{\xref{436}}{Ablative of certain words with or without a Preposition}

\b{\xref{437}}{Ablative with \latin{fīdō} and \latin{cōnfīdō}}
\c{\xref{438}}{with \latin{nītor}, \latin{innixus}, \latin{subnīxus},
\latin{adquiēsō}, \latin{stō}, \latin{cōnstō},
\latin{cōnsistō}, \latin{contentus}}

\medskip

\a{IV}{Of Composite Origin}

\b{\xref{439}}{Ablative of the Time at or within Which}

\c{\xref{440}}{Rarer Ablative of Duration of Time}

\b{\xref{441}}{Ablative of Respect}

\c{\xref{442}}{Ablative with \latin{dignus} and \latin{indignus}}

\b{\xref{443}}{Descriptive Ablative}

\b{\xref{444}}{Ablative of Cause or Reason}

\b{\xref{445}}{Ablative of the Way or Manner}

\b{\xref{446}}{Ablative with Verbs meaning \english{carry}, \english{hold},
\english{keep}, \english{receive}, etc.}

\end{synopsis}

\negbigskip
\negbigskip

\headingE{The Separative Ablative}

\vskip-1.5\bigskipamount

\headingC{Ablative with Separative Prepositions\protect\footnotemark}

\vskip-\bigskipamount

\footnotetext{For summarized statements for all prepositions, see
  \xref{455}–\xref{458}.}

\section

The Ablative is always used with the Separative Prepositions
\latin{ā}, \latin{ab} or \latin{abs}, \latin{dē}, \latin{ē} or
\latin{ex}, \latin{sine}.
\begin{examples}

\latin{iter ab Ararī āverterant},
\english{had turned away from the Arar};
\apud{B.~G.}{1, 16, 3}.

\latin{ab initiō},
\english{from the beginning};
\apud{Liv.}{1, 5, 5}.

\latin{dē mūrō iacta},
\english{thrown down from the wall};
\apud{B.~G.}{2, \emend{179}{32}{31}, 4}.

\latin{sine exercitū},
\english{without an army};
\apud{B.~G.}{1, 34, 3}.

\end{examples}

\subsubsection

\latin{Ā}, \latin{ab}, \latin{abs}.—\latin{Ab} is used before vowels
and~\phone{h}, \latin{ā} before consonants.  But before most
consonants \latin{ab} may also be used.  \latin{Abs} is common only in
the phrase \latin{abs tē} (for which \latin{ā tē} is also frequent).
\begin{examples}

\latin{ab Aquītānīs}, \apud{B.~G.}{1, 1, 2};
\latin{ā Belgīs}, \apud{}{1, 1, 2};
\latin{ā dextrō cornū}, \apud{}{1, 52, \emend{180}{1}{2}}; and also
\latin{ab decumānā portā}, \apud{}{3, \emend{181}{25}{23}, 2};
\latin{abs tē}, \apud{}{5, 30, 2}.

\end{examples}

\subsubsection

\latin{Ē}, \latin{ex}.—\latin{Ex} is used before vowels
and~\phone{h}; both \latin{ē} and \latin{ex} before consonants, but
more frequently \latin{ex}.
\begin{examples}

\latin{ex eō}, \apud{B.~G.}{1, 6, 3};
\latin{ē fīnibus}, \apud{}{1, 5, 1};
\latin{ex fīnibus}, \apud{}{4, 1, 4}.

\end{examples}

\subsubsection

\latin{Procul}, \english{far} (always with \latin{ab} in Ciceronian
prose), may take the Ablative in poetry and later prose; thus
\latin{procul negōtiīs}, \english{far from business cares};
\apud{Epod.}{2, 1}.

\section

The Separative Ablative with a Preposition is used to express a
variety of ideas.  Notice especially:

\subsection

\emph{The Agent of the Passive Voice}, with \latin{ab}
(cf.\ \apud{\emph{John}}{I, 6}, “there was a man sent from God”):
\latin{quod ab Gallīs sollicitārentur}, \english{because they were
  being urged on \emph{(from)} by the Gauls}, \apud{B.~G.}{2, 1, 3};
\latin{ab elephantīs obtrītī}, \english{trampled upon by the
  elephants}, \apud{Liv.}{21, 5, 15}.

\begin{minor}

\subsubsection

The Ablative with \latin{ab} is sometimes used with an active verb, to
\emph{suggest} the passive idea.  Thus \latin{ā tantō cecidisse virō},
\english{to have fallen \emph{(slain)} by so great a man},
\apud{Ov.\ Met.}{5, 192}.

\subsubsection

Agents are properly \emph{persons} (or \emph{animals}).  But things
may be \emph{personified}; thus \latin{superārī ab hīs virtūtibus},
\english{to be surpassed by these virtues}, \apud{Cat.}{2, 11, 25};
\latin{laesus fallācī piscis ab hāmō}, \english{the fish hurt by the
  deceitful hook}, \apud{Ov.\ Pont.}{2, 7, 9}.  Cf.\ \apud{Aen.}{3,
  533}.

\subsection

\emph{The Point of View from Which}, with \latin{ab} or \latin{ex}
(our English conception is generally that of the \emph{place
  \textsc{where}}).  Thus:
\begin{mexamples}

\latin{ā tergō, ā novissimō agmine},
etc., (from)
\english{on the rear}

\latin{ā latere},
(from) \english{on the side}

\latin{ā fronte},
(from) \english{on the front}

\latin{ex (ab) hāc parte},
(from) \english{on this side};
\latin{ex (ab) utrāque parte},
\english{on both sides}, etc., etc.

\latin{initium capit ā},
\english{begins \emph{(from)} at}, etc., etc.

\end{mexamples}

\begin{examples}

\latin{ex hāc parte pudor pugnat, illinc petulantia; hinc fidēs,
  illinc fraudātiō},
\english{on this side decency fights, on the other impudence; here
  financial faith, there robbery};
\apud{Cat.}{2, 11, 25}.
(Note the same conception in \latin{hinc}, \latin{illinc}.)

\latin{prope ā meīs aedibus},
\english{near \emph{(reckoned from)} my house};
\apud{Pis.}{11, 26}.

\latin{“ain tū tē valēre?”}
\latin{“Pol ego haud perbene ā pecūniā,”}
\english{“are you well, do you say?”}
\english{“Not so very excellently well in point of \emph{(from the
    point of view of)} money”};
\apud{Aul.}{186}.

\end{examples}

\end{minor}

\subsection

\emph{The Condition \emph{or} Situation from \emph{or} out of Which}, with
\latin{dē} or \latin{ex}: \latin{ex vinculīs causam dīcere},
\english{to plead his cause in chains}, \apud{B.~G.}{1, 4, 1} (speak
\emph{from} his position in chains); \latin{fīēs dē rhētore cōnsul},
\english{from professor, you shall become consul}, \apud{Iuv.}{7,
  197}; \latin{dē templō carcerem fierī}, \english{that a prison
  should be made out of a temple}, \apud{Phil.}{5, 7, 18}.  Compare
the expression of the Condition into Which, \xref[1]{384}, and of the
Condition in Which, \xref[1]{434}.

\subsection

\emph{The Material of Which a thing is made}, with \latin{ex} (also,
in poetry, with \latin{dē}): \latin{factae ex rōbore}, \english{made
  of oak}, \apud{B.~G.}{3, 13, 3}; \latin{pōcula ex aurō},
\english{cups of gold}, \apud{Verr.}{4, 27, 62}; \latin{fuit dē
  marmore templum}, \english{there was a temple of marble},
\apud{Aen.}{4, 457}. (Cf.\ the Genitive of Material, \xref{349}.)

\begin{minor}

\subsubsection

The poets freely omit the preposition.  Thus \latin{templa saxō
  strūcta vetustō}, \english{the temple built of ancient stone};
\apud{Aen.}{3, 84}.

\end{minor}

\section
\subsection

The Ablative is always used with the Prepositions\footnote{For
  summarized statements for all prepositions, see
  \xref{455}–\xref{458}.}  \latin{cōram}, \latin{palam}, \latin{prae},
and \latin{prō}.\footnote{The original feeling was that of
  separation.  Thus \latin{prō castrīs}, \english{in front, reckoning
    from the camp}.  \latin{Cōram}, \english{in face of}, followed the
  analogy of \latin{prō}, \english{in front of}.  \latin{Palam}
  followed that of \latin{cōram}.  \latin{Clam}, as the opposite, did
  the same.  The Accusative with \latin{clam} is due to the analogy of
  \latin{cēlō} (\xref{393}).}
\begin{examples}

\latin{legiōnēs prō castrīs cōnstituit},
\english{drew up the legions in front of the camp};
\apud{B.~G.}{4, 35, 1}.

\latin{prō profugā vēnit},
\english{came as a deserter};
\apud{B.~G.}{3, \emend{182}{18}{16}, 3}.

\latin{cūr prō istō pugnās?}
\english{why do you fight for him \emph{(in defence of him)}?}
\apud{Verr.}{4, 36, 79}.

\latin{cōram generō meō},
\english{in the presence of my son-in-law};
\apud{Pis.}{6, 12}.

\latin{palam populō},
\english{in the presence of the people};
\apud{Liv.}{6, 14, 5}.

\end{examples}

\begin{minor}

\subsubsection

\latin{Palam} is \emph{generally an adverb}, but after Cicero’s time
  occasionally a preposition.

\end{minor}

\subsection

\latin{Clam}, \english{secretly}, is regularly an adverb in Ciceronian
Latin.  In early and later Latin, it is either an adverb, or a
preposition with the Accusative (\emph{without the knowledge of}).

\subsection

\latin{Tenus}, \english{up to} (postpositive), is rare till after
Cicero.  It generally takes the Ablative, but sometimes the Genitive.
Thus \latin{capulō tenus}, \english{up to the hilt}, \apud{Aen.}{2,
  553}; \latin{genūs tenus}, \english{up to the knee}, \apud{Liv.}{44,
  40, 8}.

\begin{minor}

\subsection

\latin{Fīnī} or \latin{fīne}, \english{up to} (prepositive or
postpositive) is in rare use as a preposition, with the Ablative or
Genitive.
\begin{examples}

\latin{fīne genūs},
\english{to the knee};
\apud{Ov.\ Met.}{10, 536}.

\latin{osse fīnī},
\english{to the bone};
\apud{Men.}{859}.

\end{examples}

\end{minor}

\headingC{Ablative with Verbs of Separation}

\section

\emph{Verbs of Separation} take an Ablative.  The Preposition, if
employed, is \latin{ab}, \latin{dē}, or \latin{ex}.  The general usage
in Ciceronian prose is as follows:

\subsection

The Preposition is freely omitted\footnote{The word “omitted” should
  not be taken as implying that the preposition \emph{ought} properly
  to be present, but only as a shorter expression in place of “not
  employed.”} with Verbs of literal Separation, \emph{if themselves
  containing a separative Preposition} (\latin{ab}, \latin{dē}, or
\latin{ex}).\footnote{So with \latin{exeō}, \latin{ēgredior},
  \latin{ēmittō}, \latin{ērumpō}.  \latin{Auferō} (in which the
  preposition is disguised) regularly takes a preposition.}
\begin{examples}

\latin{castrīs ēgressī},
\english{going out from the camp};
\apud{B.~G.}{2, 11, 1}.

\latin{ē castrīs ēgressī},
\english{going out from the camp};
\apud{B.~G.}{1, 27, 4}.

\end{examples}

\subsubsection

Otherwise a Preposition is regularly used in Ciceronian
prose.\footnote{So \latin{veniō}, \latin{adveniō}, \latin{discēdō},
  \latin{prōcēdō}, \latin{proficīscor}, \latin{prōgredior},
  \latin{dīgredior}, \latin{redeō}, \latin{referō}, \latin{revertor}.}
\begin{examples}

\latin{prōcēdit ē praetōriō},
\english{comes out from the general’s quarters};
\apud{Verr.}{5, 41, 106}.

\latin{ab urbe proficīscī},
\english{to set out from the city};
\apud{B.~G.}{1, 7, 1}.

\end{examples}

\emph{Exceptions} are rare; thus \latin{oppidō fugit},
\apud{B.~C.}{3, 29, 1}; \latin{Italiā cēdit}, \apud{Att.}{9, 10, 4},
and the fixed phrase \latin{manū mittere}, as in \apud{Mil.}{19, 56}.

\subsection

The Preposition is freely omitted with Verbs expressing \emph{either}
literal or figurative Separation, \emph{if in very common use in both
  senses}.\footnote{So \latin{arceō}, \english{keep off} and
  \english{prevent}; \latin{cēdō}, \latin{dēcēdō}, and \latin{excēdō},
  \english{go from} and \english{withdraw}; \latin{moveō},
  \english{move}; \latin{pellō}, \english{drive away} and
  \english{banish}; \latin{expellō}, \english{drive out} and
  \english{remove}; \latin{solvō}, \english{loose} and \english{free};
  \latin{abeō}, \english{go away}, \english{depart from},
  \english{resign}; \latin{abstineō}, \english{hold off} and
  \english{abstain}; \latin{dēiciō}, \english{cast down};
  \latin{dēsistō}, \english{stand aside} and \english{desist};
  \latin{dēturbō} and \latin{prōturbō}, \english{push off from} and
  \english{deprive}; \latin{exclūdō}, \english{shut out} and
  \english{prevent}; \latin{interclūdō}, \english{shut off} and
  \english{cut off}; \latin{expediō}, \english{get from under foot}
  and \english{release}; \latin{prohibeō}, \english{hold off} and
  \english{prevent}.}
\begin{examples}

\latin{dē mūrō sē dēiēcērunt},
\english{leaped from the wall}
(threw themselves down from);
\apud{B.~C.}{1, 18, 3}.

\latin{mūrō dēiectī},
\english{driven down from the wall};
\apud{B.~G.}{7, 28, 1}.

\latin{nē dē honōre dēicerer},
\english{that I should not be deprived of the honor}
(driven from it);
\apud{Verr.}{A.\ Pr.\ 9, 25}.

\latin{eā spē dēiectī},
\english{deprived of this hope};
\apud{B.~G.}{1, 8, 4}.

\end{examples}

\begin{minor}

\emph{Exception}: \latin{dēfendō}, \english{fend off} and
\english{defend}, always take \latin{ab}.

\end{minor}

\subsubsection

Otherwise, a Preposition is regularly used in Ciceronian
prose.\footnote{So with \latin{abdūcō} and \latin{dēdūcō},
  \latin{āmoveō}, \latin{dēmoveō} and \latin{removeō}, \latin{āvertō},
\latin{dēmō}, \latin{dētrahō}, \latin{discēdō}, \latin{ēiciō},
\latin{prōpulsō}, \latin{repellō}, \latin{sēcernō}, \latin{tollō}.
\latin{Absum} and \latin{dēpellō} generally take a preposition
(\latin{absum} may also take a Dative of Reference, as in
\apud{B.~G.}{1, 36, 5}; \xref{366}).  A few words occur too seldom to
admit of any statement.}
\begin{examples}

\latin{ab officiō discessūrum},
\english{would depart from his duty};
\apud{B.~G.}{1, 40, 2}.

\end{examples}

\subsection

The Preposition is regularly omitted with Verbs expressing
\emph{figurative Separation only}.\footnote{Such are verbs meaning
  \english{strip}, \english{despoil}, \english{defraud},
  \english{grudge}, \english{deprive}, \english{bereave},
  \english{interdict}, \english{absolve}, \english{relieve},
  \english{free}, \english{be free}, \english{relax}, \english{leave
    off}, \english{refrain}, \english{abdicate}.  Thus \vrb{nūdō},
  \vrb{spoliō}, \vrb{exuō}, \vrb{fraudō}, \vrb{invideō}, \vrb{prīvō},
  \vrb{orbō}, \vrb{interdīcō}, \vrb{absolvō}, \vrb{levō} and
  \vrb{relevō}, \vrb{līberō}, \vrb{vacō}, \vrb{laxō} and \vrb{relaxō},
  \vrb{supersedeō}, \vrb{abdicō}.  Also, in poetry and later prose,
  \vrb{viduō}, \vrb{exhērēdō}, etc., etc.}
\begin{examples}

\latin{magistrātū sē abdicāvit},
\english{abdicated \emph{(resigned from)} his office};
\apud{Cat.}{3, 6, 15}.

\latin{proeliō supersedēre},
\english{to refrain from battle};
\apud{B.~G.}{2, 8, 1}.

\end{examples}

\noindent
\emph{Exceptions}:
\begin{enumerate*}

\item

\vrb{Temperō}, \english{refrain}, and the passive of \vrb{intermittō},
\english{leave off}, take \latin{ab}.  \vrb{Servō}, \english{rescue},
and \vrb{vindicō}, \english{deliver}, take \latin{ab} or \latin{ex}.
\vrb{Vacō}, \english{be free from}, and \vrb{laxō}, \english{loose},
either take or omit \latin{ab}.  \latin{Līberō}, \english{free},
rarely takes \latin{ex}.
\begin{examples}

\latin{temperātūrōs ab maleficiō},
\english{would refrain from mischief};
\apud{B.~G.}{1, 7, \emend{183}{5}{4}}.

\end{examples}

\item

\latin{Caveō}, \english{beware} (of), takes \latin{ab} in Ciceronian
Latin, and either \latin{ab} or the bare Ablative in early Latin.
\begin{examples}

\latin{monuērunt ā venēnō ut cavēret},
\english{warned him to beware of poison};
\apud{Fin.}{5, 22, 64}.
Cf.\ \latin{cavē malō}, \english{beware of harm}, \apud{Pers.}{835}.

\end{examples}

\end{enumerate*}

\section[Regular Expression of the Place Whence]

The Place Whence is regularly expressed in Ciceronian prose as shown
in \xref[1 and \emph{a}, 2 and \emph{a}]{408}.

\section[Remarks on the Ablative with Verbs of Separation]
\subsection

With most\linebreak
 Verbs of Separation, whether literal or figurative, a
preposition is used with words denoting \emph{persons}.
\begin{examples}

\latin{manūs ā tē abstinēre},
\english{to keep their hands off from you};
\apud{Vat.}{4, 10}.

\end{examples}

\begin{minor}

\subsection

The poets freely use the Ablative without a preposition in any
combination expressing or suggesting separation.  This is true even if
no verb is employed, and even if the word used denotes a person.
\begin{examples}

\latin{adsurgēns flūctū},
\english{rising from the wave};
\apud{Aen.}{1, 535}.

\latin{antrō lātrāns},
\english{barking from the cave};
\apud{Aen.}{6, 400}.

\latin{marītī Tyrō},
\english{suitors from Tyre};
\apud{Aen.}{4, 36}.

\latin{dēiectam coniuge tantō},
\english{robbed of so great a spouse};
\apud{Aen.}{3, 317}.

\end{examples}

\subsection

For the Place Whence with names of Towns, Small Islands, etc.,
see~\xref{451}.

\end{minor}

\headingC{Ablative with Adjectives of Separation}

\section

\emph{Adjectives of Separation} take the Ablative either with or
without~\latin{ab}.
\begin{examples}

\latin{vacua ob omnī perīculō},
\english{free from all danger};
\apud{Prov.\ Cons.}{12, 30}.

\latin{nōn vacua mortis perīculō},
\english{not free from mortal danger};
\apud{Cat.}{4, 1, 2}.

\end{examples}

\begin{minor}

\subsubsection

In Ciceronian prose, these adjectives are \latin{līber},
\english{free}, \latin{pūrus}, \english{pure}, \latin{nūdus},
\english{stripped}, \latin{orbus}, \english{bereft}, \latin{vacuus},
\english{empty}.\footnote{Also, in later poetry, \latin{cassus},
  \latin{siccus}, \latin{viduus}, and others.

  \latin{Immūnis}, \english{exempt} (in Cicero with Objective
  Genitive; \xref{354}), after Cicero takes either the Genitive or, as
  implying want, the Ablative, the latter with or without a
  preposition (e.g.\ \latin{immūnis mīlitiā}, \english{exempt from
    service}; \apud{Liv.}{1, 43, 8}).}

\subsubsection

In later poetry, and, to some extent, in later prose, the above
adjectives may also take the Genitive (\xref{348}).
\begin{examples}

\latin{sceleris pūrus},
\english{free from guilt};
\apud{Carm.}{1, 22, 1}.

\end{examples}

\end{minor}

\headingC{Ablative with Verbs and Adjectives of Difference or Aversion}

\section

\emph{Verbs of Difference} or \emph{Aversion} take the Ablative with
\latin{ab}.  \latin{Aliēnus}, \english{foreign}, may either take or
omit the Preposition.
\begin{examples}

\latin{ab eō dissentiō},
\english{I differ from him};
\apud{Pomp.}{20, 59}.

\latin{quod abhorret ā meīs mōribus},
\english{which is foreign to my ways};
\apud{Cat.}{1, 8, 20}.

\latin{aliēna ā dignitāte},
\english{inconsistent with your dignity};
\apud{Fam.}{4, 7, 1}.

\latin{aliēnum dignitāte imperī},
\english{inconsistent with the dignity of the realm};
\apud{Prov.\ Cons.}{8, 18}.

\end{examples}

\begin{minor}

\subsubsection

\latin{Alius}, \english{else}, \english{other than}, is regularly
followed by \latin{atque} (\latin{ac}), or, if negatived, by
\latin{nisi}, \latin{quam}, or \latin{praeter}; but it \emph{may} take
the Ablative without a preposition, as in \latin{alium sapiente},
\apud{Ep.}{1, 16, 20} (very rarely in prose).

\subsubsection

\latin{Aliēnus} may also take a Genitive (\xref[\emph{c}]{339}) or
  Dative (\xref[III]{362}).

\subsubsection

A few of these verbs (e.g.\ \vrb{dissentiō}) may also take the
construction of Contention (Ablative with \latin{cum}; \xref[4]{419}).

\subsubsection

The later writers freely employ the Dative with these verbs (\xref[2,
  \emph{c}]{363}).

\end{minor}

\headingC{Ablative of Parentage or Origin}

\section

\emph{Parentage} or \emph{Origin}\footnote{The verb employed in
  Ciceronian Latin is
  \latin{nāscor}.  The participles are \latin{nātus},
  \latin{prōgnātus}, \latin{ortus}; also, in later Latin,
  \latin{genitus}, \latin{generātus}, \latin{crētus}, \latin{satus},
  \latin{ēditus}, \latin{oriundus}, and others.} is expressed by the Ablative,
generally without a Preposition.
\begin{examples}

\latin{amplissimō genere nātus},
\english{born of a very noble stock};
\apud{B.~G.}{4, 12, 4}.

\latin{quō sanguine crētus?}
\english{from what blood sprung?}
\apud{Aen.}{2, 74}.

\end{examples}

\subsubsection

A preposition (generally \latin{ex}) is sometimes used with the noun,
especially if this denotes a parent.  Before a pronoun, the
preposition is regular.

\subsubsection

Remoter origin is expressed by \latin{ortus} with \latin{ab}, or
\latin{prōgnātus} with \latin{ex}.
\begin{examples}

\latin{Belgās esse ortōs ā Germānīs},
(he learned that) \english{the Belgae were descended from the
  Germans};
\apud{B.~G.}{2, 4, 1}.

\end{examples}

\headingC{Ablative of Accordance}

\section

That \emph{in Accordance with which} one acts or judges may be
expressed by the Ablative of certain words, regularly without a
Preposition.
\begin{examples}

\latin{cōnsuētūdine suā Caesar VI legiōnēs expedītās dūcēbat},
\english{according to his custom, Caesar, as he marched, kept six
  legions in fighting order};
\apud{B.~G.}{2, \emend{113}{19}{18}, 2}.

\latin{tuō cōnsiliō faciam},
\english{I will act in accordance with your plan};
\apud{Rud.}{962}.

\latin{mūnus meā sententiā magnum},
\english{a great gift, in my opinion}
(according to my way of thinking);
\apud{Off.}{3, 33, 121}.

\end{examples}

\subsubsection

So especially, in Ciceronian Latin, \latin{mōre} (\latin{mōribus}) and
\latin{cōnsuētūdine}, \english{according to custom}, \latin{cōnsiliō},
(according to) \english{with a plan}, \latin{sententiā} (\latin{meā},
etc.)\ (according to) \english{in \emph{(my, etc.)}\ opinion},
\latin{lēge}, \english{by law} (these rarely with a preposition);
\latin{iūdiciō} and \latin{animō}, (according to) \english{in the
  judgment \(of\)}, \latin{iussū} (\latin{iniussū} by analogy),
\latin{voluntāte}, \latin{rogātū}, \latin{admonitū},
\latin{arbitrātū}, or \latin{concessū}, \english{by the order},
\english{desire}, \english{request}, \english{advice},
\english{decision}, or \english{consent \(of\)}, \latin{accītū} or
\latin{missū}, \english{by the summons} or \english{sending \(of\)}
(these without a preposition).\footnote{The poets add other words.
  Thus \latin{imperiō}, \english{by the order \(of\).}

  For \latin{lēge} meaning \english{with the condition}, see
  \xref[\emph{b}]{436}.  For \latin{voluntāte}, \english{voluntarily}
  (originally Ablative of Accordance, but in effect expressing Manner,
  see~\xref{445}).}

\subsubsection

In general, Accordance is expressed by \latin{dē} or \latin{ex} with
the Ablative.
\begin{examples}

\latin{quōs ex senātūs cōnsultō convenit\dots},
\english{in accordance with which decree of the Senate it has all the
  time been proper};
\apud{Cat.}{1, 2, 4}.

\end{examples}

\headingC{Ablative of the Standard}

\section

The \emph{Standard} from which one starts in measuring or judging is
regularly expressed by the Ablative without a Preposition.
\begin{examples}

\latin{quī verbīs contrōversiās, nōn aequitāte dīiūdicās},
\english{who decide controversies according to \emph{(= by)} words,
  not according to justice};
\apud{Caecin.}{17, 49}.

\latin{magnōs hominēs virtūte mētīmur nōn fortūnā},
\english{we measure great men by their high aims, not by their luck};
\apud{Nep.\ Eum.}{1, 1}.

\end{examples}

\begin{minor}

\subsubsection

But \latin{ex} is sometimes used.  Thus \latin{amīcitiās ex commodō
  aestimāre}, \english{to judge friendships from the standard of
  advantage}; \apud{Sall.\ Cat.}{10, 5}.

\end{minor}

\headingC{Ablative with a Comparative}

\section

A \emph{Comparative Adjective} is often followed by the Ablative.

But \latin{quam} \english{may} always be used, and regularly \emph{is}
used if the first of the two things compared is in any case except the
Nominative or Accusative.
\begin{examples}

\latin{vītā cārior},
\english{dearer than life};
\apud{Cat.}{1, 11, 27}.

\latin{audācior quam Catilīnā},
\english{more overweening than Catiline};
\apud{Phil.}{2, 1, 1}.

\latin{tibi, multō maiōrī quam Āfricānus fuit},
\english{to you, a much greater man than Africanus was};
\apud{Fam.}{5, 7, 3}.

\end{examples}

\begin{minor}

\subsubsection

The relative pronoun with definite antecedent is regularly in the
Ablative after a Comparative.
\begin{examples}

\latin{Aenēās, quō iūstior alter nec pietāte fuit nec bellō maior},
\english{Aeneas, than whom no man was ever juster in piety or greater in war};
\apud{Aen.}{1, 544}.

\end{examples}

\subsubsection

Comparison may be expressed in poetry by the use of \latin{ante},
\latin{praeter}, etc.
\begin{examples}

\latin{ante aliōs immānior},
\english{more monstrous than \emph{(before)} the rest};
\apud{Aen.}{1, 347}.

\end{examples}

\subsubsection

One of the two things compared is often suppressed.
\begin{examples}

\latin{esse graviōrem fortūnam Sēquanōrum quam reliquōrum},
\english{the fate of the Sequani was harder than \emph{(the fate)} of the rest};
\apud{B.~G.}{1, 32, 4}.

\end{examples}

\end{minor}

\subsubsection

\latin{Plūs}, \latin{minus}, \latin{amplius}, and \latin{longius} may
be used as Comparative Adjectives with an Ablative, \emph{or} as
Adverbs, without effect upon the case.

\begin{examples}

\latin{utī nōn amplius quīnīs aut sēnīs mīlibus passuum interesset},
\english{so that there was not more than fix or six miles between};
\apud{B.~G.}{1, 15, \emend{90}{5}{3}}.

\latin{Sabim flūmen ā castrīs suīs nōn amplius mīlia passuum X abesse},
\english{that the Sambre was not above ten miles distant from his camp};
\apud{B.~G.}{2, \emend{184}{16}{15}, 1}.

\end{examples}

\begin{minor}

\subsubsection

Certain Ablatives are regularly used for brevity in place of clauses.
Thus \latin{aequō}, \latin{exspectātiōne}, \latin{necessāriō},
\latin{opīniōne}.\footnote{Also, in later Latin, \latin{dictō},
  \latin{fidē}, \latin{solitō}, \latin{spē}, \latin{vērō}, and other
  words.} The same usage holds with comparative adverbs.
\begin{examples}

\latin{nē plūs aequō quid in amīcitiam congerātur},
\english{lest more than \emph{(what is)} right should be heaped upon
  friendship};
\apud{Am.}{16, 58}.

\latin{longius necessāriō},
\english{farther than was necessary};
\apud{B.~G.}{7, 16, 3}.

\end{examples}

\end{minor}

\section

A \emph{Comparative Adverb} is ordinarily followed by \latin{quam}.
\begin{examples}

\latin{cum possit clārius dīcere quam ipse},
\english{though he could speak louder than \emph{(the leading
    character)} himself};
\apud{Caecil.}{15, 48}.

\end{examples}

\begin{minor}

\subsubsection

Nouns of \emph{time} are regularly in the Ablative after comparative
adverbs.
\begin{examples}

\latin{longius annō remanēre},
\english{to remain more than a year};
\apud{B.~G.}{4, 1, 7}.

\end{examples}

\subsubsection

The poets use the Ablative freely with comparative adverbs.
\begin{examples}

\latin{quam Iūnō fertur terrīs magis omnibus coluisse},
\english{which Juno is said to have fostered more than all other lands}
 (for \latin{magis quam terrās omnīs});
\apud{Aen.}{1, 15}.

\end{examples}

\end{minor}

\headingE{The Sociative Ablative}

\headingC{Ablative of Accompaniment, with
  \latin{cum}\protect\footnotemark}

\footnotetext{For summarized statements for \emph{all} prepositions,
  see \xref{455}–\xref{458}.}

\section

The Ablative is always used with the Sociative Preposition
\latin{cum}, \english{with}.

\begin{examples}

\latin{cum lēgātīs vēnit},
\english{came with the ambassadors};
\apud{B.~G.}{4, 27, 2}.

\latin{cum febrī domum rediit},
\english{came home with a fever};
\apud{De~Or.}{3, 2, 6}.

\latin{dēsinant obsidēre cum gladiīs cūriam},
\english{let them cease to invest the senate-house with swords}
(in their hands);
\apud{Cat.}{1, 13, 32}.

\end{examples}

\subsubsection

\latin{Cum} is regularly put \emph{after} a personal, reflexive, or
relative pronoun, and forms one word with it; thus \latin{mēcum},
\latin{sēcum}, \latin{quibuscum}.

\begin{minor}

\subsubsection

In poetry and later prose, \latin{simul}, \emph{together with}, is
sometimes used with the Ablative.  Thus \latin{simul hīs dictīs},
(together) \english{with these words}; \apud{Aen.}{5, 357}.

\end{minor}

\section

The Ablative with \latin{cum}, \english{with}, is used to express a
variety of ideas.  The most important are the following:

\subsection

\emph{Union}, \emph{Agreement}, or \emph{Companionship}: \latin{cum
  proximīs cīvitāt\emend{29}{ī}{i}bus pācem cōnfirmāre}, \english{to make
  peace with the neighboring states}, \apud{B.~G.}{1, 3, 1};
\latin{prūdentiam cum ēloquentiā iungere}, \english{to join prudence
  with eloquence}, \apud{Tusc.}{1, 4, 7}.

\subsection

\emph{Intercourse}, \emph{Dealing}, etc.:
\latin{is ita cum Caesare ēgit},
\english{he pleaded with Caesar as follows};
\apud{B.~G.}{1, 13, 3}.

\subsection

\emph{Comparison}:
\latin{neque hanc cōnsuētūdinem vīctūs cum illā comparandam (esse)},
\english{and that this manner of living was not to be compared with the other};
\apud{B.~G.}{1, 31, 11}.

\subsection

\emph{Contention} or \emph{Variance}:
\latin{cum Germānīs contendunt},
\english{they contend with the Germans};
\apud{B.~G.}{1, 1, 4}

\headingC{Ablative of Accompaniment, with or without \latin{cum}}

\section

In \emph{military language}, Accompaniment after Verbs of coming or
going may be expressed by an Ablative \emph{with} or \emph{without}
\latin{cum}.

But \latin{cum} \emph{must} be used if the Noun stands without a
modifier, or with a Numeral.
\begin{examples}

\latin{cum iīs cōpiīs quās ā Caesare accēperat pervēnit},
\english{arrived with the forces which he had received from Caesar};
\apud{B.~G.}{3, \emend{185}{17, 1}{15, 5}}.

\latin{eō pedestribus cōpiīs contendit},
\english{hurries thither with the infantry};
\apud{B.~G.}{3, 11, 5}.

\latin{uterque cum equitātū venīret},
\english{that both should come with cavalry}:
\apud{B.~G.}{1, 42, \emend{186}{4}{5}}.

\latin{cum hīs quīnque legiōnibus īre},
\english{to go with these five legions};
\apud{B.~G.}{1, 10, 3}.

\end{examples}

\headingC{Ablative Absolute}

\section

An Ablative Noun or Pronoun, with a Predicate word in the same case,
may be used in loose connection with the rest of the sentence.

The Predicate may be a Noun, a Participle, or an Adjective.

The Ablative Absolute is (like the English Nominative Absolute, as in
“this having been done”) strictly a \emph{neutral} construction,
telling nothing about the real relation between the facts stated in it
and the facts stated in the rest of the sentence.  In English,
however, we must ordinarily translate so as to \emph{show} these
relations.  Hence the following headings are convenient:

\subsection

\emph{\(Mere\) Time}: \latin{M.\ Messālā M.\ Pīsōne cōnsulibus},
\english{in the consulship of Marcus Messala and Marcus Piso} (Messala
and Piso being\footnote{Note that Latin has no participle
  corresponding to English “being.”} consuls); \apud{B.~G.}{1, 2,
  1}.

\subsection

\emph{\(Mere\) Situation}: \latin{ea īnscientibus ipsīs fēcisset},
\english{had done this without their knowledge} (they not knowing);
\apud{B.~G.}{1, 19, 1}.

\subsection

\emph{Situation and Time}: \latin{omnibus rēbus comparātīs diem
  dīcunt}, \english{every thing being ready, they \emph{(then)}
  appoint a day}; \apud{B.~G.}{1, 6, \emend{96}{4}{3}}.

\subsection

\emph{Situation and Cause}: \latin{mercātōribus iniūriōsius tractātīs
  bella gessērunt}, \emph{waged war when \(\emph{and} because\) our
  traders had been somewhat rudely treated}; \apud{Pomp.}{5, 11}.

\subsection

\emph{Situation and Opposition}: \latin{id paucīs dēfendentibus
  expugnāre nōn potuit}, \english{he was unable to take this \(town\),
  though its defenders were but few}; \apud{B.~G.}{2, 12, 2}.

\subsection

\emph{Condition}:
\latin{semper exīstimābitis vīvō P.\ Clōdiō nihil eōrum vōs vīsūrōs
  fuisse},
\english{you will always think that, if Publius Clodius were alive,
  you would never have seen any of these things};
\apud{Mil.}{28, 78}.

\begin{minor}

\subsubsection

\latin{Nisi}, \latin{quasi}, \latin{tamquam}, \latin{velut}, etc., may
be used.
\begin{examples}

\latin{nisi mūnītīs castrīs},
\english{unless the camp were fortified};
\apud{B.~G.}{2, \emend{112}{20, 3}{19, 2}}.

\end{examples}

\end{minor}

\subsection

\emph{Means}:
\latin{id ratibus ac lintribus iūnctīs trānsībant},
\english{were crossing this \(river\) by tying together rafts and
  boats};
\apud{B.~G.}{1, 12, 1}.

\subsection

\emph{Manner}:
\latin{incitātō equō sē hostibus obtulit},
\english{rushed upon the enemy at full speed}
(his horse being speeded);
\apud{B.~G.}{4, 12, 6}.

\begin{minor}

\subsubsection

The later writers often use an Infinitive or a Subordinate Clause as
the principal member of an Ablative Absolute; and they also often use
a Participle \emph{impersonally}.
\begin{examples}

\latin{impetrātō ut manērent},
\english{\(leave\) being obtained to remain};
\apud{Liv.}{9, 30, 10}.

\latin{lībātō},
\english{after a libation had been made};
\apud{Aen.}{1, 737}.

\end{examples}

\subsubsection

In general, the Ablative Absolute is used only where its noun or
pronoun denotes a person or thing mentioned nowhere else in the same
clause.  Yet exceptions sometimes occur, generally for the sake of
clearness or emphasis.
\begin{examples}

\latin{vōsne ego patiar cum mendīcīs nūptās mē vīvō virīs?}
\english{shall I suffer you to be the wives of beggar-men while I am
  alive?}
\apud{Stich.}{132}.
Similarly \latin{turribus excitātīs, tamen hās}, \apud{B.~G.}{3, 14, 4}.

\end{examples}

\end{minor}

\headingC{Ablative of Attendant Circumstances}

\section

An Ablative Noun with a modifier may be used to express
\emph{Situation}, \emph{Circumstances}, or \emph{Result}.

The examples fall into two main classes:

I.\enskip Expressing Situation (English “with,” “in,” “under”).  No
Preposition is used.

Thus
\latin{imperiō nostrō}, \english{under our sovereignty};
\latin{aestū magnō}, \english{in great heat};
\latin{parī} (\latin{magnō}, \latin{quō}, etc.) \latin{intervāllō},
 \english{at an equal distance};
\latin{nūllīs impedīmentīs}, \english{without baggage};
\latin{magnō comitātū}, \english{with a great retinue};
\latin{frequentissimō senātū}, \english{in a crowded meeting of the
  senate};
\latin{tantō conventū}, \latin{tantā frequentiā}, \latin{magnō
  cōn\-ses\-sū}, etc.;
\latin{hōc}, \latin{hāc}, or \latin{hīs} with various nouns.

\begin{examples}

\latin{minus facile eam rem imperiō nostrō cōnsequī poterant},
\english{could less easily attain this under our sovereignty};
\apud{B.~G.}{2, 1, 4}.

\latin{hāc cōntiōne, hōc populō nōn verērer},
\english{with an assemblage like this, with a people like this, I should have no fear};
\apud{Leg.\ Agr.}{2, 37, 101}.

\latin{dīcit frequentissimō senātū cōnsul},
\english{the consul says in a crowded meeting of the senate};
\apud{Leg.\ Agr.}{1, 8, \emend{187}{26}{101}}.
Cf.\ \apud{Arch.}{2, 3}; \apud{Mil.}{24, 66}.

\latin{ubi fidē pūblicā dīcere iussus est},
\english{when he had been invited to turn state’s evidence}
(speak with a pledge from the state);
\apud{Sall.\ Cat.}{47, 1}.

\end{examples}

\subsubsection

To this class probably belong the following Ablatives accompanied by a
Genitive of the person, or a Possessive Pronoun: \latin{ductū},
\latin{imperiō}, \latin{auspiciō}, \english{under the lead},
\english{authority}, or \english{auspices \(of\)};
\latin{contuberniō}, \english{in association \(with\)};
\latin{voluntāte} or \latin{concessū}, \english{with the approval} or
\english{consent \(of\)}.

\begin{minor}

\subsubsection

Rarely, a noun is so used \emph{without} a modifier.  Thus
\latin{intervāllō restitūta}, \english{restored after \emph{(with)} an
  interval}; \apud{Leg.\ Agr.}{1, 9, 27}.  The use is less rare in
poetry.  Thus \latin{servitiō ēnīxae}, \english{having borne a child
  in slavery}; \apud{Aen.}{3, 327}.

\subsubsection

The poets employ the construction in bold combinations.  Thus
\latin{paribus cūrīs vestīgia fīgit}, \english{wrapped in like cares
  \emph{(with like cares)} plants his footsteps}; \apud{Aen.}{6, 159}.

\end{minor}

II.\enskip Expressing Circumstances or Result (English “with” or “to”).
The Preposition, if used, is \latin{cum}.  It is

\subsection

Regularly omitted with the most common phrases.  Thus
\latin{ōminibus}, \english{with\ellipsis omens}; \latin{clāmōre},
\english{with shouting}; \latin{plausū}, \english{with applause};
\latin{convīciō}, \english{with abuse}; \latin{si\-len\-tiō}, \english{in
  \emph{(with)} silence}; \latin{pāce} or \latin{veniā}, \english{with
  the permission \(of\)}; \latin{commodō} or \latin{incommodō},
\english{with advantage} or \english{disadvantage \(to\)};
\latin{damnō} or \latin{iactūrā}, \english{with the loss \(of\)}.
\begin{examples}

\latin{hīs ōminibus, cum tuā perniciē proficīscere ad impium bellum},
\english{with these omens, and to your ruin, set out to wage your impious war};
\apud{Cat.}{1, 13, 33}.

\latin{quod commodō reī pūblicae facere posset},
\english{as far as he could do so with \emph{(resulting)} advantage to the commonwealth};
\apud{B.~G.}{1, 35, 4}.

\latin{exercitum duārum cohortium damnō dēdūcit},
\english{leads his army back with a loss of two cohorts};
\apud{B.~G.}{6, 44, 1}.

\end{examples}

\begin{minor}

\subsubsection

When used without a modifier, these words (except \latin{silentiō})
generally take \latin{cum}.  Thus \latin{cum plausū}, \apud{Phil.}{2,
  34, 85}; \latin{cum clāmōre}, \apud{Verr.}{5, 36, 93}.

\end{minor}

\subsection

Used or omitted indifferently with phrases moderately common.

Thus \latin{(cum) perīculō}, \english{with danger \(to\)};
\latin{(cum) dolōre} or \latin{aerumnā}, \english{to the grief} or
\english{sorrow \(of\)};
\latin{(cum) glōriā}, \english{to the glory \(of\)};
\latin{(cum) invidiā}, \english{to the unpopularity \(of\)}.
\begin{examples}

\latin{vīdī quantō meō dolōre},
\english{with what grief to myself did I see\dots!}
\apud{Phil.}{1, 4, 9}.
(Cf.\ \apud{Cat.}{4, 1, 2}, \english{to my grief}.)

\latin{quantō cum dolōre vīdi!}
\english{with what grief did I see\dots!}
\apud{Marc.}{6, 16}.

\end{examples}

\subsection

Regularly used with the least common phrases.
\begin{examples}

\latin{magnō cum lūctū cīvitātis simulācrum tollendum locātur},
\english{to the great grief of the state, a contract is made for the
  removal of the statue};
\apud{Verr.}{4, 34, 76}.
Similarly \latin{cum tuā perniciē} under II, 1, above.

\end{examples}

\begin{note}[Note to 1–3]

The poets and later writers vary the usage \emph{for the mere sake} of
variety.  Thus \latin{cum bonīs ōminibus}, \apud{Liv.}{Praef.\ 13};
\latin{maiōre perniciē}, \apud{}{21, 35, 1}.

\end{note}

\headingC{Ablative of Means or Instrument (Instrumental Ablative)}

\section

\emph{Means} and \emph{Instrument} may be expressed by the Ablative.
\begin{examples}

\latin{gladiīs pugnātum est},
\english{the battle was fought with swords};
\apud{B.~G.}{1, 52, 4}.

\latin{litterīs certior fīēbat},
\english{was informed by \emph{(means of)} despatches};
\apud{B.~G.}{2, 1, 1}.

\latin{id animō contemplāre quod oculīs nōn potes},
\english{contemplate in \emph{(with)} your mind what you cannot with
  your eyes};
\apud{Dei.}{14, 40}.

\latin{suō sumptō},
\english{at his own expense}
(by his expenditure);
\apud{B.~G.}{1, 18, 5}.

\latin{magnō dolōre adficiēbantur},
\english{were greatly distressed}
(were affected with great grief);
\apud{B.~G.}{1, 2, \emend{188}{4}{5}}.

\end{examples}
Similarly with \latin{ōrnātus}, \english{equipped}, \latin{praeditus},
\english{endowed}, \latin{onustus}, \english{laden}.

\subsubsection

Persons are often thought of as Means.
\begin{examples}

\latin{eā legiōne mīlitibusque quī ex prōvinciā convēnerant, mūrum perdūcit},
\english{with this legion, and the soldiers who had assembled from the province, he constructs a wall};
\apud{B.~G.}{1, 8, 1}.

\latin{iacent suīs testibus},
\english{they are overthrown by means of their own witnesses};
\apud{Mil.}{18, 47}.

\end{examples}

\subsubsection

The Ablative of Means probably appears in such phrases as \latin{quid
  illō fīet?}  \english{what will \emph{(be made with =)} become of
  him?} \apud{Att.}{6, 1, 14}; \latin{sī quid eō factum esset},
\english{if anything should happen to him}, \apud{Pomp.}{20, 59}. (By
analogy, \latin{quid tē futūrust?}  \english{what will become of you?}
\apud{Ph.}{137}, etc.)

With \latin{faciō}, the Dative of the Indirect Object (\xref{365}) may
also be used.  Thus \latin{quid huic hominī faciās?} \english{what is
  one to do with \emph{(to)} such a man?} \apud{Caecin.}{11, 30}.

\begin{minor}

\subsubsection

The poets often use the Ablative of Means to make the governing word
\emph{imply} more than it strictly says (“forced” construction).
\begin{examples}

\latin{Aeacidae tēlō iacet Hector},
\english{Hectors lies \emph{(slain)} by Achilles’ spear};
\apud{Aen.}{1, 99}.

\end{examples}

\subsubsection

Means may also be expressed by \latin{per} with the Accusative.
\begin{examples}

\latin{cōnsuluistī mē per litterās},
\english{you consulted me by letter};
\apud{Phil.}{2, 40, 102}.

\end{examples}

\end{minor}

\headingC{Ablative of Degree of Difference}

\section

\emph{Degree of Difference} is expressed by the Ablative.

The construction is freely used with Comparatives and various Adverbs,
less freely with Superlatives.
\begin{examples}

\latin{mīlibus passuum duōbus ultrā eum},
\english{two miles beyond him}
(beyond by two miles);
\apud{B.~G.}{1, 48, 2}.

\latin{paucīs ante diēbus},
\english{a few days before};
\apud{Cat.}{3, 1, 3},

\latin{quō dēlictum maius est, eō poena est tardior},
\english{the greater the fault, the slower the punishment}
(by how much greater\dots, by that much slower\dots);
\apud{Caecin.}{3, 7};
cf.\ \apud{B.~G.}{1, 14, 1}, and \apud{Pomp.}{20, 59}.

\latin{eō minus, quod memoriā tenēret},
\english{the less \emph{(so)} because he remembered\dots};
\apud{B.~G.}{1, 14,~1}.

\end{examples}

\begin{minor}

\subsubsection

In such examples as \latin{eō minus, quod\dots}, probably both Degree
of Difference (\xref{424}) and Cause (\xref{444}) were felt by the
Romans (\emph{by so much the less}, \emph{because}, and \emph{on that
  account}, namely \emph{because}).

\end{minor}

\headingC{Ablative of Plenty or Want}

\section

Certain Adjectives and Verbs of \emph{Plenty} or \emph{Want} may take
the Ablative.
\begin{examples}

\latin{erant plēna laetitiā omnia},
\english{everything was full of joy};
\apud{B.~C.}{1, 74, 7}.

\latin{montem hominibus complērī iussit},
\english{ordered the mountain to be filled with men};
\apud{B.~G.}{1, 24, 3}.

\latin{urbe ērudītissimīs hominibus adfluentī},
\english{a city overflowing with scholars};
\apud{Arch.}{3, 4}.

\latin{metū suppliciōrum carēre},
\english{to be free from the fear of penalties};
\apud{Mil.}{2, 5}.

\latin{omnibus egēre rēbus},
\english{were in want of everything};
\apud{B.~C.}{3, 32, 4}.

\end{examples}

\subsubsection

So, in Ciceronian Latin, the adjectives\footnote{Also, in later Latin,
  \latin{cōpiōsus}, \latin{crēber}, \latin{dīves}, \latin{fētus},
  \latin{frequēns}, \latin{opulentus}, and others.  Similarly
  \latin{expers} (with Genitive of Want in Ciceronian Latin;
  \xref{347}) may take the Ablative in later writers
  (\apud{Sall.\ Cat.}{33, 2}); and \latin{exhērēs} and \latin{immūnis}
  (with Objective Genitive in Ciceronian Latin) may take the Ablative
  of Want.} \latin{cōnfertus}, \latin{differtus}, \latin{refertus},
\latin{opīmus}, \latin{inānis}, rarely \latin{plēnus} and
\latin{inops}; and the verbs \latin{abundō}, \latin{redundō},
\latin{adfluō}, \latin{circumfluō}, \latin{careō}, \latin{egeō} and
its compounds, and the compounds of \suffix{-pleō}.

\subsubsection

Some of these words may also take the Genitive of Plenty or Want
(\latin{plēnus}, \latin{inops}, and \latin{indigeō} regularly;
\latin{egeō}, \latin{compleō}, and \latin{impleō} rarely).  See
\xref[\relax and~\emph{b}]{347}.

\pagebreak

\headingC{Ablative of the Route}

\section

The \emph{Route by Which} may be expressed by the Ablative of certain
words.
\begin{examples}

\latin{Aurēliā viā profectus est},
\english{he set out by the Aurelian road};
\apud{Cat.}{2, 4, 6}.

\latin{terrā Macedoniam petit},
\english{proceeded to Macedonia by land};
\apud{Liv.}{24, 40, 17}.

\latin{Padō trāiectus},
\english{crossing \emph{(by)} the Po};
\apud{Liv.}{21, 56, 9}.
Cf.\ \latin{fretō trāiēcit}, \apud{}{22, 31, 7}.

\end{examples}

\subsubsection

These words are especially \latin{colle}, \latin{flūmine},
\latin{fretō}, \latin{itinere}, \latin{iugō}, \latin{marī},
\latin{pon\-te}, \latin{portā},\footnote{With \latin{portā}
  (\latin{portīs}), this construction, not that of separation, is
  regularly used with verbs of motion (“by,” not “from”).}
\latin{terrā}, \latin{vadō}, \latin{viā}, and the plurals of most of
them.  \latin{Adversus} or \latin{secundus} is often added
(e.g.\ \apud{B.~G.}{2, \emend{110}{19}{18}, 8},
\latin{adversō colle}, by the hill
opposing, = \english{up the hill}).\footnote{Later writers use a
  larger list of words.  Thus \latin{agrō}, \latin{angustiīs},
  \latin{līmite}, \latin{fīnibus}, \latin{lītore}, \latin{ōrā}\versionA*{,
  \latin{stagnō}}, \latin{palūde}\versionB*{, \latin{pelagō}},
  \latin{rīpā}, and the names of rivers, mountains, countries, and
  city gates.}

\subsubsection

In general, the Route is expressed by \latin{per} and the Accusative,
sometimes even with the above words.
\begin{examples}

\latin{per angustiās et fīnīs Sēquanōrum suās cōpiās trādūxerant},
\english{had brought their forc\-es through the pass and the territory
  of the Sequani};
\apud{B.~G.}{1, 11, 1}.

\end{examples}

\begin{minor}

\subsubsection

The Ablative of Route often expresses the \emph{Space over Which},
closely approaching the Accusative of Extent of Space (\xref[I]{387})
in meaning.
\begin{examples}

\latin{tantō spatiō secūtī quantum efficere potuērunt},
\english{following over as large a space as they could accomplish};
\apud{B.~G.}{4, 35, 3}.

\end{examples}

\subsubsection

The poets extend the construction to other words.
\begin{examples}

\latin{āere lāpsa quiētō},
\english{gliding through the quiet air};
\apud{Aen.}{5, 216}.
Cf.\ \latin{volat per āera magnum} (the regular prose construction),
\apud{Aen.}{1, 300}.

\latin{prōspectum pelagō petit},
\english{seeks an outlook over the deep};
\apud{Aen.}{1, 181}.
Similarly \latin{altō prōspiciēns},
\english{looking out over the deep};
\apud{Aen.}{1, 126}.

\end{examples}

\end{minor}

\headingC{Ablative of Price or Value}

\section
\subsection

\emph{Definite} Price or Value \emph{must} be expressed by the
Ablative; also \emph{Indefinite} Price or Value, if the word used is a
Substantive of serious meaning.
\begin{examples}

\latin{dēnāriīs III aestimāvit},
\english{valued it at three denarii};
\apud{Verr.}{3, 92, 215}.

\latin{parvō pretiō redēmpta},
\english{bought at \emph{(with)} a low price};
\apud{B.~G.}{1, 18, 3}.

\latin{vēndidit aurō patriam},
\english{sold his country for gold};
\apud{Aen.}{6, 621}.

\end{examples}

\subsection

\emph{Indefinite} Price or Value may be expressed by \emph{either} the
Genitive (\xref{356}) \emph{or} the Ablative of:
\begin{enuma}

\item
Certain Adjectives. Thus \latin{quantī} or \suffix{-ō}, \latin{magnī}
or \suffix{-ō}, \latin{parvī} or \suffix{-ō}, \latin{minimī} or
\suffix{-ō}.

\item

Certain Substantives not used with serious meaning.  Thus
\latin{nihilī} or \suffix{-ō}, \english{zero}, \latin{naucī} or
\suffix{-ō}, \english{a peascod}.
\begin{examples}

\latin{“quantī ēmptae?” “Parvō.” “Quantī ergō?” “Octussibus,”}
\english{“how much did it cost?” “O, not much.” “How much,
  then?” “Eight pence”};
\apud{Sat.}{2, 3, 156}.

\latin{magnō ēmerat},
\english{had bought at a high price};
\apud{Verr.}{3, 30, 71}.

\latin{nōn nihilō aestimandum},
\english{not to be reckoned as worthless};
\apud{Fin.}{4, 23, 62}.

\end{examples}

\end{enuma}

\begin{note}[Remark]

The Genitive construction (\xref{356}) originally expressed Value, and
then was extended to express Price.  The Ablative construction
originally expressed Price (\emph{means} by which the purchase was
made), and then was extended to express Value.  The two thus approach
each other closely (Genitive of Value or Price, Ablative of Price or
Value).

\end{note}

\headingC{Ablative of the Penalty or Fine}

\section

Verbs of \emph{punishing} or \emph{fining} may take an Ablative of the
\emph{Penalty} or \emph{Fine}.
\begin{examples}

\latin{tergō ac capite pūnīrētur},
\english{be punished with stripes and death};
\apud{Liv.}{3, 55, 14}.

\latin{multātōs agrīs},
\english{mulcted of their fields};
\apud{B.~G.}{7, \emend{189}{54, 4}{54, 3}}.

\end{examples}

\begin{minor}

\subsubsection

This is the fixed construction for definite sums of money, for
fractions, and for expressions of the \emph{class} of punishment (like
\emph{chains}, \emph{exile}, \emph{death}).

\subsubsection

Verbs of \emph{condemning} regularly take a Genitive of the Penalty or
Fine; but, by a natural confusion with verbs of \emph{punishing} or
\emph{fining}, they also occasionally take the Ablative
\latin{capite}, \english{life}, and the Ablative of multiples
(“\emph{eightfold},” etc.).

\end{minor}

\headingC{Ablative of the Object, with Certain Verbs}

\section

\latin{Ūtor}, \latin{fruor}, \latin{fungor}, \latin{potior},
\latin{vēscor}, and their compounds \emph{take their Object in the
  Ablative}.
\begin{examples}

\latin{tōtīus Galliae imperiō potīrī},
\english{to get control of all Gaul};
\apud{B.~G.}{1, 2, 2}.

\latin{fruī vītā},
\english{to enjoy life};
\apud{Cat.}{4, 4, 7}.

\end{examples}

\begin{minor}

\subsubsection

A Second Object is sometimes used.
\begin{examples}

\latin{populō Rōmānō disceptātōre ūtī volō},
\english{I wish to employ the Roman people as umpire};
\apud{Leg.\ Agr.}{1, 7, 23}.
Similarly
\latin{īsdem ducibus ūsus},
\apud{B.~G.}{2, 7, 1}.

\end{examples}

\subsubsection

In early and later Latin, \latin{ūtor}, \latin{fruor},
\latin{frungor}, \latin{potior}, and \latin{vēscor} may take the
Accusative, like any other Verb of Transitive force.

\subsubsection

\latin{Potior} sometimes takes the Genitive in Ciceronian
Latin. See~\xref{353}.

\subsubsection

\latin{Epulor}, \english{feast}, may take the Ablative in poetry, on
the analogy of \latin{vēscor}.
\begin{examples}

\latin{dapibus epulāmur opīmīs},
\english{we feast on a rich banquet};
\apud{Aen.}{3, 224}.

\end{examples}

\end{minor}

\headingC{Ablative with \latin{opus est} and \latin{ūsus est}}

\section
\subsection

\latin{Opus est} and \latin{ūsus est}, \english{there is need}, may be
followed by an \emph{Ablative of the Thing Needed}.
\begin{examples}

\latin{quid opus est tortōre?}
\english{what is the need of an inquisitor?}
\apud{Mil.}{21, 57}.

\latin{nunc vīribus ūsus (est)},
\english{now there is need of strength};
\apud{Aen.}{8, 441}.

\end{examples}

\begin{minor}

\subsubsection

The construction with \latin{ūsus est} is rare after early Latin.

\end{minor}

\subsection

A Participle expressing the \emph{Leading Idea of its Phrase}
(\xref{333}) is often added to the Noun after \latin{opus est}.  A
Participle may also be used \emph{impersonally} in this construction.
\begin{examples}

\latin{nē exīstumārent sibi perditā rē pūblicā opus esse},
\english{they must not think \emph{(said he)} that he had need of
  ruining the commonwealth \emph{(of the commonwealth ruined)}};
\apud{Sall.\ Cat.}{31, 7}.

\latin{erat nihil cūr properātō opus esset},
\english{there was no reason why there must be haste};
\apud{Mil.}{\emend{133}{18}{19}, 49}.

\end{examples}

\begin{minor}

\subsubsection

\latin{Opus} is also used \emph{as a Predicate}, especially if the
thing needed is expressed by a neuter pronoun or adjective.
\begin{examples}

\latin{quaecumque ad oppugnātiōnem opus sunt},
\english{whatever things are necessary for the siege}
(are a need);
\apud{B.~G.}{5, 40, \emend{190}{5}{6}}.

\end{examples}

\subsubsection

By a mixture of constructions, \latin{opus} may stand in the
Predicate, while itself governing an Ablative Participle.
\begin{examples}

\latin{sī quid opus factō esset},
\english{if anything should be necessary to be done};
\apud{B.~G.}{1, 42, \emend{191}{5}{6}}.

\end{examples}

\subsubsection

For the Supine in \suffix{-ū} with \latin{opus} or \latin{ūsus}, see
\xref[2]{619}; for the Infinitive, \xref{585}.

\end{minor}

\headingC{Ablative with Special Verbs and Participles}

\section

The Sociative Ablative without \latin{cum} may be used with certain
verbs of \emph{exchanging}, \emph{mixing}, \emph{accustoming}, or
\emph{joining}.

These are \latin{mūtō}, \latin{commūtō}, and \latin{permūtō},
\english{exchange}; \latin{misceō}, \latin{commisceō}, and
\latin{cōnfundō}, \english{mix}; \latin{adsuēfaciō} and
\latin{adsuēscō}, \english{make} (or \english{be}) \english{familiar};
and the Participles \latin{iūnctus} and \latin{coniūnctus},
\english{joined}.
\begin{examples}

\latin{pāce bellum mūtāvit},
\english{has exchanged war for \emph{(with)} peace};
\apud{Sall.\ Cat.}{58, 15}.\footnote{The cases might be interchanged
  (\latin{pācem bellō}) with the same meaning.  Only the context can
  determine the sense.}

\latin{frūsta commixta merō},
\english{bits of food mixed with wine};
\apud{Aen.}{3, 633}.

\latin{nūllō officiō adsuēfactī},
\english{not trained in \emph{(familiarized with)} any duties};
\apud{B.~G.}{4, 1, 9}.

\latin{miseria dēdecore coniūncta},
\english{misery joined with disgrace};
\apud{Phil.}{3, 14, 35}.

\end{examples}

\begin{minor}

\subsubsection

All of these words except \vrb{mūtō}, \vrb{adsuēfaciō}, and
\vrb{adsuēscō}\versionA{ occasionally}\versionB*{ may also} take
\latin{cum}.

\subsubsection

Other constructions also occur.  Thus \vrb{mūtō} and \vrb{commūtō}
sometimes take \latin{prō} with the Ablative; \vrb{misceō} and
\vrb{commisceō} sometimes the Dative in poetry; \vrb{adsuēfaciō} and
\vrb{adsuēscō} sometimes the Dative, or \latin{ad} with the
Accusative; and the Participles \latin{iūnctus} and \latin{coniūnctus}
sometimes the Dative, especially of a person.

\subsubsection

Other parts of the verbs \vrb{iungō} and \vrb{coniungō} regularly take
\latin{cum} with the Ablative (\xref[1]{419}), or, less frequently in
prose, the Dative of Relation (\xref{362}), or \latin{ad} with the
Accusative.  The poets use the Dative freely.

\subsubsection

The poetic word \latin{suēscō} takes the Dative. See~\xref[2,
  \emph{d})]{363}.

\end{minor}

\section

The Ablative is used with \latin{frētus}, \english{depending upon},
\latin{contineor}, \english{be made up of}, \latin{comitātus},
\english{attended}, \latin{stīpātus}, \english{surrounded}.
\begin{examples}

\latin{frētus vōbīs},
\english{depending upon you};
\apud{Pomp.}{19, 58}.

\latin{nōn vēnīs et nervīs et ossibus continentur},
\english{\emph{(the gods)} are not made of veins and sinews and bones};
\apud{N.~D.}{2, 23, 59}.

\latin{aliēnīs virīs comitāta},
\english{attended by other women’s husbands};
\apud{Cael.}{14, 34}.

\latin{stīpātus armātīs},
\english{surrounded by armed men};
\apud{Phil.}{2, 3, 6}.

\end{examples}

\begin{note}[Remark]

This construction is descended from an Ablative of Means,
\latin{frētus} originally meaning \english{supported \emph{(by)}}, and
\latin{contineor} \english{be held together \emph{(by)}}.

\end{note}

\headingE{The Locative Ablative}

\headingC{Locative Ablative with Prepositions\protect\footnotemark}

\footnotetext{For summarized statements for \emph{all} prepositions,
  see \xref{455}–\xref{458}.}

\section[Regular Expression of the Place Where]

The Ablative is used with \latin{in} and~\latin{sub} to express the
\emph{Place Where} something \emph{is} or \emph{is done}.

The meaning may be either literal or figurative.
\begin{examples}

\latin{in silvīs abditī latēbant},
\english{were lying hidden in the woods};
\apud{B.~G.}{2, \emend{109}{19}{18}, 6}.

\latin{in spē victōriae},
\english{in the hope of victory};
\apud{B.~G.}{3, \emend{129}{26}{24}, 4}.

\latin{tē hortor ut maneās in sententiā},
\english{I urge you to stand by \emph{(remain in)} your proposition};
\apud{Pomp.}{24, 69}.

\latin{sub monte cōnsēdit},
\english{encamped under the mountain};
\apud{B.~G.}{1, 48, 1}.

\end{examples}

\subsubsection

The poets freely omit the preposition \latin{in} with \emph{any} noun,
and the later prose-writers follow them to some extent,
\begin{examples}

\latin{bellum geret Ītaliā},
\english{will wage war in Italy};
\apud{Aen.}{1, 263}.

\latin{prōmissīs maneās},
\english{stand by your promises};
\apud{Aen.}{2, 160}.

\latin{sēde rēgiā sedēns},
\english{sitting in the royal seat};
\apud{Liv.}{1, 41, 6}.

\end{examples}

\subsubsection

The Accusative is used with \latin{in} and \latin{sub} to express the
\emph{Place Whither} something \emph{moves}. See~\xref{381}.

\begin{minor}

\subsubsection

With a verb of \emph{placing}, the emphasis may lie upon the resulting
\emph{Place Where}, and in this case the Ablative is used with
\latin{in} and \latin{sub}.
\begin{examples}

\latin{saxa in mūrō conlocābant},
\english{were placing stones on the wall};
\apud{B.~G.}{2, \emend{192}{29}{28}, 3}.

\end{examples}

\subsubsection

\latin{Sub} regularly takes the Accusative when meaning \english{just
  before}, \english{just after}, or \english{about}
(\xref[\emph{b}]{381}).

\subsubsection

For the occasional Ablative with \latin{subter} in poetry,
see~\xref[\emph{a}]{382}.

\subsubsection

For the Place Where with names of Towns, Small Islands, etc.,
see~\xref{449}.

\end{minor}

\section

The Ablative with \latin{in} is used to express a variety of
figurative ideas.  The most important are the following:

\subsection

\emph{The Condition} or \emph{Situation in Which}:
\latin{magnō in aere aliēnō},
\english{greatly in debt};
\apud{Cat.}{2, 8, 18};
\latin{Iugurtham in catēnīs habitūrum},
\english{would have Jugurtha in chains};
\apud{Sall.\ Iug.}{64, 5}.
(Cf.\ \xref[1]{384}, and \xref[3]{406}.)

\subsection

\emph{The Field in Which} (the idea is close to that of the Respect in
Which; \xref{441}):
\latin{in omnibus vītae partibus honestus},
\english{honorable in every department of life};
\apud{Font.}{18, 41};
\latin{quid mē in hāc rē facere voluistī?}
\english{what did you want me to do in this matter?}
\apud{Ph.}{291}.
So regularly with the Gerundive or Gerund (\xref[IV]{612}).

\subsection

\emph{The Person in Whose Case}:
\latin{quantō hoc magis in fortissimīs cīvibus facere dēbēmus!}
\english{how much more ought we to do this in the case of our bravest
  citizens!}
\apud{Mil.}{34, 92}.

\section

%%* unfortunate line break

The Ablative is regularly used with \latin{super} in the sense of
\english{concerning},\allowbreak—rarely in its other senses (\english{upon},
\english{at}, \english{in addition to}; \xref{383}).
\begin{examples}

\latin{hāc super rē scrībam ad tē},
\english{I will write you on this point};
\apud{Att.}{16, 6, 1}.

\end{examples}

\begin{minor}

\subsubsection

In poetry the Ablative is sometimes used with \latin{super} in other
senses than \emph{concerning}.  Thus \latin{fronde super viridī},
\english{upon a pile of green leaves}, \apud{Ecl.}{1, \emend{193}{80}{81}};
\latin{super hīs}, \english{in addition to these things},
\apud{Sat.}{2, 6, 3}; \latin{nocte super mediā}, \english{at dead of
  night}; \apud{Aen.}{9, 61}.

\subsubsection

With verbs of \emph{placing}, \latin{super}, \english{upon}, may take
the Ablative. Cf.~\xref[\emph{c}]{433}.

\end{minor}

\headingC{Locative Ablative with or without a Preposition}

\section

With a number of words in very common use, the \emph{Place Where}
(literal or figurative) may be expressed by the Ablative either
\emph{with} or \emph{without} \latin{in}.

So especially with \latin{locō}, \latin{parte}, \latin{regiōne},
\latin{spatiō}, \latin{lītore}, \latin{cornū}, \latin{operibus},
\latin{parietibus}, \latin{librō}, \latin{numerō}, \latin{statū},
\latin{initiō}, \latin{prīncipiō}, \latin{vestīgiō}, and any Noun
modified by \latin{medius}, \latin{tōtus}, \latin{omnis},
\latin{cūnctus}, or \latin{ūniversus}; also, in poetic and later
Latin, with \latin{mediō} used substantively.
\begin{examples}

\latin{eō locō},
\english{in that place};
\apud{B.~G.}{6, 27, 4};
and
\latin{in eō locō},
\apud{B.~G.}{5, 7, \emend{194}{3}{2}}.

\latin{apertō ac plānō lītore},
\english{on the open and level shore};
\apud{B.~G.}{4, 23, 6};
and
\latin{in lītore mollī atque apertō},
\english{on the smooth and open shore},
\apud{}{5, 9, 1}.

\latin{tōtā Galliā},
\english{throughout Gaul};
\apud{B.~G.}{5, 55, 3};
and \latin{tōtā in Asiā}, \apud{Pomp.}{\emend{131}{2}{3}, 7}.

\end{examples}

\subsubsection

In general, the preposition is more likely to be used when the noun is
accompanied by a pronoun or descriptive adjective.  But with
\latin{medius}, \latin{tōtus}, etc., the preposition is not common.

\begin{note}

With a verb of \emph{motion}, Ablatives of this class often in effect
express \emph{the space over which}; cf.~\xref[\emph{c}]{426}.  Thus
\latin{tōtā Asiā vagātur},
\english{wanders \emph{(in =)} through the whole of Asia},
\apud{Phil.}{11, 2, 6};
\latin{impedītiōribus locīs secūtī},
\english{following over somewhat difficult ground};
\apud{B.~G.}{3, \emend{128}{28}{26}, 4}.

\end{note}

\begin{minor}

\subsubsection

The following Locative Ablatives are used \emph{without} a prepostion
in Ciceronian Latin:
\latin{dextrā}, \english{on the right},
\latin{laevā} and \latin{sinistrā}, \english{on the left},
\latin{corpore}, \english{on} or \english{in the body},
\latin{animō} and \latin{animīs}, \english{in mind}
(but \latin{in animō} with \latin{est} and \latin{habeō}),
\latin{memoriā}, \english{in} or \english{within the memory},
\latin{linguā}, \english{in the language},
\latin{nōmine} and \latin{speciē}, \english{under the name} or
\english{pretext},
\latin{lēge} and \latin{condiciōne}, \english{under the condition}
(for \latin{lēge}, \english{by law}, see \xref[\emph{a}]{414}).
Later, \latin{sub}, \english{under}, is sometimes added to
\latin{nōmine}, \latin{speciē}, \latin{lēge}, and \latin{condiciōne}.
\begin{examples}

\latin{deus inclūsus corpore hūmānō},
\english{a god inclosed in a human body};
\apud{Div.}{1, 31, 67}.

\latin{patrum nostrōrum memoriā},
\english{within the memory of our fathers};
\apud{B.~G.}{1, 12, 5}.

\latin{memoriā tenētis},
\english{you remember}
(hold in memory);
\apud{Cat.}{3, 8, 19}.

\latin{quī ipsōrum linguā Celtae appellantur},
\english{who in their own language are called Celts};
\apud{B.~G.}{1, 1, 1}.

\latin{obsidium nōmine},
\english{under the name of hostages};
\apud{B.~G.}{3, 2, 5}.

\end{examples}

\end{minor}

\headingC{Locative Ablative with Certain Verbs and Participles}

\section

\latin{Fīdō} and \latin{cōnfīdō}, \english{trust}, may take the
Ablative.
\begin{examples}

\latin{multum nātūrā locī cōnfīdēbant},
\english{they had great confidence in the nature of the country};
\apud{B.~G.}{3, 9, 3}.

\end{examples}

\subsubsection

\latin{Fīdō} and \latin{cōnfīdō} also take the Dative (\xref{362}; so
regularly of a \emph{person} in Ciceronian Latin).

\subsubsection

\latin{Diffīdō}, \english{distrust}, takes the Dative in Ciceronian
Latin, and both the Dative and the Ablative in later writers.

\section
\subsection

The Ablative, generally without \latin{in}, is used with
\latin{nītor}, \english{rely upon}, and \latin{stō} and
\latin{cōnstō}, \english{abide by}.
\begin{examples}

\latin{dolō nīterentur},
\english{rely upon treachery};
\apud{B.~G.}{1, 13, \emend{89}{6}{4}}.
(With \latin{in}, \apud{Mil.}{7, 19}.)

\latin{sī quī eōrum dēcrētō nōn stetit},
\english{if any one does not abide by their decree};
\apud{B.~G.}{6, 13,~6}.
(With \latin{in}, \apud{Fin.}{1, 14, 47}.)

\end{examples}

\subsection

\vrb{Adquiēscō}, \english{take pleasure in}, takes the Ablative with
or without \latin{in} with about equal frequency.
\begin{examples}

\latin{senēs in adulēscentium cāritāte adquiēscimus},
\english{in old age we take pleasure in being liked by young people};
\apud{Am.}{27, 101}.
(Without \latin{in}, \apud{Mil.}{37, 102}.)

\end{examples}

\begin{minor}

\subsubsection

The Participles \latin{innīxus} and \latin{subnīxus},
\english{leaning} or \english{relying upon}, take the Ablative without
a preposition.
\begin{examples}

\latin{scūtīs innīxī},
\english{leaning upon their shields};
\apud{B.~G.}{2, \emend{195}{27}{26}, 1}.

\latin{adrogantiā subnīxī},
\english{relying upon their pride};
\apud{De~Or.}{1, 58, 246}.

\end{examples}

\subsubsection

In later Latin, the finite forms of \vrb{innītor} (not occurring in
Ciceronian Latin), as well as the form \latin{innīxus}, may take the
Dative, on the principle of \xref{376}, or the Ablative, as
above. Thus
\latin{innītitur hastae}, \apud{Ov.\ Met.}{14, \emend{30}{665}{819}};
\latin{incolumitāte innītī}, \apud{\emend{31}{Tac.}{Tac.\ Ann.}}{15, 60};
\latin{hastā innīxus}, \apud{Liv.}{4, 19, 4};
\latin{cūrae innīxa}, \apud{Quintil.}{6, 1, 35}.

\subsubsection

Other verbs of like meaning take a preposition; thus \latin{sī in eō
  manērent}, \apud{B.~G.}{1, 36, 5}.  But a poet may omit it, as in
\latin{prōmissīs maneās}, \apud{Aen.}{2, 160}.

\end{minor}

\subsection

The Ablative, regularly with \latin{in}, is used with \latin{cōnstō}
and \latin{cōnsistō}, when meaning \english{depend upon}, and
\latin{cōnsistō} when meaning \english{consist in}.
\begin{examples}

\latin{monuit victōriam in eārum cohortium virtūte cōnstāre},
\english{pointed out that victory depended upon the valor of these
  cohorts};
\apud{B.~C.}{3, 89, \emend{196}{3}{4–5}};
cf.\ \apud{B.~G.}{7, 84, 4}.

\latin{vīta omnis in vēnātiōnibus atque in studiīs reī mīlitāris
  cōnsistit},
\english{their whole life consists in hunting and military pursuits};
\apud{B.~G.}{6, 21, 3}.

\end{examples}

\begin{minor}

\subsubsection

But the Ablative without \latin{in} also occurs.
\begin{examples}

\latin{cēterārum rērum studia et doctrīnā et praeceptīs et arte cōnstāre},
\english{that in other fields intellectual pursuits depend upon
  principles, precepts, and art};
\apud{Arch.}{8, 18}.

\end{examples}

\subsubsection

\latin{Cōnstō}, \english{consist of}, takes the construction of
Material.  See~\xref[4\versionA{, \emph{b}}]{406}.

\end{minor}

\subsection

The Ablative is used with \latin{contentus}, \english{content},
\english{satisfied}.
\begin{examples}

\latin{contentus hāc inīquitāte nōn fuit},
\english{was not content with this iniquity};
\apud{Verr.}{2, 38, 94}.

\end{examples}

\begin{minor}

\subsubsection

\latin{Contentus} originally meant \english{self-restrained
  \(in\)}. Cf.\ \latin{in illā cupiditāte continēbātur},
\english{restrained herself within that desire} (was content with it);
\apud{Clu.}{5, 11}.

\end{minor}

\begin{minor}

\subsection

\latin{Intentus} is used with the Ablative (probably Locative) in
\latin{aliquō negōtiō intentus}, \english{deeply engaged in some
  occupation or other}; \apud{Sall.\ Cat.}{2, 9}.  Commonly it takes
the Dative (\english{stretched toward} = \english{intent upon};
see~\xref{376}), or \latin{ad} or \latin{in} with the Accusative.

\end{minor}

\headingE{Ablative Constructions of Composite Origin (Fusion)}

\headingC{Ablative of the Time at or within Which}

\section

\emph{The Time at} or \emph{within Which} anything is or is done may
be expressed by the Ablative without a Preposition.
\begin{examples}

\latin{diē septimō pervenit},
\english{arrives on the seventh day};
\apud{B.~G.}{1, 10, \emend{87}{5}{4}}.

\latin{superiōre aestāte cognōverat},
\english{had learned the previous summer};
\apud{B.~G.}{5, 8, 3}.

\latin{bellō vacātiōnēs valent},
\english{in time of war, exemptions hold};
\apud{Phil.}{8, 1, 3}.

\latin{comitiīs},
\english{at the election};
\apud{Cat.}{1, 5, 11}.
Similarly with words denoting \emph{games} or \emph{feasts}, as
\latin{lūdīs}, \latin{gladiātōribus}, \latin{epulīs},
\latin{pulvīnāribus}.

\end{examples}

\subsubsection

The Preposition \latin{in} is regularly used:
\begin{enumerate}

\item
With a word denoting a \emph{time of life}, unless this is accompanied
by a modifier.  Thus \latin{in pueritiā}, \english{in boyhood},
\apud{Verr.}{1, 18, 47}; but \latin{extrēmā pueritiā}, \english{at the
  end of boyhood}, \apud{Pomp.}{10, 28}.

\item
With a word denoting an \emph{office}, unless this is accompanied by a
numeral.  Thus \latin{in cōnsulātū nostrō}, \english{in my
  consulship}, \apud{Arch.}{11, 28}; but \latin{quārtō cōnsulātū},
\english{in his fourth consulship}, \apud{Sen.}{13, 43}.

\item
In phrases expressing \emph{situation}.  Thus \latin{in tālī tempore},
\english{in such a state of affairs}, \apud{Sall.\ Cat.}{48, 5};
\latin{in cīvīlī bellō}, \english{in a civil war}, \apud{Phil.}{2, 19,
  47} (but \latin{secundō Pūnicō bellō}, \english{in the second Punic
  war}, \apud{Off.}{1, 13, 40}, because only the \emph{Time at Which}
is meant).

\item
With a \emph{numeral}.  Thus \latin{ter in annō}, \english{thrice a
  year}; \apud{Rosc.\ Am.}{46, 132}.

\end{enumerate}

\begin{minor}

\subsubsection

The Time at Which is sometimes expressed by \latin{cum} with the
Ablative.
\begin{examples}

\latin{cum prīmā lūce in campum currēbat},
\english{with the first \emph{(streak of)} light he was running into
  the forum};
\apud{Att.}{4, 3, 4}.

\end{examples}

\subsubsection

The Time at Which may also be expressed by \latin{ad} or \latin{sub}
(in later Latin with \latin{circā} likewise), and the Time within
Which by \latin{intrā}, with the Accusative.
\begin{examples}

\latin{sub occāsum sōlis sē recēpērunt},
\english{toward sunset they retired};
\apud{B.~G.}{2, 11, 6}.

\latin{intrā annōs XIIII},
\english{in fourteen years};
\apud{B.~G.}{1, 36, 7}.

\end{examples}

\end{minor}

\headingC{Rarer Ablative of Duration of Time}

\section

The Ablative is occasionally used to express \emph{Duration of Time}.
\begin{examples}

\latin{tōtā nocte continenter iērunt},
\english{went without break all night};
\apud{B.~G.}{1, 26, 5}.

\latin{quī vīgintī annīs āfuit},
\english{who was absent twenty years};
\apud{Bacch.}{2}.

\end{examples}

\headingC{Ablative of the Respect in Which}

\section

The \emph{Respect in Which} the meaning of a Verb or Adjective is to
be taken is expressed by the Ablative, regularly without a Preposition.

This Ablative answers the question, \english{In what?}
\english{Wherein?}
\begin{examples}

\latin{cum virtūte omnibus praestārent},
\english{since they surpassed all in bravery};
\apud{B.~G.}{1, 2, 2}.

\latin{numerō ad duodecim},
\english{about twelve in number};
\apud{B.~G.}{1, 5, 2}.

\latin{alterō oculō capitur},
\english{is blinded in one eye};
\apud{Liv.}{22, 2, 11}.

\latin{maiōrēs nātū},
\english{the elders}
(greater in respect of birth);
\apud{B.~G.}{2, \emend{197}{13, 2}{12, 7}}.
Similarly with \latin{maximus}, \latin{minor}, and \latin{minimus},
\english{oldest}, \english{younger}, \english{youngest}.

\end{examples}

\subsubsection

The preposition \latin{in} is occasionally used with abstract words.
Thus \latin{similem in fraude et malitiā}, \english{alike in knavery
  and wickedness}, \apud{Rosc.\ Com.}{7, 20}.  Cf.\ \latin{mōribus
  similēs}, \english{alike in character}, \apud{Clu.}{16, 46}.

\begin{minor}

\subsubsection

\latin{In} is \emph{regularly} used with a pronoun, unless this is a
relative.  Thus \latin{nōs nōn modo nōn vincī ā Graecīs verbōrum
  cōpiā, sed esse in eā etiam superiōrēs}, \english{that we are not
  only not surpassed by the Greeks in wealth of vocabulary, but are even
  superior in this}; \apud{Fin.}{3, 2, 5}.

\subsubsection

The Respect in Which the meaning of a \emph{noun} is to be taken must
in general be expressed by the Genitive of Application (\xref{354}),
or the Ablative with \latin{in}.  Thus \latin{virtūte praestārent} (in
example above), but \latin{praestantiam virtūtis} (see \xref{354}) or
\latin{in virtūte}.

\emph{Apparent Exceptions} occur in a few combinations.  Thus
\latin{hominēs nōn rē, sed nōmine}, \english{human beings \emph{(=
    human)} not in fact, but in name}; \apud{Off.}{1, 30, 105}.

\subsubsection

Respect may also be expressed by \latin{ad} with the Accusative.  Thus
\latin{sitū praeclārō ad aspectum}, \english{with a site splendid in
  aspect}; \apud{Verr.}{4, 52, 117}.

\end{minor}

\headingC{Ablative with \latin{dignus} and \latin{indignus}}

\section

\latin{Dignus} and \latin{indignus}, \english{worthy} and
\english{unworthy}, are followed by the Ablative.
\begin{examples}

\latin{cognitiōne dignum},
\english{worthy of acquaintance};
\apud{Arch.}{3, 5}.

\latin{suppliciō dignī},
\english{deserving punishment};
\apud{Cat.}{3, 9, 22}.

\latin{indigna homine līberō},
\english{unworthy of a free man};
\apud{Rab.\ Perd.}{5, 16}.

\end{examples}

\subsubsection

The poets and later prose writers employ the same construction with
\latin{dignor}, \english{think worthy}.  Thus \latin{haud tālī mē
  dignor honōre}, \english{I do not deem myself worthy of such an
  honor}; \apud{Aen.}{1, 335}.

\headingC{Descriptive Ablative}

\section

\emph{Kind} or \emph{External Aspect} may be expressed by the Ablative
of a Noun accompanied by a modifier; also, in a few phrases,
\emph{Situation} or \emph{Mental Condition}.

The construction may be either appositive or predicative.
\begin{examples}

\latin{C.\ Valerium Procillum, summā virtūte adulēscentem},
\english{Gaius Valerius Procillus, a young man of the greatest courage};
\apud{B.~G.}{1, 47, 4}.

\latin{C.\ Gracchus, clārissimō patre, avō, maiōribus},
\english{Gaius Gracchus, a man with a distinguished father, grandfather, and ancestors in general};
\apud{Cat.}{1, 2, 4}.

\latin{“sed quā faciēst?” “dīcam tibi: macilentō ōre, nāsō acūtō, corpore albō, oculīs nigrīs,”}
\english{“but of what appearance is he?” “I’ll tell you: he is a
  man with a spare face, a sharp nose, white skin, and black eyes”};
\apud{Capt.}{646}.

\latin{relīquit quōs virōs!  quantō aere aliēnō!}
\english{what men he left behind him!  how deep in debt}
(in how great debt)!
\apud{Cat.}{2, 2, 4}.  (Situation.)

\latin{equidem cum spē sum maximā, tum maiōre etiam animō},
\english{I for my part am in a state not only of the greatest hope,
  but of a still greater determination};
\apud{Q.~Fr.}{1, 2, 5, 16}.
(Mental Condition.)

\end{examples}

\begin{minor}

\subsubsection

In Ciceronian Latin this Ablative is generally attached to a
\emph{class}-name in apposition with the name of the person (as in the
first example above).  In later Latin, it is more freely attached to
the name of the person (as in the second example above).

\subsubsection

\latin{Statūra}, \latin{fōrma}, and \latin{corpus}, as really
expressing the idea of \emph{Kind}, may be used with either the
Genitive or the Ablative.  Thus \latin{hominēs tantulae statūrae},
\english{men of such diminutive stature} (= such puny men),
\apud{B.~G.}{2, \emend{198}{30}{29}, 4};
\latin{quā faciē, quā statūrā}, \english{of what appearance},
\english{of what stature}, \apud{Phil.}{2, 16, 41}.

\subsubsection

\latin{Genus} is not much used in the Ablative, \latin{modus} never.

\subsubsection

Groups containing adjectives in \suffix{-is} or the adjective
\latin{pār} are almost always in the Ablative.  Thus \latin{cōnstantiā
  singulārī}, \english{of exceptional steadfastness}; \apud{Pomp.}{23,
  68}.

\end{minor}

\headingC{Ablative of Cause or Reason}

\section

\emph{Cause} or \emph{Reason} may be expressed by the Ablative without
a Preposition.

\begin{examples}

\latin{cūrīs aeger},
\english{sick with anxiety};
\apud{Aen.}{1, 208}.

\latin{metū relictās urbīs},
\english{cities abandoned because of fear};
\apud{Pomp.}{11, 32}.

\latin{meā restitūtiōne laetātus est},
\english{rejoiced in my return};
\apud{Planc.}{10, 25}.

\end{examples}

\subsubsection

The construction is especially frequent with verbs and adjectives of
\emph{taking pleasure}, \emph{rejoicing}, \english{boasting}, or the
opposite.\footnote{E.g.\ \vrb{angor}, \vrb{bacchor}, \latin{dēlector},
  \vrb{doleō}, \vrb{exsiliō}, \vrb{exsultō}, \vrb{gaudeō},
  \vrb{glōrior}, \vrb{laetor}, \vrb{maereō}, \latin{mē iactō}; and the
  adjectives \latin{beātus}, \latin{fēlīx}, \latin{īnfēlīx},
  \latin{laetus}, \latin{maestus}, \latin{miser}.}

\begin{minor}

\subsubsection

The prepositions \latin{dē}, \latin{ex}, and \latin{in} are
occasionally used with one or another of these words.  Thus
\latin{ex vulnere aeger},
\english{sick from a wound},
\apud{Rep.}{2, 21, 38};
\latin{ex commūtātione rērum doleant},
\english{suffer from a change of fortune},
\apud{B.~G.}{1, 14, 5};
\latin{ut in hōc sit laetātus quod\dots},
\english{so that he took pleasure in the fact that\dots},
\apud{Phil.}{11, 4, 9}.

\subsubsection

Cause may also be expressed by \latin{ob}, \latin{per}, or
\latin{propter} with the Accusative.  Thus \latin{ob eās rēs},
\english{on account of these achievements},
\apud{B.~G.}{2, \emend{120}{35}{34}, 4}.

\subsubsection

\latin{Causā} and \latin{grātiā}, common with Genitives
(\xref[\emph{d}]{339}), were themselves originally Ablatives of Cause.

\end{minor}

\headingC{Ablative of Way or Manner}

\section

\emph{Way} or \emph{Manner} may be expressed by the Ablative, as
follows:

\subsection

With certain \emph{very common} Nouns, by the Ablative without a
Preposition.  These are especially: \latin{arte}, (\latin{parī},
etc.)\ \latin{certāmine}, \latin{cōnsiliō} (\english{intentionally}),
\latin{cāsū}, \latin{dolō}, \latin{fraude}, \latin{fūrtō},
\latin{iūre}, \latin{iniūriā}, \latin{meritō}, (\latin{hōc},
etc.)\ \latin{modō} or \latin{mōre}, \latin{ope} and \latin{opibus},
\latin{ōrdine}, (\latin{hōc}, etc.)\ \latin{pactō}, \latin{paucīs},
\latin{ratiōne}, \latin{rītū}, \latin{sponte}, \latin{vī} and
\latin{vīribus}, \latin{viā}, \latin{vitiō}, \latin{voluntāte}
(\english{voluntarily}).
\begin{examples}

\latin{sīve cāsū sīve cōnsiliō},
\english{accidentally or by intention};
\apud{B.~G.}{1, 12, 6}.

\latin{iūre an iniūriā},
\english{rightly or wrongly};
\apud{Mil.}{11, 31}.

\latin{aliquō modō},
\english{some way or other};
\apud{Arch.}{5, 10}.

\end{examples}

\begin{minor}

\subsubsection
The poets extend the usage.  Thus \latin{rīmīs}, \english{in chinks},
\apud{Aen.}{1, 123}; \latin{cumulō}, \english{in a heap},
\apud{Aen.}{1, 105}; \latin{cursū}, \english{on the run},
\apud{Aen.}{5, 265}.

\end{minor}

\subsection

With other nouns, if \emph{Concrete}, by the Ablative without a
Preposition.
\begin{examples}

\latin{nūdō corpore pugnāre},
\english{to fight with the body unprotected};
\apud{B.~G.}{1, 25, 4}.

\latin{aequō animō moriar},
\english{I shall meet death with a calm mind};
\apud{Cat.}{4, 2, 3}.

\latin{statuit nōn proeliīs neque aciē, sed aliō mōre bellum gerundum},
\english{decided that the war must be carried on, not with engagements
  or in battle array, but in some other manner};
\apud{Sall.\ Iug.}{54, 5}.

\latin{pedibus proeliantur},
\english{they fight on foot};
\apud{B.~G.}{4, 2, 3}.

\end{examples}

\subsection

With other nouns, if \emph{Abstract}, by the Ablative with \latin{cum}
if no Adjective is used, and either with or without \latin{cum} if an
Adjective \emph{is} used.
\begin{examples}

\latin{sī utrumque cum cūrā fēcerīmus},
\english{if we do both things with care};
\apud{Quintil.}{10, 7, 29}.

\latin{magnā cum cūrā suōs fīnīs tuentur},
\english{defend their boundaries with great care};
\apud{B.~G.}{7, 65, 3}.

\latin{id summā cūrā conquīrimus},
\english{this we search for with the greatest care};
\apud{Ac.}{2, 3, 7}.

\end{examples}

\begin{minor}

\subsubsection

Occasionally, other turns of expression are used.  Thus \latin{ad} (or
\latin{in}) \latin{hunc modum}, \english{in this way}; \latin{per
  vim}, \english{by violence}; \latin{per iocum}, \english{in jest}.

\end{minor}

\headingC{Ablative with Verbs meaning \english{carry}, \english{hold},
\english{keep}, \english{receive}, \english{pour}, \english{depend}}

\section
\subsection

Verbs meaning \emph{carry}, \emph{hold}, \emph{keep}, or
\emph{receive},\footnote{\vrb{Ferō}, \vrb{portō}, \vrb{gerō},
  \vrb{vehō}, \vrb{sustineō}, \vrb{gestō}; \latin{mē teneō},
  \latin{mē contineō}; \latin{accipiō}, \latin{recipiō}.

  The Ablatives most used are \latin{equō}, \latin{nāve},
  \latin{castrīs}, \latin{vāllō}, \latin{fīnibus}, \latin{oppidō},
  \latin{urbe}, \latin{portū}, \latin{tēctō} (and their plurals).}
and Verbs meaning\linebreak \emph{pour},\footnote{\latin{Fundō} and
  \latin{lībō}.} may be followed by the Ablative.
\begin{examples}

\latin{quam equīs vexerat},
\english{which \emph{(legion)} he had brought on horseback};
\apud{B.~G.}{1, 43, 2}.

\latin{castrīs sēsē tenēbat},
\english{was keeping himself in his camp};
\apud{B.~G.}{3, 17, \emend{199}{5}{9}}.

\latin{oppidīs recipere},
\english{receive them in their towns};
\apud{B.~G.}{2, 3, 3}.

\latin{vīna fundēbat paterīs},
\english{was pouring wine from the sacrificial bowls};
\apud{Aen.}{5, 98}.

\end{examples}

\begin{minor}

\subsubsection

\latin{In} is occasionally used with some of these words.  Thus
\latin{equus in quō vehēbar},
\english{the horse on which I was riding},
\apud{Div.}{2, 68, \emend{32}{14}{140}};
\latin{tempestātēs quae nostrōs in castrīs continērent},
\english{storms that kept our men in camp};
\apud{B.~G.}{4, 34, 4}.

\end{minor}

\subsection

\latin{Pendeō}, \english{hang}, \english{depend}, takes \latin{in} or
a separative Preposition when used with literal force, and either a
Preposition or the bare Ablative when used with figurative force.
\begin{examples}

\latin{ex ūnīus vītā pendēre},
\english{hung upon the life of one man};
\apud{Marc.}{7, 22}.

\latin{quae spē exiguā pendet},
(our safety), \english{which hangs upon a slight hope};
\apud{Flacc.}{2, 4}.

\end{examples}

\headingG{Two Ideas Suggested by a Single Ablative}

\section

An Ablative may suggest \emph{two ideas} at the same time.
\begin{examples}

\latin{superiōribus proeliīs exercitātī},
\english{trained in \(\emph{and} by\) preceding battles};
\apud{B.~G.}{2, \emend{114}{20, 3}{19, 2}}.
(Time and Means.)

\latin{quōrum adventū Rēmīs studium prōpugnandī accessit},
\english{at \(\emph{and} because of\) their coming, the Remi felt
  fresh energy for the attack};
\apud{B.~G.}{2, 7, 2}.
(Time and Cause.)

\latin{tranquillō silet},
\english{in calm it lies silent};
\apud{Aen.}{5, 127}.
(Time and Situation.)

\end{examples}

\chapter*[Place-Constructions with Names of Towns, etc.]{Place-Constructions}

\headingC{\textbfsc{with names of towns}, \latin{domus}, \latin{rūs},
  etc.}

\contentsentry{C}{Place-Constructions with Names of Towns,
  \latin{domus}, \latin{rūs}, etc.}

\begin{minor}

\section[\textsc{Introductory}] % 448

A few classes of words were in such constant use to express
place-relations that the preposition never became regular with them.
These are: Names of Towns and Small Islands, the words for
\english{home} and \english{country}, and a few others.  Though the
constructions belong to three different cases, they will be best
remembered together.

\end{minor}

\section

To express the \emph{Place Where}, names of Towns and Small Islands
are put in the Locative, which in the Singular Number of the First or
Second Declension is identical with the Genitive, and elsewhere with
the Ablative.
\begin{examples}

\latin{Rōmae cōnsulēs, Carthāgine quotannīs annuī bīnī rēgēs creābantur},
\english{at Rome consuls were elected yearly, at Carthage two annual kings};
\apud{Nep.\ Hann.}{7, 4}.

\latin{nātus Athēnīs},
\english{born at Athens};
\apud{Iuv.}{3, 80}.

\latin{Cyprī vīsum},
\english{seen at Cyprus};
\apud{B.~C.}{3, 106, 1}.

\end{examples}

\subsection

Similarly \latin{domī}, \english{at home}, \latin{humī}, \english{on
  the ground}, \latin{bellī} and \latin{mīlitiae}, \english{in war},
\latin{rūrī} or \latin{rūre}, \english{in the country}, \latin{forīs},
\english{out of doors}, \latin{marī}, \english{at sea}.
\latin{Terrā}, \english{on land} (seldom standing alone) follows the
apparent case of \latin{marī}.
\begin{examples}

\latin{illī domī remanent},
\english{the others remain at home};
\apud{B.~G.}{4, 1, 5}.

\latin{rūrī adsiduus fuit},
\english{he was constantly in the country};
\apud{Rosc.\ Am.}{29, 81}.

\latin{terrā marīque},
\english{on land and sea};
\apud{Cat.}{2, 5, 11}.

\end{examples}

\begin{minor}

\subsubsection

A Locative \latin{terrae} is also sometimes used in later Latin;
e.g.\ \latin{sacra terrae cēlāvimus}, \english{we hid the sacred
  objects in the earth}; \apud{Liv.}{5, 51, 9}.  Similarly, probably,
\latin{sternitur terrae}, \english{stretches himself upon the earth};
\apud{Aen.}{11, 87}.

\subsubsection

\latin{Animī}, \english{in mind} (in origin a Locative), and, by
analogy, the Genitive \latin{mentis}, are used with verbs and
adjectives of Mental Condition to express Respect.  Thus
\latin{furēns animī}, \english{raging in his heart}, \apud{Aen.}{5, 202};
\latin{pendet animī}, \english{is uncertain in mind}, \apud{Tusc.}{4, 16, 35}.

\subsubsection

The poets and some later prose writers use the construction
of~\xref{449} somewhat boldly.  Thus \latin{Crētae cōnsīdere},
\english{to settle in Crete} (a \emph{large} island),
\apud{Aen.}{3, \emend{200}{161}{162}};
\latin{Rōmae Numidiaeque}, \english{in Rome and Numidia}\emend{82}{;}{,}
\apud{Sall.\ Iug.}{33, 4}.

\end{minor}

\section

To express the \emph{Place Whither}, names of Towns and Small Islands
are put in the Accusative without a Preposition.
\begin{examples}

\latin{Rōmam revertisse},
\english{returned to Rome};
\apud{Mil.}{23, 61}.

\latin{Dēlum vēnit},
\english{came to Delos};
\apud{Verr.}{1, 17, 46}.

\end{examples}

\begin{minor}

\subsubsection

So sometimes Greek geographical names (as \latin{Bosphorum},
\apud{Mur.}{16, 34}), including \latin{Aegyptus}, \english{Egypt}
(\apud{N.~D.}{3, 22, 56}).

\subsubsection

Similarly \latin{domum}, \english{home},\footnote{Similarly we say in
  English “go home,” not “go to home.”} and \latin{rūs},
\english{to the country}.
\begin{examples}

\latin{domum reditiōnis spē},
\english{hope of returning home};
\apud{B.~G.}{1, 5, 3}.

\latin{domum rediērunt},
\english{went home again};
\apud{B.~G.}{1, 29, 3}.

\latin{rūs ībō},
\english{I am going to the country};
\apud{Eun.}{216}.

\end{examples}

\subsubsection

Latin expresses the relations of Place with exactness, no matter how
many words may be used.
\begin{examples}

\latin{rēs ad Chrȳsogonum in castra L.\ Sullae Volāterrās dēfertur},
\english{the matter is reported to Chrysogonus in the camp of Lucius
  Sulla at Volaterrae}
(in the Latin, \english{to\ellipsis to\ellipsis to\dots});
\apud{Rosc.\ Am.}{7, 20}.

\end{examples}

\end{minor}

\section

To express the \emph{Place Whence}, names of Towns and Small Islands
are put in the Ablative without a Preposition.
\begin{examples}

\latin{Rōmā profectus est},
\english{set out from Rome};
\apud{Mil.}{10, 27}.

\latin{Dēlō proficīscitur},
\english{sets out from Delos};
\apud{Verr.}{1, 18, 46}.

\end{examples}

\subsubsection

Similarly \latin{domō}, \english{from home}, \latin{rūre},
\english{from the country}.
\begin{examples}

\latin{domō dūxerat},
\english{he had brought from home};
\apud{B.~G.}{1, 53, 4}.

\latin{rūre advenit},
\english{comes in from the country};
\apud{Hec.}{\emend{33}{190}{191}}.

\end{examples}

\begin{minor}

\subsubsection

Letters are regularly dated \emph{from} a place.  Thus
\latin{Nōn.\ Nov.\ Brundisiō}, \english{(from) Brindisi, November~5};
\apud{Fam.}{14, 12}.

\end{minor}

\headingC{The Appositive with Names of Towns Where, Whither, or
  Whence}

\section

When an Appositive like \latin{urbs}, \latin{oppidum}, etc., is to be
added to the name of the Town \emph{Where}, \emph{Whither}, or
\emph{Whence}, the full expression with the Preposition is regularly
used.
\begin{examples}

\latin{Albae, in urbe opportūnā},
\english{at Alba, a convenient city};
\apud{Phil.}{4, 2, 6}.

\latin{Tarquiniōs, in urbem flōrentissimam},
\english{to Tarquinii, a very prosperous city};
\apud{Rep.}{2, 19, 34}.

\latin{Tusculō, ex clārissimō mūnicipiō},
\english{from Tusculum, a very splendid town};
\apud{Font.}{18, 41}.

\end{examples}

\begin{minor}

\subsubsection

Exceptions occur.  Thus \latin{Antiochīae, celebrī quondam urbe},
\english{at Antioch, a once populous city}, \apud{Arch.}{3, 4};
\latin{Capuam, urbem amplissimam}, \english{to Capua, a very
  flourishing city}, \apud{Leg.\ Agr.}{2, 28, 76}.

\end{minor}

\headingC{Occasional Use of the Preposition with Names of Towns, etc.}

\section

A Preposition may be used with the Name of a Town:

\subsection

To express \emph{the neighborhood in}, \emph{to}, or \emph{from
  which}.
\begin{examples}

\latin{ad Cannās pugnam},
\english{the battle at \emph{(i.e.\ near)} Cannae};
\apud{Liv.}{22, 58, 1}.

\latin{ad Genāvam pervenit},
\english{arrives before Geneva};
\apud{B.~G.}{1, 7, 1}.

\latin{ab Zāmā discēdit},
\english{withdraws from the neighborhood of Zama};
\apud{Sall.\ Iug.}{61, 1}.

\end{examples}

\begin{minor}

\subsubsection

With a noun, the \emph{Adjective} is frequent to express the
\emph{neighborhood in which}.  Thus \latin{post Cannēnsem pugnam},
\english{after the battle at Cannae}; \apud{Liv.}{23, 1, 1}.

\end{minor}

\subsection

To express \emph{the point reckoned from} or \emph{toward}.
\begin{examples}

\latin{ā Bibracte nōn amplius mīlibus passuum XVIII aberat},
\english{was not more than eighteen miles from Bibracte};
\apud{B.~G.}{1, 23, 1}.

\end{examples}

\subsection
Occasionally \emph{for sharper contrast}.

\begin{examples}

\latin{ab Arīminō Arrētium mittit},
\english{sends from Rimini to Arezzo};
\apud{B.~C.}{1, 11, 4}.

\end{examples}

\begin{minor}

\subsubsection

But at times the preposition seems to be used simply for the sake of
variety (especially in poetry and later prose).  Thus \latin{et ab
  Trallibus et ā Magnēsiā et ab Ephesō ad dēdendās urbīs vēnērunt},
\english{\(ambassadors\) came from Tralles, from Magnesia, from Ephesus,
  to surrender their cities}, \apud{Liv.}{37, 45, 1}; \latin{ab domō},
  \english{from home}, \apud{Liv.}{25, 31, 3}.

\end{minor}

\headingC{Domī, domum, domō, etc., with Modifiers}

\section
\subsection

\latin{Domī}, \latin{domum}, and \latin{domō} may be modified by a
Possessive Genitive or a Possessive Pronoun or Adjective.  Thus
\latin{domī Caesaris}, \latin{domī meae}, \latin{domī aliēnae},
\english{at Caesar’s house}, \english{at my house}, \english{at the house
  of another}.

\subsection

When \latin{domus} means a house regarded simply as a \emph{building},
a Preposition is regularly used in the above constructions.  Thus
\latin{arma omnia in domum Gallōnī contulit}, \english{packed all the
  arms into the house of Gallonius}; \apud{B.~C.}{2, 18, 2}.

\subsection

In the ordinary meaning of \emph{house} or \emph{home}, either the
bare case or the Preposition may be used, if the Noun is accompanied
by a modifier.  Thus \latin{domī Caesaris} and \latin{in domō
  Caesaris}; \latin{M.\ Laecae domum} and \latin{in M.\ Laecae domum}
(\apud{Cat.}{1, 4, 8}).

\subsection

“At a person’s house” may also be expressed by \latin{apud} or
\latin{ad} with the name of the person.  Thus \latin{apud M.\ Laecam},
\english{at the house of Marcus Laeca}, \apud{Cat.}{2, 6, 12};
\latin{ad M’.\ Lepidum}, \english{at the house of Manius Lepidus},
\apud{Cat.}{1, 8, 19}.

\chapter{Summary of the Uses of Cases with Prepositions}

\contentsentry{C}{Summary of Case-Uses with Prepositions}

\section

The Accusative is always used with the Prepositions \latin{ad},
\latin{adversus} and \latin{adversum}, \latin{ante}, \latin{apud},
\latin{circā}, \latin{circiter} and \latin{circum}, \latin{cis} and
\latin{citrā}, \latin{contrā}, \latin{ergā}, \latin{extrā},
\latin{īnfrā}, \latin{inter}, \latin{intrā}, \latin{iūxtā},
\latin{ob}, \latin{penes}, \latin{per}, \latin{pōne} and \latin{post},
\latin{praeter}, \latin{prope}, \latin{propter}, \latin{secundum},
\latin{suprā}, \latin{trāns}, \latin{ultrā}, \latin{versus}
(\xref{380}).

\begin{minor}

\subsubsection

\latin{Propius} and \latin{proximē} may, like \latin{prope}, take the
Accusative (\xref[\emph{b}]{380}).

\subsubsection

\latin{Versus} follows its noun.  But this is generally preceded by
another preposition (\latin{ad} or \latin{in}) unless it is the name
of a Town or Small Island (\xref[\emph{a}]{380}).

\end{minor}

\section

The Ablative is always used with the Prepositions \latin{ab},
\latin{dē}, \latin{ex}, and \latin{sine}; \latin{cōram},
\latin{palam}, \latin{prae}, and \latin{prō}; \latin{cum} (\xref{405},
\xref{407}, \xref{418}).

\begin{minor}

\subsubsection

\latin{Procul} and \latin{simul} may take the Ablative in poetry and
later prose (\xref[\emph{c}]{405}; \xref[\emph{b}]{418}).

\end{minor}

\section
\subsection

The Accusative is used with \latin{in} and \latin{sub} to express the
Place \emph{Whither something moves} (\xref{381}), the Ablative to
express the Place \emph{Where something is} or \emph{is done}
(\xref{433}).

\subsection

The Accusative is regularly used with \latin{subter},
\english{beneath} (\xref{382}).  In poetry, the Ablative \emph{may}
also be used to express the Place \english{beneath Which something is}
or \emph{is done} (\xref[\emph{a}]{382}).

\subsection

The Accusative is regularly used with \latin{super} in the sense of
\english{upon}, \english{at}, or \english{in addition}, the Ablative
in the sense of \english{concerning} (\xref{383}, \xref{435}).

\section
\subsection

\latin{Prīdiē} and \latin{postrīdiē}, \english{the day before} and
\english{the day after}, generally take the Accusative (of
Time-Relation), but sometimes the Genitive (of Connection,
\xref[\emph{c}]{380}).

\begin{minor}

\subsection

\latin{Clam}, \english{secretly}, is regularly an Adverb in Cicero’s
time, but takes the Ablative once.  In early Latin it is either an
Adverb, or a Preposition with the Accusative (\xref[2]{407}).

\subsection

\latin{Palam} is \emph{generally an Adverb}, but occasionally a
Preposition with the Ablative after Cicero’s time (\xref[1,
  \emph{a}]{407}).

\subsection

\latin{Tenus}, \english{up to} (postpositive; rare in Cicero’s time),
generally takes the Ablative, but sometimes the Genitive
(\xref[3]{407}).

\end{minor}

\chapter*{General Forces of the Latin Moods and Tenses}

\contentsentry{B}{General Forces of the Moods and Tenses}

\begin{minor}

\section[\textsc{General Introduction}]
\subsection

The Latin Subjunctive is made up of remains of two moods which in the
parent speech had different forms: the Subjunctive, expressing the two
distinct ideas of Will and Anticipation (I and~II under \xref{462}),
and the Optative, expressing the five distinct ideas of Wish,
Obligation or Propriety, Natural Likelihood, Possibility, and Ideal
Certainty (III–VII under \xref{462}).

The probable development of these forces of the two moods was as
follows:

\begin{enuma}

\item

In its earliest use in the parent speech, the Subjunctive probably
expressed Will.\footnote{\emph{Will} has regard to something felt by
  the speaker to lie within his control; \emph{Wish}, to something
  felt to lie outside of his control.} Next, it was also used to
express Anticipation (Expectation, Futurity).  Compare English “you
will” and “he will,” the regular form for the Future, and the
(unfortunately) growing use of “I will” in place of “I shall” (the
proper Future form), as in “I will be late, if I don’t hurry.”

\pagebreak

\item

In its earliest use in the parent speech, the Optative probably
expressed Wish (Desire, etc.).\footnotemark[\thefootnote] Next, it was
also used to express something \emph{generally desirable}, i.e.\ an
act that was obligatory or proper in a \emph{class of cases}
(“should,” “ought,” as in “the priest should put on a white robe
in sacrificing,” the original feeling being “it is desirable that
the priest should,” etc.).  Next, the use of the mood was extended to
\emph{individual} cases of obligation or propriety.  Next, the mood
was employed, just as English “should” and “ought” may be, to
express what was \emph{naturally likely} to happen, as in “there
should be white violets next week.”  Next, in cases where there were
difficulties in the way, the force of natural likelihood was weakened
to that of \emph{possibility} (“may perhaps”).  And finally, in cases
where the circumstances were strongly favorable, this same force of
natural likelihood was strengthened to that of a \emph{certainty of
  the mind}, i.e. an \emph{ideal certainty} (“would certainly”).

\end{enuma}

\subsection

The Latin Subjunctive inherited all these powers.  In addition,
several constructions (VIII–XII under \xref{462}) arose from two or
more sources each (Composite Origin; \xref[3]{315}), and others (XIII
and~XIV under \xref{462}) through the influence of one or more
constructions upon another (Analogy; \xref[4]{315}).

\end{minor}

\section

Mood is the expression, through the form of the Verb, of certain
\emph{attitudes of mind} toward an act or state.  Thus:
\begin{examples}

\latin{adestō},
\english{let him be present}
(attitude of commanding)

\latin{adsit},
\english{may he be present!}
(attitude of wishing)

\latin{nē adsit},
\english{lest he be present}
(attitude of fearing)

\latin{adest},
\english{he is present}
(attitude of recognizing a fact)

\end{examples}

\section

In English, mood-ideas are expressed mainly by auxiliaries.  Thus, “I
\emph{will} go,” “you \emph{shall} go,” “he \emph{should} go,”
“he \emph{may} go,” “he \emph{would} go,” etc.  In Latin, they are
expressed mainly by the mere \emph{form} (\emph{mood}) of the verb.

\begin{minor}

\subsubsection

But many attitudes of mind can be expressed only by special words,
\emph{combined with} an Infinitive, e.g.\ the attitude of Hesitation,
as in \latin{dubitō adesse}, \english{I hesitate to be present}; the
attitude of Suspicion, as in \latin{suspicor eum adesse}, \english{I
  suspect that he is present}; the attitude of Haste, as in
\latin{properō adesse}, \english{I hasten to be present}.

\subsubsection

Certain other ideas can be expressed \emph{either} by the mood
\emph{or} by a special word, combined with an Infinitive.  Thus one
may say either \latin{eat}, \english{let him go} (Volitive;
\xref[3]{501}) or \latin{volō eum īre}, \english{I want him to go}
(\xref{587}).  In the former, the \emph{mood} is volitive, in the
latter, the \emph{meaning} of the verb \latin{volō}.

\end{minor}

\pagebreak

\section

The Latin moods, with the principal ideas of which they are the
expression, are as follows:

\subtitle{\textsc{Table of the Principal Forces of the Latin Moods}}

\medskip

\noindent\term{Imperative}

Of \term{Peremptory Command} (as in \emph{work hard}, \emph{succeed}).

\medskip

\noindent\term{Subjunctive}

\begin{enumI}

\itemhead{A.} \emph{Simple}

\item
Of \term{Will} (Volitive Subjunctive, as in \emph{I \textsc{will}
  succeed, he \textsc{shall} succeed}).

\item
Of \term{Anticipation} (Anticipatory Subjunctive, as in \emph{until I
  \textsc{shall} succeed, he \textsc{shall} succeed}, \emph{etc.}).

\item
Of \term{Wish} (Optative Subjunctive, as in \emph{\textsc{may} I
  succeed!  \textsc{may} he succeed!}).

\item
Of \term{Obligation} or \term{Propriety} (as in \emph{he
  \textsc{should} succeed}, meaning \emph{it is his duty to succeed}).

\item
Of \term{Natural Likelihood} (as in \emph{he \textsc{should} succeed},
meaning \emph{he is likely to succeed}).

\item
Of \term{Possibility} (Potential Subjunctive, as in \emph{perhaps he
  \textsc{may} succeed}).

\item
Of \term{Ideal Certainty} (as in \emph{he \textsc{would} succeed}).

\smallskip

\itemhead{B.} \emph{Composite \(Fusion\)}

\item

Of \term{Actuality} (Fact) in Consecutive Clauses (as in \emph{so that
  he succeeds}).

\item

Of \term{Condition} (as in \emph{if he should succeed}).

\item

Of \latin{Proviso} (as in \emph{let him only succeed}, \emph{provided
  he succeeds}).

\item

Of \term{Request} or \term{Entreaty} (as in \emph{let him do this}).

\item

Of \term{Consent} or \term{Indifference} (as in \emph{let him do it},
\emph{he may do it}).

\smallskip

\itemhead{C.} \emph{By Analogy}

\item

Of \term{Indirect Discourse} (generally no change in English).

\item

By \term{Attraction} (generally no change in English).

\end{enumI}

\medskip

\noindent\term{Indicative}

Of \term{Actuality}, i.e.\ Fact (as in \emph{he \textsc{has}
  succeeded}, \emph{\textsc{is} succeeding}, etc.).

\medskip

\begin{minor}

\subsubsection

The Volitive Subjunctive is so named from \latin{volō}, \english{I
  will} (cf.\ English “volition”); the Anticipatory from the English
word “anticipate,” i.e.\ \english{look forward to},
\english{expect}, \english{foresee}; the Optative from \latin{optō},
\english{I wish}; the Potential from \latin{possum}, \english{I can}
or \english{may}.  The Subjunctive of Ideal Certainty is so named
because, though it \emph{asserts} just as much as the Indicative does, it
does not, like that mood, assert a \emph{fact}, but only a
\emph{mental} certainty,—a certainty that something \emph{would be}
true, or \emph{would have been} true, under conditions that may be
imagined.

\end{minor}

\section

In certain uses the Present and Future Indicative resemble the
Subjunctive (\xref{571}, \xref{572}).  In certain others, the Present
Indicative resembles the Future Indicative (\xref{571}).

\headingG{General Uses of the Negative Particles (for Reference)}

\contentsentry{C}{General Uses of the Negatives \latin{nē} and
  \latin{nōn}}

\section
\subsection

The Sentence-Negative for Imperative, Volitive, or Optative ideas is
\latin{nē}; for other ideas,\footnote{All these others\versionB*{ (in
    the finite verb)} are ideas of Statement (or corresponding
  Questions), except the Anticipatory idea, which was
  \emph{originally} one of Statement.} \latin{nōn}.

For \latin{nē}, the corresponding \emph{connective} (\english{and
  not}, \english{nor}) is \latin{nēve} or \latin{neu}; for
\latin{nōn}, it is \latin{neque} or \latin{nec}.

\subsubsection

\latin{Nē\ellipsis quidem}, \latin{nihil}, \latin{numquam}, \latin{nēmō},
and \latin{nūllus}, \english{not even}, \english{nothing},
\english{never}, etc., are used with all kinds of mood-ideas.

Thus, with a Volitive,
\latin{nihil fēcerīs},
\english{do nothing},
\apud{Att.}{7, 8, 2};
\latin{numquam sīrīs},
\english{never permit},
\apud{Liv.}{1, 32, 7};
with an Indicative,
\latin{nihil fēcit},
\english{he did nothing},
\apud{Verr.}{5, 5, 11};
\latin{numquam patiētur},
\english{he will never allow \(it\)},
\apud{Phil.}{6, 3, 6}.

\begin{minor}

\subsubsection

\textbf{Exceptional Uses with Imperative, Volitive, or Optative
  ideas.}  In Ciceronian Latin \latin{neque} (for \latin{nēve})
occurs, though \emph{after positive expressions} only, as follows:
with the Imperative once (\latin{habē\ellipsis nec\ellipsis exīstimā},
\apud{Att.}{12, 22, 3}); in independent Prohibitions (\xref[3]{501})
often (\latin{moveor\ellipsis nec\ellipsis putāverīs}, \apud{Ac.}{2, 46,
  141}); in independent Requests (\xref[1]{530}) occasionally
(e.g.\ \latin{respuātur nec\ellipsis haereat}, \apud{Cael.}{6, 14}); in
dependent Volitive Clauses occasionally (e.g.\ in the clause of
Purpose, \latin{ut\ellipsis praetermittam neque appellem}, \apud{Verr.}{3,
  48, 115}).

In poetic and later Latin \latin{neque} is used more freely for
\latin{nēve}, and even after \emph{negatives}.

In double Prohibitions, \latin{neque\ellipsis neque\dots}, as well as
\latin{nē\ellipsis nēve\dots}, are occasionally employed in all periods
(e.g.\ \latin{neque dēfīat neque supersit}, \apud{Men.}{221};
\latin{neque dēdiderīs nec posuerīs}, \apud{Rep.}{6, 23, 25}).

\subsubsection

In poetry after Cicero’s time, \latin{nōn} is occasionally used in
Wishes in the true Optative (\xref[1]{511}) without \latin{utinam}
(e.g.\ \latin{nōn intermisceat}, \apud{Ecl.}{10, 5}), and even with
the Imperative (e.g.\ \latin{nōn onerāte}, \apud{Ov.\ A.~A.}{3, 129}).

\subsubsection

\latin{Nōn} is freely used in all periods to negative the meaning of a
single word.
\begin{examples}

\latin{pauca nūntiāte meae puellae nōn bona dicta},
\english{take a brief message, not a kindly one, to my mistress};
\apud{Catull.}{11, 15}.

\end{examples}

\end{minor}

\subsection

But the Negative \emph{changed} in certain constructions:
\begin{enuma}

\item

In consequence of the \emph{weakening} of an original force.

\begin{minor}

Thus the feeling of Volition was weak in many Questions of Volitive
origin (\xref{503}) and wholly disappeared in the Exclamation of
Surprise.  Hence \latin{nōn} came to be the negative in \emph{all} these
Questions or Exclamations.

\end{minor}

\item

In consequence of the \emph{change} of an original force.

\begin{minor}

Thus the Optative and Volitive Subjunctives gave rise, in certain kinds
of sentences, to the idea of Obligation (“ought,” “should”; see
\xref{512}).  But this idea is one of \emph{statement}, and, \emph{as such},
naturally took the negative \latin{nōn} or \latin{neque}.  Similarly,
the Subjunctive with \latin{utinam} is of Potential descent, and must
originally have taken the negatives \latin{nōn} and \latin{neque}; but
it came to have the meaning of a Wish and, in consequence, to take
\latin{nē} and \latin{nēve} (\xref[1]{511}).

\end{minor}

\end{enuma}

\subsection

On the other hand, an original Negative may \emph{survive} in
occasional or even frequent use, \emph{alongside of} a new one.
\begin{minor}

Thus the original \latin{nē}, as well as \latin{nōn}, is found in
Statements of Obligation or Propriety (\xref[1]{513}), and the
original \latin{nōn}, as well as \latin{nē}, in Wishes with
\latin{utinam} (\xref[1]{511}).

\end{minor}

\subsection

The negative for the Infinitive, Participle, Gerund, and Gerundive is
\latin{nōn}.

\headingB{General Forces of the Latin Tenses}

\chapter{A. Ordinary Forces}

\section

Tense is the expression, through the form of the Verb, of \emph{ideas
  of time}.

\section
\subsection

\term{Tenses of the Stage.}  An act may be represented as \emph{in a
  certain stage of advancement} at a time which is in mind, namely as
completed, in progress,\footnote{\label{ftn:s466:} The phrases
  \emph{in progress}, \emph{progressive}, \emph{going on}, and
  \emph{incomplete} all mean substantially the same thing, and will be
  used interchangeably.} or yet to come. Thus:
\begin{examples}

\latin{aedificāveram},
\english{I had built}
(act completed)

\latin{aedificābam},
\english{I was building}
(act in progress)

\latin{aedificātūrus eram},
\english{I was going to build}
(act yet to come)

\end{examples}

\begin{minor}

\subsubsection

The Tenses of the Stage may also be called \emph{Tenses of the
  Situation} (State of Affairs), or \emph{Descriptive Tenses}, since
they tell \emph{how things were}, \emph{are}, or \emph{will be}, at
the time which is in mind.  These phrases will be used
interchangeably.

\end{minor}

\subsection

\term{Aoristic Tenses.}  \emph{Or}, an act may be represented \emph{in
  summary} (i.e.\ as a \emph{whole}).  Thus:
\begin{examples}

\latin{aedificāvī},
\english{I built}

\end{examples}

\section
\subsection

An act is generally seen as in a certain stage only \emph{when
  referred to} some particular time which is in mind.  Hence the
tenses of the stage are generally \emph{Relative} (i.e.\ relatively
\emph{present}, relatively \emph{past}, or relatively \emph{future}).

\begin{minor}

\subsubsection

The particular time \emph{with reference to which} an act is seen as
in a certain stage may conveniently be called either the Point of
Reference or the Point of View.

\end{minor}

\subsection

\versionA{An act can be thought of as a whole only if looked at
  \emph{without} reference to any particular time.  Hence the aoristic
  tenses are \emph{Absolute}.}

\versionB*{An act thought of as a whole (i.e.\ aoristically) may be
  looked at either without, or with, reference to a particular time,
  i.e.\ either \emph{Absolutely} or \emph{Relatively}.

\begin{minor}

\subsubsection

The aoristic tenses of the Indicative are always absolute (examples in
\xref{478}).  The Subjunctive tenses, when used with aoristic force,
are sometimes absolute (examples in \xref{478}), sometimes relative
(examples in \xref[\emph{b}]{477}).

\end{minor}
}

\headingB{Meanings of the Tenses of the Indicative, in Detail}

\contentsentry{C}{Tenses of the Indicative}

\section

The tenses of the Indicative are as follows:\footnote{The tenses of
  the Passive correspond, e.g.\ \latin{domus aedeficābātur},
  \latin{aedificāta erat}, \latin{aedificanda erat}, \english{the
    house was building}, \english{had been built}, \english{was going
    to be built}.}

\subsection

The \term{Present Indicative} represents an act as \emph{in progress at
the time of speaking} (Progressive Present).  Thus \latin{aedificat},
  \english{he is building}.

\begin{minor}

\subsubsection

The \term{Present Indicative} may also represent a present act
\emph{seen aoristically}.  Thus \latin{aedificat}, \english{he
  builds}.

\subsubsection

The \term{Present Indicative} may express a permanent truth or custom
(Universal Present).  Thus \latin{libenter hominēs id quod volunt
  crēdunt}, \english{men readily believe what they want to believe};
\apud{B.~G.}{3, \emend{123}{18}{16}, 6}.

\subsubsection

For the Historical use of the Present, see~\xref[1]{491}.

\end{minor}

\subsection

The \term{Imperfect Indicative} represents an act as \emph{in progress
  at a past time}.  Thus \latin{aedificābat}, \english{he was
  building}.

\subsection

The \term{Future Indicative} represents an act as \emph{in progress at
  a future time}.  Thus \latin{aedificābit}, \english{he will be
  building}.

\begin{minor}

\subsubsection

The \term{Future Indicative} may also represent a \emph{future act
  seen aoristically}.  Thus \latin{aedificābit}, \english{he will
  build}.

\end{minor}

\subsection

The \term{Perfect Indicative}, in the sense of a \emph{Present
  Perfect},\footnote{Often called the Perfect Definite.}  represents
an act as, \emph{at the time of speaking}, \emph{completed}.  Thus
\latin{aedificāvit}, \english{he has built}.
\begin{minor}

\subsubsection

The \term{Perfect Indicative}, in the sense of a \emph{Past
  Aorist},\footnote{\label{ftn:s468:3}Often called the Perfect
  Indefinite, or the Historical Perfect.} represents a \emph{past act
  seen aoristically}.  Thus \latin{aedificāvit}, \english{he built}.

\end{minor}

\subsection

The \term{Past Perfect Indicative} (commonly called Pluperfect)
represents an act as, \emph{at a past time}, \emph{completed}.  Thus
\latin{aedificāverat}, \english{he had built}.

\subsection

The \term{Future Perfect Indicative} represents an act as, \emph{at a
  future time}, \emph{completed}.  Thus \latin{aedificāverit},
\english{he will have built}.

\subsection

The \term{Periphrastic Futures} represent acts as, \emph{in the
  present}, \emph{past}, or \emph{future} respectively, \emph{yet to
  come}.\footnote{The \emph{periphrastic futures} of the Active and
  Passive, taken together, may conveniently be called the Tenses of
  Impending Action.}  Thus \latin{aedificātūrus est}, \latin{erat}, or
\latin{erit}, \english{he is}, \emph{was}, or \emph{will be},
\emph{about to build}.

\begin{note}

Notice that the Present Indicative serves for both the Present Aorist
and the Present Progressive (\latin{aedificō}, \english{build} and
\english{am building}); the Future for the Future Aorist and the
Future Progressive (\latin{aedificābō}, \english{shall build} and
\english{shall be building}); and the Perfect for the Past Aorist and
the Present Perfect (\latin{aedificāvī}, \english{built} and
\english{have built}).

\end{note}

\headingB{Meanings of the Tenses of the Subjunctive, in Detail}

\contentsentry{C}{Tenses of the Subjunctive}

\section

Each Subjunctive tense has the force \emph{of the Indicative tense of
  the same name}; and, in addition, each has a \emph{future} force.
Accordingly,

\section
\subsection

The Imperfect Subjunctive expresses an act as, \emph{at a certain past
  time}, (1)~in progress, or (2)~yet to come; the Past Perfect
expresses an act as, \emph{at a certain past time}, (1)~already
completed, or (2)~yet to come (and thought of as in a completed
state\footnote{\label{ftn:244:1}Note that the Past Perfect Subjunctive
  thus fills the place of a \emph{Future Perfect Subjunctive from a
    past point of view}, and the Perfect Subjunctive the place of a
  \emph{Future Perfect Subjunctive from a present or future point of
    view}.}); the Present expresses an act as, \emph{at the present
  time}, (1)~in progress, or (2)~yet to come; and the Perfect
expresses an act as, \emph{at the present time}, (1)~already
completed, or (2)~yet to come (and thought of as in a completed
state\footnotemark[\thefootnote]).

The Subjunctive has no special tenses for the third great division of
time,—the Future,—but uses over again the tenses belonging to the
Present, namely, the so-called Present and Perfect.

\textsc{Summary.} The Imperfect and Past Perfect Subjunctive are
\emph{tenses of a past point of view}, while the Present and Perfect
Subjunctive are \emph{tenses of the present} or \emph{future point of
  view}.

\begin{minor}

\subsubsection

In Wishes, Conditions, and Conclusions, the Imperfect and Past Perfect
Subjunctive refer to either the \emph{present} or the \emph{past}, and
represent the act as \emph{contrary to fact}.
See~\xref[\emph{a}]{510}; \xref[\emph{a}, remark]{581}.

\end{minor}

\subsection

The Subjunctive has its \emph{Aorists} also, with the same names as
the Aorists of the Indicative, namely, the Perfect and the Present;
thus \latin{rogās cūr ae\-de\-fi\-cā\-ve\-ram}, \english{you ask why I
  \textsc{built}}; \latin{rogās cūr aedificem}, \english{you ask why I
  \textsc{build}}; \latin{aes aliēnum faciō, ut aedificem}, \emph{I am
  borrowing money, in order that I \textsc{may} build}.  The
Imperfect, too, may be used with aoristic meaning; thus \latin{aes
  aliēnum fēci, ut aedificārem}, \english{I borrowed money, in order
  that I \textsc{might build}}.

\subsection

Like the Indicative tenses, the tenses of the Subjunctive have the
power of expressing an act or state relatively (i.e.\ as relatively
\emph{past}, relatively \emph{present}, or relatively \emph{future} or
\emph{subsequent}\footnote{\label{ftn:s470:2}In Consecutive Clauses
  (\xref[3, \emph{a}]{519}; \xref[1, \emph{e}]{521}), the act mostly
  takes place \emph{after} that which brought it about,
  i.e.\ \emph{subsequently}.}).

\subsection

The Subjunctive possesses periphrastic forms, corresponding to those
of the Indicative, to express an act as, \emph{at a certain time}, yet
to come, e.g.\ \latin{aedificātūrus esset}, or \latin{sit},
\english{he was}, \english{is}, or \english{will be},
\english{\textsc{going} to build}.

\begin{minor}

\subsubsection

These Periphrastic Futures are used when the other forms would be
ambiguous; hence in Indirect Questions of Fact (\xref{537}),
\versionA*{ in} Consecutive Clauses (\xref{521}),\versionA*{ in}
Causal-Adversative Clauses (\xref{523}), 
and\versionB*{ (generally)}\versionA*{ in}
\latin{quīn}-Clauses after \latin{nōn dubitō}
(\xref[3, \emph{b}]{521}). Thus:
\begin{examples}

\remember\1{\latin{rogābit}}

\latin{\1{rogāvit} quid factūrus essem},
\english{asked what I was going to do}
(past situation).

\latin{\1{rogat} \ditto[quid] \ditto[factūrus] sim},
\english{asks what I am going to do}
(present situation).

%%* unfortunate line break

\latin{\1{rogābit} \ditto[quid] \ditto[factūrus] sim},
\english{will ask what I am \emph{(shall then be)} going to do}
(future situation).

\end{examples}

\end{minor}

\headingB{Meanings of the Tenses of the Imperative}

\contentsentry{C}{Tenses of the Imperative}

\section

The so-called Present Imperative refers to \emph{the immediate
  future}, the Future Imperative to \emph{the more remote future}.
\begin{examples}

\latin{aedeficā}, \english{build} (now).

\latin{cum redieris, aedificātō},
\english{build after you return}.

\end{examples}

\headingB{Meanings of the Tenses of the Infinitive}

\contentsentry{C}{Tenses of the Infinitive}

\section

The tenses of the Infinitive express an act as, at the time of the
verb on which they depend, \emph{completed} (Perfect Infinitive),
\emph{in progress} (Present Infinitive), or \emph{yet to come} (Future
Infinitive).  They cannot, of themselves alone, show in which of the
three divisions of time the act expressed by them belongs.  They are
thus purely \emph{relative}.

\begin{sidebyside*}
Present, \latin{aedificāre}, \english{to be building}
&
\latin{aedificārī}, \english{to be building} (to be being built)
\\
Perfect, \latin{aedificāvisse}, \english{to have built}
&
\latin{aedificātus esse}, \english{to have been built}
\\
Future, \latin{aedificātūrus esse}, \english{to be going to
  \emph{(intending to}) build}
&
\latin{aedificātum īrī}, \english{to be going to be built}
\end{sidebyside*}

\subsubsection

Like the Indicative and Subjunctive tenses, the tenses of the
Infinitive have the secondary power of expressing an act as
\emph{prior}, \emph{contemporaneous}, or \emph{future} to the time
which is in mind.

Thus
\latin{dīcit sē aedificāvisse},
\english{he says that he has built}
(he says \latin{aedificāvī},
\english{I have built});
\latin{dīxit sē aedificāvisse},
\english{he said that he had built};
\latin{dīcit aedeficāre},
\english{he says that he is building}
(he says \latin{aedificō});
\latin{dīxit sē aedificāre},
\english{he said that he was building};
\latin{dīcit sē aedificātūrum esse},
\english{he says that he shall build}
(he says \latin{aedificābō}, or \latin{aedificātūrus sum});
\latin{dīxit sē aedificātūrum esse},
\english{he said that he should build}.

\subsubsection

These three tenses may also be used aoristically in dependence upon
the present tense of a verb of \emph{saying}, \emph{thinking}, or the
like.  Thus \latin{dīcit sē aedificāvisse}, \english{he says that he
  built} (he says \latin{aedificāvī}, \english{I built}).

\subsubsection

For verbs having no Future Infinitive, this form is replaced by
\latin{fore} or \latin{fu\-tū\-rum esse} with \latin{ut} and the
Subjunctive, in either voice; and the same equivalent \emph{may} be
used for the Future Infinitive of \emph{any verb}.
\begin{examples}

\latin{magnam in spem veniēbat fore utī pertinācia dēsisteret},
\english{\(Caesar\) was coming to have great hopes that \(Ariovistus\)
  would give up his obstinacy}
(that it would be the case that he would\dots);
\apud{B.~G.}{1, 42, 3}.

\latin{futūrum utī tōtīus Galliae animī ā sē āverterentur},
(he said)
\english{that the affections of the whole of Gaul would be turned away
  from him};
\apud{B.~G.}{1, 20, 4}.

\end{examples}

\begin{minor}

\subsubsection

The auxiliary \latin{posse} with the Present Infinitive of \emph{any}
verb may form an equivalent for the Future Infinitive.
\begin{examples}[9pt]

\latin{tōtīus Galliae sēsē potīrī posse spērant},
\english{they hope to be able to master the whole of Gaul}
(= \latin{sēsē potītūrōs esse spērant}), \english{they hope that they
  will master\dots};
\apud{B.~G.}{1, 3,~\emend{201}{8}{7}}.

\end{examples}

\end{minor}

\headingB{Meanings of the Tenses of the Participle}

\contentsentry{C}{Tenses of the Participle}

\section

The tenses of the Participle express an act as, at the time of the
main verb, \emph{completed} (Perfect Passive Participle), \emph{in
  progress} (Present Active Participle), or \emph{yet to come} (Future
Active and Future Passive Participle).  They are thus, like the
tenses of the Infinitive (\xref{472}), purely relative.
\begin{Longtable}[c]{l@{ }l}

Present Active,  & \latin{aedificāns}, \english{building} \\
Perfect Passive, & \latin{aedificātus}, \english{built}\\
Future Active,   & \latin{aedificātūrus}, \english{about to build}\\
Future Passive,  & \latin{aedificandus}, \english{about to be built}

\end{Longtable}

\headingB{Uses of Indicative, Subjunctive, and Imperative Tenses\\
in Combinations of Verbs}

\begin{minor}

\section[\textsc{\small Introductory}]
\subsection

The subordinate act generally belongs in \emph{the same temporal
  scene} with the main act, and so necessarily in the same great
division of time with it (\emph{both} in the past, \emph{both} in the
present, or \emph{both} in the future).  Naturally, it is generally
looked at as it was, is, or will be, \emph{at the time of that act},
and so is expressed by a \emph{relative} tense.  Hence the facts noted
in~\xref{476}.

\subsection
But the subordinate act \emph{may} belong \emph{in a different
  division of time} from the main act, or, though belonging in the
same division of time, it \emph{may} be looked at \emph{absolutely},
so far as tense is concerned.  Hence the facts noted in~\xref{478}.

\subsection
Rarely, there is a purely mechanical harmony of tenses.
See~\xref{480}.

\end{minor}

\section

Any combination of tenses is possible for which the corresponding
combination of \emph{meanings} is possible.  In addition, combinations
with purely mechanical harmony sometimes occur.  The possibilities may
be tabulated as follows:
\smallskip
\begin{Longtable}{cll}
\pushright\1{III.}

\emph{A}.
& \group{With true tense-force\\(Indicative or Subjunctive)}
& \groupL{\1{I.}\enskip Acts in Temporal Relation (\xref{476})\\
          \1{II.}\enskip Acts not in Temporal Relation (\xref{478})}
\\[\bigskipamount]

\emph{B}.
& \group{Without true tense-force\\(Subjunctive only)}
& \groupL{\1{III.}\enskip Tenses in Mechanical Harmony (\xref{480})}
\end{Longtable}

\headingE{Usual Combinations of Tenses\\ (“Sequence of Tenses”)}

\contentsentry{C}{Usual Combinations of Tenses (“\textnormal{Sequence}”)}

\subtitle{(Acts \emph{in temporal relation})}

\section

A main\footnote{I.e.\ any tense on which another depends.} tense of
the past\footnote{\label{ftn:247:2}\label{ftn:s476:2}The tenses of the
  past, Indicative or Subjunctive (Perfect Aorist, Past Perfect, and
  Imperfect), are often called “secondary” or “historical,” and the
  tenses of the present or future (Present, Future, Present Perfect,
  and Future Perfect), “primary.”} is generally accompanied by a
dependent Imperfect or Past Perfect, and a main tense of the present
or future by a dependent Present, Perfect, Future, or Future
Perfect.\footnote{Since Periphrastic Futures contain an \latin{erat},
  \latin{est}, etc., this statement includes them.}
\begin{sidebyside}

\cc{1}{\textsc{Indicative}}
& \cc{1}{\textsc{Subjunctive}}
\endhead

\latin{Helvētiī reliquōs Gallōs virtūte praecē\-dunt, quod ferē
  cotīdiānīs proeliīs cum Ger\-mā\-nīs contendunt}, \english{the
  Helvetians surpass the rest of the Gauls in prowess, because they
  engage in almost daily encounters with the Germans}; \apud{B.~G.}{1,
  1, 4}.
&
\latin{id autem difficile nōn est, cum tantum equitātū valeāmus},
\english{this, however, is not difficult, since we are so strong in
  calvalry};
\apud{B.~C.}{3, 86, 4}.
\\
\latin{Caesar ālāriōs omnīs in cōnspectū hos\-ti\-um cōnstituit, quod
  minus mul\-ti\-tū\-di\-ne mīlitum legiōnāriōrum prō hos\-ti\-um
  numerō valēbat}, \english{Caesar placed all his auxiliaries in sight
  of the enemy, because he was weak in the number of his legionaries
  as compared with that of the enemy}; \apud{B.~G.}{1, 51, 1}.  &
\latin{hī cum per sē minus valērent, quod an\-tī\-qui\-tus summa auctōritās
  erat in Hae\-du\-īs, Germānōs atque A\-ri\-o\-vis\-tum si\-bi adiūnxerant},
\english{the latter, since they were not strong in themselves, because
  in early times the largest influence lay with the Haeduans, had
  bound the Germans and Ariovistus to themselves}; \apud{B.~G.}{6, 12,~2}.
\end{sidebyside}

\section

These relations between main and subordinate verbs appear not only
when the latter are subordinate in form, but also when, though
subordinate in \emph{feeling}, they are \emph{independent} in
\emph{form} (paratactic; \xref{227}); for the relations are, in fact,
relations of thought.  And they hold for \emph{Indicatives and
  Subjunctives alike}.
\begin{examples}

\latin{\textsc{relinquēbātur} ūna per Sēquanōs via, quā Sēquanīs
  invītīs propter angustiās īre nōn \textsc{poterant}.  Hīs cum suā
  sponte persuādēre nōn \textsc{possent}, lēgātōs ad Dumnorīgem
  Haeduum \textsc{mittunt}, ut eō dēprecātōre ā Sēquanīs
  \textsc{im\-pe\-trā\-rent}.\linebreak  Dumnorīx apud Sēquanōs plūrimum
  \textsc{poterat}},
\english{there \textsc{was left} only the way through the land of the
  Sequani; and by this, on account of the narrowness of the pass, they
  \textsc{were unable} to go without the consent of the Sequani.
  Failing \emph{(when they \textsc{were unable})} to persuade the
  latter by themselves, they send \(= \textsc{sent}\) ambassadors to
  Dumnorix the Haeduan, in order that, through his intercession, they
  \textsc{might obtain} what they wanted of the Sequani.  Dumnorix
  \textsc{possessed} great influence with the Sequani};
\apud{B.~G.}{1, 9, 1–3}.
(The externally independent \latin{relinquēbātur} and the dependent
\latin{poterant} and \latin{possent} alike express a \emph{past
  situation}, i.e.\ the situation existing at the time when
\latin{mittunt} took place; and \latin{poterat} again expresses
situation for the next main act, to which the narrative moves on.
\latin{Poterant} and \latin{possent} differ only in mood, being
\emph{identical in point of tense-meaning}.  \latin{Impetrārent}
expresses an act belonging in the same general temporal scene with the
rest, but yet to come,—a \emph{past purpose}\emend{34}{).}{.)}

\latin{quāpropter \textsc{dēcernite} dīligenter, ut
  \textsc{īnstituistis}, ac fortiter.  \textsc{Habētis} eum cōnsulem
  quī pārēre vestrīs dēcrētīs nōn \textsc{dubitet}},
\english{therefore decide with careful thought, as you \textsc{have
    begun}, and boldly.  You \textsc{have} a consul who \textsc{has}
  no hesitation in following your decisions};
\apud{Cat.}{4, 11, 24}.
(\latin{Dēcernite} expresses a command looked at from the
\emph{present}; \latin{īnstituistis}, \latin{habētis}, and \latin{nōn
  dubitet} express the \emph{present} situation, under which the act
of \latin{dēcernite} is to be performed.)

\end{examples}

\subsubsection

If the meaning is that of Contrariety to Fact (\xref{581}) the
Imperfect and Past Perfect are necessarily employed \emph{after a main
  verb of any time}, except as shown in \xref[\emph{b}, 2)]{581}.
\begin{examples}

\latin{moriar, sī magis gaudērem, sī id mihi accidisset},
\english{may I die, if I should be more pleased if it had happened to myself};
\apud{Att.}{8, 6, 3}.

\end{examples}

\subsubsection

\versionA{The relative tenses of the Indicative all express
  \emph{situation}.  So do the relative tenses of the Subjunctive,
  when used with the same force as the corresponding tenses of the
  Indicative.  When used with future force, they may express either
  the idea of future (or subsequent) \emph{situation}, or a mere
  \emph{aoristic} idea for future (or subsequent) time.}

\versionB*{The relative tenses of the Indicative all express
  \emph{situation}; the aoristic\linebreak tenses of the Indicative do not
  (\xref[2, \emph{a}]{467}).

The Subjunctive tenses, when used with relative force, may express
either the idea of situation, or the aoristic idea.  Thus, either a
situation, or an act seen in summary, may be put as relatively future
to a past time.}

\begin{minor}

Thus \latin{ut suppeteret} in \apud{B.~G.}{1, 3, 1} expresses a
past-future \emph{situation}; \latin{nē committeret}, \apud{B.~G.}{1,
  22, 3}, a past-future act seen \emph{aoristically}; \latin{ut nōn
  possent}, \apud{B.~G.}{3, 15, 3}, a subsequent \emph{situation} in
the past (in \emph{tense}, \latin{possent} = \latin{poterant});
\latin{ut redintegrārent}, \apud{B.~G.}{2, \emend{202}{27}{26}, 1}, a result seen
\emph{aoristically}, but \emph{in temporal relation} (namely, as
\emph{subsequent}) to the time of the main verb.  With the last,
compare the absolute tense in \latin{ut āmīserit}, \xref{478}, and the
explanation there given.

\end{minor}

\subsubsection

In any expression of thought, the most important acts or states are
selected for the principal statements, and expressed by
\emph{absolute} tenses (\xref[2]{467}), which may therefore be called
\emph{principal} (or \emph{leading}) tenses.  The side-lights upon
these principal acts or states are expressed by \emph{relative} tenses
(\xref[1]{467}), which may therefore be called \emph{auxiliary} (or
\emph{helping}) tenses.  Thus, in the first example above,
\latin{mittunt} is a principal tense, while \latin{relinquēbātur},
\latin{poterant}, and \latin{possent} are auxiliary tenses.

\headingE{Less Usual Combinations of Tenses\\
    (“E\lowercase{xceptions to the} S\lowercase{equence}”)}

\contentsentry{C}{Less Usual Combinations (“\textnormal{Exceptions to
    the Sequence}”)}

\subtitle{(Acts \emph{not} in temporal relation)}

\section

A main tense is sometimes accompanied by a tense belonging to a
different division of time, or by an absolute tense belonging to the
same division of time.
\begin{sidebyside}

\cc{1}{\textsc{Indicative}}
& \cc{1}{\textsc{Subjunctive}}
\endhead

\latin{id hōc facilius iīs persuāsit, quod un\-di\-que locī nātūrā
  Helvētiī con\-ti\-nen\-tur},
\english{he \textsc{found} it easier to persuade them for the reason
  that the Helvetians, by the very character of the country,
  \textsc{are hemmed} in on all sides};
\apud{B.~G.}{1, 2, 3}.
(Main act in past, while the reason is an ever-present one.)
&
\latin{fīlius pertimuit nē ea rēs mihi nocēret, cum praesertim adhūc
  stilī poenās dem},
\english{my son \textsc{feared} that the affair might do me harm,
  especially since I \textsc{am} still \textsc{paying} the penalty for
  my writing};
\apud{Fam.}{6, 7, 1}.
(Past fear, with reason still present.)
\\
\latin{nunc incipiunt crēdere fuisse hominēs Rōmānōs hāc quondam
  continentiā, quod iam nātiōnibus exterīs incrēdibile vidēbātur};
\english{now they \textsc{begin} to believe that there once were
  Romans possessed of this self-restraint, which thing \textsc{was
    beginning to seem} incredible to foreign nations};
\apud{Pomp.}{14, 41}.
&
\latin{cuius reī tanta est vīs ut Ithacam sapientissimus vir
  immortālitātī an\-te\-pō\-ne\-ret},
\english{so great \textsc{is} the power of this \emph{(love of
    country)} that the wisest of men preferred his Ithaca to
  immortality};
\apud{De~Or.}{1, 44, 196}.
(In tense, \latin{antepōneret} = \latin{antepōnēbat}.)
\\
\latin{ab senātū impetrātum (est); tantum āfuit ut ex incommodō aliēnō
  occāsiō peterētur},
\english{the request \textsc{was} granted by the senate; so far
  \textsc{were} they from taking advantage of another’s dilemma};
\apud{Liv.}{4, 58, 2}.
(\latin{Āfuit} is in the same temporal scene with \latin{impetrātum
  (est)}, but is looked at absolutely.)
&
\latin{ita est mulcātus ut vītam āmīserit},
\english{he \textsc{was} so maltreated that he \textsc{lost} his
  life};
\apud{Mil.}{14, 37}.
(\latin{Āmīserit} is in the same temporal scene with \latin{est
  mulcātus}, but is looked at absolutely.  Similarly \latin{ut
  dēfuerit}, \apud{B.~G.}{2, \emend{115}{21}{20}, 5}.)
\\
\latin{superiōra illa, quamquam ferenda nōn fuērunt, tamen, ut potuī,
  tulī},
\english{the earlier things, though they \textsc{were} intolerable, I
  nevertheless \textsc{bore}, as well as I \textsc{could}};
\apud{Cat.}{1, 7, 18}.
(The tense of \latin{fuērunt} is absolute.)
&
\latin{cum ab hōrā septimā ad vesperum pug\-nā\-tum sit, āversum hostem
  vidēre nēmō potuit},
\english{though the battle \textsc{lasted} from the seventh hour till
  evening, nobody \textsc{could} catch sight of an enemy’s back};
\apud{B.~G.}{1, 26, 2}
\\
\latin{id fēcit, quod nōluit eum locum vacāre},
\english{he \textsc{did} this because he \textsc{did} not wish this
  territory to lie open};
\apud{B.~G.}{1, 28, \emend{203}{4}{3}}.
(The tense of \latin{nōluit} is absolute.)
&
\latin{fuit mīrificā vigilantiā, quī suō tōtō cōnsulātū somnum nōn
  vīderit},
\english{he \textsc{was} a wonderfully wide-awake man, for in his
  whole consulship he \textsc{knew} no sleep};
\apud{Fam.}{7, 30, 1}.
\end{sidebyside}

\subsubsection

Unrelated tenses are less frequent in Subjunctive than in Indicative\linebreak
clauses, because the bond of thought is generally closer between a
Subjunctive clause and the main sentence.

\begin{minor}

Thus a Purpose necessarily exists \emph{at the time of the main act}
which is performed in order to bring it about, and its tense will
accordingly be a relative one.

\end{minor}

\section

The combination of a Present with a Past or Future Aorist, or of these
with each other, is natural and common.\footnote{This is because it is
  the very nature of the aorists to express an act as it looks
  \emph{from the present}. The mind, standing \emph{at} the present,
  looks easily in either direction.}
\begin{examples}

\latin{illī aliēnum, quia poēta fuit, post mortem etiam expetunt},
\english{they claim a foreigner, even after his death, because he was
  a poet};
\apud{Arch.}{9, 19}.

\latin{quid fēcerim, nārrābō},
\english{I’ll tell you what I did};
\apud{De~Or.}{2, 48, 198}.

\end{examples}

\pagebreak

\subsubsection

Such a Past Aorist may of course be accompanied by dependent tenses of
the Past.  The Past Aorist thus often serves as a \emph{bridge of
  passage} from a past temporal scene to a present one, or \emph{vice
  versa}.
\begin{sidebyside}

\cc{1}{\textsc{Dependent Indicative}}
& \cc{1}{\textsc{Dependent Subjunctive}}\\

\latin{\textsc{Quaerō} cūr bona quae \textsc{possidēbat} nōn
  \textsc{vēndiderit}},
\english{my question \textsc{is} why he \textsc{did} not sell the
  goods of which he \textsc{was} possessed};
\apud{Quinct.}{24, 76}.
&
\latin{\textsc{Quaerāmus} quae tanta vitia \textsc{fuerint} in ūnicō
  fīliō, quārē is patrī \textsc{dis\-pli\-cē\-ret}},
\english{\textsc{Let} us \textsc{inquire} what so great faults there
  \textsc{were} in this only son, that he \textsc{was} obnoxious to
  his father};
\apud{Rosc.\ Am.}{14, 41}.
Similarly \apud{Cat.}{3, 9, 21}, and often.
\end{sidebyside}

\headingE{(Rare) Mechanical Harmony of Subjunctive Tenses}

\contentsentry{C}{(Rare) Mechanical Harmony of Subjunctive Tenses}

\section

A Subjunctive tense is sometimes put, without true tense-meaning, into
mechanical harmony with that of a Subjunctive main verb.

This happens especially in the Subjunctive by (Mechanical) Attraction
(\xref{539}), and in Indirect Questions depending upon constructions
Contrary to Fact.
\begin{examples}

\latin{respondērem sī, quem ad modum parātī essēmus, scīrem},
\english{I should answer, if I knew in what fashion we were
  \emph{(i.e. \english{are})} prepared};
\apud{Att.}{7, 18, 1}.

\end{examples}

\headingG{Alternative Tense-Usages}

\contentsentry{C}{Alternative Tense-Usages}

\section[Tenses in Clauses Dependent on a Present Perfect]

The Present Perfect covers both the past act and the present result.
Hence an act dependent upon a Present Perfect may be seen
\emph{either} in connection with the Past \emph{or} in connection with
the Present, and consequently either kind of tense may be used.
\begin{sidebyside}

\cc{1}{\textsc{Thought mainly concerned}}
& \cc{1}{\textsc{Thought mainly concerned}}\\

\cc{1}{\textsc{with the Present}}
& \cc{1}{\textsc{with the Past}}\\[\smallskipamount]

\latin{haec tibi \textsc{scrīpsī}, quia dē omnibus quae mē vel
  dēlectant vel angunt tēcum loquī \textsc{soleō}},
\english{this I \textsc{have written} to you, for the reason that I
  \textsc{am} in the habit \emph{(present reason)} of talking over
  with you everything that \textsc{gives} me pleasure or annoyance},
(continued on the right.)
&
\latin{deinde, quod dūrum \textsc{exīstimābam} tē frau\-dā\-re voluptāte
  quam ipse \textsc{ca\-pi\-ē\-bam}},
\english{and, secondly,\versionA{ for the reason that}\versionB*{ (I
    have written) because} it \textsc{seemed} to me
  \emph{(reason of the time of beginning the writing)} unkind to cheat
  you of the pleasure which I myself \textsc{was taking}};
\apud{Plin.\ Ep.}{5, 1, 12}.
\\
\latin{rērum nātūrā nūllam nōbīs \textsc{dedit} cognitiōnem fīnium, ut
  ūllā in rē statuere \textsc{possīmus}, ‘quātenus,’}
\english{nature \textsc{has} not equipped us with power to draw the
  line so that we \textsc{are} in any matter able to determine ‘how
  far’};
\apud{Ac.}{2, 29, 92}.
(Present Result; that which nature has accomplished \emph{is} not
that\dots)
&
\latin{mentēs enim hominum au\-dā\-cis\-si\-mō\-rum scelerātae ac nefāriae nē
  vō\-bīs no\-cē\-re \textsc{possent}, ego \textsc{prōvīdī}},
\english{for I \textsc{have} taken precautions, to the end that the
  wicked and abominable purposes of abandoned men \textsc{should} do
  you no harm};
\apud{Cat.}{3, 12, 27}.
(Past Aim; in what I have done, my purpose \emph{was}\dots)
\end{sidebyside}

\pagebreak

\section[Permanent Truths in Clauses Dependent on a Verb of the Past]

\subsection

That which is permanently true was of course true in the past, and, if
connected in thought with a past act, will generally be seen and
stated as it \emph{then was} (i.e.\ by a tense of past
situation).\footnote{E.g.\ you \textsc{were} a kind-hearted fellow:
  that’s why you helped me.}

\subsection

But a permanent truth will occasionally be seen and stated as such
(i.e.\ by a tense of present situation), in spite of its being
connected in thought with a past act.\footnote{E.g.\ you \textsc{are}
  a kind-hearted fellow: that’s why you helped me.}
\begin{sidebyside}

\cc{1}{\textsc{Indicative}}
& \cc{1}{\textsc{Subjunctive}}\\[\smallskipamount]

\cc{2}{(1) \term{Permanent truth in its aspect at a past time} (tenses
  of past situation)}\\[\smallskipamount]

\latin{mōns altissimus impendēbat},
\english{a lofty mountain \textsc{overhung}};
\apud{B.~G.}{1, 6, 1}.
(It still does, of course, when Caesar writes; but \emph{that} fact is
not the important one.)
&
\latin{certior factus est montīs quī im\-pen\-dē\-rent ā maximā multitūdine
  te\-nē\-rī},
\english{he was informed that the mountains which \textsc{overhung}
  were held by a very large body of men};
\apud{B.~G.}{3, 2, 1}.
(The \emph{tense}-meaning of \latin{impendērent} is the same as that
of \latin{impendēbat} opposite.)
\\[\medskipamount]

\cc{2}{(2) \term{Permanent truth in its general aspect} (present
  tenses)}\\[\smallskipamount]

\latin{id (frūmentum) erat perixiguum, quod sunt loca aspera ac
  montuōsa},
\english{the grain was very scanty, because the district \textsc{is}
  rough and mountainous};
\apud{B.~C.}{3, 42, 5}.
&
\latin{hīc, quantum in bellō fortūna possit, cognōscī potuit},
\english{at this juncture it was possible to recognize how great
  \textsc{is} the power of Fortune in war};
\apud{B.~G.}{6, 35, 2}.
\end{sidebyside}

\begin{note}

Both the Romans and we of English speech more frequently describe
permanent facts of \emph{external} nature by putting them in the same
temporal scene with the main act; but we are more likely than the
Romans to put permanent facts of \emph{human} nature as always true
(present tense).

\end{note}

\headingB{Tenses of the Subjunctive Depending upon an Infinitive}

\section

The Tenses of Subjunctive Clauses depending upon an Infinitive express
meaning in the same way as Subjunctive Clauses depending upon Finite
Verbs (\xref{475}–\xref{482}), and the combinations are accordingly
the same.

It should be borne in mind that the Perfect Infinitive, like the
Perfect Indicative, may be used either as a Past Aorist or as a
Present Perfect.
\begin{examples}

\latin{dīcō patefactum esse Pontum, quī anteā clausus fuisset},
\english{I say that Pontus was laid open, which before that time had
  been closed};
\apud{Pomp.}{8, \emend{204}{20}{20–21}}.
(The tense of \latin{fuisset} is relative, the point of view being
that of the Past Aorist Infinitive \latin{patefactum esse}.
Cf.\ \latin{hī cum valērent}, \xref{476}.)

\latin{cuius adventū ipsō, tametsī ille ad maritimum bellum vēnerit,
  tamen impetūs hostium repressōs esse intellegunt},
\english{by whose mere arrival, though he came for a war by sea, they
  know that none the less the attacks of \(these\) enemies were
  checked};
\apud{Pomp.}{5, 13}.
(The tense of \latin{vēnerit} is absolute.  Cf.\ \latin{cum pugnātum
  sit}, \xref{478}.)

\latin{id mihi īnstituisse videntur quod neque in vulgus disciplīnam
  efferrī velint,
\linebreak
 neque\dots},
\english{this \emph{(custom)} they seem to me to have established for
  the reason that they do not wish their knowledge to be spread
  abroad, nor\dots};
\apud{B.~G.}{6, 14, 4}.
(\latin{Quod velint} is put as a permanent truth in its general
aspect.  Cf.\ \xref[(2)]{482}.)

\end{examples}

\chapter{B. Special Forces Gained by Various Tenses}

\contentsentry{C}{Special Forces \emend{70}{g}{G}ained by \emend{71}{v}{V}arious Tenses}

\section[Tenses of Habitual\footnotemark Action, or of
  Attempted\footnotemark Action]

\footnotetext{Also called “Repeated” or “Customary.”}

\footnotetext{Also called “Conative.”}

The tenses expressing action as \emph{going on} (Imperfect, Present,
Future) gained also the power of expressing \emph{habitual action} or
\emph{attempted action}.
\begin{examples}

\latin{Carthāgine quotannīs bīnī rēgēs creābantur},
\english{at Carthage two kings used to be elected annually};
\apud{Nep.\ Hann.}{7, 4}.
(Habitual action.)

\latin{quī poenam removet},
\english{who is for removing the penalty};
\apud{Cat.}{4, 4, 7}.
(Attempted action.  Similarly \latin{faciēbās}, \english{you were
  trying to do}; \apud{Cat.}{1, \emend{205}{5}{6}, 13}.)

\latin{C.\ Flāminiō restitit agrum Gallicum dīvidentī},
\english{resisted Gaius Flaminius, who was trying to apportion the
  Gallic territory};
\apud{Sen.}{4, 11}.
(Attempted action.)

\end{examples}

\begin{minor}

\subsubsection

But a past habit \emph{may} be looked at aoristically, and so be
expressed by the Past Aorist (Perfect).
\begin{examples}

\latin{maiōrēs sīc habuērunt},
\english{our ancestors held this view};
\apud{Cato Agr.}{Intr.~1}.

\end{examples}

\end{minor}

\section

Expressions of duration of time (e.g.\ \latin{iam diū}, \latin{iam
  dūdum}, or a noun of time), when added to a tense of action in
progress (Imperfect, Present, or Future) show the action to have been
\emph{already} going on for the amount of time indicated.
\begin{examples}

\latin{tē iam dūdum hortor},
\english{I have long been urging you};
\apud{Cat.}{1, 5, 12}.

\latin{iam dūdum flēbam},
\english{I had long been weeping};
\apud{Ov.\ Met.}{3, \emend{206}{656}{654}}.

\latin{cum iam amplius hōrīs sex pugnārētur},
\english{when the battle had now been going on for more than six
  hours};
\apud{B.~G.}{3, 5, 1}.

\latin{sēcum ipse diū volvēns},
\english{having pondered for a long time};
\apud{Sall.\ Iug.}{113, 1}.

\end{examples}

\section
\subsection

The Imperfect may be used to express the discovery of a state of
affairs \emph{existing before}.
\begin{examples}

\latin{“quid agitur, Aeschine?” “Ehem, pater mī, tū hīc erās?”}
\english{“what’s going on, Aeschinus?” “Why, father, were you
  here?”}
\apud{Ad.}{\emend{35}{901}{905}}.

\end{examples}

\subsection

The Future may be used to express the discovery of a state of affairs
\emph{now existing}.
\begin{examples}

\latin{sīc erit},
\english{you’ll find it so}
(it will be so);
\apud{Ph.}{801}.

\end{examples}

\section

In several verbs the Present Perfect, Past Perfect, and Future Perfect
have come to express a present, past, or future \emph{state}.  Thus
\latin{nōvī}, (\english{have learned}) \english{know},
%
\versionA*{\latin{nōveram}, \english{knew},
%
\latin{nōverō}, \english{shall know},
%
\latin{cognōvī}, \english{know},}
%
\latin{cōnsuēvī},%
    \versionA*{ (\english{have formed the habit}) \english{am in the habit}}%
    \versionB{ \english{am accustomed}},
%
\latin{meminī},\versionA*{ (\english{have recollected})} \english{remember},
%
\latin{ōdī},\versionA*{ (\english{have come to dislike})} \english{hate}%
\versionA{. Similarly \latin{coepī}, \english{begin}}%
\versionB*{, \latin{coepī}, \english{begin}, etc.
Similarly, sometimes, in other verbs.  Thus
  \latin{cōnstiterant}, \english{had taken their stand}, =
  \english{were standing}; \apud{B.~G.}{1, 24, 3}}.

\section

The Perfect of Experience\footnote{Also called the “Gnomic
  Perfect.”} is sometimes used in the place of a general present.
\begin{examples}

\latin{lūdus enim genuit īram},
\english{for contests in sport beget hatred}
(have in the past begotten);
\apud{Ep.}{1, 19, 48}.

\end{examples}

\section

The Perfect is sometimes used to indicate an act or State as \emph{no
  longer existing}.
\begin{examples}

\latin{fuit Īlium},
\english{Ilium is no more}
(Ilium once was);
\apud{Aen.}{2, \emend{207}{324}{325}}.

\end{examples}

\section

\term{Energetic} or \term{Emphatic Perfect}.  Tenses of completed
action are often employed instead of tenses of incomplete action, to
express \emph{haste}, \emph{thoroughness}, or \emph{positiveness}.
(Cf.\ English “begone” for “go.”)
\begin{examples}

\latin{“rape mē: quid cessās?” “Fēcerō,”}
\english{“hurry me there: why are you so slow?” “I’ll do it at
  once”};
\apud{Ph.}{882}.

\latin{periimus},
\english{we are dead and buried};
\apud{Trin.}{515}.
(We have perished.  Cf.\ \latin{perierīs} in \xref[1]{511}.)

\latin{sit īnscrīptum in fronte ūnīuscuiusque, quid dē rē pūblicā
  sentiat},
\english{be it written once for all on every man’s forehead what are
  his sentiments with regard to the Commonwealth};
\apud{Cat.}{1, 13, 32}.

\latin{illōs monitōs volō},
\english{I want them to understand well\dots};
\apud{Cat.}{2, 12, 27}.

\latin{tē interfectum esse convenit},
\english{you ought to be killed and have done with it};
\apud{Cat.}{1, 2, 4}.
(\latin{Interficī} would have meant simply \english{be killed}.
Cf.\ \latin{trucīdārī}, \xref[3, \emph{a}]{582}.)

\latin{neque ego ausim},
\english{nor should I for a moment venture},
\apud{Sat.}{1, 10, 48}.

\latin{nē dubitārīs mittere},
\english{do not hesitate at all to send it};
\apud{Att.}{1, 9, 2}.

\end{examples}

\subsubsection

In dependent clauses and in the Future Perfect Indicative (except in
\latin{vī\-de\-rō}, \latin{vīderis}, etc.)\ this tense-use mostly passes
away, after early Latin.  Elsewhere it remains common in Ciceronian
and later prose; but in Prohibitions (\xref[3]{501}) and Softened
Assertions (\xref[1, \emph{b}]{519}) the tense seems to have become
nearly or quite stereotyped, and must thus have lost much of its
original sharpness.

\section[Picturesque Uses of the Tenses]

By the use of tenses properly belonging to the present point of view,
a past scene may be brought before the mind as \emph{now existing},
with its events \emph{now} taking place, its purposes \emph{now}
entertained, etc., as follows:

\subsection

A past event may be represented as now taking place, or a past
situation as now existing (Historical Present\footnote{This use might
  at any time arise through liveliness of imagination.  But it more
  probably is a survival from an early use (see \ftn*{303}{},
  footnote).} or Present Perfect).
\begin{examples}

\latin{quod iussī sunt, faciunt},
\english{they do as commanded}
(what they have been told to do, they do);
\apud{B.~G.}{3, 6, 1}.

\end{examples}

\begin{minor}

\subsubsection

The Historical Present is often used, with less vividness, in the
\emph{annalistic style}, giving the effect of copying from records
made from time to time as the events occurred.
\begin{examples}

\latin{Silvius deinde rēgnat.  Is Aenēam Silvium creat},
\english{next Silvius reigns.  He begets Aeneas Silvius};
\apud{Liv.}{1, 3, 6}.

\end{examples}

\end{minor}

\subsection

In subordinate clauses of any kind, attached to such picturesque
tenses, the same effect may be given\footnote{When the dependent
  clause \emph{precedes}, the pictureseque tense is less common.}
(e.g.\ a past purpose may be represented as \emph{now} entertained),
or the sober tenses of the past may be employed.
\begin{examples}

\latin{quaecumque ad oppugnātiōnem opus sunt, noctū comparantur},
\english{whatever \textsc{is} need\-ed \emph{(picturesque tense)} for the
  siege \textsc{is got together} \emph{(picturesque tense)} at night};
\apud{B.~G.}{5, 40, \emend{208}{5}{6}}.

\latin{Dumnorīgī custōdēs pōnit, ut quibuscum loquātur scīre possit},
\english{he \textsc{sets} spies \emph{(picturesque)} over Dumnorix,
  that he \textsc{may be} able \emph{(picturesque)} to learn with whom
  he \textsc{is communicating} \emph{(picturesque)}};
\apud{B.~G.}{1, 20, \emend{92}{6}{5}}.

\latin{Helvētiī cum id quod ipsī diēbus vīgintī aegerrimē cōnfēcerant,
  illum ūnō diē fēcisse intellegerent, lēgātōs ad eum mittunt},
\english{the Helvetians, when they \textsc{were aware} \emph{(sober
    tense)} that what they themselves \textsc{had} with the utmost
  difficulty \textsc{accomplished} \emph{(sober tense)} in twenty
  days, he had done in one day, \textsc{send} ambassadors to him
  \emph{(picturesque tense)}};
\apud{B.~G.}{1, 13, \emend{88}{12}{2}}.

\end{examples}

\subsection

In poetry, a condition and conclusion which are really contrary to
fact are sometimes picturesquely presented as still undetermined
(i.e.\ as lying \emph{in the future}).
\begin{examples}

\latin{volat Diōrēs, spatia et sī plūra supersint, trānseat prior},
\english{Diores flies along, and were there to be more space, he would
  be first to cross};
\apud{Aen.}{5, 325}.
Cf.\ the sober \latin{cēpissent praemia, nī fūdisset}, \apud{}{5,
  232}.

\end{examples}

\section[Tenses of Rapid Action]

The Past Perfect and the picturesque Present Perfect are occasionally
used to indicate the \emph{rapid succession of events}, intervening
acts being passed over.
\begin{examples}

\latin{vixdum dīmidium dīxeram, intellēxerat},
\english{hardly had I said the half, he had understood}
(= he understood in an instant);
\apud{Ph.}{594}.

\latin{intonuēre polī},
\english{instantly the heavens thunder};
\apud{Aen.}{1, 90}.

\end{examples}

\section[Epistolary Tenses]

In letters, acts are sometimes put as they \emph{will appear} to the
receiver.  Hence the Imperfect or the Past Aorist instead of the
Present, and the Past Perfect instead of the Present Perfect.
\begin{examples}

\latin{haec ego scrībēbam hōrā noctis nōnā: Milō campum iam tenēbat.
  Mārcellus candidātus ita stertēbat, ut ego vīcīnus audīrem},
\english{I am writing \emph{(was writing)} this at the ninth hour of
  the night.  Milo is already in the field.  Marcellus, who is a
  candidate, is snoring so loud that I hear him next door};
\apud{Att.}{4, 3, 5}.
(= \latin{scrībō}, \latin{tenet}, \latin{stertit}, \latin{audiam}.)

\end{examples}

\begin{minor}

\subsubsection

When the epistolary tenses are used, the expressions of time of course
change accordingly. “Yesterday” (\latin{herī}) becomes “the day
before” (\latin{prīdiē}), and “to-day” (\latin{hodiē}) becomes
“that day” (\latin{eō diē}).

\end{minor}

\section

In general, Latin expresses relations of time more exactly than
English.
\begin{examples}

\latin{quibus ego sī ēdictum praetōris ostenderō, concident},
\english{if I show them the praetor’s edict, they will fall};
\apud{Cat.}{2, 3, 5}.
(In Latin, more exactly, \emph{shall have shown}, because this act
comes first.)

\latin{nihil est maius quam ut faveat ōrātōrī is quī audiet},
\english{nothing is more important than that the man that hears shall
  be favorably disposed toward the speaker};
\apud{De~Or.}{2, 42, 178}.
(In Latin, more exactly, \english{the man that shall hear}, matching
the real time of \latin{faveat}.)

\end{examples}

\chapter{The Imperative}

\contentsentry{B}{Uses of the Imperative}

\section
\subtitle{\textsc{Synopsis of the Principal Uses of the Imperative}}

%% VISUAL FORMATTING

\medskip

\begingroup

\leftskip2\normalparindent
\parindent-\normalparindent

Command, Advice or Suggestion, Consent or Indifference,\\ 
Request or Entreaty, Prayer (\xref{496}), Concession, Proviso, Condition
(\xref{497}).

\endgroup

\section

The Imperative expresses \emph{Peremptory Command}, \emph{Advice} or
\emph{Suggestion}, \emph{Consent} or \emph{Indifference},
\emph{Request} or \emph{Entreaty}, or \emph{Prayer}.  The negative is
\latin{nē}.

The Present refers to the \emph{immediate} future, the future to some
\emph{distinctly future} time, or to \emph{future time in general}
(hence regularly used in laws, treaties, and maxims).
\begin{examples}

\latin{līctor, conligā manūs},
\english{lictor, bind his hands};
\apud{Liv.}{1, 26, 7}.
(Command.)

\latin{mihi crēde, oblīvīscere caedis atque incendiōrum},
\english{take my advice, put bloodshed and conflagration out of your
  mind};
\apud{Cat.}{1, 3, 6}.
(Advice.)

\latin{tibi permittō: posce},
\english{I give you permission: ask her in marriage};
\apud{Trin.}{384}.
(Consent.)

\latin{dīc sōdēs},
\english{tell me, please};
\apud{Ep.}{1, 16, 31}.
(Request.)

\latin{audī Iuppiter},
\english{hear thou, Jupiter};
\apud{Liv.}{1, 32, 10}.
(Prayer.)

\latin{crās petitō, dabitur},
\english{ask to-morrow, you shall have it};
\apud{Merc.}{770}.

\end{examples}

\begin{minor}

\subsubsection

The Imperative is sometimes accompanied by \latin{age}
(\latin{agite}), \english{come}.
\begin{examples}

\latin{vāde age vocā zephyrōs},
\english{come, go and call the breezes},
\apud{Aen.}{4, 223}.

\end{examples}

\subsubsection

\latin{Quīn}, \english{pray do}, is often prefixed to the Imperative
in early Latin.  The usage is rare in Cicero, but revives in later
Latin.  (For the origin of the force of \latin{quīn}, see
\xref[\emph{a}, remark]{545}.)
\begin{examples}

\latin{quīn omitte mē},
\english{do let me alone};
\apud{Ph.}{486}.
Similarly \apud{Aen.}{4, 547}.

\latin{quīn sīc attendite},
\english{pray look at the matter thus};
\apud{Mil.}{29, 79}.

\end{examples}

\subsubsection

The Future Imperatives \latin{mementō}, \english{bear in mind}
(e.g.\ \apud{Cat.}{2, 3, 5}), and \latin{scītō}, \english{know}
(e.g.\ \apud{Cat.}{2, 10, 23}), are used in place of the Presents,
which are rare or lacking.  \latin{Habētō} is used in the sense of
\english{you are to understand} (e.g.\ \apud{Am.}{2, 10}).

\subsubsection

The Imperative is not used in Prohibitions except in early Latin,
legal Latin, poetry, and (rarely) later prose.

\end{minor}

\section

The Imperative is often used:

\subsection

As a Substantive Sentence (cf.\ \xref[3, \emph{c}]{502}).
\begin{examples}

\latin{tū tacētō: hoc optimum est},
\english{keep quiet: that is best};
\apud{Rud.}{1029}.

\end{examples}

\subsection

In Concessions, Provisos, or Conditions (cf.\ \xref{532}, \xref{529},
\xref[1]{504}).
\begin{examples}

\latin{estō: at certē\dots},
\english{be it so: yet at any rate\dots};
\apud{Heaut.}{572}.
(Concession.)

\latin{spectā, tum sciēs},
\english{look, and then you’ll know};
\apud{Bacch.}{1023}
(= if you look).

\end{examples}

\section

Since the Imperative expresses a Direct Command, it cannot be used in
Indirect Discourse, but \emph{must be replaced by the Volitive
  Subjunctive} (Subjunctive of Command, \xref[3]{501}, becoming
dependent, \xref{538}).

\chapter{The Subjunctive}

\contentsentry{B}{Uses of the Subjunctive}

\section
\subtitle{\textsc{Synopsis of the Principal Uses of the Subjunctive}}

\begin{subjunctivesynopsis}

\cc{1}{\textbfsc{independent sentences}}
& \cc{1}{\textbfsc{dependent clauses}} \\[\smallskipamount]
\endhead

\cc{2}{\normalsize\textbf{Volitive Subjunctive}} \\[\smallskipamount]

Resolve (\xref[1]{501})

Proposal, Suggestion, or Exhortation (\xref[2]{501})

Command or Prohibition (\xref[3]{501})
&
Volitive Determinative Clause (\xref[1]{502})

Volitive Descriptive Clause (\xref[1]{502})

Clause of Plan or Purpose (\xref[2]{502})

Volitive Substantive Clause (\xref[3]{502})

Dependent Clause of Fear or Anxiety (\xref[4]{502})

Commands and Prohibitions in Indirect Discourse (\xref{538})
\\\cc{2}{}\\[\medskipamount]%\separator

Question of Deliberation or Perplexity, etc.\ (\xref{503})

Question or Exclamation of Surprise or Indignation (\xref{503})
&
Dependent Question of Deliberation or Perplexity, etc.\ (\xref{503})
\\\cc{2}{}\\[\medskipamount]%\separator

Volitive Condition (\xref[1]{504})
&
Generalizing Clause in the Second Person Singular Indefinite
(\xref[2]{504})

Clause of Imaginative Comparison with \latin{quasi},
etc.\ (\xref[3]{504})
\\\cc{2}{}\\[\medskipamount]%\separator

Subjunctive with \latin{nēdum}, \english{still less} (\xref{505})
\\\cc{2}{}\\[\medskipamount]%\separator

\pagebreak

\cc{2}{\normalsize\textbf{Anticipatory Subjunctive}} \\[\smallskipamount]

(No independent uses)
&
Anticipatory Determinative Clause (\xref[1]{507})

Anticipatory Descriptive Clause (\xref[1]{507})

Anticipatory Substantive Clause with \latin{ut} (\xref[2]{507})

Indirect Question of Anticipation (\xref[3]{507})

Clause of Anticipated Act with \latin{antequam} or \latin{priusquam}:
\begin{indented}

Act anticipated and \english{prepared for}  (\xref[4, \emph{a}]{507})

Act anticipated and \english{forestalled}   (\xref[4, \emph{b}]{507})

Act anticipated and \english{insisted upon} (\xref[4, \emph{c}]{507})

Act anticipated and \english{deprecated}    (\xref[4, \emph{d}]{507})

\end{indented}
Clause of Anticipated Act with \latin{dum}, \latin{dōnec}, or
\latin{quoad} (\xref[5]{507})

Past-Future Clauses in general (\xref{508}; \xref{509})
\\\cc{2}{}\\[\medskipamount]%\separator

\cc{2}{\normalsize\textbf{Optative Subjunctive}} \\[\smallskipamount]

Wish (\xref[1]{511})

Optative Condition (\xref[1, \emph{b}]{511})
&
Optative Substantive Clause (\xref[2]{511})
%\\\cc{2}{}\\[\medskipamount]%\separator

\\

\cc{2}{\normalsize\textbf{Subjunctive of Obligation or Propriety}}
\\[\smallskipamount]

Statement or Question of Obligation or Propriety (\xref[1]{513})
&
Dependent Question of Obligation or Propriety (\xref[1]{513})

Clause of Obligation or Propriety with \latin{quod}, \latin{quārē},
etc.\ (\xref[2]{513})

Relative Clause or \latin{ut}-Clause after \latin{dignus},
etc.\ (\xref[3]{513})

Clause with \latin{ut} after \latin{tantī}, etc.\ (\xref[4]{513})

Substantive Clause of Obligation or Propriety (\xref[5]{513})
\\\cc{2}{}\\[\medskipamount]%\separator

% \pagebreak

\cc{2}{\normalsize\textbf{Subjunctive of Natural Likelihood}}
\\[\smallskipamount]

Question of Natural Likelihood (\xref[1]{515})
&
Clause of Natural Likelihood with \latin{quī}, \latin{quārē},
etc.\ (\xref[2]{515})

Substantive Clause of Natural Likelihood with \latin{ut}
(\xref[3]{515})
\\\cc{2}{}\\[\medskipamount]%\separator

\pagebreak

\cc{2}{\normalsize\textbf{Potential Subjunctive}}
\\[\smallskipamount]

Potential Statement or Question (\xref[1]{517})
&
Potential Relative Clause (\xref[2]{517})

Potential Substantive Clause (\xref[3]{517})
\\\cc{2}{}\\[\medskipamount]%\separator

\cc{2}{\normalsize\textbf{Subjunctive of Ideal Certainty}}
\\[\smallskipamount]

Statement or Question of Ideal Certainty (\xref[1]{519})

Softened Statement or Question (\xref[1, \emph{b}]{519})
&
Determinative Clause of Ideal Certainty (\xref[2]{519})

Descriptive Clause of Ideal Certainty (\xref[2]{519})

Clause of Ideally Certain Result (\xref[3]{519})

Substantive Clause of Ideal Certainty (\xref[4]{519})
\\[\medskipamount]

\cc{2}{}\\

Conclusions of Ideal Certainty:
\begin{indented}
Less Vivid Future (\xref[1, \emph{a}]{519}; \xref{580})

Contrary to Fact (\xref[1, \emph{a}]{519}; \xref{581})
\end{indented}
%\\\cc{2}{}\\[\medskipamount]%\separator

\\

\cc{2}{\normalsize\textbf{Subjunctive Constructions of Composite
    Origin}}
\\[\smallskipamount]

(No independent uses)
&
Descriptive Clause of Actuality (Fact) with \latin{quī}, etc., or
\latin{cum} (\xref[1]{521})

Clause of Actual Result (Fact) with \latin{ut}, \latin{ut nōn}, or
\latin{quīn} (\xref[2]{521})

Substantive Clause of Actuality (Fact) with \latin{ut}, \latin{ut
  nōn}, or \latin{quīn} (\xref[3, \emph{a} and \emph{b}]{521})

Derivatives of the Descriptive Clause of Fact:
\begin{indented}

Restrictive \latin{quī}-Clause (\xref{522})

Causal or Adversative \latin{quī}-Clause (\xref{523})

Descriptive \latin{cum}-Clause of Situation (\xref{524})

Descriptive \latin{cum}-Clause of Situation, with an Accessory Causal
or Adversative Idea (\xref{525})

Purely Causal or Adversative \latin{cum}-Clause (\xref{526})
\end{indented}
\\\cc{2}{}\\[\medskipamount]%\separator

&
Subjunctive Conditions:
\begin{indented}
Less Vivid Future (\xref{528}; \xref{580})

Contrary to Fact (\xref{528}; \xref{581})
\end{indented}
Dependent Clause of Proviso (\xref{529})
\\\cc{2}{}\\[\medskipamount]%\separator

Subjunctive of Request (\xref[1]{530})
&
Substantive Clause of Request (\xref[2]{530})
\\\cc{2}{}\\[\medskipamount]%\separator

Subjunctive of Consent or Indifference (\xref[1]{531})

Concession of Indifference (\xref[1]{532})
&
Substantive Clause of Consent or Indifference (\xref[2]{531})

Concession of Indifference with \latin{quamvīs} or \latin{quamlibet}
(\xref[2]{532})
%\\\cc{2}{}\\[\medskipamount]%\separator

\\

\cc{2}{\normalsize\textbf{Subjunctive Constructions due to the
    Influence of Other Constructions}}
\\[\smallskipamount]

&
Subjunctive in Subordinate Clauses in Indirect Discourse in:
\begin{indented}
Statements of Fact (\xref{535})

Conditions of Fact (\xref{536})

Questions of Fact (\xref{537})

Commands and Prohibitions (\xref{538})
\end{indented}
Subjunctive by Attraction to a Subjunctive or Infinitive Clause
(\xref{539})

Subjunctive of Repeated Action (\xref{540})
\\\cc{2}{}\\[\medskipamount]%\separator

Generalizing Statement of Fact in Second Singular Indefinite
(\xref{542})
\end{subjunctivesynopsis}

\headingE{The Volitive Subjunctive}

\contentsentry{C}{The Volitive Subjunctive}

\section

The Volitive Subjunctive represents an act or state as \emph{willed}
or \emph{wanted}.  Hence it is used in expressions of \emph{Demand},
\emph{Intention}, or \emph{Endeavor} (English “\emph{I}
\textsc{will},” “\emph{you} \textsc{shall},” “\emph{you}
\textsc{are to},” “\emph{I} \textsc{want} \emph{you to},”
etc.).  The negative is regularly \latin{nē}.

\subsubsection

In independent sentences, the Volitive Subjunctive expresses the will
of \emph{the speaker only}.  In dependent clauses, it regularly
expresses the will of the subject or agent of the principal clause.

\subsubsection

The Present and Perfect generally express a \emph{present} or
\emph{future} demand, intention, or endeavor; the Imperfect and Past
Perfect a \emph{past} demand, intention, or endeavor.

\begin{note}

The \emph{performance} of the act expressed by the Volitive
Subjunctive in the literal uses lies in time relatively \emph{future}.
In the figurative uses (\xref{504}–\xref{505}) the act imaginatively
commanded may lie in time \emph{relatively past}, \emph{relatively
  present}, or \emph{relatively future}.

\end{note}

\section

The Volitive Subjunctive may be used in independent declarative
sentences:

\begin{minor}

\subsection

To express a \term{Resolve} for the speaker’s own action (rarely, and
mainly with \vrb{crēdō} or \vrb{opīnor}).
\begin{examples}

\latin{maneam opīnor},
\english{I’ll stay, I think};
\apud{Trin.}{1136}.

\latin{sed opīnor quiēscāmus},
\english{but I think I’ll stop};
\apud{Att.}{9, 6, 2}.

\end{examples}

\end{minor}

\subsubsection

The regular construction is the Future Indicative (\xref{572}).

\subsection

To express a \term{Proposal}, \term{Suggestion}, or
\term{Exhortation}.
\begin{examples}

\latin{vide sī hoc ūtibile magis dēputās: ipsum adeam Lesbonīcum},
\english{see if you think this idea more practical: I will go to
  Lesbonicus himself};
\apud{Trin.}{748}.
(\latin{Adeam} is a Proposal or Suggestion.)

\latin{resīdāmus, sī placet},
\english{we will take seats, if you please}
(= let us take seats);
\apud{Fin.}{3, 2, 9}.
(\latin{Resīdāmus} is an Exhortation.)

\end{examples}

\subsection

To express a \term{Command} or \term{Prohibition}.
\begin{examples}

\latin{sēcēdant improbī},
\english{let the ill-disposed withdraw};
\apud{Cat.}{1, 13, 32}.

\latin{nē trānsierīs Hibērum! nē quid reī tibi sit cum Saguntīnīs},
\english{do not cross the Ebro!  Let there be no interference on your
  part with the Saguntines};
\apud{Liv.}{21, 44, 6}.

\end{examples}

\subsubsection

In Ciceronian and later prose,
\begin{enumerate}

\item
If addressed to a \emph{general} second person, Commands and
Prohibitions are expressed by the Present Subjunctive.
\begin{examples}

\latin{istō bonō ūtāre, dum adsit; cum absit, nē requīrās},
\english{use this blessing while you have it; when it is gone, do not
  mourn for it};
\apud{Sen.}{10, 33}.

\end{examples}

\item

If addressed to an \emph{individual} second person (or persons),
Commands are expressed by the Imperative; while Prohibitions may be
expressed by the Perfect Subjunctive, or, in a roundabout way, by
\latin{cavē} with a dependent Subjunctive (\xref[3, \emph{b}]{502}),
\latin{vidē} with a dependent \latin{nē}-clause (\xref[3,
  \emph{a}]{502}), or \latin{nōlī} with the Infinitive (\xref{586}).
  The Perfect Subjunctive is the most peremptory or emphatic form, and
  \latin{nōlī} the most courteous.
\begin{examples}

\latin{hoc facitō, hoc nē fēcerīs},
\english{this do, this do not do};
\apud{Div.}{2, 61, 127}.

\latin{nē dubitārīs mittere},
\english{do not hesitate to send};
\apud{Att.}{1, 9, 2}.

\latin{cavē ignōscās, cavē tē misereat},
\english{beware of forgiving, beware of feeling pity};
\apud{Lig.}{5, 14}.

\latin{cavē audiam istuc ex tē},
\english{don’t let me hear that from you}
(= don’t say it);
\apud{Stich.}{37}.

\latin{nōlīte dubitāre},
\english{do not hesitate}
(be unwilling to);
\apud{Pomp.}{23, 68}.

\end{examples}

\end{enumerate}

\subsection

In early Latin, and in the poetical style, both Imperative and
Subjunctive are freely used in any kind of command or prohibition.
Cf.\ \xref[\emph{d}]{496}.

\section

The Volitive Subjunctive may be used in dependent clauses:

\subsection

In \term{Relative Clauses},
determinative\footnote{\label{ftn:260:1}That is, telling \emph{what}
  person or thing is meant.} or
descriptive.\footnote{\label{ftn:260:2}That is, telling \emph{what
    kind} of person or thing is meant (also called “characterizing”
  clauses).}
\begin{examples}

\latin{“cavē.”\dots\ “Quid est quod caveam?”}
\english{“look out.” “What is it that I am to look out for?”}
\apud{Rud.}{\emend{36}{828}{833}}.
(Determinative.)

\latin{Mago locum mōnstrābit quem īnsideātis},
\english{Mago will show you the place which you are to take for an
  ambuscade};
\apud{Liv.}{21, 54, 3}.
(Determinative.)

\latin{saepe stilum vertās, iterum quae digna legī sint scrīptūrus},
\english{use the eraser often, if you mean to write things that shall
  be worth reading a second time};
\apud{Sat.}{1, 10, 72}.
(Descriptive.)

\end{examples}

\subsection

In \term{Clauses of Plan or Purpose},\footnote{\label{ftn:260:3}Such
  clauses are often called “final.”} with \latin{quī},
etc.,\footnote{Any relative may be used.  Thus \latin{ubi},
  \latin{unde}.} \latin{quō}, \latin{ut}, or \latin{nē}.
\begin{examples}

\latin{equitātum quī sustinēret impetum mīsit},
\english{he sent cavalry who were to check the attack}
(= to check);
\apud{B.~G.}{1, 24, 1}.
(Past Purpose.)

\latin{id quō maiōre faciant animō},
\english{that they may do it with greater courage};
\apud{B.~G.}{7, 66, 6}.
(Purpose, picturesquely represented as Present.)

\latin{mihi timōrem ēripe; sī est vērus, nē opprimar; sīn falsus, ut
  timēre dēsinam},
\english{free me of fear; if it is well-founded, that I may not be
  crushed, but if false, that I may cease to fear};
\apud{Cat.}{1, 7, 18}.
(Present Purpose.)

\end{examples}

\begin{minor}

\subsubsection

A Clause of Purpose may be preceded by an adverb of manner or degree,
or by \latin{eō cōnsiliō}, \latin{eā causā}, \latin{idcircō}, etc.
\begin{examples}

\latin{librum petō ā tē ita corrigās nē mihi noceat},
\english{I beg of you to correct the book in such  a way that it shall
  not do me harm};
\apud{Fam.}{6, 7, 6}.

\latin{eō cōnsiliō, ut expugnārent},
\english{with the plan that they should storm};
\apud{B.~G.}{2, 9, 4}.

\end{examples}

\subsubsection

\latin{Quō} is generally used with a comparative, as in
\apud{B.~G.}{7, 66, 6} above.

\subsubsection

A Clause of Purpose is sometimes used parenthetically.
\begin{examples}

\latin{ac nē longum sit, \ellipsis iussimus},
\english{and, to be brief, we ordered\dots}
(in order to be brief, I say at once, we ordered);
\apud{Cat.}{3, 5, 10}.

\end{examples}

\end{minor}

\subsection

In \term{Substantive Clauses}:
\begin{enuma}

\item

With verbs of \emph{will} or \emph{endeavor}.\footnote{Such verbs (or
  phrases) express: (1)~\emph{Will in its simplest form},
  e.g.\ \latin{volō}, \latin{nolō}, \latin{mālō}; (2)~\emph{Demand},
  \emph{Command}, or \emph{Direction}, e.g.\ \latin{flāgitō},
  \latin{postulō}, \latin{poscō}, \latin{imperō}, \latin{mandō},
  \latin{moneō} and its compounds, \latin{hortor} and its compounds,
  \latin{ēdīcō}, \latin{dīcō}, \latin{respondeō}, \latin{scrībō},
  \latin{mittō} (\english{send instructions}), \latin{certiōrem
    faciō}, \latin{prōnūntio}, rarely \latin{iubeō} and \latin{vetō};
  (3)~\emph{Intention}, \emph{Plan}, \emph{Purpose}, or
  \emph{Agreement}, e.g.\ \latin{dēcernō}, \latin{in animum indūcō},
  \latin{animus} or \latin{cōnsilium est}, \latin{scīscō},
  \latin{statuō}, \latin{cēnseō}, \latin{pacīscor}, \latin{convenit}
  (\emph{is is agreed}), \latin{placet} (\emph{it is decided}; in its
  original meaning \english{is pleasing}, this belongs
  under~\emph{c}), \latin{iūs est bellī}; (4)~\emph{Endeavor on One’s
    Own Part}, e.g.\ \latin{labōrō} and its compounds, \latin{īnstō},
  \latin{certō}, \latin{nītor} and its compounds, \latin{videō} and
  \latin{prōvideō}, \latin{cūrō}, \latin{cōnsulō}, \latin{tendō} and
  its compounds, \latin{faciō} and its compounds, \latin{cōnsequor}
  and \latin{adsequor}, \latin{agō}, \latin{operam dō},
  \latin{committō}, \latin{teneō} (\english{insist}) and
  \latin{obtineō}, \latin{est in manū} (\english{it is in one’s
    power}); (5)~\emph{The Giving of an Impulse to Another},
  e.g.\ \latin{moveō}, \latin{incitō}, \latin{suādeō} and
  \latin{persuādeō}, \latin{impellō}, \latin{addūcō}, \latin{indūcō},
  \latin{cōgō}, and \latin{subigō}.}  The connective, if one is used,
is \latin{ut}\footnote{\label{ftn:261:2}\latin{Ut}, when used
  in\versionA{ these constructions}\versionB*{ substantive clauses}, is
  purely formal, having come in, merely \emph{as the
    opposite of} \latin{nē}, from Clauses of Purpose, where it
  originated.  By a natural second step, it was sometimes added to
  \latin{nē} itself\versionB*{ (likewise in clauses of purpose)}.
  \versionA*{Thus \latin{ut nē sit impūne}, \apud{Mil.}{12, 31}.}} or
\latin{nē}.
\begin{examples}

\latin{nē fīliī quidem hoc nostrī rescīscant volō},
\english{I want not even our sons to hear of this};
\apud{Ph.}{819}.
Cf.\ \latin{volō ut faciās}, \apud{Bacch.}{989, \emph{a}}.

\latin{tē hortor ut maneās in sententiā, nēve vim pertimēscās},
\english{I urge you to stand by your opinion, and not to fear
  violence};
\apud{Pomp.}{24, 69}.

\latin{efficiēmus nē nimis aciēs vōbīs cordī sint},
\english{we’ll see to it that you shall not like the battle-line too
  well};
\apud{Liv.}{8, 7, 6}.\footnote{\latin{Faciō}, \latin{efficiō}, and
  \latin{perficiō} may be followed by either a Volitive or a
  Consecutive Clause (\xref[3, \emph{a}]{521}), according as the act
  is presented as \emph{aimed at}, or as \emph{accomplished}.}

\latin{vide nē peccēs},
\english{see that you don’t do a wrong};
\apud{Ph.}{803}.

\end{examples}

\begin{note}[Note 1]

The original Volitive force is often lost, so that the clause
becomes a mere \emph{verb-noun}.
\begin{examples}

\latin{poenam sequī oportēbat, ut ignī cremārētur},
\english{the punishment of being burned alive would follow};
\apud{B.~G.}{1, 4, 1}.

\end{examples}

\end{note}

\begin{note}[Note 2]

A \latin{nē}-Clause with \latin{videō} or \latin{vīsō} may suggest a
\emph{Possibility}.
\begin{examples}

\latin{vidē nē tuum fuerit},
(see to it that it was not your duty)
\english{consider whether it was not your duty}
(= possibly it was);
\apud{Fin.}{3, 3, 10}.
Similarly \apud{Pomp.}{22, 63}.

\end{examples}

\end{note}

\item

With verbs of \emph{hindrance}, \emph{prevention}, or
\emph{check}.\footnote{(1)~\emph{Hindrance}, \emph{Prevention},
  \emph{Check}, or \emph{Falling Short}, e.g.\ \latin{impediō},
  \latin{prohibeō}, \vrb{obstō}, \vrb{obsistō}, \vrb{officiō},
  \vrb{dēterreō}, \vrb{teneō}, \latin{facere nōn possum}, or
  \latin{nōn possum} alone, \latin{nōn est in manū}, \latin{paulum},
  \latin{nōn longē}, etc.\ with \latin{abest} (\latin{quīn});
  (2)~\emph{Avoidance}, e.g.\ \latin{caveō}, \latin{vītō}, \latin{temperō},
  \latin{mē contineō}, \latin{mē ēripiō}, \latin{resistō},
    \latin{repugnō}, \latin{nōn cūnctandum est}, \latin{haud dubia rēs
      vidētur}, \latin{nūlla mora est} (these last with \latin{quīn});
    (3)~\emph{Refusal} or \emph{Hesitation}, e.g.\ \latin{recūsō},
    \latin{dubitō}.} The connective is \latin{nē}, \latin{quīn}, or
  \latin{quōminus}.

\latin{Quīn} is used only after a negative, \latin{quōminus} after
either a negative or a positive, \latin{nē} generally only after a
positive.\footnote{\label{ftn:262:1}The conjunction \latin{quīn}
  (\latin{quī}, \english{whereby}, + \latin{ne}) meant originally
  \english{whereby not}.  \latin{Quōminus} likewise meant
  \english{whereby the less}, \english{whereby not} (\latin{minus}
  being only a weakened negative).

   In all its uses as a conjunction, \latin{quīn} is employed only
   after a negative idea, expressed or implied.}

\begin{examples}

\latin{quis umquam hoc senātor recūsāvit nē putāret?}
\english{what senator ever refused to think this?}
\apud{Clu.}{55, 150}.
The same verb \latin{recūsō}, \english{negatived}, is used
with \latin{quīn} in \apud{B.~G.}{4, 7, 3}, and with \latin{quōminus}
in \apud{B.~G.}{1, 31, 7}.

\latin{cavē nē negēs},
\english{beware of refusing};
\apud{Catull.}{61, 152}.

\latin{dēterrēre nē frūmentum cōnferant},
\english{were deterring them from collecting grain};
\apud{B.~G.}{1, 17, 2}.

\latin{quīn dīcant, nōn est: meritō ut nē dīcant, id est},
\english{that they shall not say it, is not \emph{(in my power)}: that
  they shall not say it with reason, that is \emph{(in my power)}};
\apud{Trin.}{105}.
(\latin{Ut nē} shows that the parallel clause with \latin{quīn} must
be Volitive in feeling.)

\latin{quīn loquar, numquam mē potes dēterrēre},
\english{you can never prevent me from speaking}
(that I shall not speak);
\apud{Amph.}{559}.

\latin{paulam āfuit quīn Vārrum interficeret},
\english{it lacked but little of his killing Varus}
(= \english{he was on the point of\dots});
\apud{B.~C.}{2, 35, 2}.
Cf.\ \latin{neque longius abesse quīn Sabīnus ēdūcat},
\apud{B.~G.}{3, \emend{209}{18}{16}, 4}.

\end{examples}

\begin{note}[Note 1]

These uses come originally from combinations like \latin{recūsō: nē
  putem}, \english{I refuse: I will not believe}; \latin{dēterreō: nē
  cōnferant}, \english{I am deterring them: they shall not collect};
etc.  They were then extended to combinations like \latin{nōn longē
  abest quīn}.

\end{note}

\begin{note}[Note 2]

\latin{Cavē}, as itself suggesting a negative idea, can be used
without \latin{nē}.  Thus \latin{cavē mentiāris}, \english{beware of
  lying}, \apud{Mil.}{22, 60}.

\end{note}

\item

With adjectives, and verbs or phrases of adjective
force.\footnote{Such verbs and phrases represent an action as
  (1)~\emph{good} or \emph{bad}, e.g.\ \latin{melius est},
  \latin{optimum est}\emend{37}{:}{;} (2)~\emph{necessary},
  \emph{seasonable}, \emph{advantageous}, \emph{sufficient},
  \emph{remaining to be done}, or \emph{lacking},
  e.g.\ \latin{necessārium est} or \latin{necesse est}, \latin{opus
    est}, \latin{tempus est}, \latin{rēfert}, \latin{interest},
  \latin{satis est}, \latin{sufficit} (but these two mostly with
  infinitive; \xref{585}), \latin{reliquum est}, \latin{relinquitur},
  \latin{restat}, \latin{sequitur} (when meaning \english{the
    next thing to do is}), \latin{superest}, \latin{abest};
  (3)~\emph{customary}, \latin{ūsitātum est}, \latin{mōs (mōris) est},
  \latin{cōnsuētūdō (cōnsuētūdinis) est}.  Many of these take the
  Infinitive also (\xref{585}), some more frequently than the
  Subjunctive.}  The connective, if one is used, is
\latin{ut}\footnote{Formal \latin{ut}.  See
  footnote~\ref{ftn:261:2}, \pageref{ftn:261:2}.} or
\latin{nē}.
\begin{examples}

\latin{iūs valeat necesse est},
\english{law must prevail};
\apud{Sest.}{42, 92}.
(Let law prevail: it is necessary.
Cf.\ \latin{tacētō: optimum est}, \xref[1]{497}.)

\latin{reliquum est ut dē fēlīcitāte dīcāmus},
\english{it remains for me to discuss the subject of good fortune};
\apud{Pomp.}{16, 47}.
(It remains that I am to discuss\dots)

\end{examples}

\begin{note}

%%* loose line

These are best called, not Substantive Volitive Clauses, but
\term{Substantive
\linebreak
 Clauses of Volitive Origin}; for with most of them
the Volitive feeling has faded out.

\end{note}

\end{enuma}

\subsection

In \term{Clauses of Fear or Anxiety}. The connectives are \latin{nē},
\english{lest} or \english{that}, and \latin{ut} (less frequently
\latin{nē nōn}), \english{lest not}, \english{that not}.
\begin{examples}

\latin{nē eius suppliciō Dīviciācī animum offenderet verēbātur},
\english{he feared that by punishing him he should offend Diviciacus};
\apud{B.~G.}{1, 19, 2}.
(Past fear about the future.)

\latin{vereor nē id fēcerint},
\english{I am afraid that they have done it};
\apud{Caecin.}{2, 4}.
(Present fear about the past.)

\latin{verērī videntur ut habeam satis praesidī},
\english{seem to fear that I have not a sufficient guard};
\apud{Cat.}{4, 7, 14}.
(Present fear about the present.)

\latin{timeō nē nōn impetrem},
\english{I fear I may not get what I ask for};
\apud{Att.}{9, 6, 6}.
(Present fear about the future.)

\end{examples}

\begin{minor}

\subsubsection

\latin{Nē}, \english{lest}, was originally a mere negative adverb (as
in \latin{nē suscēnseat: ti\-meō}, \english{he must not be angry: I am
  afraid}, i.e.\ \english{I am afraid that he will be angry}).
\latin{Nē nōn}, \english{lest not} is the natural opposite of
\latin{nē}.  \latin{Ut}, which means the same as \latin{nē nōn},
probably came into use as the \emph{formal} opposite of \latin{nē}
(footnote~\ref{ftn:261:2}, \pageref{ftn:261:2}).

\subsubsection

The original volitive feeling has entirely faded out from the
construction.

\end{minor}

\section

The Volitive Subjunctive may be used in Questions of Deliberation or
Perplexity; in Questions asking for Instructions; and in Questions or
Exclamations of Surprise or Indignation.  The negative is \latin{nōn}.

The Questions may be independent or dependent.
\begin{examples}

\latin{ēloquar, an sileam?}
\english{shall I speak, or shall I keep silence?}
\apud{Aen.}{3, 39}.

\latin{est certum quid respondeam},
\english{what I shall answer is clear};
\apud{Arch.}{7, 15}.

\latin{quid Rōmae faciam?}
\english{what shall I do in Rome \emph{(= can I)}?}
\apud{Iuv.}{3, 41}.

\latin{“scrībe.” “Quid scrībam?”}
\english{“write.” “What shall I write?”}
\apud{Bacch.}{731}.

\latin{quid faciam imperā},
\english{command me what to do};
\apud{Ph.}{223}.

\latin{“tū nārrā.” “Scelus!  Tibi nārret?”}
\english{“you tell him.” “You rascal! he tell the story under your
  orders?”}
\apud{Ph.}{1000}.

\latin{tū rēbus omnibus cōpiōsus sīs, et dubitēs!}
\english{you a man provided with everything,—and you hesitate!}
\apud{Cat.}{2, 8, 18}.

\end{examples}

\subsubsection

The last example represents the extreme point of development reached
by the construction, in which nothing remains either of the
interrogative idea or of the original idea of Will.

\subsubsection

The construction is sometimes introduced by \latin{ut} or
\latin{utin}\footnote{This type has probably arisen from a Question of
  Perplexity (“how shall?”).  But it \emph{may} have arisen from a
  Potential Question (“how can?”) or through an ellipsis (e.g. “Is
  it possible that?”).} (\latin{utī} plus the interrogative
\enclitic{-ne}), as in \latin{tū ut umquam tē corrigās}, \english{the
  idea of your ever reforming!}  \apud{Cat.}{1, 9, 22}.

\section

The Volitive Subjunctive may be used figuratively (negative \latin{nōn}):

\subsection

In \term{Independent Conditions} (cf.\ the Imperative, \xref[2]{497}).
\begin{examples}

\latin{experiātur: tēctō recipiet nēmō},
\english{let him try: no one will admit him to his house};
\apud{Verr.}{2, 10, 26}.
Similarly \latin{sineret dolor}, \apud{Aen.}{6, 31}.
(Individual Condition, Less Vivid Future.)

\latin{mersēs profundō, pulchrior ēvenit},
\english{sink it in the depths, it comes forth fairer};
\apud{Carm.}{4, 4, 65}.
(Generalizing Condition, in any time.)

\end{examples}

\subsection

In \term{Generalizing Clauses in the Second Person Singular
  Indefinite}, after \latin{sī} or a relative of any kind.
\begin{examples}

\latin{haec quō diē fēcerīs necessāria, eadem, sī cotīdiē fēcisse tē
  reputēs, inānia videntur, multō magis cum sēcesserīs},
\english{these things seem necessary on the day on which you have done
  them, and yet, if you reflect that you have been doing them day
  after day, they appear frivolous, and much more so when you have
  retired into the country};
\apud{Plin.\ Ep.}{1, 9, 3}.
(\latin{Fēcerīs}, \latin{reputēs}, and \latin{sēcesserīs} are all
examples.  “You” is in each case “anybody.”)

\end{examples}

\subsubsection

This Subjunctive originally expressed a \emph{command of the imagination}
\linebreak
(“let”), but it became a mere sign of indefiniteness.

\subsection

In \term{Imaginative Comparisons}, with words meaning “as
if.”\footnote{\latin{Quasi}, \latin{tamquam}, \latin{tamquam sī},
  \latin{velut sī}, and (less frequently) \latin{ac sī} and \latin{ut
    sī}.  Also, in poetic and later Latin, \latin{ceu}, \latin{nōn
    aliter quam sī}, \latin{sīcutī}, \latin{velut}, \latin{perinde
    ac}, etc.}
\begin{examples}

\latin{est obstandum, velut sī ante Rōmāna moenia pugnēmus},
\english{we must make our stand, as if fighting before the walls of
  Rome};
\apud{Liv.}{21, 41, 15}.

\latin{metus cēpit, velut sī iam ad portās hostis esset},
\english{fear seized upon them, as if the enemy were already at their gates};
\apud{Liv.}{21, 16, 2}.

\end{examples}

\subsubsection

The tenses of the present (Present and Perfect) are used \emph{if the
  imagined act is placed in the present or future}, the tenses of the
past (Imperfect and Past Perfect) \emph{if it is placed in the past}.

\begin{minor}

\subsubsection

The construction probably in the beginning expressed a \emph{command
  of the imagination} (“imagine us to be fighting,” etc.),
\emph{without any question about the fact}; and the usage, once
established, remained fairly constant.

\subsubsection

Still it would often be felt that the imagined act was really
\emph{contrary to the actual fact} (see Conditions, \xref{581}); and
accordingly the Imperfect and Past Perfect occur.
\begin{examples}

\latin{proinde habēbō ac sī scrīpsissēs\dots},
\english{I shall regard it as if you had written\dots};
\apud{Att.}{3, 13, 1}.
Similarly \latin{quasi nōn nōssēs}, \apud{Ph.}{388}.

\end{examples}

\end{minor}

\section

The Subjunctive is used with \latin{nēdum} (rarely \latin{nē}),
\english{still less}.\footnote{The construction is probably of Volitive
  origin, but its exact history is not clear.}
\begin{examples}

\latin{vix intellegere potuī: nēdum satis sciam quō modō mē tuear},
\english{I was scarcely able to understand; still less do I know how
  to defend myself};
\apud{Liv.}{40, 15, 14}.
Similarly \latin{nē illī temperārent},
\apud{Sall.\ Cat.}{11, 8}.

\end{examples}

\headingE{The Anticipatory Subjunctive}

\contentsentry{C}{The Anticipatory Subjunctive}

\section

The Anticipatory Subjunctive represents an act as \emph{foreseen},
\emph{expected}, \emph{looked forward to} (English “shall” in all
persons).  The negative is \latin{nōn}.

This use of the Subjunctive had died out in independent sentences
before the beginnings of the literature.

\begin{minor}

\subsubsection

The Present and Perfect express a \emph{present} or \emph{future}
anticipation, the Imperfect and Past Perfect a \emph{past}
anticipation.

The Perfect is thus a Future Perfect for the present or future, the
Past Perfect a Future Perfect for the past.

\end{minor}

\section

The Anticipatory Subjunctive is used in dependent clauses as follows:

\subsection

In \term{Relative Clauses}, determinative or descriptive.
\begin{examples}

\latin{exspectandus erit quī lītēs incohet annus tōtīus populī},
\english{I shall have to wait for the year that shall \emph{(= will)}
  start afresh upon the suits of the whole people};
\apud{Iuv.}{16, 41}
(= the coming year.  Determinative clause).

\latin{nunc est ille diēs quom (= cum\footnote{\latin{Cum}, as a
    relative referring to an antecdent of time, of course has the same
  constructions as any other relative.}) glōria maxuma sēsē nōbīs
  ostendat},
\english{this is the day when the supreme glory is to \emph{(= will)}
  manifest itself to us};
\apud{Enn.\ Ann.}{414, 4}.
(This is that expected day.  Determinative clause.\versionA*{
  Similarly, though in indirect discourse, \latin{diem quō condant},
  \apud{Aen.}{7, 145}.})

\versionA*{\latin{nāscētur pulchrā Troiānus orīgine Caesar, imperium Ōceanō,
  fāman quī ter\-mi\-net astrīs},
\english{there will be born a Trojan of noble origin, Caesar, who
  shall \emph{(prophetic, = will)} make the Ocean the boundary of his
  dominion, the stars the boundary of his fame};
\apud{Aen.}{1, 286}.
(A Trojan of what kind?  A Trojan that shall\ellipsis\  Cf.\ \latin{quae
  verteret}, expressing a \emph{past} Anticipation, \apud{Aen.}{1,
  20}.)}

\versionB{\latin{nāscētur Troiānus, fāmam quī terminet astrīs},
\english{there will be born a Trojan, who shall \emph{(prophetic, =
    will)} make the stars the boundary of his fame};
\apud{Aen.}{1, 286}.  (A Trojan of what kind?  A Trojan that
shall.\dots\ Cf.\ \latin{quae verteret}, expressing a \emph{past}
Anticipation, \apud{Aen.}{1, 20}.)}

\versionA*{\latin{venient annīs saecula sērīs quibus Ōceanus vincula rērum laxet
  et ingēns pateat tellūs},
\english{a time will come in years remote when Ocean shall
  \emph{(prophetic, = will)} relax the bonds that bind the world, and
  the great globe lie open};
\apud{Sen.\ Med.}{\emend{210}{378}{375}}.
(A time of what nature?  A time when\ellipsis shall.)}

\end{examples}

\subsubsection

The Future Indicative has\versionA{ almost completely} driven the
Anticipatory Subjunctive\versionB*{ almost completely} out of the
determinative clause, and tends to drive it out of the descriptive
clause, as in \latin{veniet aetās cum premet}, \apud{Aen.}{1,
  283}. Cf.\ also \latin{quī\ellipsis ferant quōrumque\ellipsis
  vidēbunt}, \apud{Aen.}{7, 98}.

\subsection
In \term{Substantive Clauses} 
\versionA{with \latin{ut}, }%
\versionB*{\term{of Anticipation}:
\subsubsection
With \latin{ut} }%
after verbs of \emph{expecting}.\footnote{\vrb{Exspectō},
  \vrb{opperior} (and, rarely, \vrb{spērō}).}
\begin{examples}

\latin{nēmō exspectet ut aliēnō labōre sit disertus},
\english{let no man expect that he will become eloquent through the
  labor of others};
\apud{Quintil.}{7, 10, 14}.

\latin{mea lēnitās hoc exspectāvit, ut id quod latēbat ērumperet},
\english{my clemency has waited for that which was concealed to break
  out};
\apud{Cat.}{2, 12, 27}.

\end{examples}

\versionB*{\subsubsection

With \latin{quīn} after verbs of \emph{doubt}, if these are negatived.
\begin{examples}

\latin{haec sī ēnūntiāta Ariovistō sint, nōn dubitāre quīn gravissimum
  supplicium sūmat},
(says) \english{he does not doubt that, if this be told to Ariovistus,
  he will inflict the severest punishment};
\apud{B.~G.}{1, 31, \emend{211}{15}{14}}.

\end{examples}}

\subsection

In \term{Indirect Questions}, after verbs of \emph{expecting},
\emph{knowing}, \emph{fearing}, or
\emph{anxiety}.\footnote{\vrb{Exspectō}, \vrb{nesciō}, \vrb{timeō};
  also the phrases \latin{mihi cūrae est}, \latin{sollicitus sum},
  etc.}
\begin{examples}

\latin{quid hostēs cōnsilī caperent exspectābat},
\english{\emph{(Crassus)} was waiting \emph{(to see)} what plan the
  enemy would form};
\apud{B.~G.}{3, \emend{212}{24}{22}, 1}.
(Past Expectancy.)

\latin{nescīs quid vesper sērus vehat},
\english{you know not what the shades of evening shall bring forth};
\apud{Varro, Sat.\ Men.}{333}.
(Present Expectancy.)

\latin{sīn (eritis secūtī) illam alteram nesciō an amplius mihi negōtī
contrahātur},
\english{but if \(you follow\) the other proposal, I am inclined to
  think that more trouble will be brought upon me};
\apud{Cat.}{4, 5, 9}.
(For the translation, see \xref[\emph{f}]{537}.)

\end{examples}

\subsubsection

With \latin{exspectō quam mox}, the construction is frequent, even in
Cicero.

\subsection

In \term{Clauses} with \latin{antequam},\footnote{\latin{Ante} and
  \latin{prius} are often separated from \latin{quam}.  (See examples
  under \emph{c}.)} \latin{priusquam}, \latin{citius quam},
\latin{potius quam}, and the like, to represent an act as:

\begin{enuma}

\item
Anticipated and \emph{prepared for}.
\begin{examples}

\latin{medicō dīligentī, priusquam cōnētur aegrō adhibēre medicīnam,
  nātūra corporis cog\-nō\-scen\-da est},
\english{a careful physician, before attempting to prescribe medicine
  for a patient, must look into his general constitution};
\apud{De~Or.}{2, 44, 186}.

\latin{priusquam ēdūceret in aciem, ōrātiōnem est exōrsus},
\english{before leading out his men into line of battle, \(he\) began
  a harangue};
\apud{Liv.}{21, 39, 1}.

\end{examples}

\begin{note}[Note 1]

The formula became a fixed one, and was then used of the regular
anticipation of one event by another in the \emph{operations of nature},
although there is in this case no true looking forward.
\begin{examples}

\latin{huius folia priusqam dēcidant, sanguineō colōre mūtantur},
\english{its leaves turn red before falling};
\apud{Plin.\ N.~H.}{14, \emend{38}{37}{11}}.

\end{examples}

\end{note}

\begin{note}[Note 2]

For the Indicative of an actual event \emph{looked back upon}, see
\xref[\emph{b}]{550}.

\end{note}

\begin{note}[Note 3]

After Cicero’s time the distinction of mood broke down, and the
Subjunctive was frequently used of an actual event.
\begin{examples}

\latin{ducentīs annīs antequam Rōman caperent, in Italiam Gallī
  trānscendērunt},
\english{two hundred years before they were to take \emph{(took)}
  Rome, the Gauls crossed into Italy};
\apud{Liv.}{5, 33, 5}.

\end{examples}

\end{note}

\item

Anticipated and \emph{forestalled}.
\begin{examples}
\latin{Rōmānus, priusquam forēs portārum obicerentur, velut agmine ūnō
  inrumpit},
\english{the Romans, before the gates could be closed, rushed in as in
  a single mass};
\apud{Liv.}{1, 14, 11}.
\end{examples}

\begin{note}

Since an event forestalled is\versionB*{ generally} one which the main
actor tries to make \emph{impossible}, the Anticipatory Subjunctive of
\latin{possum} (with the Infinitive) is sometimes used\versionA{ in
  this construction} (as in \apud{B.~G.}{6, 3, 2}, \latin{priusquam
  convenīre possent})\versionA{,} in place of the simple verb in the
Subjunctive (\latin{priusquam convenīrent}).

\end{note}

\item

Anticipated and \emph{insisted upon}.
\begin{examples}

\latin{nōn prius ducēs ex conciliō dīmittunt quam sit concessum},
\english{they do not \emph{(= will not)} let the leaders leave the
  council until the concession is made};
\apud{B.~G.}{3, \emend{124}{18}{16}, 7}.
Cf.\ \latin{nec prius absistit quam fundat}, \apud{Aen.}{1, 192}.

\end{examples}

\begin{note}

To give this meaning the main verb must be negatived.

\end{note}

\item

Anticipated and \emph{deprecated}.
\begin{examples}

\latin{animam ommittunt prius quam locō dēmigrent},
\english{they die sooner than \emph{(= rather than)} leave their
  post};
\apud{Amph.}{240}.

\latin{potius quam id nōn fīat, ego dabō},
\english{rather than not have it come off, I’ll give the money myself};
\apud{Pseud.}{554}.
\versionB*{Cf.\ \latin{prius quam ut}, \apud{Lig.}{12, 34}.}

\end{examples}

\begin{note}[Note to {\upshape \emph{a}})-{\upshape \emph{d}})]

The Future Perfect Indicative is also used in these constructions (as
in \latin{antequam cognōverō}, \apud{Sen.}{6, 18}), the Future
Indicative only very rarely in Ciceronian prose (thus \latin{citius
  quam extorquēbit}, \apud{Lig.}{5, 16}; in poetry more commonly,
e.g. \latin{ante quam dabitur}, \apud{Aen.}{9, 115}).  For the
frequent \emph{Present} Indicative in the same general sense,
see~\xref{571}.

\end{note}

\end{enuma}

\subsection

In \term{Clauses} with \latin{dum}, \latin{dōnec}, or \latin{quoad},
\english{until}, to represent an act as \emph{looked forward to}.
\begin{examples}

\latin{mānsūrus patruom pater est dum adveniat},
\english{your father is going to wait till your uncle shall arrive};
\apud{Ph.}{480}.
(Present Expectation.)

\latin{dum reliquae nāvēs eō convenīrent exspectāvit},
\english{he waited till the other ships should arrive};
\apud{B.~G.}{4, 23, 4}.
(Past Expectation.)

\end{examples}

\begin{minor}

\subsubsection

The Future Perfect Indicative is also used in this sense, the Future
Indicative not in Ciceronian prose (poetical example \latin{dōnec
  dabit}, \apud{Aen.}{1, 273}).  For the frequent \emph{Present}
Indicative in the same general sense, see~\xref{571}.

\subsubsection

For the Indicative of an actual event \emph{looked back upon},
see~\xref[\emph{b}]{550}.

\subsubsection

After Cicero’s time the distinction broke down, and the Subjunctive was
frequently used of an actual event.  Cf.\ \xref[4, n.~3]{507}.
\begin{examples}

\latin{hoc plūribus (diēbus), dōnec hominēs subīret timendī pudor},
\english{this \emph{(took place)} on a number of \(days\), until men
  began to be ashamed of being afraid};
\apud{Plin.\ Ep.}{9,~33,~6}.

\end{examples}

\subsubsection

\latin{Dum}, \latin{dōnec}, and \latin{quoad}, meaning \english{so
  long as}, take the Indicative (\xref[\emph{b}]{550}).

\end{minor}

\section

In general,\footnote{\label{ftn:s508:}The only exceptions are
  assertions and conditions expressing an \emph{actual past intention}
  (periphrastic forms, as in \emph{they were \textsc{going}
    to\dots}; \english{if they were \textsc{going} to}).} \emph{all}
past-future ideas must, if expressed by a Finite Verb, be in the
Anticipatory Subjunctive; for \emph{no other means of expression
  exists}.

\subsubsection

There are thus three possible ways of expressing Futurity to the
Present, and only one way of expressing Futurity to the Past:
\begin{Tabular}{@{}c@{}c@{}}

\textsc{Point of View Past}
& \textsc{Point of View Present}
\\[\smallskipamount]

\begin{tabular}{@{}c@{}}
Anticipatory Subjunctive,\\Imperfect or Past\\ (Future) Perfect
\end{tabular}
&
\groupL{1. Indicative Future or Future Perfect \\
        2. Present Indicative with future force (see \xref{571})\\
        3. Anticipatory Subjunctive, Present or (Future) Perfect}

\end{Tabular}

\section

Accordingly, the Anticipatory Subjunctive of the past is extremely
common in constructions\footnote{With any relative pronoun, or
  relative or conditional conjunction.} in which it would not be used,
or \emph{need not} be used, if the point of view were present or
future. Thus:

\emph{Past-Future Determinative Clauses}:
\begin{examples}

\latin{aderat iam annus quō prōcōnsulātum Āfricae sortīrētur},
\english{the year was now at hand, in which he should draw the
  proconsulate of Africa as his lot};
\apud{Tac.\ Agric.}{42}.

\latin{omnīnō bīduum supererat, cum exercituī frūmentum mētīrī oportēret},
\english{two days in all were left \emph{(before the time)} when
  rations would have to be issued to the army};
\apud{B.~G.}{1, 23, 1}.

\end{examples}

\emph{Past-Future Conditions}:
\begin{examples}

\latin{nostrī, sī ab illīs initium trānseundī fieret, parātī erant},
\english{our men were ready, if they should begin to cross};
\apud{B.~G.}{2, 9, 1}.

\latin{erat ūnum iter, Ilerdam sī revertī vellent, alterum, sī
  Tarracōnem peterent},
\english{there was one way if they should choose to return to Lerida,
  another if they should make for Tarragona};
\apud{B.~C.}{1, 73, 2}.

\end{examples}

\subsubsection

It often \emph{happens} that such past anticipations are indirect
expressions of some one’s speech or thought,—i.e.\ are in Indirect
Discourse (\xref[2]{534}).
\begin{examples}

\latin{ubi intellēxit diem īnstāre quō diē frūmentum mīlitibus mētīrī
  oportēret},
\english{when he saw that the day was at hand on which rations would
  have to be given out to the soldiers};
\apud{B.~G.}{1, 16, 5}.
(\latin{Diem quō oportēret} is to the past what \latin{diēs quō
  oportēbit} would be to the present.)

\latin{Xerxēs praemium prōposuit quī invēnisset novam voluptātem},
\english{Xerxes offered a reward to the man who should invent a new
  pleasure};
\apud{Tusc.}{5, 7, 20}.

\end{examples}

\headingE{The Optative Subjunctive}

\contentsentry{C}{The Optative Subjunctive}

\section

The Optative Subjunctive represents an act as \emph{wished} or
\emph{desired} (English “may,” “would that,” etc.).

\subsubsection

The Present and Perfect deal with the future, and so express a wish
that \emph{may be realized}.  The Imperfect and Past Perfect deal with
the present and past, and so express a wish \emph{contrary to fact}.

The Imperfect generally refers to the present, and the Past Perfect to
the past.  But occasionally the Imperfect (especially in poetry)
expresses a past act, and the Past Perfect an act completed in the
present.

\begin{note}[Remark]

The Imperfect and Past Perfect originally expressed a wish in time
\emph{future to a past time}.  This is still generally the case in
dependent clauses.  Thus \latin{optābam ut adesset}, \english{I wished
  that he might be present}.

\end{note}

\subsubsection

The Perfect may express a hope that something \emph{has been done}.

\section

The Optative Subjunctive is used:

\subsection

In \term{Wishes}.  These may be introduced by \latin{utinam}, and
generally \emph{are} so introduced, if in the Imperfect or Past
Perfect.  The negative is regularly \latin{nē}, but with
\latin{utinam} sometimes
\latin{nōn}.\footnote{\label{ftn:s511:1}Wishes with \latin{utinam},
  \latin{ut}, and \latin{quī} were originally \emph{Potential
    Questions} (“how might\dots?”).  Hence the original negative was
  \latin{nōn}.}
\begin{examples}

\latin{sint beātī},
\english{may they be happy!}
\apud{Mil.}{34, 93}.

\latin{perierīs},
\english{may you perish utterly!}
\apud{Men.}{295}.
(Emphatic Perfect.)

\latin{utinam spem implēverim},
\english{I hope I may have fulfilled his expectation};
\apud{Plin.\ Ep.}{1, 10, 3}.
(Present Perfect.)

\latin{utinam ille omnīs sēcum suās cōpiās ēdūxisset!}
\english{would that he had led out all his forces with him!}
\apud{Cat.}{2, 2, 4}.

\latin{obruerent Rutulī tēlīs!}
\english{would that the Rutuli had laid \emph{(me)} low with their
  darts!}
\apud{Aen.}{11, 162}.

\latin{utinam fīliī nē dēgenerāssent!}
\english{would that the sons had not degenerated!}
\apud{Prov.\ Cons.}{8, 18}.

\latin{utinam susceptus nōn essem!}
\english{would that I had not been allowed to live at birth!}
\apud{Att.}{11, 9, 3}.

\end{examples}

\begin{minor}

\subsubsection

In poetry, especially in early Latin, \latin{ut} and
\latin{quī}\footnotemark[\thefootnote] may also be used, the latter in
\emph{Imprecations} (Curses) only.
\begin{examples}

\latin{quī illum dī omnēs perduint!}
\english{may all the gods confound him!}
\apud{Ph.}{123}.

\end{examples}

\subsubsection

A Wish may be used to express an independent condition.
\begin{examples}

\latin{mē quoque, quā frātrem, mactāssēs, improbe, clāvā!  Esset, quam
  dederās, morte solūta fidēs},
\english{would that you had killed me, wretch, with the same club with
  which you killed my brother!  The promise you had given would then
  have been annulled by death};
\apud{Ov.\ Her.}{10, 77}.

\end{examples}

\end{minor}

\subsection

In \term{Substantive Clauses}, after verbs of \emph{wishing},
\emph{desiring}, etc.\footnote{The commonest of these are \vrb{optō},
  and, in poetry and later prose, \latin{cupiō}, \latin{vōtum est}.}
The connective, if one is used, is \latin{ut} or \latin{nē}.
\begin{examples}

\latin{optēmus ut eat in exilium},
\english{let us hope that he is going into exile};
\apud{Cat.}{2, 7, 16}.
(Present Wish.)

\latin{fuit optandum Caecīnae ut contrōversiae nihil habēret},
\english{it was desirable for Caecina to have no controversy};
\apud{Caecin.}{9, 23}.
(Past Wish.)

\end{examples}

\headingE{The Subjunctive of Obligation or Propriety}

\contentsentry{C}{The Subjunctive of Obligation or Propriety}

\section

The Subjunctive of Obligation or Propriety represents an act as
\emph{obligatory}, \emph{proper}, or \emph{reasonable} (English
“ought”, “should”).

\subsubsection

The original negative, \latin{nē}, is sometimes still found in
\emph{statements} (\xref[1]{513}), not elsewhere.  But, even here,
\latin{nōn} became more common, since this is the negative that
regularly \emph{belongs} to statements (\xref[1]{464}, and footnote).

\subsubsection

The Present expresses a \emph{present} obligation or propriety, the
Imperfect and Past Perfect a \emph{past} obligation or propriety,
unfulfilled.

\section

The Subjunctive of Obligation or Propriety is used:

\subsection

In \term{Statements} and \term{Questions}.

The interrogative words, if used, are \latin{quid}, \latin{quidnī},
\latin{quāre}, \latin{quamobrem}, or
\latin{cūr}.\footnote{\latin{Quīn}, as in \latin{quīn rogem?}\
  \english{why shouldn’t I ask?}\ \apud{Mil.\ Gl.}{426}, is rarely used
  in questions of obligation or propriety.  In dependent clauses, it
  is frequent.}
\begin{examples}

\latin{quid facere dēbuistī? frūmentum nē ēmissēs},
\english{what ought you to have done? You ought not to have bought the grain};
\apud{Verr.}{3, 84, 195}.

\latin{“nōn ego illī argentum redderem?” “Nōn redderēs,”}
\english{“oughtn’t I to have paid in the money to him?” “You ought not”};
\apud{Trin.}{133}.

\latin{ā lēgibus nōn recēdāmus},
\english{we should not swerve from the laws};
\apud{Clu.}{57, 155}.

\latin{nōn eō sīs cōnsiliō},
\english{you should not adopt this opinion};
\apud{Fam.}{9, 16, 7}.

\latin{hunc ego nōn admīrer?}
\english{ought I not to admire a man like this?}
\apud{Arch.}{8, 18}.

\latin{quid ego tē invītem},
\english{why should I urge you?}
\apud{Cat.}{1, 9, 24}.
(Direct Question of Present Obligation.)

\latin{nōn videō cūr nōn audeam},
\english{I don’t see why I should not venture};
\apud{Sen.}{21, 77}.
(Indirect Question of Present Obligation.)

\latin{cūr dēspērārent?}
\english{why \emph{(he asked)} should they despair?}
\apud{B.~G.}{1, 40, 4}.
(Indirect Question of Past Obligation.)

\end{examples}

\begin{minor}

\subsubsection

In Statements, this construction seems to be less frequent in tenses of
the present than in tenses of the past.

\end{minor}

\subsection

In \term{Dependent Clauses}, with \latin{quod}, \latin{quārē},
\latin{quamobrem}, \latin{cūr}, or \latin{quīn} (the last only after a
negative idea, expressed or implied).
\begin{examples}

\latin{nihil est quod pōcula laudēs},
\english{there is no reason why you should praise the cups}
(nothing with reference to which you ought\dots);
\apud{Ecl.}{3, 48}.

\latin{satis esse causae arbitrābātur quārē in eum animadverteret},
\english{he thought there was reason enough why he should punish him};
\apud{B.~G.}{1, 19, 1}.

\latin{quid est quamobrem putēs\dots?}
\english{what reason is there why you should think\dots?}
\apud{Verr.}{2, 20, 49}.

\end{examples}

\subsection

In \term{Relative Clauses} (rarely in clauses with \latin{ut}) after
\latin{dignus}, \latin{indignus}, \latin{aptus}, or \latin{idōneus}.
\begin{examples}

\latin{erit dignior locus ūllus quī hanc virtūtem excipiat?}
\english{will there be any place more worthy to harbor such virtue?}
(any place worthier that it should harbor\dots?);
\apud{Mil.}{37, 101}.
Similarly \latin{idōneus quī}, \apud{Pomp.}{19, 57}.

\latin{nōn sum dignus ut fīgam pālum in parietem},
\english{I am not fit to drive a spike into a wall}
(not fit that I should drive);
\apud{Mil.\ Gl.}{1140}.

\end{examples}

\begin{minor}

\subsubsection

\latin{Quārē}, \latin{quamobrem}, and \latin{cūr} are also
occasionally used with \latin{dignus}, etc.
\begin{examples}

\latin{nihil enim dignum faciēbat, quārē eius fugae comitem mē
  adiungerem},
\english{for he was doing nothing worthy to make me add myself as an
  associate in his flight}
(no worthy thing, on account of which I should\dots);
\apud{Att.}{9, 10, 2}.

\end{examples}

\end{minor}

\subsection

In \term{Clauses} with \latin{ut} \emend{39}{\textbf{or}}{or}
\latin{ut nōn} after \latin{tantī}, \english{worth so much}, and
similar expressions.
\begin{examples}

\latin{est ergō ūlla rēs tantī aut commodum ūllum tam expetendum, ut
  virī bonī et splendōrem et nōmen āmittās?}
\english{is anything then worth so much, or is any advantage so
  desirable, that one should \emph{(= ought to)} give up the proud
  distinction of the name of “good man”?}
\apud{Off.}{3, 20, 82}.

\latin{nūlla studia tantī ut amīcitiae officium dēserātur},
\english{no studies are so important that friendship’s due ought to be
  withheld};
\apud{Plin.\ Ep.}{8, 9, 2}.

\end{examples}

\subsection

In \term{Substantive Clauses}, without connective, or with \latin{nē}
(rare) or \latin{quīn} (the latter after a negative idea
only).\footnote{So with \vrb{oportet}, \latin{aequum}, \latin{iūstum}
  or \latin{iūs est}, \vrb{mereor}, \vrb{decet}, \vrb{dēdecet}.}
\begin{examples}

\latin{multa oportet discat},
\english{he ought to learn many things};
\apud{Quinct.}{17, 56}.

\latin{nūllō modō aequom vidētor quīn quod peccārim potissimum mihi id
  obsit},
\english{it doesn’t seem at all just that my wrongdoing should not
  damage me rather than any one else};
\apud{Trin.}{\emend{213}{588}{587–588}}.

\latin{quārē meditēre cēnseō},
\english{wherefore I think that you should consider};
\apud{Phil.}{2, 37, 95}.
Similarly (in irony) \latin{vereāminī cēnseō}, \apud{Cat.}{4, 6, 13}.

\end{examples}

\headingE{The Subjunctive of Natural Likelihood}

\contentsentry{C}{The Subjunctive of Natural Likelihood}

\section

The Subjunctive of Natural Likelihood represents an act as
\emph{likely to take place} (English “should,” “might well,”
“naturally would,” etc.). The negative is \latin{nōn}.

\subsubsection

The Present and Perfect express a natural likelihood in the
\emph{present or future}; the Imperfect and Past Perfect, a natural
likelihood in the \emph{past}.

\section

The Subjunctive of Natural Likelihood is used:

\subsection

In \term{Questions}, with \latin{quid}, \latin{quidnī}, \latin{quī}
(\english{how?}), \latin{quārē}, \latin{quamobrem}, or \latin{cūr}.
\begin{examples}

\latin{quid enim ōdisset Clōdium Milō, segetem ac māteriem suae
  glōriae?}
\english{why should Milo have hated Clodius, who furnished him the
  field and the occasion of his glory?}
\apud{Mil.}{13, 35}.

\latin{quārē dēsinat esse macer?}
\english{why \emph{(under such circumstances)} should he cease to be
  lean?}
\apud{Catull.}{89, 4}.
(= naturally he would remain lean.)

\latin{“inepta, nescīs quid sit āctum?” “Quī sciam?”}
\english{“you stupid, don’t you know what has taken place?” “How
  should I know?”}
\apud{And.}{\emend{214}{791}{792}}.

\end{examples}

\subsection

In \term{Dependent Clauses}, with \latin{quī}, \latin{quārē},
\latin{quamobrem}, \latin{cūr}, \latin{quīn}, or \latin{ut}.
\begin{examples}

\latin{videō causās esse permultās quae istum impellerent},
\english{I recognise the existence of a great many causes that would
  naturally be impelling him};
\apud{Rosc.\ Am.}{33, 92}.
(Natural working in the past.)

\latin{quantumvīs quārē sit macer inveniēs},
\english{you’ll find every reason in the world why he should be lean};
\apud{Catull.}{89, 6}.
Cf.\ \latin{quārē dēsinat}, \xref[1]{515}.

\latin{ille erat ut ōdisset accūsātōrem suum},
\english{there was \emph{(reason)} that he should \emph{(naturally)}
  hate his accuser};
\apud{Mil.}{13, 35}.

\end{examples}

\subsection

In \term{Substantive Clauses} with \latin{ut}.
\begin{examples}

\latin{vērī simile nōn est, ut ille homō religiōnī suae pecūniam
  antepōneret},
\english{it is not likely that such a man would set money above his
  conscience};
\apud{Verr.}{4, 6, 11}.

\end{examples}

\headingE{The Potential Subjunctive}

\contentsentry{C}{The Potential Subjunctive}

\section

The Potential Subjunctive expresses \emph{Possibility} or
\emph{Capacity} (English
\linebreak
“may,” “might,” “can,” “could”).  The
negative is \latin{nōn}.

\subsubsection

The Present and Perfect express a \emph{present} or \emph{future}
possibility or capacity, the Imperfect and Past Perfect a \emph{past}
possibility or capacity.

\section

The Potential Subjunctive is used especially:

\subsection

In \term{Independent Sentences}, but only where a negative is implied,
or in the Second Person Singular Indefinite, or with \latin{quis},
\latin{aliquis}, \latin{vix}, \latin{facile}, or \latin{forsitan}.
\begin{examples}

\latin{quis clādem illīus noctis fandō explicet}?
\english{who could set forth in words the ruin of that night}?
\apud{Aen.}{2, 361}.
(Present Capacity; = no one could.)

\latin{cuneō hoc agmen disiciās},
\english{with a wedge, one could split this line};
\apud{Liv.}{22, 50, 9}.
(Present Capacity.)

\latin{ea perītīs amnis eius vix fidem fēcerint},
\english{this could scarcely gain any credence at all among those who
  know this river};
\apud{Liv.}{21, 47, 5}.
(Present Capacity, emphatic tense.)

\latin{Servius, frāter tuus, facile dīceret, hic versus Plautī nōn
  est, hic est},
\english{your brother Servius could easily say \emend{40}{‘}{“}That verse isn’t
  Plautus’s, this one is\emend{41}{’}{”}};
\apud{Fam.}{9, 16, 4}.
(Past Capacity.)

\latin{aliquis dīcat mihi},
\english{some one may say to me};
\apud{Sat.}{1, 3, 19}.
(Possibility.)

\end{examples}

\begin{minor}

\subsubsection

But the Future Indicative is much more common with \latin{quis} and
\latin{aliquis}, as in \latin{dīcet aliquis}, \english{some one will
  say}, \apud{Pis.}{28, 68}.

\end{minor}

\subsection

In \term{Relative Clauses}, after expressions of \emph{existence} or
\emph{non-existence}.\footnote{Thus after \vrb{est}, \vrb{habeō},
  \latin{nōn est}, \latin{nōn habeō}, etc.

  These clauses are really \emph{descriptive}, expressing that of
  which the antecedent is \emph{capable}, or for which it is
  \emph{available} or \emph{suitable}.}
\begin{examples}

\latin{est unde haec fīant},
\english{I have means with which it can be done};
\apud{Ad.}{\emend{42}{122}{123}}.

\latin{nihil erat quō famem tolerārent},
\english{there was no means by which they could relieve their
  starvation};
\apud{B.~G.}{1, 28, 3},

\latin{ūnum angustum et difficile, vix quā singulī carrī dūcerentur},
\english{one \emph{(way was)} narrow and difficult, by which carts
  could hardly be hauled one at a time};
\apud{B.~G.}{1, 6, 1}.

\end{examples}

\begin{minor}

\subsubsection

The potential feeling of the clause is clearly shown by its
parallelism with clauses with \latin{possum} in the Subjunctive of
Actuality (\xref[1]{521}) with a dependent Infinitive.  Thus
\latin{unde agger comportārī posset} (instead of
\latin{comportārētur}), \latin{nihil erat reliquum}, \english{there
  was nothing left from which a rampart cauld be got together};
\apud{B.~C.}{2, 15, 1}; cf.\ \apud{B.~G.}{2, \emend{105}{25}{24}, 1};
\apud{}{4, 29, 4}.

\end{minor}

\subsection

In \term{Substantive Clauses} after \latin{fierī potest}.
\begin{examples}

\latin{fierī potest ut rēctē quis sentiat, et id quod sentit polītē
  ēloquī nōn possit},
\english{it may happen that a man may think correctly, and yet be
  unable to express his thoughts in a finished manner};
\apud{Tusc.}{1, 3, 6}.

\end{examples}

\begin{minor}

\subsubsection

This is the only way in Latin of saying “may” or “can,” except
with \latin{possum} used personally, or as shown under~\xref[1]{517}.

\end{minor}

\headingE{The Subjunctive of Ideal Certainty\protect\footnotemark}

\footnotetext{Possibility, Natural Likelihood, and Ideal Certainty (act
possible, probable, or ideally certain) often lie close together, so
that a given example may seem to belong to any or all of the three
forces.  Thus \latin{nēmō crēdat} might mean either \emph{no one could
  believe}, \emph{no one would be likely to believe}, or \emph{no one
  would believe}.

  At the \emph{extremes} of their forces, on the other hand,
  Possibility and Ideal Certainty are widely separated.  Thus in
  \latin{nōn ille nōbīs} under \xref[1]{519}, the meaning is not
  \emph{possibly he would not have appointed\dots}, but \emph{he
    certainly would not have appointed\dots, \textsc{not he}}.}

\contentsentry{C}{The Subjunctive of Ideal Certainty}

\section

The Subjunctive of Ideal Certainty declares that, under imagined or
imaginable circumstances, something \emph{would take place} (or
\emph{would have taken\linebreak
 place}), or asks a corresponding question
(English “I should,” “you would,” “he would,” etc.).  The
negative is \latin{nōn}.

\subsubsection

The Present and Perfect express an Ideal Certainty in time
\emph{future to the present}, the Imperfect and Past Perfect an Ideal
Certainty in time \emph{future to a past time}.  The Perfect is
accordingly a Future Perfect for the present, the Past Perfect a
Future Perfect for the past.  Thus, \latin{ille id faciat},
\english{he would do this} (e.g.\ if he should be called upon);
\latin{ille id fēcerit}, \english{he would assuredly do this}
(emphatic Perfect).

\subsubsection

\textbf{New Force developed by the Imperfect and Past Perfect
  Subjunctive.}  In addition, the Imperfect and Past Perfect
Subjunctive gained the power of expressing an ideal certainty
\emph{contrary to fact}, the \emph{Imperfect} referring generally to
\emph{present} time, and the \emph{Past Perfect} to either \emph{past}
or \emph{present} time.  Thus \latin{ille id faceret}, \english{he would be
  doing this} (e.g.\ if he had been called upon); \latin{ille id
  fēcisset}, \english{he would have done this}, now or in the past
(e.g.\ if he had been called upon).  For the origin of this force of
the tenses, see \xref[\emph{a}, rem.]{581}

\section

The Subjunctive of Ideal Certainty is used:

\subsection

In \term{Statements} and \term{Questions}.
\begin{examples}

\latin{ecquis id dīxerat? Certē nēmō},
\english{would anybody dream of saying this? Surely nobody would};
\apud{Tusc.}{1, 36, 87}.
(Emphatic Perfect.  Act future.)

\latin{īre per ignīs et gladiōs ausim},
\english{I should have courage to go through fire and sword};
\apud{Ov.\ Met.}{8, 76}.

\latin{nōn ille nōbīs Sāturnālia cōnstituisset},
\english{he would not have appointed the Saturnalia for us, not he};
\apud{Cat.}{3, 7, 17}.
(Contrary to fact; for they \emph{have been} appointed.)

\end{examples}

\subsubsection

A frequent use is in Subjunctive Conclusions. See \xref{574},
\xref{580}, \xref{581}.

\subsubsection

This Subjunctive is often used merely to \emph{soften a statement}.
\begin{examples}

\latin{ego quae in rem tuam sint ea velim faciās},
\english{I should like you to do that which is for your interest};
\apud{Ph.}{449}.
(Compare the unsoftened \latin{Syrō ignōscās volō},
\english{I \textsc{want} you to forgive Syrus};
\apud{Heaut.}{\emend{215}{1066}{1067}}.)

\latin{velītis iubeātisne haec sīc fierī?}
\english{would it be your wish and command that this course be taken?}
\apud{Liv.}{22, 10, 2}.
(Softened Question.)

\end{examples}

\begin{minor}

\subsubsection

\term{Virtual Wishes.}  The Softened Statements \latin{velim},
\latin{vellem}, \latin{mālim}, \latin{māllem}, with \versionA{a
  Substantive}\versionB*{an Infinitive or} Volitive Clause attached,
are equivalent to a Subjunctive of Wish.
\begin{examples}

\latin{virum mē nātam vellem},
\english{I should like to have been born a man}
(= would that I had been born a man);
\apud{Ph.}{792}.
Similarly \latin{māllem ēdūxisset}, \apud{Cat.}{2, 3, 5}.

\end{examples}

\end{minor}

\subsection

In \term{Relative Clauses}, determinative or descriptive.
\begin{examples}

\latin{ūnō verbō dīc, quid est quod mē velīs},
\english{tell me in a word what the thing is which you would like of
  me};
\apud{And.}{45}.
(Determinative.)

\latin{fēcērunt id servī Milōnis quod suōs quisque servōs in tālī rē
  facere voluisset},
\english{Milo’s slaves did just that which, in similar circumstances,
  any one would have wished his slaves to do};
\apud{Mil.}{10, 29}.
(Determinative.)

\latin{nīl est aequē quod faciam lubēns},
\english{there is nothing that I should do with so much pleasure};
\apud{Ph.}{565}.
(Descriptive.)

\latin{profectus id temporis, cum iam Clōdius, sī quidem eō diē Rōmam
  ventūrus erat, redīre potuisset},
\english{he set out at an hour when Clodius, if he really meant on
  that day to come to Rome, might already have been \emph{(would have
    been able to be)} on his way back};
\apud{Mil.}{10, 28}.
(Descriptive.)

\end{examples}

\subsection

In \term{Clauses of Ideally Certain Result}, with \latin{ut} or
\latin{ut nōn}.
\begin{examples}

\latin{adeō variant auctōrēs ut vix quicquam adfirmāre ausus sim},
\english{authorities differ so much that I should hardly dare to make
  any statement at all};
\apud{Liv.}{22, 36, 1}.
(Present Ideal Certainty about the future.)

\latin{rēs tamen ab Āfrāniānīs hūc erat dēducta, ut, sī priōrēs montīs
  attigissent, ipsī perīculum vītārent, impedīmenta servāre nōn
  possent},
\english{things, however, had been brought to such a pass by Afranius
  and his men, that, if they should be the first to reach the hills,
  they themselves would escape from danger, but would be unable to
  save their baggage};
\apud{B.~C.}{1, 70, 2}.
(Past-future Ideal Certainty.)

\end{examples}

\subsubsection

The Descriptive Clause and the Result Clause both express something
that \emph{would naturally follow from the character of the
  antecedent}.  Hence they may be called \emph{Consecutive Clauses}.

\begin{minor}

\subsubsection

In modern English we have to use the Conjunction “that” (after
“such,” “so,” etc.)\ to express the full consecutive idea.  In
Shakespeare’s time, the bare Relative “who” or “that” could do
this.  A comparison will make the feeling of the Latin plainer.
\begin{examples}

“Who is here so base that would be a bondman?”
\apud{Shakespeare, Jul.\ Caes.}{3, 2}.

\latin{quis est tam impius quī nōn fateātur?}
\english{who is so impious that \(he\) would not admit\dots?}
\apud{Har.\ Resp.}{10, 20}.

\end{examples}

\end{minor}

\subsection

In \term{Substantive Clauses of Ideal Certainty}:
\begin{enuma}

\item

With \latin{ut} or \latin{ut nōn}, after verbs of \emph{bringing
  about} or of \emph{existence}.
\begin{examples}

\latin{unde fit ut mālim frāterculus esse gigantis},
\english{whence it results that I should prefer to be the little
  brother of a son of the soil};
\apud{Iuv.}{4, 98}.

\end{examples}

\item

With \latin{quīn} after verbs or phrases of \emph{doubt} or
\emph{ignorance}, if these are negatived, or imply a
negative.\footnote{So especially after \latin{nōn dubitō}, \latin{nōn
    dubium est}, \latin{quis dubitat} (implies “no one doubts”),
  \latin{num dubium est}, \latin{nōn ignōrō}, \latin{quis ignōrat}.

  After an affirmative the Infinitive is used (example in \xref{589}),
and the later writers often use it even after a negative.}
\begin{examples}

\latin{quod ille sī repudiāssset, dubitātis quīn eī vīs esset adlāta?}
\english{if he had refused, do you doubt that violence would have been
  offered him?}
\apud{Sest.}{29, 62}.
(Here \latin{dubitātis} = \latin{dubitāre nōn potestis}.)

\end{examples}

\end{enuma}

\chapter{Subjunctive Constructions of Composite Origin (Fusion)}

\headingB{The Subjunctive of Actuality (Fact)}

\contentsentry{C}{The Subjunctive of Actuality in Consecutive Clauses}

\section

The Subjunctive of Actuality represents an act or state as a fact. The
negative is \latin{nōn}.

\subsubsection

In Subjunctive Clauses of Actuality, the Present expresses either a
present or future \emph{situation} or a present or future act seen
\emph{aoristically}, the Imperfect either a past \emph{situation} or a
past act seen \emph{aoristically}, \emph{but in temporal relation to
  the main act}.  The Past Aorist (Perfect), on the other hand,
expresses a past act, etc., seen \emph{absolutely}.

The Present Perfect and Past Perfect express an act as \emph{in a
  completed state} at a present or past time respectively.

\section

The Subjunctive of Actuality is used only in dependent clauses
\emph{of consecutive nature} (\xref[1, \emph{e}]{521}), as follows:

\subsection

In \term{Descriptive Clauses of Actuality (Fact)}.

Any relative may be used (e.g.\ \latin{quī}, \latin{cum}, \latin{ubī},
\latin{unde}).  \latin{Quīn},\footnote{\label{ftn:s521:}This
  \latin{quīn} is of the same origin as the conjunction \latin{quīn},
  \english{that not} (footnote~\ref{ftn:262:1},
  p.~\pageref{ftn:262:1}), but is used in place of the declined
  relatives \latin{quī nōn}, \latin{quae nōn}, or \latin{quod nōn}.
  It may be employed in any construction in which \latin{quī nōn} is
  possible, e.g.\ in \xref[2]{519}.}  \english{who\ellipsis not},
\english{that\ellipsis not}, may be employed in place of \latin{quī nōn},
etc., but only after a negative idea, expressed or implied.
\begin{examples}

\latin{sī quis est tālis quī mē accūset},
\english{if there is any one of such a disposition that he blames me};
\apud{Cat.}{2, 2, 3}.
(Present state of affairs.)

\latin{at sunt quī dīcant},
\english{but there are men that say};
\apud{Cat.}{2, 6, 12}.

\latin{num quis est tam dēmēns quī arbitrētur?}
\english{is there any one so mad \emph{(who thinks)} as to think?}
\apud{Mil.}{28, 78}.

\latin{is sum, quī istōs plausūs semper contempserim},
\english{I am one that has always despised such applause}
(I am such that I have\dots);
\apud{Phil.}{1, 15, 37}.
(Present Perfect.)

\latin{nēmō fuit quīn vīderit},
\english{there was no one that did not see};
\apud{Verr.}{5, 54, 140}.
(Past Aorist, expressing the time absolutely.)

\latin{fuit tempus cum Germānōs Gallī virtūte superārent},
\english{there was a time when the Gauls surpassed the Germans in
  courage};
\apud{B.~G.}{6, 24, 1}.

\latin{in ea tempora nātus es, quibus firmāre animum expediat
  cōnstantibus exemplīs},
\english{your life has fallen upon times in which it is well to
  fortify the mind through examples of firmness}
(times such that in them\dots);
\apud{Tac.\ Ann.}{16, 35}.
(Present state of affairs.)

\latin{in id saeculum Rōmulī cecidit aetās, cum iam minor fābulīs
  habērētur fidēs},
\english{the life of Romulus fell upon an age when less credence was
  given to fables};
\apud{Rep.}{2, 10, 18}.
(Past state of affairs.)

\latin{erit illud profectō tempus cum tū amīcissimī benevolentiam
  dēsīderēs},
\english{there will surely come a time when you will miss the kindness
  of a devoted friend};
\apud{Mil.}{26, 69}.
(Future state of affairs.)

\end{examples}

\subsubsection

\emph{These clauses follow incomplete descriptive
  words,\footnote{\latin{Tālis}, \english{such}, \latin{tantus},
  \english{so great}, \latin{hic}, \latin{ille}, \latin{is}, or
  \latin{iste}, \english{such}, \latin{ūnus} or \latin{sōlus},
  \english{the only one}, or \latin{tam}, \latin{adeō}, or
  \latin{ita}, \english{so}, with an adjective.}  or negative or
indefinite expressions, or questions implying a negative}.

\begin{note}

Because of the kind of words or phrases after which the subjunctive
descriptive clause is used, it is \emph{essential}, i.e.\ it cannot be
left out without making the sentence incomplete.  Cf.\ the \emph{free}
descriptive clause, \xref{569}.

\end{note}

\subsubsection

The Subjunctive in such descriptive clauses is \emph{always} necessary
after a negative, and after words meaning \english{such} or
\english{so}.\footnotemark[\thefootnote]

After indefinite positive antecedents,\footnote{E.g.\ \latin{sunt
    quī}, \latin{multī sunt quī}, \latin{quīdam sunt quī}.} the
Indicative (which was the original mood) never was wholly driven out,
though the Subjunctive became more common.  Thus \latin{sunt multī quī
  Graecās nōn ament litterās}, \apud{Ac.}{2, 2, 5}, but \latin{sunt
  multī quī ēripiunt\dots}, \apud{Off.}{1, 14, 43}.

\subsubsection

These clauses all tell \emph{what kind of} a person or thing is meant;
i.e., they are really \emph{complex adjectives}.  For the contrasting
Determinative Clauses (Indicative), which tell \emph{what} person or
thing is meant, see~\xref{550}.

\begin{note}

Notice (in the last four examples) that a \emph{time} may be
described, as well as anything else, and that the mechanism is the
same, except that the temporal relative \latin{cum} may be used, as
well as a form of \latin{quī}, for such an antecedent.  Thus one may
say \latin{in id saeculum quō}, or \latin{in id saeculum cum}.

\end{note}

\subsubsection

For \latin{maior quam quī}, etc., with the Subjunctive, see
2,~\emph{c}, below.

\subsubsection

The Descriptive Clause of Actuality, the Clause of Actual Result
(see~2, below), and the Substantive Clause of Actuality (see~3, below)
all express something that \emph{follows from the character of the
  antecedent}.  Hence these clauses and the clauses derived from them
may be called \term{Consecutive Clauses}.  But in the Descriptive
Clause of Actuality the original consecutive feeling is often faint,
or even non-existent. (So in \latin{sunt quī dīcant}, above.)

\subsubsection

\latin{Quod sciam}, etc.  The Subjunctive is used in phrases meaning
\emph{so far as I know}, \emph{so far as I have heard},
etc. (\latin{quod} or \latin{quantum sciam}, \latin{quod exstet},
\latin{quod quidem sēnserim}, \latin{quod audierim}, etc.), since
these phrases generally follow negative or indefinite words.
\begin{examples}

\latin{numquam dictum ab illō, quod sciam},
\english{never, so far as I know, has it been said by him};
\apud{Fin.}{2, 26, 82}.

\end{examples}

\subsection

In \term{Clauses of Actual Result (Fact)}, with \latin{ut}, \latin{ut
  nōn}, or \latin{quīn}.  \latin{Quīn} is used only after a negative
idea, expressed or implied.
\begin{examples}

\latin{neque enim is es, Catilīna, ut tē pudor umquam ā turpitūdine
  revocārit},
\english{you are not such a man, Catiline, that shame has ever held
  you back from dishonor};
\apud{Cat.}{1, 9, 22}.
(Present Perfect.)

\latin{nec tam sum dēmēns ut nesciam quid sentiātis},
\english{nor am I so mad as not to know what you think};
\apud{Mil.}{27, 72}.

\latin{hostium tam parātus (fuit) ad dīmicandum animus, ut etiam ad
  galeās induendās tempus dēfuerit},
\english{the spirit of the enemy was so ready for battle that time
  failed even for putting on the helmets};
\apud{B.~G.}{2, \emend{116}{21}{20}, 5}.
(Tense aoristic, and absolute.)

\latin{tanta rērum commūtātiō est facta ut nostrī proelium redintegrārent},
\english{so great a change was made that our men renewed the fight};
\apud{B.~G.}{2, \emend{216}{27}{26}, 1}.
(Tense aoristic, but relative to that of \latin{facta est}.)

\latin{mōns altissimus impendēbat, ut perpaucī prohibēre possent},
\english{a high mountain overhung, so that even a very small number
  were able to stop the way};
\apud{B.~G.}{1, 6, 1}.
(Tense of past situation.)

\latin{numquam tam male est Siculīs quīn aliquid facētē dīcant},
\english{things never go so badly with the Sicilians that they haven’t
  some witty thing to say};
\apud{Verr.}{4, 43, 95}.

\latin{eiusmodī tempus erat ut hominēs impūne occīderentur},
\english{the time was such that men were being killed with impunity};
\apud{Rosc.\ Am.}{29, 80}.

\enlargethispage{\baselineskip}

\latin{iīs temporibus fuērunt ut eōrum lūctum ipsōrum dignitās
  cōnsōlārētur},
\english{\emph{(Paullus and Cato)} lived in such times that their high
  position consoled their grief};
\apud{Fam.}{4, 6, 1}.

\end{examples}

\subsubsection

These clauses generally follow incomplete descriptive
words\footnote{\latin{Tālis}, \english{such}, \latin{tantus},
  \english{so great}, \latin{hic}, \latin{ille}, \latin{is}, or
  \latin{iste}, \english{such}, or \latin{tam}, \latin{adeō},
  \latin{sīc}, or \latin{ita}, \english{so}, with an adjective or
  adverb.

  When following an incomplete \emph{adverbial} modifier, or a verb
  without modifier, these clauses describe the character of the
  \emph{act} or \emph{state} expressed by that verb.}; but they may
also follow a verb having no modifier, as in the example \latin{mōns
  impedēbat, ut\dots} above.

\subsubsection

There is no essential difference between the Descriptive Clause of
Actuality and the Clause of Actual Result, when both express the
character of a person, thing, or time.  E.g.\ \latin{is sum quī
  contempserim} of \xref[1]{521}, and \latin{is es ut revocārit} of
\xref[2]{521}, correspond exactly in meaning; as also do \latin{tam
  dēmēns quī} of \xref[1]{521}, and \latin{tam dēmēns ut} of
\xref[2]{521}.

\subsubsection

A Comparative with \latin{quam} may be followed by a Consecutive
\latin{quī}- or \latin{ut}-Clause of Actuality, with the meaning of
\emph{more\ellipsis than such as to\dots}, \emph{too\ellipsis to}, etc.
\begin{examples}

\latin{maior sum quam cui possit fortūna nocēre},
\english{I am too great for fortune to have power to harm me}
(greater than one such that fortune is able);
\apud{Ov.\ Met.}{6, 195}.
Similarly \latin{rēs est vīsa maior quam ut},
\apud{Liv.}{22, 51, 3}.

\latin{nōn longius hostēs aberant quam quō tēlum adigī posset},
\english{the enemy was not farther away than a javelin could be
  thrown}
(than a point such that to it\dots);
\apud{B.~G.}{2,~\emend{111}{21}{20},~3}.

\end{examples}

\subsubsection

\latin{Ita ut} with the Subjunctive may express a Limitation.
\latin{Ita ut} may also express the Way by Which, and \latin{(ita) ut
  nōn}, or \latin{quīn}, an Act \emph{not} Accompanying the main act.
\begin{examples}

\latin{quī ita concēdunt, ut vōbīscum de amōre reī pūblicae certent},
\english{who yield only to the extent of vying \emph{(so that they
    vie)} with you in love for the Commonwealth};
\apud{Cat.}{4, 7, 15}.

\latin{ita ēlūdit ut contendat\dots},
\english{he escapes by urging\dots}
(in such a way that he urges);
\apud{Plin.\ Ep.}{1, 20, 6}.

\latin{ingenium ita laudō ut nōn pertimēscam},
\english{I praise his ability without being overawed by it}
(in such a way that I am not overawed);
\apud{Caecil.}{13, 44}.
Similarly \apud{Pomp.}{7,~19}.

\end{examples}

\subsection

In \term{Substantive Clauses of Actuality (Fact)}:
\begin{enuma}

\item

With \latin{ut} or \latin{ut nōn}, after verbs of \emph{bringing about}
  or of \emph{existence}.\footnote{Such verbs (or phrases) express: (1)~the
  \emph{Bringing About} of something, e.g.\ \latin{faciō},
  \latin{efficiō}, \latin{cōnficiō}, \latin{perficiō}, \latin{cōgō},
  \latin{persuādeō}; (2)~a \emph{Conclusion Brought About}
  (i.e.\ proved), e.g.\ \latin{efficitur}, \latin{sequitur},
  \latin{relinquitur}, \latin{restat}; (3)~a \emph{Fact Occurring or
    Existing}, e.g.\ \latin{fit} (\english{it is brought about},
  \english{the result is}), \latin{accidit}, \latin{contingit},
  \latin{obtingit}, \latin{ēvenit} (\english{it happens}),
    \latin{est} (\english{it is
    the case that}), \latin{accēdit} (\english{it is the case in
    addition that}), \latin{rārum}, \latin{novum}, and the like with
  \latin{est} (\english{it is rarely the case that}, etc.),
  \latin{tantum abest ut} (\english{it is so far from being the case
    that}), \latin{vērum}, \latin{falsum}, and the like with
  \latin{est} (\english{it is true or false that}); (4)~\emph{Existing
    Custom}, e.g.\ \latin{mōs} or \latin{mōris est},
  \latin{cōnsuētūdō} or \latin{cōnsuētūdinis est}, \latin{commūne
    est}.

  Verbs like \vrb{faciō}, \vrb{efficiō}, or \vrb{cogō}, may be
  followed by either the Volitive Subjunctive (\xref[3,
    \emph{a}]{502}), or the Subjunctive of Actuality, according as the
  writer or speaker is thinking of an act as \emph{to be} brought out,
  or as \emph{actually} brought about.  (Cf.\ \latin{efficiēmus nē},
  under \xref[3, \emph{a}]{502}.)}
\begin{examples}

\latin{sed ut possim facit ācta vīta},
\english{but my past life makes me able}
(makes that I am able);
\apud{Sen.}{11, 38}.
(Present state of affairs.  In tense, \latin{possim} = \latin{possum}.)

\latin{hīs rēbus fiēbat,\footnote{The rise of the meaning of Fact out
    of Effect (Result) is due to such phrases as \latin{effectum est
      ut}, \english{it has been brought about that}, = \english{it is
      now the fact that\dots}} ut minus lātē vagārentur},
\english{the result of this was that their wanderings were over a
  narrower territory};
\apud{B.~G.}{1, 2, 4}.
(Past state of affairs.  In tense, \latin{vagārentur} =
\latin{vagābantur}.)

\latin{populī Rōmānī hanc esse cōnsuētūdinem, ut sociōs grātiā,
  dignitāte, honōre auctiōrēs vellet esse},
\english{it was \emph{(said he)} the way of the Roman people to desire
  \emph{(that it desired)} its allies to be magnified in influence,
  dignity, and honor};
\apud{B.~G.}{1, 43, 8}.

\end{examples}

\begin{note}

The Substantive \latin{ut}-Clause of Actuality is often a mere
\emph{verb-noun}.
\begin{examples}

\latin{id quod ipsī diēbus XX aegerrimē cōnfēcerant, ut flūmen
  trānsīrent},
\english{what they themselves had with difficulty accomplished in
  twenty days, namely, the crossing of the river};
\apud{B.~G.}{1, 13, 2}.

\end{examples}

\end{note}

\item

With \latin{quīn}, after verbs or phrases of \emph{doubt} or
\emph{ignorance},\footnote{So especially after \latin{nōn dubitō}, \latin{nōn
    dubium est}, \latin{nōn ignōrō}, \latin{quis dubitat}, \latin{num
    dubium est}, \latin{quis ignōrat}, \latin{nōn abest suspīciō}.} if
these are negatived\versionB*{, or imply a negative}.
\begin{examples}

\latin{nōn dubitat quīn brevī sit Troia peritūra},
\english{he does not doubt that Troy will soon fall};
\apud{Sen.}{10, 31}.
(Periphrastic Future; see \xref[4, \emph{a}]{470}.)

\latin{neque abest suspīciō, quīn ipse sibi mortem cōnscīverit},
\english{nor is suspicion lacking that he took his own life};
\apud{B.~G.}{1, 4, 4}.
(Past Aorist.)

\end{examples}

\begin{note}

The Infinitive also may be used (\xref{589}), and, after verbs not
negatived, always \emph{is} used until after Cicero’s time.

\end{note}

\end{enuma}

\headingG{New Meanings \emend{78}{d}{D}eveloped by the Consecutive
  \latin{quī}-Clause}

\headingC{Restrictive Relative Clause}

\section

A Subjunctive Relative Clause may be used to
\emph{restrict the application of the antecedent}.
\begin{examples}

\latin{omnium ōrātōrum, quōs quidem ego cognōverim, acūtissimum},
\english{the keenest of all orators, at least of such as I have known};
\apud{Brut.}{48, 180}.
(So generally with \latin{quidem}.)

\latin{M.\ Antōnī, omnium ēloquentissimī quōs ego audierim},
\english{Marcus Antonius, the most eloquent of all whom I have heard};
\apud{Tusc.}{5, 19, 55}.

\end{examples}

\begin{minor}

\subsubsection

Without \latin{quidem}, the Determinative Indicative is much more
common; see~\xref{550}.

\end{minor}

\headingC{Causal or Adversative Relative Clause}

\section

A Relative Clause in the Subjunctive may be used to express
\emph{Cause} or \emph{Opposition}.\footnote{The word “cause” is used
  for brevity (here and in \xref{525} and \xref{526}) in place of
  “cause or reason,” and the word “opposition” in place of
  “opposition or contrast.”}
\begin{examples}

\latin{ferreī sumus, quī quicquam huic negēmus},
\english{we are hard-hearted, that we deny him anything};
\apud{Phil.}{8, 8, 25}\emend{83}{}{.}
(Causal; = I say hard-hearted \emph{because}\dots)\emend{84}{.}{}

\latin{illī autem, quī omnia dē rē pūblicā praeclāra sentīrent,
  negōtium suscēpērunt},
\english{and they, since they had only the noblest sentiments toward
  the state, undertook the task};
\apud{Cat.}{3, 2, 5}.
(Causal.)

\latin{tum Cethēgus, quī paulō ante aliquid dē gladiīs ac sīcīs
  respondisset, repente conticuit},
\english{then Cethegus, although a little before he had made some
  reply about the swords and daggers, suddenly became silent};
\apud{Cat.}{3, 5, 10}.
(Adversative.)

\end{examples}

\subsubsection

As compared with the Tacit Causal or Adversative Clause (Indicative;
\xref[\allowbreak \emph{a}]{569}) which merely \emph{suggests} the idea of cause
or opposition without calling attention to it, the Subjunctive Clause
may be called the \term{Explicit Causal} or \term{Adversative Clause}.

\subsubsection

The Causal \latin{quī}-Clause is often introduced by \latin{ut}
(\latin{utpote}), \latin{quippe}, or \latin{praesertim} (\english{as},
\english{in fact}, \english{especially}, etc.).
\begin{examples}

\latin{magna pars Fīdēnātium, ut quī colōnī additī Rōmānīs essent,
  Latīnē sciēbant},
\english{a good many of the people of Fidenae, inasmuch as they had
  been annexed to the Romans as colonists, understood Latin};
\apud{Liv.}{1, 27, 9}.

\end{examples}

\headingG{New Meanings \emend{79}{d}{D}eveloped by the Consecutive
  \latin{cum}-Clause}

\headingC{Descriptive \latin{cum}-Clause of Situation}

\section

A Subjunctive \latin{cum}-Clause may be used to \emph{describe the
  Situation under Which} the main act took place.

The tenses are necessarily those of past situation (Imperfect or Past
Perfect).

\emph{Original type}.\footnote{These examples are simply additional
  instances of the kind seen in \xref[1]{521}.}
\begin{examples}

\latin{accēpit agrum temporibus iīs cum iacērent pretia praediōrum},
\english{he got the land at a time when prices were down};
\apud{Rosc.\ Com.}{12, 33}.

\latin{epistolae tum datae sunt cum ego mē nōn bellē habērem},
\english{the letters were sent at a time when I was not feeling well};
\apud{Att.}{5, 11, 7}.

\end{examples}

\emph{Narrative type}.\footnote{Essentially the same thing as the
  original type, but employed in a new way, namely in narrating.}
\begin{examples}

\latin{ipsī ad mē, cum iam dīlūcēsceret, dēdūcuntur}, \english{the men
  themselves were brought to me as day was breaking}; \apud{Cat.}{3,
  3, 6}.  (\latin{Dīlūcēsceret} is narrated, just as much as
\latin{dēdūcuntur} is.)

\end{examples}

\subsubsection

The Descriptive \latin{cum}-Clause of Situation is often equivalent to
a Participle.
\begin{sidebyside}
\latin{prō castrīs fortissimē pugnāns occīditur},
\english{he is killed fighting bravely in front of the camp};
\apud{B.~G.}{5, 37, 5}.
&
\latin{in secundō proeliō cecidit Critiās cum fortissimē pugnāret},
\english{in the second battle Critias fell fighting bravely};
\apud{Nep.\ Thras.}{2, 7}.
\\

\latin{Antiochum saepe disputantem au\-di\-ē\-bam},
\english{I used often to hear Antiochus arguing};
\apud{Ac.}{2, 4, 11}.
&
\latin{L.\ Flaccum audīvī cum dīceret (= dīcentem)\dots},
\english{I have heard Lucius Flaccus \emph{(saying)} say\dots};
\apud{Div.}{1, 46, 104}.
\end{sidebyside}

\subsubsection

\term{The Descriptive \latin{cum}-Clause of Situation in its Lightest
  Form.}  The construction, as the examples under~\emph{a} indicate,
may at the extreme of its development show the feeling of Situation
but faintly.

\begin{minor}

\subsubsection

The Descriptive \latin{cum}-Clause of Situation stands in sharp
contrast with the Determinative \latin{cum}-Clause
(\xref[\emph{a}]{550}) which simply \emph{defines} the time of the
main act.

\subsubsection

In the future the \latin{cum}-Clause of Situation, unless clearly
consecutive as in \latin{erit illud tempus cum} (\xref[1]{521}), takes
the Indicative.  Thus \latin{cum poterit}, \apud{Cat.}{1, 2, 5}.

\subsubsection

For the Indicative in \latin{cum}-Clauses of Situation in the present,
see \xref[\allowbreak note~1]{569}.

\end{minor}

\headingC{\latin{Cum}-Clause of Situation, with Accessory Causal or
  Adversative Idea}

\section

The Descriptive \latin{cum}-Clause of Situation may be used \emph{with
  an accessory idea} of Cause or Opposition.
\begin{examples}

\latin{hīs cum suā sponte persuādēre nōn possent, lēgātōs ad
  Dumnorīgem mittunt},
\english{when \emph{(and because)} they could not persuade them by
  their own efforts, they sent \emph{(send)} ambassadors to Dumnorix};
\apud{B.~G.}{1, 9, 2}.

\latin{nam cum id posset īnfitiārī, repente praeter opīniōnem omnium
  cōnfessus est},
\english{for when \emph{(and in spite of the fact that)} it was in his
  power to deny, suddenly, contrary to what everybody was looking for,
  he confessed};
\apud{Cat.}{3, 5, 11}.

\end{examples}

\subsubsection

Since the idea of Situation is the \emph{original} one, the preference
should always be given to it in explaining instances where it is still
present.  Thus the above should not be explained merely as causal or
adversative clauses.

\headingC{The Purely Causal or Adversative \latin{cum}-Clause}

\section

A Subjunctive \latin{cum}-Clause may be used, in any tense, to express
\emph{Cause} or \emph{Opposition}.\footnote{The construction arose in
  that of Situation, as in~\xref{525}.  The use of it in cases where
  the idea of Situation was weak, and that of Cause or Opposition
  strong, led to this last type in which the latter idea alone is
  emphasized.  The same cause led to the complete freedom of the tense.}
\begin{examples}

\latin{quae cum ita sint, Catalīna, perge},
\english{since this is so, Catiline, proceed};
\apud{Cat.}{1, 5, 10}.
(Causal.)

\latin{cum ea ita sint, tamen sēsē pācem esse factūrum},
\english{though this is so, yet \emph{(he says)} he will make peace};
\apud{B.~G.}{1, 14, 6}.
(Adversative.)

\end{examples}

\begin{minor}

\subsubsection

The Causal \latin{cum}-Clause, like the Causal \latin{quī}-Clause, may
be introduced by \latin{utpote}, \latin{quippe}, or
\latin{praesertim} (\emph{as}, \emph{in fact}, \emph{especially},
etc.).  \latin{Praesertim} sometimes \emph{follows} \latin{cum}.
\begin{examples}

\latin{cum praesertim videam\dots},
\english{especially since I see\dots};
\apud{Cat.}{3, 12, 28}.

\end{examples}

\end{minor}

\headingC{\latin{Cum}-Clauses in Early Latin}

\section

In early Latin, all \latin{cum}-Clauses, whether narrative, causal, or
adversative, still took the Indicative.  Occasional examples are to be
found even in Cicero’s time and later.  Thus Virgil uses the older
construction, for its old-fashioned effect, in several places, as:
\begin{examples}

\latin{postera cum stellās fugārat diēs, sociōs in coetum advocat
  Aenēās},
\english{when the next dawn had chased away the stars, Aeneas called
  \emph{(calls)} his comrades to an assembly};
\apud{Aen.}{5, 42}.
(In Cicero, this would naturally have been \latin{fugāsset}; compare
\latin{cum dīlūcēsceret}, in~\xref{524}.)

\end{examples}

\headingB{The Subjunctive in Conditions}

\contentsentry{C}{The Subjunctive in Conditions}

\section

\emph{Conditions and Conclusions of all kinds are treated together,
  for convenience, in \xref{573}–\xref{582}}.

\headingB{The Subjunctive of Proviso}

\contentsentry{C}{The Subjunctive of Proviso}

\section

The Subjunctive may be used with \latin{modo}, \latin{dum}, or
\latin{dum modo}, \english{only}, \english{so long as}, \english{so
  long as only}, to express a \emph{Proviso}.  The negative is
\latin{nē} (sometimes, in later Latin, \latin{nōn}).
\begin{examples}

\latin{id Rōmānī, modo nē quid movērent, aequō satis animō
  (ferēbant)},
\english{the Romans were well enough satisfied with this, provided
  only they might remain inactive};
\apud{Liv.}{21, 52, 4}.

\latin{magnō mē metū līberābis, dum modo inter mē atque tē mūrus
  intersit},
\english{you will free me from great fear, if only there shall be a
  wall between you and me};
\apud{Cat.}{1, 5,~10}.

\end{examples}

\headingB{The Subjunctive of Request or Entreaty}

\contentsentry{C}{The Subjunctive of Request or Entreaty}

\section

The Subjunctive may be used to express Request or Entreaty (negative
\latin{nē}):

\subsection

In \term{Independent Sentences}.
\begin{examples}

\latin{iam accipiat, hanc dūcat},
\english{do let him have the money at once, and marry the girl};
\apud{Ph.}{677}.

\end{examples}

\begin{minor}

\subsubsection

The Second Person is almost wholly confined to poetry.

\begin{examples}

\latin{sīs fēlīx},
\english{be thou propitious};
\apud{Aen.}{1, 330}.

\latin{sī tibi vidētur, dēs eī fīliam tuam nūptum},
\english{if you approve, give him your daughter in marriage};
\apud{Nep.\ Paus.}{2, 3}.
(Written to a king).

\end{examples}

\end{minor}

\subsection

In \term{Substantive Clauses}, after verbs or phrases of
\emph{Requesting}, \emph{Begging}, \emph{Imploring}, etc.\footnote{The
  most common of the verbs are \vrb{rogō}, \vrb{ōrō}, \vrb{precor},
  \vrb{obsecrō}, \vrb{impetrō}, \vrb{quaerō}, \vrb{petō}.

It is often hard to determine whether in a given Substantive Clause
the idea of Request is uppermost, or that of Will (\xref[3]{502}).
The distinction is unimportant, since with verbs of weaker meaning the
idea of Will would always tend to \emph{shade into} that of Request.}
\begin{examples}

\latin{Dīviciācus Caesarem obsecrāre coepit nē quid gravius in frātrem
  statueret},
\english{Diviciacus began to entreat Caesar not to pass too severe
  judgment upon his brother};
\apud{B.~G.}{1, 20, 1}.

\end{examples}

\headingB{The Subjunctive of Consent or Indifference}

\contentsentry{C}{The Subjunctive of Consent or Indifference}

\section

The Subjunctive may be used to express \emph{Consent},
\emph{Acquiescence}, or \emph{Indifference} (negative \latin{nē}):

\subsection

In \term{Independent Sentences}.
\begin{examples}

\latin{vīn mē crēdere? Fiat},
\english{do you wish me to believe it? So be it};
\apud{Ph.}{810}.

\latin{moriar nī putō tē mālle ā Caesare cōnsulī quam inaurārī},
\english{may I die \emph{(= I am willing to die)} if I don't believe
  you would rather have Caesar ask your advice than make you a
  millionaire};
\apud{Fam.}{7, 13, 1}.
(Compare the boys’ phrase “I hope to die if it isn’t true.”)

\latin{sibi habeant arma},
\english{they may have their arms};
\apud{Sen.}{16, \emend{217}{58}{57}}.

\end{examples}

\subsection

In \term{Substantive Clauses}, after verbs of \emph{Consent},
\emph{Acquiescence}, or \emph{Indifference}.\footnote{The most common
  of these are \latin{concēdō}, \latin{sinō}, \latin{permittō},
  \latin{licet}.}
\begin{examples}

\latin{huic permīsit utī in hīs locīs legiōnem conlocāret},
\english{he gave him permission to station his legion in those parts};
\apud{B.~G.}{3, 1, 3}.

\latin{quae iam mēcum licet recognōscās},
\english{and these things you may now recall with me}
(it is permitted that you recall);
\apud{Cat.}{1, 3, 6}.

\end{examples}

\section

The Subjunctive may be used to express a \emph{Concession of
  Indifference} (“Concessive” Subjunctive):

\subsection

In \term{Independent Sentences} (negative \latin{nē}).
\begin{examples}

\latin{nē sit sānē summum malum dolor; malum certē est},
\english{grant that pain is not the greatest evil; an evil at any rate
  it is};
\apud{Tusc.}{2, 5, 14}.

\end{examples}

\begin{minor}

\subsubsection

This construction, and the dependent form of it in~2, generally
expresses a concession made merely \emph{for the sake of the
  argument}, and are thus the opposite of the concession of
\emph{fact} (Indicative; \xref[\emph{a}]{556}).

\end{minor}

\subsection

In \term{Dependent Concessions of Indifference}, with \latin{quamvīs}
or \latin{quamlibet}, \english{as much as you please}, \english{even
  though} (negative \latin{nōn}).
\begin{examples}

\latin{illa quamvīs rīdicula essent, sīcut erant, tamen rīsum nōn
  mōvērunt},
\english{no matter how amusing this may have been, as in fact it was,
  nevertheless it didn’t raise a laugh};
\apud{Fam.}{7, 32, 3}.
(Concession of a state of things in the past.)

\latin{senectūs quamvīs nōn sit gravis, tamen aufert eam viriditātem
  in quā etiam nunc erat Scīpiō},
\english{old age, no matter though it be not burdensome, nevertheless
  takes away the freshness which Scipio still possessed};
\apud{Am.}{3, 11}.
(Concession in the general present.)

\end{examples}

\begin{minor}

\subsubsection

\term{Concession of Indifference with \vrb{licet}.}  \vrb{Licet},
\english{it is permitted}, is often used as a Conjunction, in a
Concession of Indifference.
\begin{examples}

\latin{fremant omnēs licet, dīcam quod sentiō},
\english{the whole world may storm at me, still I will say the thing I
  think}
(though the whole world should storm);
\apud{De~Or.}{1, 44, 195}.

\end{examples}

\subsubsection

A Subjunctive Clause with \latin{ut}, \english{even though}, may
express a Concession of Indifference.\footnote{This \latin{ut} is
  probably merely the formal opposite of \latin{nē}
  (cf.\ \ftn*{261}{2}); but the clause \emph{may} originally have been
  dependent (“granting \emph{that}”).}
\begin{examples}

\latin{ac iam ut omnia contrā opīniōnem acciderent, tamen sē plūrimum
  nāvibus posse},
\english{then, too, even though everything should turn out contrary to
  their expectation, \emph{(they felt)} that they were very powerful in
  ships};
\apud{B.~G.}{3, 9, 6}.

\end{examples}

\subsubsection

For the Concession of Fact with \latin{quamquam}, see
\xref[\emph{a}]{556}.  For the same with \latin{etsī},
\latin{tametsī}, etc., see \xref[8]{582}.  For the breakdown of the
distinction between \latin{quamvīs} and \latin{quamquam},
see~\xref{541}.

\end{minor}

\chapter{Subjunctive Constructions Due to the Influence\\ of Other
  Constructions \textup(Analogy\textup)}

\headingB{The Subjunctive of Indirect Discourse}

\contentsentry{C}{The Subjunctive of Indirect Discourse}

\section

When the words or thoughts of any one are reported exactly as spoken
or thought, they are said to be in \term{Direct
  Discourse}.\footnote{Also called \latin{Ōrātiō Rēcta}.}  When they
are made to depend on a verb of saying, thinking, etc.\ (expressed or
implied), they are said to be in \term{Indirect
  Discourse}.\footnote{Also called \latin{Ōrātiō Oblīqua}.}

\subsubsection

In Indirect Discourse, the first and second persons generally change
to the third (\latin{ego} to \latin{sē}, \latin{meus} to \latin{suus},
\latin{hic} and \latin{iste} to \latin{ille}, etc.).  The same applies
to subordinate clauses.

\section
\subsection

As explained in \xref{589},

\emph{Principal Statements} in Indirect Discourse are expressed by the
Infinitive, regularly with a Subject Accusative.\footnote{This
  construction is mentioned here for convenience; but the
  \emph{principle} is simply that of \xref{589}–\xref{593}, which see
  for details and a list of governing verbs.}
\begin{examples}

\latin{Dumnorīgem dēsignārī sentiēbat},
(Caesar) \english{was aware that Dumnorix was meant};
\apud{B.~G.}{1, 18, 1}.
(What Caesar thought was: \latin{Dumnorīx dēsignātur},
\english{Dumnorix is meant}.)

\end{examples}

\subsubsection

The Infinitive of Indirect Discourse often follows a verb which does
not suggest this idea.  The \emph{Infinitive itself} is, in such a
case, the \emph{sign} of the idea.
\begin{examples}

\latin{sēsē omnēs flentēs Caesarī ad pedēs prōiēcērunt; nōn minus sē
  contendere\dots},
\english{all threw themselves, in tears, at Caesar’s feet: they were
  not less urgent \emph{(they said)}\dots};
\apud{B.~G.}{1, 31, 2}.

\end{examples}

\begin{minor}

\subsubsection

All \emph{Conclusions} (being \emph{Statements}) must go into the
Infinitive in Indirect Discourse.  See especially \xref[\emph{b},
  1)]{581}.

\end{minor}

\subsection

Subordinate Clauses \emph{representing Indicatives} or
\emph{Imperatives} are put in the Subjunctive in Indirect Discourse.
These are:
\begin{enumI}

\item
Subordinate Statements of Fact, including Clauses of Reason with
\linebreak
\latin{quod}, \latin{quia}, \latin{quoniam}, or \latin{quandō}
(\xref{535}).

\item
Conditions of Fact (\xref{536}).

\item
Questions of Fact (\xref{537}).

\item
Commands or Prohibitions (\xref{538}).

\end{enumI}

\subsubsection

The negative is the same as in corresponding clauses or sentences in
Direct Discourse, i.e.\ \latin{nē} for commands or prohibitions, and
\latin{nōn} for all other clauses.  (Cf.~\xref{464}.)

\subsubsection

For comparison, the corresponding Indicative or Imperative forms of
\emph{Direct} Discourse will be given for each of the Subjunctive examples.

\section
\subsection

\term{Subordinate Statements of Fact in Indirect Discourse}
\begin{sidebyside}

\cc{1}{\textsc{Indirect Discourse}}
& \cc{1}{\textsc{Direct Discourse}}\\[\smallskipamount]

\latin{vehementer eōs incūsāvit; sē cum sōlā decimā legiōne itūrum, dē
  quā nōn dubitāret},
\english{he rebuked them roundly, \emph{(and said)} that he would go
  with the Tenth Legion alone, about which he had no doubt};
\apud{B.~G.}{1, 40, \emend{218}{15}{1…15}}.
&
\latin{cum sōlā decimā legiōne ībō, dē quā nōn dubitō},
\english{I will go with the Tenth Legion alone, about which I have no
  doubt.}
\end{sidebyside}

\subsubsection

\term{Informal Indirect Discourse.}  The fact that a statement is
quoted may be shown by the mood alone, even if there is no verb of
saying or thinking in the main sentence.
\begin{sidebyside}
\latin{cotīdiē Caesar Haeduōs frūmentum quod essent pollicitī
  flāgitāre},
\english{Ceasar was dunning the Haedui daily for the grain which
  \emph{(as he reminded them)} they had promised};
\apud{B.~G.}{1, 16, 1}.
&
\latin{frūmentum quod estis pollicitī},
(give me) \english{the grain which you have promised}.
\end{sidebyside}

\begin{minor}

\subsubsection

Forward-Moving and Parenthetical Relative Clauses of Fact (\xref{566}
and \xref{567}), since they are additional statements of fact, may be
expressed in Indirect Discourse by the Infinitive.  In the majority of
cases, however, the general mould of the sentence throws such a clause
into the Subjunctive.  An example of each kind follows:
\begin{examples}

\latin{nōn sustinēre dēserere officiī suī partīs, in quō tamen suō
  dolōrī modum im\-pō\-ne\-re},
(Cornutus said) \english{that he could not endure to desert the duties
  of his office; in which, however \emph{(= but in this)} he set
  bounds to his own grief};
\apud{Plin.\ Ep.}{9, 13, 16}.

\enlargethispage{\baselineskip}

\latin{scīre sē illa esse vēra, nec quemquam ex eō plūs dolōris
  capere, proptereā quod per sē crēvisset; quibus opibus ad minuendam
  grātiam ūterētur},
(said) \english{that he knew this to be true, and that no one suffered
  more grief from the fact, for the reason that \(his brother\) had
  grown through his help; which resources he was using to lessen his
  influence};
\apud{B.~G.}{1, 20, 2}.
(Might have been written \latin{quibus ūtī}, \english{which he was
  using}.)  Similarly the parenthetical \latin{quī diēs futūrus
  esset};
\apud{Cat.}{1, 3, 7}.

\end{examples}

\subsubsection

An Infinitive construction is often kept up after a Relative or
\latin{quam} depending upon an Infinitive.  In such a case, the
Infinitive is often expressed but once.
\begin{examples}

\latin{tē suspicor īsdem rēbus quibus mē ipsum commovērī},
\english{I suspect that you are troubled by the same things by which I
  myself am};
\apud{Sen.}{1, 1}.

\end{examples}

\subsubsection

Clauses expressing statements \emph{inserted by the narrator himself}
are really not a part of the Indirect Discourse, and therefore are
expressed by the Indicative.
\begin{examples}[6pt]

\latin{nūntiātum est Ariovistum ad occupandum Vesontiōnem, quod est
  oppidum maximum Sēquanōrum, contendere},
\english{it was announced that Ariovistus was hurring to take
  possession of Besançon, which is the largest town of the Sequani};
\apud{B.~G.}{1, 38,~1}.

\end{examples}

\end{minor}

\subsection

\textbf{Clauses of Reason with \latin{quod}, \latin{quia},
  \latin{quoniam},} or \textbf{\latin{quandō}, in Indirect Discourse}

These are mostly only a \emph{particular kind} of statement of fact,
distinguished from the others for convenience.
\begin{sidebyside}

\cc{1}{\textsc{Indirect Discourse}}
& \cc{1}{\textsc{Direct Discourse}}\\[\smallskipamount]

\latin{Caesar respondit eō sibi minus du\-bi\-tā\-ti\-ōn\-is darī,
  quod memoriā tenēret\dots},
\english{Caesar answered that he felt less hesitation, because he
  remembered\dots};
\apud{B.~G.}{1, 14, 1}.
&
\latin{mihi minus dubitātiōnis datur, quod memoriā teneō\dots},
\english{I feel less hesitation, because I remember\dots}
\\
\latin{grātulāris mihi quod accēperim augurātum},
\english{you congratulate me on having been made an augur};
\apud{Plin.\ Ep.}{4, 8, 1}.
&
\latin{grātulor tibi quod augurātum accēpistī},
\english{I congratulate you on having been made an augur}.
\end{sidebyside}

\subsubsection

\term{Subjunctive of Quoted Reason.}  By a kind of informal Indirect
Discourse, the Subjunctive is used with \latin{quod}, \latin{quia},
\latin{quoniam}, or \latin{quandō} to express a reason \emph{given by
  another than the speaker}.
\begin{examples}

\latin{supplicātiō dēcrēta est, quod Italiam bellō līberāssem},
\english{a thanksgiving was decreed because I had saved Italy from
  war};
\apud{Cat.}{3, 6, 15}.
(This was what the senate said, in passing the decree.)

\end{examples}

\begin{note}[Note 1]

To give the \emph{speaker’s} reason, the Indicative is
used. See~\xref{555}.

\end{note}

\begin{note}[Note 2]

The speaker may quote a reason as \emph{given or felt by himself at
  another time}, and will then use the Subjunctive.

\end{note}

\versionB*{\begin{note}[Note 3]

By a natural confusion, \latin{dīcō} and \latin{exīstimō} are
sometimes put in the Subjunctive in a \latin{quod}-Clause of Reason.
\begin{examples}

\latin{rediit quod sē oblītum nesciō quid dīceret},
\english{he came back, because he said he had forgotten something}
(properly \latin{quod oblītus esset}, \english{because}, as he said,
\english{he had forgotten}); \apud{Off.}{1, 13, 40}.  Similarly
\latin{quod exīstimārent}, \apud{B.~G.}{1, 23, 3}.

\end{examples}

\end{note}}

\subsubsection

\term{Subjunctive of Rejected Reason.}  The Subjunctive is used with
\latin{nōn quod}, \latin{nōn quia}, \latin{nōn quoniam}, \latin{nōn
  quō}, \latin{nōn quīn}, etc., to express a reason \emph{imagined as
  possibly given} by some one, but \emph{rejected} by the
speaker.\footnote{This construction, though no longer a Subordinate
  Statement of Fact, has arisen \emph{out of} such a statement.}  The
true reason is then sometimes added in the Indicative.
\begin{examples}

\latin{nōn idcircō eōrum ūsum dīmīseram, quod iīs suscēnsērem, sed
  quod eōrum mē suppudēbat},
\english{I had given up my intercourse with them \emph{(my books)};
  not that I was angry at them, but because I felt somewhat ashamed of
  myself in their presence};
\apud{Fam.}{9, 1, 2}.

\end{examples}

\versionA{\begin{note}[Note]

By a natural confusion, \latin{dīcō} is sometimes put in the
Subjunctive in a \latin{quod}-Clause of Reason.
\begin{examples}

\latin{rediit quod sē oblītum nesciō quid dīceret},
\english{he came back, because he said he had forgotten something}
(properly \latin{quod oblītus esset}, \english{because}, as he said,
\english{he had forgotten});
\apud{Off.}{1, 13, 40}.
Similarly \latin{quod exīstimārent}; \apud{B.~G.}{1, 23, 3}.

\end{examples}

\end{note}}

\section
\subtitle{\textbf{Conditions of Fact in Indirect Discourse}}

\begin{sidebyside}

\cc{1}{\textsc{Indirect Discourse}}
& \cc{1}{\textsc{Direct Discourse}}\\[\smallskipamount]

\latin{respondit sī obsidēs ab iīs sibi dentur, sēsē\footnote{Compare
    with example to the right, and note the changes of person.} cum
  iīs\footnotemark[\thefootnote] pācem esse factūrum},
\english{he answers that if hostages shall be given him by them, he
  will make peace with them};
\apud{B.~G.}{1, 14, 6}.
(Condition really future to a past time, but picturesquely put as
future to the present.)
&
\latin{sī obsidēs ā vōbīs mihi dabuntur, vōbīscum pācem faciam},
\english{if hostages are \emph{(shall be)} given me by you, I will
  make peace with you}.
(More Vivid Future Condition; \xref[\emph{a}]{579}.)
\\

\latin{eōs incūsāvit: \ellipsis sī quōs adversum proe\-li\-um commovēret, hōs
  reperīre posse},
\english{he rebuked them:\ellipsis \emph{(saying)} that, if the defeat
  disheartened any among them, these could ascertain\dots};
\apud{B.~G.}{1, 40, \emend{219}{8}{1…8}}.
(Condition of Fact, in time \emph{relatively} present to the past
point of view.)
&
\latin{sī quōs adversum proelium commovet, hī reperīre possunt},
\english{if the defeat disheartens any among you, they can ascertain}.
(Condition of Fact in the present; \xref{579}.)

\end{sidebyside}

\subsubsection

\term{Informal Indirect Discourse}.  The expression is often informal,
the indirectness of the Condition being shown only by the
Subjunctive itself.
\begin{sidebyside}

\latin{sī quid dīcere vellet, fēcī potestātem},
\english{I gave him an opportunity, if he wanted to say anything};
\apud{Cat.}{3, 5, 11}.
Cf.\ \latin{quī velint}; \apud{Aen.}{5, 291}.
&
\latin{sī quid dīcere vīs, potestātem habēs},
\english{if you wish to say anything, you have an opportunity}.
(Condition of Fact in the Present.)
\end{sidebyside}

\section
\subtitle{\textbf{Questions of Fact in Indirect Discourse}}
\begin{sidebyside}

\cc{1}{\textsc{Indirect Discourse}}
& \cc{1}{\textsc{Direct Discourse}}\\[\smallskipamount]

\latin{Ariovistus respondit\dots; quid sibi vellet?  cūr in suās
  possessiōnēs venīret?}
\english{Ariovistus answered\dots; \emph{(asking)} what he
  \emph{(Caesar)} wanted; why he \emph{(Caesar)} came into his
  possessions};
\apud{B.~G.}{1, 44, \emend{220}{8}{1…8}}.
&
\latin{quid tibi vīs?  quid in meās possessiōnēs venīs?}
\english{what do you want?  why do you come into my possessions?}
\end{sidebyside}

\subsubsection

\pagebreak

For Rhetorical Questions of Fact in Indirect Discourse, see
\xref[\emph{a}]{591}.

\subsubsection

The Indirect Question of Fact in the Subjunctive may be used with
\emph{any} verb or expression capable of suggesting the interrogative
idea.  The underlying principle is the same as in the above.
\begin{examples}

\latin{quaesīvī quid dubitāret},
\english{I asked why he hesitated};
\apud{Cat.}{2, 6, 13}.

\latin{incertī, quō fāta ferant},
\english{uncertain whither the fates are carrying us};
\apud{Aen.}{3, 7}.

\end{examples}

\subsubsection

Indirect Questions are of substantive nature.  See the example.

\subsubsection

Note the following usages in Indirect Questions:
\begin{enum1*}

\item

The Future Indicative is represented by the Periphrastic Future
(\xref[4, \emph{a}]{470}).
\begin{examples}

\latin{antequam, ista quō ēvāsūra sint, vīderō},
\english{before I see where this is going to turn out};
\apud{Att.}{14, 19, 6}.
(The question is, \latin{quō ēvādent?})

\end{examples}

\item

\latin{Num} does not differ from \enclitic{-ne} in meaning.
\begin{examples}

\latin{quaerō num exīstimēs},
\english{I ask whether you think};
\apud{Clu.}{23, 62}.

\end{examples}

\item

\latin{Ut}, \english{how}, is freely
used.\footnote{\label{ftn:s537:}\latin{Ut} is used also in direct
  \emph{Exclamations}, but not in direct \emph{Questions}, except in
  early Latin and imitations of it.}

\begin{examples}

\latin{docēbat ut omnī tempore tōtīus Galliae prīncipātum Haeduī
  tenuissent},
(Caesar) \english{informed him how the Haedui had constantly held the
  chief position in all Gaul};
\apud{B.~G.}{1, 43, \emend{221}{6}{6–7}}.

\end{examples}

\end{enum1*}

\begin{minor}

\subsubsection

Several interrogative phrases may be used as \emph{indefinites},
without effect upon the mood.  So especially, in Ciceronian Latin,
\latin{nesciō quis} (\latin{quō pactō}, etc.), \latin{mīrē quam}, etc.
\begin{examples}

\latin{nesciō quō pactō ērūpit},
\english{has in some way or other burst forth};
\apud{Cat.}{1, 13, 31}.

\end{examples}

\subsubsection

\latin{Nesciō an} in Ciceronian Latin \emph{generally} implies “I
rather think that\dots” (cf.\ English “I don’t know but”; example
under \xref[3]{507}).  In later Latin, it has its original neutral
meaning (“I don’t know whether\dots”).

\subsubsection

The original Indicative is still sometimes found in Indirect Questions
or Exclamations in poetry (especially in early Latin), and in late
colloquial prose.
\begin{examples}

\latin{sciō quid dictūras (= dictūra es)},
\english{I know what you are going to say};
\apud{Aul.}{174}.

\latin{viden ut geminae stant vertice cristae},
\english{see how upon his head the double plumes stand up};
\apud{Aen.}{6, 779}.

\end{examples}

\end{minor}

% \pagebreak

\section
\subtitle{\textbf{Commands and Prohibitions in Indirect Discourse}}
\begin{sidebyside}

\cc{1}{\textsc{Indirect Discourse}}
& \cc{1}{\textsc{Direct Discourse}}\\[\smallskipamount]

\latin{respondit\dots; cum vellet, congrederētur},
\english{he answered\dots; when he wanted, let him come on};
\apud{B.~G.}{1, 36, 7}.
&
\latin{cum volēs, congredere},
\english{when you want \emph{(shall want)}, come on}.
\\

\latin{nūntius vēnit bellum Athēniēnsīs in\-dīxis\-se: quārē venīre nē
  dubitāret},
\english{a message came that the Athenians had declared war: wherefore
  he should not hesitate to come};
\apud{Nep.\ Ages.}{4, 1}.
&
\latin{Athēniēnsēs bellum indīxērunt: quārē venīre nōlī dubitāre},
\english{the Athenians have declared war: wherefore do not hesitate to
  come}.
(For the usage in direct prohibitions, see \xref[3, \emph{a}]{501}.)

\end{sidebyside}

\vskip-\bigskipamount

\subsubsection

Verbs of saying like \latin{dīcō} and \latin{respondeō} may take a
Volitive Clause, on the principle of \xref[3, \emph{a})]{502}.

\headingB{The Subjunctive by Attraction}

\contentsentry{C}{The Subjunctive by Attraction}

\section

A Dependent Clause attached to a Subjunctive or Infinitive Clause, and
conceived as forming an essential part of the thought conveyed by it,
is put in the Subjunctive.
\begin{examples}

\latin{cum ita balbus esset, ut eius ipsīus artis, cui studēret,
  prīmam litteram nōn posset dīcere},
\english{though he stammered so much, that he could not pronounce the
  first letter of the very art that he was studying};
\apud{De~Or.}{1, 61, 260}.

\latin{mōs est Syrācūsīs, ut, sī quā dē rē ad senātum referātur, dīcat
  sententiam quī velit},
\english{it is the custom at Syracuse that, when any matter is taken
  up in the senate, any one that desires speaks};
\apud{Verr.}{4, 64, 142}.

\latin{mōs est Athēnīs laudārī in cōntiōne eōs quī sint in proeliīs
  interfectī},
\english{it is the custom at Athens to pronounce a public eulogy over
  those who have fallen in battle};
\apud{Or.}{44, 151}.

\latin{quicquid increpuerit, Catilīnam timērī, nōn est ferendum},
\english{it is intolerable that, whatever sound is heard, Catiline
  should have to be feared};
\apud{Cat.}{1, 7, 18}.

\end{examples}

\headingB{The Subjunctive of Repeated Action}

\contentsentry{C}{The Subjunctive of Repeated Action}

\section

The Subjunctive is sometimes used in subordinate clauses, to express
\emph{Repeated Action}.

Any Relative or Conjunction may be used; but the earliest examples are
mostly with \latin{cum}.
\begin{examples}

\latin{vexillum prōpōnendum, quod erat īnsigne cum ad arma concurrī
  oportēret},
\english{the flag had to be displayed, which was the signal, when
  \emph{(ever)} the soldiers must gather to arms};
\apud{B.~G.}{2, \emend{222}{20}{19}, 1}.
Cf.~\apud{}{5, 19, 2}.

\latin{saepe, cum ipse tē cōnfirmāssēs, subitō ipse tē retinēbās},
\english{often, when you had nerved yourself, you would suddenly check
  yourself};
\apud{Quinct.}{11, 39}.

\latin{quod ubi dīxisset, hastam in fīnīs ēmittēbat},
\english{after saying which, \(the priest\) used to cast a spear into
  their territory};
\apud{Liv.}{1, 32, 13}.

\latin{est vulgus cupiēns voluptātum, et, sī eōdem prīnceps trahat,
  laetum},
\english{the populace is fond of pleasure, and delighted if the chief
  ruler leads in that direction};
\apud{Tac.\ Ann.}{14, 14}.

\end{examples}

\subsubsection

In Cicero’s time, the older construction (Indicative; \xref{579}) is
much more common than the Subjunctive.  After Cicero, the Subjunctive
became equally common in tenses of the past, but remained less common
in tenses of the present.

\headingB{The Later Subjunctive with \latin{quamquam}\\
and Indicative with \latin{quamvīs}}

\contentsentry{C}{The (Later) Subjunctive with \latin{quamquam}}

\section

After Cicero, \latin{quamquam} and \latin{quamvīs} are used with
either Indicative or Subjunctive, often without distinction of
meaning.
\begin{examples}

\latin{quamquam movērētur},
\english{although he was moved};
\apud{Liv.}{36, 34, 6}.

\latin{quamvīs īnfestō animō pervēnerās},
\english{no matter in how hostile a spirit you had arrived};
\apud{Liv.}{2, 40, 7}.
Similarly \latin{quamvīs dēiēcit}, \apud{Aen.}{5, \emend{223}{541}{542}}.

\end{examples}

\begin{minor}

\subsubsection

For the regular Ciceronian constructions (\latin{quamquam} Indicative,
\latin{quam\-vīs} Subjunctive), see \xref{556}; \xref[2]{532}.

\subsubsection

\latin{Quamvīs} and, after Cicero, \latin{quamquam} are often used
with other parts of speech than verbs, as in \latin{quamvīs retentus},
\apud{Plin.\ Ep.}{10, 15}; \latin{quamquam parcissimus}, \apud{}{10,
  9}.

\end{minor}

\headingB{The Subjunctive Generalizing Statement of Fact\\
in the Second Singular \emend{72}{}{Person }Indefinite}

\contentsentry{C}{The Subjunctive Generalizing Statement of Fact in
  the Second Person Singular Indefinite}

\section

A General Statement of Fact is sometimes expressed by a Subjunctive in
the \emph{Second Person Singular Indefinite}.
\begin{examples}

\latin{ubi mortuus sīs, ita sīs ut nōmen cluet},
\english{when you’re dead, dead you are in the true sense of the
  word};
\apud{Trin.}{496}.
(The second \latin{sīs} has the force of \latin{es}.)

\latin{qui hostēs patriae semel esse coepērunt, eōs cum ā perniciē reī
  pūblicae rep\-pu\-le\-rīs, nec vī coercēre nec beneficiō plācāre possīs},
\english{if men have once begun to be enemies of their country, then,
  when you have stopped them from destroying the state, you can
  neither constrain them by force nor reconcile them by kindness};
\apud{Cat.}{4, 10, 22}.
(\latin{Possīs} has the force of \latin{potes}.)

\end{examples}

\subsubsection

The Indicative is also used in this sense.

\pagebreak

\chapter{The Indicative}

\contentsentry{B}{Uses of the Indicative}

\section
\subtitle{\textsc{Synopsis of the Principal Uses of the Indicative}}

\begin{indicativesynopsis}

\cc{1}{\textbfsc{independent sentences}}
& \cc{1}{\textbfsc{dependent clauses}} \\[\smallskipamount]
\endhead

Statement or Question of Fact (\xref{545})

(Including Conclusions of Fact; \xref{546}, \xref{579})

&

\centerline{\small\textbf{Essential Clauses, and others derived from them}}

\smallskip

Determinative Clause of Fact: determining the
\begin{indented*}
    person or thing, with \latin{quī}, etc.\ (\xref[and footnote~2]{550})

    kind or amount, with \latin{quālis}, \latin{quantus} (\xref[and
    ftn.]{550})

    manner or degree, with \latin{ut} or \latin{quam} (\xref[and
    ftn.]{550})

    time at which, with \latin{quī} or \latin{cum} (\xref[and
    \emph{a}]{550})

    time before which, with \latin{antequam} or \latin{priusquam}
    (\xref[and \emph{b}]{550})

    time after which, with \latin{postquam} (\xref[and ftn.]{550})

    time from which, with \latin{ex quō} or \latin{ut} (\xref[and
    ftn.]{550})

    time up to which, with \latin{dum}, \latin{dōnec}, or
    \latin{quoad} (\xref[and \emph{b}]{550})

    time during which, with \latin{dum}, \latin{dōnec}, \latin{quoad},
    or \latin{quam diū} (\xref[and \emph{b}]{550})

    time included in the reckoning, with \latin{cum} or \latin{quod}
    (\xref[and ftn.]{550})

    Clause of Equivalent Action, with \latin{quī}, \latin{cum},
    etc. (\xref{551})
\end{indented*}

Substantive \latin{quod}-Clause of Fact (\xref[1]{552})
\begin{indented*}
    \latin{Quod}-Clause of Respect (\xref[2]{552})
\end{indented*}

Substantive \latin{cum}-Clause (\xref{553})

\smallskip

\centerline{\small\textbf{Clauses Less Closely Attached}}

\smallskip

Clause of Cause or Reason, with \latin{quod}, \latin{quia},
    etc.\ (\xref{555})

Adversative Clause of Fact, with \latin{quamquam} (\xref{556})

Aoristic Narrative Clause, with \latin{ubi}, \latin{ut},
    \latin{postquam}, \latin{simul atque}, etc.\ (\xref{557})

Narrative Clause of Situation, with \latin{ubi}, \latin{ut}, or
    \latin{postquam} (\xref{558})

\latin{Dum}-Clause of Situation (\xref{559})

Narrative Clause, with \latin{dum}, \latin{dōnec}, or \latin{quoad}
    (\xref{560})

\latin{Ut}-Clause of Accordance or Reason (\xref{562})

\latin{Ut}-Clause of Harmony or Contrast (\xref{563})

Parallel \latin{cum\ellipsis tum\dots} (\english{not only\ellipsis but
    also\dots}) (\xref{564})

\smallskip

\centerline{\small\textbf{Free Clauses}}

\smallskip

Forward-moving Relative Clause, with \latin{quī}, \latin{cum},
    etc.\ (\xref{566})
\begin{indented*}
    “\latin{Cum inversum}” (\xref[\emph{a}]{566})
\end{indented*}

Parenthetical Clause and “Asides” (\xref{567})

Loosely Attached Descriptive Clause (\xref{568})

Free Descriptive Clause (\xref{569})
\begin{indented*}
    Tacit Causal or Adversative Clause (\xref[\emph{a}]{569})
\end{indented*}
\\\cc{2}{}\\[\medskipamount]%\separator

Independent Conditions of Fact (\xref[\emph{b}]{545})
& Conditions of Fact (\xref{570}, \xref{579})

\end{indicativesynopsis}

\section

The Indicative mood represents an act or state as a fact.  It may
accordingly be used to \emph{state} a fact, to \emph{assume} a fact,
or to \emph{inquire} whether something is a fact (negative
\latin{nōn}).
\begin{examples}

\latin{vēnit},
\english{he has come}
(Declarative)

\latin{sī vēnit},
\english{if he has come}
(Conditional)

\latin{vēnit?}
\english{has he come?}
(Interrogative)

\end{examples}

\subsubsection

The Indicative may also be used in Exclamations (cf.\ \xref[3,
  \emph{a}]{228}).

\chapter{The Indicative in Independent Sentences}

\section

The Indicative may be used in independent sentences to \emph{declare}
something to be a fact, to \emph{inquire} whether something is a fact,
or to \emph{exclaim} about a fact.
\begin{examples}

\latin{fuistī apud Laecam},
\english{you were at Laeca’s house};
\apud{Cat.}{1, 4, 9}.

\latin{quid tacēs?}
\english{why are you silent?}
\apud{Cat.}{1, 4, 8}.

\end{examples}

\begin{minor}

\subsubsection

A \term{Virtual Command} or \term{Exhortation} may be expressed by an
Indicative question with \latin{cūr nōn} or \latin{quīn}, \english{why
  not?}
\begin{examples}

\latin{quīn cōnscendimus equōs?}
\english{why don’t we mount our horses?}
(= let’s mount our horses);
\apud{Liv.}{1, 57, 7}.
Similarly \latin{quīn exercēmus}, \apud{Aen.}{4, 99}.

\end{examples}

\begin{note}[Remark]

From such uses, \latin{quīn} gets the force of urgency, and is then
used with the Imperative also.  See \xref[\emph{b}]{496}.

\end{note}

\subsubsection

An apparently independent statement or question sometimes forms a
Condition.
\begin{examples}

\latin{negat quis: negō},
\english{somebody says “no”: so do I}
(= \textsc{if} somebody says “no”);
\apud{Eun.}{252}.

\end{examples}

\end{minor}

\section

A Statement or Question of Fact to which a Condition is attached is
called a \emph{Conclusion of Fact}. See \xref{573}, \xref{579}.

\chapter{The Indicative in Dependent Clauses}

\section

The Indicative may be used in dependent clauses to \emph{declare}
(\emph{state}) something as a fact, or to \emph{assume} something as a
fact (cf.\ \xref[3, \emph{b}]{228}).

\headingB{A. Dependent Statements\footnotemark of Fact}

\footnotetext{An indicative declarative clause may either \emph{convey
    information} of a fact not hitherto known to the hearer (or
  reader), or may \emph{make use of} a fact supposed to be already
  known by him.  The word “statement” covers both these
  possibilities.}

\section

Dependent Statements of Fact may be subdivided as follows:
\begin{enumI}[III]

\item
Determinative Clauses of Fact, and constructions derived from them.
These, in their very nature, are closely attached to the main sentence
(\emph{essential}).

\item
Clauses of Fact less closely attached, but still dependent.

\item
Clauses of Fact loosely attached; in reality dependent only in form.

\end{enumI}

\headingG{I.\enskip Determinative Clauses of Fact, and Derived Constructions}

\section

The Indicative is used in closely attached (essential) clauses in the
following constructions:

\section

\term{Determinative Clauses of Fact},
determining\footnote{\label{ftn:s550:1}That is, telling \emph{what}
  person, thing, time, etc., is meant.  The Determinative Clause
  pieces out an incomplete pronominal word.  It is therefore
  \emph{pronominal} in its nature, as against the Descriptive Clause,
  which has the force of an \emph{adjective}.} an antecedent idea of
any kind.\footnote{\label{ftn:s550:2}Thus a person or thing
  (\latin{quī}), kind or amount (\latin{quālis}, \latin{quantus}),
  manner or degree (\latin{ut}, \latin{quam}, \english{as}), time
  which (\latin{quī} or \latin{cum}), time \emph{at} which (ablative
  of \latin{quī}, or \latin{cum}), time \emph{before} which
  (\latin{antequam} or \latin{priusquam}), time \emph{after} which
  (\latin{postquam}), time \emph{from} or \emph{since} which
  (\latin{ex quō} or \latin{ut}), time \english{up to} which
  (\latin{dum}, \latin{dōnec}, \latin{quoad}, \english{until}), time
  \emph{during} which (\latin{dum}, \latin{dōnec}, \latin{quoad},
  \latin{quam diū}, \english{so long as}), time included in the
  reckoning (\latin{cum} or \latin{quod}).}
\begin{examples}

\latin{eā legiōne quam sēcum habēbat},
\english{with the legion \emph{(what legion? The one)} which he had
  with him};
\apud{B.~G.}{1, 8, 1}.

\latin{et vīvēs ita ut vīvis},
\english{and you shall live as you are living now}
(= in \emph{that} way in \emph{which});
\apud{Cat.}{1, 2, 6}.

\latin{quī fuit in Italiā temporibus īsdem quibus L.\ Brūtus patriam
  līberāvit},
\english{who was in Italy at the time at which Lucius Brutus freed his
  country};
\apud{Tusc.}{4, 1, 2}.

\latin{haec Crassī cum ēdita ōrātiō est quattuor et trīgintā tum
  habēbat annōs},
\english{at the time when this oration of his was published, Crassus
  was thirty-four years old};
\apud{Brut.}{43, 161}.
Similarly \latin{cum Caesar in Galliam vēnit},
\apud{B.~G.}{6, 12, 1}.

\latin{sī tum cum lēx ferēbātur in Italiā domicilium habuissent},
\english{if, at the time when the law was being passed, they had their
  domicile in Italy};
\apud{Arch.}{4, 7}.

\latin{sex annīs ante quam ego nātus sum},
\english{six years before I was born};
\apud{Sen.}{14, 50}.

\latin{annō postquam ego nātus sum},
\english{one year after I was born};
\apud{Sen.}{4, 10}.

\latin{mānsit in pactō usque ad eum fīnem, dum iūdicēs reiectī sunt},
\english{he stood by the agreement until the judges were rejected}
(up to that limit, namely until\dots);
\apud{Verr.}{A.\ Pr.\ 6, 16}. %%??

\latin{ex eō tempore quō pōns īnstituī coeptus est},
\english{from the time when the bridge began to be built};
\apud{B.~G.}{4, 18, 4}.
Cf.\ \latin{ut ērūpit}, \apud{Cat.}{3, 1, 3}.

\latin{quoad potuit, restitit},
\english{as long as he could, he resisted};
\apud{B.~G.}{4, 12, \emend{224}{6}{5}}.

\latin{vīcēnsimus annus est, cum omnēs scelerātī mē petunt},
\english{it is now the twentieth year \emph{(in which)} that all
  malefactors have been attacking me};
\apud{Phil.}{12, 10, 24}.
Cf.\ \latin{septima vertitur aetās cum}, \apud{Aen.}{5, 626}.

\end{examples}

\subsubsection

Among the more important constructions of this class is the
\term{Determinative \latin{cum}-Clause}, as in the fourth and fifth
examples.

The majority of the Determinative \latin{cum}-Clauses have their verb
in the Perfect (Past Aorist), as in the fourth example.  But clauses
with the Imperfect or Past Perfect are also found, forming a
Determinative Clause of \emph{Situation}, as in the fifth example.
\versionB*{(See also \xref[\emph{d}]{524}.)}

\begin{note}[Note 1]

This very common construction stands in sharp contrast to the
\emph{Descriptive} \latin{cum}-Clause of Situation (Subjunctive;
\xref{524}).  The Indicative \latin{cum}-Clause \emph{defines}
(\emph{dates}) the time at which the main act took place; the
Subjunctive \latin{cum}-Clause \english{describes} the time (gives its
\emph{character}).

\end{note}

\begin{note}[Note 2]

A \latin{quī}-clause or \latin{cum}-clause may sometimes, though
primarily determinative, convey an \emph{accessory} idea of
description, or cause, or opposition, and \emph{vice versa}.
\begin{examples}

\latin{in eō librō quī est dē tuendā rē familiārī},
\english{in that book which deals with the management of the household};
\apud{Sen.}{17, 59}.
(The speaker primarily tells \emph{which} of his books he means; but
incidentally he describes it.)

\latin{an tibi tum imperium hoc esse vidēbātur, cum populī Rōmānī
  lēgātī capiēbantur?}
\english{did this seem to you at that time to be an empire, when
  ambassadors of the Roman people were being taken captive?}
\apud{Pomp.}{17, 53}.

\end{examples}

\end{note}

\begin{note}[Note 3]

\term{Rhetorical Determinative Clause.} The Determinative \latin{quī}-
or \latin{cum}-Clause is sometimes deliberately chosen, for rhetorical
effect, where a descriptive, or causal, or adversative clause would be
equally natural, or more natural.

This clause is often used to \emph{introduce} a sentence in a
\emph{non-committal} manner, the relation between it and the main verb being
left to be discovered when the latter is reached.  It may then be
called the \emph{Introductory Neutral} \latin{quī}- or
\latin{cum}-\emph{Clause}.

The latter use is more common with \latin{quī} than with \latin{cum}.
\begin{examples}

\latin{ego sum ille cōnsul cui nōn cūria umquam  vacua mortis perīculō
  fuit},
\english{I am that consul for whom the senate-house has never been
  free from mortal peril};
\apud{Cat.}{4, 1, 2}.
(Rhetorical, in place of a descriptive clause, with \latin{fuerit},
\english{I am one for whom}.)

\latin{etenim, cum mediocribus multīs grātuītō cīvitātem in Graeciā
  hominēs impertiēbant, Rēgīnōs crēdō, quod scaenicīs artificibus
  largīrī solēbant, id huic summā ingenī prae\-di\-tō glōriā nōluisse},
\english{for, when in Greece men were freely granting citizenship to
  many ordinary persons, the people of Regium, I suppose, were
  unwilling to bestow upon this man, the possessor of the highest
  intellectual distinction, that which they were in the habit of
  bestowing upon stage performers};
\apud{Arch.}{5, 10}.
(Both the \latin{cum}-clause and the \latin{quod}-clause are
introductory and neutral.)

\end{examples}

\end{note}

\subsubsection

Other especially important Clauses of this class are the
\term{Determinative Clauses with antequam \textrm{or} priusquam},
\english{before}, and \latin{dum}, \latin{dōnec}, or \latin{quoad},
\english{until} or \english{so long as}, as in examples six, seven,
eight, and ten under \xref{550}.  In these, the verb states an actual
event looked back upon, \emph{before which}, or \emph{until which},
etc., the main act took place.  They thus stand in sharp contrast to
the anticipatory subjunctive clauses with these connectives (\xref[4
  and 5]{507}), which represent acts, not as actual, but as
\emph{looked forward to}.\footnote{\label{ftn:s550:3}In the sense of
  \emph{so long as}, \latin{dum}, \latin{dōnec}, and \latin{quoad},
  together with \latin{quamdiū}, take an Indicative when referring to
  future time, unless (\xref{509}) the main verb is in the past.  Thus
  \latin{quamdiū quisquam erit, quī tē dēfendere audeat, vīvēs},
  \english{so long as there shall be any one who shall dare to defend
    you, you shall live}; \apud{Cat.}{1, 2, 6}.}

\begin{minor}

\subsubsection

In the construction of the Time after Which, the \latin{post} of
\latin{postquam} sometimes governs a noun.  The same idea may also be
expressed by an ablative noun of time, with a relative in the same
case.
\begin{examples}

\latin{post diem quārtum quam est in Britanniam ventum},
\english{four days after they came to England};
\apud{B.~G.}{4, 28, 1}.

\latin{diēbus decem, quibus māteria coepta est comportārī},
\english{within ten days after the material began to be brought
  together}
(within the ten days within which);
\apud{B.~G.}{4, 18, 1}.

\end{examples}

\end{minor}

\section

\term{Clause of Equivalent Action}, with \latin{quī}, \latin{quod},
\latin{cum}, or \latin{ubi}.
\begin{examples}

\latin{errāstis quī spērāstis},
\english{you were mistaken in hoping};
\apud{Leg.\ Agr.}{1, 7, \emend{225}{23}{22–23}}.
(Your hoping was a mistake.)

\latin{cum quiēscunt, probant},
\english{in acquiescing, they approve};
\apud{Cat.}{1, 8, \emend{226}{2}{21}}.
(Their acquiescence is equivalent to approval.)

\latin{bene fēcistī quod lībertum in animum recēpistī},
\english{you have done well in taking your freedman into your good
  graces again};
\apud{Plin.\ Ep.}{9, 24, 1}.

\end{examples}

\section
\subsection
\term{Substantive \latin{quod}-Clause}.
\begin{examples}

\latin{illud mihi occurrit, quod uxor ā Dolābellā discessit},
\english{this \emph{(fact)} occurs to me, \emph{(name\-ly)} that
  Dolabella’s wife has left him}; \apud{Fam.}{8, 6,
  1}.\footnote{\label{ftn:296:1}\label{ftn:s552:1}When it explains a
  substantive, as in this example (\latin{illud quod}), the clause is
  often called “Explicative.”}

\latin{adde quod ingenuās didicisse fidēliter artīs ēmollit mōrēs, nec
  sinit esse ferōs},
\english{add that to have learned faithfully the liberal arts refines
  the manners, nor suffers them to be boorish};
\apud{Ov.\ Pont.}{2, 9, \emend{227}{49}{47–48}}.
Similarly \latin{accēdēbat quod dolēbant}, \apud{B.~G.}{3, 2, 5}.

\end{examples}

\begin{minor}

\subsubsection

A frequent form of the \latin{quod}-Clause is the condensed expression
\latin{quid quod\dots?} \english{what \emph{(of the fact)} that\dots?}
\begin{examples}

\latin{quid quod tē ipse in custōdiam dedistī?}
\english{what of your giving yourself into custody}
(what of the fact that\dots)?
\apud{Cat.}{1, 8, 19}.

\end{examples}

\end{minor}

\subsection

\term{\latin{Quod}-Clause\footnote{The \latin{quod} of this
    construction and of \xref[1]{552} was originally a Relative
    Pronoun.  As regards case, it stood in \emph{no} tangible relation
    to the verb of its clause.  Accordingly it echoed the prevailing
    case of its antecedent, namely the Nominative-Accusative form.} of
  Respect} (“as to the fact that”).
\begin{examples}

\latin{quod scīre vīs quā quisque in tē fidē sit et voluntāte,
  difficile dictū est dē singulīs},
\english{as to your desiring to know what loyalty and good will this
  and that man have toward you, it is difficult to say this of
  individuals}
(as to this, namely, that you desire);
\apud{Fam.}{1, 7, 2}.
Similarly \latin{quod petiēre}, \apud{Aen.}{2, 180}, and (in Indirect
Discourse) \latin{quod glōriārentur}, \apud{B.~G.}{1, 14, 4}.

\end{examples}

\begin{minor}

\subsubsection

This clause is only a special form of the one given in~1 above.

\end{minor}

\section

\term{Substantive \latin{cum}-Clause}\footnote{This construction has
  come down from a time when \latin{cum} (earlier form \latin{quom};
  cf.\ \latin{quod}) had not yet gained its temporal force.} (\latin{cum}
meaning \emph{that}).
\begin{examples}

\latin{hoc mē beat, quom perduellīs vīcit},
\english{this gives me pleasure, \emph{(namely)} that he has conquered
  his enemies};
\apud{Amph.}{\emend{43}{644}{642}}.

\end{examples}

\begin{minor}

\subsubsection

In Ciceronian Latin, this clause is as regular as the
\latin{quod}-Clause (\xref{555}) with verbs and phrases of
\emph{thanking}, \emph{congratulating}, \emph{rejoicing},
\emph{praising}, and the like (cf.\ English “rejoice that”).
\begin{examples}

\latin{tē, cum istō animō es, satis laudāre nōn possum},
\english{I cannot praise you enough for having such resolution};
\apud{Mil.}{36, 99}.

\end{examples}

\end{minor}

\headingG{II.\enskip Clauses of Fact less closely attached, but still really
  dependent}

\section

The Indicative is used, in clauses less closely attached, in the
following constructions:

\section

\term{Clause of Cause or Reason}, with \latin{quod}, \latin{quia},
\latin{quoniam}, \latin{quandō}, \emph{because},
\emph{since}.\footnote{The construction with \latin{quod} arose out of
  the one in \xref[1]{552}, through examples like \latin{laetae id quod mē
    aspexerant}, \english{glad with reference to this, namely, that
    they had seen me} (i.e.\ \english{because});
  \apud{Hec.}{\emend{44}{368}{369}}
  (cf.\ \versionB*{\latin{id maesta est}, }\xref[\emph{a}]{388}).}
\begin{examples}

\latin{Caesar, quod memoriā tenēbat L.\ Cassium cōnsulem occīsum ab
  Helvētiīs, concēdendum nōn putābat},
\english{Caesar, because he remembered that Lucius Cassius the consul
  had been killed by the Helvetians, thought that the request should
  not be granted};
\apud{B.~G.}{1, 7, \emend{228}{4}{3}}.

\end{examples}

\subsubsection

The Subjunctive is used with these words to express a Quoted or
Rejected Reason.  (Informal Indirect Discourse; see \xref[2, \emph{a}
  and \emph{b}]{535}.)

\section

\term{Adversative Clause of Fact}, with \latin{quamquam} (“although
in fact”).
\begin{examples}

\latin{illōs, quamquam sunt hostēs, tamen monitōs volō},
\english{although they are enemies, yet I wish them to be well
  warned};
\apud{Cat.}{2, 12, 27}.

\end{examples}

\subsubsection

When this Clause concedes an objection made by an adversary, it
becomes a \term{Concession of Fact} (\emph{although it \textsc{is
    true} that}).  The construction is thus in contrast with that of
the Concession of Indifference (Concession for the Sake of the
Argument) with \latin{quamvīs} (\xref[2]{532}), which means \emph{no
  matter how much}, \emph{even though}, and does not deal with the
question whether the thing conceded is true or not.

\begin{minor}

\subsubsection

For “corrective” \latin{quamquam}, \latin{etsī}, \latin{tametsī},
see \xref[7]{310}.

\end{minor}

\section

\term{Aoristic Narrative Clause}, with \latin{ubi}, \latin{ut},
\latin{postquam},\footnote{The form \latin{posteā quam} is more frequent
  in Cicero, \latin{postquam} in Caesar.} or \latin{simul atque}, and
an aorist tense.
\begin{examples}

\latin{ubi dē eius adventū Helvētiī certiōrēs factī sunt, lēgātōs ad eum
  mittunt},
\english{when the Helvetians were informed of his coming, they sent
  \emph{(send)} ambassadors to him};
\apud{B.~G.}{1, 7, 3}.

\latin{id ubi vident, mūtant cōnsilium},
\english{when they see this, they change their plan};
\apud{B.~C.}{2, 11, 2}.
\latin{(Vident} is an Historical Present.)

\end{examples}

\begin{minor}

\subsubsection

Less common Introductory words or phrases for this clause are
\latin{ut prīmum}, \latin{ut semel}, \latin{ubi prīmum},
\latin{simul}, \latin{cum prīmum} (\latin{prīmus}, \latin{prīma},
etc.).

\end{minor}

\section

\term{Narrative Clause of Situation}, with \latin{ubi}, \latin{ut},
\latin{postquam},
\versionB*{ or \latin{simul atque}, }%
and a tense of past situation (\versionA{the }less common\versionA{
  usage}).
\begin{examples}

\latin{postquam rēs eōrum satis prōspera vidēbātur},
\english{when now their affairs seemed in a prosperous condition};
\apud{Sall.\ Cat.}{6, 3}.
Cf.\ \apud{B.~G.}{7, 87, 5}.

\end{examples}

\section

\term{\latin{Dum}-Clause of Situation.}  The tense is regularly the
\emph{Present}, no matter what the tense of the main Verb may be.
\begin{examples}

\latin{dum haec geruntur, Caesarī nūntiātum est},
\english{while these things were going on, word was brought to
  Caesar\dots};
\apud{B.~G.}{1, 46, 1}.

\end{examples}

\begin{minor}

\subsubsection

Out of the \latin{dum}-Clause of Situation arises the
\term{\latin{dum}-Clause of the Way by Which}.  Thus \latin{hī dum
  aedificant, in aes aliēnum incidērunt}, \english{while \emph{(= by)}
  building houses, these men have fallen into debt}; \apud{Cat.}{2, 9,
  20}.

\subsubsection

A \latin{dum}-Clause is often used to express a \term{Situation of
  which Advantage is to be taken}.  Thus \latin{abīte, dum est
  facultās}, \english{escape while there is opprtunity};
\apud{B.~G.}{7, 50, 6}.

\subsubsection

In later Latin, the Imperfect is sometimes used in the
\latin{dum}-Clause of Situation.  Thus \latin{dum cōnficiēbātur},
\apud{Nep.\ Hann.}{2, 4}.

\end{minor}

\section

\latin{Narrative Clause} with \latin{dum}, \latin{dōnec}, or
\latin{quoad}, \english{until}.  The tense is regularly the Perfect
(past aorist).
\begin{examples}

\latin{neque fīnem sequendī fēcērunt, quoad equitēs praecipitēs hostīs
  ēgērunt},
\english{nor did they stop the pursuit, until the cavalry drove the
  enemy headlong}
(= they pursued, and finally\dots);
\apud{B.~G.}{5, 17, 3}.

\end{examples}

\begin{minor}

\subsubsection

In such a clause, the verb tells a new fact in the narration just as
much as the main verb does.  The construction is more common than that
of \xref[\emph{b}]{550}.

\end{minor}

\section

\term{Narrative Clause} with \latin{antequam} or \latin{priusquam}.
The tense is regularly the Perfect (past aorist).
\begin{examples}

\latin{neque prius fugere dēstitērunt quam ad flūmen Rhēnum
  pervēnērunt},
\english{nor did they cease to flee until they came to the Rhine}
(= they kept on fleeing, and finally they came\dots);
\apud{B.~G.}{1, 53, 1}.

\end{examples}

\begin{minor}

\subsubsection

In such a clause the verb tells a new fact in the narration just as
much as the main verb does.  The force is possible only when the main
verb is negatived.

\end{minor}

\section

\term{\latin{Ut}-Clause of Accordance or Reason} (English “as” =
“for”).
\begin{examples}
\latin{haec ex oppidō vidēbantur, ut erat ā Gergoviā dēspectus in
  castra},
\english{these things were seen from the town, as there was a prospect
  from Gergovia into the camp};
\apud{B.~G.}{7, 45, 4}.

\latin{hōrum auctōritāte fīnitimī adductī (ut sunt Gallōrum subita
  cōnsilia), Trebium retinent},
\english{led by their influence \(for the resolutions of the Gauls are
  quickly taken\), their neighbors detain Trebius};
\apud{B.~G.}{3, 8, 3}.

\end{examples}

\section

%%* unfortunate forced break

\term{\latin{Ut}-Clause of Harmony or Contrast} (\latin{ut\ellipsis ita}
or \latin{sīc\dots}, \english{as\ellipsis so\dots}, or 
\english{while\allowbreak\ellipsis yet\dots}).
\begin{examples}

\latin{ut magistrātibus lēgēs, ita populō praesunt magistrātūs},
\english{as the laws are superior to the magistrates, so the
  magistrates are superior to the people};
\apud{Leg.}{3, 1, 2}.

\latin{ut ad bella suscipienda Gallōrum alacer est animus, sīc mollis
  ad calamitātēs perferendās mēns eōrum est},
\english{while the spirit of the Gauls is quick to undertake war, yet
  their mind is not sturdy for enduring reverses};
\apud{B.~G.}{3, \emend{126}{19}{17}, 6}.

\end{examples}

\section

\term{Parallel \latin{cum} and \latin{tum}} (\english{while\ellipsis at
  the same time\dots}, \english{not only\dots, but also\dots}).
\begin{examples}

\latin{cum omnis iuventūs eō convēnerant, tum nāvium quod ubīque
  fuerat coēgerant},
\english{not only had all the young men gathered there, but they had
  got together all the ships there had been anywhere};
\apud{B.~G.}{3, \emend{229}{16}{15}, 2}.
(Originally \emph{when\ellipsis at the same time\dots})

\end{examples}

\begin{minor}

\subsubsection

A slight emphasis is thrown upon the second member.

\subsubsection

The presence of the idea of \emph{Contrast} (a sort of
\emph{Opposition}) sometimes brings about the use of the Subjunctive
(\xref{526}).

\subsubsection

When the same verb is meant in both clauses, it is expressed but once.
Sometimes no verb at all is used (\term{Adverbial
  \latin{cum}\ellipsis\latin{tum}}).
\begin{examples}

\latin{cum illa certissima vīsa sunt argūmenta, tum multō certiōra
  illa},
\english{not only did these evidences seem very sure, but still surer
  the following};
\apud{Cat.}{3, 5, 13}.

\latin{cum cārum, tum dulce},
\english{not only dear, but sweet};
\apud{Cat.}{4, 7, 16}.

\end{examples}

\end{minor}

\headingG{III.\enskip Clauses of Fact loosely attached; in reality
  dependent only in Form (Free Clauses)}

\section

The Indicative is used in clauses very loosely attached (in reality
completely independent), in the following constructions:

\section

\term{Forward-moving Relative Clause}, with \latin{quī}, \latin{cum},
\latin{ut} (\english{as}), etc.  Such a clause \emph{advances the
  thought}, just as an independent sentence beginning with \latin{et
  is}, \latin{et tum}, \latin{et sīc}, etc., would do.
\begin{examples}

\latin{nec hercule, inquit, sī ego Serīphius essem, nec tū sī
  Athēniēnsis, clārus umquam fuissēs; quod eōdem modō dē senectūte
  dīcī potest},
\english{I should never have been renowned, said he, if I were a
  Seriphian, nor, by Jove, would you have been, if you were an
  Athenian.  Which \emph{(= and this)} may be said in like manner of
  old age};
\apud{Sen.}{3, 8}.

\latin{spērāns Pompeium interclūdī posse; ut accidit\dots},
\english{hoping that Pompey could be cut off; as \emph{(= and this)}
  happened};
\apud{B.~C.}{3, 41, 3}.

\latin{litterās recitāstī, quās tibi ā C.\ Caesare missās dīcerēs; cum
  etiam es argūmentātus},
\english{you read a letter, which you said had been sent you by Gaius
  Caesar; whereupon \emph{(= and then)} you went so far as to
  argue\dots};
\apud{Dom.}{S.~9, 22}.

\end{examples}

\subsubsection

Out of this use arises the common use in which the \latin{cum}-Clause
follows the main clause (hence called “\term{cum inversum}”), and
expresses an act that comes in upon an existing state of affairs.
\begin{examples}

\latin{iam montānī conveniēbant, cum repente cōnspiciunt hostīs},
\english{already the mountaineers were gathering, when suddenly they
  see the enemy};
\apud{Liv.}{21, 33, 2}.
Similarly \latin{cum cognōscunt}, \apud{B.~G.}{6, 7, 2};
\latin{cum reddit}, \apud{Aen.}{2, 323}.

\end{examples}

\section

\term{Parenthetical Clauses, and “Asides.”}  A Parenthetical Clause
with \latin{quī}, \latin{cum}, \latin{ut}, etc., may be used to insert
into a sentence some fact which is of interest by the way. Such
clauses are really independent sentences.

Or, a clause with \latin{quī}, \latin{cum}, etc., may insert
\emph{between sentences} something which for the moment carries the
mind away from the direct progress of the thought.  Such “Asides”
are really independent sentences.
\begin{examples}

\latin{intereā ā lacū Lemannō, quī in flūmen Rhodanum īnfluit, ad
  montem Iūram, quī fīnīs Sēquanōrum ab Helvētiīs dīvidit, mūrum
  perdūcit},
\english{meanwhile he builds a wall running from Lake Leman, which
  empties into the Rhone, to Mount Jura, which separates the lands of
  the Sequani from the Helvetians};
\apud{B.~G.}{1, 8, 1}.
(In place of the two words \latin{quī} we might have had \latin{hic
  lacus} and \latin{hic mōns}.)

\latin{Gallia sub septentriōnibus, ut ante dictum est, posita est},
\english{Gaul, as has been said above, lies to the north};
\apud{B.~G.}{1, 16, 2}.
(Parenthetical Clause. In place of \latin{ut}, we might have had
\latin{id}.)

\latin{quaestor deinde quadrienniō post factus sum, quem magistrātum
  gessī cōn\-su\-li\-bus Tuditānō et Cethēgō, cum quidem ille admodum senex
  suāsor lēgis Cinciae dē dōnīs et mūneribus fuit},
\english{then four years later I was made quaestor,
\allowbreak
  —which office, by
  the way, I held in the consulship of Tuditanus and Cethegus,—at
  which time, by the way, he, though very old, was an active promoter
  of the Cincian law about gifts and bribes};
\apud{Sen.}{4, 10}. (Two successive “Asides.”)

\end{examples}

\begin{note}

The forward-moving Clause advances the thought: the Parenthetical
Clause and the “Aside” delay it for the moment.

\end{note}

\section

\term{Loosely Attached Descriptive Clause}, with \latin{quī} or
\latin{cum}.  A Descriptive Clause that might have been in the
Subjunctive (\xref[1]{521}) is sometimes purposely \emph{attached
  loosely}, with the feeling of a forward-moving statement.

\begin{examples}

\latin{nōn nūllī sunt in hōc ōrdine, quī aut ea quae imminent nōn
  videant, aut ea quae vident dissimulent; quī spem Catilīnae mollibus
  sententiīs aluērunt},
\english{there are a number of men in this body, who either do not see
  that which is hanging over our heads, or conceal that which they do
  see; who \emph{(= and these)} by their half-hearted expressions of
  opinion have fed the hopes of Catiline};
\apud{Cat.}{1, 12, 30}.
(The first clause is closely attached, the second loosely.)  Similarly
\latin{erat alia vehemēns opīniō, quae animōs pervāserat},
\apud{Pomp.}{9, 23}.

\latin{ūnus et alter diēs intercesserat, cum rēs parum certa
  vidēbātur},
\english{a couple of days had passed, in which \emph{(= and in this
    time)} the matter seemed rather indefinite};
\apud{Clu.}{26, 72}.

\end{examples}

\begin{minor}

\subsubsection

Similar loosely attached Causal or Adversative Clauses occur.

\subsubsection

This Loosely Attached Descriptive Clause, which might be replaced by
the Subjunctive, must be distinguished from the following, in which
the Subjunctive \emph{could not be used}, unless an independent
sentence with the same meaning would take this mood.

\end{minor}

\section

\term{Free\footnote{\label{ftn:302:}Free clauses are clauses that can
    be left out without making the sentence grammatically incomplete.
    They are opposed to \emph{essential} (i.e.\ \emph{necessary})
    clauses of various kinds.} Descriptive Clause}.  After an
antecedent complete in itself, a relative clause (with \latin{quī},
\latin{cum}, etc.)\ is really an independent statement, and
accordingly takes \emph{whatever mood the statement in itself
  requires},—generally the Indicative.
\begin{examples}

\latin{imāgō avī tuī, clārissimī virī, quī amāvit patriam},
\emph{the likeness of your grandfather, a most eminent man, who loved
  his country};
\apud{Cat.}{3, 5, 10}.

\latin{relinquēbātur ūna per Sēquanōs via, quā Sēquanīs invītīs īre
  nōn poterant},
\english{there remained only the way through the country of the
  Sequani, by which \emph{(= and by this)} they could not pass without
  the consent of the Sequani};
\apud{B.~G.}{1, 9, 1}.

\latin{dōnec ad haec tempora, quibus nec vitia nostra nec remedia patī
  possumus, perventum est},
\english{until we reached the present time, in which we can endure
  neither our defects nor the remedies applied to them};
\apud{Liv.}{1, Praef.~9}.
(\latin{Cum} might have been used, in place of \latin{quibus}.)

\end{examples}

\begin{note}[Note 1]

A Descriptive Clause is necessarily a free one when it refers
immediately to an antecedent \emph{complete in itself}, e.g.\ a word
denoting a person (as \latin{Cicerō}, \latin{ego}, \latin{tū}); a noun
with a determinative or possessive pronoun (as \latin{hic homō},
\latin{hōc tempore}), or the adverb \latin{nunc}.  Hence the mood is
Indicative in clauses of present situation (\latin{nunc cum}, etc.),
unless the idea of \emph{cause} or \emph{opposition} is to be brought
out, in which case the Subjunctive is used.

\end{note}

\begin{note}[Note 2]

After an antecedent \emph{not} complete in itself, a Descriptive
Clause of Fact \emph{must} be in the Subjunctive (unless it expresses
a Condition; \xref{579}).  The reason for this difference is that the
\emph{Subjunctive} Descriptive Clause of Fact is of consecutive origin
(\xref[1, \emph{e}]{521}), and gets its mood in that way; while the
Free Descriptive Clause is \emph{not} of consecutive origin.

\end{note}

\subsubsection

These free descriptive clauses often \emph{suggest} the causal or
adversative idea, and may then be called \term{Tacit Causal} or
\term{Adversative Clauses}, in opposition to Explicit Causal or
Adversative Clauses (\xref{523}) in which the \emph{mood calls
  attention} to the relation.

\begin{examples}

\latin{ō tē ferreum, quī illīus perīculīs nōn movēris!}
\english{O you hard-hearted man, who are not moved by his dangers!}
\apud{Att.}{13, \emend{45}{30, 1}{29, 3}}.
(Might have been \latin{quī nōn moveāris}.  Cf.\ \latin{ferreī sumus,
  quī negēmus} under \xref{523}.)

\latin{nisi vēro ego vōbīs cessāre nunc videor, cum bella nōn gerō},
\english{unless indeed I seem to you to be a laggard in these days, in
  which I am not carrying on war};
\apud{Sen.}{6, 18}.
(Might have been \latin{cum bella nōn geram}, \english{since I am not
  carrying on war}.)

\end{examples}

\headingB{B. Dependent Conditions of Fact}

\section

The Indicative may be used in Conditions which \emph{assume something
  to be a fact}.

\emph{Conditions and Conclusions of all kinds are, for convenience,
  treated together in \xref{573}–\xref{582}.}

\chapter{Special Uses of the Present, Perfect, and Future Indicative}

\section

The Freer Present Indicative may be used to express a number of ideas
which are \emph{usually, or sometimes, expressed by
  other moods or tenses}.  The negative is
\latin{nōn}.\footnote{\label{ftn:303:}These uses have probably come
  down from a time when only a single set of verb-forms existed,
  expressing distinctions of person and number, but none of mood or
  tense.  Compare the use of the English verb by a foreigner who has
  learned only one form.}

These are especially the ideas of Resolve, Deliberation, Perplexity,
etc., Anticipation (with \latin{dum}, \latin{dōnec}, \latin{quoad},
\latin{antequam}, \latin{priusquam}, etc.), Consent, Future Condition
(with \latin{sī}, etc.), or Vivid Statement about the future or the
past (the latter is called the Historical Present; \xref[1]{491}).
\begin{examples}

\latin{quid agō?  Rūrsusne procōs inrīsa priōrēs experiar?}
\english{what am I to do?  Am I now, insulted \emph{(by Aeneas)}, to
  try once more my former suitors?}
\apud{Aen.}{4, 534}. (Perplexity; cf.\ \xref{503}.)

\latin{nunc, antequam ad sententiam redeō, dē mē pauca dīcam},
\english{now, before I return to the voting, I wish to say a few words
  about myself};
\apud{Cat.}{4, 10, 20}.
(Act anticipated and prepared for; cf.\ \xref[4, \emph{a}]{507}.)

\latin{sed mihi vel tellūs optem prius īma dehīscat, ante, Pudor, quam
  tē violō},
\english{but I should wish the depths of earth to yawn for me, before
  I wrong thee, Modesty!}
\apud{Aen.}{4, 24}.
(Act deprecated; cf.\ \xref[4, \emph{d}]{507}.)

\latin{sī in eādem mente permanent, eq quae merentur exspectent},
\english{if they remain of the same mind, let them expect that which
  they deserve};
\apud{Cat.}{2, 5, 11}.
(Future condition; cf.\ the equivalent \latin{sī permanēbunt},
\apud{Cat.}{2, 8, 18}.)

\end{examples}

\begin{minor}

\subsubsection

Under the influence of the Present, the Present Perfect is sometimes
used to express the same ideas, but with greater energy or emphasis
(\xref{490}).
\begin{examples}

\latin{sī eundem mox in aestimandā fortūnā vestrā habueritis, vīcimus,
  mīlitēs},
\english{if you have the same \emph{(spirit)} presently in judging of
  your own fate, we have already conquered, soldiers};
\apud{Liv.}{21, 43, 2}.  (\latin{Vīcimus} is energetic.)

\end{examples}

\subsubsection

In Cicero, the Present Indicative is more common than the Subjunctive
after \latin{antequam} and \latin{priusquam}.

\subsubsection

As in the case of the Anticipatory Subjunctive (\xref[4,
  note~1]{507}), the formula that came into use in cases of true
anticipation was naturally used for the \emph{operations of nature} as
well, as in the following:
\begin{examples}

\latin{membrīs ūtimur priusquam didicimus cuius ea causā ūtilitātis
  habeāmus},
\english{we use our limbs before we have learned for what use we
  possess them};
\apud{Fin.}{3, 20, 66}.

\end{examples}

\end{minor}

\section

The Future Indicative may be used to express a number of ideas which
are \emph{generally}, \emph{or sometimes}, \emph{expressed by the
  Subjunctive}. The negative is \latin{nōn}.

These are especially the ideas of Resolve, Exhortation, Command or
Prohibition, Deliberation or Perplexity, Surprise or Indignation,
Consent or Acquiescence.
\begin{examples}

\latin{nōn feram},
\english{I shall not \emph{(= will not)} bear it};
\apud{Cat.}{1, 5, 10}.

\latin{sinite īnstaurāta revīsam proelia; nunquam omnēs hodiē moriēmur
  inultī},
\english{let me go back and see the conflict set on foot again.  We
  shall not all die unavenged to-day, ah no};
\apud{Aen.}{2, \emend{230}{668}{669–670}}.
(Hortatory; = let us not.)

\latin{referēs ergō haec et nūntius ībis Pēlīdae},
\english{you will \emph{(= shall)} report this, then, and will go as a
  messenger to the son of Peleus};
\apud{Aen.}{2, \emend{231}{546}{547}}.
(Command.)

\latin{quōs Sīdoniā vix urbe revellī rūrsus ventīs dare vēla iubēbō?}
\english{shall I \emph{(= can I)}, who have with difficulty torn my
  men from the Sidoniam city, again bid them give their sails to the
  wind?}
\apud{Aen.}{4, 545}.
(Perplexity.)

\latin{dēdēmus ergō Hannibalem?}
\english{shall we, then, give up Hannibal?}
\apud{Liv.}{21, 10, 11}.
(Indignation; = surely you don’t mean this!)
Cf. \latin{patiēre?}
\apud{Cat.}{1, 11, 27}.

\end{examples}

\begin{minor}

\subsubsection

In many of these uses, the Future may conveniently be called the
\emph{Volitive Future Indicative} (so in the first three examples).

\end{minor}

\headingB{Summary of Conditions and Conclusions}

\contentsentry{C}{Summary of Conditions and Conclusions}

\headingG{Indicative and Subjunctive}

\section

A Conclusion is a \emph{conditioned} statement.

The Condition (Assumption\footnote{The word “condition” is
  convenient, as being in common use.  The word \emph{assumption}
  would more exactly fit the mental operaton, would balance the verb
  \emph{assume}, and would perfectly express the character of the
  first type (assumption of fact).}) \emph{assumes} something as true
(or realized), and the Conclusion \emph{asserts} something as true (or
realized) only \emph{if} the thing assumed is true (or realized).

\section

Conclusions may be either Statements of \emph{Fact} (Indicative) or
Statements of an \emph{Ideal Certainty} (Subjunctive).

The corresponding Conditions will be either Assumptions of \emph{Fact}
(Indicative) or \emph{Ideal} Assumptions (Subjunctive).

\section
\subtitle{\textsc{Table of Conditions and Conclusions}}

%% VISUAL FORMATTING

\smallskip

\noindent
\begin{tabular*}{\textwidth}{@{}l@{ }l@{}}

\emph{A}.\enskip Conditions and Conclusions \emph{of Fact}.
& \groupL{%
    \vbox{%
        \hsize 2.25in
        \leftskip 1em
        \parindent -1em
        \baselineskip13pt
        \raggedright
        In any time. \emph{Indicative, in any tense}.%
    }%
  } \\

\emph{B}.\enskip \emph{Ideal} Conditions and Conclusions.
& \groupL{
    \vbox{%
        \hsize 2.25in
        \leftskip 1em
        \parindent -1em
        \baselineskip13pt
        \raggedright
        In future, and so realizable.  \emph{Present or Perfect
        Subjunctive}.} \\
    \vbox{%
        \hsize 2.25in
        \leftskip 1em
        \parindent -1em
        \baselineskip13pt
        \raggedright
        In Present or Past, and so unrealized (contrary to fact).
        \emph{Imperfect or Past Perfect Subjunctive}.
    }
  }

\end{tabular*}

\section

Any kind of Condition and Conclusion may be used either (1)~with
individual\footnote{\label{ftn:s567:}Often called “particular.”}
Meaning, or (2)~with generalizing Meaning.  The form is in general the
same.

\begin{minor}

\subsubsection

The only exceptions to this rule are: the Generalizing Condition in
the Second Person Singular Indefinite (always Subjunctive;
\xref[2]{504}), and the Subjunctive of Repeated Action (not yet common
in Cicero, and never common in tenses of the present; \xref{540}).

\end{minor}

\section

Conditions may be introduced by a Relative\footnote{The oldest way of
  expressing a Condition was doubtless by the use of the Relative (the
  simplest of all connectives), not by \latin{sī}.} or an
equivalent\footnote{\label{ftn:s577:3}Connectives like \latin{cum},
\latin{dum}, \latin{antequam}, \latin{postquam}, \latin{quotiēns},
\latin{quotiēnscumque}, etc.} (\emph{Conditional} or \emph{Assumptive}
  Clauses; \xref[2]{228}), or by \latin{sī}, \latin{nisi}, \latin{nī},
  or \latin{sīn}.\versionA*{ In what follows, the two kinds will be
    treated together.\par} The negative is \latin{nōn}.

\versionB*{%
\subsubsection

The tense of the Condition often expresses the act as \emph{in a
  finished state} at the time of the tense of the
Conclusion. (Cf.\ \xref{494}.)}

\headingC{\latin{Sī}, \latin{sī nōn}, \latin{nisi},
  \latin{nī}, and \latin{sīn}.  Meanings and Uses.}

\section
\subsection

\latin{Sī} means \emph{in case}, \emph{if} (cf.\ \latin{sī-c},
\english{in that case}).

\subsection

The negative of \latin{sī} is \latin{sī nōn}, \english{if not}, if a
single word is especially negatived, or \latin{nisi},
\english{unless}, if the whole condition is negatived,
\begin{examples}

\latin{sī stāre nōn possunt, corruant},
\english{if \(these men\) are unable to stand, let them fall};
\apud{Cat.}{2, 10, 21}.
(\latin{Nōn possunt} = \latin{nequeunt}.)

\latin{dēsilīte, inquit, commīlitōnēs, nisi vultis aquilam hostibus prōdere},
\english{“leap down, fellow soldiers,” he said, “unless you wish to
  betray the eagle to the enemy”};
\apud{B.~G.}{4, 25, 3}.

\end{examples}

\subsection

When a \emph{second} Condition is opposed to the first, it is
introduced, if positive, by \latin{sīn}, \english{but if}
(\apud{Cat.}{1, 7, 18}); if negative, by \latin{sī nōn}, \english{if
    not} (\apud{B.~G.}{1, 35, 4}), or \latin{sī minus}, \english{if
    not}, \english{otherwise} (\apud{B.~G.}{2, 9, \emend{232}{4}{5}};
    \apud{Cat.}{1, 5, 10}), the latter being regular where the verb is omitted.

\begin{minor}

\subsubsection

\latin{Nisi} is often used ironically of an afterthought.  Thus
\latin{nisi forte}, \english{unless perhaps} (\apud{Cat.}{4, 10, 21});
\latin{nisi vērō}, \english{unless indeed} (\apud{Cat.}{4, 6, 13}).

\subsubsection

\latin{Nisi} often means merely \english{except}, \english{but}.
\begin{examples}

\latin{nihil cōgitant nisi caedem},
\english{they think of nothing but bloodshed};
\apud{Cat.}{2, 5, 10}.

\end{examples}

\end{minor}

\subsection

\latin{Nī}, \english{unless}, is sometimes used in place of
\latin{nisi}, mainly in the poetical or later style.

\begin{minor}

\subsection

A Condition may be introduced by \latin{ita}, \latin{eā condiciōne},
etc.
\begin{examples}

\latin{ita senectūs honesta est, sī sē ipsa dēfendit},
\english{old age is honorable \emph{(on these terms, namely)} if it
  defends itself};
\apud{Sen.}{11, 38}.

\end{examples}

\subsection

A Condition is often contained in a Noun, an Adjective, a Participle,
an Adverb, an Ablative Absolute, etc.
\begin{examples}

\latin{nūlla alia gēns nōn obruta esset},
\english{no other race would have failed to be crushed};
\apud{Liv.}{22, 54, 10}.
(If it had been any other race, it would have been crushed.)

\end{examples}

\end{minor}

\headingG{Conditions and Conclusions, in Detail}

\headingC{First Class: Conditions and Conclusions of Fact, in Any Time}

\section

\emph{Conditions and Conclusions of Fact} are expressed by the
Indicative.
\linebreak
They may be in any time, and so in any tense; and the two
parts may also \emph{differ} in tense.
\begin{examples}

\latin{sī occīdī, rēctē fēcī; sed nōn occīdi},
\english{if I killed him, I killed him justly; but I did not kill him};
\apud{Quintil.}{4, 5, 13}.
(Time the same in both.)

\latin{vindicābitis vōs, sī mē potius quam fortūnam meam fovēbātis},
\english{you will avenge me, if it was I, rather than my fortunes,
  that you were courting};
\apud{Tac.\ Ann.}{2, 71}.
(Time differing in the two.)

\latin{quotiēnscumque mē petīstī, per mē tibi obstitī},
\english{as often as your attack has been aimed at me, I have resisted
  you with my own resources};
\apud{Cat.}{1, 5, 11}.
(Generalizing; present perfect tense.\footnote{In order to be
  generalizing, a sentence needs only to be true of \emph{every case
    in a given class}, not necessarily of every case \emph{everywhere}
  and \emph{always}.})

\latin{neque, cum aliquid mandārat, cōnfectum putābat},
\english{nor, when he had given a commission, did he regard it as
  executed};
\apud{Cat.}{3, 7, 16}.
(Generalizing in a tense of the past.)

\latin{beātus est nēmō, quī eā lēge vīvit},
\english{no man is happy who lives on such terms};
\apud{Phil.}{1, 14, 35}.

\latin{nam cum hostium cōpiae nōn longē absunt, etiamsī inruptiō nūlla
  facta est},
\english{for when an enemy’s force is not far off, agriculture is
  abandoned, even if no incursion has been made};
\apud{Pomp.}{6, 15}.

\end{examples}

\begin{note}[Note 1]

In the generalizing clause, the idea of condition (the \emph{assuming}
of something as true) is necessarily always present.  This idea
regularly takes precedence of all other ideas,—whether descriptive, or
causal, or adversative.  The real meaning in the last example but one
is: \emph{\textsc{if any man} lives on such terms, then that man is
  not happy}; in the last example, \emph{\textsc{if} an enemy’s force
  is not far off}.  (Note the parallelism of \latin{cum} and
\latin{etiamsī}.)

\end{note}

\begin{note}[Note 2]

Yet the \emph{habit} of using the Subjunctive after negative or
indefinite antecedents (\xref[1, \emph{b}]{521}) is so strong that the
Romans occasionally did employ it, even in a Generalizing Clause, after
such antecedents.  So especially with \latin{quī quidem} and
\latin{quī modo}.
\begin{examples}

\latin{quī reī pūblicae sit hostis, fēlīx esse nēmō potest},
\english{no man can be happy who is an enemy to the commonwealth};
\apud{Phil.}{2, 26, 64}.
Similarly \latin{quem inrētīssēs}, \apud{Cat.}{1, 6, 13};
\latin{quī modo sit}, \apud{Cat.}{4, 8, 16} (contrast \latin{quī modo
  audīvit}, \apud{Dei.}{6, 16}).

\end{examples}

\end{note}

\subsubsection

\term{The More Vivid Future Condition and Conclusion} is simply one
particular form of the Condition and Conclusion of Fact, in which
\emph{both} are in the \emph{future}, as in the examples following:
\begin{examples}

\latin{sī accelerāre volent, cōnsequentur},
\english{if they \emph{(shall choose to)} will make haste, they will
  overtake him};
\apud{Cat.}{2, 4, 6}.

\latin{quī sibi fīdet, dux reget exāmen},
\english{the man that shall trust himself will lead and rule the
  swarm};
\apud{Ep.}{1, 19, 22}.
(Generalizing in the future.)

\end{examples}

\headingC{Second Class: Less Vivid Future Conditions and Conclusions}

\section

\emph{Less Vivid Future Conditions and Conclusions} are expressed by
the Present or Perfect Subjunctive (really Future and Future Perfect
in meaning).
\begin{examples}

\latin{quibus ego sī mē restitisse dīcam, nimium mihi sūmam},
\english{if I should say that it was I that withstood them, I should
  be claiming too much};
\apud{Cat.}{3, 9, 22}.

\latin{quī dīcat prō illō ‘nē fēcerīs,’ ‘nōn fēcerīs,’ in idem incidat
  vitium},
\english{a man who should say ‘\latin{nōn fēcerīs}’ instead of
  ‘\latin{nē fēcerīs}’ would fall into the same error};
\apud{Quintil.}{1, 5, 50}.
(Generalizing, = \emph{any} man who\dots, \emph{if any} man\dots)

\latin{nihil enim prōficiant, nisi admodum mentiantur},
\english{for if they \emph{(namely, \emph{traders})} should fail to
  lie roundly, they would make nothing};
\apud{Off.}{1, 42, 150}.
(Generalizing.)

\end{examples}

\subsubsection

There are thus (counting in the Present Indicative; \xref{571}) three
ways of expressing a future Condition and Conclusion:
\begin{examples}

\term{Less Vivid}:
\latin{sī veniat, gaudeam},
\english{if he should come, I should be glad}.

\term{More Vivid}:
\latin{sī veniet, gaudēbō},
\english{if he shall come, I shall be glad}.

\term{With the Freer Present}:
\latin{sī venit, gaudēbō},
\english{if he comes, I shall be glad}.

\end{examples}

\subsubsection

\term{Past-Future Condition and Conclusion.} When the point of view is
in the \emph{past}, the tenses of the Subjunctive are of course the
Imperfect and Past Perfect (really Future and Future Perfect to the
past; see \xref{470}).
\begin{examples}

\latin{at tum sī dīcerem, nōn audīrer},
\english{but at that time \emph{(it was certain that)} I should not be
  listened to, if I were to speak};
\apud{Clu.}{29, 80}.
(For the tense-feeling, compare the \emph{N.~Y.\ Evening Post}, June
16, 1891: “But it was now nearly six o’clock, and it \emph{would}
surely \emph{be} dark before we could scale the heights of Demetrias
and return to Volo.”)

\latin{habēbat Tigellius hoc\ellipsis sī conlibuisset, ab ōvō usque ad
  māla citāret ‘Iō Bacche,’}
\english{Tigellius had this habit\dots; if the fancy were to take him,
  he would sing ‘Ho Bacchus’ from soup to pudding};
\apud{Sat.}{1, 3, 3}.
(Generalizing.)

\end{examples}

\begin{note}

No distinction of the degree of vividness can be made in \emph{Past}
Future Conditions and Conclusions, since only the Subjunctive is here
possible (\xref{508}).

\end{note}

\subsubsection

A Past-Future Conclusion may also be expressed by the use of a
\emph{Past Periphrastic Future} form of the Indicative.
\begin{examples}

\latin{quia, sī armentum in spēluncam compulisset, vēstīgia dominum eō
dēductūra erant, bovēs caudīs in spēluncam trāxit},
\english{because, if he should drive the herd into the cave, their
  tracks would \emph{(were going to)} lead their master thither,
  \(Cacus\) dragged them into the cave by their tails};
\apud{Liv.}{1, 7, 5}.

\latin{quem sī tenērent nostrī, pābulātiōne prohibitūrī hostīs
  vidēbantur},
\english{and if our men should hold this hill, it seemed that they
  would keep the enemy from foraging}
(they seemed to be going to keep\dots);
\apud{B.~G.}{7, 36, 5}.

\end{examples}

\headingC{Third Class: Conditions and Conclusions Contrary to Fact,\\
  in the Present or Past}

\section

\emph{Conditions and Conclusions Contrary to Fact} are expressed by
the Imperfect or Past Perfect Subjunctive.

The Imperfect expresses an \emph{act} or \emph{state} in the present
or past (generally in the present), and the Past Perfect a
\emph{completed} act, in the present or past.
\begin{examples}

\latin{servī meī sī mē istō pactō metuerent, domum meam relinquendam
  putārem},
\english{if even my slaves feared me in this fashion, I should think
  that I ought to leave my home};
\apud{Cat.}{1, 7, 17}.
(Present.)

\latin{sī hoc optimum factū iūdicārem, ūnīus ūsūram hōrae gladiātōrī
  istī ad vīvendum nōn dedissem},
\english{if I thought this the best course to take, I should not have
  granted this cutthroat the enjoyment of one hour of life};
\apud{Cat.}{1, 12, 29}.
(\latin{Sī iūdicārem} refers both to the past and to the present.)

\latin{neque diūtius Numidae resistere quīvissent, nī peditēs magnam
  clādem facerent},
\english{nor would the Numidians have been able to hold out any
  longer, had not the infantry effected a great slaughter};
\apud{Sall.\ Iug.}{59, 3}.
(\latin{Facerent} refers to the past.)

\latin{praeterita aetās quamvīs longa cum efflūxisset, nūlla
  cōnsōlātiō permulcēre posset stultam senectūtem},
\english{when the past,—no matter how long,—was over, no consolation
  could comfort a fool’s \emph{(= \emph{any} fool’s)} old age};
\apud{Sen.}{3, 4}.
(Generalizing: “when” = “in any case in which.”)

\end{examples}

\subsubsection

A Conclusion Contrary to Fact may also be expressed by the use of a
\emph{Past Periphrastic Future} form of the Indicative (\latin{-tūrus
  fuī}, \latin{eram}, etc.).
\begin{examples}

\latin{quōs ego, sī tribūnī mē triumphāre prohibērent, testīs
  citātūrus fuī},
\english{whom, in case the tribunes had opposed my celebrating a
  triumph, I should have summoned as witnesses};
\apud{Liv.}{38, 47, 4}.

\end{examples}

\begin{note}[Remark]

This construction has arisen out of the true Past-Future construction
(\emph{was going to\dots, if\ellipsis should}; see \xref[\emph{c}]{580}).

The use of the Imperfect and Past Perfect \emph{Subjunctive} in the
more common construction arose in the same way out of the past-future
force.  Compare \latin{tum sī dīcerem, nōn audīrer} (under
\xref[\emph{b}]{580}), originally meaning \english{if I were at that
  time to speak, I should not be heard}, but easily suggesting the
meaning \english{if I \textsc{had} at that time spoken, I
  \textsc{should} not have been heard}.

\end{note}

\subsubsection

The Periphrastic Future form supplies a means of expression where the
Subjunctive cannot be used, or where a different tense is wanted:
\begin{enumerate}

\item

A \term{Conclusion Contrary to Fact in Indirect Discourse} is
expressed by \latin{fuisse} (very rarely \latin{esse}) \emph{with the
  Future Participle}, active or passive.
\begin{examples}

\latin{Ariovistus respondit: sī quid ipsī ā Caesare opus esset, sēsē
  ad eum ventūrum fuisse},
\english{Ariovistus replied: if he himself wanted anything of Caesar,
  he \emph{(Ariovistus)} would have come to him};
\apud{B.~G.}{1, 34, 2}.
(In Direct Discourse, \latin{sī quid mihi ā Caesare opus esset, ego ad
  eum vēnissem}.)

\end{examples}

\item

A Conclusion Contrary to Fact, where a Subordinate Clause in the
Perfect Subjunctive is desired, is expressed by \latin{fuerim}, etc.,
\emph{with a Future Participle}, active or passive.\footnote{The
  growing fondness for the aorist in result clauses makes this
  construction common in later Latin in Conclusions Contrary to Fact
  (e.g.\ \latin{ut}, \latin{nisi\ellipsis fuisset}, \latin{repetītūrus
    fuerit}, \apud{Liv.}{22, 32, 3}).}
\begin{examples}

\latin{dīc quidnam factūrus fuerīs, sī eō tempore cēnsor fuissēs},
\english{tell me what you would have done, if you had been censor at
  that time};
\apud{Liv.}{9, 33, 7}.
The Past Perfect may be retained; cf.\ the tense in \xref[4,
  \emph{b})]{519}.

\end{examples}

\end{enumerate}

\begin{note}

The tense of the \emph{Condition} Contrary to Fact is never changed
under any circumstances.  See the examples above.

\end{note}

\subsubsection

\term{Highly Improbable Conclusion}.  The Imperfect or Past Perfect is
sometimes used to express a Conclusion which, since the Condition is
contrary to fact, is very \emph{unlikely to be realized}.
\begin{examples}

\latin{quod ego sī verbō adsequī possem, istōs ipsōs ēicerem},
\english{if I had it in my power to accomplish this by a word, I
  should drive out these very men};
\apud{Cat.}{2, 6, 12}.
This the speaker does not mean to do. (Not \emph{I should have driven
  out} nor \emph{I should now be driving out}, but \emph{I should
  proceed to drive out}.)

\end{examples}

\begin{minor}

\subsubsection

\term{Early and Poetic Conditions and Conclusions Contrary to Fact.}
In early Latin, Conditions and Conclusions Contrary to Fact may be
expressed by the \emph{Present} and \emph{Perfect}; and the poets
sometimes employ the construction at a later period.
\begin{examples}

\latin{sī ēcastor nunc habeās quod dēs, alia verba praehibeās: nunc
  quia nihil habēs\dots},
\english{good gracious!  if you had anything to give, your language
  would be different.  As it is, since you haven’t anything\dots};
\apud{As.}{188}.

\latin{dēliciās tuās, nī sint inēlegantēs, vellēs dīcere},
\english{you would wish to tell of your pleasures, if they were not
  discreditable};
\apud{Catull.}{6, 1}.

\end{examples}

\begin{note}[Remark]

This construction is a survival of the earliest type, in use before
the Imperfect and Past Perfect Subjunctive came into existence.  This
earliest type could make no distinction of time.

\end{note}

\subsubsection

The Indicative Past Perfect, or Imperfect, is sometimes used to
represent an act as \emph{sure to have taken place}, except for a
certain condition, expressed or implied.
\begin{examples}

\latin{praeclārē vīcerāmus, nisi Lepidus recēpisset Antōnium},
\english{we had won a splendid victory, had not Lepidus given Antony
  shelter}
(the victory was already won, \emph{but}\dots);
\apud{Fam.}{12, 10, 3}.

\latin{iam tūta tenēbam, nī gēns crūdēlis ferrō invāsisset},
\english{I should surely have laid hold upon safety \emph{(was already
    laying hold)}, had not the cruel race attacked me with the sword};
\apud{Aen.}{6, 358}.

\end{examples}

\end{minor}

\headingC{General Notes on Conditions and Conclusions}

\section
\subsection

\term{Mixed Conditions and Conclusions.}  Any thinkable combination of
types may be employed; or the Conclusion may take the form of a
Command, a Wish, a Statement of Obligation, etc.
\begin{examples}

\latin{quae supplicātiō sī cum cēterīs supplicātiōnibus cōnferātur,
  hoc interest},
\english{if this thanksgiving should be compared with the rest, there
  is this difference};
\apud{Cat.}{3, 6, 15}.
(Condensed for “there is this difference, \emph{as would be found},
if the comparison should be made.”)

\latin{sī dēferantur et arguantur, pūniendī sunt},
\english{if they should be reported and convicted, they are to be
  punished};
\apud{Plin.\ Ep.}{10, 97}.

\latin{vincite, sī vultis},
\english{have your way, if you will};
\apud{B.~G.}{5, 30, 1}.

\latin{sī amābat, adservāret diēs noctīsque},
\english{if he really was in love with her, he should have watched
  over her day and night};
\apud{Rud.}{379}.

\end{examples}

\subsubsection

The mixed form is especially common where the inherent \emph{meaning} of the
main verb suggests the future idea, as with \latin{dēbeō},
\latin{possum}, \latin{studeō}, \latin{volō}, etc.
\begin{examples}

\latin{intrāre, sī possim, castra hostium volō},
\english{I mean, if I should be successful, to enter the camp of the
  enemy}
(= I shall\dots, if\dots);
\apud{Liv.}{2, 12, 5}.

\end{examples}

\subsection

\term{Loosely Attached Conditions.}  A Less Vivid Future Condition may
be \emph{loosely attached} to the main clause.
\begin{examples}

\latin{auscultō, sī quid dīcās},
\english{I am listening, in case you should have anything to say};
\apud{Trin.}{148}.
(Future to the present.)

\latin{hanc sī nostrī trānsīrent, hostēs exspectābant},
\english{the enemy were waiting, in case our men should cross this
  \emph{(swamp)}};
\apud{B.~G.}{2, 9, 1}.
(Future to the past,)

\end{examples}

\begin{minor}

\subsubsection

Such Conditions often suggest the idea “to see whether,” or “in the
hope that.”

\subsubsection

Out of examples like the last arises the true \term{Indirect Question
  of Fact with \latin{sī}}.
\begin{examples}

\latin{vide sī quid opis potes adferre},
\english{see if you can help};
\apud{Ph.}{553}.
(For the mood, see \xref[\emph{g}]{537}.)

\latin{incerta sī Iuppiter velit},
\english{uncertain whether it is the will of Jove};
\apud{Aen.}{4, 110}.

\latin{quaesīsse sī equitēs ēvāsissent},
\english{asked if the calvary had escaped};
\apud{Liv.}{39, 50, 7}.

\end{examples}

\end{minor}

\subsection

\term{Special Idioms} with Verbs or Phrases expressing Obligation,
Possibility, and the like, and certain other Phrases made up of a
neuter Adjective with \latin{est}, or equivalents:\footnote{So
  e.g.\ with \latin{dēbeō}, \latin{decet}, \latin{oportet},
  \latin{covenit}, \latin{possum}, \latin{licet}; \latin{aequum},
  \latin{melius}, \latin{optimum}, \latin{iūstum}, \latin{pār est};
  \latin{longum}, \latin{facile}, \latin{grave est}; the Future
  Passive Participle with \latin{est}; and \latin{est} with the
  Descriptive Genitive.  Similarly, in poetry, with \latin{tempus est},
  etc.}

\begin{enumerate}

\item

An \emph{actually existing} Obligation, Possibility, etc., in whatever
time, is expressed by an Indicative of the appropriate
tense;\footnote{\label{ftn:311:2}In corresponding expressions in
  English we inflect the Infinitive to make variations of tense (“I
  ought \emph{to do} it,” “\emph{to have done} it,” etc.).  The
  Romans inflected the main verb (“it \emph{is} my duty to do it,”
  “it \emph{was} my duty to do it,” etc.).  Thus \latin{id facere
    dēbuī}, \english{I ought to have done it}.

  But of course the Infinitive may be be used in an emphatic tense
  (\xref{490}), or \latin{iam prīdem} may be added (\xref{485}), or
  both, as in \latin{quod iam prīdem factus esse oportuit},
  \english{which ought \textsc{long ago} to have been done \textsc{and
      done with}}; \apud{Cat.}{1, 2, 5}.} an Obligation, Possibility,
etc., which, in some imagined case, \emph{would} exist, or \emph{would
  have} existed, by a Subjunctive of the appropriate tense.

In such uses, the Imperfect Indicative expresses an actually existing
present Obligation or Possibility not fulfilled, the Perfect an actual
past Obligation or Possibility not fulfilled, the Past Perfect an
Obligation or Possibility actually existing in past time, and prior to
a point which is in mind.  The tenses of the Subjunctive, when used
with these expressions, are simply those of the regular Subjunctive
Conclusion (Less Vivid Future, or Contrary to Fact, as the case may
be).  Compare the contrasting forms in the following table:

\medskip

\emph{Examples of Contrasting Uses}:
\begin{sidebyside}

\cc{1}{\textsc{Indicative}}
& \cc{1}{\textsc{Subjunctive}}
\endhead

\latin{possum persequī permulta oblec\-tā\-men\-ta rērum rūsticārum,
  sed\dots},
\english{I might treat of a great many pleasures of farm life; but\dots};
\apud{Sen.}{16, 55}.
(I \textsc{have it} in my power to treat.)
&
\latin{sī scierīs, scīsse tē quis arguere possit?}
\english{supposing you to have known, who could prove that you had
  known?}
\apud{Fin.}{2, 18, 59}.
(Who, in that case, \textsc{would} have it in his power?)
\\

\latin{quibus vōs absentibus cōnsulere dē\-bē\-tis},
\english{for whose interests you ought to consult in their absence};
\apud{Pomp.}{7, 18}.
(It \emph{is} an actual obligation, open to fulfilment.)
&
\latin{haec sī tēcum patria loquātur, nōnne im\-pe\-trā\-re dēbeat?}
\english{if your country should thus speak with you, ought she not to
  prevail?}
\apud{Cat.}{1, 8, 19}.
(Thus English.  The Latin idea is, \emph{Would it not} in that case
\emph{be} an obligation?)
\\

\latin{quōs ferrō trucīdārī oportēbat},
\english{who ought to be butchered with the sword};
\apud{Cat.}{1, 4, 9}.
(It \emph{is} an actual obligation, unfulfilled.)
&
\latin{quae sī dīceret, tamen ignōscī nōn opor\-tē\-ret},
\english{if he said this, still it would not be right to forgive};
\apud{Verr.}{1, 27, 70}.
(It \emph{would} in that case still \emph{be} an obligation.)
\\

\latin{melius fuerat prōmissum patris nōn esse servātum},
\english{it would have been better that the father’s promise should
  not be kept};
\apud{Off.}{3, 25, 94}.
(It actually \emph{was}, before the time thought of, the better
thing.)
&
\latin{nōnne melius multō fuisset quiētam ae\-tā\-tem trādūcere},
\english{would it not have been much better to spend my life in
  quiet?}
\apud{Sen.}{23, 82}.
(It \emph{would have been} better, in the case supposed in the
previous sentence.)
\\

\latin{dēlērī tōtus exercitus potuit, sī fugientēs persecūtī victōrēs
  essent},
\english{the entire army might have been destroyed, if the victors had
  followed up the fugitives};
\apud{Liv.}{32, 12, 6}.
(It \emph{was} possible to destroy them, but it was not done.)
&
\latin{nisi labōre mīlitēs essent dēfessī, omnēs hostium cōpiae dēlērī
  potuissent},
\english{if the soldiers had not been tired out, the entire force of
  the enemy might have been destroyed};
\apud{B.~G.}{7, 88, 6}.
(It \emph{would} in that case \emph{have been} possible to destroy them.)
\end{sidebyside}

\begin{note}[Note 1]

The Indicative may be used, even when accompanied by a Condition
Contrary to Fact, if the Conclusion is true \emph{independently} of
the Condition.
\begin{examples}

\latin{quodsī Rōmae Cn.\ Pompeius prīvātus esset, tamen is erat
  dēligendus},
\english{now if
  Gnae\-us Pompey were in Rome, and a private citizen,
  still he would be the right person to choose};
\apud{Pomp.}{17, 50}.
(Is the right person as it is, or would be even in the supposed case.)

\end{examples}

\end{note}

\begin{note}[Note 2]

Constructions corresponding to the above Indicative types of course
occur in Indirect Discourse also.
\begin{examples}

\latin{sī alicuius iniūriae sibi cōnscius fuisset, nōn fuisse
  difficile cavēre},
(he said that)
\english{if he had been conscious of any wrongdoing, it would have
  been easy to be on his guard};
\apud{B.~G.}{1, 14, 2}.
(He said: \latin{“sī cōnscius fuissem, nōn fuit difficile.”}
Cf.\ \latin{facile fuit quattuor duplicāre},
\english{it would have been easy to double the four};
\apud{Div.}{2, 18, 42}.)

\end{examples}

\end{note}

\begin{note}[Note 3]

The poets occasionally \emph{force} the Indicative construction, using
it as the equivalent of a Conclusion (sometimes even of a Condition)
Contrary to Fact.
\begin{examples}

\looseness=-1
\latin{sī nōn alium iactāret odōrem, laurus erat},
\english{if it did not cast a different perfume, it were \emph{(would
    be)} a laurel tree};
\apud{Georg.}{2, 132}.
Similarly \latin{Castor erās}, \apud{Mart.}{5, 38, 6}.

\end{examples}

\end{note}

\item

With certain adjectives with \latin{est} (or
\latin{sunt}),\footnote{Thus \latin{longum est}, \latin{facile est}.}
the Present Indicative is the \emph{fixed idiom} in Ciceronian Latin,
as against the Present Subjunctive, which is not used.
\begin{examples}

\latin{difficile est hoc dē omnibus cōnfirmāre, sed tamen est certum
  quid respondeam},
\english{it would be difficult to maintain this in the case of all
  \emph{(Latin, “it is difficult”)}, but still it is clear what I am
  to answer};
\apud{Arch.}{7, 15}.

\end{examples}

\end{enumerate}

\subsection

A Condition may itself form a Conclusion for another Condition.
\begin{examples}

\latin{moriar sī magis gaudeam, sī id mihi accidisset},
\english{may I die if I should take more pleasure if it had happened
  to myself};
\apud{Att.}{8, 6, 3}.

\end{examples}

\subsection

A Condition with \latin{sī} or \latin{ō sī} may express a
\term{Virtual Wish}.\footnote{That is, a wish in \emph{force}, though
  not in \emph{form}.}
\begin{examples}

\latin{sī nunc sē ille aureus rāmus ostendat},
\english{if now that golden branch would show itself}
(= would that\dots);
\apud{Aen.}{6, 187}.

\end{examples}

\subsection

A Condition with \latin{sī modo}, \english{if only}, is equivalent to
a Proviso (\xref{529}).  Either mood may be used, according to the
feeling.
\begin{examples}

\latin{opprimī dīcō patientiā, sī modo est aliqua patientia},
\english{I assert that \(pain\) is overcome by endurance, if only
  there is some endurance};
\apud{Tusc.}{2, 14, 33}.

\end{examples}

\subsection

\latin{Sī} is sometimes used with the force of \latin{etsī},
\english{even if} (concessive).
\begin{examples}

\latin{nōn possum, sī cupiam},
\english{I cannot, even if I should desire};
\apud{Verr.}{4, 40, \emend{46}{88}{87}}.

\end{examples}

\subsection

\latin{Etsī}, \latin{tametsī}, and \latin{etiamsī}, \english{even if},
are often equivalent to \emph{although} (\term{Vir\-tu\-al Adversative
  Clause}).  Either mood may be used, according to the feeling.
\begin{examples}

\latin{etsī nōndum eōrum cōnsilium cognōverat, tamen suspicābātur},
\english{though \emph{(even if)} he did not yet know their plan, still
  he was suspicious};
\apud{B.~G.}{4, 31, 1}.

\end{examples}

\subsection

\latin{Sī quidem},\footnote{Also written \latin{sīquidem}.  (In later
  poetry, sometimes \latin{sĭquidem}.)} \english{if indeed}, gains the
force of \emph{for} or \emph{since} (\term{Virtual Clause of Reason}).
\begin{examples}

\latin{in agrīs erant tum senātōrēs, sī quidem arantī L.\ Quīnctiō
  Cincinnātō nūntiātum est eum dictātōrem esse factum},
\english{there were senators living in the country at that time; for
  \emph{(if indeed)} the news that he had been appointed dictator was
  brought to Lucius Quinctius Cincinnatus while ploughing};
\apud{Sen.}{15, 56}.

\end{examples}

\subsection

A Definition may be expressed by an Indicative Clause with \latin{quī}
or \latin{cum} (originally simply a generalizing clause; see
\xref{576}–\xref{579}).
\begin{examples}

\latin{vir bonus est is quī prōdest quibus potest, nocet nēminī},
\english{the good man is the one who helps whom he can, and harms nobody};
\apud{Off.}{3, 15, 64}.

\latin{is est triumphus vērus, cum bene dē rē pūblica meritīs
  testimōnium ā cōnsēnsū cīvitātis datur},
\english{that is the true triumph, when those who have deserved well
  of the state receive evidence of this from the unanimous feeling of
  its citizens};
\apud{Phil.}{14, 5, 13}.

\end{examples}

\chapter{The Infinitive}

\contentsentry{B}{Uses of the Infinitive}

\section
\subtitle{\textsc{Synopsis of the Principal Uses of the Infinitive}}

\medskip

\begin{enumI}[VIII]

\item
With Adjectives with \latin{est}, and Verbs or Phrases of similar
force, as in “it is base to\dots” (\xref{585}).

\item
With Verbs or Phrases expressing attitude or position with reference
to performing an act, as in “I wish to” (\xref{586}).

\item
With Verbs or Phrases expressing attitude or position toward the
performing of an act by another, as in “I wish \emph{you} to”
(\xref{587}).

\item
With Verbs or Phrases of perceiving, saying, thinking, or knowing, as
in “I see that you\dots” (\xref{589}).

\item
With Verbs or Phrases of feeling, as in “I am glad that you\dots”
(\xref{594}).

\item
Historical Infinitive (\xref{595}).

\item
Exclamatory Infinitive (\xref{596}).

\item
As Subject, Predicate, or Object of certain Verbs, or as an Appositive
(\xref[1]{597}).

\end{enumI}

\section

The Infinitive is in effect a Verbal Noun, capable of standing in
various case-relations.

As a Noun, it may have a Neuter Adjective or Pronoun agreeing with it
(\xref[3]{58}; example under \xref[1, \emph{b}]{597}).

As a Verb, it may govern Cases, and may itself be modified by an
Adverb.

\subsubsection

The negative is \latin{nōn}.

\subsubsection

For the general forces of the tenses, see~\xref{472}.

\subsubsection

According to the sense intended, the Infinitive may be Active or
Passive; it may, or may not, be attended by a Subject
Accusative;\footnote{A classification of the Infinitive on the basis
  of its having or not having a Subject Accusative is unserviceable,
  since many verbs may take either construction \emph{without
    essential difference of meaning}.  Thus one may say either
  \latin{cupiō clēmēns esse} or \latin{cupiō mē esse clēmentem}.} and,
if Passive, it may, or may not, be attended by a Predicate Noun or
Adjective.

\subsubsection

In most of its uses, the Infinitive stands to the verb or phrase on
which it depends in the Relation of Subject, or Object, or Accusative
of Respect.  In such examples, it is of Substantive nature
(cf.~\xref{238}).

\subsubsection

In the Future Active and the Perfect Passive Indicative, the auxiliary
\latin{esse} is often omitted (\xref[7]{164}).

\headingF{Prose Uses of the Infinitive in All Periods}

\headingC{I}

\section

The Infinitive is used with \emph{Adjectives with \latin{est}}, and
  Verbs and Phrases of similar force.\footnote{Thus \latin{nefās est},
    \english{it is wrong}, has the same force as \latin{nefāstum est};
    \latin{mōs est}, \english{it is customary}, as \latin{ūsitātum
      est}; \latin{tempus est}, \english{it is time}, as
    \latin{tempestīvum est}.}

These expressions represent an action as (1)~\emph{advantageous} or
\emph{important}; (2)~\emph{necessary} or \emph{obligatory};
(3)~\emph{customary} or \emph{permissible}; (4)~\emph{seemly} or
  \emph{shameful}, \emph{pleasant} or \emph{tiresome}, \emph{easy} or
  \emph{difficult},\footnote{E.g.\ (1)~\latin{iuvat}, \latin{expedit},
    \latin{ūtile est}, \latin{condūcit}, \latin{prōdest},
    \latin{rēfert}, \latin{interest}; (2)~\latin{necesse} or
    \latin{ne\-ces\-sā\-ri\-um est}, \latin{opus} or \latin{ūsus est},
    \latin{tempus est} (\english{it is seasonable to}, \english{it is
      high time that}), \latin{oportet}, \latin{convenit}, \latin{iūs}
    or \latin{iūstum est}, \latin{fās}, \latin{nefās}, or
    \latin{nefārium est}, \latin{pār}, \latin{rēctum}, \latin{aequum},
    \latin{inīquum}, etc., \latin{est}; (3)~\latin{mōs} (\latin{mōris})
    or \latin{cōnsuētūdō} (\latin{cōnsuētūdinis}) or \latin{ūsitātum
      est}, \latin{meum} (\latin{tuum}, etc.) \latin{est},
      \latin{licet}; (4)~\latin{decet} or \latin{dēdecet},
      \latin{convenit}, \latin{laus est}, \latin{turpe} or
      \latin{praeclārum est}, \latin{scelus} or \latin{facinus est},
      \latin{displicet} (\english{is disagreeable}), \latin{dēlectat},
      \latin{taedet}, \latin{paenitet}, \latin{pudet}, \latin{piget},
      \latin{rēfert}, \latin{interest}, \latin{iūcundum},
      \latin{grātum}, \latin{grave}, \latin{molestum},
      \latin{miserum}, \latin{longum est}, \latin{facile} or
      \latin{difficile est}, \latin{satis} or \latin{satius est},
      \latin{optābile}, \latin{bonum} or \latin{malum est},
      \latin{vidētur} (\english{seems best}), \latin{praestat}
      (\english{is better}), \latin{est} or \latin{rēs est} with the
      Genitive (\english{is the part of}), \latin{proprium est}
      (\english{is peculiar to}), etc.

    Similarly other words in later Latin. Thus \latin{vincit}
    (\english{is better}).} etc., etc.
\begin{examples}

\latin{commodissimum vīsum est mittere},
\english{it seemed most advantageous to send};
\apud{B.~G.}{1, 47, 4}.

\latin{tempus est abīre mē},
\english{it is time that I should go}
(to go is seasonable);
\apud{Tusc.}{1, 41, 99}.

\end{examples}

\subsubsection

In many phrases, this Infinitive may either have, or not have, a
Subject Accusative; thus \latin{tempus est abīre} or \latin{tempus est
  nōs abīre}.

\subsubsection

When the Subject of the Infinitive is indefinite (\emph{one}, \emph{a
  man}, \emph{people}), it is not expressed.  But a Predicate Noun or
Adjective may nevertheless be used, \emph{belonging in thought} to the
indefinite Subject.
\begin{examples}

\latin{nōn esse cupidum pecūnia est},
\english{not to be covetous is wealth};
\apud{Par.}{6, 3, 51}.

\end{examples}

\subsubsection

When \latin{licet}, \latin{expedit}, etc., are followed by a Dative
and Infinitive, the Predicate of the Infinitive may be in the
Accusative, or it may be attracted into the Dative.
\begin{examples}

\latin{cīvī Rōmānō licet esse Gāditānum},
\english{it is permitted to a Roman citizen to be a citizen of Cadiz};
\apud{Balb.}{12, 20}.

\latin{mihi neglegentī esse nōn licet},
\english{I am not allowed to be careless};
\apud{Att.}{1, 17, 6}.

\end{examples}

\subsubsection

Such an Infinitive sometimes has a Neuter Adjective or Pronoun in
agreement (cf.~\xref[3]{58}).
\begin{examples}

\latin{cum vīvere ipsum turpe sit},
\english{when merely to be alive is disgraceful};
\apud{Att.}{13, 28, 2}.

\end{examples}

\begin{minor}

\subsubsection

Verbs or phrases of this class suggesting that the action is
\emph{wanted} or \emph{urged} may also take a
%
\versionA{Volitive Substantive Clause (\xref[3, \emph{c}]{502}, and
  lists). }%
%
\versionB*{Subjunctive Substantive Clause. }%
%
So especially \latin{interest}, \latin{rēfert}, \latin{oportet},
\latin{licet}, \latin{necesse}, \latin{opus}, \latin{ūsus}, or
\latin{tempus est}, \latin{melius} or \latin{optimum est}.  Thus one
may say either \latin{tempus est nōs abīre} or \latin{tempus est ut
  abeāmus}.

\subsubsection

Many verbs and phrases of this class \emph{shade into} meanings
belonging to the second or the third class.  So \latin{tempus est
  abīre} \emph{suggests} the meaning \emph{I am inclined to go}.

\end{minor}

\headingC{II}

\section

The Infinitive is used with Verbs or Phrases expressing
\emph{attitude} or \emph{position with reference to performing an
  act}.

The personal verbs of this class express the ideas of
(1)~\emph{wishing} or \emph{not wishing}; (2)~\emph{determining},
\emph{planning}, or \emph{endeavoring}; (3)~\emph{beginning} or
\emph{omitting}, \emph{persevering} or \emph{ceasing};
(4)~\emph{hastening} or \emph{delaying}; (5)~\emph{daring},
\emph{hesitating}, \emph{fearing}; (6)~\emph{knowing how} or
\emph{learning how}; (7)~\emph{remembering to} or \emph{seeming to};
(8)~\emph{being accustomed to}, \emph{having the power to}, or
\emph{being under obligation to}.\footnote{E.g.\ (1)~\latin{volō},
  \latin{mālō}, \latin{nōlō}, \latin{cupiō}, \latin{optō},
  \latin{dēsīderō}, \latin{sustineō}, \latin{recūsō};
  (2)~\latin{statuō}, \latin{cōnstituō}, \latin{īnstituō},
  \latin{dēcernō}, \latin{animum indūcō}, \latin{cōnsilium capiō},
  \latin{cōgitō}, \latin{meditor}, \latin{studeō}, \latin{in animō
    habeō}, \latin{dēstinō}, \latin{parō}, \latin{cōnor},
  \latin{nītor}, \latin{mōlior}, \latin{labōrō}, \latin{temptō};
  (3)~\latin{coepī}, \latin{incipiō}, \latin{mittō} and
  \latin{omittō}, \latin{neglegō}, \latin{pergō}, \latin{persevērō},
  \latin{īnstō}, \latin{dēsinō}, \latin{dēsistō}, \latin{cessō};
  (4)~\latin{festīnō}, \latin{properō}, \latin{mātūrō},
  \latin{contendō}, \latin{moror}, \latin{cūnctor}; (5)~\latin{audeō},
  \latin{dubitō}, \latin{vereor}, \latin{metuō}, \latin{timeō};
  (6)~\latin{sciō}, \latin{nesciō}, \latin{discō}; (7)~\latin{meminī},
  \latin{recordor}, \latin{oblīvīscor}, \latin{videor}
  (\english{seem}); (8)~\latin{soleō}, \latin{adsuēscō},
  \latin{cōnsuēscō}, \latin{possum}, \latin{queō}, \latin{nequeō},
  \latin{dēbeō}.

  Also, in poetic and later Latin (1)~\latin{ārdeō}, \latin{dignor},
  \latin{gaudeō}, \latin{laetor}; (2)~\latin{poscō}; (3)~\latin{sūmō};
  (4)~\latin{praecipitō}; (5)~\latin{horreō}; (8)~\latin{sufficiō},
  \latin{valeō}, etc.}

The impersonal verbs or phrases express \emph{determination},
\emph{inclination}, or \emph{whim}.\footnote{E.g.\ \latin{certum},
  \latin{dēstinātum}, \latin{cōnsilium} or \latin{in animō est},
  \latin{venit in mentem}, \latin{placet}, \latin{iuvat},
  \latin{libet}.

  Also, in poetic and later Latin, \latin{cūra} or \latin{cūrae est},
  \latin{est animus}, \latin{fert animus}, \latin{amor} or
  \latin{cupīdō est}, \latin{subit īra}, \latin{mēns est}, \latin{spēs
    est accēnsa}, etc.}
\begin{examples}

\latin{maiōrī partī placuit castra dēfendere},
\english{the majority wanted to defend the camp}
(to defend it was pleasing to them);
\apud{B.~G.}{3, 3, 4}.

\latin{ad hunc lēgātōs mittī placet?}
\english{do we want ambassadors to be sent to such a man as this?}
(= \latin{mittere placet?})
\apud{Phil.}{5, 9, 25}.

\latin{eās nātiōnēs adīre volēbat},
\english{he wished to visit those tribes};
\apud{B.~G.}{3, 7, 1}.

\latin{nōlīte dubitāre},
\english{pray, do not hesitate}
(be unwilling to\dots);
\apud{Pomp.}{23, 68}.
(Courteous Prohibition.  See~\xref[\emph{a}, 2]{501}.)

\latin{mātūrat proficīscī},
\english{he makes haste to set out};
\apud{B.~G.}{1, 7, 1}.

\latin{dēbēre sē suspicārī},
\english{he was bound \emph{\(he said\)} to suspect};
\apud{B.~G.}{1, 44, 10}.

\end{examples}

\subsubsection

With most of the personal verbs of this class, the Infinitive
\emph{completely fills out} the meaning (as in \latin{volō īre},
\english{I wish to go}).  Hence it is called the Complementary
Infinitive.

\subsubsection

Some of these verbs may either have, or not have, the Reflexive
Pronoun as Subject Accusative.

If such a Subject Accusative is used, a Predicate Noun or Adjective
must of course be in the Accusative; if not, it must go back to the
Subject of the main verb for its agreement (generally therefore in the
Nominative).
\begin{examples}

\latin{grātum sē vidērī studet},
\english{aims to seem grateful};
\apud{Off.}{2, 20, 70}.

\latin{fierī studēbam doctior},
\english{I aimed to become wiser};
\apud{Am.}{1, 1}.

\end{examples}

\subsubsection

Impersonal verbs or phrases of this class suggesting that the action is
\emph{wanted} or \emph{urged} may also take a Volitive Substantive
Clause (\xref[3, \emph{a}]{502}).  Thus one may say either
\latin{placuit eī lēgātōs mittere} or \latin{placuit eī ut lēgātōs
  mitteret} (\apud{B.~G.}{1, 34, 1}).

\subsubsection

Most verbs of \emph{wishing} or \emph{not wishing}, when used to
express attitude toward performing an act \emph{oneself}, take only
the Infinitive (thus \latin{volō}).  But \latin{recūsō} may also take
the Subjunctive with \latin{nē} or \latin{quōminus}, or, if negatived,
with \latin{quōminus} or \latin{quīn}; see \xref[3, \emph{b})]{502}.
(In Cicero’s time only the \emph{negative} form \latin{nōn recūsō},
etc., takes the Infinitive.)

\subsubsection

Several verbs of \emph{determining}, \emph{planning}, or
\emph{endeavoring} take either the Infinitive or the Volitive
Subjunctive (\xref[3, \emph{a}]{502}).  So \latin{cōnstituō},
\latin{labōrō}.

\subsubsection

The Participle \latin{parātus} may take an Infinitive (thus in
\apud{B.~G.}{1, 44, 4}) just as any other part of \latin{parō} may do.
Later, the Participles of \latin{suēscō}, \latin{adsuēscō},
\latin{adsuēfaciō}, and \latin{soleō} (\latin{suētus},
\latin{adsuētus}, \latin{adsuēfactus}, \latin{solitus}) came to be
used similarly with the Infinitive.  For the large extension of this
usage, see~\xref[2, \emph{c})]{598}.

\subsubsection

Several verbs belong both to this class and to the following one;
e.g.\ \latin{placet}, \latin{volō}, \latin{cupiō}, \latin{optō},
\latin{studeō} (thus “I wish to do a thing,” and “I wish \emph{you}
to do a thing”).

\headingC{III}

\section

The Infinitive is used with certain Verbs expressing \emph{attitude}
or \emph{position toward the performing of an act by another}.

Verbs of this class express the ideas of (1)~\emph{wishing} or
\emph{not wishing}; (2)~\emph{commanding} or \emph{impelling};
(3)~\emph{permitting}, \emph{prohibiting}, or \emph{preventing};
(4)~\emph{teaching} or
\emph{accustoming}.\footnote{\label{ftn:s587:1}E.g.\ (1)~\latin{volō},
  \latin{mālō}, \latin{nōlō}, \latin{cupiō}, \latin{optō},
  \latin{dēsīderō}, \latin{studeō}, \latin{placet}; (2)~\latin{iubeō},
  \latin{cōgō}; (3)~\latin{patior}, \latin{permittō} (oftener with
  \latin{ut}-clause), \latin{sinō}, \latin{prohibeō}, \latin{impediō};
  (4)~\latin{doceō}, \latin{adsuēfaciō}.

  Other verbs are also so used by Cicero or Caesar, but rarely, though
  freely by the poets; thus \latin{expetō}, \latin{moneō},
  \latin{admoneō}, \latin{hortor}, \latin{faciō} (\english{cause} or
  \english{force}), \latin{suādeō}, \latin{dēterreō} (in passive),
  \latin{ēdoceō}.

  Others are so used only in poetry and later prose;
  e.g.\ (2)~\latin{stimulō}, \latin{poscō}, \latin{tendō},
  \latin{foveō}\emend{47}{}{,} \latin{invītō}, \latin{impellō},
  \latin{suādeō}; (3)~\latin{patior}; (4)~\latin{mōnstrō}
  (\english{show how}), \latin{ērudiō}.}
\begin{examples}

\latin{iter patefierī volēbat},
\english{he wished the road to be opened};
\apud{B.~G.}{3, 1, \emend{233}{3}{2}}.

\latin{Pompeius rem ad arma dēdūcī studēbat},
\english{Pompey’s aim was that the matter should be brought to the
  settlement of arms};
\apud{B.~C.}{1, 4, \emend{234}{4}{5}}.

\latin{Dīviciācum vocārī iubet},
\english{he orders Diviciacus to be summoned};
\apud{B.~G.}{1, 19, 3}.

\latin{sī hic ōrdō placēre dēcrēverit tē īre in exsilium},
\english{if this body should decide it to be its pleasure that you
  should go into exile};
\apud{Cat.}{1, 8, 20}.

\end{examples}

\begin{minor}

\subsubsection

Since verbs used with this meaning imply that something is
\emph{wanted} or \emph{desired}, many may also take a Volitive or
Optative Substantive Clause (\xref[3]{502} and \xref[2]{511}).

\subsubsection

\latin{Imperō}, \english{command}, regularly takes a Volitive
Substantive Clause; but in a few places (as \apud{Cat.}{1, 11, 27};
\apud{B.~G.}{7, 60, 3}) it takes an Infinitive of passive form (either
true passive or deponent).

\latin{Iubeō}, \english{order}, and \latin{vetō}, \english{forbid},
regularly take the Infinitive; but in a few places (as \apud{Verr.}{2,
  67, 16}) they take a Volitive Substantive Clause.

\subsubsection

Many other verbs, of the same general force as those of Class~3, take
only the Volitive Substantive Clause (\xref[3]{502}).

\end{minor}

\section

The Infinitive may also be used with the \emph{Passive} of many verbs
of this class, e.g.\ with \latin{iubeor}, \latin{prohibeor},
\latin{vetor}.
\begin{examples}

\latin{arma trādere iussī},
\english{being ordered to give up their arms};
\apud{B.~G.}{3, \emend{235}{21}{19}, 3}.

\end{examples}

\headingC{IV}

\section

The Infinitive is used to express a \emph{Statement} after Verbs or
Phrases of \emph{perceiving}, \emph{saying}, \emph{thinking},
\emph{knowing}, and the like.

\begin{minor}

These express or imply the ideas of (1)~\emph{seeing}, \emph{feeling},
or \emph{hearing}; (2)~\emph{saying}, \emph{proving},
\emph{conceding}, or \emph{denying}; (3)~\emph{accusing} or
\emph{acquitting}; (4)~\emph{thinking}, \emph{believing},
\emph{suspecting}, or \emph{doubting}; (5)~\emph{remembering} or
\emph{knowing}; (6)~\emph{learning} or \emph{informing};
(7)~\emph{confessing} or \emph{pretending}; (8) \emph{swearing},
\emph{threatening}, \emph{hoping}, or
\emph{promising}.\footnote{\label{ftn:318:}E.g.\ (1)~\latin{videō},
  \latin{sentiō}, \latin{audiō}, \latin{manifestum est}, \latin{nōn mē
    fallit}; (2)~\latin{dīcō}, \latin{dēclārō}, \latin{nārrō},
  \latin{adfirmō}, \latin{fāma est}, \latin{dēmōnstrō}, \latin{probō},
  \latin{vērum} or \latin{falsum est}, \latin{cōnstat},
  \latin{concēdō}, \latin{negō}, \latin{convenit}, \english{it is
    agreed that}, \latin{sequitur}, \latin{efficitur}, \english{it is
    made out that}; (3)~\latin{arguō}, \latin{incūsō},
  \latin{īnsimulō}, \latin{dēfendō}; (4)~\latin{putō},
  \latin{arbitror}, \latin{opīnor}, \latin{statuō} and
  \latin{cōnstituō} (with Infinitive and Future Passive Participle),
  \latin{cēnseō}, \latin{exīstimō}, \latin{iūdicō}, \latin{crēdō},
  \latin{dūcō}, \latin{fīdō}, \latin{diffīdō}, \latin{suspicor},
  \latin{habeō} (in the sense of \english{understand}),
  \latin{dubitō}, \latin{mīrum est}, \latin{vērī simile est};
  (5)~\latin{recordor}, \latin{meminī}, etc., \latin{memoriā teneō},
  \latin{intellegō}, \latin{sciō}, \latin{nesciō}, \latin{ignōrō};
  (6)~\latin{discō}, \latin{inveniō}, \latin{cognōscō},
  \latin{ignōrō}, \latin{reperiō}, \latin{certior fīō},
  \latin{certiōrem faciō}, \latin{nūntiō} and its compounds,
  \latin{moneō} (\english{inform that}), \latin{suādeō} and
  \latin{persuādeō} (\english{persuade that}); (7)~\latin{fateor},
  \latin{cōnfiteor}, \latin{fingō}, \latin{simulō}; (8)~\latin{iūrō},
  \latin{minor}, \latin{spērō}, \latin{spem habeō}, etc.,
  \latin{cōnfīdō}, \latin{cōnfirmō}, \latin{polliceor},
  \latin{prōmittō}.

  Other verbs are found in poetry and later Latin, as
  \latin{prōspiciō}, \latin{repetō}, \latin{mōnstrō}.}

\begin{examples}

\latin{biennium satis esse dūxērunt},
\english{thought two years to be enough};
\apud{B.~G.}{1, 3, 2}.

\latin{Caesar sēsē eōs cōnservātūrum (esse) dīxit},
\english{Caesar said that he would leave them unharmed};
\apud{B.~G.}{2, \emend{236}{15}{14}, 1}.

\latin{memoriā tenēbat L.\ Cassium occīsum (esse) ab Helvētiīs},
\english{he remembered that Lucius Cassius had been slain by the
  Helvetians};
\apud{B.~G.}{1, 7, \emend{237}{4}{3}}.

\latin{quis ignōrābat Q.\ Pompeium fēcisse foedus?}
\english{who was ignorant that Quintus Pompey had made the treaty?}
\apud{Rep.}{3, 18, 28}.

\end{examples}

\subsubsection

A number of verbs or phrases may take the Infinitive, if the idea of
\emph{saying} or \emph{thinking} is implied, or the Subjunctive, if
the idea of \emph{resolving} or \emph{directing} is implied
(\xref[3]{502}); and the two constructions may even be used together.
Thus:
\begin{examples}

\latin{cōnstituērunt optimum esse domum suam quemque revertī, et\ellipsis
  undique convenīrent},
\english{determined that it was best that all should return to their
  homes, and\ellipsis should assemble from all sides};
\apud{B.~G.}{2, 10, 4}.

\end{examples}

\end{minor}

\section
\subsection

The Infinitive may also be used with the \emph{Passive} of many verbs of this
class, e.g.\ with \latin{arguor}, \latin{dīcor}, \latin{exīstimor},
\latin{iūdicor}, \latin{putor}, \latin{videor}.
\begin{examples}

\latin{centum pāgōs habēre dīcuntur},
\english{are said to have a hundred cantons};
\apud{B.~G.}{4, 1, 4}.

\end{examples}

\begin{minor}

\subsubsection

Passive forms compounded with a Participle are generally in the
impersonal construction.  Similarly \latin{crēditur}, \english{it is
  believed}.  But \latin{videor} is preferred to \latin{vidētur}.
\begin{examples}

\latin{cui Āpuliam attribūtam esse erat indicātum},
\english{to whom it had been shown that Apulia had been assigned};
\apud{Cat.}{3, 6, 14}.

\end{examples}

\end{minor}

\subsection

When the main verb is personal, all predicate forms must of course be
in the Nominative, if the Subject is.
\begin{examples}

\latin{nōn minōrem laudem exercitus meritus (esse) vidēbātur},
\english{the army seemed to have earned no less praise};
\apud{B.~G.}{1, 40, 5}.

\end{examples}

\section

Such Statements, because made indirectly (see \xref{533},
\xref[1]{534}), are said to be in Indirect Discourse.  \emph{Every
  Principal Statement in Indirect Discourse is expressed by the
  Infinitive}.

\begin{minor}

\subsubsection

A Rhetorical Question of Fact (\xref{235}), since it is
\emph{equivalent to a Statement} of Fact, is expressed in Indirect
Discourse by an Infinitive.
\begin{examples}

\latin{num etiam recentium iniūriārum memoriam dēpōnere posse?}
\english{could he \emph{\(he asked\)} put aside the memory of recent
  wrongs also?}
\apud{B.~G.}{1, 14, 3}.
(The original \latin{num possum?} \english{can I?} really meant
\latin{nōn possum}, \english{I cannot}.)

\end{examples}

\begin{note}

This usage is confined to questions which originally were in the first
or third person.

\end{note}

\subsubsection

For the Conclusion Contrary to Fact in Indirect Discourse, see
\xref[\emph{b}, 1)]{581}.

\subsubsection

For the occasional Infinitive in a subordinate Indirect Statement, see
\xref[1, \emph{b}]{535}.

\subsubsection

For the Infinitive after a Relative or \latin{quam}, see \xref[1,
  \emph{c}]{535}.

\subsubsection

For the Infinitive (instead of a Participle) with verbs of seeing or
representing, see~\xref[1]{605}.

\end{minor}

\section

The Infinitive in Indirect Discourse regularly has a Subject; but this
is sometimes omitted, especially if it is a Reflexive Pronoun.  The
omission of \latin{is} is rare.
\begin{examples}

\latin{ignōscere imprūdentiae dīxit},
\english{said that he forgave their indiscretion};
\apud{B.~G.}{4, 27, 5}.

\end{examples}

\begin{minor}

\subsubsection

When the Subject is thus omitted, the poets sometimes make a Predicate
Adjective or Participle agree with the Subject of the main verb.
\begin{examples}

\latin{sēnsit mediōs dēlāpsus in hostīs},
\english{saw that he had fallen into the midst of the enemy};
\apud{Aen.}{2, 377}.

\end{examples}

\end{minor}

\section[Tenses]

The tenses in Indirect Discourse have their regular meanings, as
explained in~\xref{472}, the Perfect Infinitive representing a
\emph{relatively past} time, the Present a \emph{relatively present}
time, the Future a \emph{relatively future} time.  For examples, see
\xref[\emph{a} and \emph{b}]{472}.

\subsubsection

Verbs or phrases of \emph{promising}, \emph{hoping}, \emph{swearing},
or \emph{threatening} look forward to the future, and therefore
generally take the Future Infinitive or \latin{posse}, with a Subject
Accusative.  Yet they sometimes take the Present Infinitive, without a
Subject (as generally in English).
\begin{examples}

\latin{spērat adulēscēns diū sē vīctūrum},
\english{the young man hopes to live a long life}
(hopes that he will live\dots);
\apud{Sen.}{19, 68}.

\latin{tōtīus Galliae sēsē potīrī posse spērant},
\english{they hope to be able to master the whole of Gaul};
\apud{B.~G.}{1, 3, \emend{94}{8}{7}}.

\latin{lēgātī veniunt quī polliceantur obsidēs dare},
\english{ambassadors come, to promise to give hostages};
\apud{B.~G.}{4, 21, 5}.

\end{examples}

\subsubsection

Verbs of \emph{remembering} may take the Present Infinitive of a
personal experience (mere act, without tense-force).
\begin{examples}

\latin{meministīne mē dīcere\dots?}
\english{do you remember my saying\dots?}
\apud{Cat.}{1, 3, 7}.

\end{examples}

\headingC{V}

\section

The Infinitive is used with Verbs or Phrases of
\emph{feeling}.\footnote{Such statements are often said to be in
  Indirect Discourse.}

These convey the ideas of (1)~\emph{pride} or \emph{wonder};
(2)~\emph{joy} or \emph{grief}; (3)~\emph{indignation},
\emph{complaint}, or
\emph{resignation}.\footnote{E.g.\ (1)~\latin{glōrior}, \latin{mīror},
  \latin{admīror}, \latin{dēmīror}; (2)~\latin{laetor},
  \latin{gaudeō}, \latin{doleō}, \latin{lūgeō}, \latin{maereō};
  \latin{acerbē}, \latin{graviter}, \latin{molestē}, etc., with
  \latin{ferō}; (3)~\latin{indignor}, \latin{expostulō},
  \latin{fremō}, \latin{queror}, \latin{facile patior}.

  Also, in poetry and later prose, (1)~\latin{laudor} (\english{be
    praised for}), (2)~\latin{gemō}, \latin{dēlector};
  (3)~\latin{tolerō} (\english{put up with}), etc.}
\begin{examples}

\latin{mīrābar crēdī},
\english{I was surprised that it was believed};
\apud{Mil.}{24, 65}.

\latin{exercitum hiemāre in Galliā molestē ferēbant},
\english{took it ill that the army was wintering in Gaul};
\apud{B.~G.}{2, 1, 3}.

\end{examples}

\begin{minor}

\subsubsection

The poets and later prose writers apply the construction also to
\emph{adjectives} of feeling, e.g.\ \latin{laetus}, \latin{maestus},
\latin{contentus}.

\subsubsection

With most of these verbs and phrases the Infinitive is in origin an
Accusative of Respect (e.g. with \latin{doleō}, \english{mourn with
  reference to the fact that}; cf.\ \latin{id maesta est},
\xref[\emph{a}]{388}).  With others, it is a direct Object or Subject
(e.g.\ it is an Object with \latin{molestē ferō}).

\subsubsection

Most of these verbs and phrases may also take a Substantive
\latin{quod}-Clause (\xref{555}).

\end{minor}

\headingC{VI.\enskip Historical Infinitive}

\section

In lively narration, the Infinitive may be used in place of an
\emph{Indicative}, \emph{Perfect}\footnote{With aoristic force.} or
\emph{Imperfect}.  Its Subject is in the Nominative.
\begin{examples}

\latin{hostēs ex omnibus partibus signō datō dēcurrere},
\english{at a given signal the enemy rushed down from every side};
\apud{B.~G.}{3, 4, 1}.
(Aoristic.)

\latin{interim cotīdiē Caesar Haeduōs frūmentum flāgitāre},
\english{meanwhile Caesar was dunning the Haedui daily for the corn};
\apud{B.~G.}{1, 16, 1}.
(Situation.)

\latin{sōlam nam perfidus ille tē colere},
\english{for the traitor used to care for you alone};
\apud{Aen.}{4, 421}.
(Habitual Action.)

\end{examples}

\headingC{VII.\enskip Exclamatory Infinitive}

\section

The Infinitive, generally with a Subject Accusative, may be used in
Exclamations of \emph{surprise}, \emph{indignation}, or \emph{regret}.
The particle \enclitic{-ne} is sometimes attached to the emphatic
word.
\begin{examples}

\latin{hoc nōn vidēre!}
\english{the idea of not seeing this!}
\apud{Fin.}{4, 27, 76}.

\latin{mēne inceptō dēsistere!}
\english{I to give up my purpose!}
\apud{Aen.}{1, 37}.

\end{examples}

\headingC{VIII.\enskip As Subject, Predicate, or Object, of Certain Verbs}

\section
\subsection

The Infinitive is also used, in all periods,
\begin{enuma}

\item

As the Subject, Predicate, or Object of Verbs meaning (1)~to \emph{be},
(2)~to \emph{befall}, or (3)~to \emph{place}.\footnote{E.g.\ (1)~\latin{est};
  (2)~\latin{cadit}, \latin{accidit}, \latin{contingit};
  (3)~\latin{pōnō}, \latin{positum} or \latin{situm est}.}
\begin{examples}

\latin{vīvere est cōgitāre},
\english{to live is to think};
\apud{Tusc.}{5, 38, 111}.
(= a definition.)

\latin{nōn cadit invidēre in sapientem},
\english{it does not happen to the wise man to feel envy};
\apud{Tusc.}{3, 10, 21}.

\latin{beātē vīvere vōs in voluptāte pōnitis},
\english{you base the happy life on pleasure};
\apud{Fin.}{2, 27, 86}.

\end{examples}

\item

As an Appositive.
\begin{examples}

\latin{sī hoc optimum factū iūdicārem, Catilīnam morte multārī},
\english{if I thought this the best course to take,
  \emph{\(namely\)} that Catiline should be put to death};
\apud{Cat.}{1, 12, 29}.

\end{examples}

\item

After \latin{inter} in the phrase \latin{interest inter}.
\begin{examples}

\latin{inter valere et aegrōtare nihil interesse},
\english{\emph{\(said\)} that there was no difference between being
  well and being ill};
\apud{Fin.}{2, 13, 43}.

\end{examples}

\end{enuma}

\subsection

The Infinitive is occasionally used with \latin{habeō}.
\begin{examples}

\latin{habeō dīcere quem dēiēcerit},
\english{I can tell whom he cast down};
\apud{Rosc.\ Am.}{33, 100}.

\latin{nihil habeō scrībere},
\english{I have nothing to write};
\apud{Att.}{2, 22, 6}.

\end{examples}

\headingB{B. Poetical and Later Prose Uses of the Infinitive}

\section
\subsection

The poets of all periods use the Infinitive freely to express Purpose:
\begin{enuma}

\item

With Verbs of \emph{motion}.\footnote{E.g.\ \latin{eō}, \latin{veniō},
\latin{abigō}.}
\begin{examples}

\latin{nōn Libycōs populāre penātīs vēnimus},
\english{we have not come to sack the homes of Libya};
\apud{Aen.}{1, 527}.
(\latin{Populāre} = \latin{ut populēmus}.)

\end{examples}

\item

With Verbs of \emph{giving} or
\emph{undertaking}.\footnote{E.g.\ \latin{dō}, \latin{dōnō},
  \latin{trādō}, \latin{ministrō}, \latin{sūmō}.}
\begin{examples}

\latin{lōrīcam dōnat habēre virō},
\english{he gave to the hero a breastplate to possess};
\apud{Aen.}{5, 260}.
(\latin{Habēre} = \latin{habendam}, \xref[2]{605}.)

\end{examples}

\end{enuma}

\subsection

The later poets use the Infinitive freely:
\begin{enuma}

\item

As the Object of Verbs of \emph{granting} or \emph{taking
  away}.\footnote{E.g.\ \latin{dō}, \latin{tribuō}, \latin{concēdō},
  \latin{reddō}, \latin{adimō}, \latin{ēripiō}, \latin{perdō}.}
\begin{examples}

\latin{tū dās epulīs accumbere dīvom},
\english{thou grantest to recline at the banquets of the gods};
\apud{Aen.}{1, 79}.

\end{examples}

\item

In place of a Subjunctive Substantive Clause.
\begin{examples}

\latin{celerāre fugam suādet} (for \latin{ut celeret suādet}),
\english{urges her to speed her flight};
\apud{Aen.}{1, 357}.

\latin{dūcī intrā mūrōs hortātur},
\english{urges that it be brought within the walls};
\apud{Aen.}{2, 33}.

\end{examples}

\item

With Adjectives, or Participles of adjective
force.\footnote{E.g.\ \latin{doctus}, \latin{doctior},
  \latin{indoctus}, \latin{docilis}, \latin{indocilis};
  \latin{callidus}, \latin{sollers}, \latin{sagāx}, \latin{cautus},
  \latin{prūdēns}, \latin{perītus}, \latin{blandus}; \latin{patiēns},
  \latin{impatiēns}; \latin{solitus}, \latin{īnsolitus};
  \latin{audāx}, \latin{timidus}; \latin{cupidus}, \latin{certus},
  \latin{sciēns}, \latin{nescius}; \latin{dignus}, \latin{aptus},
  \latin{idōneus}; \latin{impiger}, \latin{piger}, \latin{celer},
  \latin{sēgnis}; \latin{bonus}, \latin{efficāx}, \latin{ūtilis},
  \latin{pār}, \latin{minor}; \latin{potēns}, \latin{impotēns}.} The
later prose-writers follow to a large extent.
\begin{examples}

\latin{certa morī},
\english{determined to die};
\apud{Aen.}{4, 564}.

\latin{sī crēdere dignum est},
\english{if the story is worthy of belief};
\apud{Aen.}{6, 173}.

\latin{legī dignus},
\english{worthy to be read};
\apud{Quintil.}{10, 1, 96}.

\latin{praestantior ciēre},
\english{more skilful in arousing};
\apud{Aen.}{6, 165}.

\end{examples}

\item

With Nouns denoting \emph{attention} or \emph{opportunity}.
\begin{examples}

\latin{dum praecipitāre potestās},
\english{while there is opportunity for flight};
\apud{Aen.}{4, 565}.

\end{examples}

\end{enuma}

\begin{minor}

\subsection

\versionA{The later poets use the Infinitive occasionally as a
  Substantive with a Verb, or after certain Prepositions governing the
  Accusative.}
\versionB*{\par The later writers, especially the poets, use the
  Infinitive occasionally as a mere Substantive depending upon a Verb,
or in the Accusative after certain prepositions.}
\begin{examples}

\latin{postquam sapere urbī vēnit nostrum},
\english{afer this philosophizing of ours came to town};
\apud{Persius}{6, 38}.

\latin{Tityon cernere erat},
\english{one could see Tityos}
(it was possible to see);
\apud{Aen.}{6, 595}.

\latin{nīl praeter plōrāre},
\english{nothing except to weep};
\apud{Sat.}{2, 5, 69}.

\end{examples}

\end{minor}

\chapter{The Participle}

\contentsentry{B}{Uses of the Participle}

\section

The Participle is a Verbal Adjective.

\subsubsection

As an Adjective, it belongs to a Substantive, and agrees with it
(\xref{320}).

\subsubsection

As a Verb, it expresses Voice, governs Cases, and may be modified by
an Adverb.  It also expresses tense-ideas, but only those of
completion, progress, or futurity (action \emph{prior},
\emph{contemporaneous}, or \emph{yet to come}; see \xref{473}
and~\xref{600}).

\subsubsection

The negative is \latin{nōn}.

\headingC{Ordinary Tense-Meanings of the Participles}

\section

The ordinary Tense-Meanings of the Participles are as follows:

\subsection

The Present Active Participle represents an act as \emph{going on} at
the time of the main verb.
\begin{examples}

\latin{Cotta pugnāns occīditur},
\english{Cotta is killed \emph{\(while\)} fighting};
\apud{B.~G.}{5, 37, \emend{238}{5}{4–5}}.

\latin{Sp.\ Maelium novīs rēbus studentem occīdit},
\english{killed Spurius Maelius \emph{\(who was\)} plotting
  revolution};
\apud{Cat.}{1, 1, 3}.

\end{examples}

\begin{minor}

\subsubsection

For the use to express attempted action, and the use to express action
already for some time in progress, see \xref{484} and~\xref{485}.

\end{minor}

\subsection

The Future Active Participle represents an act as \emph{intended} or
\emph{impending} at the time of the main verb.
\begin{examples}

\latin{sed nōn est itūrus},
\english{but he does not itend to go}
(is not intending to go);
\apud{Cat.}{2, 7, 15}.

\end{examples}

\begin{minor}

\subsubsection

In Ciceronian prose, the Future Active Participle is almost wholly
confined to the Periphrastic Conjugation, as above.

\begin{note}

The only exceptions are the occasional use of \latin{ventūrus} and
\latin{futūrus} as Adjectives (\xref{248}), and a single example
expressing Purpose.

\end{note}

\end{minor}

\subsection

The Future Passive Participle represents an act as, at the time of the
main verb, \emph{necessary}, \emph{proper}, or \emph{intended}.
\begin{examples}

\latin{aciēs erat īnstruenda},
\english{the line of battle had to be formed}
(was to be formed);
\apud{B.~G.}{2, \emend{239}{20}{19}, 1}.

\latin{quod multō magis est admīrandum},
\english{which is much more to be wondered at};
\apud{Cat.}{1, 3, 7}.

\end{examples}

\begin{minor}

\subsubsection

The Impersonal Future Passive Participle with \latin{est} is very
common.  It governs a Dative or Ablative, if other parts of the
verb do.
\begin{examples}

\latin{mīlitibus dē nāvibus dēsiliendum erat},
\english{the soldiers had to leap down from the ships};
\apud{B.~G.}{4, 24, 2}.

\latin{resistendum senectūtī est},
\english{one must resist old age};
\apud{Sen.}{11, 35}.

\end{examples}

\subsubsection

In a few examples, the Future Passive Participle has the force of a
present passive.\footnote{Probably this was the original force.  So,
  e.g., \latin{vir honōrandus} may have meant originally \emph{a man
    honored}, next \emph{a man honorable}, and finally \emph{a man to
    be honored}.}
\begin{examples}

\latin{volvenda diēs},
\english{time rolling on}
(being rolled on);
\apud{Aen.}{9, 7}; cf.~\apud{}{1, 269}.

\end{examples}

\end{minor}

\subsection

The Perfect Passive Participle represents an act as \emph{already}
completed at the time of the main verb.
\begin{examples}

\latin{quō proeliō sublātī audācius subsistere coepērunt},
\english{\(having been\) cheered up by this engagement, they began to
  resist with more spirit};
\apud{B.~G.}{1, 15, \emend{240}{3}{2}}.

\end{examples}

\pagebreak

\headingC{Occasional Tense-Meanings of the Participles}

\section

The tense-meaning is sometimes shifted, as follows:

\subsection

The Perfect Passive Participles of a few Deponent or Semi-Deponent
Verbs gain naturally a \emph{present meaning}; e.g.\ \latin{arbitrātus},
\english{having come to think},—and so \english{thinking}.  The use
is then extended to other verbs.\footnote{The most important
  participles of the kind with which the use arose are
  \latin{arbitrātus} and \latin{ratus}, \latin{cōnfīsus},
  \latin{diffīsus}, \latin{gāvīsus}, \latin{solitus}, \latin{veritus}.
  The most important to which the use was extended later are
  \latin{amplexus}, \latin{ausus}, \latin{cōnātus}, \latin{complexus},
  \latin{ēmēnsus}, \latin{imitātus}, \latin{locūtus}, \latin{pālātus},
  \latin{secūtus}, \latin{sōlātus}, \latin{ūsus}.  The poets use the
  idiom with great freedom.}
\begin{examples}

\latin{īsdem ducibus ūsus Numidās subsidiō oppidānīs mittit},
\english{employing the same men as guides, he sends the Numidians to
  the relief of the inhabitants of the town};
\apud{B.~G.}{2, 7, 1}.  Similarly \latin{complexī}, \apud{Cat.}{2, 5,
  10}.

\end{examples}

\subsection

The later writers, especially the poets, extend the usage to passive
verbs used reflexively\footnote{Thus with \latin{abscissus},
  \latin{circumfūsus}, \latin{conversus}, \latin{effūsus},
  \latin{percussus}, \latin{prōtēctus}, \latin{tūnsus}.} (\xref[3]{288}),
and from these to true passive verbs.
\begin{examples}

\latin{trīstēs et tūnsae pectora palmīs},
\english{sad, and beating their breasts with their hands};
\apud{Aen.}{1, 481}.  (Reflexive use of verb.)  Similarly
\latin{prōtēctī}, \apud{Aen.}{2, 444}.

\latin{portam conversō cardine torquet},
\english{turns the gate upon its revolving hinge};
\apud{Aen.}{9, 724}.
(True passive verb.)
Similarly
\latin{vectōs}, \apud{}{6, 335};
\latin{invectus}, \apud{}{1, 155}.

\latin{servum caesum mediō ēgerat Circō},
\english{had driven a slave under the lash \emph{\(being beaten\)}
  through the midst of the Circus};
\apud{Liv.}{2, 36, 1}.
(\latin{Caesum} = \english{being beaten}, instead of \english{having
  been beaten}.)

\end{examples}

\headingC{Forms Lacking and how they are Supplied}

\section
\subsection

Latin has no Perfect \emph{Active} Participle.  It therefore cannot
directly express such an idea as \emph{having done so and so}.

Indirectly, the idea may be expressed by a clause with \latin{ubi},
etc., or \latin{cum}, by an Ablative Absolute, or by a Perfect Passive
Participle attached to the Object of the main verb.

\begin{minor}

\subsubsection

But the Perfect Passive Participle of \emph{Deponent} or
\emph{Semi-Deponent} Verbs has of course the perfect active meaning
(\xref[\emph{a}]{291}).  Thus \latin{cōnspicātus}, \english{having
  seen}.

\end{minor}

\subsection

Latin has no Present Passive Participle. The place of this is supplied
by a relative clause, a clause with \latin{ubi}, etc., or \latin{cum},
or \latin{dum}.

\headingB{A. Common Uses of the Participle in All Periods}

\begin{minor}

\section[\textsc{Introductory}]

Since the Participle can be attached, directly or indirectly, to any
verb, the combination of the two affords a means by which the speaker
or writer can present two acts (or states) \emph{together} to his
hearer or reader, without in any way indicating what the actual
relation of the two acts to each other is.  That relation, if any
exists, is left for the hearer or reader \emph{to feel}.

Because of this adaptability to easy and condensed expression, the
Participle has a wide use in Latin.

\end{minor}

\section

In its most common uses, the Participle is employed:

\subsection

For \emph{compactness}, in place of a coördinate clause.
\begin{examples}

\latin{Germānī hostīs locō dēpellunt; fugientīs persequuntur},
\english{the Germans dislodge the enemy from their position, and, as
  these flee, pursue them};
\apud{B.~G.}{7, 67, 5}.
(= \latin{illī fugiunt; Germānī persequuntur}.  The enemy flee, and
the Germans pursue them.  A \emph{new} fact is narrated by
\latin{fugientīs}.)

\end{examples}

\begin{minor}

\subsubsection

The Participle is often used \emph{to repeat} something already stated
in a Finite Verb.
\begin{examples}

\latin{exercitum fundit, fūsum persequitur},
\english{he routs the army, and, after routing it, pursues}
(pursues it, routed);
\apud{Liv.}{1, 10, 4}.

\end{examples}

\end{minor}

\subsection

To express \emph{Situation},\footnote{The Participle never expresses
  the mere idea of time.} with or without a causal or adversative
suggestion.
\begin{examples}

\latin{occīsus est ā cēnā rediēns},
\english{was killed on his way back from dinner};
\apud{Rosc.\ Am.}{34, 97}.
(No relation suggested.)

\latin{stantem urbem relīquit},
\english{he left the city still standing};
\apud{Cat.}{2, 1, 2}.
(No relation suggested.)

\latin{illum exercitum contemnō, conlēctum ex senibus
  dēspērātīs\dots},
\english{I think little of that army, patched up as it is of desperate
  old men\dots};
\apud{Cat.}{2, 3, 5}.
(Causal relation suggested.  \latin{Conlēctum = quia conlēctus est}.)

\latin{ut eum cupientēs tenēre nequeāmus},
\english{so that, though we wish to restrain it \emph{\(namely,
    laughter\)}, we cannot};
\apud{De~Or.}{2, 58, 235}.
(Adversative relation suggested.  \latin{Cupientēs} = \latin{quamquam
  cupimus} or \latin{quamvīs cupiāmus}.)

\end{examples}

\begin{note}

The Participle is used abundantly, in either of the above ways (1 and
2), to narrate an Event or Situation, as preparation for the narration
of the main event.  It may then be called the \term{Narrative
  Participle}.

Thus used, it forms an equivalent for either an Aoristic Narrative
Clause with \latin{ubi}, etc.\ (\xref{557}) or a Narrative
\latin{cum}-Clause of Situation (\xref{524}).  (There may of course be
an \emph{accessory} causal or adversative idea, as in
\latin{cum}-Clauses; \xref{525}.)
\begin{examples}

\latin{ā quō nōn receptus ad mē venīre ausus es},
\english{and when he did not take you in, you had the hardihood to
  come to me};
\apud{Cat.}{1, 8, 19}.
(\latin{Nōn receptus} = \latin{ubi nōn receptus es}, or \latin{cum nōn
  receptus essēs}.)

\end{examples}

\end{note}

\subsection

To express a \emph{Condition}.
\begin{examples}

\latin{damnātum poenam sequī oportēbat, ut ignī cremārētur},
\english{the punishmint of being burned alive must follow his
  conviction};
\apud{B.~G.}{1, 4, 1}.
(Must follow, \emph{if} he should be convicted. \latin{Damnātam} =
\latin{sī damnātus esset}.)

\latin{quis potest, mortem metuēns, esse nōn miser?}
\english{what man, fearing death \emph{\(= i.e.\ \emph{if} he fears
    death\)} can help being wretched?}
\apud{Tusc.}{5, 6, 15}.
(\latin{Metuēns} = \latin{quī metuit}, or \latin{sī metuit}.
Generalizing Condition; \xref{579}.)

\end{examples}

\subsection

To express the \emph{Way by Which} (\emph{Means}).
\begin{examples}

\latin{facit amīcitia adversās (rēs) partiēns leviōrēs},
\english{friendship makes misfortune lighter by dividing it};
\apud{Am.}{6, 22}.

\end{examples}

\subsection

To express \emph{Manner}.
\begin{examples}

\latin{flentēs implōrābant},
\english{they begged with tears};
\apud{B.~G.}{1, 51, \emend{241}{2}{3}}.

\end{examples}

\subsection

To express an \emph{Act Not Accompanying the main act} (English
“without \blank[.45em]ing”).  A negative must be added.
\begin{examples}

\latin{miserum est nihil prōficientem angī},
\english{it is a wretched thing to suffer without accomplishing
  anything};
\apud{N.~D.}{3, 6, 14}.

\end{examples}

\subsection

In place of \emph{a Relative Clause}, as follows:
\begin{enuma}

\item

In place of a Determinative Clause (\xref{550}).
\begin{examples}

\latin{sepultūrum occīsōrum},
\english{the burial of those who had been slain}
(\latin{occīsōrum} = \latin{eōrum quī occīsī erant});
\apud{B.~G.}{1, 26, 5}.

\end{examples}

\item

In place of a Descriptive Clause (\xref[1]{521}; \xref{569}).
\begin{examples}

\latin{dē bene meritīs cīvibus},
\english{regarding citizens who have served you well};
\apud{Mil.}{2, 4}.

\end{examples}

\item

In place of a Parenthetical Clause (\xref{567}).
\begin{examples}

\latin{mortem igitur omnibus hōrīs impendentem timēns quī poterit
  animō cōnsistere?}
\english{if a man fears death,—which at every moment hangs over
  us,—how can he be steady in mind?}
\apud{Sen.}{20, 74}.
(\latin{Impendentem} = \latin{quae impendet}.)

\end{examples}

\end{enuma}

\headingB{B. Special Idioms of The Participle in all Periods}

\section

The Romans were fond of the use of the Participles with certain kinds
of Verbs, as follows:

\subsection

The Present Active Participle with verbs of \emph{seeing},
\emph{hearing}, or \emph{representing}.\footnote{The most common are
  \latin{videō}, \latin{audiō}, \latin{faciō}, \latin{fingō},
  \latin{indūcō} (\english{bring upon the stage}).}

\begin{examples}

\latin{vidēre hanc urbem concidentem},
\english{to see this city falling};
\apud{Cat.}{4, 6, 11}.

\end{examples}

\begin{note}

The Infinitive also may be used with these verbs.  In the Passive
Voice the Infinitive alone is possible, since there is no present
passive participle.
\begin{examples}

\latin{quōs videō volitāre in forō},
\english{whom I see flitting about in the forum};
\apud{Cat.}{2, 3, 5}.

\latin{cōnstruī ā deō atque aedificārī mundum facit},
\english{\emph{\(Plato\)} represents the world as being constructed and
  built by God};
\apud{N.~D.}{1, 8, 19}.

\end{examples}

\end{note}

\subsection

The Future Passive Participle to express Purpose with verbs of
\emph{giving}, \emph{leaving}, or \emph{marking}\versionB*{ (and
  many others in poetry)}.\footnote{The most common are \latin{dō},
  \latin{dēferō}, \latin{trādō}, \latin{praebeō}, \latin{concēdō},
  \latin{relinquō}, \latin{dēnotō}.}
\begin{examples}

\latin{hōs Haeduīs custōdiendōs trādit},
\english{these he hands over to the Haedui to be guarded};
\apud{B.~G.}{6, 4, 4}.

\end{examples}

\subsection

The Perfect Passive Participle (emphatic or energetic; \xref{490})
with verbs of \emph{wishing}.
\begin{examples}

\latin{sē probātum voluit},
\english{he wished himself well approved};
\apud{Caecin.}{36, 103}.

\end{examples}

\subsection

The Perfect Passive Participle with certain verbs of \emph{giving} or
\emph{making},\footnote{The most common are \latin{dō}, \latin{reddō},
  \latin{faciō}, \latin{cūrō}.} to represent something as \emph{put
  into} a completed condition (emphatic or energetic).
\begin{examples}

\latin{sī quī voluptātibus dūcuntur, missōs faciant honōrēs},
\english{people who are led by pleasure must give the honors of life a
  complete dismissal};
\apud{Sest.}{66, 138}.

\end{examples}

\subsection

The Perfect Passive Participle with verbs of \emph{having},
\emph{holding}, or \emph{possessing},\footnote{The most common are
  \latin{habeō}, \latin{teneō}, \latin{possideō}.} to represent
something as \emph{ready} or \emph{kept} in a completed condition.
\begin{examples}

\latin{ducēs comprehēnsōs tenētis},
\english{you hold the leaders under arrest};
\apud{Cat.}{3, 7, 16}.

\latin{certōs hominēs dēlēctōs ac dēscrīptōs habēbat},
\english{he had certain men selected and appointed}
(= he had selected, etc.);
\apud{Cat.}{3, 7, 16}.

\end{examples}

\begin{minor}

\subsubsection

With \latin{habeō}, the construction approaches closely to that of our
English perfect with \emph{have}, which is descended from it.

\end{minor}

\headingB{New Uses of the Participles in Later Latin}

\section

In later Latin, the Present Participle may be used to express
\emph{Purpose}.
\begin{examples}

\latin{lēgātī missī (sunt) auxilium ōrantēs},
\english{ambassadors were sent \emph{\(asking\)} to ask for help};
\apud{Liv.}{21, 6, 2}.
Similarly \latin{scītantem}, \apud{Aen.}{2, 114}.

\end{examples}

\begin{minor}

\subsubsection

This use is an extension of a true present use, as in \latin{vēnērunt
  auxilium ōrantēs}, \english{they came asking \(and, \emph{of course},
  to ask\) help}.

\end{minor}

\section

After Cicero’s time, the Future Participle, Active or Passive, gains a
wide use.

In addition to its older use in the Periphrastic Conjugation, it is
employed to express something as \emph{destined}, \emph{anticipated},
or \emph{purposed}, or to take the place of a \emph{condition}, a
\emph{conclusion}, or a \emph{relative clause}.
\begin{examples}

\latin{sēsē medium iniēcit peritūrus in agmen},
\english{and flung himself into their ranks—to die} (destined to die);
\apud{Aen.}{2, 408}.

\latin{sī peritūrus abīs},
\english{if you are going away to perish};
\apud{Aen.}{2, 675}.

\latin{dā mānsūram urbem},
\english{grant a city that shall abide};
\apud{Aen.}{3, 85}.

\end{examples}

\headingB{The Participle as Expressing the Leading Idea of its Phrase}

\section

The Participle originally expressed the less important idea of the
\linebreak
phrase to which it belongs, as in the examples above.

But in three uses the Participle came to express the \emph{leading
  idea} of the phrase (cf.~\xref{333}).  The English equivalent for it
is then a Verbal Noun, governing an Object.  These uses are:

\subsection

The Perfect Passive Participle with a Noun, depending on \latin{opus
  est}.  See \xref[2]{430}.

\subsection

The Perfect Passive or Present Active Participle with a Noun,
depending upon a Preposition, or in the Genitive, or, less frequently,
in the Nominative.
\begin{examples}

\latin{cum dē homine occīsō (= dē hominis caede) quaerātur},
\english{when there is an investigation about a man killed},
i.e.\ \english{about the killing of a man};
\apud{Mil.}{3, 8}.

\latin{post hanc urbem conditam},
\english{since the founding of this city};
\apud{Cat.}{3, 6, 15}.

\latin{ob īram interfectī dominī},
\english{through anger at the killing of his master};
\apud{Liv.}{21, 2, 6}.
(\latin{Interfectī dominī} = \latin{caedis dominī}.)
Cf.\ \apud{Aen.}{2, 413}.

\latin{fugiēns Pompeius mīrābiliter hominēs movet},
\english{Pompey’s flight is stirring people up extraordinarily};
\apud{Att.}{7, 11, 4}.

\end{examples}

\begin{minor}

\subsubsection

The construction is sometimes found in English, mainly in poetry.
Cf.\ Oliver Wendell Holmes: “Our midnight is Thy smile withdrawn.”

\end{minor}

\subsection

The Future Passive Participle with a Noun.

The Future Passive Participle with a Noun, when used to convey the
leading idea in its phrase, receives a \emph{new name}, that of the
“Gerundive,” and will therefore be treated under that heading.  The
related construction of the Gerund will be treated at the same time.

\chapter{The Gerundive and the Gerund}

\contentsentry{B}{Uses of the Gerundive and Gerund}

\section

The \term{Gerundive} is the Future Passive Participle, \emph{after it
  has gained the power of conveying the leading idea in its phrase}.
\begin{examples}
\latin{in iīs libellīs quōs dē contemnendā glōriā scrībunt},
\english{in the essays which they write about despising glory}
(about glory being despised);\footnote{See \xref[3, \emph{b}, and
    footnote]{600}.}
\apud{Arch.}{11, 26}.
(\latin{Dē contemnendā glōriā} = \latin{dē contemptiōne glōriae}.)

\latin{exercendae memoriae grātiā},
\english{for the sake of exercising the memory}
(for the sake of memory to be exercised);
\apud{Sen.}{11, 38}.
(\latin{Exercendae memoriae} = \latin{exercitātiōnis memoriae}.)

\end{examples}

\section

The Gerundive is thus nearly the equivalent of a Verbal Noun.  But it
is not yet a \emph{complete} Verbal Noun.  Instead of depending
directly on the word which governs the phrase, and itself governing
the other word of the phrase, as in the English “about despising
glory,” it is still subordinate to that other word, and has to agree
with it (as in \latin{dē contemnendā glōriā}).  It is in
\emph{thought} the leading word, but not yet \emph{grammatically} so.

Naturally, it came in time to take this one step further, and became a
complete Verbal Noun, in the Neuter Gender.

\section

The \term{Gerund} is a \emph{complete verbal noun}.

As a Verb, it has the power, if transitive, of governing a Noun or
Pronoun; as a Noun, it is itself governed in case.

\begin{note}

The Gerundive and Gerund differ from the true Future Passive
Participle\footnote{The traditional usage, by which the name
  “Gerundive” is employed instead of the name “Future Passive
  Participle\emend{48}{}{,}” is confusing.  Obviously, the word Gerundive
  should be restricted to uses which have exact parallels in uses of
  the Gerund.} in four ways:
\begin{enumerate*}

\item They express the leading idea of their phrase.

\item They convey no idea of necessity or obligation.

\item They are active in feeling, not passive.\footnote{Thus
  \latin{Carthāgō dēlenda est} means \english{Carthage must be
    destroyed} (passive), while \latin{spēs Car\-thā\-gi\-nis dēlendae}
  (Gerundive) means \english{the hope of destroying Carthage}
  (active).}

\item They accordingly cannot take any construction of the agent.

\end{enumerate*}

\end{note}

\smallskip

\suppresschapterspace

\headingB{Common Uses of the Gerundive and Gerund in All Periods}

\section

The Gerundive and Gerund exist only in the Genitive, Dative,
Accusative, and Ablative cases.  The case-uses, so far as they go, are
in general the same as those of Nouns.

In Ciceronian Latin, the principal uses are as follows:
\begin{enumI}[III]

\item

\term{Genitive}.  After any Noun or Adjective that can govern a
Genitive \emph{Noun}.

\item

\term{Dative}.  After any Adjective or Phrase that can govern a Dative
\emph{Noun}; also after certain \emph{official phrases}, and after
\latin{sum} or \latin{adsum}.

\item

\term{Accusative}.  After \emph{Prepositions}, mainly
\latin{ad};\footnote{Rarely with \latin{ante}, \latin{circā},
  \latin{ergā}, \latin{in}, \latin{inter}, \latin{ob},
  \latin{propter}, \latin{super}.} and after Verbs of
\emph{arranging}, \emph{contracting}, or \emph{giving a
  contract}.\footnote{\latin{Cūrō} = \english{have a thing done},
  \latin{condūcō} = \english{take a contract}, \latin{locō} =
  \english{give a contract}, etc.}

\item

\term{Ablative}. To express \emph{Means}, \emph{Circumstances}, or
\emph{Cause}, and after \emph{Prepositions}, mainly \latin{dē},
\latin{ex}, \latin{in}.\footnote{Rarely with \latin{cum}, \latin{prō},
  \latin{super}.}

\end{enumI}

\smallskip

\emph{Examples of the four case-uses}:
\begin{sidebyside}

\cc{1}{\textbf{GERUNDIVE}}
& \cc{1}{\textbf{GERUND}}
\endhead

\cc{2}{\textsc{I.\enskip Genitive}} \\[\smallskipamount]

\latin{cupiditās bellī gerendī},
\english{desire of carrying on war};
\apud{B.~G.}{1, 41, 1}.
(Objective Genitive; \xref{354})
&
\latin{hominēs bellandī cupidī},
\english{men desirous of fighting};
\apud{B.~G.}{1, 2, \emend{242}{4}{5}}.
(Objective Genitive; \xref{354}.)
\\

\latin{neque cōnsilī habendī}
(continued on right)
&
\latin{neque arma capiendī spatiō datō},
\english{time being given neither for taking counsel nor for seizing
  their arms};
\apud{B.~G.}{4, 14, 2}.
\\

\latin{difficultātēs bellī gerendī},
\english{difficulties in carrying on the war};
\apud{B.~G.}{3, 10, 1}.
(Genitive of Connection; cf.\ \latin{difficultātēs bel\-lī},
\xref{339}.)
&
\latin{difficultās nāvigandī},
\english{difficulty in navigating};
\apud{B.~G.}{3, 12, 5}.
(Genitive of Connection; \xref{339}.)
\\

\latin{praedae} (= \latin{praedandī}) \latin{ac bellī īnferendī causā},
\english{for the sake of plunder and making war};
\apud{B.~G.}{5, 12, 2}.
&
\latin{praedandī causā},
\english{for the sake of plundering};
\apud{B.~G.}{2, \emend{107}{17}{16}, 4}.
\\[\medskipamount]

\cc{2}{\textsc{II.\enskip Dative}} \\[\smallskipamount]

\latin{locum oppidō condendō cēpērunt},
\english{they chose a place for founding a town};
\apud{Liv.}{39, 22, 6}.
(Dative of Object for Which; cf.\ \xref{361}.)
&
\latin{quem quisque pugnandō locum cē\-pe\-rat},
\english{the place that each had taken for fighting};
\apud{Sall.\ Cat.}{61, 2}.
(Dative of Object for Which; cf.\ \xref{361}.)
\\

\latin{sunt nōn nūllī acuendīs puerōrum ingeniīs nōn inūtilēs lūsūs},
\english{there are certain games that are not bad for sharpening the
  wits of boys};
\apud{Quintil.}{1, 3, 11}.
(Dative of Direction, \xref{362}.)
&
\latin{aqua ūtilis bibendō},
\english{water good for drinking};
\apud{Plin.\ N.~H.}{31, 59}.
(Dative of Direction; \xref{362}.)
\\
\latin{cōnsul plācandīs dīs dat operam},
\english{the consul devotes his attention to appeasing the gods};
\apud{Liv.}{22, 2, 1}.
(Dative of Indirect Object; \xref{365}.)
&
\latin{is cēnsendō fīnis factus est},
\english{this was made the ending of \emph{\(for\)} the taking of the
  census};
\apud{Liv.}{1, 44, 2}.
(Dative of Indirect Object; \xref{365}.)
\\
\latin{(cōnsul) comitia conlēgae subrogandō habuit},
\english{the consul held an election for the appointing of a
  colleague};
\apud{Liv.}{2, 8, 3}.
(Dative after an official phrase; \xref[II]{612}.)
&
\latin{cum solvendō cīvitātēs nōn essent},
\english{since the states were not equal to paying}
(not solvent);
\apud{Fam.}{3, 8, 2}.
(Special idiom, after \latin{sum} or \latin{adsum}, \xref[II]{612}.)
\\[\medskipamount]

\cc{2}{\textsc{III.\enskip Accusative}} \\[\smallskipamount]

\latin{ad hās rēs cōnficiendās sibi trīduī spa\-ti\-um daret},
\english{that he should give them three days’ time for accomplishing
  this};
\apud{B.~G.}{4, 11, 3}.
(Purpose; cf.\ \xref[3]{384}.)
&
\latin{nūllum sibi ad cognōscendum spatium relinquunt},
\english{leave themselves no time for investigating};
\apud{B.~G.}{7, 42, 1}.
(Purpose; cf.\ \xref[3]{384}.)
\\
\latin{ad bella suscipienda Gallōrum alacer est animus},
\english{the temper of the Gauls is keen for undertaking wars};
\apud{B.~G.}{3, \emend{127}{19}{16}, 6}.
(Figurative Direction; cf.\ \xref[2]{384}.)
&
\latin{cum hostēs nostrōs mīlitēs alacriōrēs ad pugnandum
  effēcissent},
\english{when the enemy had made our soldiers keener for fighting};
\apud{B.~G.}{3, \emend{243}{24}{22}, 5}.
(Figurative Direction; cf.\ \xref[2]{384}.)
\\

\latin{pontem in Ararī faciendum cūrat},
\english{he sees to the building of a bridge over the Saône};
\apud{B.~G.}{1, 13, 1}.\footnote{True Gerundive construction; for the
  leading idea is carried by the grammatically subordinate word
  \latin{faciendum}.  Compare the constrasting Participial use in
  \xref[2]{605}.}
%\\[\medskipamount]

\\

\cc{2}{\textsc{IV.\enskip Ablative}} \\[\smallskipamount]

\latin{loquendī ēlegantia augētur legendīs ōrā\-tō\-ri\-bus et poētīs},
\english{distinction in speech is increased by reading the orators and
  poets};
\apud{De~Or.}{3, 10, 39}.
(Means, \xref{423}; cf.\ \apud{B.~G.}{3, 25, 1}.)
&
\latin{(memoria) excolendō augētur},
\english{memory is built up by using it};
\apud{Quintil.}{11, 2, 1}.
(Means, \xref{423}; cf.\ \apud{B.~G.}{4, 13, 5}.)
\\
\latin{cum plausum meō nōmine recitandō de\-dis\-set},
\english{when \(the people\) had applauded at the reading of my name};
\apud{Att.}{4, 1, 6}.
(Circumstances; \xref[1]{422}.)
&
\latin{imperandō sociīs in tantum adductus perīculum},
\english{brought into such danger in \emph{\(by\)} directing the
  allies};
\apud{Verr.}{1, 27, 70}.
(Means, becoming Circumstances, \xref[1]{422}.)
\\
\latin{in eā (voluptāte) spernendā virtūs maximē cernitur},
\english{manliness is best seen in the despising of pleasure};
\apud{Leg.}{1, 19, 52}.
(Field in Which, with \latin{in}; \xref[2]{434}.)
&
\latin{industria in agendō},
\english{energy in action}
(in acting);
\apud{Pomp.}{11, 29}.
(Field in Which, with \latin{in}; \xref[2]{434}.)
\end{sidebyside}

\begin{note}[Note 1]

The Gerundive or Gerund in the Ablative of Means or Circumstances
sometimes approaches the force of a Participle.  In later Latin, the
\emph{Gerund} is frequent with this force.
\begin{examples}

\latin{aliīs frūctum libīdinum nōn modo impellendō vērum etiam
  adiuvandō pol\-li\-cē\-bā\-tur},
\english{to others he promised the enjoyment of their lusts, not only
  urging them but also aiding them};
\apud{Cat.}{2, 4, 8}.
Cf.\ \latin{fandō}, \apud{Aen.}{2, 6};
\latin{tuendō}, \apud{Aen.}{1, 713}.

\end{examples}

\end{note}

\begin{note}[Note 2]

\looseness=-1
Rarely, the Gerund is used as an appositive, as in \latin{rēs
  dīversissimās, pārendum atqe imperandum}, \english{two very
  different things, obeying and commanding}; \apud{Liv.}{21, 4,~3}.

\end{note}

\section

Where the phrase contains a Noun or Pronoun, the Gerundive is more
common than the Gerund in Ciceronian Latin.  But either construction
\emph{may} be employed, except as follows:

\subsection

The Gerundive alone is employed in the Dative or after a Preposition.
Hence one must say, e.g.: \latin{plācandīs dīs dat operam, ad eās rēs
  cōnficiendās, in voluptāte spernendā}, etc., as above.

\subsection

The Gerund alone is employed:
\begin{enuma}

\item

With a Neuter Adjective used substantively.
\begin{examples}

\latin{artem vēra ac falsa dīiūdicandī},
\english{the art of distinguishing true things from false things};
\apud{De~Or.}{2, 38, 157}.
(Not \latin{vērōrum ac falsōrum dīiūdicandōrum}, which might be taken
to mean \english{of distinguishing true men from false men}.)

\end{examples}

\item

If the verb used is Intransitive.
\begin{examples}

\latin{hominī cupidō satisfaciendī reī pūblicae},
\english{a man desirous of doing his duty to the commonwealth};
\apud{Fam.}{10, 18, 1}.

\end{examples}

\begin{note}

The Deponent Verbs \vrb{ūtor}, \vrb{fruor}, \vrb{fungor},
\vrb{potior}, and \vrb{vēscor}, being really transitive in meaning
(\xref[\emph{b}]{429}), can take either construction.
\begin{examples}

\latin{spem potiundōrum castrōrum},
\english{hope of taking the camp};
\apud{B.~G.}{3, 6, 2}.

\latin{quārum potiendī spē},
\english{by the hope of gaining which};
\apud{Fin.}{1, 18, 60}.

\end{examples}

\end{note}

\end{enuma}

\negbigskip

\section

The Reflexive Genitives \latin{meī}, \latin{tuī}, \latin{suī},
  \latin{nostrī}, and \latin{vestrī} throw an accompanying Gerundive
  into \emph{the same form}, without regard to the actual gender or number of
  the person or persons meant.
\begin{examples}

\latin{suī opprimendī causā},
\english{for the sake of crushing them};
\apud{B.~G.}{1, 44, 10}.

\end{examples}

\begin{note}[Remark]

\latin{Meī}, \latin{nostrī}, etc., were originally Neuter Singular
Adjectives used substantively.  Hence the usage.

\end{note}

\headingB{Rarer Constructions of the Gerund or Gerundive}

\headingC{Objective Genitive with the Gerund}

\section

Occasionally, though rarely in Cicero, the Gerund takes an Objective
Genitive, just as an ordinary Verbal Noun may do.
\begin{examples}

\latin{exemplōrum ēligendī potestās},
\english{a chance for the selecting of examples};
\apud{Inv.}{2, 2, 5}.
(= \latin{exemplōrum ēlēctiōnis potestās}.  Cf.\ \latin{ēlēctiō
  verbōrum}, \apud{Or.}{20, 68}.)

\end{examples}

\headingC{The Genitive of the Gerundive in Expressions of Purpose}

\section

A Gerundive in the Descriptive Genitive, while strictly depending upon
a Noun, may \emph{suggest} the \emph{purpose} of an act.
\begin{examples}

\latin{paucōs post diēs quam ad bellum renovandum mīserant lēgātōs,
  pācis petendae ōrātōrēs mīsērunt},
\english{a few days after they had sent commanders to renew the war,
  they sent \emph{\(ambassadors of the peace-asking kind\)}
  ambassadors to ask for peace};
\apud{Liv.}{36, 27, 2}.\footnote{Similarly, \latin{nāvēs dēiciendī
    operis} (the reading of the better family of manuscripts) will be
  found in many texts in \apud{B.~G.}{4, 17, 10}.  Cf.\ \latin{suī
    commodī}, \apud{B.~G.}{5, 8, 6} (the reading of the same family).}

\latin{cētera in XII minuendī sūmptūs sunt},
\english{the remaining provisions in the Twelve Tablets are for the
  lessening of expense};
\apud{Leg.}{2, 23, 59} (same Genitive, in the predicate).  Similarly
\latin{cōnservandae lībertātis}, \apud{Sall.\ Cat.}{6, 7}.

\end{examples}

\chapter{The Supine}

\contentsentry{B}{Uses of the Supine}

\begin{minor}

\section[\textsc{Introductory}]

The Supine is a Verbal Noun of the Fourth Declension.  It has but two
forms in common use, one in~\suffix{-um} and one in~\suffix{-ū}.  The
form in~\suffix{-um} is an Accusative, expressing an action thought of
as the End of Motion (cf.~\xref{450}).  The form in~\suffix{-ū} is an
Ablative, generally expressing Respect (\xref{441}).

\end{minor}

\headingC{The Supine in \suffix{-um}}

\section

The Supine in~\xref{-um} is used to express Purpose after \emph{Verbs
  of \emend{85}{m}{M}otion}, and a few others \emph{implying} motion, real or
figurative.\footnote{These others are \latin{vocō} and \latin{revocō},
  \latin{dare} and \latin{conlocāre} with \latin{nūptum} (\emph{give
    or place to marry}, i.e.\ \english{in marriage}) and \latin{recipiō}
  with \latin{sessum} (\english{receive to sit}, i.e.\ \english{help to
    a seat}).

Virgil employs the construction with poetic boldness after
\latin{fortūnā ūtī} (\english{use our opportunity to})
in~\apud{Aen.}{9, \emend{244}{241}{240}}.}
\begin{examples}

\latin{lēgātōs ad Caesarem mittunt rogātum auxilium},
\english{they send ambassadors to Caesar to ask help};
\apud{B.~G.}{1, 11, \emend{245}{3}{2}}.

\latin{nōn Graiīs servītum mātribus ībō},
\english{I shall not go to play the slave to Grecian dames};
\apud{Aen.}{2, 786}.

\end{examples}

\begin{minor}

\subsubsection

The Supine in \suffix{-um} may itself be followed by any construction
which any other part of the Verb may take, e.g.\ a Direct Object, a
Dative, a Substantive Clause, an Indirect Question, etc.

\end{minor}

\headingC{The Supine in \suffix{-ū}}

\section

The Supine in~\suffix{-ū} is used:

\subsection

To express Respect with Adjectives,\footnote{Most frequently with
  \latin{facilis}, \latin{difficilis}, \latin{gravis},
  \latin{mīrābilis}, \latin{incrēdibilis}, \latin{honestus},
  \latin{turpis}, \latin{ūtilis}, \latin{iūcundus}, \latin{optimus}.

  The Supines mostly commonly occurring are \latin{dictū},
  \latin{factū}, \latin{audītū}, \latin{vīsū}, \latin{cognitū}.}  and
with \latin{fās} or \latin{nefās}.
\begin{examples}

\latin{perfacile factū},
\english{a very easy thing to do}
(in the doing);
\apud{B.~G.}{1, 3, \emend{246}{6}{5}}.

\latin{sī hoc fās est dictū},
\english{if this is right to say};
\apud{Tusc.}{5, 13, 38}.

\end{examples}

\begin{minor}

\subsubsection

Some of these Adjectives may also take the Gerund with \latin{ad}, as
in \latin{facile ad crēdendum}, \apud{Tusc.}{1, 33, 78}.

\end{minor}

\subsection

Occasionally after \latin{opus est}, \latin{dignus} or
\latin{indignus}.
\begin{examples}

\latin{quod scītū opus est},
\english{which it is necessary to know}
(which there is need of knowing);
\apud{Inv.}{1, 20, 28}.

\latin{nihil dignum dictū},
\english{nothing worth mentioning};
\apud{Liv.}{4, 30, 4}.

\end{examples}

\begin{minor}

\subsubsection

Ordinarily, \latin{opus est} takes the Perfect Passive Participle
(\xref[2]{430}) and \latin{dig\-nus} or \latin{indignus} a Subjunctive
\latin{quī}- or \latin{ut}-Clause (\xref[3]{513}).

\end{minor}

\begin{note}[Note 1]

The Supine in~\suffix{-ū} cannot take a Direct Object; for the thing
which is to be done is the \emph{Subject} of the statement.—But an
Infinitive of Statement or an Indirect Question sometimes forms an
\emph{apparent} Object of the Supine (really the Subject of the main
verb).
\begin{examples}

\latin{difficile est dictū quantō in odiō sīmus apud exterās
  nātiōnēs},
\english{it is difficult to say how foreign nations hate us}
(how much they hate us is difficult to say);
\apud{Pomp.}{22, 65}.

\end{examples}

\end{note}

\headingB{Word-Order}

\contentsentry{B}{Word-Order}

\begin{minor}

\section[\textsc{Introductory}]

In English, in which there is little inflection, word-order is largely
fixed.  Thus the idea “Caesar conquered Pompey” can be expressed
only in this order (“Pompey conquered Caesar”) would mean the
opposite).  In Latin, in which relations are largely expressed by
inflection, there is in the main no \emph{necessary} order.  Thus
\latin{Caesar Pompeium superāvit}, \latin{Pompeium Caesar superāvit},
and \latin{superāvit Pompeium Caesar} all tell the same fact, and
differ only with regard to the emphasis placed upon one part or
another.

Emphasis is expressed also by stress and by pitch.  But the written
sentence cannot indicate these means.

\end{minor}

\section

Emphasis may be obtained either by putting an important thing before
the hearer immediately, or by holding it back for a time, to stimulate
his curiosity.  Hence,

\emph{The most emphatic places in a sentence, clause, or group, are
  the first and the last.}  The places next these are relatively next
in emphasis, and so on.

\section

If no \emph{special} emphasis is to be given to any part, the subject
and the act are the most important things.  Hence they stand first and
last respectively.  Their modifiers naturally stand near them.

\chapter[Normal Word Order]{Normal Order}

\section

Accordingly, the \emph{normal\footnote{The words “regular” and
    “regularly,” “general” and “generally” are avoided in most
    of the following statements; for the actual majority of cases
    under a given class may perfectly well be on the side of the
    \emph{rhetorical} order.  Cf.\ \xref{625}.} order} of the sentence
is:
\begin{center}
\medskip
\emph{Subject, modifiers of the subject, modifiers of the verb, verb}.
\medskip
\end{center}

\latin{L.\ Flaccus et C.\ Pomptīnus praetōrēs meritō laudantur},
\english{Lucius Flaccus and Gaius Pomptinus, the praetors, are
  deservedly praised}; \apud{Cat.}{3, 6, 14}.

\subsubsection

The normal order of the modifiers of the verb and the verb itself is:
\begin{enumerate*}

\item
Remoter modifiers (time, place, situation, cause, means, etc.).

\item
Indirect object.

\item
Direct object.

\item
Adverb.

\item
Verb.

\end{enumerate*}

\subsubsection

But this exact order is not common, since there is almost always some
special shade of emphasis to disturb it. Cf.~\xref{625}.

\section
\subsection

Adjectives and genitives normally follow their nouns.\footnote{The
  general idea is given first, and this is then narrowed by a
  descriptive conception.  The same usage has come down in French.}
\begin{examples}

\latin{aetās puerīlis},
\english{the age of boyhood} (the boyish age);
\apud{Arch.}{\emend{49}{1, }{}3, 4}.

\latin{dīlātiōnem comitiōrum},
\english{the postponement of the election};
\apud{Pomp.}{1, \emend{50}{1, }{}2}.

\end{examples}

\begin{minor}

\subsubsection

\latin{Ūllus} and \latin{nūllus} normally precede their nouns.  Thus
\latin{nūllum malum}, \english{no evil}, \apud{Cat.}{4, 7, 15}.

\subsubsection

Certain combinations have settled into a stereotyped order.  Thus
\latin{cīvis Rōmānus}, \latin{pontifex maximus}, \latin{rēs pūblica};
\latin{senātūs cōnsultum}, \latin{plēbis scītum}, \latin{tribūnus
  plēbis}.  The genitive regularly precedes \latin{causā} and
\latin{grātiā}, \english{for the sake of}.

\end{minor}

\subsection

Determinative and intensive pronouns, and adjectives of quantity
or precision, normally precede their nouns.

So \latin{hic}, \latin{is}, \latin{iste}, \latin{ille}; \latin{ipse};
\latin{ūnus}, \latin{duo}, etc.; \latin{omnis}, \latin{tōtus},
\latin{ūniversus}, \latin{cūnctus}, \latin{multus}, \latin{tantus};
\latin{proximus}, \latin{superior},\footnote{Some of these,
  e.g.~\latin{hic}, \latin{is}, etc., form a constituent part of the
  thought, and so are not easily held in suspense.  Others, like
  \latin{multus} and \latin{tantus}, are naturally emphatic. The same
  usage has come down in French.} etc.
\begin{examples}

\latin{hic locus},
\english{this place};
\apud{Pomp.}{1, 2}.

\latin{omnis hic locus},
\english{this entire place};
\apud{Cat.}{3, 10, \emend{247}{14}{24}}.

\latin{ūniversus senātus cēnsuit\dots},
\english{the whole senate voted\dots},
\apud{Sull.}{49, 136}.

\end{examples}

\subsubsection

\latin{Ille} meaning “the famous” normally follows its noun; but it
regularly goes \emph{with} an adjective or appositive, wherever this
may stand.
\begin{examples}

\latin{Mēdeā illa},
\english{the famous Medea},
\apud{Pomp.}{9, 22}.

\latin{Catō ille sapiēns},
\english{Cato, the famous sage};
\apud{Div.}{1, 15, 28}.

\latin{sapientī illī Catōnī},
\english{the famous sage Cato};
\apud{Leg.}{2, 2, 5}.

\end{examples}

\subsection

Possessive and indefinite pronouns, and ordinal numerals, normally
follow their nouns.
\begin{examples}

\latin{avī tuī},
\english{of your grandfather};
\apud{Cat.}{3, 5, 10}.

\latin{cāsū aliquō},
\english{by some chance};
\apud{Cat.}{1, 6, 16}.

\latin{hōrā quārtā},
\english{at the fourth hour};
\apud{B.~G.}{4, 23, 2}.

\end{examples}

\subsection

Words depending upon a modifier of a noun, or upon a noun accompanied
by a modifier, are generally put between the two, the whole being thus
tied into a single mass (like an algebraic quantity within brackets).
\begin{examples}

\latin{īnfestam reī pūblicae pestem},
\english{a plague dangerous to the state};
\apud{Cat.}{1, 5, 11}.

\latin{duās urbīs huic imperiō īnfestissimās},
\english{two cities most dangerous to this realm};
\apud{Cat.}{4, 10, 21}.

\latin{complūrēs eiusdem āmentiae sociōs},
\english{many associates in the same madness};
\apud{Cat.}{1, 4, 8}.

\end{examples}

\subsection

Appositive nouns and appositive adjectives normally follow their
substantives.
\begin{examples}

\latin{Ennius et sapiēns et fortis et alter Homērus},
\english{Ennius, wise and brave and a second Homer};
\apud{Ep.}{2, 1, 50}.

\end{examples}

\subsection

Vocatives normally stand after one or more words.
\begin{examples}

\latin{quid est, Catilīna?}
\english{How is this, Catiline?}
\apud{Cat.}{1, \emend{248}{5, 13}{6, 12}}.

\end{examples}

\subsection

Interrogative words normally stand first in their clauses.
\begin{examples}

\latin{quem ignōrāre arbitrāris\dots?}
\english{who, think you, is ignorant\dots?}
\apud{Cat.}{1, 1, 1}.

\end{examples}

\subsection

Relative pronouns and conjunctions normally stand first in their
clauses.
\begin{examples}

\latin{proximī sunt Germānīs, quī trāns Rhēnum incolunt},
\english{they are next to the Germans, who live beyond the Rhine};
\apud{B.~G.}{1, 1, 4}.

\latin{sī tē comprehendī iusserō},
\english{if I have you arrested};
\apud{Cat.}{1, 2, 5}.

\end{examples}

\subsubsection

The conjunctions \enclitic{-que} and~\enclitic{-ve}, being enclitics,
cannot stand first.  See~\xref[1, \emph{b}]{307}.

\subsubsection

\latin{Autem}, \latin{enim}, and \latin{vērō} follow the first word or
phrase.  So, generally, does \latin{igitur}, though it sometimes
stands first.  \latin{Tamen} stands either first, or after an emphatic
word.

\subsection

Determinative words referring to something in the preceding sentence
stand, like relatives, at the beginning (first word, or in the first
phrase).
\begin{examples}

\latin{ad eās rēs cōnficiendās Orgetorīx dēligitur.  Is sibi
  lēgātiōnem ad cīvitātīs sus\-cē\-pit.  In eō itinere\dots},
\english{Orgetorix is chosen to carry out these plans.  He
  \emph{\(this man\)} undertook an embassy to the various states.
  Upon this journey\dots};
\apud{B.~G.}{1, 3, 3}.

\end{examples}

\subsection

Relative clauses generally follow the phrase containing the
antecedent; but often they are inserted into that phrase.

\begin{examples}

\latin{ad ea castra quae suprā dēmōnstrāvimus contendent},
\english{hastens to the camp which I have mentioned above};
\apud{B.~G.}{7, 83, 8}.

\latin{ad eās quās dīximus mūnītiōnēs pervēnērunt},
\english{arrived at the fortifications which I have mentioned};
\apud{B.~G.}{3, \emend{249}{26}{24}, 2}.

\end{examples}

\subsubsection

For the relative clause preceding its antecedent, see~\xref[5]{284}.

\subsection

Conditions and conditional relative clauses generally precede the
main clause, or are inserted in it.  They rarely follow.

\subsection

Prepositions regularly precede the words which they govern.

\begin{minor}

\subsubsection

Exceptions occur mainly in poetry, mostly with dissyllabic
prepositions.  Thus \latin{tē propter}, \apud{Aen.}{4, 320}.

\subsubsection

For \latin{mēcum}, \latin{quibuscum}, etc., see~\xref[\emph{a}]{418}.

\subsubsection

For \enclitic{-que} with monosyllabic prepositions,
see~\xref[1, \emph{b}]{307}.

\end{minor}

\subsection

Most adverbs normally stand just before the words they modify.
\begin{examples}

\latin{tam improbus},
\english{so worthless};
\apud{Cat.}{1, 2, 5}.

\end{examples}

\subsubsection

\latin{Quidem}, \latin{quoque}, \latin{dēnique}, and \latin{dēmum}
follow the word they modify.  So, generally, do \latin{ferē},
\latin{fermē}, \latin{paene}, and \latin{prope}; \latin{potius} and
\latin{potissimum}; and \latin{tantum} in the sense of \emph{only}.

\begin{examples}

\latin{aequō ferē spatiō}, \english{at about an equal distance};
\apud{B.~G.}{1, 43, 1}.

\end{examples}

\subsection

\latin{Nōn} regularly stands just before the word it modifies.

\subsection

The first person precedes the other two, and the second the third.
\begin{examples}

\latin{sī tū et Tullia valētis, ego et suāvissimus Cicerō valēmus},
\english{if you and Tullia are well, so are my dear boy and~I} (in
Latin, I and my boy);
\apud{Fam.}{14, 5, 1}.

\end{examples}

\subsection

\latin{Inquam}, \latin{inquit}, etc., stand after one or more of the
quoted words.
\begin{examples}

\latin{“est vērō,” inquam, “nōtum quidem signum,”}
\english{“it is indeed,” said~I, “a well-known seal”};
\apud{Cat.}{3, 5, 10}.

\end{examples}

\chapter{Rhetorical Order}

\section

But the so-called normal arrangement is really rare, since the speaker
or writer generally \emph{has} some special emphasis to put upon some
part of the sentence (\emph{rhetorical order}).

This may be effected:
\begin{enumI}

\item
By reversing the normal order.

\item
By the juxtaposition of like or contrasting words.

\item
By
\versionA{the separation of connected words }%
\versionB*{postponement }%
to produce suspense.

\end{enumI}

\emph{Examples} (contrast those in~\xref[1–7]{624}):
\begin{examples}

\latin{līs haec},
\english{\textsc{this particular} suit};
\apud{Clu.}{41, 116}.

\latin{nōn est ista mea culpa sed temporum},
\english{it is not \textsc{my} fault, but that of the times};
\apud{Cat.}{2, 2, 3}.

\latin{senātus ūniversus iūdicāvit},
\english{the senate judged, \textsc{to a man}};
\apud{Clu.}{49, 136}.

\latin{iacet ille},
\english{he lies \textsc{prostrate}} (prostrate he lies);
\apud{Cat.}{2, 1, 2}.

\latin{latrōnī quae potest īnferrī iniūsta nex?}
\english{\textsc{upon a brigand} what death can be inflicted that is
  not \textsc{deserved}?}
\apud{Mil.}{\emend{132}{5}{4}, 10}.

\latin{nōn est saepius in ūnō homine summa salūs perīclitanda reī
  pūblicae},
\english{it is not right that a \textsc{single} person should
  repeatedly be allowed to endanger the \textsc{highest} welfare of
  the commonwealth};
\apud{Cat.}{1, 5, 11}.

\latin{M.\ Tullī, quid agis?}
\english{\textsc{Marcus Tullius}, what are you doing?}
\apud{Cat.}{1, 11, 27}.

\latin{Q.\ Maximum senem adulēscēns dīlēxī},
\english{I loved Quintus Maximus, in his old age and my youth};
\apud{Sen.}{4, 10}.

\latin{magna dīs immortālibus habenda est grātia},
\english{\textsc{great gratitude} is due to the immortal gods};
\apud{Cat.}{1, 5, 11}.

\end{examples}

\subsubsection

A double emphasis is of course possible.
\begin{examples}

\latin{cupiō mē esse clēmentem},
\english{my \textsc{desire} is to be \textsc{merciful}};
\apud{Cat.}{1, 2, 4}.

\end{examples}

\subsubsection

On the other hand, the putting of a word into an emphatic position
often throws another into an unusual place \emph{without} special
emphasis upon that other.
\begin{examples}

\latin{vīvēs, et vīvēs ita ut vīvis},
\english{you shall live, and live \textsc{in the same way as now}};
\apud{Cat.}{1, 2, 6}.  (\latin{Ita} is emphatic, but the \latin{vīvēs}
immediately preceding it merely repeats the first \latin{vīvēs},
without emphasis.)

\end{examples}

\subsubsection

In the compound tenses, the auxiliary \latin{sum} may, according to
the needs of the sentence, be placed anywhere, without emphasis upon
itself.

\section

An emphatic word is often taken out of a dependent clause and put
before the connective, especially if it belongs in thought to both the
dependent and the main clause.
\begin{examples}

\latin{servī mehercule meī sī mē istō pactō metuerent, domum meam
  relinquendam putārem},
\english{good heavens!  if \textsc{even my slaves} feared \textsc{me}
  in this fashion, I should think I ought to leave my home};
\apud{Cat.}{1, 7, 17}.

\latin{Caesarī cum id nūntiātum esset, mātūrat ab urbe proficīscī},
\english{when this had been announced to Caesar, he made
  \emph{\(makes\)} haste to set out from the city};
\apud{B.~G.}{1, 7, 1}.  Contrast~\apud{}{1, 50, 4}, in which the
emphasis does \emph{not} lie upon the actor.

\end{examples}

\begin{minor}

\subsubsection

Sometimes many words of the dependent clause precede the connective.

\begin{examples}

\latin{per omnia nive opplēta cum sēgniter agmen incēderet},
\english{as the army was marching sluggishly through a country covered
  with snow};
\apud{Liv.}{21, 35, 7}.

\end{examples}

\end{minor}

\section
\subsection

The Romans liked to separate a group of words consisting of a noun and
modifier, by inserting the governing word.  The effect is to throw a
little more emphasis upon the modifier, by leaving it for the moment
in suspense.
\begin{examples}

\latin{eōdem ūsī cōnsiliō},
\english{following the same plan};
\apud{B.~G.}{1, 5, \emend{95}{4}{3}}.

\latin{proptereā quod aliud iter habērent nūllum},
\english{since other way they had \textsc{none}};
    \apud{B.~G.}{1, 7, 3}.
Double emphasis; for \latin{nūllum} is not only put after \latin{iter}
instead of preceding it (\xref[1, \emph{a}]{624}), but is held longer
in suspense by the insertion of \latin{habērent}.

\end{examples}

\subsection

The Romans liked to put pronouns early in a clause, to group them
together, and even to insert them into groups with which they have no
direct connection.
\begin{examples}

\latin{huic ego mē bellō ducem profiteor},
\english{for this war I announce myself as leader};
\apud{Cat.}{2, 5, 11}.

\latin{magnō mē metū līberābis},
\english{you will relieve me of great fear};
\apud{Cat.}{1, 5, 10}.

\end{examples}

\begin{minor}

\subsubsection

In Adjurations, \latin{per} is often separated from its object by a
pronoun.
\begin{examples}

\latin{per ego hās lacrimās tē ōrō},
\english{by these tears I beseech you};
\apud{Aen.}{4, 314}.

\end{examples}

\subsubsection

The groups \latin{suus quisque} and \latin{sibi quisque} always take
this order.

\end{minor}

\subsection

After neuters and adverbs, the Genitive of the Whole is usually held
back for several words.
\begin{examples}

\latin{dīxistī paulum tibi esse etiam nunc morae},
\english{you said that you were still suffering a little delay};
\apud{Cat.}{1, 4, 9}.

\end{examples}

\subsection

An adjective or pronoun belonging to a noun governed by a monosyllabic
preposition is often placed before the preposition.
\begin{examples}

\latin{quem ad finem?}
\english{to what limit?}
\apud{Cat.}{1, 1, 1}.

\latin{magnō cum dolōre},
\english{with great grief};
\apud{Phil.}{1, 12, 31}.

\end{examples}

\section

When two pairs of words are in contrast with each other, the members
may be arranged either in \term{Parallel Order} or in \term{Cross
  Order}.\footnote{Called \emph{chiasmus}, from the Greek letter~Χ, in
  which the lines are crossed.}
\begin{examples}

\latin{puerīlī speciē, sed senīlī prūdentiā},
\english{of boyish appearance, but of an old man’s wisdom};
\apud{Div.}{2, 23, 50}.  (Parallel Order.)

\latin{prō vītā hominis nisi hominis vīta reddātur},
\english{unless for the life of a man a man’s life  be paid};
\apud{B.~G.}{6, 16, \emend{250}{3}{2}}. (Cross Order.)

\end{examples}

\section

In English the general tendency is to \emph{complete the thought}, as
far as possible, as each part of the sentence is spoken or written.

In Latin, on the contrary, the general tendency is to hold first one
thing and then another \emph{in temporary suspense} as the sentence moves
from part to part.\footnote{It is all-important to bear this in mind
  in reading.  The student should remember that the \emph{chances} are
  that a given word, phrase, or clause is not explained by anything he
  has yet reached, but by \emph{something that is yet to come}.}
Accordingly,

\subsection

Most kinds of clauses normally precede that which they modify.
\begin{examples}

\latin{Alcō, precibus aliquid mōtūrum ratus, cum ad Hannibalem noctū
  trānsīsset, post\-quam nihil lacrimae movēbant, apud hostem mānsit},
\english{Alco, thinking that he could accomplish something by
  entreaties, after going to Hannibal by night, and finding that tears
  did not move him, remained with the enemy};
\apud{Liv.}{21, 12, 4}.
\end{examples}

\subsubsection

But when two clauses \emph{of a different character} modify the same
verb, one generally precedes this, and the other follows it.
\begin{examples}

\latin{hīs cum suā sponte persuādēre nōn possent, lēgātōs ad
  Dumnorīgem mittunt, ut eō dēprecātōre impetrārent},
\english{when they found themselves unable to persuade these people
  by their own influence, they sent \emph{\(send\)} ambassadors to
  Dumnorix, in order to obtain their wish through his mediation};
\apud{B.~G.}{1, 9, 2}.

\end{examples}

\subsection

Substantive and consecutive clauses normally follow the word on which
they depend.
\begin{examples}

\latin{persuāsit ut exīrent},
\english{persuaded them to emigrate};
\apud{B.~G.}{1, 2, 1}.

\latin{hīs rēbus fiēbat ut\dots},
\english{the result was, that\dots};
\apud{B.~G.}{1, 2, 4}.

\end{examples}

\section

A carefully constructed sentence of some length, with suspense kept up
until the end, is called a \term{Period}, and the style is called the
\term{Periodic Style}.  See, for example, the sentence \latin{Alcō},
etc., \xref[1]{629}; \latin{Caesar—ūtī possent},
\apud{B.~G.}{2, \emend{99}{25}{24}, 1–2};
and the first two sentences of \apud{Cat.}{3, 1}.

\begin{minor}

\subsubsection

Such a sentence generally requires to be broken up into two or more
sentences in English.

\end{minor}

\headingB{Figures of Syntax and Rhetoric\footnotemark}
\footnotetext{A sharp distinction between the two classes is often
  impossible.}

\setcounter{chapter}{0}

\numchapter*{Figures of Syntax}

\contentsentry{B}{Figures of Syntax}

\section
\subsection

\term{Ellípsis} is the omission of one or more words.
\begin{examples}

\latin{Aeolus haec contrā},
\english{thus Aeolus \(spoke\) in reply};
\apud{Aen.}{1, 76}.

\end{examples}

\begin{minor}

\subsubsection

The words most commonly omitted are \latin{dīcō}, \latin{loquor},
\latin{agō}, \latin{faciō}.  See example under~\xref[\emph{a}]{222}.

\end{minor}

\subsection

\term{Brachýlogy} is brevity of expression.
\begin{examples}

\latin{vir bonus dīcī dēlector ego ac tū (dēlectāris)},
\english{I like to be called a good man, just as you \(do\)};
\apud{Ep.}{1, 16, 32}.

\end{examples}

\subsection

\term{Condensed Comparison} is a form of brachylogy in which a thing
is compared with a characteristic, or a characteristic with a thing.
\begin{examples}

\latin{hārum est cōnsimilis caprīs figūra},
\english{their shape is like \(that of\) goats};
\apud{B.~G.}{6, 27, 1}.

\end{examples}

\subsection

\term{Pléonasm} is the use of unnecessary words.
\begin{examples}

\latin{sīc ōre locūta est},
\english{thus she spoke with her lips};
\apud{Aen.}{1, 614}.

\end{examples}

\subsection

\term{Hendíadys}\footnote{\,Ἓν διὰ δυοῖν, “one thing through two.”} is
the expression of one complex idea through the use of two nouns
connected by a conjunction.
\begin{examples}

\latin{mōlem et montīs altōs},
\english{a mass of lofty moutains};
\apud{Aen.}{1, 61}.

\end{examples}

\subsection

\term{Sýnesis} (“sense”) is construction according to sense, not
according to form. (See~\xref{325}.)
\begin{examples}

\latin{pars in fugam effūsī},
\english{a part were scattered in flight};
\apud{Liv.}{27, 1, 12}.

\end{examples}

\subsection

\term{Zeúgma} (“joining”) is the government of two words by a word
which strictly applies to only one of them.
\begin{examples}

\latin{Danaōs et laxat claustra Sinōn},
\english{Sinon unbars the doors and \(sets free\) the Greeks};
\apud{Aen.}{2, 258}.

\end{examples}

\subsection

\latin{Anacolúthon} (“lack of sequence”) is a change of construction
in a sentence, by which the first part is left without government.
\begin{examples}

\latin{nōs omnēs, quibus est aliquis obiectus labōs, omne quod est
  intereā tempus lucrōst},
\english{all of us before whom trouble lies,—\emph{\(for us\)} the
  time between is gain};
\apud{Hec.}{\emend{51}{286}{287}}.
(The nominative construction is not followed out.)

\end{examples}

\subsection

\term{Enállage} is the exchange of one part of speech for another, or
of one gender, number, etc., for another.
\begin{examples}

\latin{populum lātē rēgem},
\english{a people sovereign far and wide} (\latin{rēgem} for
\latin{rēgnantem});
\apud{Aen.}{1, 21}.

\end{examples}

\subsection

\term{Hypállage} is an exchange of grammatical relations.

\latin{dare classibus austrōs},
\english{to give the winds to the fleet} (instead of \english{give the
  fleet to the winds});
\apud{Aen.}{3, 61}.

\subsection

\term{Prolépsis}\footnote{Πρόληψισ, “taking in advance.”} is the use
of a word in advance of that which explains it.
\begin{examples}

\latin{submersās obrue puppīs},
(o’erwhelm the sunken ships)
\english{o’erwhelm the ships so that they sink};
\apud{Aen.}{1, 69}.

\end{examples}

\subsection

\term{Hýsteron Próteron}\footnote{\,Ὕστερον πρότερον, “the last
  first.”} is the reversing of the logical order.
\begin{examples}

\latin{moriāmur et in media arma ruāmus},
\english{let us die and rush into the midst of arms};
\apud{Aen.}{2, 353}.

\end{examples}

\subsection

\term{Hypérbaton} is a change in the natural order of words.
\begin{examples}

\latin{per omnīs tē deōs ōro},
\english{I pray you by all the gods};
\apud{Carm.}{1, 8, 1}.

\end{examples}

\subsection

\latin{Anástrophe} (“turning around”) is the placing of a
preposition after its case.  See~\xref[12, \emph{a}]{624}.

\subsection

\term{Tmésis} (“cutting”) is the separating of the parts of a
compound word.
\begin{examples}

\latin{quae mē cumque vocant terrae},
\english{what lands soever bid me come};
\apud{Aen.}{1, 610}.

\end{examples}

\numchapter*{Figures of Rhetoric}

\contentsentry{B}{Figures of Rhetoric}

\section
\subsection

\term{Lítotes} is the rhetorical softening of an expression by the
denial of the opposite idea.  The effect is increased emphasis.
\begin{examples}

\latin{nōn ignāra malī},
\english{not ignorant of suffering};
\apud{Aen.}{1, 630}.

\end{examples}

\subsection

\latin{Hypérbole} is exaggeration.
\begin{examples}

\latin{ventīs ōcior},
\english{swifter than the winds};
\apud{Aen.}{5, 319}.

\end{examples}

\subsection

\term{Oxymóron} is the putting together of two apparently
contradictory ideas.
\begin{examples}

\latin{īnsānientis sapientiae},
\english{of a mad wisdom};
\apud{Carm.}{1, 34, 2}.

\end{examples}

\subsection

\term{Irony} is the intentional saying of the opposite of what is
really meant.
\begin{examples}

\latin{bone custōs},
\english{excellent guardian} (for \english{bad guardian});
\apud{Ph.}{287}.

\end{examples}

\subsection

\term{Anáphora} is the use of the same or closely similar words in the
same place in successive clauses.
\begin{examples}

\latin{tū flectis amnīs, tē vīdit īnsōns Cerberus},
\english{thou turnest torrents from their course, on thee Cerberus
  looked and did no harm};
\apud{Carm.}{2, 19, 17}.

\end{examples}

\subsection

\term{Chiásmus} is the arranging of pairs of words in the opposite
order.  See example in~\xref{628}.

\subsection

\term{Antíthesis} is the setting of contrasting things against each
other.
\begin{examples}

\latin{speciē blanda, reāpse repudianda},
\english{in aspect charming, in reality objectionable};
\apud{Am.}{13, 47}.

\end{examples}

\subsection

\term{Synécdoche} is the use of a part for the whole.
\begin{examples}

\latin{mūcrōne coruscō},
\english{with flashing sword} (strictly \english{point});
\apud{Aen.}{2, 333}.

\end{examples}

\subsection

\term{Metónymy} (“shift of name”) is the use of a name in place of
another to which it is related.
\begin{examples}

\latin{furit Volcānus},
\english{Vulcan \emph{\(i.e.\ the fire\)} rages};
\apud{Aen.}{5, 662}.

\latin{tremit puppis},
\english{the stern \emph{\(i.e.\ the ship\)} trembles};
\apud{Aen.}{5, 198}. (Part for the whole.)

\latin{aere},
\english{with the bronze} (i.e.\ with the bronze prow);
\apud{Aen.}{1, 35}. (Material for the thing made of it.)

\end{examples}

\subsection

A \term{Transferred Epithet} is an epithet not strictly belonging to
that to\linebreak which it is attached, but transferred from something connected
with this in thought.
\begin{examples}

\latin{mare vēlivolum},
\english{the sail-flying sea} (for sail-covered);
\apud{Aen.}{1, 224} (“sail-flying” really applies to the ships, not
to the sea).

\end{examples}

\subsection

\term{Climax} (“a ladder”) is a steady rise of force\versionA*{ till
  the end of the sentence is reached}.

\begin{examples}

\latin{nihil agis, nihil mōlīris, nihil cōgitās, quod nōn ego nōn modo
  audiam sed etiam videam plānēque sentiam},
\english{you do nothing, you \textsc{attempt} nothing, you
  \textsc{\large think} of nothing, that I fail, I will not merely say to
  hear of, but even to \textsc{see} and to \textsc{\Large understand
    completely}};
\apud{Cat.}{1, 3, 8}.

\end{examples}

\subsection

\term{Eúphemism} is the use of a less disagreeable expression in place
of a more disagreeable one.
\begin{examples}

\latin{sī quid accidat Rōmānīs},
\english{if anything}\versionA*{ (for \english{any disaster})} \english{should
  happen to the Romans}\versionB*{ (instead
  of \english{if they should be defeated})}\unskip;
\apud{B.~G.}{1, 18, 9}.

\end{examples}

\subsection

\term{Métaphor} is the figurative use of words.
\begin{examples}

\latin{sentīna reī pūblicae},
\english{the dregs of the state};
\apud{Cat.}{1, 5, 12}.

\end{examples}

\subsection

\term{Állegory} is continued metaphor.
\begin{examples}

\latin{ō nāvis, referent in mare tē novī flūctūs\dots; fortiter occupā
  portum},
\english{O ship, yet other billows will carry thee out to sea\dots; be
  brave and make the port};
\apud{Carm.}{1, 14, 1}\versionB{.}%
\versionA*{ (\apud{Quintilian}{8, 6, 44}, explains that the ship is the
  state, the billows the civil wars, and the port peace and harmony).}%
\versionB{ (The ship is the state, the billows the civil wars,
  etc.)}

\end{examples}

\subsection

\term{Símile} is illustration by comparison.
\begin{examples}

\latin{ac velutī magnō in populō cum coorta est sēditiō, gravem sī
  forte virum quem cōnspexēre, silent, sīc pelagī cecidit fragor},
\english{and as, when a riot has broken out among a great rabble, if
  they chance to see some man of weight, they are hushed, so ceased
  the tumult of the waters};
\apud{Aen.}{1, 148}.

\end{examples}

\subsection

\term{Aposiopésis} (“silence”) is a breaking off in a sentence.
\begin{examples}

\latin{quōs ego—, sed mōtōs praestat compōnere flūctūs},
\english{whom I—, but it is better to calm the angry waves};
\apud{Aen.}{1, 135}.

\end{examples}

\subsection

\term{Apóstrophe} is an impassioned turning aside from the previous
form of thought, to address some person or thing.
\begin{examples}

\latin{citae Mettum in dīversa quadrīgae distulerant (at tū dictīs,
  Albāne, manērēs)},
\english{the swift chariots had torn Mettus asunder \(but thou, O
  Alban, shouldst have kept thy word\)};
\apud{Aen.}{8, 643}.

\end{examples}

\subsection

\term{Personification} is the treating of inanimate things as persons.
\begin{examples}

\latin{haec sī tēcum patria loquātur},
\english{if your country should thus plead with you};
\apud{Cat.}{1, 8, 19}.

\end{examples}

\subsection

\term{Alliteration} is the repetition of single sounds, generally
consonants.
\begin{examples}

\latin{vī victa vīs},
\english{force has been foiled by force};
\apud{Mil.}{11, 30}.

\end{examples}

\subsection

\term{Onomatop\'œia} is the
\versionA*{use of words the sound of which corresponds with the thing
  signified.}
\versionB{matching of sound to sense.}
\begin{examples}

\latin{magnō cum murmure montis},
\english{with a mighty murmuring of the mountain};
\apud{Aen.}{1, 55}.

\end{examples}

\versionB*{\subsection

The \term{Figūra Etymologica} combines words of kindred origin but
different meanings.
\begin{examples}

\latin{sēnsim sine sēnsū},
\english{gradually and imperceptibly};
\apud{Sen.}{11, 38}.

\end{examples}}

\part{Versification}

\contentsentry{B}{Rhythm; Ictus; the Foot; the Verse}

\section

\term{Rhythm} is the regular recurrence of sound-groups that take the
same amount of time (quantity\footnote{In Latin, as in languages
  spoken to-day, the poet, using in the main the pronunciation of daily
  speech, so arranged his words that, for any reader, they made
  rhythm.

  Quantity is accordingly not a matter of verse alone, but a matter of
  \emph{Pronunciation} in general, and is so treated in this grammar
  (\xref{16}–\xref{40}).})\emend{52}{}{.}

\section

\term{Ictus} (from Latin \latin{ictus}, \english{a blow}) is the
natural \emph{stress} or \emph{pulse-beat} which, whenever there is
such a regular recurrence of groups of sound, is given to the same
place in each group.

\begin{minor}

\subsubsection

Ictus is simply \emph{stress of voice}.  It does not differ in
character from word-accent or sense-stress, but is due to a different
cause.

\end{minor}

\section

A rhythmical sound-group is called a \term{Foot}.

\section

A succession of feet arranged according to a fixed scheme is called a
\term{Verse}.

\section

The two kinds of feet which the student meets in his earlier reading
in Latin are:
\begin{examples}

The \term{Dactyl}, or \scan{lss} (\music{HQQ}\kern.8pt), as in \latin{dēsuper}.

The \term{Spondee}, or \scan{ll} (\music{HH}\kern.8pt), as in \latin{īrae}.

\end{examples}

\subsubsection

These two kinds of feet take the same time in pronunciation (namely
four units);\footnote{\label{ftn:344:2}The shortest unit of
  pronunciation is technically called \latin{mora}, \english{delay}.
  The ancient Roman grammarians tell us that a long syllable contained
  two \latin{morae}, and this statement is consistent with what we
  find in Latin poetry. The same of course holds, in a general way,
  for prose, though the proportion must have been less exact.} for the
two short syllables in the Dactyl, \emph{together}, occupy as much
time in pronunciation as the long syllable.  In beating time,
accordingly, one would give four beats to either of these feet.

\begin{minor}

\subsubsection

Two other feet of which the student will need to know the names early
are the \term{Trochee}, or~\scan{ls}, as in \latin{inde} or
\latin{prīmus}, and the \latin{Iambus}, or~\scan{sl}, as \latin{amō}
or \latin{dolēns}.  In beating time one would give three beats to
either of these feet.

\end{minor}

\section

The word \term{Metre} strictly means a \emph{measure} in the
composition of a verse.  But it is more generally used for a
\emph{kind} of metrical system, whatever this may be.  Thus we might
say of a given system “this metre is dactylic.”

\begin{minor}

\subsubsection

The two kinds of metre which the student meets in his earlier reading
in Latin are the Dactylic Hexameter and the Dactylic Pentameter.

\end{minor}

\headingC{The Dactylic Hexameter}

\contentsentry{B}{The Dactylic Hexameter and Dactylic Pentameter}

\section

The \term{Dactylic Hexameter} is made up of six Dactyls or Spondees.
\begin{examples}

\latin{multā ∣ mōle do∣cendus a∣prīcō ∣ parcere ∣ prātō};
\apud{Ep.}{1, 14, 30}.

\end{examples}

\subsubsection

The last foot \emph{must be} a Spondee.  The fifth foot
\emph{generally} is a Dactyl.  The other feet may be either Dactyls or
Spondees.

The length of the final syllable of the verse is of no
consequence,\footnote{The last foot, therefore, though it is
  convenient to call it a Spondee, will often be made up of a long
  syllable plus a short (\scan{ls}), i.e.\ will strictly be a
  Trochee.} since there is regularly a slight pause at the end
(see~\xref[n.~3]{641}).

The scheme may be thus indicated (the second form showing the
relative length of the syllables in musical notation):
\begin{Tabular}{l@{\enskip}l@{}l|l@{}l|l@{}l|l@{}l|l@{}l|l@{}l}

& \scan{l} & \scan{C}
& \scan{l} & \scan{C}
& \scan{l} & \scan{C}
& \scan{l} & \scan{C}
& \scan{l} & $\mathop{\text{\scan{ss}}}\limits\sp{\text{(\scan{l})}}$
& \scan{l} & \scan{c}
\\

\multicolumn{11}{c}{}\\[-\medskipamount]

& \music{H} & \music{QQ}
& \music{H} & \music{QQ}
& \music{H} & \music{QQ}
& \music{H} & \music{QQ}
& \music{H} & \music{QQ}
& \music{H} & \music{H}
\\[1pt]

or
& \music{h} & \music{h}
& \music{h} & \music{h}
& \music{h} & \music{h}
& \music{h} & \music{h}
& \llap{(\kern.6pt}\music{h} & \music{h}\rlap)
& \music{h} & \music{q}

\end{Tabular}

\begin{minor}

Observe that there are \emph{four} beats to the measure, not, as in
the English hexameter, three.

\end{minor}

\begin{minor}

\subsubsection

Verses with a spondee in the fifth foot (“spondaic verses”) are
rare.
\begin{examples}

\latin{cōnstitit ∣ atque \slur\ ocu∣līs Phrygi∣a \slur\ agmina ∣
  circum∣spexit}; \apud{Aen.}{2, 68}.

\end{examples}

\end{minor}

\subsubsection

\term{Variety of Effect} is produced by the more skilful poets (in
this respect Virgil is first) by varying the proportion of dactyls to
spondees.  An accumulation of dactyls gives an effect of rapidity of
action, or of excitement of feeling; while an accumulation of spondees
gives the effect of slow or difficult motion, of depression, of fear,
etc., etc.  Examples of extreme cases follow, the first describing the
swift galloping of horses, the second the fearful aspect of the
monster Polyphemus:
\begin{examples}

\latin{Quadrupe∣dante pu∣trem soni∣tū quatit ∣ ungula ∣ campum};
\apud{Aen.}{8, 596}.

\latin{Mōnstrum \slur\ hor∣rendum, \slur\ īn∣fōrme \slur\ in∣gēns, cui
  ∣ lūmen ad∣ēmptum}; \apud{Aen.}{3, 658}.

\end{examples}

\subsubsection

The best poets aim not to let many words end with the end of a foot.
But in the fifth foot this is not avoided.
\begin{examples}

\latin{urbs an∣tīqua fu∣it, Tyri∣ī tenu∣ēre co∣lōnī};
\apud{Aen.}{1, 13}.

\end{examples}

\section
\subsection

\term{Caesura} (“cutting”) is the ending of a word \emph{before} the
end of the foot.

\begin{minor}

\subsubsection

The word which thus cuts the foot by its ending may be of any length;
see \latin{urbs}, \latin{fuit}, \latin{antīqua}, and \latin{tenūere}
in the verse above.

There \emph{may} be a caesura in every foot, as in the verse above.

\end{minor}

\subsection

\term{Diaresis} (“dividing”) is the ending of a word \emph{with} the
end of the foot (marked~\#).  Thus in the first foot of
\begin{examples}

\latin{et soror \# et con∣iūnx, ū∣nā cum ∣ gente tot ∣ annōs};
\apud{Aen.}{1, 47}.

\end{examples}

\begin{minor}

\subsubsection

Diaeresis is thus the opposite of Caesura.

\end{minor}

\section

The \term{Principal Caesura} (marked~∥)\emend{53}{}{,} commonly called
simply \emph{the} Cae\-su\-ra, is a caesura which falls at a natural pause
in the verse, not far from the middle.

This natural pause may be for the sake of the sense as well as the
sound, or merely for the sound (i.e.\ for an agreeable breaking of the
long verse into parts).\footnote{Cf.\ the following verses from
  Longfellow’s \emph{Evangeline}, Part~I.  In the first, the caesura
  is for the sense as well as the sound.  In the second it is for the
  sound only.
\begin{verse}
Columns of pale blue smoke, ∥ like clouds of incense arising.\\ Sweet
was her breath as the breath ∥ of kine that feed in the meadows.
\end{verse}\baselineskip0pt}

\begin{minor}

\subsubsection

The Caesura is called \term{Masculine}, when it falls after the first
syllable of the foot, \term{Feminine} (from the softer effect), when
it falls after the second syllable of the foot.  See the principal
caesuras under~\emph{b}, below.

\subsubsection

The Principal Caesura is generally in the third
foot,\footnote{Technically called \emph{penthemimeral},
  i.e. \emph{after the fifth half}.}  less frequently in the
fourth.\footnote{Technically called \emph{hephthemimeral},
  i.e. \emph{after the seventh half}.}

\emph{In the Third Foot}:
\begin{center}

\latin{turbine ∣ corripu∣it ∥ scopu∣lōque \slur\ īn∣fīxit a∣cūtō};
\apud{Aen.}{1, 45}.\\

(The caesura here is masculine.)\\[\medskipamount]

\latin{ō pas∣sī gravi∣ōra ∥ da∣bit deus ∣ hīs quoque ∣ fīnem};
\apud{Aen.}{1, 199}.

(The caesura here is feminine.)

\end{center}

\emph{In the Fourth Foot}:

\begin{center}

\latin{Tȳdī∣dē, mē∣ne \slur\ Īlia∣cīs ∥ oc∣cumbere ∣ campīs};
\apud{Aen.}{1, 97}.\\

(The caesura here is masculine.)

\end{center}

\subsubsection

Sometimes there are two or even three Caesuras.  And it may be
impossible to say which is the most important one.
\begin{examples}

\latin{exper∣tī; ∥ revo∣cāte \slur\ ani∣mōs, ∥ maes∣tumque
  ti∣mōrem}; \apud{Aen.}{1, 202}.

\latin{īnsig∣nem ∥ pie∣tāte ∥ vir∣rum ∥ tot ad∣īre la∣bōrēs};
\apud{Aen.}{1, 10}.

\end{examples}

\begin{note}[Note 1]

In order not to leave the parts of the verse unbalanced, a caesura in
the fourth foot is often accompanied by another in the second
foot,\footnote{Technically called \emph{trithemimeral},
  i.e.\ \emph{after the third half}.}  as above, or by a diaeresis,
with natural pause of sense,\footnote{Though the words Caesura and
  Diaeresis \emph{may} apply to any foot (see~\xref[1, \emph{a})]{640},
  they are ordinarily used of verse-pauses only, as in the present
  section.} in the first or second foot, as in
\begin{examples}

\latin{ast ego \# quae dī∣vum \slur\ incē∣dō ∥ rē∣gīna Io∣visque};
\apud{Aen.}{1, 46}.

\latin{in pup∣pim ferit; \# excuti∣tur, ∥ prō∣nusque ma∣gister};
\apud{Aen.}{1, 115}.

\end{examples}

\end{note}

\begin{note}[Note 2]

When a diaeresis with sense-pause falls at the end of the fourth foot,
it is called the \term{Bucolic Diaeresis}.\footnote{Because especially
used by the bucolic (i.e.\ pastoral) poets.}
\begin{examples}

\latin{dīc mihi, ∣ Dāmoe∣tā, ∥ cu∣ium\footnote{The first syllable of
    \latin{cuium} is long, though the vowel is not.  (Pronounce
    \latin{cui-ium}; see \xref[\break 2, \emph{a}]{29}.)} pecus? \# An
  Meli∣boeī?}  \apud{Ecl.}{3, 1}.

\end{examples}

\end{note}

\begin{note}[Note 3]

The Romans regularly made a slight pause at the end of a verse, as is
shown by the fact that a vowel in that place was ordinarily not
slurred (\xref{646}) into an initial vowel in the next verse.
\begin{examples}

\latin{Carthā∣gō \slur\ Ītali∣am con∣trā ∥ Tibe∣rīnaque ∣ longē}

\latin{Ōstia, ∣ dīves o∣pum ∥ studi∣īsque \slur\ as∣perrima ∣ bellī};
\apud{Aen.}{1, 12 and 13}.

\end{examples}

\end{note}

\begin{note}[Note 4]

\term{Hypermetric} (i.e.\ over-measure) \term{Verses}. Occasionally a
poet puts an \emph{extra syllable} at the end of a verse, slurring it
into a vowel beginning the next verse.  The slurring is in this case
called \term{Synapheia} (“\emph{joining}”).

\begin{examples}

\latin{iactē∣mur, doce∣ās: ∥ ig∣nārī \slur\ homi∣numque
  lo∣cōrumque \slur\ }

\latin{\slur errā∣mus}; \apud{Aen.}{1, 332}.

\end{examples}

\end{note}

\end{minor}

\headingC{The Dactylic Pentameter}

\section

The \term{Dactylic Pentameter}\footnote{The name, which is ancient, is wrong.
  The verse is really a twice-clipped Hexameter.} is an hexameter with
a pause replacing the second long syllable of the spondee in the third
and sixth feet.

\subsubsection

The Pentameter is regularly used in alternation with the Hexameter.
The two together form the \term{Elegiac
  Stanza}.\footnote{\label{ftn:s642:6}Also called Elegiac Distich
  (“distich” means “\emph{containing two verses}”).}

\subsubsection

In the first half of the Dactylic Pentameter, spondees may be used in
place of Dactyls.  In the second, only Dactyls are possible.

\subsubsection

The first half always ends with a long syllable, and this syllable
always ends a word.

\subsubsection

The scheme of the Elegiac Stanza is therefore as follows:
\begin{examples}
$\left\{
    \begin{tabular}[c]{@{}ll@{}l@{}l@{}l@{}r@{}l@{}l@{}}
                    &           &         &          &          &(\scan{l})\\
        Hexameter:  & \scan{lC|}&\scan{lC}&\scan{|lC}&\scan{|lC}&\scan{|lss}&\scan{|lc}\\
        Pentameter: & \scan{lC|}&\scan{lC}&\scan{|l^}&\scan{|lss}&\scan{|lss}&\scan{|c^}
    \end{tabular}
\right.$

\medskip

\groupL{%
  \text{Hexameter: \latin{sponte su∣ā car∣men nume∣rōs veni∣ēbat ad ∣ aptōs}}\\
  \text{Pentameter: \latin{et quod ∣ temptā∣bam ^ ∣ scrībere ∣ versus e∣rat ^};}\\
   \text{\apud{Ov.\ Trist.}{4, 10, 25–26}.}}

\end{examples}

\begin{minor}

\subsubsection

Variety of effect is sought, and division of words between feet is
made, in the Pentameter, as in the Hexameter (\xref[\emph{c},
  \emph{d}]{639}).

\subsubsection

In Ovid, the last word of the Pentameter is generally one of two
syllables.

\subsubsection

In Ovid, the sense is usually complete at the end of each stanza.

\end{minor}

\section

\term{Scanning} is the dividing of a verse into feet in reading, without
reference to word-accent or sense, as in~\xref[1]{645}.

\headingC{Relation of Ictus to Accent}

\contentsentry{B}{Relation of Ictus to Word-Accent}

\section

The writers of the Dactylic Hexameter generally made accent and ictus
fall together in the last two feet, as in \latin{c\stress{on}deret ∣
  \stress{u}rbem}; \apud{Aen.}{1, 5}.

\begin{minor}

\subsubsection

A monosyllabic ending like \latin{prae∣r\stress{u}ptus \stress{a}∣quae
  m\stress{ō}ns}, \apud{Aen.}{1, 105}, is rare, and is meant always to
produce an unexpected and striking effect.

\end{minor}

\section

With regard to the Roman way of reading the feet in which the ictus
fell upon syllables that did not have the accent, there are two
opinions, and consequently two systems of reading.

\subsection

\textbf{First System}.  When accent and ictus fell upon different
syllables, the former was completely lost.  Thus, in the two following
verses from Ennius and Horace, the words ordinarily pronounced
\latin{ant\stress{ī}quīs}, \latin{aust\stress{ē}rum}, and
\latin{st\stress{u}diō} are, upon this system, to be pronounced
\latin{\stress{a}ntīqu\'īs}, \latin{\stress{au}stēr\stress{u}m}, and
\latin{studi\stress{ō}}:
\begin{examples}

\latin{m\stress{ō}ribus ∣ \stress{a}ntī∣qu\stress{ī}s rēs ∣ st\stress{a}t
  Rō∣m\stress{ā}na vi∣r\stress{ī}sque}; \apud{Enn.\ Ann.}{425}.

\latin{m\stress{o}lliter ∣ \stress{au}stē∣r\stress{u}m studi∣\stress{ō}
  fal∣l\stress{e}nte la∣b\stress{ō}rem}; \apud{Sat.}{2, 2, 12}.

\end{examples}

\begin{minor}

\subsubsection

This system of reading (“scanning”) was until recently almost
universal, and is still the one generally used.

\end{minor}

\subsection

\textbf{Second System}.\footnote{The one preferred by the authors of
  this grammar.}  When accent and ictus fell upon different syllables,
both were heard, the latter being, however, the lighter of the two, so
that the essential character of the word was not changed.

\subsubsection

Similarly \emph{sense-stress} may fall upon a syllable that does not
have the ictus.

\subsubsection

In the following examples, ictus is represented by a circle (or, if
lighter, by a point), while accent and sense-stress are represented by
dashes (thus \stress{} or \stress*{}, the shorter ones indicating lighter
stress).  Where ictus and accent fall together, only one sign is used.

\begin{examples}

\latin{m\ictus{ō}ribus ∣ \ictus*{an}t\stress{ī}∣q\ictus*{u}īs
  r\stress*{ē}s ∣ st\ictus*{a}t Rō∣m\ictus{ā}na vi∣r\ictus{ī}sque};
\apud{Enn.\ Ann.}{425}.

\latin{m\ictus{o}lliter ∣ \ictus*{au}st\stress{ē}∣r\ictus*{u}m
  st\stress{u}di∣\ictus*{ō} fal∣l\ictus{e}nte la∣b\ictus{ō}rem};
\apud{Sat.}{2, 2, 12}.

\end{examples}

\subsubsection

The effect of this separation of accent (as well as of sense-stress)
from ictus may be illustrated from modern poetry, in which it is
fairly frequent, and occasions no trouble to any reader.  Examples
will be seen in all but the first, second, and fifth of the following
verses (in these three, accent and ictus fall together):
\begin{examples}

S\ictus{o}mewhat ∣ b\ictus{a}ck from the ∣ v\ictus{i}llage ∣ str\ictus{ee}t,

St\ictus{a}nds ∣ the old-f\/\ictus{a}∣shioned c\ictus{ou}n∣try-s\ictus*{ea}t;

Acr\ictus{o}ss ∣ its \ictus*{a}n∣t\stress*{i}que p\ictus{o}r∣tic\ictus*{o};
% 
% \smallskip
% 
\qquad\qquad Longfellow, \emph{Old Clock on the Stairs}.

\bigskip

\goodbreak

\ictus{O}nly an ∣ \ictus*{un}s\stress{ee}n ∣ pr\ictus{e}sence ∣
f\ictus{i}lled the ∣ \ictus{ai}r;

\smallskip

\qquad\qquad Longfellow, \emph{Hawthorne}.

\bigskip

S\ictus{o} it ∣ \ictus{i}s; yet ∣ l\ictus*{e}t us ∣ s\ictus{i}ng

H\ictus{o}nor ∣ t\ictus*{o} the ∣ \ictus*{o}ld
b\stress{o}w-∣str\ictus*{i}ng;
% 
% \smallskip
% 
\quad\quad
Keats, \emph{Robin Hood}.

\bigskip

W\stress{e}ll h\ictus*{a}th ∣ h\stress*{e} d\ictus*{o}ne ∣ who
h\ictus*{a}th ∣ s\stress{ei}zed h\ictus{a}p∣pin\ictus*{e}ss

\hbox to 220pt{\hfil·\hfil·\hfil·\hfil·\hfil·\hfil·\hfil·}

\vskip3pt

H\stress{e} d\ictus*{o}th ∣ w\stress*{e}ll t\ictus{oo}, ∣ who
k\ictus{ee}ps ∣ th\stress{a}t l\ictus{a}w ∣ the m\ictus{i}ld

B\stress{i}rth-g\ictus*{o}d∣dess \ictus*{a}nd ∣ the
\ictus*{au}s∣t\stress{e}re f\/\ictus{a}tes ∣ f\/\stress*{i}rst
g\ictus{a}ve;

\smallskip

\qquad\qquad
Matthew Arnold, \emph{Fragment of an Antigone}.

\end{examples}

\begin{minor}

\subsubsection

As a practical matter in using this system, it is best at first to
give a \emph{strong} word-accent, and to try to \emph{avoid} giving
verse-ictus.  Our mental constitution being what it is, a light
verse-pulse (as upon “and” in the last verse from Arnold) will
almost inevitably be given; and this is all that ought ever to be
given in such a case.

If the pronunciation is truly quantitative (see~\xref{36}, \xref{37}),
it will be comparatively easy to keep word-accent as in prose.  To
this end, it will be a help to the student to read \emph{slowly} and
\emph{very tranquilly}, until he has become familiar with the flow of
the verse.

\end{minor}

\chapter{Pronunciations to Be Noticed, though Not
Peculiar to Poetry}

\contentsentry{B}{Pronunciations to be \emend{73}{n}{N}oticed}

\section

\textbf{Slurring}.\footnote{\label{ftn:350:1}Technically called by the
  Greeks and Romans \term{Synaloepha}, or \english{smearing together}.
  The word \term{Elision} (\latin{Ēlīsiō}) is used only by the
  \emph{later} Roman Grammarians.}  As in daily speech (\xref[1]{34}),
a final vowel or diphthong followed by a word beginning with a vowel
or~\phone{h} was slurred or \emph{run into} the vowel of the following
word.\footnote{The final vowel, or vowel with~\phone{m}, was
  \emph{not} cut out.}

This was done so completely that no appreciable extra time was taken,
even in the case of a long vowel or diphthong.  Only the
\emph{quality} of the sound was clearly heard.  The resulting
\emph{quantity} was entirely that of the initial vowel of the
following word.

\section

\term{Hiatus} (“\emph{having the mouth open}”) is the opposite of
slurring, i.e.\ it is the giving of a vowel sound \emph{in full} at
the end of a word, before an initial vowel or~\phone{h}.  (It may be
marked thus:~\hiatus.)

\subsection

It is regularly used in the case of the Interjections \latin{ō},
\latin{āh}, \latin{heu}, \latin{prō}.
\begin{examples}

\latin{ō pater, ∣ \b{ō} \hiatus\ homi∣num rē∣rumque \slur\ ae∣terna
  po∣testās}; \apud{Aen.}{10, 18}.

\end{examples}

\subsubsection

It is occasionally used in other words after the principal caesura, or
before a stop, or anywhere before Greek words (rarely otherwise).
\begin{examples}

\latin{et vē∣ra \slur\ inces∣sū patu∣it de\b{a}. \hiatus\ ∣ Ille \slur\ ubi
  ∣ mātrem}; \apud{Aen.}{1, 405}.

\latin{quid struit? ∣ aut quā ∣ sp\b{ē} \hiatus\ ini∣mīca \slur\ in ∣
  gente mo∣rātur?} \apud{Aen.}{4, 235}.

\latin{tūne \slur\ il∣le \slur\ Aenē∣ās, quem ∣ Dardani∣\b{ō} \hiatus\
  An∣chīsae}; \apud{Aen.}{1, 617}.

\end{examples}

\section

\term{Semihiatus}, or \term{Half Hiatus}, is the giving of \emph{half}
a long vowel sound (namely a corresponding \emph{short} sound),
instead of slurring completely, at the end of a word before an initial
vowel, or vowel with~\phone{h}.
\begin{examples}

\latin{victor a∣pud rapi∣dum Simo∣enta sub ∣ Īli\b{ŏ} ∣ altō};
\apud{Aen.}{5, 261}.

\end{examples}

\begin{minor}

\subsubsection

\phone{Ae} is the only diphthong that admits Hiatus or Half Hiatus.

\end{minor}

\section[Iambic Shortening]

The poets, especially the comic, satiric, and epigrammatic poets,
often availed themselves of the tendency in popular speech to shorten
a long syllable after an accented short syllable (change of
\scan{śl} to \scan{śs}. See \xref[5, note]{28}).

\begin{examples}

\latin{tū cav\b{ĕ} ∣ nē minu∣ās; tū, ∣ nē ma∣ius faci∣ās id};
\apud{Sat.}{2, 3, 177}.
(\latin{Cavĕ} for \latin{cavē}.)

\end{examples}

\section

\term{Syncope} (“\emph{cutting-out}”) is the omission of a short
unaccented vowel.

\begin{examples}

\latin{excide∣rant ani∣mō; manet ∣ altā ∣ mente re∣postum} (for
\latin{repositum}); \apud{Aen.}{1, 26}.

\end{examples}

\chapter{Pronunciations Peculiar to Poetry}

\section[Unconscious Compression of Syllables of Extra Length]

It often happens that a syllable, besides containing a long vowel,
contains a consonant, or even two consonants, at the end, as in \latin{āc-tus},
\latin{sānc-tus}.  A similar thing may happen at the end of a word
before another beginning with a consonant, as in \latin{deōs Latiō},
\apud{Aen.}{1, 6}.  In daily speech, there was additional length in
such cases.  In verse, there must have been (as in modern verse in
similar cases) an \emph{unconscious compression} of each sound, which
would bring the whole into the time belonging to the syllable in the
regular march of the verse.  This, however, would still leave the
vowel perceptibly different from a short vowel.

\section[Occasional Use of Old-fashioned Pronunciations]

The Roman poet occasionally employed pronunciations which, though once
in regular use, had passed away in daily speech:

\subsection

In place of the pronunciations \latin{mihi}, \latin{tibi},
\latin{sibi}, \latin{ibi}, \latin{ubi}, the old pronunciations
\latin{mihī}, \latin{tibī}, \latin{sibī}, \latin{ibī}, \latin{ubī},
might be used (\xref[3]{28}).
\begin{examples}

\latin{mūsa mi∣\b{ī} cau∣sās memo∣rā, quō ∣ nūmine ∣ laesō};
\apud{Aen.}{1, 8}.

\end{examples}

\subsection

In place of such regular pronunciations as \latin{arat},
\latin{videt}, \latin{erat}, \latin{peteret}, \latin{ferar},
\latin{a\-mor}, etc., the old pronunciations \latin{arāt},
\latin{vidēt}, \latin{erāt},\footnote{Similarly \latin{subiīt},
  \apud{Aen.}{8, 363}, but for a different reason (\xref[3,note]{152}).
  Virgil uses these long forms in \suffix{-t} only in the first
  syllable (“thesis”) of the second, third, or fourth foot.}
\latin{peterēt}, \latin{ferār}, \latin{amōr},\footnote{\latin{Puēr} of
  \apud{Ecl.}{9, 66}, which never had the long~\phone{e} in speech, is
  to be explained by~\xref{654}.} \latin{pātēr}, etc., might be used
(\xref[note]{26})\versionB*{, especially in the
  caesura}.\footnote{\label{ftn:s654:3}This usage is
  technically called Diástole, or “\emph{drawing
    out}.”}\savefootnote
\begin{examples}

\latin{quī tene∣ant, nam \slur\ in∣culta vi∣d\b{ē}t,
  homi∣nēsne fe∣raene}; \apud{Aen.}{1, 308}.

\latin{Pergama ∣ cum pete∣r\b{ē}t
  in∣conces∣sōsque \slur\ hyme∣naeōs}; \apud{Aen.}{1, 651}.

\latin{et dīs ∣ cāra fe∣r\b{ā}r et ∣ vertice ∣ sīdera
  ∣ tangam}; \apud{Met.}{7, 61}.

\latin{omnia ∣ vincit A∣m\b{ō}r: et ∣ nōs cē∣dāmus A∣mōrī};
\apud{Ecl.}{10, 69}.

\end{examples}

\subsection

In the Third Person Plural of the Perfect Indicative Active an old
penult with short~\phone{e} (\suffix{-ĕrunt}) is occasionally used
by the poets.\footnote{\label{ftn:s652:4}Technically called
  “Systole,” or “\emph{drawing together},” i.e.\ shortening.}
\begin{examples}

\latin{obstipu∣ī, stet\b{ĕ}∣runtque co∣mae \slur\ et vōx ∣
  faucibus ∣ haesit}; \apud{Aen.}{2, 774}.

\end{examples}

\section[Employment of Pronunciations Coming into Use in Daily Speech]

Common speech tended to shorten the~\phone{i} before~\suffix{-us} in
Pronominal Genitives (\xref[\break note]{21}).  The poets sometimes take
advantage of this pronunciation.
\begin{examples}

\latin{ūn\b{ĭ}us ∣ ob nox∣am \slur\ et furi∣ās A∣iācis
  O∣īl\t{e}{i}}; \apud{Aen.}{1, 41}.

\end{examples}

\section[Lengthening of Syllables Short in Daily Speech\savedfootnote]

In the first place
(“thesis”)\footnote{\label{ftn:351:5}\label{ftn:s654:5}The accented
  part of the foot.  The remainder is called the “arsis.”} of any
foot, a syllable which had never regularly been long in daily speech
might be lengthened.\footnote{Most of the syllables so lengthened come
  before a natural pause, generally the caesura.}  This happens
especially with the enclitic \enclitic{-que}, and the endings
\ending{-a}, \ending{-er}, \ending{-is}, \ending{-us},
and~\ending{-ur}.\footnote{Occasionally also with \ending{-ul},
  \ending{-ut}, \ending{-it}, as in
\latin{procūl}, \apud{Aen.}{8, 98};
\latin{capūt}, \apud{}{10, 394};
\latin{facīt}, \apud{Ecl.}{7, 23}.}
\begin{examples}

\latin{līmina∣qu\b{ē} lau∣rusque de∣ī, tō∣tusque mo∣vērī};
\apud{Aen.}{3, 91}.

\latin{dōna de∣hinc au∣rō gravi∣\b{ā} sec∣tōque \slur\ ele∣phantō};
\apud{Aen.}{3, 464}.

\latin{per ter∣ram, \slur\ et ver∣sā pul∣v\b{ī}s īn∣scrībitur
  ∣ hastā}; \apud{Aen.}{1, 478}.

\latin{et dī∣repta do∣m\b{ū}s et ∣ parvī ∣ cāsus I∣ūlī};
\apud{Aen.}{2, 563}.

\latin{lītora ∣ iactē∣t\b{ū}r odi∣īs Iū∣nōnis a∣cerbae};
\apud{Aen.}{1, 668}.

\end{examples}

\section[Separation of a Mute from a Following Liquid]

The mute may be pronounced with the preceding vowel, adding a unit to
the time, instead of being pronounced, as usually, in the same impulse
with the liquid (\xref[2,note]{14}).
\begin{examples}

\latin{aut tere∣brāre ca∣vās ute∣rī \slur\ et temp∣tāre la∣te\b{b}-rās};
\apud{Aen.}{2, 38}. (Contrast

\latin{tum levis ∣ haut ul∣trā late∣\b{b}rās iam ∣ quaerit i∣māgō};
\apud{Aen.}{10, 663}.)

\end{examples}

\section
\subsection

\textbf{Consonantal \phone{i} and~\phone{u} Pronounced as Vowels.}
Consonantal~\phone{i} and~\phone{u} may be pronounced more fully,
becoming vowels~(\xref{2}).
\begin{examples}

\latin{nunc mare ∣ nunc silu∣ae} (\scan{lss|lss|l^});
\apud{Epod.}{13, 2}.  (\latin{Siluae} for \latin{silvae}.)

\end{examples}

\subsection

\textbf{Vowels \phone{i} and~\phone{u} Pronounced as Consonants.}
The vowels \phone{i} and~\phone{u} may be compressed, thus becoming
consonants~(\xref{2}).  This pronunciation throws the preceding
consonant back into the preceding syllable, and makes that syllable
long, even if in ordinary pronunciation it is short.
\begin{examples}

\latin{aedifi∣cant sec∣tāque \slur\ in∣texunt ∣ ab-iete ∣ costās};
\apud{Aen.}{2, 16}. (Pronounce \phone{ab-yete}.)

\latin{cōnūb∣iō iun∣gam stabi∣lī propri∣amque di∣cābō};
\apud{Aen.}{1, 73}.  (Pronounce \phone{cō\-nūb∣yō}.)

\end{examples}

\section[Inventions of New Pronunciations]

For a few words that had to be used in poetry, but were difficult or
impossible in their ordinary pronunciation, a new one might be
devised.  Thus Virgil has \latin{Ăsiae} in \apud{Aen.}{3, 1}, but
\latin{Āsia} in \apud{}{7, 701}; \latin{Ītaliam} in~\apud{}{1, 2}, but
\latin{Ĭtalī} in~\apud{}{1, 109}; \latin{Prīamidēn}
in~\apud{}{6, 494}, but \latin{Prĭamēïa} in~\apud{}{2, 403}.

\section[Contraction of Vowels\protect\footnotemark]
\footnotetext{\label{ftn:352:3}Technically called Synizésis, or
  Synaéresis, \emph{a taking-together}.}

Difficult words are sometimes made possible to use through the
contraction of two vowels.  Thus \latin{Īlion\t{e}{ī}} in
\apud{Aen.}{1, 120}; \latin{alv\t{e}{ō}} in \apud{}{6, 412};
\latin{sc\t{i}{ō}} in \apud{}{3, 602}; \latin{d\t{e}{h}inc}
in~\apud{}{1, 131} (contrast \latin{dōna de∣hinc} in~\apud{}{3, 464}).

\section[Tmesis (“\emph{cutting in two}”)]

A poet often obtains variety, and sometimes can employ a word not
otherwise possible to use, by cutting a compound into two parts. Thus
\latin{hāc celebrāta tenus (hāctenus celebrāta}), \apud{Aen.}{5, 603};
\latin{super ūnus eram} (\latin{supereram} would be impossible in the
Dactylic Hexameter); \apud{Aen.}{2, 567}.

\backmatter

\part{Appendix}

\chapter{The Roman Calendar}

\contentsentry{B}{The Roman Calendar}

\section

The Romans divided time, as we do, by years, months, days, and hours.

\section

A given year as date was indicated either:

\subsection

By the names of the consuls in the Ablative Absolute with
\latin{cōnsulibus} (see first example in~\xref{421}); or, less
commonly

\subsection

By the number of the year as reckoned from the supposed date of the
founding of the city (753~\bc).
\begin{examples}

\latin{annō trecentēnsimō quīnquāgēnsimō post Rōmam conditam},
\english{in the three hundred and fiftieth year after the founding of
  Rome}; \apud{Rep.}{1, 16, 25}.

\end{examples}

\begin{minor}

\subsubsection

To convert to our reckoning, subtract from 754 (upon the principle
explained in footnote~\ref{ftn:a} below). Thus the date in the example
above is 754 − 350 = 404.

\end{minor}

\section

The months were \latin{Iānuārius}, \latin{Februārius},
\latin{Mārtius}, \latin{Aprīlis}, \latin{Maius}, \latin{Iū\-ni\-us},
\latin{Iūlius}, \latin{Augustus}, \latin{September}, \latin{Octōber},
\latin{November}, \latin{December}.\footnote{Originally adjectives.
  Thus (\latin{mēnsis}) \latin{Iānuārius}.}

\begin{minor}

\subsubsection

The names \latin{Iūlius}, \english{July}, and \latin{Augustus},
\english{August}, were first given under Augustus, in honor
respectively of Julius Caesar and Augustus himself.  Before this time
these months were called respectively \latin{Quīnctīlis} and
\latin{Sextīlis}.\footnote{The Roman year originally began with March.
  Hence the old names of \latin{Quīnctīlis} (fifth month), and
  \latin{Sextīlis} (sixth), and the names of the remaining months
  (\latin{September}, \english{the seventh} month, \latin{Octōber},
  \english{the eighth}, \latin{November}, \english{the ninth},
  \latin{December}, \english{the tenth}).}

\end{minor}

\section

After the reform of the Calendar by Julius Caesar in 46~\bc, the
number of days assigned to the various months was as now.

\section

Days were reckoned from three fixed points in the month: the Kalends,
or first day, and the Nones and Ides, respectively the seventh and
fifteenth days in March, May, July, and October, the fifth and
thirteenth in the other months\footnote{Before the reform of the
  calendar, March, May, July, and October were reckoned as of 31~days
  each, February of~28, and the rest of~29.  The greater length of the
  first-mentioned months is the reason why the Nones and the Ides were
  put correspondingly later in them.}  (\latin{Kalendae},
\latin{Nōnae},\footnote{\label{ftn:a}So called because it was the
  ninth day, by the Roman way of reckoning (which includes the day
  reckoned~\emph{to}), before the Ides.  Thus the 7th is the ninth day
  back in the row 7, 8, 9, 10, 11, 12, 13, 14,~15.} \latin{Īdūs},
abbreviated \abbrev{K.} or \abbrev{Kal.}, \abbrev{Nōn.}, \latin{Īd.}).

\section

The various days of the month are reckoned as such and such a day
\emph{before} one of these fixed points.
The day immediately before
the fixed points was so named, namely \latin{prīdiē}
(\latin{Kalendās}, \latin{Nōnās}, or \latin{Īdūs}), \emph{the day
  before} (\emph{the Kalends}, etc.).  Other days were designated by
their number before the fixed points, both days being counted in the
reckoning.  Thus, while January~31 was \latin{prīdiē Kal.\ Feb.}\ (the
day before the first of February), January~30 was \latin{diēs tertius
  ante Kal.\ Feb.}\ (the third day back in the row—30, 31,~1).  The
case is similar with the days before the Nones or Ides.

Hence the rule for changing a modern date (except the day immediately
before a fixed point, or \latin{prīdiē}) is:

\subsection

For days before the Nones or Ides, add one to the date of the Nones or
Ides in the given month, and subtract the given number.

Thus Jan.~2 = 5 (date of Nones in Jan.)\ + 1 − 2 = the 4th day before
\latin{Nōn.\ Iān.}

\subsection

For days before the Kalends, add two\footnote{This is because one has
  to reckon in not only the last day of the month, but also the first
  of the next (\latin{Kalendae}).  Hence the days reckoned are 28, 29,
  30, 31,~1, so that 28 is the fifth day back.} to the number of days
in the month concerned, and subtract the given number.

Thus Jan.\ 28 equals 31 + 2 − 28 = the 5th day before
\latin{Kal.\ Feb.}

\section

The grammatical form for the Kalends, Nones, and Ides as dates is the
Ablative of the Time at Which (\xref{439}).  Thus \latin{Kalendīs
  Februāriīs}, (on) \english{February 1st}.

\section

For the other days two forms are in common use.  Thus:
\begin{Tabular}[][l]{l@{ }l}

Jan.~29 = \latin{quārtō (diē ante)} & \latin{Kal.\ Feb.}\ =
\latin{IV Kal.\ Feb.}, \emph{or}\\

Jan.~29 = \latin{ante diem quārtum} & \latin{Kal.\ Feb.}\ = \latin{a.~d.~IV
  Kal.\ Feb.}

\end{Tabular}

\begin{minor}

\subsubsection

The second way is perhaps descended from an original \latin{ante (diē
  quārtō) Kalendās Februārias}, \emph{before \(namely on the fourth
  day\) the Kalends of February}.  The Ablative would easily pass over
to the Accusative, in consequence of its position immediately after
\latin{ante}.

\end{minor}

\section

The second of these forms is the more common.  It is thought of as one
word, so that \latin{ex}, \latin{in}, or \latin{ad} may be used before
it.  Thus “from January~29 to November~3” = \latin{ex a.~d. IV
  Kal.\ Feb.\ usque ad a.~d. III Nōn.\ Nov.}

\section

In leap year an extra day was inserted after Feb.~24 (\latin{a.~d. VI
Kal.\ Mārt.}), which was called \english{the sixth day over again},
i.e.\ \latin{a.~d. bissextum Kal.\ Mārt.}  Hence leap year was called
\latin{annus bissextīlis}.  After this day the reckoning went on as
usual.

\begin{minor}

\subsubsection

Before the reform, the year (355 days) was short of the true year.  To
make up for the difference, an extra month (\latin{mēnsis
  intercalāris}) of varying length (27 or 28~days), was inserted by
the Pontifices after the 23rd of February, the rest of February being
then omitted.

\end{minor}

\section
\subsection

The day was divided into two sets of twelve hours each, one running
from sunrise to sunset, the other from sunset to sunrise.  Thus the
first hour is \latin{hōra prīma} (at night \latin{hōra prīma noctis}),
the second, \latin{hōra secunda}, the third, \latin{hōra tertia}, etc.
But it is often impossible for us to tell whether, for a given hour,
the Romans meant at the \emph{end} of that hour (\latin{hōra prīma} =
seven o’clock), or \emph{within} that hour (\latin{hōra prīma} =
between six and seven).

\begin{minor}

\subsubsection

The hours differed greatly in length at different times in the year.

\end{minor}

\subsection

In camp the night was divided into four watches of three Roman hours
each (\latin{vigilia prīma}, \latin{secunda}, \latin{tertia},
\latin{quārta}).

\section
\headingB{Calendar}

\begin{calendar}
\hline
\hline

\scriptsize\textsc{days of}
& \M{3}{c|}{\scriptsize\textsc{march, may, july,}}
& \M{3}{c|}{\scriptsize\textsc{january, august,}}
& \M{3}{c|}{\scriptsize\textsc{april, june,}}
& \M{3}{c}{\scriptsize\textsc{february}\footnote{The forms in brackets are for \emph{leap year}.}}

\\[-1.5\jot]

\scriptsize\textsc{our month}
& \M{3}{c|}{\scriptsize\textsc{october}}
& \M{3}{c|}{\scriptsize\textsc{december}}
& \M{3}{c|}{\scriptsize\textsc{september,}}
& 
\\[-1.5\jot]

& \M{3}{c|}{}
& \M{3}{c|}{}
& \M{3}{c|}{\scriptsize\textsc{november}}
& 
\\

\hline

1
& \M{3}{c|}{\textbf{Kal.}}
& \M{3}{c|}{\textbf{Kal.}}
& \M{3}{c|}{\textbf{Kal.}}
& \M{3}{c}{\textbf{Kal.}}
\\

2
& a.d. & VI & Nōn
& a.d. & IV & Nōn
& a.d. & IV & Nōn
& a.d. & IV & Nōn
\\

3
& a.d. & V & \ditto
& a.d. & III & \ditto
& a.d. & III & \ditto
& a.d. & III & \ditto
\\

4
& a.d. &  IV & \ditto
&      & prīd. & \ditto
&      & prīd. & \ditto
&      & prīd. & \ditto
\\

5
& a.d. & III   & \ditto
&      & \textbf{Nōn.} &
&      & \textbf{Nōn.} &
&      & \textbf{Nōn.} &
\\

6
&      & prīd. & \ditto
& a.d. & VIII & Īd.
& a.d. & VIII & Īd.
& a.d. & VIII & Īd.
\\

7
&      & \textbf{Nōn.} & 
& a.d. & VII & \ditto
& a.d. & VII & \ditto
& a.d. & VII & \ditto
\\

8
& a.d. & VIII & Īd.
& a.d. & VI & \ditto
& a.d. & VI & \ditto
& a.d. & VI & \ditto
\\

9
& a.d. & VII & \ditto
& a.d. &   V & \ditto
& a.d. &   V & \ditto
& a.d. &   V & \ditto
\\

10
& a.d. &  VI & \ditto
& a.d. &  IV & \ditto
& a.d. &  IV & \ditto
& a.d. &  IV & \ditto
\\

11
& a.d. &    V & \ditto
& a.d. &  III & \ditto
& a.d. &  III & \ditto
& a.d. &  III & \ditto
\\

12
& a.d. &   IV & \ditto
&      &  prīd. & \ditto
&      &  prīd. & \ditto
&      &  prīd. & \ditto
\\

13
& a.d. &  III & \ditto
&      &  \textbf{Īd.} &
&      &  \textbf{Īd.} &
&      &  \textbf{Īd.} &
\\

14
&      &  prīd. & \ditto
& a.d. &    XIX & Kal.
& a.d. &  XVIII & Kal.
& a.d. &    XVI & Kal.
\\

15
&      &    \textbf{Īd.} &
& a.d. &  XVIII & \ditto
& a.d. &   XVII & \ditto
& a.d. &     XV & \ditto
\\

16
& a.d. &   XVII & Kal.
& a.d. &   XVII & \ditto
& a.d. &    XVI & \ditto
& a.d. &    XIV & \ditto
\\

17
& a.d. &    XVI & \ditto
& a.d. &    XVI & \ditto
& a.d. &     XV & \ditto
& a.d. &   XIII & \ditto
\\

18
& a.d. &     XV & \ditto
& a.d. &     XV & \ditto
& a.d. &    XIV & \ditto
& a.d. &    XII & \ditto
\\

19
& a.d. &    XIV & \ditto
& a.d. &    XIV & \ditto
& a.d. &   XIII & \ditto
& a.d. &     XI & \ditto
\\

20
& a.d. &    XIII & \ditto
& a.d. &    XIII & \ditto
& a.d. &     XII & \ditto
& a.d. &       X & \ditto
\\

21
& a.d. &     XII & \ditto
& a.d. &     XII & \ditto
& a.d. &      XI & \ditto
& a.d. &      IX & \ditto
\\

22
& a.d. &      XI & \ditto
& a.d. &      XI & \ditto
& a.d. &       X & \ditto
& a.d. &    VIII & \ditto
\\

23
& a.d. &       X & \ditto
& a.d. &       X & \ditto
& a.d. &      IX & \ditto
& a.d. &     VII & \ditto
\\

24
& a.d. &      IX & \ditto
& a.d. &      IX & \ditto
& a.d. &    VIII & \ditto
& a.d. &      VI & \ditto
\\

25
& a.d. &    VIII & \ditto
& a.d. &    VIII & \ditto
& a.d. &     VII & \ditto
& a.d. &  V [VI] & \ditto
\\

26
& a.d. &     VII & \ditto
& a.d. &     VII & \ditto
& a.d. &      VI & \ditto
& a.d. &  IV [V] & \ditto
\\

27
& a.d. &      VI & \ditto
& a.d. &      VI & \ditto
& a.d. &       V & \ditto
& a.d. & III [IV] & \ditto
\\

28
& a.d. &       V & \ditto
& a.d. &       V & \ditto
& a.d. &      IV & \ditto
& \M{2}{r@{ }}{prīd.\ Kal.\ [III]} & \ditto
\\

29
& a.d. &      IV & \ditto
& a.d. &      IV & \ditto
& a.d. &     III & \ditto
& \M{2}{r@{ }}{[prīd.\ Kal.]} & \ditto
\\

30
& a.d. &     III & \ditto
& a.d. &     III & \ditto
&      &   prīd. & \ditto
&
\\

31
& &   prīd. & \ditto
& &   prīd. & \ditto
& &   & 
&
\\

\hline
\hline

\end{calendar}

\chapter{Roman Money, Weights, and Measures}

\contentsentry{B}{Roman Money, Weights, and Measures}

\headingG{Roman Measures of Money and Weight}

\section

The original unit of weight and value was the \latin{as}, a mass of
copper, weighing nearly one pound, or \latin{libra}.  This was divided
into twelve ounces (\latin{ūn\-ci\-ae}).

The following table shows the more important fractions:
\begin{Tabular}[\small]{clrl}

\multicolumn{2}{l}{\emph{Ounces}}
& \multicolumn{2}{l}{\emph{Ounces}}
\\

\vfrac{1}{2} &\latin{sēmiūncia} (\latin{sēmis} = \english{a half})

& 7 & \latin{septūnx} (\latin{septem ūnciae})
\\

 1 & \latin{ūncia}
& 8 & \latin{bessis} or \latin{bes}
\\

 2 & \latin{sextāns} (\english{a sixth})
& 9 & \latin{dōdrāns} (\latin{dēquadrāns}, \english{a fourth off})
\\

 3 & \latin{quadrāns} (\english{a fourth}); also \latin{terūncius}
& 10 & \latin{dēxtāns} (\latin{dēsextāns}, \english{a sixth off})
\\

 4 & \latin{triēns} (\english{a third})
& 11 & \latin{deūnx} (\latin{deūncia}, \english{an ounce off})
\\

 5 & \latin{quīncūnx} (\latin{quīnque ūnciae})
& 12 & \latin{as} (of money, \latin{libra} of weight)
\\

 6 & \latin{sēmis} or \latin{sēmissis} (\english{a half})

\end{Tabular}

\section
\subsection

For any kind of thing, these terms may be used to express fractions
having~12 for a denominator.  Thus \vfrac{1}{6} = \latin{sextāns},
\vfrac{5}{12} = \latin{quīncūnx}, \vfrac{3}{4} = \latin{dōdrāns}.

\subsection

Fractions having~1 for a numerator may be indicated by an ordinal with
or without \latin{pars}.  Thus \vfrac{1}{2} = \latin{dīmidia} or
\latin{dīmidia pars} (also \latin{dīmidium}), \vfrac{1}{3} =
\latin{tertia} or \latin{tertia pars}.\emend{54}{\footnotemark[1]}{}

\subsection

Fractions having a denominator greater by~1 than the numerator may be
indicated by a cardinal number with \latin{partēs}.  Thus \vfrac{2}{3}
= \latin{duae partēs}.

\subsection

Other fractions are indicated by the cardinal for a numerator and the
ordinal for a denominator.  Thus \vfrac{2}{5} = \latin{duae quīntae}.

\subsection

Fractions may also be indicated by addition. Thus \vfrac{3}{4} =
\latin{dīmidia et quārta} (\vfrac{1}{2} + \vfrac{1}{4}).

\subsection

Proportions in inheritances are indicated by any of these forms,
with~\latin{ex}.  Thus \latin{hērēs ex asse}
(\apud{Plin.\ Ep.}{5, 1, 9}), \english{heir to the whole}; \latin{hērēs
  ex parte quārtā} (\emph{ibid}.), \english{heir to a fourth};
\latin{hērēs ex triente}, \english{heir to a third}, etc.

\section

The \latin{as} was reduced till, at the close of the Second Punic War,
it weighed but one ounce.  Its value was then a little less than two
cents (or about 1\emph{d}.~English).

\section
\subsection

Other coins were the \latin{sēstertius}, a small silver coin, the
\latin{dēnārius}, a larger silver coin, and the \latin{aureus} or gold
piece.  The sum of a thousand sesterces was called \latin{sēstertium}
(originally a Genitive Plural, “of sesterces”).  The word
\latin{nummus} (“coin”) is often attached to \latin{sēstertius} or
\latin{aureus}.  When used alone, \latin{nummus} stands for
\latin{sēstertius}.  The table is as follows:\footnote{Since values
  frequently changed, a table can be only approximate.}
\smallskip
\begin{Tabular}[\small]{r@{\,\,}l@{\,=\,}l}

2\,\vfrac{1}{2}
& \latin{assēs}
& 1 \latin{sēstertius}\footnote{\latin{Sēmis tertius}, \english{the
      third part of a half}, i.e.\ two whole numbers +~\vfrac{1}{2}.}
  (a little more than 4 cents, or 2\emph{d}.\ English money).
\\

4
& \latin{sēstertiī}
& 1 \latin{dēnārius}\footnote{\latin{Dēnārius}, a piece of money
  containing ten \latin{assēs} (\latin{dēnī}); cf.~“tenpence.”}
  (a little more than 16~cents, or 8\emph{d}.\ English money).
\\

25
& \latin{dēnāriī}
& 1 \latin{aureus} (about \$4, or 17\emph{s}.\ English money).
\\

1000
& \latin{sēstertiī}
& 1 \latin{sēstertium}
  (about \$42.50, or £8\,10\emph{s}.\ English money).

\end{Tabular}

\subsection

The reckoning of money was by the sesterce and its multiples, as
follows:
\begin{enuma}\parindent\normalparindent

\item

Up to 2000, by sesterces.  Thus \latin{trīgintā sēstertiī}, 30
\english{sesterces}; \latin{trecentī sē\-ster\-tiī},
300~\english{sesterces}.

\item

From 2000 to 1,000,000, by \emph{thousands} of sesterces, i.e.\ by
\latin{sēstertia}.  The numeral used was generally the distributive
(\emph{sometimes} the ordinal).  Thus: \latin{bīna (duo) sēstertia},
2000~\english{sesterces}.

\item

From 1,000,000 upwards, by \english{hundreds of thousands} of
sesterces, i.e.\ by \latin{centēna mīlia
  sēstertium}.\footnote{\latin{Sēstertium} is here a true genitive
  plural.}  The numeral used was the adverb.  Thus \latin{deciēns
  centēna mīlia sēstertium} = ten times 100,000, = 1,000,000.

But the words \latin{centēna mīlia} are generally omitted, and
sometimes even the word \latin{sēstertium}.  Thus \latin{deciēns
  sēstertium}, or simply \latin{deciēns}, = 1,000,000.

\end{enuma}

\subsection

The sign~HS was used for either a \latin{sēstertius} or a
\latin{sēstertium}, the difference being ordinarily shown by the use
of cardinal and distributive numerals respectively.  With an
abbreviation in Roman numerals, a straight mark drawn above means
\latin{sēstertia}. Thus:
\begin{examples}

\latin{HS XXX} = \latin{trīgintā sēstertiī}, 30 sesterces

\latin{HS $\overline{\latin{XXX}}$}
    = \latin{trīcēsima sēstertia}, 30,000 sesterces

\end{examples}

\section
\subtitle{\textsc{Roman Measures of Length}}
\smallskip
\begin{Tabular}{r@{\,\,}l@{\,=\,}l}

4 & \latin{digitī} (“finger-breadths”)
& 1 \latin{palmus} (“palm”)
\\

4 & \latin{palmī}
& 1 \latin{pēs} (11.6 inches)
\\


2\,\vfrac{1}{2} & \latin{pedēs}
& 1 \latin{gradus} (“step”)
\\

2 & \latin{gradūs}
& 1 \latin{passus} (“pace”)\footnote{One double pace, that is, one
  easy step with each foot, or a little less than 5~feet.  Hence
  \latin{mīlle passūs}, or \latin{mīlle passuum} = a little less than
  one English mile.  (The Roman mile has been estimated at 4851~feet.
The English mile = 5280~feet.)}
\\

1000 & \latin{passūs}
& \latin{mīlle passūs} or \latin{mīlle passuum} (“mile”)

\end{Tabular}

\subsubsection

A \latin{stadium} (from a Greek word) was an eighth of a Roman mile (a
little less than our furlong).

\subsubsection

The unit of measure of land was a \latin{iūgerum} (translated
\english{acre}, but really a little less than \vfrac{2}{3} of an
acre), an area of 240 by 120~feet.

\pagebreak

\section
\subtitle{\textsc{Roman Measures of Capacity}}
\smallskip

\begin{Tabular}[\small]{@{}
                        r@{\,\,}l@{\,=\,}l
                        @{\qquad}
                        r@{\,\,}l@{\,=\,}l
                        @{}}

\multicolumn{3}{c}{\emph{Liquid Measure}}
& \multicolumn{3}{c}{\emph{Dry Measure}}
\\[\smallskipamount]

1\,\vfrac{1}{2} & \latin{cyathī}\footnote{\latin{Cyathus} meant originally
  \english{small ladle}.}
  & 1 \latin{acētābulum}
& 1\,\vfrac{1}{2} & \latin{cyathī} & 1 \latin{acētābulum}
\\

  2 & \latin{acētābula} & 1 \latin{quārtārius}
& 2 & \latin{acētābula} & 1 \latin{quārtārius}
\\

  2 & \latin{quārtāriī} & 1 \latin{hēmīna}
& 2 & \latin{quārtāriī} & 1 \latin{hēmīna}
\\

  2 & \latin{hēmīnae}   & 1 \latin{sextārius} (about a pint)
& 2 & \latin{hēmīnae}   & 1 \latin{sextārius}
\\

  6 & \latin{sextārīi} & 1 \latin{congius}
& 8 & \latin{sextārīi} & 1 \latin{sēmodius}
\\

  4 & \latin{congiī}  & 1 \latin{urna}
& 2 & \latin{sēmodiī} & 1 \latin{modius} (about a peck)
\\

  2 & \latin{urnae} & 1 \latin{amphora}
\\

 20 & \latin{amphorae} & 1 \latin{culleus}

\end{Tabular}

\subsubsection

A \latin{sextārius} (pint) thus contained 12~\latin{cyathī}
(\vfrac{3}{2} × 2 × 2 × 2).

\chapter{Roman Names}

\contentsentry{B}{Roman Names}

\section
\subsection

The Roman regularly had three names: the \latin{praenōmen}, or
\emph{first name} (our “given name”), the \latin{nōmen}, or
\emph{principal name}, and the \latin{cognōmen}, or \emph{additional
name}.  Thus:
\begin{center}
\begin{tabular}{ccc}
\latin{praenōmen}   & \latin{nōmen}     & \latin{cognōmen} \\
\latin{Mārcus}      & \latin{Tullius}   & \latin{Cicerō}
\end{tabular}
\end{center}

\begin{enuma}

\item

The \latin{praenōmen} indicates the individual, the \latin{nōmen} the
\latin{gēns}, or largest unit of related persons (our “last name”),
the \latin{cognōmen}, the family, or smaller unit of related persons.

\item
The \latin{nōmen} always ends in~\suffix{-ius}.  Thus \latin{Tullius},
\latin{Cornēlius}, \latin{Iūlius}.

\item

The \latin{cognōmen} originally indicated some personal
peculiarity.  Thus \latin{Scae\-vo\-la}, \english{left-handed},
\latin{Cicerō}, \english{chick-pea}, or \english{wart},
\latin{Balbus}, \english{lisping}.  But of course these names lost all
personal application as they were passed down, just as have our names
White, Brown, Armstrong, etc.

\end{enuma}

\subsection

A second cognomen was sometimes added to commemorate an achievement.
Thus \latin{Cornēlius Scīpiō Āfricānus} (\english{conqueror of
  Africa}).

\begin{minor}

\subsubsection

From the Fourth Century, this was often called an \latin{agnōmen}.

\end{minor}

\subsection

The \latin{praenōmina}, with their abbreviations, are:
\begin{Tabular}{l@{ }l
                @{\qquad\qquad}
                l@{ }l
                @{\qquad\qquad}
                l@{ }l}

  \latin{A.}    & \latin{Aulus}
& \latin{L.}    & \latin{Lūcius}
& \latin{Q.}    & \latin{Quīntus}
\\

  \latin{App.}  & \latin{Appius}
& \latin{M.}    & \latin{Mārcus}
& \latin{Sex.}  & \latin{Sextus}
\\

  \latin{C.}    & \latin{Gāius}
& \latin{M’.}   & \latin{Manius}
& \latin{Ser.}  & \latin{Servius}
\\

  \latin{Cn.}   & \latin{Gnaeus}
& \latin{Mām.}  & \latin{Māmercus}
& \latin{Sp.}   & \latin{Spurius}
\\

  \latin{D.}    & \latin{Decimus}
& \latin{N.}    & \latin{Numerius}
& \latin{T.}    & \latin{Titus}
\\

  \latin{K.}    & \latin{Kaesō}
& \latin{P.}    & \latin{Pūblius}
& \latin{Ti(b).} & \latin{Tiberius}
\end{Tabular}

\subsection

An adopted son took the name of the adoptive father, adding his own
gentile name in the form of an adjective in~\suffix{-ānus}.  Thus
\latin{L.\ Aemilius Paulus}, being adopted by \latin{P.\ Cornēlius
  Scīpiō}, became \latin{P.\ Cornēlius Scīpiō Aemiliānus}.

\begin{minor}

\subsubsection

But irregular methods ultimately came into fashion.  Thus when Pliny
the Younger, whose name had been \latin{P.\ Caecilius Secundus}, was
adopted by his uncle \latin{C.\ Plīnius Secundus}, instead of taking
the name \latin{C.\ Plīnius Secundus Caecili\emend{55}{a}{ā}nus} (as
by the older usage he would have done), he took the name
\latin{C.\ Plīnius Caecilius Secundus}.

\end{minor}

\subsection

Women had no \latin{praenōmina}, but were called by the feminine form
of the name of the \latin{gēns}.  Thus the daughter of \latin{Mārcus
  Tullius Cicerō} was called \latin{Tullia}.  If there were two
daughters, they were distinguished as the “elder” and the
“younger” (thus \latin{Tullia Maior}, \latin{Tullia Minor}).  If
there were other daughters, the later-born were called “third”
(\latin{Tertia}), “fourth” (\latin{Quārta}), etc.

\chapter{Hidden Quantity}

\contentsentry{B}{Hidden Quantity}

\section

List of words containing a long vowel before two or more consonants.
Omitted are:
\begin{enumerate}

\item
Words containing \phone{ns}, \phone{nf}, \phone{nx}, \phone{nct},
before which the vowel is always long.  See~\xref{18}.

\item
Verbs in~\suffix{-scō}, in all but three of which the vowel before the
suffix is long.  See~\xref[4]{23}.

\item
Shortened Perfect forms in \suffix{-āsse}, \suffix{-ēsse},
\latin{-īsse}, \suffix{-āstī}, \suffix{-ēstī}, \latin{-īstī}, etc., in
which the vowel before~\phone{s} is always long.  See~\xref[1]{163},
and footnote~\ref{ftn:96:1}.

\item
Nominatives in \suffix{-x}, \suffix{-ps}, \suffix{-bs}, before which
the vowel is long if long in the other cases, as
\latin{lēx}, Gen.\ \latin{lēgis}; \latin{Cyclōps},
Gen.\ \latin{Cyclōpis}; \latin{plēbs}, Gen.\ \latin{plē\-bis}.

\item
Derivatives in \suffix{-ātrum}, \suffix{-ābrum}, etc.  See~\xref[2]{23}.

\item
Compounds, derivatives, and parallel formations of words containing a
long vowel.
See \xref{22}, \xref{24}.  Thus \latin{ōrnō} implies
\latin{ōrnāmentum}, \latin{lūxus} implies \latin{lūxuria},
\latin{āctum} implies \latin{āctus} (\suffix{-ūs}), \latin{āctiō},
\latin{āctor}, etc.

\item
Proper names and rare words.

\end{enumerate}

But several words belonging under 5), 6), or~7) are, for greater
convenience, included in the list.

\begin{hiddenquantity}

āctum, āctiō, etc.

Adrāstus

Āfrica, Āfrī, etc.

Alcēstis

Ālēctō

alīptēs

Amāzōn

anguīlla

Aquīllius

arātrum

ārdeō, ārsī, etc.

āthla

āthlētēs

ātrium

Ātrius

\medskip

bārdus

Bēdriacum

bēstia

bimēstris

bovīllus

Būthrōtum

\medskip

candēlābrum

catēlla, \english{chain}

catīllus

chīrūrgus

cicātrīx

Cīncius

clātrī

Clytēm(n)ēstra

Cnōssus

comēstum

cōmptum, etc.

cōntiō

corōlla

crābrō

Crēssa

crībrum

crīspus

crūsta, crūstum

\medskip

dēlūbrum

dēmptum

dēxtāns

Diēspiter

dīgladior

dīgredior

dōdrāns

dolābra

\medskip

ēbrius

ēmptum, etc.

ēsca

Ēsquiliae

Etrūscus

exīstimō

\medskip

fāstus, \english{court-day}

favīlla

fēstus

fīxī, fīxum

flābrum

-flīxī, -flīctum

flūctus

flūxī, flūxus

fōrma

frāctum, frāgmen

-frīxī

frūctus

frūstrā

frūstum

fūrtim, fūrtum

fūstis

\medskip

geōgraphia

geōrgicus

glōssārium

\medskip

Hellēspontus

hibīscum

hīllae

hōrnus

hōrsum

Hymēttus

\medskip

Īllyria

inlūstris

intrōrsum

involūcrum

Iōlcus

istōrsum

iūglāns

iūrgō

iūstus

iūxtā

\medskip

lābrum, \english{basin}

lāmna

lāpsus

lārdum

Lārs

lārva

lātrīna

lātrō, \english{bark}

lavābrum

lavācrum

lēctum (from legō)

lēmna

lēmniscus

Lēmnos

lentīscus

lībra

līctor

lūbricus

lūctus

lūstrum, \english{expiation}

lūstrō

lūxī

lūxus, \english{luxury}

Lycūrgus

\medskip

mālle, etc.

Mānlius

Mārcellus

Mārcus

Mārs

Mārsī

māssa

mercēnnārius

Mētrodōrus

mētropolis

mīlle

mīlvus

Mōstellāria

mūcrō

mūscus

\medskip

Nārnia

nārrō

nāsturcium

nefāstus

nīxus

nōlle, etc.

nōndum

nōngentī

nōnne

Nōrba

nūllus

nūndinae

nūntiō, nūntius

nūptum, nūptiae

nūtriō, nūtrīx

\medskip

Oenōtria

ōlla

Onchēstus

Opūs, Opūntis

ōrca

orchēstra

ōrdior

ōrdō

ōrnō

ōscitāns

ōsculum, ōsculor

Ōstia

ōstium

ovīllus

Ōxus

\medskip

pāctum (from pangō)

palimpsēstus

palūster

pāstillus

pāstum, pāstor, etc.

pēgma

perīclitor

Permēssus

Phoenīssa

pīstum, pīstor, etc.

plēctrum

plōstellum

Pōlliō

Polymēstor

pōsca

prāgmaticus

Prāxitelēs

prēndō

prīmōrdium

prīnceps

prīscus

prīstinus

Procrūstēs

prōmptum, etc.

prōrsum

prōsper, prōsperus

prōstibulum

Pūblicola

pūblicus

Pūblius

pulvīllus

pūrgō

pūstula

\medskip

quārtus

quīncūnx

quīndecim

quīnquātrūs

quīnque, quīntus

Quīntiliānus, Quīntus

quōrsum

\medskip

rāstrum

reāpse

rēctum, rēctor, etc.

rēgnum

rēxī

rīxa

rōscidus

Rōscius

rōstrum

Rōxānē

rūctō, rūctus, etc.

rūrsum

rūsticus

\medskip

Sārsina

scēptrum

sēgnis

sēmēstris

sēmūncia

sēscūncia

Sesōstris

sēsqui-

sēstertius

Sēstius

Sēstos

simulācrum

sinistrōrsus

sīstrum

sōbrius

Sōcratēs

sōlstitium

sōspes

sōspita

stāgnum

stīlla

strūxī, strūctum, etc.

sublūstris

suīllus

sūmptum, etc.

sūrculus

sūrsum

Sūtrium

\medskip

tāctum, etc.

Tartēssus

Tecmēssa

tēctum, etc.

Telmēssus

Tēmnos

tēxī

theātrum

Thrēssa

Tīllius

trāxī

trīstis

\medskip

ūllus

ūncia

ūndecim

ūsūrpō

\medskip

vāllum, vāllus

vāsculum

vāstus

Vēctis

vēgrandis

Vēlābrum

Venāfrum

vēndō

vērnus

vēstibulum

vēstīgium

vīxī, vīctus

vīlla

vīllum

vīndēmia

Vīpsānius

vīscus

\medskip

Xenophōn, -ōntis

\end{hiddenquantity}

\Unnumbered{Catalogue of Verbs}
\label{verbcat}
\markboth{Catalogue of Verbs}{Catalogue of Verbs}
\markthird{}
\thispagestyle{dropfolio}

\contentsentry{B}{Catalogue of Verbs}

Most verbs of the First and Fourth Conjugations with principal parts
of the usual type are omitted; and of the Denominatives of the Second
Conjugation and the Inchoatives only a few are given. Compounds are
not noted unless they present some irregularity in formation, or a
change in the form of the root-syllable (see \xref{41}, \xref{42}). In
such cases the variation is shown under the simple verb. Some
compounds are also given separately with cross references to the
simple verb, but generally only at the beginning of the list
(compounds of \latin{ad} and \latin{con}), by way of illustration.  A
prefixed hyphen indicates that the form occurs only in compounds (not
necessarily in all compounds).

Forms which are unusual and may well be omitted by a student in
memorizing the principal parts are inclosed in~(~). Some very rare
forms are omitted entirely.  Perfect forms in \suffix{-iī} beside
\suffix{-īvī} are not ordinarily noted. For the forms making up
the Principal Parts, especially the fourth, see~\xref{150}.  When the
Future Active Participle does not follow the formation of the Perfect
Passive Participle (\xref{182}), it is added in~(~). Forms inclosed
in~[~] indicate the derivation or formation. The abbreviations Dep.,
Def., Impers., Irreg.\ are used for Deponent, Defective, Impersonal,
and Irregular.

\medskip

\begin{verbcat}

\latin{abdō}, see~\latin{dō}.

\latin{abiciō}, see~\latin{iaciō}.

\latin{abigō}, see \latin{agō}.

\latin{abluō}, see \suffix{-luō}.

\latin{abnuō}, see \suffix{-nuō}.

\latin{aboleō}, \english{destroy}, abolēre, abolēvī,
abolitum.

\latin{abolēscō}, \english{vanish}, abolēscere, abolēvī.

\latin{abripiō}, see \latin{rapiō}.

\latin{abscīdō}, see \latin{caedō}.

\latin{abstineō}, see \latin{teneō}.

\latin{accendō}, see \suffix{-cendō}.

\latin{accidō}, see \latin{cadō}.

\latin{accīdō}, see \latin{caedō}.

\latin{accipiō}, see \suffix{capiō}.

\latin{accumbō}, see \latin{-cumbō}.

\latin{acuō}, \english{sharpen}, acuere, acuī, acūtum.

\latin{addō}, see \latin{dō}.

\latin{adficiō}, see \latin{faciō}.

\latin{adflīgō}, see \suffix{-flīgō}.

\latin{adgredior}, see \latin{gradior}.

\latin{adhibeō}, see \latin{habeō}.

\latin{adiciō}, see \latin{iaciō}.

\latin{adigō}, see \latin{agō}.

\latin{adimō}, see \latin{emō}.

\latin{adipīscor}, see \latin{apīscor}.

\latin{adliciō}, see \suffix{-liciō}.

\latin{adluō}, see \suffix{-luō}.

\latin{adnuō}, see \suffix{-nuō}.

\latin{adolēscō}, see \latin{alēscō}.

\latin{adquīrō}, see \latin{quaerō}.

\latin{adsideō}, see \latin{sedeō}.

\latin{agnōscō}, see \latin{nōscō}.

\latin{agō}, \english{move}, agere, ēgī, āctum.  So
circum-agō, per-agō, praeter-agō, sat-agō.  But ab-igō,
ab-igere, ab-ēgī, ab-āctum; so ad-igō, amb-igō, ex-igō,
prōd-igō, red-igō, sub-igō, trāns-igō.  Note also
cōgō, cōgere, coēgī, co-āctum; dēgō, dēgere.

\latin{aiō}, \english{say}.  Def.  \xref[1]{198}.

\latin{albeō}, \english{be white}, albēre [albus].

\latin{albēscō}, \english{become white}, albēscere.

\latin{alēscō}, \english{grow up}, alēscere.
co-alēscō, co-alēscere, co-aluī (old cōlēscō, cōlēscere, cōluī);
ad-olēscō, \english{grow up}, ad-olēscere, ad-olēvī, ad-ultum;
ex-olēscō, ex-olēscere, ex-olēvī, ex-olētum;
in-olēscō, sub-olēscō in Pres.\ Syst.\ only.
See also obsolēscō.

\latin{algeō}, \english{be cold}, algēre, alsī.

\latin{algēscō}, \english{get cold}, algēscere, alsī.

\latin{alō}, \english{nourish}, alere, aluī, altum (alitum mostly
late).

\latin{ambiō}, see \latin{eō}.

\latin{amiciō}, \english{wrap about}, amicīre, amictum.
(Perf.\ rare, amicuī, amixī.)

\latin{amō}, \english{love}, -āre, -āvī, -ātum.

\latin{amplector}, see \suffix{-plector}.

\latin{angō}, \english{choke}, angere.

\latin{aperiō}, \english{open}, aperīre, aperuī, apertum.

\latin{apīscor}, \english{attain}, apīscī, aptus sum.  Dep.
ad-ipīscor, ad-ipīscī, ad-eptus sum; so
ind-ipīscor, red-ipīscor.

\latin{arceō}, \english{confine}, arcēre, arcuī.
Cpds.\ -erceō, -ercēre, -ercuī, -ercitum.

\latin{arcessō} (sometimes accersō), \english{send after},
arcessere, arcessīvī, arcessītum.

\latin{ārdeō}, \english{blaze}, ārdēre, ārsī, ārsūrus.

\latin{ārdēscō}, \english{blaze up}, ārdēscere, ārsī,
(ex)-ārsūrus.

\latin{āreō}, \english{be dry}, ārēre.

\latin{ārēscō}, \english{become dry}, ārēscere, (ex)-āruī.

\latin{arguō}, \english{make known}, arguere, arguī (argūtus,
Adj.).

\latin{arō}, \english{plough}, -āre, -āvī, -ātum.

\latin{arripiō}, see \latin{rapiō}.

\latin{ascendō}, see \latin{scandō}.

\latin{ascrībō}, see \latin{scrībō}.

\latin{aspergō}, see \latin{spargō}.

\latin{aspiciō}, see \suffix{-spiciō}.

\latin{attineō}, see \latin{teneō}.

\latin{attingō}, see \latin{tangō}.

\latin{audeō}, audēre, ausus sum.  Semi-Dep. (Perf.\ Subj.\ ausim,
\xref[5]{163}.)

\latin{audiō}, \english{hear}, audīre, audīvī,
audītum.

\latin{auferō}, see \latin{ferō}.

\latin{augeō}, \english{increase}, augēre, auxī, auctum.

\latin{avē}, \english{hail}. Def.  \xref{200}.

\bigskip

\latin{balbūtiō}, \english{stammer}, balbūtīre.

\latin{bibō}, \english{drink}, bibere, bibī, pōtum.

\latin{blandior}, \english{coax}, blandīrī, blandītus
sum.  Dep. [blandus].

\bigskip

\latin{cadō}, \english{fall}, cadere, cecidī, cāsūrus.
Cpds.~-cidō, -cidere, -cidī, -cāsum.

\latin{caedō}, \english{cut}, caedere, cecīdī, caesum.
Cpds.\ -cīdō, -cīdere, -cīdī, -cīsum.

\latin{caleō}, \english{be warm}, calēre, caluī, calitūrus.

\latin{calēscō}, \english{grow warm}, calēscere, -caluī.

\latin{candeō}, \english{be bright}, candēre, canduī.

\latin{candēscō}, \english{grow bright}, candēscere, -canduī.

\latin{cāneō}, \english{be gray}, cānēre [cānus].

\latin{cānēscō}, \english{grow gray}, cānēscere, cānuī.

\latin{canō}, \english{sing}, canere, cecinī (Partic.\ supplied
by cantātum from cantō).  Cpds.\ -cinō, -cinere, -cinuī
(rarely -cecinī).

\latin{capessō}, \english{seize eagerly}, capessere,
capessīvī, capessītum [capiō, \xref[4]{212}].

\latin{capiō}, \english{take}, capere, cēpī, captum.  So
ante-capiō.  But in other cpds.\ -cipiō, -cipere, -cēpī,
-ceptum.

\latin{careō}, \english{be without}, carēre, caruī,
caritūrus.

\latin{carpō}, \english{pluck}, carpere, carpsī, carptum.
Cpds.\ -cerpō, -cerpere, -cerpsī, -cerptum.

\latin{caveō}, \english{take care}, cavēre, cāvī, cautum.

\latin{cedo}, \english{give}.  Def.  \xref{200}.

\latin{cēdō}, \english{depart}, cēdere, cessī, cessum.

\suffix{-cellō}, \english{rise}, -cellere (celsus, Adj.).  Ante-, ex-,
prae-, re-.

\suffix{-cendō}, \english{burn}, -cendere, -cendī, -cēnsum
       [*candō; cf.~candeō].  Ac-, in-, suc-.

\latin{cēnseō}, \english{rate}, \english{think}, cēnsēre,
cēnsuī, cēnsum.

\latin{cernō}, \english{separate}, \english{decide}, cernere,
crēvī, -crētum (certus, Adj., rarely Partic.).

\latin{cieō}, \english{stir up}, ciēre, cīvī, citum.  But
ac-ciō, ac-cīre, ac-cīvī, ac-cītum; other
cpds.\ vary between -ciō, -cīre, -cītum, and -cieō,
-ciēre, -citum.

\latin{cingō}, \english{gird}, cingere, cīnxī, cīnctum.

\latin{clāreō}, \english{be bright}, clārēre [clārus].

\latin{clārēscō}, \english{grow bright}, clārēscere.

\latin{claudeō}, \english{limp}, claudēre (also claudō, claudere)
      [claudus].

\latin{claudō}, \english{shut}, claudere, clausī, clausum.
Cpds.\ -clūdō, -clūdere, -clūsī, -clūsum.

\latin{clepō}, \english{steal}, clepere, clepsī (rare verb).

\latin{clueō}, \english{be said}, cluēre (rare verb).

\latin{coepī}, \english{began}, coeptum (early Latin coepiō,
coepere).  Def.  \xref[2]{199}.

\latin{coerceō}, see \latin{arceō}.

\latin{cognōscō}, see \latin{nōscō}.

\latin{cōgō}, see \latin{agō}.

\latin{colō}, \english{cultivate}, colere, coluī, cultum.

\latin{combūrō}, see \latin{ūrō}.

\latin{comminīscor}, \english{devise}, comminīscī, commentus
sum.  Dep.\ [men- \emph{in} me-min-ī, etc.].

\latin{cōmō}, \english{comb}, cōmere, cōmpsī, cōmptum
      [emō].

\latin{comperiō}, see \suffix{-periō}.

\latin{compescō}, \english{restrain}, compescere, compescuī.

\latin{complector}, see \suffix{-plector}.

\latin{compleō}, see \suffix{-pleō}.

\latin{comprimō}, see \latin{premō}.

\latin{concidō}, see \latin{cadō}.

\latin{concīdō}, see \latin{caedō}.

\latin{concinō}, see \latin{canō}.

\latin{concipiō}, see \latin{capiō}.

\latin{conclūdō}, see \latin{claudō}.

\latin{concupīscō}, \english{long for}, -cupīscere,
-cupīvī, -cupītum [cupiō].

\latin{concutiō}, see \latin{quatiō}.

\latin{condō}, \english{establish}, condere, condidī, conditum
      [cf.~dō].  Perf.\ of abs-condō, abs-condī.

\latin{cōnficiō}, see \latin{faciō}.

\latin{cōnfiteor}, see \latin{fateor}.

\latin{cōnfringō}, see \latin{frangō}.

\latin{congredior}, see \latin{gradior}.

\latin{congruō}, \english{agree}, congruere, congruī [con-gruō;
  cf.~in-gruō].

\latin{coniciō}, see \latin{iaciō}.

\latin{cōnīveō}, \english{blink}, cōnīvēre
(cōnīxī, cōnīvī, rare).

\latin{conquīrō}, see \latin{quaerō}.

\latin{cōnspiciō}, see \suffix{-spiciō}.

\latin{cōnstituō}, see \latin{statuō}.

\latin{cōnsulō}, \english{consult}, cōnsulere, cōnsuluī,
cōnsultum.

\latin{contineō}, see \latin{teneō}.

\latin{contingō}, see \latin{tangō}.

\latin{coquō}, \english{cook}, coquere, coxī, coctum.

\latin{corripiō}, see \latin{rapiō}.

\latin{crēdō}, \english{believe}, crēdere, crēdidī,
crēditum [cf.~dō].

\latin{crepō}, \english{rattle}, crepāre, crepuī
(crepāvī rare), crepitum.

\latin{crēscō}, \english{grow}, crēscere, crēvī, crētum.

\latin{cubō}, \english{recline}, cubāre, cubuī (cubāvī
rare), cubitum.

\latin{cūdō}, \english{strike}, cūdere, -cūdī, -cūsum.

\suffix{-cumbō}, \english{recline}, -cumbere, -cubuī, -cubitum.
Ac-, con-, etc.

\latin{cupiō}, \english{desire}, cupere, cupīvī,
cupītum.

\latin{currō}, \english{run}, currere, cucurrī, cursum.  In
cpds.\ Perf.\ -cucurrī and -currī, the latter more common.

\bigskip

\latin{dēbeō}, see \latin{habeō}.

\latin{decet}, \english{it is fitting}, decēre, decuit.  Impers.

\latin{dēfendō}, see \suffix{-fendō}.

\latin{dēgō}, see \latin{agō}.

\latin{dēleō}, \english{destroy}, dēlēre, dēlēvī,
dēlētum.

\latin{dēmō}, see \latin{emō}.

\latin{dīcō}, \english{say}, dīcere, dīxī, dictum.
Imperat.\ dīc, \xref[1]{164}.

\latin{diribeō}, see \latin{habeō}.

\latin{discō}, \english{learn}, discere, didicī.

\latin{discutiō}, see \latin{quatiō}.

\latin{distinguō}, see \latin{stinguō}.

\latin{dīvidō}, \english{divide}, -videre, -vīsī,
-vīsum.

\latin{dō}, \english{give}, dare, dedī, datum.
Irreg. \xref{197}.  So circum-dō, satis-dō, etc.  But ab-dō,
ab-dere, ab-didī, ab-ditum; so ad-dō, con-dō, crēdō,
dē-dō, dī-dō, ē-dō, in-dō, ob-dō, per-dō,
prō-dō, red-dō, sub-dō, trā-dō, vēn-dō; in these is
contained also, in part, another verb~-dō, meaning \english{put},
and related to faciō.

\latin{doceō}, \english{teach}, docēre, docuī, doctum.

\latin{doleō}, \english{suffer}, dolēre, doluī, dolitūrus.

\latin{domō}, \english{tame}, domāre, domuī, domitum.

\latin{dormiō}, \english{sleep}, dormīre, dormīvī,
dormītum.

\latin{dūcō}, \english{lead}, dūcere, dūxī, ductum.
Imperat.\ dūc, \xref[1]{164}.

\bigskip

\latin{edō}, \english{eat}, ēsse, ēdī, ēsum (but
com-ēstum beside com-ēsum).  Irreg.  \xref{196}.

\latin{ēdō}, see \latin{dō}.

\latin{egeō}, \english{want}, egēre, eguī.  Ind-igeō,
ind-igēre, ind-iguī [ind-, \xref[9]{51}].

\latin{ēliciō}, see \suffix{-liciō}.

\latin{ēmineō}, \english{project}, ēminēre, ēminuī
      [cf.~ēminus].

\latin{emō}, \english{take}, \english{buy}, emere, ēmī,
ēmptum.  Co-emō, inter-emō or inter-imō, per-emō or
per-imō, ad-imō, dir-imō, ex-imō, red-imō.  Cf.~also
dēmō, \english{take away}, dēmere, dēmpsī, dēmptum; so
cōmō, prōmō, sūmō.

\latin{eō}, \english{go}, īre, iī (īvī), itum.
Irreg.  \xref{194}.  So in cpds., except ambiō, \english{go around},
ambīre, ambīvī, ambītum.

\latin{ēsuriō}, \english{be hungry}, ēsurīre,
ēsurītūrus [edō, \xref[3]{212}].

\latin{excellō}, see \suffix{-cellō}.

\latin{excutiō}, see \latin{quatiō}.

\latin{exerceō}, see \latin{arceō}.

\latin{exolēscō}, see \latin*{alēscō}.

\latin{experior}, see \suffix{-perior}.

\latin{explōdō}, see \latin{plaudō}.

\latin{exstinguō}, see \suffix{-stinguō}.

\latin{exuō}, \english{take off}, exuere, exuī, exūtum
      [ex-uō; cf.~ind-uō].

\bigskip

\latin{facessō}, \english{fulfil}, \english{depart}, facessere,
facessīvī (facessī), facessītum [faciō,
  \xref[4]{212}].

\latin{faciō}, \english{make}, facere, fēcī, factum.
Imperat.\ fac, \xref[1]{164}; faxō, faxim, \xref[5]{163}.  For
passive, see fīō.  So bene-faciō, cale-faciō, etc.,
\xref[3]{31}; \xref[3]{218}\emend{56}{,}{.}
But in prepositional cpds.\ -ficiō, -ficere, -fēcī, -fectum.

\latin{fallō}, \english{deceive}, fallere, fefellī (falsus, Adj.).
Re-, Perf.\ re-fellī [*fal-nō, \xref[\emph{D}]{168}.]

\latin{farciō}, \english{stuff}, farcīre, farsī, fartum
(farctum rare).  Cpds.\ -ferciō or -farciō, -fertum.

\latin{fateor}, \english{confess}, fatērī, fassus sum.  Dep.
Cpds.\ -fiteor, -fitērī, -fessus sum.

\latin{faveō}, \english{favor}, favēre, fāvī, fautum.

\suffix{-fendō}, \english{strike}, -fendere, -fendī,
-fēnsum.  Dē-, of-.

\latin{feriō}, \english{strike}, ferīre.

\latin{ferō}, \english{carry}, ferre, tulī (tetulī),
lātum.  Irreg.  \xref{193}.  So cpds., e.g.,

\begin{cpds}

ad-ferō, at-tulī, al-lātum (ad-lātum);

au-ferō, abs-tulī, ab-lātum;

cōn-ferō, con-tulī, con-lātum (col-lātum);

dif-ferō, dis-tulī, dī-lātum;

ef-ferō, ex-tulī, ē-lātum;

īn-ferō, in-tulī, in-lātum;

of-ferō, ob-tulī (rarely obs-tulī), ob-lātum\emend{57}{.}{;}

re-ferō, re-ttulī (\xref[1]{43}), re-lātum (rel-lātum).

\end{cpds}

\latin{ferveō}, \english{boil}, fervēre (fervī, ferbuī
rare), (fervō, fervere, poetical).

\latin{fīdō}, \english{trust}, fīdere, fīsus sum.  Semi-Dep.

\latin{fīgō}, \english{trust}, fīgere, fīxī,
fīxum.

\latin{findō}, \english{split}, findere, fidī, fissum.

\latin{fingō}, \english{mould}, fingere, fīnxī, fictum.

\latin{fīniō}, \english{finish}, fīnīre,
fīnīvī, fīnītum [fīnis].

\latin{fīō}, fierī, factus sum, used as passive of faciō.
Irreg.  \xref{195}.

\latin{flectō}, \english{turn}, flectere, flexī, flexum [flec-tō,
  \xref[\emph{E}]{168}].

\latin{fleō}, \english{weep}, flēre, flēvī, flētum.

\suffix{-flīgō}, \english{dash}, -flīgere, -flīxī,
-flīctum. Ad-, cōn-, etc.

\latin{flō}, \english{blow}, flāre, flāvī, flātum.

\latin{flōreō}, \english{bloom}, flōrēre, flōruī
      [flōs].

\latin{fluō}, \english{flow}, fluere, flūxī (flūxus, Adj.).

\latin{fodiō}, \english{dig}, fodere, fōdī, fossum.

\latin{(for)}, \english{speak}, fārī, fātus sum.
Def.  \xref[3]{198}.

\latin{foveō}, \english{warm}, \english{cherish}, fovēre,
fōvī, fōtum.

\latin{frangō}, \english{break in pieces}, frangere, frēgī,
frāctum.  Cpds.\ -fringō, -fringere, -frēgī, -frāctum.

\latin{fremō}, \english{growl}, fremere, fremuī.

\latin{frendō}, \english{crush}, frendere, frēsum (fressum).

\latin{fricō}, \english{rub}, fricāre, fricuī, frictum
(fricātum).

\latin{frīgeō}, \english{be cold}, frīgēre.

\latin{frīgēscō}, \english{grow cold}, frīgēscere,
-frīxī.

\latin{fruor}, \english{enjoy}, fruī, frūctus sum
(fruitūrus).  Dep.

\latin{fugiō}, \english{flee}, fugere, fūgī, fugitūrus.

\latin{fulciō}, \english{support}, fulcīre, fulsī, fultum.

\latin{fulgeō}, \english{flash}, fulgēre, fulsī (fulgō,
fulgere, poet.).

\latin{fundō}, \english{pour}, fundere, fūdī, fūsum.

\latin{fungor}, \english{perform}, fungī, fūnctus sum.  Dep.

\latin{furō}, \english{rage}, furere.

\bigskip

\latin{gaudeō}, \english{rejoice}, gaudēre, gāvīsus sum.
Semi-Dep.

\latin{gemō}, \english{groan}, gemere, gemuī.

\latin{gerō}, \english{carry}, gerere, gessī, gestum.

\latin{gignō}, \english{beget}, gignere, genuī, genitum
      [gi-gn-ō, \xref[\emph{B}]{168}].

\latin{glīscō}, \english{swell}, glīscere.

\latin{gradior}, \english{step}, gradī, gressus sum.  Dep.
Cpds.\ -gredior, -gredī, -gressus.

\bigskip

\latin{habeō}, \english{hold}, habēre, habuī, habitum.
Cpds.\ -hibeō, -hibēre, -hibuī, -hibitum.  Cf.~also praebeō
(rarely praehibeō), praebēre, praebuī, praebitum; dēbeō
(from dē-hibeō), dēbēre, dēbuī, dēbitum.

\latin{haereō}, \english{stick}, haerēre, haesī, haesūrus.

\latin{hauriō}, \english{drain}, haurīre, hausī, haustum
(hausūrus).  (Imperf.\ haurībant, \xref[4]{164}.)

\latin{havē}, see \latin{avē}.

\latin{hebeō}, \english{be blunt}, hebēre.

\latin{hīscō}, \english{gape}, hīscere [hiō].

\latin{horreō}, \english{bristle}, \english{be afraid}, horrēre,
horruī.

\bigskip

\latin{iaceō}, \english{lie}, iacēre, iacuī.

\latin{iaciō}, \english{throw}, iacere, iēcī, iactum.  So
super-iaciō.  But in other cpds.\ -iciō, -icere, -iēcī,
-iectum.  For the length of the first syllable in cpds.,
see~\xref[1]{30}.

\latin{īcī}, \english{struck}, ictum (īcō, īcere, early
Latin).

\latin{imbuō}, \english{wet}, imbuere, imbuī, imbūtum.

\latin{immineō}, \english{project}, imminēre [cf.\ ē-mineō].

\latin{indigeō}, see \latin{egeō}.

\latin{indulgeō}, \english{be kind}, indulgēre, indulsī.

\latin{induō}, \english{put on}, induere, induī, indūtum
      [ind-uō; cf.\ ex-uō].

\latin{ingruō}, \english{fall upon}, ingruere, ingruī
      [in-gruō; cf.\ con-gruō.]

\latin{inquam}, \english{say}.  Def.  \xref[2]{198}.

\latin{inveterāscō}, \english{become fixed}, -āscere, -āvī
      [in-veterō, vetus].

\latin{iubeō}, \english{order}, iubēre, iussī, iussum.

\latin{iungō}, \english{join}, iungere, iūnxī, iūnctum.

\latin{iuvō}, \english{aid}, iuvāre, iūvī, iūtum
(iuvātūrus, but ad-iūtūrus).

\bigskip

\latin{lābor}, \english{slip}, lābī, lāpsus sum.  Dep.

\latin{lacessō}, \english{excite}, lacessere, lacessīvī,
lacessītum [laciō; cf.\ -liceō].

\latin{laedō}, \english{hurt}, laedere, laesī, laesum.
Cpds.\ -līdō, -līdere, -līsī, -līsum.

\latin{lambō}, \english{lick}, lambere (lambuī rare).

\latin{langueō}, \english{be weak}, languēre.

\latin{languēscō}, \english{become weak}, languēscere, languī.

\latin{largior}, \english{lavish}, largīrī, largītus
sum.  Dep.  [largus.]

\latin{lateō}, \english{lie hid}, latēre, latuī.

\latin{lavō}, \english{bathe}, lavāre, lāvī, lautum or
lōtum (rarely lavātum).  (Early and poet.\ lavō, lavere.)
Ē-lavō.  Cf.~also -luō.

\latin{legō}, \english{collect}, \english{read}, legere, lēgī,
lēctum.  So ad-legō, inter-legō, prae-legō, re-legō,
sub-legō, trāns-legō; pel-legō or per-legō (also pel-ligō,
per-ligō).  But intel-legō, intel-legere, intel-lēxī,
intel-lēctum, and so neg-legō (rarely Perf.\ intellēgī,
neglēgī); dī-ligō, dī-ligere, dī-lēxī,
dī-lēctum; col-ligō, col-ligere, col-lēgī,
col-lēctum, and so dē-ligō, ē-ligō, sē-ligō.

\latin{libet} (early lubet), \english{it is pleasing}, libēre,
libuit or libitum est.  Impers.

\latin{liceō}, \english{be for sale}, licēre, licuī.

\latin{liceor}, \english{bid}, licērī, licitus sum.  Dep.

\latin{licet}, \english{it is permitted}, licēre, licuit or licitum
est.  Impers.

\suffix{-liciō}, \english{lure}, -licere, -lexī, -lectum.
       [*laciō; cf.~lacessō.]  So ad-liciō, in-liciō,
       pel-liciō (per-liciō).  But ē-liciō, ē-licere,
       ē-licuī, ē-licitum.

\latin{lingō}, \english{lick}, lingere, līnxī, līnctum.

\latin{linō}, \english{besmear}, linere, lēvī, litum.

\latin{linquō}, \english{leave}, linquere, līquī, -lictum.

\latin{liqueō}, \english{be fluid}, liquēre, licuī.

\latin{līquor}, \english{be fluid}, līquī.  Dep.

\latin{loquor}, \english{speak}, loquī, locūtus sum.  Dep.

\latin{lūceō}, \english{be light}, lūcēre, lūxī [lūx].

\latin{lūdō}, \english{play}, lūdere, lūsī, lūsum.

\latin{lūgeō}, \english{mourn}, lūgēre, lūxī.

\latin{luō}, \english{loose}, \english{atone for}, luere, luī.

\latin{-luō}, \english{wash}, -luere, -luī, -lūtum [lavō].
Ab-, ad-, con-, etc.

\bigskip

\latin{madeō}, \english{be wet}, madēre, maduī.

\latin{maereō}, \english{grieve}, maerēre.

\latin{mālō}, \english{prefer}, mālle, māluī [volō].  Irreg.
\xref{192}.

\latin{mandō}, \english{chew}, mandere, mandī, mānsum.

\latin{maneō}, \english{remain}, manēre, mānsī, mānsum.

\latin{medeor}, \english{remedy}, medērī.  Dep.

\latin{meminī}, \english{remember}. Def. \xref[1]{199}.

\latin{mentior}, \english{deceive}, mentīrī, mentītus sum.  Dep.

\latin{mereō}, \english{deserve}, merēre, meruī, meritum; also
Dep.~mereor.

\latin{mergō}, \english{dip}, mergere, mersī, mersum.

\latin{mētior}, \english{measure}, mētīrī, mēnsus sum.  Dep.

\latin{metō}, \english{mow}, metere, messuī, messum.

\latin{metuō}, \english{fear}, metuere, metuī.

\latin{micō}, \english{shake}, micāre, micuī.  So ē-,
inter-; but dī-micō, -āre, -āvī (-uī rare), -ātum.

\latin{mingō}, \english{make water}, mingere, mīnxī, mictum.

\latin{minuō}, \english{lessen}, minuere, minuī, minūtum.

\latin{misceō}, \english{mix}, miscēre, miscuī, mixtum.

\latin{misereor}, \english{pity}, miserērī, miseritus sum (misertus).
Dep.

\latin{miseret}, \english{excites pity in}, miseruit.  Impers.

\latin{mittō}, \english{send}, mittere, mīsī, missum.

\latin{molō}, \english{grind}, molere, moluī, molitum.

\latin{moneō}, \english{advise}, monēre, monuī, monitum.

\latin{mordeō}, \english{bite}, mordēre, momordī, morsum.

\latin{morior}, \english{die}, morī (sometimes morīrī, \xref[1]{165}),
mortuus sum (moritūrus). Dep.

\latin{moveō}, \english{move}, movēre, mōvī, mōtum.

\latin{mulceō}, \english{stroke}, mulcēre, mulsī, mulsum.

\latin{mulgeō}, \english{milk}, mulgēre, mulsī, mulsum.

\bigskip

\latin{nancīscor}, \english{get}, nancīscī, nactus or nānctus sum.
Dep.

\latin{nāscor}, \english{be born}, nāscī, nātus sum.  Dep.

\latin{necō}, \english{slay}, necāre, necāvī (necuī rare), necātum.
Ē-necō (ē-nicō rare), ē-necāre, ē-necuī, ē-nectum (ē-nicāvī, ē-necātum
rare).

\latin{nectō}, \english{bind}, nectere, nexuī (nexī), nexum [nec-tō,
  \xref[\emph{E}]{168}].

\latin{neglegō}, see \latin{legō}.

\latin{neō}, \english{spin}, nēre, nēvī.

\latin{nequeō}, see \latin{queō}.

\latin{ninguit (ningit)}, \english{it snows}.  Impers.

\latin{niteō}, \english{shine}, nitēre, nituī.

\latin{nītor}, \english{lean on}, \english{strive}, nītī, nīxus or
nīsus sum.

\latin{nō}, \english{swim}, nāre, nāvī.

\latin{noceō}, \english{harm}, nocēre, nocuī, nocitum.

\latin{nōlō}, \english{will not}, nōlle, nōluī [volō].  Irreg.
\xref{192}.

\latin{nōscō} (early gnōscō), \english{know}, nōscere, nōvī, nōtum.
(For nōsse, nōram, etc., see~\xref[2]{163}.)  So inter-, per-, prae-,
ignōscō; but agnitum from agnōscō (also ad-gnōscō) and cognitum from
cognōscō.

\latin{nūbō}, \english{veil}, \english{marry}, nūbere, nūpsī, nūptum.

\suffix{-nuō}, \english{nod}, -nuere, -nuī.  Ab-, ad- (an-), in-, re-.

\bigskip

\latin{oblīvīscor}, \english{forget}, oblīvīscī, oblītus sum.  Dep.

\latin{oboediō}, \english{obey}, oboedīre, oboedīvī, oboedītum.

\latin{obsolēscō}, \english{wear out}, \english{go out of use},
obsolēscere, obsolēvī, obsolētum [alēscō or soleō, or both].

\latin{occulō}, \english{hide}, occulere, occuluī, occultum [*celō;
  cf.~cēlō, cēlāre].

\latin{ōdī}, \english{hate}, ōsūrus.  Def. \xref[1]{199}.

\latin{oleō}, \english{smell}, olēre, oluī.

\latin{operiō}, \english{cover}, operīre, operuī, opertum.

\latin{oportet}, \english{it is necessary}, oportēre, oportuit.
Impers.

\latin{opperior}, see \suffix{-perior}.

\latin{ōrdior}, \english{begin}, ōrdīrī, ōrsus sum.  Dep.

\latin{orior}, \english{arise}, orīrī, ortus.  Dep.  Pres.\ Syst.,
except Infin., usually of Third Conj., \xref[1]{165}.

\bigskip

\latin{pacīscor}, \english{bargain}, pacīscī, pactus sum.  Dep.
dē-pecīscor, dē-pectus, or dē-pacīscor, dē-pactus.

\latin{paenitet}, \english{it repents}, paenitēre, paenituit.  Impers.

\latin{palleō}, \english{be pale}, pallēre, palluī.

\latin{pandō}, \english{open}, pandere, pandī, passum or pānsum.
Dis-pendō or dis-pandō, dis-pessum or dis-pānsum; ex-pandō, ex-pānsum
(ex-passum).

\latin{pangō}, \english{fix}, pangere, pānxī and pēgī, pāctum.  Also
Perf.\ pepigī, \english{agree}; cf.~pacīscor.  Cpds.\ -pingō,
-pingere, -pēgī, -pāctum.

\latin{parcō}, \english{spare}, parcere, pepercī (parsī), parsūrus.
Com-percō (com-parcō), com-persī.

\latin{pāreō}, \english{appear}, pārēre, pāruī.

\latin{pariō}, \english{bring forth}, parere, peperī, partum
(paritūrus).

\latin{partior}, \english{divide}, partīrī, partītus sum.  Dep.
      [pars.]

\latin{parturiō}, \english{be in travail}, parturīre, parturīvī
      [pariō, \xref[3]{212}].

\latin{pāscō}, \english{feed}, pāscere, pāvī, pāstum.

\latin{pateō}, \english{be open}, patēre, patuī.

\latin{patior}, \english{endure}, patī, passus sum.  Dep.  per-petior,
per-petī, per-pessus.

\latin{paveō}, \english{fear}, pavēre, pāvī.

\latin{paviō}, \english{strike}, pavīre.

\latin{pectō}, \english{comb}, pectere, pexī, pexum [pectō,
  \xref[\emph{E}]{168}].

\latin{pellō}, \english{strike}, pellere, pepulī, pulsum [*pel-nō,
  \xref[\emph{D}]{168}].  In cpds.\ Perf.\ -pulī; re-ppulī
(\xref[1]{43}) from re-pellō.

\latin{pendeō}, \english{hang down}, pendēre, pependī.  In
cpds.\ Perf.\ -pendī, Partic.\ prō-pēnsum.

\latin{pendō}, \english{weigh}, pendere, pependī, pēnsum.  In
cpds.\ Perf.\ -pendī.

\latin{percellō}, \english{cast down}, -cellere, -culī, -culsum.

\latin{perdō}, \english{destroy}, perdere, perdidī, perditum [dō].

\latin{pergō}, see \latin{regō}.

\suffix{-periō}, \latin{-perior}:

\begin{cpds}

\latin{com-periō}, \english{learn}, -perīre, -perī, -pertum.

\latin{com-perior}, \english{learn}, -perīrī, -pertus sum.  Dep.

\latin{ex-perior}, \english{try}, -perīrī, -pertus sum.  Dep.

\latin{op-perior}, \english{await}, -perīrī, -pertus sum.  Dep.

\latin{re-periō}, \english{find}, re-perīre, re-pperī (\xref[1]{43}),
re-pertum.

\end{cpds}

\latin{petō}, \english{seek}, petere, petīvī or petiī, petītum.

\latin{piget}, \english{it grieves}, pigēre, piguit or pigitum est.
Impers.

\latin{pingō}, \english{paint}, pingere, pīnxī, pictum.

\latin{pīnsō}, \english{pound}, pīnsere, pīnsuī (pīnsiī), pīstum
(pīnsītum).

\latin{placeō}, \english{please}, placēre, placuī, placitum.
Com-placeō, per-placeō, but dis-pliceō.

\latin{plangō}, \english{strike}, plangere, plānxī, plānctum.

\latin{plaudō}, \english{clap}, plaudere, plausī, plausum.  Ap-plaudō,
circum-plaudō, but ex-plōdō, sup-plōdō.

\latin{plectō}, \english{plait}, plectere, plexī, plexum [plec-tō,
  \xref[\emph{E}]{168}].

\latin{-plector}, \english{embrace}, -plectī, -plexus sum.  Dep.  Am-,
circum-, com-.

\latin{-pleō}, \english{fill up}, -plēre, -plēvī, -plētum.  Com-, ex-,
im-, etc.

\latin{plicō}, \english{fold up}, plicāre, -plicāvī or -plicuī,
-plicātum or -plicitum.

\latin{pluit}, \english{it rains}, pluere, pluit and plūvit.  Impers.

\latin{polleō}, \english{be powerful}, pollēre.

\latin{polliceor}, see \latin{liceor}.

\latin{polluō}, \english{soil}, polluere, polluī, pollūtum [cf.~luēs].

\latin{pōnō}, \english{place}, pōnere, posuī, positum [*po-s(i)nō].

\latin{porriciō}, \english{offer in sacrifice}, porricere, porrectum
      [iaciō; form influenced by porrigō].

\latin{poscō}, \english{demand}, poscere, poposcī.

\latin{possideō}, see \latin{sedeō}.

\latin{possum}, \english{be able}, posse, potuī.  Irreg.  \xref{191}.

\latin{potior}, \english{become master of}, potīrī, potītus sum.
Dep. [potis.]  Pres.\ Syst., except Infin., usually of Third Conj.,
\xref[1]{165}.

\latin{pōtō}, \english{drink}, pōtāre, pōtāvī, pōtum (pōtātum).

\latin{praebeō}, see \latin{habeō}.

\latin{prandeō}, \english{lunch}, prandēre, prandī, prānsum.

\latin{prehendō}, \english{seize}, prehendere, prehendī, prehēnsum,
and prēndō, prēndere, prēndī, prēnsum [prae-hendō, pre-hendō
  (p.~\pageref{ftn:9:}, footnote), prēndō].

\latin{premō}, \english{press}, premere, pressī, pressum.
Cpds.\ -primō, -primere, -pressī, -pressum.

\latin{proficīscor}, \english{set out}, proficīscī, profectus sum.
Dep. [faciō.]

\latin{profiteor}, see \latin{fateor}.

\latin{prōmineō}, \english{project}, prōminēre, prōminuī
      [cf.~ē-mineō].

\latin{prōmō}, \english{produce}, prōmere, prōmpsī, prōmptum [emō].

\latin{pudet}, \english{it shames}, pudēre, puduit or puditum est.
Impers.

\latin{pungō}, \english{prick}, pungere, pupugī, pūnctum.  In
cpds.\ Perf.\ -pūnxī.

\bigskip

\latin{quaerō}, \english{seek}, quaerere, quaesīvī, quaesītum.
Cpds.\ -quīrō, etc.

\latin{quaesō}, \english{beseech}, quaesumus.  Def.  \xref{200}.

\latin{quatiō}, \english{shake}, quatere, \na, quassum.
Cpds.\ -cutiō, -cutere, -cussī, -cussum.

\latin{queō}, \english{can}, quīre, quīvī, quitum,
\xref[\emph{c}]{194}.

\latin{queror}, \english{complain}, querī, questus sum.  Dep.

\latin{quiēscō}, \english{become quiet}, quiēscere, quiēvī (quiētus,
Adj.).

\bigskip

\latin{rādō}, \english{scrape}, rādere, rāsī, rāsum.

\latin{rapiō}, \english{seize}, rapere, rapuī, raptum.  Cpds.\ -ripiō,
-ripere, -ripuī, -reptum.  For sur-ripiō early Latin has sur-rupiō,
Perf.\ surrupuit and surpuit.

\latin{regō}, \english{direct}, regere, rēxī, rēctum.  Cpds.\ -rigō,
-rigere, rēxī, -rēctum.  But pergō (*per-(ri)gō), pergere, per-rēxī,
per-rēctum; surgō (early sur-rigō), surgere, sur-rēxī, sur-rēctum;
rarely porgō beside por-rigō.

\latin{reminīscor}, \english{remember}, reminīscī.  Dep. [meminī.]

\latin{reor}, \english{think}, rērī, ratus sum.  Dep.

\latin{rēpō}, \english{creep}, rēpere, rēpsī.

\latin{rīdeō}, \english{laugh}, rīdēre, rīsī, rīsum.

\latin{rigeō}, \english{be stiff}, rigēre, riguī.

\latin{rōdō}, \english{gnaw}, rōdere, rōsī, rōsum.

\latin{rudō}, \english{roar}, rudere.

\latin{rumpō}, \english{break}, rumpere, rūpī, ruptum.

\latin{ruō}, \english{tumble down}, ruere, ruī, -rutum (ruitūrus).

\bigskip

\latin{saepiō}, \english{hedge in}, saepīre, saepsī, saeptum.

\latin{saliō}, \english{leap}, salīre, saluī.  Cpds.\ -siliō, -silīre,
-siluī (early -suluī; late -siliī, -silīvī).

\latin{salvē}, \english{hail}.  Def.  \xref{200}.

\latin{sanciō}, \english{ratify}, sancīre, sānxī, sānctum.

\latin{sapiō}, \english{taste of}, \english{be wise}, sapere, sapīvī.
Cpds.\ -sipiō, etc.

\latin{sarciō}, \english{repair}, sarcīre, sarsī, sartum.

\latin{scabō}, \english{scrape}, scabere, scābī (rare verb).

\latin{scalpō}, \english{scrape}, scalpere, scalpsī, scalptum.

\latin{scandō}, \english{climb}, scandere. Cpds.\ -scendō, -scendere,
-scendī, -scēnsum.

\latin{scindō}, \english{tear}, scindere, scidī, scissum.

\latin{sciō}, \english{know}, scīre, scīvī, scītum.  (Imperf.\ scībam,
Fut.\ scībō, \xref[4, 5]{164}.)

\latin{scīscō}, \english{approve}, scīscere, scīvī, scītum.

\latin{scrībō}, \english{write}, scrībere, scrīpsī, scrīptum.

\latin{sculpō}, \english{carve}, sculpere, sculpsī, sculptum.

\latin{secō}, \english{cut}, secāre, secuī, sectum.

\latin{sedeō}, \english{sit}, sedēre, sēdī, sessum.  Circum-sedeō,
super-sedeō; but in other cpds.\ -sideō, -sidēre, -sēdī, -sessum.

\latin{sentiō}, \english{feel}, sentīre, sēnsī, sēnsum.

\latin{sepeliō}, \english{bury}, sepelīre, sepelīvī, sepultum.

\latin{sequor}, \english{follow}, sequī, secūtus sum.  Dep.

\latin{serō}, \english{sow}, serere, sēvī, satum.  Cpds.\ -serō,
-serere, -sēvī, -situm [*si-sō, \xref[\emph{B}, \emph{a}]{168}].

\latin{serō}, \english{entwine}, serere, -seruī, sertum.

\latin{serpō}, \english{creep}, serpere, serpsī.

\latin{sīdō}, \english{sit down}, sīdere, -sēdī (-sīdī), -sessum.

\latin{sileō}, \english{be still}, silēre, siluī.

\latin{sinō}, \english{permit}, sinere, sīvī or siī, situm.
(Perf.\ Subj.\ sīrīs, sīrit beside sierīs, sīverīs; \xref[5]{163}.)

\latin{sistō}, \english{sit}, sistere, stitī, statum.

\latin{soleō}, \english{be wont}, solēre, \na, solitus sum.  Semi-Dep.

\latin{solvō}, \english{release}, solvere, solvī, solūtum [luō].

\latin{sonō}, \english{sound}, sonāre, sonuī, sonātūrus (sonō, sonere,
rare).

\latin{sorbeō}, \english{suck in}, sorbēre, surbuī (rarely -sorpsī).

\latin{spargō}, \english{scatter}, spargere, sparsī, sparsum.
Cpds.\ -spergō, -spergere, -spersī, -spersum.

\latin{spernō}, \english{scorn}, spernere, sprēvī, sprētum.

\suffix{-spiciō}, \english{spy}, -spicere, -spexī, -spectum [speciō, a
  rare verb].  Aspiciō (ad-), circum-, cōn-, etc.

\latin{splendeō}, \english{shine}, splendēre.

\latin{spondeō}, \english{promise}, spondēre, spopondī, spōnsum.  In
cpds.\ Perf.\ -spondī.

\latin{spuō}, \english{spit}, spuere, -spuī, -spūtum.

\latin{statuō}, \english{set}, statuere, statuī, statūtum [status].
Cpds.\ -stituō, -stituere, -stituī, -stitūtum.

\latin{sternō}, \english{spread out}, sternere, strāvī, strātum.

\latin{stertō}, \english{snore}, stertere, -stertuī.

\latin{stinguō}, \english{prick}, \english{put out}, stinguere,
-stīnxī, -stīnctum.  Distinguō, ex-, etc.

\latin{stō}, \english{stand}, stāre, stetī, stātūrus.  In
cpds.\ Perf.\ -stitī, e.g. prae-stitī, re-stitī, etc.; but
anti-stetī, circum-stetī, super-stetī.  Partic.\ prae-stitum and
prae-stātum.

\latin{strepō}, \english{make a noise}, strepere, strepuī.

\latin{strīdeō}, \english{hiss}, strīdēre, strīdī.  Also strīdō,
strīdere.

\latin{stringō}, \english{bind tight}, stringere, strīnxī, strictum.

\latin{struō}, \english{heap up}, struere, strūxī, strūctum.

\latin{studeō}, \english{be eager}, studēre, studuī.

\latin{stupeō}, \english{be dazed}, stupēre, stupuī.

\latin{suādeo}, \english{advise}, suādēre, suāsī, suāsum.

\latin{suēscō}, \english{become used}, suēscere, suēvī, suētum.

\latin{sūgō}, \english{suck}, sūgere, sūxī, sūctum.

\latin{sum}, \english{be}, esse, fuī.  Irreg.  \xref{153}.

\latin{sūmō}, \english{take}, sūmere, sūmpsī, sūmptum [emō].

\latin{suō}, \english{sew}, suere, suī, sūtum.

\latin{surgō}, see \latin{regō}.

\bigskip

\latin{taceō}, \english{be silent}, tacēre, tacuī, tacitum.
Cpds.\ -ticeō, etc.

\latin{taedet}, \english{it disgusts}, taedēre, taesum est.  Impers.

\latin{tangō}, \english{touch}, tangere, tetigī, tāctum.
Cpds.\ -tingō, -tingere, -tigī, -tāctum.

\latin{tegō}, \english{cover}, tegere, tēxī, tēctum.

\latin{temnō}, \english{scorn}, temnere, -tempsī, -temptum.

\latin{tendō}, \english{stretch}, tendere, tetendī, tentum (late
tēnsum, but extēnsum, ostēnsum common beside extentum, ostentum).  In
cpds.\ Perf.\ -tendī.

\latin{teneō}, \english{hold}, tenēre, tenuī.  Cpds.\ -tineō, -tinēre,
-tinuī, -tentum.

\latin{tergeō}, \english{wipe}, tergēre, tersī, tersum (tergō, tergere
rare).

\latin{terō}, \english{rub}, terere, trīvī, trītum.

\latin{terreō}, \english{frighten}, terrēre, terruī, territum.

\latin{texō}, \english{weave}, texere, texuī, textum.

\latin{timeō}, \english{be afraid}, timēre, timuī.

\latin{tinguō} (\latin{tingō}), \english{wet}, tinguere, tīnxī,
tīnctum.

\latin{tollō}, \english{lift}, tollere, sus-tulī, sub-lātum.
      [*tol-nō, \xref[\emph{D}]{168}.]

\latin{tondeō}, \english{shear}, tondēre, \na, tōnsum.  Perf.\ of
at-tondeō, at-tondī; of dē-tondeō, dē-tondī (dē-totondī rare).

\latin{tonō}, \english{thunder}, tonāre, tonuī (at-tonitus, Adj.).
Usually impers.

\latin{torqueō}, \english{twist}, torquēre, torsī, tortum.

\latin{torreō}, \english{dry up}, torrēre, torruī, tostum.

\latin{trahō}, \english{draw}, trahere, trāxī, tractum.

\latin{tremō}, \english{tremble}, tremere, tremuī.

\latin{tribuō}, \english{assign}, tribuere, tribuī, tribūtum.

\latin{trūdō}, \english{shove}, trūdere, trūsī, trūsum.

\latin{tueor}, \english{watch}, tuērī, tūtus sum.  Dep.

\latin{tumeō}, \english{be swollen}, tumēre.

\latin{tundō}, \english{pound}, tundere, (tutudī), tūnsun or tūsum.
Perf.\ re-ttudī (\xref[1]{43}) from re-tundō.

\bigskip

\latin{ulcīscor}, \english{avenge}, ulcīscī, ultus sum.  Dep.

\latin{ugeō}, \english{push}, urgēre, ursī.

\latin{ūrō}, \english{burn}, ūrere, ussī, ustum.  Note amb-ūrō and
(formed after this) comb-ūrō.

\latin{ūtor}, \english{use}, ūtī, ūsus sum.  Dep.

\bigskip

\latin{vādō}, \english{go}, vādere, -vāsī, -vāsum.

\latin{valeō}, \english{be strong}, valēre, valuī, valitūrus.

\latin{vehō}, \english{carry}, vehere, vexī, vectum.

\latin{vellō}, \english{tear}, vellere, vellī (vulsī), vulsum.

\latin{vēndō}, \english{sell}, vēndere, vēndidī [vēnum + dō].

\latin{vēneō}, \english{be sold}, vēnīre, vēniī [vēnum + eō].

\latin{veniō}, \english{come}, venīre, vēnī, ventum.

\latin{vereor}, \english{revere}, verērī, veritus sum.  Dep.

\latin{vergō}, \english{slope}, vergere.

\latin{verrō}, \english{sweep}, verrere, verrī, versum.  Early
vorrō, etc.

\latin{vertō}, \english{turn}, vertere, vertī, versum.  Early vortō,
etc.  Dep. re-vertor has Perf.\ re-vertī.

\latin{vēscor}, \english{feed upon}, vēscī.  Dep.

\latin{vesperāscō}, \english{become evening}, vesperāscere, vesperāvī
      [vesper].

\latin{vetō}, \english{forbid}, vetāre, vetuī, vetitum.  Early votō,
etc.

\latin{videō}, \english{see}, vidēre, vīdī, vīsum.

\latin{vigeō}, \english{be strong}, vigēre, viguī.

\latin{vinciō}, \english{bind}, vincīre, vīnxī, vīnctum.

\latin{vincō}, \english{conquer}, vincere, vīcī, victum.

\latin{vīsō}, \english{look after}, vīsere, vīsī, vīsum.

\latin{vīvō}, \english{live}, vīvere, vīxī, -vīctum.

\latin{volō}, \english{wish}, velle, voluī.  Irreg.  \xref{192}.

\latin{volvō}, \english{roll}, volvere, volvī, volūtum.

\latin{vomō}, \english{vomit}, vomere, vomuī, vomitum.

\latin{voveō}, \english{vow}, vovēre, vōvī, vōtum.

\end{verbcat}

\Unnumbered{Index}
\markboth{Index}{Index}
\markthird{}

\contentsentry{B}{Index}

The references are to sections, unless the page (p.)\ is mentioned.
The principal abbreviations used are:
aor.\ = aorist or aoristic;
cl.\ = clause;
constr.\ = construction;
cpd.\ = compound;
compar.\ = comparative;
dep.\ = dependent;
det.\ = determinative;
descr.\ = descriptive;
end.\ = ending;
expr.\ = expressed;
ftn.\ = footnote;
imper.\ = imperative;
imperf.\ = imperfect;
ind.\ = indirect;
n.\ = note;
narr.\ = narrative;
opt.\ = optative;
partic.\ = participle;
reg.\ = regularly;
subj.\ = subjunctive;
vol.\ = volitive;
w.\ = with;
wh. = which.

\begin{algindex}

\latin{ā}, \latin{ab}, \latin{abs},
    in cpds., \xref[1]{51};
    use, \xref{405} and~\emph{a}, \xref[1, 2]{406}, \xref{408}.

Ablative,
    \emph{Form}:
        abl. sing., decl.~III, in \ending{-e} or~\ending{i},
        \xref[\emph{a}]{75}, \xref[2]{88};
        of adjs., \xref[1]{118};
        in advs., \xref[1, 3, 4]{126};
        abl. pl., decl.~I, in \ending{-ābus}, \xref[4]{66};
        decl.~IV, in~\ending{-ubus}, \xref[1]{97}.
    \enskip
    \emph{Syntax}: see synopsis, \xref{404}.

Absolute tenses,
    \xref[2]{467}\versionA{, \xref{478}}\versionB*{, \xref[b, c]{477}}.

Absolute use of trans. verbs,
    \xref[\emph{a}]{289}.

Abstract nouns,
    form, \xref[2, 4]{206}, \xref[2]{207};
    defined, \xref[5]{240};
    w. concrete meaning, \ibid[\emph{a}];
    pl. of, \xref[n.]{103}, \xref[5, \emph{b}]{240}.

Absurd question,
    w. \latin{an}, \xref{236}.

\suffix{-ābus},
    in decl.~I, \xref[4]{66}.

\latin{ac},
    see \latin{atque}.

Accent,
    \xref{31}–\xref{33};
    in verse, \xref{645}.

Accompaniment, abl. of,
    \xref{418}–\xref{420}.

Accordance,
    abl. of, \xref{414};
    \latin{ut}-cl. of, \xref{562}.

Accusative,
    \emph{Form}:
        acc. sing. end., \xref[1, n.]{62};
        in \suffix{-im}, \xref[\emph{a}]{75}, \xref[1]{88};
        in \suffix{-a} in Greek nouns, \xref{95}, examples;
        acc. pl. in \suffix{-īs}, \xref[\emph{a}]{75}, \xref[3]{88},
            \xref[4]{118};
        acc. as adv., \xref[5,6,7]{126};
        acc. pl. neut. of adj. of decl.~III, \xref[1]{118}.
    \enskip
    \emph{Syntax}:
        see synopsis, \xref{379}.

“Accusing,” constrs. w.,
    \xref{342}, \xref{343}, \xref[1]{397}.

Acquiescence, how expr.,
    see Consent.

Act anticipated, \latin{antequam}, etc.,
    w. subj., \xref[4, \emph{a}\)–\emph{d}\)]{507};
    w. indic., \ibid[n.], \xref{571}.

Action, nouns of,
    \xref[2, 3]{206}.

Active,
    see Voice.

Actuality (fact), subj. of,
    \xref{520}, \xref{521}.

\latin{ad},
    forms in cpds., \xref[2]{51};
    w. acc., \xref{380}, \xref[6]{364};
    cpds. of, w. dat., \xref{376}.

\latin{adeō},
    meaning, \xref[7]{302};
    \latin{adeō ut}, \xref[2, \emph{a}]{521}.

\suffix{-adēs}, suffix,
    \xref[3]{207}.

Adjectives,
    \emph{Form}:
        decl.~I and~II, \xref{110}–\xref{112};
        decl.~III, \xref{113}–\xref{118};
        comparison, \xref{119}–\xref{123};
        pronom. adj., \xref{112}, \xref{143};
        derivation of, \xref{208}–\xref{210};
        numerals, \xref{130}–\xref{133};
        verbal adjs., \xref{146}.
    \enskip
    \emph{Syntax}:
        adj. defined, \xref{221};
        used w. force of advs., \xref{245};
        as substs., \xref{249}, \xref{250};
        pred., \xref{230};
        comparison of, \xref{241};
        denoting a part, \xref{244};
        agreement of, \xref{320};
        case w. \latin{nihil}, \latin{aliquid}, etc., \xref[\emph{a}]{346};
        neut. pl. of, w. gen., \xref{357}.

\latin{admoneō}, constr. w.,
    \xref{351}.

Adverbs,
    \emph{Form}:
        \xref{124}–\xref{127}, \xref{293};
        compar., \xref{128}, \xref{129};
        numeral advs., \xref{133}.
    \enskip
    \emph{Syntax}:
        \xref{294}–\xref{295};
        forces in comparison, \xref{300};
        two comparatives, \xref{301}.

Adverbial accusative,
    \ftn{209}{2};
    clauses, \xref{239};
    prefixes, \xref{51}, \xref[1]{218}.

Adversative conjuctions,
    \xref{310}.

Adversative idea,
    expr. by abl. absolute, \xref[5]{421};
    by partic., \xref[2]{604};
    by \latin{quī}-cl. w. subj., \xref{523}, w. indic., \xref[\emph{a}]{569};
    by \latin{cum}-cl. w. subj., \xref{525}, \xref{526};
    w. indic., \xref[\emph{a}]{569};
    by \latin{quamquam}-cl. w. indic., \xref{556}.

\latin{Aenēās}, decl., \xref{68}.

\latin{aequē ac},
    \xref[2, \emph{a}]{307};
    w. \latin{sī}, w. subj., \xref[3]{504}.

Agency, nouns of,
    \xref[1]{80}, \xref[1]{206}.

Agent,
    expr. by abl. w. \latin{ab}, \xref[1]{406};
    by dat., \xref{373}.

Agreement,
    of nouns, prons., adjs., and partics., \xref{316}–\xref{327};
    of pred. depending on infin. w. \latin{putor}, \latin{videor},
        etc., \xref[2]{590};
    poetic nom. in pred. of infin. for acc., \xref[\emph{a}]{592};
    agreement of verbs, \xref{328}–\xref{332};
    agreement w. antecedent of rel., \xref[1, \emph{a}]{328}.

\latin{aiō},
    conj., \xref[1]{198};
    \latin{ain}, form, \xref[{1, \emph{b}\), n.~3}]{231}.

\latin{aliquis}, \latin{aliquī},
    decl., \xref[2]{142};
    use, \xref[2]{276}.

\latin{aliter atque (ac)},
    \xref[2, \emph{a}]{307}.

\latin{alius},
    decl., \xref[\emph{a}]{112};
    meaning, \xref{279};
    as recipr. pron., \xref{265};
    w. \latin{atque} or \latin{ac}, \xref[2, \emph{a}]{307}.

Alliteration,
    \xref[19]{632}.

Alphabet, \xref{1}.

\latin{alter},
    decl., \xref[\emph{a}, \emph{c}]{112};
    meaning, \xref[1, 2]{279};
    as recipr. pron., \xref{265}.

Alternative questions, \xref{234}.

\prefix{amb-}, \prefix{am-},
    \xref[3]{51}, \xref[1, \emph{b}\)]{218}.

\latin{ambō},
    decl., \xref[2, n.]{131};
    use, \xref{277}.

\latin{amō}, conj., \xref{155}.

\latin{amplius}, w. abl., or without effect on case, \xref[\emph{d}]{416}.

\latin{an},
    \xref{234};
    in absurd questions, \xref{236}.

Anacolúthon, \xref[8]{631}.

Analogy, working of, \xref[4]{315}.

Anáphora, \xref[5]{632}.

Anástrophe, \xref[14]{631}.

\latin{Anchīsēs}, decl., \xref{68}.

\latin{Andromachē}, decl., \xref{68}.

Animals, gend. of names of, \xref[2]{59}.

\latin{animī}, \english{in mind}, \xref[\emph{c}]{449}.

Answers, forms of, \xref{232}, \xref{233}.

\latin{ante},
    forms in cpds., \xref[4]{51};
    w. acc., \xref{380};
    cpds. of, w. dat., \xref{376};
    in expressions of time, w. acc., \xref{380}, example,
                            or abl., \xref{424}, example;
    as adv., \xref[\emph{c}]{303}.

Antecedent,
    defined, \xref[\emph{a}]{281};
    omission of, \xref[1]{284};
    incomplete, \xref[1, \emph{a}]{521};
    repeated, \xref[4]{284};
    attracted to rel., \ibid[6], \xref{327}.

Antepenult,
    \xref[2]{31}.

\latin{antequam} or \latin{priusquam},
    w. subj., \xref[4, \emph{a}\)–\emph{d}\)]{507};
    w. fut. perf. or fut. indic., \ibid[n.];
    w. pres. indic., \xref{571};
    w. past teneses of indic., \xref[\emph{b}]{550}.

Anticipation,
    expr. by subj., \xref{506}–\xref{509};
    by fut. perf. or fut. indic., \xref[4, n. to \emph{a}\)–\emph{d}\)]{507};
    by pres. indic., \xref{571}.

Anticipatory subjunctive,
    \xref{506}–\xref{509}.

Aoristic tenses,
    \xref[2]{466}, \xref[2]{467};
    of indic., \xref[n.]{468};
    of subj., \xref[2]{470}.

“Apodosis,” see Conclusion, \xref{573}–\xref{582}.

Application, gen. of, \xref{354}.

“Appositive genitive,” \xref{341}.

Appositive words,
    \xref[2]{317};
    agreement of, \xref[I]{319}, \xref[II]{320};
    w. names of towns where, whither, whence, \xref{452};
    attracted by dat., \xref[3]{326};
    often put w. a rel., \xref{327};
    acc. in apposition to a sentence, \xref{395};
    nom. instead of voc., \xref{401};
    position of, \xref[5]{624}.

\latin{apud},
    w. ac., \xref{380}, \xref[4]{454}.

Arsis,
    \ftn{351}{5}.

Article, lacking in Latin,
    \xref[\emph{e}]{221}.

\suffix{-ās},
    old gen. sing. in, \xref[1]{66}.

\latin{-āscō},
    verbs in, \xref[\emph{F}, \emph{a}]{168}, \xref[2]{212}.

Asides,
    \latin{quī}, \latin{cum}, etc., in, \xref{567}.

“Asking,” see “Inquiring” and “Requesting.”

Aspirates,
    \xref[5]{6}, \xref{11}, \xref{12}, \xref[2, n.]{14}

Assimilation of consonants, \xref{49}–\xref{51}.

Assocation of ideas, \xref[2]{315}.

Asyndeton, \xref[I, \emph{a}]{305}.

\latin{at}, \latin{at enim}, etc.,
    \xref[1, \emph{a}–\emph{c}]{310}.

\latin{atque} or \latin{ac},
    \xref[2]{307};
    choice of forms, \ibid[3, \emph{c}];
    used w. \latin{īdem}, \latin{alius}, etc., \ibid[2, \emph{a}].

\latin{atquī}, \xref[3]{310}.

Attempted action, tenses of, \xref{484}.

Attendant circumstances, abl. of, \xref{422}.

Attraction,
    agreement of prons., adjs., and partics. by, \xref[1–5]{326};
    of verb by, \xref{332};
    adj. attracted into rel. cl., \xref[7]{284};
    appositive attracted into rel. cl., \xref{327};
    subj. by attraction, \xref{539}.

Attributive words,
    \xref[1]{317};
    agreement, \xref[I]{320}.

\latin{audeō}, semi-depon., \xref{161}.

\latin{audiō},
    conj., \xref{159};
    w.\ \latin{cum}-cl., \xref[\emph{a}]{524};
    w. partic., \xref[1]{605};
    w. infin., \ibid[n.]

\latin{aut},
    \xref[1, 3, \emph{a}]{308};
    correlative, \xref{309}.

\latin{autem},
    \xref[2, \emph{a}, \emph{b}]{310};
    position, \xref[8, \emph{b}]{624}.

Auxiliary and principal tenses, \xref[\emph{c}]{477}.

Auxiliary verb, \xref{153}, \xref[8]{164}.

\bigskip

\latin{bellī}, \english{in war}, \xref[\emph{a}]{449}.

\latin{bene}, compar., \xref{129}.

\latin{bonus},
    decl., \xref{110};
    compar., \xref{122}.

\latin{bōs},
    decl., \xref{92}.

Brachýlogy,
    \xref[2]{631}.

Bucolic diaeresis,
    \xref[\emph{c}, n. 2]{641}.

\bigskip

\latin{Caesar},
    decl., \xref[3]{80};
    \latin{Caesarēs}, pl., \xref[n.]{103}

Caesura,
    \xref{640}, \xref{641};
    masc. and fem., \xref[\emph{a}]{641}.

Calendar,
    \xref{660}–\xref{671}.

Calends, \latin{Kalendae}, \xref{664}.

“Can,” “could,”
    how expr., see Capacity.

Capacity,
    expr. by potential subj., \xref{516}, \xref{517};
    by \latin{possum} w. infin., \xref{586}.

\latin{caput},
    decl., \xref{76}, \xref[5]{77};
    gen. of penalty, \xref{343};
    abl. of penalty, \xref[\emph{b}]{428}.

Cardinal numbers,
    \xref{130}, \xref{131}.

Cases,
    forms, \xref{61}, \xref{62};
    endings, \xref{62}–\xref{64};
    earliest meanings of, \xref{334}.

\latin{causā}, \english{on account of},
    case, \xref[\emph{d}]{444};
    w. gen., \xref[\emph{d}]{339};
    w. gerundive, \xref[I]{612}.

Causal-adversative \latin{quī}- or \latin{cum}-cl.,
    in subj., \xref{523}, \xref{525}, \xref{526};
    in indic., \xref[\emph{a}]{569}.

Cause or reason
    expr. by abl., \xref{444};
    by abl. absolute, \xref[4]{421};
    by prep. phrases, \xref[\emph{b}, \emph{c}]{444};
    by subj. \latin{quī}- or \latin{cum}-cl., \xref{523}, \xref{525},
    \xref{526};
    by indic. \latin{quī}- or \latin{cum}-cl., \xref[\emph{a}]{569};
    by cl. w. \latin{quod}, \latin{quia}, \latin{quoniam}, or
        \latin{quandō}, \xref{555};
    by \latin{nōn quia}, \latin{nōn quod}, etc., w. subj., \xref[2,
      \emph{b}]{535};
    by partic., \xref[2]{604}.

\latin{cavē},
    in prohibitions, \xref[3, \emph{a}, 2\)]{501}, \xref[3,
      \emph{b}\)]{502};
    without \latin{nē}, \ibid[n.~2];
    w. short~\ending{-e}, \xref[2, \emph{b}\)]{28}.

\enclitic{-c(e)}, particle,
    \xref[n.]{32}, \xref{33}, \xref[2, \emph{c}]{138}.

\latin{cēlō},
    constrs. w., \xref{393}.

\latin{cēnseō},
    w. vol. col., \xref[3, \emph{a}\)]{502};
    w. cl. of obligation or propriety, \xref[5]{513};
    w. infin., \xref{589}.

\latin{cētera},
    acc. of respect, \xref[\emph{a}]{389}.

\latin{cēterī},
    meaning, \xref[1, \emph{a}]{279}.

“Characteristic” and “characterizing clause,”
    see Descriptive clause.

Charge,
    gen. of, \xref{342}.

Chiasmus,
    \xref{628}.

\latin{circā}, \latin{circum}, \latin{circiter},
    w. acc., \xref{380};
    as advs., \xref[\emph{c}]{303}.

\prefix{circum-},
    form in cpds., \xref[5]{51};
    w. acc., \xref{380};
    cpds. of, w. acc., \xref{386}, \xref[2]{391};
    w. dat., \xref{376}.

Circumstances or situation,
    expr. by abl., \xref{422};
    by abl. absolute, \xref{421};
    by partic., \xref[2]{604};
    by \latin{cum}-cl., \xref{524}, \xref{525}.

\latin{cis} and \latin{citrā},
    w. acc., \xref{380}.

\latin{citerior},
    compar., \xref{123}.

Cities, gend. of names of, \xref[2]{58}.

\latin{clam}, adv., or prep. w. abl. or acc., \xref[2]{458}.

Clause,
    definitions:
        principal or dep. (subordinate), \xref[1]{224};
        coördinate, \xref{225};
        det., \ftn{260}{1};
        descr., \ftn{260}{2};
        conditional, \xref[2]{228};
        free, \ftn{302}{}, ftn.;
        subst., \xref{238};
        adv., \xref{239};
        individual and generalizing, \xref{576}, \xref{577}.

Climax,
    \xref[11]{632}.

\latin{coepī},
    conj., \xref[2]{199};
    voice of infin. w., \ibid

Cognate acc.,
    see Kindred meaning, \xref{396}.

\latin{cognōvī} etc.,
    forces of tenses, \xref{487}.

\latin{cōgō},
    w. acc., \xref[1]{397};
    w. vol. cl., \xref[3, \emph{a}\)]{502};
    w. infin., \xref{587};
    w. cl. of actuality, \xref[3, \emph{a}\)]{521}.

Collective noun,
    \xref[3]{240};
    agreement w., \xref{325}, \xref[1]{331}.

\prefix{com-}, see \prefix{con-}.

Combinations of tenses,
    usual, \xref{476}, \xref{477};
    less usual, \xref{478};
    mechanical harmony of subj. tenses, \xref{480};
    tenses depending on pres. perf., \xref{481};
    permanent truths depending on past tenses, \xref{482}.

Command,
    expr. by imper., \xref{496}, \xref[3, \emph{b}]{501};
    by subj., \xref[3, \emph{a}, \emph{b}]{501};
    by fut. indic., \xref{572};
    in ind. disc., \xref{538}.

Common nouns,
    \xref[2]{240}.

“Common” syllable,
    \xref[5, n.]{28}

\latin{commonefaciō}, \latin{commoneō},
    constr. w., \xref{351}.

\latin{commūnis},
    w. either gen. or dat., \xref[\emph{c}]{339}.

Comparative,
    case constrs. w., \xref{416}, \xref{417};
    w. \latin{quam} and \latin{quī}- or \latin{ut}-cl., \xref[2,
      \emph{c}]{521}.

Comparatives,
    decl., \xref{116}, \xref{118};
    formation, see Comparison.

Comparison:
    of adjs.,
        formation, \xref{119}–\xref{123};
        forces of degrees, \xref[1–4]{241};
        two compars., \xref{242};
    comparison of advs.,
        \xref{128}–\xref{129};
        forces of degrees, \xref{300};
        two compars., \xref{301}.

Comparison, imaginative,
    w. \latin{quasi}, etc., and subj., \xref[3]{504}.

Complementary infinitive,
    defined, \xref[\emph{a}]{586}.

Complex sentence,
    \xref[3]{223}.

Composition of words,
    \xref{213}–\xref{218};
    quantity in cpds., \xref{24};
    accent in, \xref[3]{31};
    vowel-change in, \xref{42};
    assimilation of prep. in, \xref{50}, \xref{51};
    redupl. perf. of cpds., \xref[\emph{D}, \emph{a}]{173}.

Composition or material,
    gen. of, \xref{349}.

Compound sentence,
    \xref[2]{223}.

Compounds of verb and prep.,
    w. dat., \xref{376}, \xref{377};
    w. dat. and acc., \xref[\emph{a}]{376};
    w. acc., \xref[2]{391};
    w. dat. or acc., \ibid[\emph{a}].

\prefix{con-}, \latin{com-},
    form in cpds., \xref[6]{51};
    cpds of, w. dat., \xref{376}.

“Conative action,” tenses of, \xref{484}.

Concern,
    dat. of, \xref{366}.

Concession of indifference,
    expr. by imper., \xref[2]{497};
    by subj., \xref[1]{532};
    by subj. cl. w. \latin{quamvīs} etc., \xref[2]{532}.

Concessive, see Concession, and Adversative.

Conclusions, see Conditions.

Concrete nouns, \xref[4]{240}.

Concrete object for wh., dat. of, \xref{361}.

“Condemning,” gen. w., \xref{342}, \xref{343}.

Condensed comparison,
    \xref[3]{631}.

Conditional sentence or cl., \xref[2]{228}, \xref{577};
    see also Conditions.

Conditions and conclusions,
    generalizing and individual distinguished, \xref{576};
    three types, \xref{575}–\xref{581};
    in ind. dis., \xref[1, \emph{b}]{534}, \xref{536}.

\latin{cōnfīdō},
    w. dat., \xref[II]{362};
    w. abl., \xref{437}.

\latin{coniciō},
    quantity of first syll., \xref[1]{30}.

Conjugation of verbs,
    \xref{54}, \xref{145}–\xref{201};
    of \latin{sum}, \xref{153}–\xref{154};
    conjs. distinguished, \xref{148};
    first conj., \xref{155};
    second, \xref{156};
    third, \xref{157};
    fourth, \xref{159};
    of depon., \xref{160};
    periphr., \xref{162};
    peculiarities in, \xref{163}–\xref{165};
    variation between conjs., \xref{165};
    of irreg. verbs, \xref{190}–\xref{197};
    of defect. verbs, \xref{198}–\xref{200};
    of impers. verbs, \xref{201}.

Conjunctions,
    origin, \xref{125};
    defined, \xref{304};
    coördinating, \xref{305} and~I;
    copulative, \xref{307};
    disjunctive, \xref{308};
    advers., \xref{310};
    inferential, \xref{311};
    subordinating, \xref{312}.

Connection,
    gen. of, \xref{339}.

\latin{cōnscius},
    gen. w., \xref{354};
    dat. w., \xref[1, \emph{b}\)]{363}.

Consecutive clauses defined,
    \xref[3, \emph{a}]{519}, \xref[1, \emph{e}]{521};
    of ideal certainty, \xref[2, 3]{519};
    of actuality, \xref[1–3]{521}.

Consent,
    expr. by imper., \xref{496};
    by subj., \xref[1, 2]{531};
    by indic., \xref{571}, \xref{572}.

\latin{cōnsistō},
    constrs. w., \xref[3]{438}.

Consonants,
    \xref{2};
    classif., \xref{6}–\xref{8}, \xref{12};
    pronunc., \xref{11};
    changes of, \xref{47}–\xref{49};
    stems in, \xref[\emph{A}]{74}, \xref{75}–\xref{86}.

\latin{cōnstituō},
    w. vol. subj. or infin., \xref{586} and~\emph{e}.

\latin{cōnstō},
    constrs. w., \xref[1, 3, \emph{a}, \emph{b}]{438}.

Construction,
    defined, \xref[3]{314}.

\latin{cōnsuēvī} etc.,
    forces of tenses, \xref{487}.

\latin{cōnsulō},
    w. dat. or acc., \xref{367}.

“Contention,”
    w. \latin{cum} and abl., \xref[4]{419};
    w. dat., \xref[2, \emph{c}\)]{363};
    w. acc., \xref[2]{397}.

\latin{contentus},
    w. abl., \xref[4]{438}.

\latin{contrā},
    w. acc., \xref{380};
    w. \latin{ateque (ac)}, \xref[2, \emph{a}]{307}.

Contraction of vowels,
    \xref{45};
    quantity resulting from, \xref{19};
    of vowels, in poetry, \xref{658}.

Contrast, \latin{ut}-cl. of, \english{while\ellipsis\(yet\)}, \xref{563}.

Coördinate clauses, \xref{225}.

Coördinate sentences, \xref[2]{223}.

Coördinating conjunctions, \xref{305}–\xref{311}.

Copula, \xref[\emph{a}]{230}.

Copulative compounds, \xref[1]{216}.

Copulative conjunctions, \xref{307}, \xref{309}.

\latin{cōram},
    abl. w. \xref[1]{407}.

Corrective
    \latin{aut}, \latin{sīve}, \latin{vel}, \xref[3, \emph{a}]{308};
    \latin{quamquam}, \latin{etsī}, \latin{tametsī}, \xref[7]{310}.

Correlatives, \xref{144}.

Countries, gend. of names of, \xref[2]{58}.

\latin{crēdō},
    w. dat., \xref[II]{362};
    w. acc., \xref[3]{364}.

\latin{cui},
    pronunc., \xref[\emph{d}]{10}, \xref[\emph{a}]{140}.

\latin{cuius},
    pronunc., \xref[2, \emph{a}]{29}, \xref[\emph{a}]{140}.

\latin{cum},
    prep., in cpds., \xref[6]{51};
    w. soc. abl., \xref{418};
    \latin{mēcum}, etc., \ibid[\emph{a}];
    ideas expr. by, \xref{419}.

\latin{cum}-clauses:
    descr. cl. of ideal certainty, \xref[2]{519};
    of actuality, \xref[1]{521};
    descr. cl. of situation, \xref{524};
    of situation, w. caus. or advers. idea, \xref{525};
    purely caus. or advers. cl., \xref{526};
    of repeated action, \xref{540};
    determining a time, \xref{550} and~\emph{a};
    of the time included in the reckoning, \xref[and \sftn{2}]{550};
    of equivalent action, \xref{551};
    subst., \xref{553};
    aor. narr. cl. (\latin{cum prīmum}), \xref[\emph{a}]{557};
    \latin{cum\ellipsis tum\dots}, \xref{564};
    \latin{cum} in forward-moving cl., \xref{566};
    “\latin{cum inversum},” \ibid[\emph{a}];
    parenthetical cl. and “asides,” \xref{567};
    looseley attached descr. cl., \xref{568};
    free descr. cl., \xref{569};
    tacit caus. or advers. cl., \ibid[\emph{a}];
    general conditions, \xref[\sftn{3}]{577};
    see also examples, \xref{579}, \xref{581}.

\latin{cupiō},
    w.~dat. or acc., \xref{367};
    w.~gen., \xref[3]{352};
    w.~subj., \xref[2]{511};
    w.~infin., \xref{586}, \xref{587}.

\latin{cūr},
    w. subj. of obligation, \xref[1, 2]{513};
    of natural likelihood, \xref[1, 2]{515}.

\latin{cūrō},
    w. vol. cl., \xref[3, \emph{a}\)]{502};
    w. gerundive, \xref[III]{612}.

Customary action,
    tenses of, \xref{484}.

\bigskip

Dactyl,
    \xref{637};
    dactylic hexameter, \xref{639};
    dactylic pentameter, \xref{642}.

Dates,
    how expr., \xref{664}–\xref{669}.

Dative,
    \emph{Form}:
        sing., decl.~IV, in~\ending{-ū}, \xref[2]{97};
        decl.~V, in \ending{-ēī} or \ending{-eī}, \xref[1]{100};
        in~\ending{-ē}, \ibid[2];
        pl., decl.~I, in~\ending{-ābus}, \xref[4]{66};
        decl.~IV, in~\ending{-ubus}, \xref[1]{97}.
    \enskip
    \emph{Syntax}:
        see synopsis, \xref{359}.

\latin{dē},
    in cpds., \xref[2]{24}.

\latin{dē},
    w. abl., \xref{405}, \xref[3, 4]{406};
    for gen. of whole, \xref[\emph{e}]{346};
    w. verbs of separation, \xref{408}.

\latin{dēbeō},
    moods and tenses of, \xref[3]{582};
    w. infin., \xref{586}.

\latin{decet},
    w. acc., \xref[\emph{a}]{390};
    w. subj., \xref[5]{513};
    w. infin., \xref{585}.

Declarative sentence or clause,
    \xref[1]{228}.

Declension,
    \xref{54}, \xref{55};
    the five decls. of nouns, \xref{63};
    endings, \xref{64};
    I, \xref{65}–\xref{68};
    II, \xref{69}–\xref{73};
    III, \xref{74}–\xref{95};
    IV, \xref{96}–\xref{98};
    V, \xref{99}–\xref{101};
    nouns variable in, \xref{107};
    of adjs., \xref{110}–\xref{118};
    of comparatives, \xref{116};
    of partics., \xref{117};
    of numerals, \xref{131};
    of prons., \xref{134}–\xref{142}.

Defective nouns,
    \xref{102}–\xref{104}, \xref{106};
    defect. compar., \xref{123};
    verbs, \xref{198}–\xref{201}.

“Defining” genitive, \xref{341}.

Degree,
    acc. of, \xref{387}.

Degree, cl. determining the, \xref[and \sftn{2}]{550}.

Degree of difference,
    abl. of, \xref{424}.

Degrees of comparison, \xref{119}.

Deliberation,
    expr. by subj., \xref{503};
    by pres. indic., \xref{571};
    by fut. indic., \xref{572}.

Demand,
    expr. by vol. subj., \xref{500}, \xref[3, \emph{a}\)]{502}.

“Demonstrative,” see Determinative.

Denominative verbs,
    \ftn{98}{1}, \xref{211};
    of conj.~I, \xref[1]{211}, \xref[n.]{212}, \xref[1, 2]{166};
    of conj.~II, \xref[3]{211}, \xref[1, \emph{c}]{167};
    of conj.~III, \xref[4]{211}, \xref[\emph{H}, 1]{168};
    of conj.~IV, \xref[2]{211}, \xref[1]{169}.

Dentals,
    \xref[2]{7}, \xref{12};
    changes of, \xref[4, 5, 8, 10]{49}.

Dependent clause,
    \xref[1 and \emph{a}]{224}.

Dependent compounds,
    \xref[3]{216}.

Deponent verbs,
    defined, \xref{145}, \xref{160};
    originally reflexive, \xref[3, \emph{b}]{288};
    voice meanings of, \xref[\emph{a}–\emph{d}]{291};
    perf. pass. partic. of, w. act. meaning, \xref[1, \emph{a}]{602}.

Deprecated act,%
    \versionA{ in subj.}
    w. \latin{antequam}, etc.,
    \versionB*{and subj., }%
    \xref[4, \emph{d}\)]{507}\versionB*{; w. indic., \xref{571}}.

“Depriving,”
    abl. w., \xref{408}.

Derivation of nouns and adjectives,
    \xref{203}–\xref{210};
    derivatives, primary and secondary, \xref{203};
    vowel-quantity in, \xref[2]{23}.

Descent,
    words denoting, \xref[3]{207}.

Descriptive compounds,
    \xref[2]{216}.

Descriptive genitive,
    \xref{355};
    abl., \xref{443};
    tenses, \xref[1, \emph{a}]{466};
    descr. cl. defined, \ftn{260}{2};
    clauses: see synopses, \xref{499}, \xref{543}.

Desideratives,
    \xref[3]{212}.

Desire,
    see Wish, \xref[1]{511}.

\latin{dēspērō},
    w. dat. or acc., \xref{367};
    w. acc., \xref[1]{391}.

\latin{dēterior},
    comparison, \xref{123}.

Determinative cl.,
    \ftn{260}{1};
    see synopses, \xref{499}, \xref{543}.

Determinative-descriptive pronouns,
    decl., \xref{137}, \xref{138};
    distinctive meanings of \latin{hic}, \latin{iste}, \latin{ille},
        \latin{is}, and of \latin{tālis}, \latin{tantus}, \latin{tot},
        \xref{271}.

\latin{deus},
    decl., \xref[5]{71}.

Diaeresis, bucolic,
    \xref[\emph{c}, n.~2]{641}.

Diastole,
    \xref[\sftn{3}]{654}.

\latin{dīc},
    imper., \xref[1]{164}.

\latin{dīcō},
    w. vol. cl., \xref[3, \emph{a}\)]{502};
    w. infin., \xref{589};
    in subj. in cl. of reason%
    \versionA{, \xref[2, \emph{b}, n.]{535}}%
    \versionB*{, \xref[2, \emph{a}, n.~3]{535}}%

\latin{Dīdō},
    decl., \xref[\emph{b}]{95}.

\latin{diēs},
    decl., \xref{99};
    gend. \xref{101}.

“Difference” or “aversion,”
    abl. w., \xref{412};
    poetic dat. w., \xref[2, \emph{c}\)]{363}.

\latin{difficilis},
    comparison, \xref[2, \emph{a}]{120}.

\latin{dignus},
    w. abl., \xref{442};
    w. subj. cl., \xref[3]{513};
    w. supine in~\ending{-ū}, \xref[2]{619};
    w. infin., \xref[2, \emph{c}\)]{598}.

Diminutive adjectives,
    \xref[1]{207}.

Diphthongs,
    \xref{5};
    pronunc. in Latin, \xref{10}.

Direct discourse,
    explained, \xref{533}.

Direct object,
    acc. of, \xref{390}, \xref{391}.

Direct reflexive,
    \xref[1]{262}.

Direction or relation expr. by dat.,
    \xref[I–III]{362};
    w. \latin{ad}, \latin{in}, etc., \xref[2]{384};
    poetic dat. of direction in space, \xref{375}.

\prefix{dis-},
    \xref[1]{24}, \xref[7]{51}, \xref[1, \emph{b}]{218}.

Disjunctive conjunctions,
    \xref{308}, \xref{309}.

“Distance,”
    see Extent of space, acc., \xref[I]{387},
    and degree of difference, abl., \xref{424}.

Distich, elegiac,
    \xref[\emph{a}, \sftn{6}]{642}.

Distributive numerals,
    \xref{133};
    used for cardinals, \xref{247}.

Distributive pronouns,
    \xref{142}, \xref{278}.

\latin{diū},
    comparison, \xref{129}.

\latin{doceō},
    constr. w., \xref{393} and~\emph{b},~1).

\latin{domus},
    decl., \xref[5]{97};
    \latin{domī}, loc. form, \xref[6]{71};
    gend., \xref[\emph{a}]{98};
    of place where, whither, whence, \xref{449}–\xref{451};
    w. modifiers, \xref{454}.

\latin{dōnec},
    see \latin{dum}.

Double consonants,
    \xref{11}.

“Double questions,”
    \xref{234}.

“Doubt,”
    w. \latin{quīn}, \xref[4, \emph{b}\)]{519};
    \xref[3, \emph{b}\)]{521}.

“Dubitative subjunctive,”
    see Deliberation.

\latin{dubitō},
    w. vol. cl., \xref[3, \emph{b}\)]{502};
    w. antic. subj., \xref[2, \emph{b}\)]{506};
    of ideal certainty, \xref[4, \emph{b}\)]{519};
    of actuality, \xref[3, \emph{b}\)]{521};
    w. infin., \xref{586}.

\latin{dūc},
    imper., \xref[1]{164}.

\latin{duim}, \latin{duīs}, \latin{duit}, etc.,
    \xref[\emph{a}]{197}.

\latin{dum}, \latin{dōnec}, \latin{quoad},
    w. antic. subj., \xref[5]{507};
    w. fut. perf. or fut. indic., \ibid[\emph{a}];
    w. pres. indic., \xref{571};
    determining time up to wh., \xref{550} and~\emph{b};
    narr. cl. w., \xref{560};
    determining time during wh., \xref{550} and~\emph{b};
    \latin{dum}-cl. of situation, \xref{559};
    replacing pres. pass. partic., \xref[2]{602};
    subj. \latin{dum}-cl. of proviso, \xref{529}.

\latin{dummodo},
    in cl. of proviso, \xref{529}.

\latin{duo},
    decl., \xref[2]{131}.

Duration of time,
    expr. by acc., \xref[II]{387};
    by \latin{per} and acc., \ibid[\emph{a}];
    by abl., \xref{440}.

\bigskip

\latin{ē},
    prep., see~\latin{ex}.

\latin{ecquis?}
    decl., \xref[\emph{b}]{141};
    use, \xref[6]{275}.

\latin{edō},
    conj., \xref{196};
    \latin{edim}, \latin{edīs}, etc., \xref[\emph{a}]{196}.

\latin{efficiō},
    w. vol. cl., \xref[3, \emph{a}\)]{502};
    w. cl. of fact, \xref[3, \emph{a}\)]{521}.

“Effort,”
    w. vol. cl., \xref[3, \emph{a}\)]{502}.

\latin{egeō},
    w. gen., \xref{347};
    w. abl., \xref[\emph{a}]{425}.

\latin{ego},
    decl., \xref{134};
    \latin{egomet}, \ibid[\emph{d}].

\latin{eius},
    pronunc. and quantity, \xref[2, \emph{a}]{29};
    \xref[\emph{a}]{137};
    \latin{eius modī}, descr. gen., \xref{355}.

Elegiac stanza, \xref[\emph{a}]{642}.

Elision,
    see Slurring.

Ellipsis,
    \xref[1]{631}.

Emphasis,
    obtained by order, \xref{625}–\xref{628}.

Emphatic future-perfect,
    \xref{490} and~\emph{a}.

Emphatic perfect,
    \xref{490}.

Enállage,
    \xref[9]{631}.

Enclitics
    defined, \xref[1]{33};
    quantity of, \xref[2, \emph{d}]{28};
    question of accent before, \xref[n.]{32}

“End of motion,”
    expr. by prep. w. acc., \xref{385};
    by poetic dat., \xref{375}.

Endeavor,
    expr. by vol. subj., \xref{500}.

Endings,
    inflectional, origin, \xref[\sftn{2}]{203};
    of nouns, \xref{64};
    of verbs, \xref{151}.

Energetic or emphatic perf.,
    \xref{490}.

\latin{enim},
    meaning and position, \xref[6]{311}.

Entreaty,
    expr. by imper., \xref{496};
    by subj., \xref{530}.

\latin{eō},
    conj., \xref{194}.

\latin{eō\ellipsis quō},
    of degree of difference, \xref{424}.

Epistolary tenses,
    \xref{493}.

\latin{equidem},
    use of, \xref[\emph{a}]{257}.

Equivalent action,
    cl. of, \xref{551}.

\latin{ergā},
    w. acc., \xref{380}, \xref[7]{364}.

\latin{ergō},
    \xref[1]{311};
    w. gen., \xref[\emph{d}]{339}.

\suffix{-ēscō},
    verbs in, \xref[\emph{F}, \emph{a}]{168}, \xref[2]{212}.

\latin{esse},
    often omitted, \xref[\emph{e}]{584}.

Essential part,
    see Attraction.

\latin{est} and noun,
    dat. of reference w., \xref[\emph{a}]{366};
    \latin{est}, w. potential descr. cl., \xref[2]{517};
    w. descr. cl. of ideal certainty, \xref[2]{519};
    of actuality, \xref[1]{521};
    w. subst. cl. of actuality, \xref[3, \emph{a}\)]{521};
    w. infin., \xref[3]{598}.

\latin{et},
    \xref[1, \emph{a}, \emph{c}]{307};
    = \latin{etiam}, \xref[2 and \emph{b}]{302};
    \latin{et\ellipsis et}, \latin{et\ellipsis neque}, \latin{neque\ellipsis et},
    \xref{309}.

\latin{etenim}, \xref[6, \emph{a}]{311}.

Ethical dative,
    \xref{372}.

\latin{etiam},
    in answers, \xref[1]{232};
    meaning and position, \xref[2]{302};
    often added to \latin{sed} or \latin{vērum}, \xref[4, \emph{b}]{310}.

\latin{etiamsī} and \latin{etsī},
    = \english{although}, \xref[8]{582}.

\latin{etsī},
    “corrective,” \xref[7]{310}.

Eúphemism,
    \xref[12]{632}.

\latin{ex} or \latin{ē},
    in cpds., \xref[8]{51};
    choice of forms, \xref[\emph{b}]{405};
    w. abl., \xref{405}, \xref{406};
    of point of view, \xref[2]{406};
    w. verbs of separation, \xref{408}.

“Exchanging,”
    w. abl., \xref{431}.

Exclamation,
    acc. of, \xref{399};
    nom. of, \ibid[\emph{a}];
    infin. of, \xref{596}.

Exclamatory sentence,
    \xref[3]{228};
    questions, \xref{503}.

Exhortation,
    expr. by vol. subj., \xref[2]{501};
    by fut. indic., \xref{572}.

Explanatory genitive,
    \xref{341}.

Explicative \latin{quod}-cl.,
    \ftn{296}{1}.

Explicit causal or adversative \latin{quī}- or \latin{cum}-cl.,
    \xref[\emph{a}]{523}, \xref{526}.

\latin{exspectō},
    w. antic. subj., \xref[2]{507};
    w infin. \xref[\emph{a}]{593}.

Extent of space,
    acc. of, \xref{387};
    abl. of, \xref[\emph{c}]{426}.

\latin{exterus}, \latin{exterior}, \latin{extrēmus},
    \xref{123}.

\latin{extrā},
    w. acc., \xref{380}.

\bigskip

\latin{fac},
    imper., \xref[1]{164}.

\latin{facilis},
    comparison, \xref[2]{120};
    w. supine in~\ending{-\emend{58}{u}{ū}}, \xref[1]{619};
    w. \latin{ad} and gerundive, \ibid[\emph{a}].

\latin{faciō},
    accent of cpds. of, \xref[3]{31};
    non-prep. forms, \xref[3]{218};
    w. vol. subj., \xref[3, \emph{a}\)]{502};
    w. cl. of actuality, \xref[3, \emph{a}\)]{521};
    w. infin., \xref[\sftn{1}]{587}, \xref[1, n.]{605}

Fact,
    indic. of, \xref{462}, \xref{544};
    subj. of, in consec. cls., \xref{520}, \xref[3, \emph{a}\),
      \emph{b}\)]{521}.

“Factitive object,”
    see Result produced.

\latin{falsus},
    comparison, \xref[\emph{a}]{123}.

\latin{famēs},
    abl., \latin{famē}, \xref[5]{88}.

\latin{familiās},
    \xref[1]{66}.

\latin{fārī},
    conj., \xref[3]{198}.

\latin{fās},
    indecl., \xref[2]{106};
    \latin{fās est}, w. infin., \xref{585};
    \latin{fās} w. supine, \xref[1]{619}.

\latin{faxō}, \latin{faxim},
    \xref[5]{163}.

Fear,
    subj. cl. of, \xref[4]{502}.

Feminine caesura,
    \xref[\emph{a}]{641}.

Feminines,
    see Gender.

\latin{fer},
    imper., \xref[1]{164}.

\latin{ferē}, \latin{fermē},
    position of, \xref[13, \emph{a}]{624}.

\latin{ferō},
    conj., \xref{193};
    dat. w., \xref{365};
    w. infin., \xref{594}.

\suffix{-fīcus},
    comparison of cpds. in, \xref[3]{120}.

\latin{fīdō},
    \xref{161};
    w. dat., \xref[II]{362};
    w. abl.\emend{59}{}{,} \xref{437}.

\latin{fīdus},
    comparison, \xref[\emph{a}]{123}.

Field in which,
    expr. by \latin{in} w. abl., \xref[2]{434}.

Fifth declension,
    \xref{99}–\xref{101}.

Figurative use of cases, moods, etc.,
    \xref[1]{315}.

Figures
    of syntax, \xref{631};
    of rhetoric, \xref{632}.

\latin{fīlia},
    decl., \xref[4]{66}.

\latin{fīlius},
    gen. and voc., \latin{fīlī}, \xref[3]{71}.

“Final clauses,”
    \ftn{260}{3}.

Final consonants,
    changes in, \xref[13]{49}.

“Fine,”
    abl. of, \xref{428}.

\latin{fīnis},
    decl., \xref{87}, \xref[2, \emph{d}]{88};
    sing. and pl., \xref{105}.

Finite forms of verb,
    defined, \xref{146}.

\latin{fīō} etc.,
    quantity of \phone{i} in, \xref[1]{21};
    conj., \xref{195};
    defect. cpds. w., \xref[\emph{a}]{195};
    abl. w., \xref[\emph{b}]{423}.

First conjugation,
    \xref{148}, \xref{155};
    pres. stem, \xref{166};
    denom., \xref[1]{211}.

First declension of nouns,
    \xref{65}–\xref{68}.

First and second declension of adjectives,
    \xref{110}–\xref{112}.

“Fitness,”
    adjs. of, w. dat., \xref{362};
    w. \latin{ad}, \xref[2, \emph{a}]{384}.

\latin{flāgitō},
    cases w., \xref{393} and~\emph{b},~2\).

Foot,
    defined, \xref{635}.

\latin{forās},
    adv., \xref[7, n.]{126}

\latin{fore}, \latin{forem}, etc.,
    \xref[1, 2]{154}.

\latin{fore} or \latin{futūrum ut\dots} = fut. infin.,
    \xref[\emph{c}]{472}.

Forestalled act,
    in subj. w. \latin{antequam} or \latin{priusquam}, \xref[4,
      \emph{b}\)]{507}.

“Forgetting,” “remembering,” and “recalling,”
    constrs. w., \xref{350}.

\latin{forīs},
    \english{out of doors}, \xref[\emph{a}]{449}.

Formal \latin{ut},
    \ftn{261}{2}.

Formation of verb-stems,
    \xref{166}–\xref{184};
    of words, \xref{202}–\xref{218}.

\latin{forsitan},
    w. potential subj., \xref[1]{517}.

Forward-moving clauses,
    \xref{566}.

Fourth conjugation,
    \xref{159};
    pres. stem, \xref{169}.

Fourth declension,
    \xref{96}–\xref{98}.

Free clause,
    defined, \ftn{302}{}, ftn.;
    free descriptive clause, \xref{569}.

Freer neuter accusative modifiers,
    \xref{397}.

Frequentatives,
    \xref[2]{166}, \xref{212}.

Fricatives,
    \xref[3]{6}, \xref{12}.

\latin{frūgī},
    compar., \xref{122};
    as adj., \ftn{190}{2}.

\latin{fruor},
    w. abl., \xref{429};
    w. acc., \ibid[\emph{b}];
    in gerundive constr., \xref[2, n.]{613}

\latin{fuī}, etc.,
    for \latin{sum}, etc., as auxiliaries, \xref[8]{164}.

Fullness,
    adjs. of, \xref[2]{209};
    see Plenty.

\latin{fungor},
    w. abl., \xref{429};
    w. acc., \ibid[\emph{b}];
    in gerundive constr., \xref[2, n.]{613}

Fusion,
    explained, \xref[3]{315}.

Future conditions,
    more vivid, \xref[\emph{a}]{579};
    less vivid, \xref{580};
    past-fut., \xref{508}, \xref{509}, \xref[\emph{b}, \emph{c}]{580}.

Future,
    \textsc{Indicative}:
        \emph{Form}, \xref{172};
        in \latin{-ībō}, \xref[5]{164};
        meanings of tense, \xref[3 and \emph{a}]{468}, \xref{484},
            \xref{485}, \xref[2]{486};
        special uses, \xref{572};
    \enskip
    \textsc{Subjunctive}:
        how replaced in Latin, \xref[1, 4 and \emph{a}]{470},
            \xref{580};
    \enskip
    \textsc{Infinitive}:
        meaning of tense, \xref[and \emph{a}, \emph{b}]{472};
        active, \emph{form}, \xref[3]{178};
        passive, \emph{form}, \ibid;
    \enskip
    \textsc{Participle}:
        active, verbal adj., \xref{146};
        in periphr. conjs., \xref{162};
        \emph{form}, \xref{182};
        meaning of tense, \xref[2]{600};
        passive, verbal adj., \xref{146};
        in periphr. conj., \xref{162};
        in conjs. III and IV, \xref[2]{164};
        \emph{form}, \xref{184};
        meaning of tense, \xref[3 and \emph{b}]{600}.

Future perfect,
    \textsc{Indicative}:
        \emph{Form}, \xref[5]{163}, \xref[6]{164}, \xref[2]{174};
        meaning, \xref[6]{468};
        as emphatic fut., \xref{490};
    \enskip
    \textsc{Subjunctive}:
        how replaced in Latin, \ftn{244}{1}.

Futures, periph.,
    see Periphrastic.

\latin{futūrum esse ut\dots} = fut. infin.,
    \xref[\emph{c}]{472}.

\latin{futūrus},
    as adj., \xref{248}.

\bigskip

\latin{gaudeō},
    \xref{161};
    w. acc., \xref[2]{397};
    w. abl., \xref[\emph{a}]{444};
    w. infin., \xref{594};
    w. \latin{quod}-cl., \xref{555}.

Gender,
    \xref{56}–\xref{59};
    decl.~I, \xref{67};
    decl.~II, \xref{72};
    decl.~III, \xref{94} (summary), \xref{78}, \xref{81}, \xref{84},
        \xref{86}, \xref{89}, \xref[2]{91};
    decl.~IV, \xref{98};
    decl.~V, \xref{101};
    nouns variable in, \xref{108}.

General “truths” or “customs,”
    expr. by pres., \xref[1, \emph{b}]{468}.

Generalizing clause,
    \xref{576};
    indic., \xref[and ftn.]{579};
    subj. in 2d sing. indef., \xref[2]{504}.

Generalizing pronouns,
    \xref{282};
    the same used w. merely indef. meaning, \xref{283}.

Genitive,
    \emph{Form}:
        sing.,
            decl.~I, in~\ending{-ās}, \xref[1]{66};
                     in~\ending{-āī}, \ibid[2];
            decl.~II, in~\ending{-ī} of nouns in \suffix{-ius},
                \suffix{-ium}, \xref[2, 3]{71};
                in \ending{-iī} of adjs. in \suffix{-ius},
                    \xref[\emph{a}]{110};
                of pronom. adjs., \xref{112};
            decl.~IV, \ending{-ī}, \xref[4, 5]{97};
            decl.~V, in \ending{-eī}, \ending{-ēī}, and~\ending{-ī},
                \xref[1, 2, 3]{100};
        pl.,
            decl.~I, in \ending{-um}, \xref[3]{66};
            decl.~II, in \ending{-um}, \xref[4]{71};
                of \latin{ducentī}, etc., in \ending{-um}, \xref[4]{131};
            decl.~III, in \ending{-um} and \latin{-ium}, \xref{75};
                in adjs., \xref[1]{118};
            decl.~IV, in \emend{60}{\ending{-ūm}}{\ending{-um}}, \xref[3]{97}.
    \enskip
    \emph{Syntax}:
        see synopsis, \xref{338}.

Gerund,
    \emph{Form}, \xref{184}.

Gerundive and gerund,
    nature of, \xref{609}–\xref{611};
    common uses, \xref{612}, \xref{613};
    gend. of gerundive, w. \latin{meī nostrī}, etc., \xref{614};
    rarer uses, \xref{615}, \xref{616}.

“Gnomic perfect,”
    \xref{488}.

Grammatical gender,
    \xref[\emph{b}]{56}.

\latin{grātiā},
    w. gen., \xref[\emph{d}]{339};
    of gerundive, \xref[I]{612}.

“Greek accusative,”
    see Acc. of respect.

Greek nouns,
    decl.~I, \xref{68};
    decl.~II, \xref{73};
    decl.~III, \xref{95}.

Growth of meanings in constructions,
    \xref{315}.

Gutturals,
    \xref[3]{7}, \xref{12};
    changes of, \xref[2, 3]{49};
    stems in, \xref{76}, \xref[1]{77}.

\bigskip

\latin{habeō},
    w. potential cl., \xref[2]{517};
    w. infin., \xref[2]{597};
    w. perf. pass. partic., \xref[5]{605}.

\latin{habētō},
    tense, how used, \xref[\emph{c}]{496}.

Habitual action,
    tenses of, \xref{484}.

\latin{haereō},
    constr. w., \ftn{193}{}, ftn..

Harmony,
    \latin{ut}-cl. of, \xref{563}.

\latin{haud}, use of, \xref[and \emph{a}]{297}.

\latin{havē},
    see \latin{avē}.

Hendiadys,
    \xref[5]{631}.

“Hesitating,”
    w. \latin{nē}, \latin{quīn}, or \latin{quōminus}, \xref[3,
      \emph{b}\)]{502};
    w. infin., \xref{586}.

Heteroclites,
    \xref{107}.

Heterogeneous nouns,
    \xref{108}.

Hexameter, dactylic,
    \xref{639}.

Hiatus,
    \xref{647}.

\latin{hic},
    quantity, \xref{30};
    decl., \xref[1]{138};
    \latin{hoc}, quantity, \xref{30};
    \latin{huius}, \latin{huic}, pronunc., \xref[\emph{d}]{10},
        \xref[\emph{b}]{138};
    meaning and uses of \latin{hic}, \xref{271}–\xref{273};
    \latin{hic\ellipsis ille}, “the former”\ellipsis “the latter,”
        \xref[2]{274};
    as indefinites, \ibid[\emph{b}].

\latin{hīc},
    adv., quantity, \xref[1]{25};
    \emph{form}, \xref[7]{127}.

Hidden quantity,
    \xref[n.~2]{16};
    list, \xref{679}.

\latin{hiem(p)s},
    \xref[7]{49};
    decl., \xref[4]{83}.

Highly improbably future conclusion,
    \xref[\emph{c}]{581}.

\latin{hinc\ellipsis illinc},
    \xref[2, first example]{406}.

Hindrance,
    w. vol. cl., \xref[3, \emph{b}\)]{502}.

Historical infinitive,
    \xref{595}.

“Historical perfect,”
    \xref[4,\emph{a}, \sftn{3}]{468}.

Historical present,
    \xref[1]{491}.

“Historical” tenses,
    \ftn{247}{2}.

\latin{hoc},
    see \latin{hic}.

“Hoping,”
    constr. w., \xref[and \emph{a}]{593}.

“Hortatory” subjunctive,
    see \xref[2]{501}.

\latin{hortor},
    w. neut. acc. pron., \xref[1]{397};
    w. vol. cl., \xref[3, \emph{a}\)]{502};
    w. infin., \xref{587}.

\latin{humī},
    loc., \xref[6]{71}, \xref[\emph{a}]{449}.

\latin{humilis},
    comparison, \xref[2]{120}.

Hypállage,
    \xref[10]{631}.

Hypérbaton,
    \xref[13]{631}.

Hypérbole,
    \xref[2]{632}.

Hypermetric verse,
    \xref[\emph{c}, n.~4]{641}.

Hýsteron próteron,
    \xref[12]{631}.

\bigskip

\phone{I},
    letter, \xref[\emph{a}, n.]{1};
    consonantal~\phone{i}, \xref{11}.

\phone{i},
    consonantal, sometimes becomes vowel in poetry, \xref[1]{656};
    vocalic, sometimes becomes consonantal in poetry, \xref[2]{656}.

\phone{i}-stems, \xref{87}–\xref{89}.

\suffix{-ia},
    suffix, \xref[2]{207}.

\latin{iaciō},
    cpds. of, spelling, and quantity of first syll., \xref[1]{30}.

\latin{iam},
    compared w. \latin{nunc}, \xref[4, 5]{302}.

\latin{iam diū}, \latin{iam prīdem}, etc.,
    w. tenses of action in progress, \xref{485}.

Iambic shortening,
    \xref[n.]{28}, \xref{649}.

Iambus,
    \xref[\emph{b}]{637}.

\ending{-ībam},
    imperf. indic. of conj.~IV, \xref[4]{164}.

\ending{-ībō},
    fut. of conj.~IV, \xref[5]{164}.

Ictus,
    \xref{634};
    relation to accent, \xref{644}, \xref{645}.

\latin{id quod},
    \xref[\emph{a}, n.~2]{325}.

Ideal certainty,
    subj. of, \xref{518}, \xref{519}.

\latin{īdem},
    decl., \xref{137};
    use, \xref{270};
    w. \latin{quī} or \latin{atque}, \ibid[b\emend{61}{.}{}];
    w. dat., \xref[2, \emph{e})]{363}.

Identifying pronoun,
    see \latin{īdem}.

Ides, \latin{Īdūs},
    \xref{664};
    use in dating, \xref{665}–\xref{671}.

\latin{idōneus},
    comparison, \xref{121};
    w. dat., \xref{362};
    w. \latin{ad} and acc., \xref[6]{364};
    w. \latin{quī} or \latin{ut}, \xref[3]{513}.

\latin{Īdūs},
    gend., \xref[a]{98}.
    See also Ides.

\latin{iēns},
    pres. act. partic. of \latin{eō}, \xref{183}.

\suffix{-ier},
    infin. in, \xref[3]{164}, \xref[2]{178}.

\latin{igitur},
    position of, \xref[2]{311}.

“Illative” conjunctions,
    \xref{311}.

\latin{ille},
    decl., \xref[1]{138};
    meaning and uses, \xref{271}–\xref[1]{274};
    \latin{hic\ellipsis ille\dots}, \xref[2]{274};
    \latin{ille} w. descr. \latin{quī}-cl., \xref[1]{521};
    w. \latin{ut}-cl., \xref[2, \emph{b}]{521};
    position of \latin{ille}, \xref[2]{624}.

\latin{illic},
    decl., \xref[2, \emph{c}]{138}.

\latin{ill\'īc},
    adv., quantity of final syllable, \xref[1]{25};
    accent, \xref[1]{32}.

Imaginative comparison,
    cl. of, \xref[3]{504}.

\latin{immō},
    \xref{233}.

\latin{immūnis},
    w. gen., \xref{354};
    w. abl., \ftn{244}{1}.

Imperative,
    endings of, \xref[\emph{b}]{151}.

Imperative,
    synopsis of uses, \xref{495};
    tenses of, \xref{496};
    imper. of command, advice, etc., \ibid;
    w. \latin{quīn}, \ibid[\emph{b}];
    in prohibitions, \ibid[\emph{d}];
    replaced by vol. subj. in ind. disc., \xref{538}.

Imperfect,
    meanings common to all forms:
        of progressive action, \xref[1 and ftn.]{466}, \xref[1]{470};
        of habitual action, \xref{484};
        of attempted action, \ibid;
        w. \latin{iam diū}, etc., \xref{485};
    \enskip
    \textsc{Indicative}:
        \emph{form}, \xref{171};
        in \ending{-ībam}, \xref[4]{164}, \xref[n.]{171};
        meaning, \xref[2]{468};
        of discovery, \xref[1]{486};
        epistolary, \xref{493};
    \enskip
    \textsc{Subjunctive}:
        \emph{form}, \xref[\emph{a}]{175};
        original meanings: progressive, \xref[1]{470};
        aor., \xref[\emph{b}]{477};
        in cls. of result, \ibid\ and \xref[2, examples]{521};
        in wishes, \xref[\emph{a}]{510};
        in conditions and conclusions, \xref{581};
        imperf. contrary to fact retained in any combination of
            tenses, \ibid[n.].

\latin{imperō},
    w.~dat., \xref{362};
    w.~dat. and acc., \xref[4]{364};
    w.~subj., \xref[3,~\emph{a})]{502};
    w.~infin., \xref[\emph{b}]{587}.

Impersonal verbs,
    \xref{201}, \xref{287};
    verbs in pass. w. dat., \xref[2]{364};
    impers. constr., generally preferred w. certain pass. infins.,
        \xref[1, \emph{a}]{590}.

\latin{impetrō},
    w. subj. cl., \xref[2]{530}.

\latin{īmus},
    \emph{lowest part of}, \xref{244}.

\latin{in},
    form in cpds., \xref[9]{51};
    w. acc., \xref{380}, \xref{381}, \xref{385};
    w. abl., \xref{433}, \xref{434};
    use w. abl. of time, \xref[\emph{a}]{439};
    w. abl. of respect, \xref[\emph{a}–\emph{c}]{441};
    often or reg. omitted w. abl. of certain words, \xref{436};
    freely omitted by poets, \xref[\emph{a}]{433};
    cpds. of, w. dat., \xref{376}.

\prefix{in-},
    negative prefix, \xref[2]{214}.

Inchoatives (“Inceptives”),
    \xref[2]{212};
    perf. of, \xref[\emph{F}, ftn.]{188}.

Incomplete action,
    tenses of, \xref[1, ftn.]{466}

Indeclinable nouns,
    \xref[2]{106};
    gend. of, \xref[3]{58}.

“Indefinite antecedents,”
    \xref[1, \emph{a}]{521}.

Indefinite idea distinguished from generalizing,
    \xref{283}.

Indefinite pronouns,
    list of, \xref{142}, \xref{276}.

Indefinite second person in conditions,
    \xref[2]{504}, \xref[\emph{a}]{576};
    in general statement of fact, \xref{542};
    w. potential subj., \xref[1]{517}.

Indefinite subject,
    \xref{286}.

Indefinite value,
    \xref{356}.

Indicative,
    general force of, \xref{462}, \xref{544};
    tenses of, \xref{468};
    tenses w. verbs and phrases of possibility, obligation, etc.,
        \xref[3, \emph{a})]{582};
    general uses, see synopsis, \xref{543};
    pres., pres. perf., and future, in special uses, \xref{571},
        \xref{572}.

Indifference,
    expr. by imper., \xref{496};
    by subj., \xref{531};
    concession of, expr. by imper., \xref[2]{497};
    by subj., \xref[1]{532};
    w. \latin{quamvīs}, \ibid[2];
    w. \latin{licet}, \ibid[\emph{a}];
    w. \latin{ut}, \ibid[\emph{b}].

\latin{indigeō},
    w. gen., \xref[\emph{a}]{347};
    w. abl., \xref[\emph{a}]{425}.

Indignation,
    expr. by subj., \xref{503};
    by fut. indic., \xref{572};
    by infin., \xref{596}.

\latin{indignus},
    w. abl., \xref{442};
    rarely w. gen., \xref[\emph{c}, example]{354};
    w. subj. rel. cl., \xref[3]{513};
    w. supine in \ending{-ū}, \xref[2]{619}.

Indirect discourse,
    defined, \xref{533}, \xref{589}, \xref{591};
    prons. and persons in, \xref[\emph{a}]{533};
    \enskip
    \textsc{Infinitive} in ind. disc.:
        tenses, \xref{593};
        list of verbs and phrases governing, \ftn{318}{}, ftn.;
        principal statements in infin., \xref[1]{534}, \xref{591};
        constrs. of, may be used w. verbs not suggesting ind. disc.,
            \xref[1, \emph{a}]{534}, \xref[1, \emph{a}]{535},
            \xref[\emph{a}]{536};
        subord. cls. in, \xref[2]{534}, \xref{535}–\xref{538}.

Indirect object,
    dat. of, \xref[and \emph{a}]{365}.

Indirect questions,
    of fact, \xref[and ftn.]{537};
    fut., how expr. in, \xref[4, \emph{a}]{470};
    indic. in, \xref[\emph{g}]{537};
    antic. subj. in, \xref[3]{507}.

“Indirect reflexive,”
    \xref[2]{262}.

Individual condition,
    see Condition.

Indo-European speech,
    \xref[\sftn{1}]{46};
    forces of cases in, \xref{334};
    of moods, \xref{459}.

\latin{indulgeō},
    w. dat., \xref{362};
    w. acc., \xref[4]{364}.

Inferential conjunctions,
    \xref{311}.

\latin{īnferior},
    comparison, \xref{123}.

\latin{īnfimus},
    \emph{lowest part of}, \xref{244}.

Infinitives,
    \emph{Form}, \xref{178}.
    \enskip
    \emph{Syntax}:
        synopsis of uses, \xref{583};
        gender, case relations, etc., \xref{584};
        tenses, \xref{472};
        in ind. disc., \xref[1]{534}, \xref{591};
        tenses w. verbs of swearing, etc., \xref[\emph{a}]{593};
        w. verbs of remembering, \ibid[\emph{b}];
        w. verbs of obligation, propriety, etc., \ftn{311}{2};
        energetic or emphatic perfs., \xref{490};
        list of verbs taking ind. disc., \ftn{318}{}, ftn.;
        ordinary uses:
            dependent, \xref{585}–\xref{594}, \xref{597},
            independent, historical, \xref{595},
            exclamatory, \xref{596};
        poetical and later prose uses, \xref{598};
        special points:
            complementary infin., \xref[\emph{a}]{586};
            omission of subject in ind. disc., \xref{592};
        w. \latin{parātus}, \latin{suētus}, etc., \xref[\emph{f}]{586};
        w. verbs of seeing, hearing, or representing, \xref[1, n.]{605};
        in rel. cls., \xref[1, \emph{b}]{535};
        after \latin{quam}, \ibid[\emph{c}].

Inflection,
    defined, \xref{54}.

Informal indirect discourse,
    subj. of, \xref[1, \emph{a}]{535}, \xref[\emph{a}]{536}.

\latin{īnfrā},
    w. acc., \xref{380}.

\latin{innīxus},
    abl. w., \xref[2, \emph{a}]{438};
    dat. w., \ibid[\emph{b}].

\latin{inops},
    w. gen., \xref[\emph{a}]{347};
    w. abl., \xref[\emph{a}]{425}.

\latin{inquam},
    conj., \xref[2]{198};
    position of, \xref[16]{624}.

“Inquiring,”
    two acc. w., \xref{393};
    indirect question w., \xref[\emph{b}]{537}.

Inquiry for instructions,
    in subj., \xref{503};
    in indic., \xref{571}, \xref{572}.

“Inseparable prepositions,”
    see \xref[1]{218}.

\latin{īnsidiae},
    pl. only, \xref[4]{104};
    dat. w., \xref[1, \emph{a})]{363}.

\latin{īnstar},
    gen. w., \xref[\emph{d}]{339}.

Instrument,
    abl. of, \xref{423}.

Instruments or means,
    ends. denoting, \xref[6]{206}.

“Instrumental ablative,”
    \xref[\emph{b}]{61}, \xref{423}.

“Integral part,”
    see Subj. by attraction.

Intensive pronoun,
    see \latin{ipse}.

Intensives (meditatives),
    \xref[4]{212}.

Intention,
    expr. by vol. subj., \xref{500}, \xref[3, \emph{a}), \emph{b}),
      and ftns.]{502};
    by fut. act. partic., \xref{607}.

\latin{intentus},
    cases w., \xref[5]{438}.

\latin{inter},
    form in cpds., \xref[10]{51};
    w. acc., \xref{380};
    cpds. of, w. dat., \xref{376};
    \latin{inter sē}, etc., \xref{266}.

\latin{interclūdō},
    w. dat., \xref[\emph{c}]{366};
    w. abl., \xref[2]{408}.

\latin{interdīcō},
    w. dat., \xref[\emph{c}]{366};
    w. abl., \xref[3]{408}.

“Interest,”
    dat. of, see Reference, dat. of, \xref{366}–\xref{369}.

\latin{interest},
    cases w., \xref{345};
    w. vol. cl., \xref[3, \emph{c})]{502};
    w. infin., \xref{585}.

\latin{interior},
    comparison, \xref{123}.

Interjections,
    \xref{221}, \xref{313}.

Interrogative pronouns, \xref{141}, \xref{275}.

Interrogative sentence,
    see Questions.

\latin{intrā},
    w. acc., \xref{380}.

Intransitive,
    see Voice and Verbs.

Intransitive verbs,
    passive of, \xref[\emph{c}]{201}, \xref[\emph{a}]{290}.

\suffix{-iō},
    verbs in, of conj.~III, \xref{158}, \xref[I]{168}.

\latin{ipse},
    decl., \xref{139};
    use, \xref{267}–\xref{269};
    agreement of, \xref{268};
    as reflex., \xref{263}, \xref{264};
    \latin{meus ipsīus}, etc., \xref[\emph{b}]{339}.

\latin{īrī},
    in fut. infin. pass., \xref[3]{178}.

Irony,
    \xref[4]{632}.

Irregular nouns of decl.~III,
    \xref{92}.

Irregular verbs,
    \xref{170};
    conj. of, \xref{190}–\xref{197}.

\latin{is},
    decl., \xref{137};
    meaning and uses, \xref{271}–\xref{274};
    sometimes instead of \latin{sē}, \xref[2, \emph{a}]{262};
    w. \latin{quī}-cl., det. \xref{550};
    descr., \xref[1]{521};
    w. \latin{ut}-cl., \xref[2, \emph{b}]{521}.

\suffix{-īs},
    acc. pl. in, \xref[\emph{a}]{75}, \xref[3]{88}, \xref[4]{118}.

\latin{-īscō},
    inchoatives in, \xref[2]{212}, \xref[\emph{F}, \emph{a}]{168}.

Islands,
    gend., \xref[2]{58};
    constrs. of place w., \xref{449}–\xref{451}.

\latin{iste},
    decl., \xref[2]{138};
    meaning, \xref{271}, \xref[4]{274}.

\latin{ita},
    in answers, \xref[1]{232};
    \latin{ita ut} or \latin{nē}, w. vol. subj., \xref[2,
      \emph{a}]{502};
    \latin{ita ut}, \latin{ut nōn}, etc., \xref[2, \emph{d}]{521};
    \latin{ita ut}, of way by wh., \ibid;
    \latin{ita sī}, \xref[5]{578}.

\latin{itaque},
    accent, \xref[2, n.]{32};
    use, \xref[3]{311}.

\latin{iter},
    decl., \xref[6]{80}.

“Iterative” subj.,
    see Repeated action.

Iterative verbs,
    see Frequentatives, \xref[1]{212}.

\suffix{-itō},
    frequentatives in, \xref[1]{212}.

\latin{iubeō},
    w. acc., \xref[1]{397};
    w. infin., \xref{587};
    w. subj., \ibid[\emph{b}];
    \latin{iubeor}, w. infin., \xref{588}.

\latin{iūgerum},
    decl., \xref[2]{107};
    as measure, \xref[\emph{b}]{676}.

\ending{-ium},
    gen. pl. in, \xref{64}, \xref[4]{88}, \xref[1]{91}, \xref[1]{118}.

\latin{iungō},
    constr. w., \xref[\emph{c}]{431}.

\latin{Iuppiter},
    decl., \xref{92}.

\latin{iūrātus},
    \english{having sworn}, \xref{161}, \xref[\emph{a}, 4)]{290}.

\latin{iūs},
    decl., \xref[1]{86};
    \latin{iūs est}, w. cl. of obligation, \xref[5]{513};
    \latin{iūs est bellī}, w. vol. cl., cf.\ \xref[3, \emph{c})]{502},
    and ftn.

\ending{-ius},
    pronom. gen., quantity of \phone{i} in, \xref[2]{21},
    \xref[n.]{112};
    \latin{-ĭus} in gen. in poetry, \xref{653}.

\suffix{-ius},
    suffix, \xref[1]{210}, \xref[2]{215};
    gen. and voc. sing. of nouns in, \xref[2, 3]{71};
    of adjs. in, \xref[\emph{a}]{110}.

\latin{iussū},
    abl. only, \xref[1]{106};
    case, \xref[\emph{a}]{414}.

\latin{iuvenis},
    decl., \xref[4]{88};
    compar., \xref{122}, \xref[\emph{b}]{123}.

\latin{iūxtā},
    w. acc., \xref{380}.

\ending{-īvī},
    perf. contracted, \xref[\emph{A}]{173};
    short forms of, \xref[1, 3]{163}.

\bigskip

\grapheme{J},
    letter, \xref[n.]{1}

Judging, person, in dat., \xref{370}.

“Jussive,”
    see Volitive subjunctive.

Juxtaposition,
    in cpds., \xref[3]{214}, \xref[2]{218}.

\bigskip

\grapheme{K},
    letter, \xref[\emph{a}]{1}.

Kindred meaning,
    acc. of, \xref{396}.

“Knowing,”
    w. infin., \xref{589}.

\bigskip

Labials,
    \xref[1]{7}, \xref{12};
    changes of, \xref[8, 10]{49};
    stems in, \xref{76}, \xref[1]{77}, \xref{90}.

\latin{laetor},
    w. neut. acc. pron., \xref[2]{397};
    w. abl., \xref[\emph{a}]{444};
    w. infin., \xref{594}.

\latin{laetus},
    w. force of adv., \xref{245};
    w. abl., \xref[\emph{a}]{444}.

Leading idea not in principal noun,
    \xref{333}.

Leap year,
    calendar for, \xref{669}, \xref{671}.

\latin{lēge},
    \english{by law}, \xref[\emph{a}]{414};
    \english{under the condition}, \xref[\emph{b}]{436}.

Length,
    measures of, \xref{676}.

Lengthening in poetry,
    \xref{654};
    see also \xref{652}.

Less vivid future condition and conclusion,
    \xref{580}.

\latin{līber},
    decl., \xref{111};
    constr. w., \xref[\emph{a}, \emph{b}]{411}.

\latin{līberī},
    pl. only, \xref[4]{104};
    gen. pl. of, \xref[4, \emph{b})]{71}.

\latin{līberō}, \english{acquit},
    w. gen., \xref{342};
    w. abl., \xref[3 and exc.~1]{408}.

\latin{licet},
    conj., \xref{201};
    w. dat., \xref{362};
    w. subj., \xref[2]{531};
    = \english{although}, \xref[2, \emph{a}]{532};
    w. infin., \xref{585};
    w. pred. dat., \ibid[\emph{c}].

“Likeness,”
    adjs. of, w. dat., \xref[III]{362};
    w. gen., \xref[\emph{c}]{339}.

\latin{linguā},
    as loc. abl., \xref[\emph{b}]{436}.

“Linguals,”
    see Dentals.

Liquids,
    \xref[1]{6};
    stems in, \xref{79}–\xref{81}.

\latin{lītore},
    w. and without \latin{in}, \xref{436}.

Lítotes,
    \xref[1]{632}.

Local point of view,
    dat. of, \xref[\emph{a}]{370}.

Locative,
    \emph{Form}, \xref[\emph{b}]{61};
    decl.~I, \xref[5]{66};
    decl.~II, \xref[6]{71};
    decl.~III, \xref{93};
    in pron. advs., \xref[7]{127};
    \latin{domī}, \latin{humī}, etc., \xref[\emph{a}]{449};
    locative of names of towns, etc., \xref{449};
    appositve to, in abl. w. prep., \xref{452}.

Locative ablative,
    \xref[2]{334}.

\latin{locō},
    w. gerundive, \xref[III]{612}.

\latin{locus},
    pl. of, \xref[2]{108};
    \latin{locō}, \latin{locīs}, in abl. w. or without prep., \xref{436}.

\latin{longius},
    w. abl., or without effect on case, \xref[\emph{d}]{416};
    w. abl. of noun of time, \xref[\emph{a}]{417}.

\latin{longum est}, etc.,
    mood, \xref[3, \emph{b})]{582}.

Loosely attached descr. cl. w. \latin{quī}, etc., \xref{568}.

\bigskip

\phone{m},
    final, in slurring, \xref{34}, \xref{646}.

\latin{magis},
    use in compar., \xref{121}.

\latin{magnopere}, \latin{magis}, \latin{maximē},
    compar., \xref{129}.

\latin{magnus},
    compar., \xref{122};
    \latin{magnī}, \ending{-ō}, of value or price, \xref[1]{356},
        \xref[2, \emph{a})]{427}.

Main (or principal) sentence or cl.,
    \xref[1]{224}.

\latin{maior},
    w. \latin{nātū}, \xref{441};
    w. \latin{quam quī} or \latin{ut}, \xref[2, \emph{c}]{521}.

\latin{maius},
    pronunc., \xref[2, \emph{a}]{29}.

\latin{male},
    quantity of~\phone{e}, \xref[2, \emph{c})]{28};
    compar., \xref{129}.

\latin{mālō},
    conj., \xref{192};
    w. vol. cl., \xref[3, \emph{a})]{502};
    w. infin., \xref{586}, \xref{587};
    \latin{mālim}, \latin{māllem}, w. subj., = a wish, \xref[1,
      \emph{c}]{519}.

\latin{malus},
    compar., \xref{122}.

\latin{maneō}, \english{abide by},
    constr. w., \xref[2, \emph{c}]{438}.

Manner,
    expr. by abl., \xref{445};
    by \latin{ad}, \latin{in}, or \latin{per} w. acc., \ibid[3, \emph{a}];
    by abl. absolute, \xref[8]{421};
    by partic., \xref[5]{604}.

Masculine caesura,
    \xref[\emph{a}]{641}.

Material,
    gen. of, \xref{349};
    may be expr. by \latin{ex} (poetic \latin{dē}) w. abl.,
        \xref[4]{406};
    prep. may be omitted in poetry, \ibid[\emph{a}].

Material,
    suffix denoting, \xref[1]{209}.

\latin{mātūrus},
    comparison, \xref[1]{120}.

\latin{maximē},
    use in comparison, \xref{121}.

\latin{maximī},
    gen. of value or price, \xref[1]{356}.

“May,” “might,”
    expr. by potential subj., \xref{516}, \xref{517};
    by \latin{possum} w. infin., \xref{586}.

Means,
    expr. by abl., \xref{423};
    by abl. absolute, \xref[7]{421};
    persons as means, \xref[\emph{a}]{423}.

“Measure of difference,”
    see \xref{424}.

Measures of money, weight, etc.,
    \xref{672}–\xref{677}.

\latin{mēcum},
    \xref[\emph{a}]{418}.

Meditatives,
    \xref[4]{212}.

\latin{medius}, \english{the middle of},
    \xref{244};
    w. abl. noun, \xref[and \emph{a}]{436}.

\latin{meī},
    gen. of \latin{ego}, reg. objective, \xref[\emph{a}]{254}.

\latin{melior},
    decl., \xref{116};

\latin{meminī},
    conj., \xref{199};
    case-constrs. w., \xref[and \emph{a}]{350};
    w. infin., \xref{589};
    force of tenses, \xref[1]{199}, \xref{487};
    \latin{mementō}, \xref[\emph{c}]{496}.

\latin{memor},
    \xref[\emph{b}]{117}, \xref[1, \emph{a}, 2)]{118};
    w. gen., \xref{354}.

\suffix{-men}, \suffix{-mentum},
    suffixes, \xref[3]{206}.

Mental action,
    obj. of, in gen., \xref{350}, \xref{351}.

\suffix{-met},
    particle, \xref[\emph{d}]{134}.

Metaphor,
    \xref[13]{632}.

Metónymy,
    \xref[9]{632}.

Metre,
    defined, \xref{638}.

\latin{metuō},
    w. dat. or acc., \xref{367};
    w. subj. cl. w. \latin{nē} or \latin{ut}, \xref[4]{502};
    w. infin., \xref{586}.

\latin{meus},
    decl., \xref[\emph{a}]{136};
    voc. \latin{mī}, \ibid

\latin{mī},
    dat. of \latin{ego}, \xref{134}.

Middle voice,
    \ftn{158}{}, ftn.;
    w. acc., \xref[\emph{b}]{390}.

\latin{mihi},
    quantity of final~\phone{i}, \xref[n.]{28}, \xref[\emph{a}]{134};
    \latin{mihī}, in poetry, \ibid, \xref[1]{652}.

\latin{mīles},
    final syll., \xref[3]{30}, \xref[4]{49};
    decl., \xref{76}.

\latin{mīlitiae}, \english{in war}, \english{in the field},
    \xref[\emph{a}]{449}.

\latin{mīlle},
    \xref[3]{131};
    use, \xref[3, 4]{131}.

\latin{minimē},
    compar., \xref{129};
    in answers, \xref[2]{232}.

\latin{mininī}, \suffix{-ō},
    of value, \xref[1]{356};
    \xref[2, \emph{a})]{427}.

\latin{minor nātū},
    \xref{441};
    \latin{minōris}, of value, \xref[1]{356}.

\latin{minus},
    comparison, \xref{129};
    w. abl., or without effect on case, \xref[\emph{d}]{416};
    w. abl. of nouns of time, \xref[\emph{a}]{417}.

\latin{mīror},
    conj., \xref{160};
    rare gen. w., \xref[3]{352}.

\latin{misceō},
    constr. w., \xref[and \emph{a}, \emph{b}]{431}.

\latin{misereor}, \latin{miserēscō},
    w. gen., \xref[2]{352}.

\latin{miseret},
    acc. and gen. w., \xref[1]{352}, \xref[\emph{a}]{390}.

Mixed conditions and conclusions,
    \xref[1]{582}.

Mixed stems,
    \xref[\emph{C}]{74}, \xref{90}, \xref{91}.

“Modest (softened) statements,”
    \xref[1, \emph{b}]{519}.

\latin{modo},
    short \suffix{-o}, \xref[4]{28};
    in cl. of proviso, \xref{529}.

\latin{moneō},
    conj., \xref{156};
    w. neut. acc., \xref[1]{397};
    w. vol. cl., \xref[3, \emph{a})]{502};
    w infin., \xref{587}.

Money,
    Roman, \xref{672}–\xref{675}.

Monosyllables,
    quantity of, \xref{25}–\xref{28}.

Moods,
    \xref{145};
    mood-sign of subj., \xref{175};
    mood defined, \xref{460};
    table of forces, \xref{462};
    general sketch of historical relations, \xref{459}.

Months,
    names of, \xref{662};
    gend., \xref[1 and \emph{a}]{58}.

\latin{mora},
    \ftn{344}{2}.

\latin{mōre},
    of accordance, \xref[\emph{a}]{414};
    of manner, \xref[1]{445}.

More vivid future condition and conclusion,
    \xref[\emph{a}]{579}.

\latin{mōs est},
    w. vol. subj., \xref[3, \emph{c})]{502};
    w. subj. cl. of actuality, \xref[3, \emph{a})]{521};
    w. infin., \xref{585}.

Motion toward,
    see Place whither, \xref{385}.

\latin{multī sunt quī},
    subj. or indic. w., \xref[1, \emph{b}]{521}.

Multiplicatives,
    \xref{133}.

\latin{multum}, \emph{much},
    \xref[6, n.]{126}, \xref[III]{387};
    comparison, \xref{129}.

\latin{multus},
    comparison, \xref{122};
    denoting a part, \xref{244}.

Mute and liquid,
    pronounced in same syll., \xref[2, n.]{14};
    separated in poetry, \xref{655}.

Mutes or stops,
    \xref[4]{6}, \xref{12};
    mute stems, \xref{76}.

\medskip

\latin{nam},
    use, \xref[6]{311}.

\enclitic{-nam},
    interrogative enclitic, \xref[2, n.]{231}

Names,
    Roman, \xref{678};
    in adoption, \ibid[4].

\latin{namque},
    \xref[6, \emph{a}]{311}.

Narrative clause,
    subj., w. \latin{cum}, \xref{524};
    indic., w. \latin{ubi}, etc., \xref{557};
    of situation, w. same, \xref{558};
    w. \latin{dum}, \latin{dōnec}, etc., \xref{560};
    w. \latin{antequam} or \latin{priusquam}, \xref{561};
    narr. partic., \xref[2, n.]{604}

Nasalized vowels,
    \xref[3]{4};
    before \phone{ns}, \xref{11}, \xref{18};
    before \phone{-m}, \xref[2]{34}.

Nasals,
    \xref[2]{6}, \xref{12};
    changes, \xref[9]{49};
    stems in, \xref{82}–\xref{84};
    pres. w. inserted \phone{n}, \xref[\emph{C}]{168}.

\latin{nātū},
    abl. only, \xref[1]{106};
    w. \latin{maior}, etc., \xref{441}.

Natural gender,
    \xref[\emph{a}]{56}.

Natural likelihood,
    subj. of, \xref{514}, \xref{515}.

\phone{nct},
    length of vowel before, \xref{18}.

\prefix{ne-},
    prefix, \xref[3]{24}.

\enclitic{-ne},
    encl., interrog., \xref{33};
    added to forms in \enclitic{-ce}, \xref[2, \emph{d}]{138};
    shortened to \enclitic{n}, \xref[n.]{28}, \xref[n.~3]{231};
    use and position, \xref[1, \emph{b})]{231};
    \latin{-ne\ellipsis an}, \xref{234};
    \latin{-ne\ellipsis -ne}, \ibid[\emph{b}];
    w. exclamatory infin., \xref{596}.

\latin{nē}, \english{surely},
    \xref[8]{302}.

\latin{nē}, \english{not}, \english{lest},
    general statement of use, \xref[1]{464};
    fitting changed meaning, \ibid[2];
    details:
        w. imper., \xref{496};
        w. subj., see especially prohibitions, \xref[3]{501};
        cl. of purpose, \xref[2]{502};
        in vol. subst. cl., \ibid[3];
        in cl. of fear, \ibid[4];
        in wishes, \xref[1]{511};
        in opt. subst. cl., \ibid[2];
        sometimes in statements of obligation or propriety,
            \xref[\emph{a}]{512}.

\latin{nē nōn},
    in cl. of fear, \xref[4]{502}.

\latin{nē\ellipsis quidem}, \english{not even};
    w. all moods, \xref[1, \emph{a}]{464};
    simply adds emphasis, \xref[2, \emph{a}]{298}.

“Nearness,”
    see dat. of relation, \xref{362}.

\latin{nec},
    see \latin{neque}.

\latin{nec enim},
    \xref[6, \emph{b}]{311}.

\latin{necesse est},
    w. vol. subj. cl., \xref[3, \emph{c})]{502};
    w. infin., \xref{585}.

\latin{necne},
    alternative in questions, \xref[\emph{a}]{234}.

\latin{nēdum}, \english{still less},
    w. subj., \xref{505}.

Negative particles,
    see \latin{nē} and \latin{nōn};
    two negatives, \xref[2]{298}.

\latin{nēmō},
    \xref[3]{106};
    for \latin{nūllus}, \xref[9, \emph{c}, \emph{d}]{276}.

\latin{nēquam},
    compar., \xref{122}.

\latin{neque}, \latin{nec}, \english{and not},
    \xref[1]{464};
    choice of forms, \xref[3, \emph{c}]{307};
    correl., \xref{309}.

\latin{nesciō},
    w. infin., \xref{586}, \xref{589};
    \latin{nesciō an}, \xref[\emph{f}]{537};
    \latin{nesciō quis}, \latin{quō pactō}, etc., \xref[4]{276};
    same not affecting mood, \xref[\emph{e}]{537}.

\latin{neuter},
    pronunc., \xref[\emph{b}]{10};
    decl., \xref[\emph{a}]{112};
    use, \xref[9]{276};
    meaning in pl., \ibid[\emph{a}].

Neuter acc. as adv.,
    \xref[6, n.]{126}, \ftn{209}{2}.

Neuter adj. as pred. w. nouns of any gend.,
    \xref[\emph{c}]{325};
    neut. pl. w. gen., \xref{357}.

Neuters,
    see Gender.

\latin{nēve}, \latin{neu},
    \xref[3]{307};
    use w. moods, see \latin{nē}.

\phone{nf},
    length of vowel before, \xref{18}.

\latin{nī},
    use, \xref[4]{578}.

\latin{nihil} (or \latin{nīl}, quantity, \xref[1]{25}, \xref{45}),
    indecl., \xref[2]{106};
    as acc. of degree, \xref[III]{387};
    constr. of adjs. w., \xref[\emph{a}]{346};
    \latin{nihil reliquī faciō}, etc., \xref[\emph{a}]{340};
    \latin{nihil abest quīn}, \xref[3, \emph{b})]{502};
    \latin{nihil est quod}, \latin{quārē}, etc., \xref[2]{513}.

\latin{nihilī},
    descr. gen., \xref[\emph{a}]{355};
    \latin{nihilī}, \suffix{-ō}, of value or price, \xref[2]{356},
        \xref[2, \emph{b})]{427}.

\latin{nisi}, \latin{nisi \emend{62}{si}{sī}}, \latin{forte}, etc.,
    \xref{577}, \xref{578};
    \latin{nisi} w. abl. absolute, \xref[6, \emph{a}]{421};
    meaning \english{except} or \english{but}, \xref[3,
      \emph{b}]{578}.

\latin{nītor},
    w. abl., \xref[1]{438}.

\latin{nōlī} in prohibitions,
    \xref[3, \emph{a}, 2)]{501}.

\latin{nōlō},
    conj., \xref{192};
    w. vol. cl., \xref[3, \emph{a})]{502};
    w. infin., \xref{586}, \xref{587}.

Nominative,
    \emph{Form}:
        decl.~III, \ftn{36}{}, ftn., \xref{75}, \xref[1]{77}, \xref{80},
            \xref{83}, \xref[n.]{86}, \xref{87}, \xref[n.]{92};
        decl.~IV, \xref{96};
        decl.~V, \xref{99}.
    \enskip
    \emph{Syntax}:
        as subject, \xref{335};
        as attributive, appositive, or pred., \xref{317}–\xref{321};
        in exclamations, \xref[\emph{a}]{399};
        for voc., \xref{401}.

\latin{nōn},
    general statement of uses, \xref[1]{464}.

\latin{nōn modo\ellipsis sed nē\ellipsis quidem},
    \xref{299}.

\latin{nōn nēmō} and \latin{nōn nūllus},
    \xref[6 and \emph{a}]{276}.

\latin{nōn quia}, etc.,
    w. subj., \xref[2, \emph{b}]{535}.

Nones, \latin{Nōnae},
    \xref{664};
    use in dating, \xref{665}–\xref{671}.

\latin{nōnne},
    interrog. particle, \xref[1, \emph{c})]{231}.

\latin{nōs} = \latin{ego}, \latin{noster} = \latin{meus},
    \xref{259}.

\latin{nostrī},
    objective, \latin{nostrum}, gen. of the whole,
        \xref[\emph{b}]{134}, \xref[\emph{a}]{254}.

Nouns,
    gender, \xref{56}–\xref{59};
    number, \xref{60};
    cases, \xref{61}–\xref{62};
    decl., \xref{63}–\xref{108};
    used only in sing., \xref{103};
    only in pl., \xref{104};
    w. different meaning in sing. and pl., \xref{105};
    defect. in case-forms, \xref{106};
    variable in decl., \xref{107};
    variable in gend., \xref{108};
    deriv. of, \xref{203}–\xref{207};
    classification of cpds., \xref{214};
    verbal nouns, \xref{146};
    noun defined, \xref{221};
    kinds of, \xref{240};
    as adjs., \ibid[2, \emph{b}];
    appos., \xref[I]{319};
    pred., \ibid[II].

\latin{nōvī}, etc.,
    force of tenses, \xref{487}.

\phone{ns},
    length of vowel before, \xref{18}.

\latin{nūlla causa est cūr}, \latin{quārē}, \latin{quīn}, etc.,
    cf.\ \xref[2]{513}.

\latin{nūllus},
    decl., \xref[\emph{a}]{112};
    use, \xref[9 and \emph{b}]{276}.

\latin{num},
    interrog. particle, \xref[1, \emph{d})]{231};
    in indirect questions, \xref[\emph{d}, 2)]{537}.

Number,
    in nouns, \xref{60};
    in verbs, \xref{145};
    nouns used only in sing., \xref{103};
    only in pl., \xref{104};
    with difference in meaning, \xref{105};
    in agreement, \xref{318}–\xref{332}.

Numerals,
    \xref{130}–\xref{133};
    uses, \ibid\ and \xref{247}.

\latin{numquis},
    decl., \xref[\emph{b}, n.]{141};
    use, \xref[1]{276}.

\latin{nunc},
    compared w. \latin{iam}, \xref[4, 5]{302}.

\latin{nūper},
    comparison, \xref{129}.

\phone{nx},
    length of vowel before, \xref{18}.

\bigskip

\latin{ō sī},
    in virtual wish, \xref[5]{582}.

\latin{ob},
    form in cpds., \xref[11]{51};
    w. acc., \xref{380};
    cpds. of, w. dat., \xref{376}.

“Obeying,”
    w. dat., \xref{362}.

\latin{obiciō},
    quantity of first syll., \xref[1]{30}.

Object,
    concrete obj. for wh., w. dat., \xref{361};
    ind. in dat., \xref{365};
    direct in acc., \xref{390}, \xref{391};
    obj. cls., see Substantive clauses.

Objective genitive,
    \xref{354}.

Obligation,
    expr. by subj., \xref{512}, \xref{513};
    by fut. pass. partic., \xref[3]{600};
    by same used impersonally, \ibid[\emph{a}];
    moods and tenses in verbs of, \xref[3]{582};
    in ind. disc., \ibid[\emph{a}),n.~2].

Obligation or propriety, subj. of, \xref{512}, \xref{513}\emend{63}{}{.}

Oblique cases,
    \xref[\emph{a}]{61}.

\latin{oblīvīscor},
    constrs. w., \xref{350}.

Obstructed consonants,
    \xref[2, \emph{b}]{14}, \xref{37}.

\latin{ōcior},
    comparison, \xref{123}.

\latin{ōdī},
    meaning of tense,
    \xref[1]{199}, \xref{487}.

Omission of verb,
    \xref[\emph{a}]{222}, \xref[1, \emph{a}]{631};
    of subject, \xref{257}, \xref{285};
    of antecedent, \xref[1]{284}.

Onomatop\'œia,
    \xref[20]{632}.

Open syllables, \xref[\emph{a}]{14};
    open vowels, \xref{3}.

\latin{opīniōne},
    after compar., \xref[\emph{e}]{416}.

\latin{oportet},
    moods and tenses of, \xref[3]{582};
    w. subj. cl., \xref[5]{513};
    w. infin., \xref{585}.

\latin{oppidum},
    in appos. to names of towns, \xref{452}.

Opposition,
    see Adversative.

Optative,
    \emph{Form}, \xref[n.]{175};
    opt. subj., uses, \xref{510}, \xref{511}.

\latin{optimum est},
    moods w., \xref[3, \emph{c})]{502}, \xref{585}.

\latin{optō},
    w. opt. cl., \xref[2]{511};
    w. infin., \xref{586}, \xref{587}.

\latin{opus est},
    w. abl., \xref[1]{430};
    of partic., \ibid[2];
    w. supine in \ending{-ū}, \xref[2]{619};
    w. vol. cl., \xref[3, \emph{c})]{502};
    w. infin., \xref{585};
    \latin{opus} as pred., \xref[2, \emph{a}, \emph{b}]{430}.

\latin{ōrātiō oblīqua},
    see indirect discourse, \xref{533}.

\latin{ōrātiō rēcta},
    see direct discourse, \xref{533}.

Ordinals,
    \xref{130}, \xref{131};
    w. \latin{quisque}, \xref[2, \emph{c})]{278}.

Origin,
    abl. of, \xref{413};
    w. \latin{ab} or \latin{ex}, \ibid[\emph{a}, \emph{b}].

\latin{ōrō},
    w. two accs., \xref{393};
    w. subj. cl., \xref[2]{530}.

Orthography,
    \xref{52}.

Oxymóron,
    \xref[3]{632}.

\bigskip

\latin{paene},
    position of, \xref[13, \emph{a}]{624}.

\latin{paenitet},
    cases w., \xref[1]{352};
    mood w., \xref{585}.

\latin{palam},
    as adv., or w. abl., \xref[1, \emph{a}]{407}.

Palatals,
    \xref[3]{7}.

\latin{pār},
    quantity in, \xref[1]{25};
    cases w., \xref[\emph{c}]{339}.

Parallel \latin{cum\ellipsis tum},
    \xref{564}.

Parallel order, \xref{628}.

Parataxis defined,
    \xref{227};
    paratactic uses, imper., \xref[2]{497};
    subj., \xref[1]{504}, \xref[1, \emph{b}]{511}, \xref[1]{530},
        \xref[1]{532};
    indic., \xref[\emph{b}]{545}.

\latin{parātus},
    constrs. w., \xref[2, \emph{a}, \emph{b}]{384},
        \xref[\emph{f}]{586}.

“Pardoning,”
    w. dat., \xref{362}.

Parentage or origin,
    constr., \xref{413}.

Parenthetical cl.,
    \xref{567};
    partic. =, \xref[7, \emph{c})]{604}.

\latin{pars},
    \xref[1, \emph{a}]{91};
    \latin{parte}, in loc. abl., \xref{436};
    \latin{partem}, acc. of respect, \xref{388}.

Part, idea of,
    denoted by adjs., \xref{244}.

Participles,
    \emph{Form},
        pres. act., \xref{183};
        decl., \xref{117}, \xref{118};
        fut. act., \xref{182};
        perf. pass., \xref{179};
        fut. pass., \xref{184}.
    \enskip
    \emph{Syntax}:
        nature, \xref{599};
        used as adjs., \xref{248};
        w. adv. force, \xref{245};
        as subst., \xref{249}, \xref{250};
        used impersonally in abl. absolute, \xref[8, \emph{a}]{421};
        perf. pass. modified by adv. or adj., \xref[2, n.]{250};
        perf. pass. w. act. meaning, \xref[\emph{a}, 4)]{290};
        w. pres. force, \xref{601};
        agreement, \xref{320};
        carrying leading idea, \xref{333}, \xref{608};
        tenses, meanings of, \xref{600}, \xref{601};
        of attempted action, \xref{484};
        w. \latin{iam diū}, etc., \xref{485};
        voice-meanings, \xref{146};
        of deponents and semi-deponents, \xref{291};
        lacking perf. act. and pres. pass. partics., how supplied,
            \xref{602};
        fut. pass. w. occasional pres. pass. force, \xref[3,
          \emph{b}]{600};
        common uses, \xref{604};
        special idioms, \xref{605};
        new uses in later Latin, \xref{606}, \xref{607}.

Particles,
    defined, \xref[\emph{d}]{221}.

“Particular conditions,”
    see \xref[and \sftn{}]{567}

\latin{partior},
    conj., \xref{160}.

Partitive apposition,
    \xref[I, \emph{a}]{319}.

“Partitive genitive,”
    \ftn{183}{}, ftn.

Parts of speech,
    \xref{53};
    defined, \xref{221}.

\latin{parum},
    comparison, \xref{129}.

\latin{parvus},
    comparison, \xref{122};
    \latin{parvī} or \ending{-ō}, of value or price, \xref[1]{356},
        \xref[2, \emph{a})]{427}.

Passive,
    see Voice.

Past aorist,
    indic., \xref[4, \emph{a}]{468};
    subj., \xref[2]{470}.

Past-future expressions,
    periphr. indic. and antic. subj., \xref[and \sftn{}]{508};
    dep. past-fut. cls. necessarily in antic. subj., \ibid;
    past-fut. condition and conclusion, in subj.,
        \xref[\emph{b}]{580};
    in periphr. fut. indic., \ibid[\emph{c}].

Past perfect,
    \textsc{Indicative}:
        \emph{Form}, \xref[1]{174};
        meaning, \xref[5]{468};
        of rapid succession of events, \xref{492};
        epistolary, \xref{493};
        instead of subj. contrary to fact, \xref[\emph{e}]{581};
    \enskip
    \textsc{Subjunctive}:
        \emph{form}, \xref[\emph{c}]{175};
        meaning, \xref[1]{470};
        in wishes, \xref[\emph{a}]{510};
        in conditions and conclusions, \xref{581}.

Patronymics,
    Greek, \xref[3]{207}.

Penalty,
    gen. of, \xref{343};
    abl. of, \xref{428}.

\latin{penes},
    w. acc., \xref{380}.

Pentameter,
    dactylic, \xref{642}.

Penult,
    \xref[2]{31}.

\latin{per},
    forms of, in cpds., \xref[12]{51};
    as prefix, \xref[\emph{a}]{218};
    w. acc., \xref{380};
    of persons as means, \ibid[\emph{d}];
    of duration of time, \xref[II, \emph{a}]{387};
    of route, \xref[\emph{b}]{426};
    of cause, \xref[\emph{c}]{444};
    of manner, \xref[3, \emph{a}]{445}.

\enclitic{-per},
    enclitic particle, \xref[9]{127}.

Peremptory command,
    expr. by imper., \xref{496}.

Perfect,
    \textsc{Indicative}:
        \emph{Form}, perf. system of, \xref[\emph{B}]{147};
        ends., \xref[\emph{a}]{151};
        short forms of, \xref{163};
        types of, \xref{173};
        pass., \xref[8]{164};
        meanings, \xref[4 and \emph{a}]{468};
        tenses of dep. verb w., \xref{476}, \xref{479}, \xref{481};
        perf. of experience (“gnomic”), \xref{488};
        of act or state no longer existing, \xref{489};
        energetic perf., \xref{490};
        picturesque, \xref[1]{491};
        of rapid succession of events, \xref{492};
        \latin{nōvī}, \latin{meminī}, \latin{ōdī}, \latin{coepī},
            etc., \xref{199}, \xref{487};
    \enskip
    \textsc{Subjunctive}:
        \emph{form}, \xref[\emph{b}]{175};
        confusion w. fut. perf. indic. forms, \xref[6]{164};
        in \ending{-sim}, \xref[5]{163};
        meanings of tense, \xref{469}, \xref{470};
        in result-cls., \xref{478}, \xref[2, examples]{521};
    \enskip
    \textsc{Infinitive}:
        \emph{form}, \xref[1–3]{178};
        w. \latin{esse} omitted, \xref[7]{164};
        meaning of tenses, \xref{472};
        in ind. disc., \xref{534}, \xref{589}, \xref{591};
    \enskip
    \textsc{Participle}:
        \emph{form}, \xref{179}, \xref{180};
        meaning of tenses, \xref{473}, \xref{600}, \xref{601};
        perf. w. verbs of wishing, \xref[3]{605}.

Perfect definite,
    see Perfect Indicative.

Period,
    definition of, \xref{630}.

Periphrastic conjugation,
    \xref{162};
    peculiarities in, \xref{163}–\xref{165};
    periphr. fut. tenses, of indic., \xref[7]{468};
    of subj., \xref[4]{470};
    when used in general, \ibid[\emph{a}];
    periphr. fut., in ind. questions, \xref[\emph{d}, 1)]{537};
    in conclusions, indic., subj., or infin., \xref[\emph{c}]{580},
        \xref[\emph{a}, \emph{b}]{581}.

Permanent truths,
    tenses of, in combination w. other tenses, \xref{482}.

\latin{permittō},
    w. subj., \xref[2]{531};
    w. infin., \xref{587}.

Perplexity,
    question of, in subj., \xref{503};
    in pres. indic., \xref{571};
    in fut. indic., \xref{572}.

Person,
    \xref{147};
    order of mention of the three persons, \xref[15]{624};
    person judging, dat. of, \xref{370};
    persons as agents, \xref[1]{406};
    as means through wh., w. \latin{per}, \xref[\emph{d}]{380};
    as means by wh., w. abl., \xref[\emph{a}]{423}.

Personal construction in passive voice,
    \latin{prohibeor}, \latin{putor}, etc., \xref[1]{590}.

Personal endings of verbs,
    \xref{151}, \xref{152}.

Personal pronouns,
    decl., \xref{134};
    use, \xref{254};
    of third pers., how replaced, \xref{255};
    when expr., when omitted, \xref{257};
    pl. of dignity, \xref{259};
    as reflex., \xref[\emph{a}]{260}.

Personification,
    \xref[18]{632}.

\latin{persuādeō},
    w. dat., \xref{362};
    w. dat. and acc., \xref[4]{364};
    w. subj., \xref[3, \emph{a})]{502}.

\latin{pertaesum est},
    constr. w., see \latin{taedet}, \xref[1]{352}.

\latin{petō},
    cases w., \xref[\emph{c}]{393};
    w. subj., \xref[2]{530}.

Phonetic changes,
    \xref{41}–\xref{51}.

Phrase,
    defined, \xref[2, \emph{b}]{224}.

Picturesque tenses,
    \xref{491}.

\latin{piget},
    w. acc. and gen., \xref[1]{352};
    w. infin., \xref{585}.

Place,
    adjs. denoting, \xref[3]{210}.

Place where, whither, or whence,
    reg. expr. by preps., \xref{433}, \xref{385}, \xref[1 and
      \emph{a}, 2 and \emph{a}]{408};
    poets may omit, \xref[\emph{a}]{433}, \xref[\emph{c}]{385},
        \xref[2]{410};
    repeated relations all expr. (\latin{ad Chrȳsogonum ad castra},
        etc.), \xref[\emph{c}]{540};
    constr. of names of towns, \latin{domus}, \latin{rūs}, etc.,
        \xref{449}–\xref{453}.

“Placing,”
    w. \latin{in} and abl., \xref[\emph{c}]{433}.

Plan,
    how expr., \xref[2]{502}.

Plants,
    gend. of names of, \xref[2]{58}.

“Pleasing,”
    w. dat., \xref{362}.

Plenty and want,
    gen. of, \xref{347};
    abl. of, \xref{425}.

Pléonasm,
    \xref[4]{631}.

Pluperfect,
    see Past perfect.

Plural,
    wanting, \xref{103};
    pl. only, \xref{104};
    of dif. meaning from sing., \xref{105}.

\latin{plūs},
    comparison, \xref{122};
    \latin{plūs}, w. abl. or without effect on case,
        \xref[\emph{d}]{416};
    \latin{plūris} and \latin{plūrimī}, gen. of value or price,
        \xref[1]{356}.

Point of reference for tenses,
    \xref[1, \emph{a}]{467}.

Point of view,
    expr. by \latin{ab} or \latin{ex}, \xref[2]{406}.

\latin{pōne},
    w. acc., \xref{380}.

\prefix{por-},
    prefix, \xref[13]{51};
    \xref[1, \emph{b}]{218}.

“Position, length by,”
    see \xref[3]{29}.

Positive degree,
    \xref{119};
    wanting, \xref{123}.

Possession,
    dat. of, \xref[and \emph{a}]{374}.

Possessive compounds,
    \xref[4]{216}.

Possessive genitive,
    \xref{339};
    in pred., \xref{346}.

Possessive pronouns,
    \emph{Form}, \xref{136};
    use, \xref{254};
    of 3d pers., how replaced, \xref{256};
    when expr., when omitted, \xref{258};
    in pl. of dignity, \xref{259};
    as reflexive, \xref[\emph{a}]{260};
    poss. pron. preferred to gen. of personal, \xref[\emph{a}]{339}.

Possibility,
    expr. by potential subj., \xref{516}, \xref{517};
    by \latin{possum} w. infin., \xref{586};
    moods and tenses in verbs of, \xref[3, \emph{a})]{582};
    in ind. disc., \ibid[n.~2].

\latin{possum},
    conj., \xref{191};
    moods and tenses of, in conclusions, \xref[3, \emph{a})]{582};
    in ind. disc., \ibid[n.~2];
    \latin{possum} w. infin., \xref{586};
    \latin{posse} w. pres. infin. = fut. infin., \xref[\emph{d}]{472}.

\latin{post},
    w. acc., \xref{380};
    cpds. of, w. dat., \xref{376};
    as adv., \xref[\emph{c}]{303};
    see also \latin{ante}\emend{64}{}{.}

\latin{posteāquam},
    w. indic., \xref{557}, \xref{558}.

\latin{posterior},
    defect. comparison, \xref{123}.

\latin{postquam},
    w. indic. det. cl., \xref{550};
    replaced by a noun w. \latin{quam} or the abl. of \latin{quī},
        \ibid[\emph{c}\emend{65}{.}{}];
    w. narr. cl., \xref{557}, \xref{558}.

\latin{postrēmus},
    \english{the last to}, \xref{243}.

\latin{postrīdiē},
    w. gen. or acc., \xref[\emph{c}]{380}.

\latin{postulō},
    cases w., \xref[\emph{b}, 2)]{393};
    w. subj., \xref[3, \emph{a})]{502}.

Potential subjunctive,
    \xref{516}, \xref{517}.

\latin{potestās},
    w. infin., \xref[2, \emph{d})]{598}.

\latin{potior},
    w. abl., \xref{429};
    w. acc., \ibid[\emph{b}];
    w. gen., \ibid[\emph{c}];
    in gerundive constr., \xref[n.]{613}

\latin{potior}, defect. adj.,
    comparison, \xref{123}.

\latin{potissimum},
    \xref[6]{302};
    position, \xref[13, \emph{a}]{624}.

\latin{potius},
    comparison, \xref{129};
    meaning, \xref[6]{302};
    position, \xref[13, \emph{a}]{624};
    \latin{potius quam}, w. subj., \xref[4, \emph{d})]{507}.

\latin{prae},
    sometimes shortened, \ftn{9}{}, ftn.;
    w. abl., \xref[1]{407};
    cpds. of, w. dat., \xref{376}.

\latin{praeceps},
    decl., \xref[\emph{b}]{117};
    w. adv. force, \xref{245}.

\latin{praesertim},
    w. \latin{quī} or \latin{cum}, \xref[\emph{b}]{523},
        \xref[\emph{a}]{526}.

\latin{praestōlor},
    w. dat. or acc., \xref[5]{364}.

\latin{praesum},
    conj., \xref{190}, \xref{191};
    w. dat., \xref{376}.

\latin{praeter},
    w. acc., \xref{380};
    w. infin., \xref[3]{598};
    cpd. of, w. acc. \xref[\emph{a}]{386}.

\latin{praetervehor},
    w. acc., \xref[\emph{a}]{386}.

Prayer,
    expr. by imper., \xref{496}.

Predicate,
    defined, \xref{229}, \xref{230}, \xref[3]{317};
    pred. verb, \xref[3, \emph{a}]{317};
    omission of, \xref[\emph{a}]{222};
    pred. acc., \xref[\emph{a}]{392};
    poss. gen. in, \xref{340};
    agreement of pred., \xref{318}–\xref{332}\emend{66}{,}{;}
    agreeing w. subject of main verb, \xref[2]{590},
        \xref[\emph{a}]{592};
    pred. attracted by dat., \xref[3]{326}, \xref[\emph{c}]{585}.

Prefixes,
    adv., spelling, \xref{51};
    lists, \xref{218}.

Prepositions,
    \emph{Form},
        \xref{125};
        assimilation in cpds., \xref{51}.
    \enskip
    \emph{Syntax}:
        definition, \xref{221}, \xref{303};
        origin and and early use, \ibid[\emph{a}];
        cpds. of, taking dat., \xref{376};
        taking acc., \xref[2]{391};
        taking dat. or acc., \ibid[\emph{a}];
        taking two accs., \xref{386};
        preps. w. acc., \xref{380}–\xref{383};
        preps. w. separative abl., \xref{405}–\xref{412};
        w. sociative abl., \xref{418}–\xref{420};
        w. locative abl., \xref{433}–\xref{436};
        w. names of towns, small islands, etc., \xref{453};
        summary of uses of cases w. preps., \xref{455}–\xref{458}.

Present:
    present system, \xref[\emph{A}]{147};
    \emph{Form} of present stem, \xref{166}–\xref{170}:
    meanings common to all forms:
        of permanent truths or customs, \xref[1, \emph{b}]{468};
        of habitual or attempted action, \xref{484};
        w. \latin{iam diū}, etc., \xref{485};
    \enskip
    \textsc{Indicative}:
        \emph{form}, \xref[1, 2]{152}, \xref{166}, \xref{170};
        progressive, \xref[1]{468};
        aor., \ibid[\emph{a}];
        historical pres., \xref[1]{491};
        special uses, \xref{571};
    \enskip
    \textsc{Subjunctive}:
        \emph{form}, \xref{175};
        pres. and fut. forces, \xref{469}, \xref[1]{470};
        w. progressive force, \ibid;
        w. aor. force, \ibid[2];
        referring to fut. in conditions and conclusions, \xref{580};
    \enskip
    \textsc{Infinitive}:
        \emph{form}, \xref{178}, in \ending{-ier}, \xref[3]{164};
        meaning, \xref[and \emph{b}]{472};
    \enskip
    \textsc{Participle}:
        \emph{form}, \xref{183};
        decl., \xref{117}, \xref{118};
        meaning, \xref{473}.

“Preventing,”
    w. subj. cl., \xref[3, \emph{b})]{502}.

Price or value,
    gen. of, \xref{356};
    abl. of, \xref{427}.

\latin{prīdiē},
    w. gen. or acc., \xref[\emph{c}]{380}.

Primary derivatives,
    \xref{203};
    nouns, \xref{206};
    adjs., \xref{208}.

“Primary” tenses,
    \xref[and \sftn{2}]{476}.

Primary verbs,
    \ftn{98}{1};
    of conj.~I, \xref[3, 4]{166};
    of conj.~II, \xref[1, \emph{a}, 2]{167};
    of conj.~III, \xref{168};
    of conj.~IV, \xref[2]{169}.

\latin{prīmō} distinguished from \latin{prīmum},
    \xref[3]{302}.

\latin{prīmus}, \english{the first to},
    \xref{243}.

\latin{prīnceps},
    decl., \xref{76}, \xref[1, 4]{77};
    \xref[1, \emph{a}, 2]{118};
    \english{the first to}, \xref{243}.

Principal and auxliary tenses,
    \xref[\emph{c}]{477}.

Principal parts of verbs,
    \xref{150}.

Principal sentence or clause,
    \xref[1]{224}.

\latin{prior},
    comparison, \xref{123};
    \english{the first to}, \xref{243}.

\latin{prius},
    comparison, \xref{129}.

\latin{priusquam},
    see \latin{antequam}.

\latin{prō},
    \xref[13]{51};
    quantity in cpds., \xref[2]{24};
    w. abl., \xref[1]{407}.

\latin{procul},
    as prep. w. abl., \xref[\emph{c}]{405}.

Progressive action,
    tenses of, \xref[and \sftn{}]{466}

\latin{prohibeō},
    w. abl., \xref[2]{408};
    w. vol. cl., \xref[3, \emph{b})]{502};
    w. infin., \xref{587};
    \latin{prohibeor} w. infin., \xref{588}.

Prohibition,
    how expr., \xref[\emph{d}]{496}, \xref[3]{501};
    in ind. disc. always subj., \xref{538}.

Prolépsis,
    \xref[11]{631}.

“Promising,”
    constr. w., \xref[\emph{a}]{593}.

Pronominal adjectives,
    \xref{112};
    use, \xref{279}.

Pronominal adverbs,
    \xref[3, 4, 6, 7]{126}, \xref{127}.

Pronouns,
    declension:
        pers., \xref{134};
        reflex., \xref{135};
        poss., \xref{136};
        det.-descr., \xref{137}, \xref{138};
        intens., \xref{139};
        rel., \xref{140};
        interrog., \xref{141};
        indef. and distrib., \xref{142};
        pronom. adj., \xref{143};
        correl., \xref{144}.
    \enskip
    \emph{Syntax}:
        prons. defined, \xref{221};
        classification, \xref{253};
        pers., \xref{254}–\xref{259};
        reflex., \xref{260}–\xref{264};
        recipr., \xref{265};
        intens., \xref{267}–\xref{269};
        identifying, \xref{270};
        det.-descr., \xref{271}–\xref{274};
        interrog., \xref{275};
        indef., \xref{276};
        collective, \xref{277};
        distrib., \xref{278};
        pronom. adjs., \xref{279};
        rel. prons. and adjs., \xref{281}–\xref{284};
        agreement, \xref{321}–\xref{325};
        referring to general substance of sentence,
            \xref[\emph{a}]{325};
        prons. in ind. disc., \xref[\emph{a}]{533}.

Pronunciation,
    general explanations, \xref{2}–\xref{8};
    of Latin, \xref{9}–\xref{11}, \xref{13}–\xref{34};
    suggestions and cautions, \xref{35}–\xref{40}.

\latin{prope},
    comparison, \xref{129};
    w. acc., \xref{380};
    position of, \xref[13, \emph{a}]{624};
    \latin{prope ā}, \xref[2]{406}.

Proper nouns, \xref[1]{240}.

\latin{propior},
    comparison, \xref{123};
    w. dat., \xref[III]{362};
    w. acc., \xref[\emph{b}]{380}.

\latin{propius},
    w. acc., \xref[\emph{b}]{380}.

Proposal,
    expr. by imper., \xref{496};
    by vol. subj., \xref[2]{501}.

“Propriety,”
    expr. by subj., \xref{512}, \xref{513};
    tenses of verbs meaning, \xref[3, \emph{a})]{582}.

\latin{proprius},
    w. gen. or dat., \xref[\emph{c}]{339}.

\latin{propter},
    w. acc., \xref{380}.

Prosody,
    see Quantity and Versification.

\latin{prōsper},
    decl., \xref[\emph{a}]{111}.

\latin{prōspiciō},
    w. dat. or acc., \xref{367}.

\latin{prōsum},
    conj., \xref{190}, \xref{191};
    w. dat., \xref{362}.

Protasis,
    see Conditions, \xref{573}–\xref{582}.

\latin{prōvideō},
    w. dat. or acc., \xref{367};
    w. subj., \xref[3, \emph{a})]{502}.

Proviso,
    cl. of, w. \latin{modo}, \latin{dum}, etc., \xref{529}.

\latin{proximē},
    w. acc., \xref[\emph{b}]{380}.

\latin{proximus},
    w. dat., \xref[III]{362};
    w. acc., \xref[\emph{b}]{380}.

\latin{pudet},
    w. acc. and gen., \xref[1]{352}, \xref[\emph{a}]{390};
    w. infin., \xref{585}.

Purpose,
    expr. by dat., \xref{360};
    by acc. w. \latin{ad} or \latin{in}, \xref[3]{384};
    by subj. cl., \xref[2]{502};
    by acc. or gerund or gerundive w. \latin{ad}, \xref[III]{612};
    by gerund or gerundive w. \latin{causā} or \latin{grātiā},
        \xref[I]{612};
    by supine in \ending{-um}, \xref{618};
    by partics., \xref[2]{605}, \xref{606}, \xref{607};
    by poetic infin., \xref[1]{598}.

\bigskip

\latin{quae rēs} for \latin{quod} or \latin{id quod},
    \xref[\emph{a}, n.~1]{325}.

\latin{quaerō},
    cases w., \xref[\emph{c}]{393}.

\latin{quaesō},
    defective, \xref[4]{200}.

\latin{quālis},
    pronom. adj., \xref{143};
    interrog., \xref[5]{275};
    rel., \xref{144};
    w. det. cl., \xref[and \sftn{2}]{550}.

\latin{quāliscumque},
    \xref{282}, \xref{283}.

\latin{quam}-cl.
    determining the degree, \xref[and \sftn{2}]{550};
    \latin{quam diū} cl. determining the time how long, \ibid;
    \latin{quam}, \latin{quam possum}, w. superls., \xref[4]{241}.

\latin{quam}, \english{than},
    w. \latin{alius} or \latin{aliter}, \xref[2, \emph{b}]{307};
    after compars., \xref{416}, \xref{417};
    \latin{quam quī} or \latin{ut} after compars., \xref[2,
      \emph{c}]{521};
    \latin{quam} after infin. and followed by acc., \xref[1,
      \emph{c}]{535}.

\latin{quamobrem},
    see \latin{quārē}.

\latin{quamquam},
    advers. cl. w., \xref{556};
    “corrective,” \xref[7]{310};
    w. subj., \xref{541}.

\latin{quamvīs},
    w. subj., \xref[2]{532};
    w. later indic., \xref{541}.

\latin{quandō},
    in cl. of cause or reason, \xref{555};
    same in ind. disc., \xref[2 and \emph{a}]{535}.

\latin{quantī} or \ending{-ō},
    of value or price, \xref[1]{356}, \xref[2, \emph{a})]{427}.

Quantity of vowels,
    \xref{16}, \xref{17};
    in syll. not final, \xref{18}–\xref{24};
    in final syll., \xref{25}–\xref{28};
    in Greek words, \xref[5]{21}, \ftn{10}{1};
    in cpds., \xref{24};
    quantity of sylls., \xref{29}, \xref{30};
    marks of quantity, \xref{16};
    evidences of, \xref[n.~2]{16};
    list of “hidden quantities,” \xref{679}.

\latin{quantus},
    pronom. adj., \xref{143};
    interrog., \xref[5]{275};
    rel., \xref{144};
    uses, \xref[I]{282};
    w. det. cl., \xref[and ftn.~2]{550}.

\latin{quārē}, \english{why}, \english{wherefore},
    w. subj. of obligation or propriety, \xref{513};
    w. subj. of natural likelihood, \xref{515}.

\latin{quasi},
    w. \latin{quīdam}, \xref[5, \emph{a}]{276};
    w. abl. absolute, \xref[6, \emph{a}]{421};
    w. subj., \xref[3]{504}.

\enclitic{-que},
    encl. particle, \xref[n.]{32}, \xref[1]{33}, \xref[1 and
      \emph{b}]{307};
    \latin{-que\dots-que}, \xref[\emph{a}]{309}.

\latin{queō},
    conj., \xref[\emph{c}]{194}.

Questions,
    classification by form, \xref{231};
    alternative, \xref{234};
    rhetorical, \xref{235};
    absurd, \xref{236};
    for subj. questions see synopsis, \xref{499};
    for indic. questions see synopsis, \xref{543}, and \xref{571},
        \xref{572}.

\latin{quī},
    indef., decl., \xref[and 1, n.]{142}

\latin{quī},
    interrog. adv., \xref[\emph{b}]{140};
    in imprecations, \xref[1, \emph{a}]{511};
    w. subj. of natural likelihood, \xref[1]{515}.

\latin{quī},
    interrog. pron., see \latin{quis}.

\latin{quī}, rel. pron.,
    decl., \xref{140};
    stem, \xref[n.]{141};
    meaning, \xref{282};
    \latin{quī}-cls., in subj., see synopsis, \xref{499};
    in indic., \xref{543};
    in conditional cls., \xref{577}.

\latin{quia}-clauses:
    indic., of cause or reason, \xref{555};
    same in ind. disc., subj., \xref[2 and \emph{a}]{535};
    of rejected reason, \ibid[\emph{b}].

\latin{quibuscum}, \english{with whom}, \english{with which},
    \xref[\emph{a}]{418}.

\latin{quīcum}, \english{with whom}, \english{with which},
    \xref[\emph{b}]{140}.

\latin{quīcumque},
    decl., \xref[\emph{d}]{140};
    meaning, \xref[II]{282};
    as indef., \xref[10]{276}, \xref{283}.

\latin{quid},
    \english{to what extent?} \xref[III]{387};
    \english{in what respect?} \xref{388};
    \english{why?}, \ibid[n.];
    \latin{quid}, \latin{quidnī}, \english{why?} \english{why not?}
        in questions of obligation or propriety, \xref[1]{513};
    of natural likelihood, \xref[1]{515}.

\latin{quid quod},
    \english{what \emph{(of the fact)} that?} \xref[1, \emph{a}]{552}.

\latin{quīdam},
    \xref[3]{142};
    use, \xref[5]{276};
    w. \latin{quasi}, \ibid[\emph{a}];
    w. \latin{ex}, \xref[\emph{e}]{346};
    \latin{quīdam sunt quī}, mood after, \xref[1, \emph{b}]{521}.

\latin{quidem},
    \xref[1]{302};
    w. \latin{tū}, \xref[\emph{a}]{257};
    w. \latin{is}, \xref[3]{274}.

\latin{quīlibet},
    \xref[9]{142};
    use, \xref[8]{276}.

\latin{quīn},
    w. indic., \xref[\emph{a}]{545};
    w. imper. \xref[\emph{b}]{496};
    w. subj. in vol. subst. cl., \xref[3, \emph{b})]{502};
    in cl. of obligation or propriety, \xref[2]{513};
    in cl. of ideal certainty, \xref[2, 4, \emph{b})]{519};
    in cl. of actuality, \xref[1, 2, 3, \emph{b})]{521}.

\latin{quippe},
    w. \latin{quī}-cl., \xref[\emph{b}]{523};
    w. \latin{cum}-cl., \xref[\emph{a}]{526}.

\latin{quis}, indef.,
    decl., \xref[1]{142};
    use, \xref[1]{276}.

\latin{quis}, interrog. pron., and \latin{quī}, interrog. adj.,
    decl., \xref{141};
    distinction not always observed, \ibid[\emph{a}],
        \xref[4, \emph{a}]{275};
    stem, \xref[\emph{b}, n.]{141};
    cpds., \xref[\emph{b}]{141}, \xref{142};
    meaning, \xref[1, 4]{275}.

\latin{quīs},
    for \latin{quibus}, \xref[\emph{c}]{140}.

\latin{quisnam}, \latin{quīnam},
    \xref[\emph{b}]{141}.

\latin{quispiam},
    \xref[4]{142};
    use, \xref[3]{276}.

\xref{quisquam},
    \xref[5]{142};
    use \xref[7]{276}.

\latin{quisque},
    \xref[6]{142};
    use, \xref[2]{278};
    in partitive apposition, \xref[I, \emph{a}]{319};
    w. pl. verb, \xref[2]{331};
    in agreement, or w. gen. of the whole, \xref[\emph{b}]{346}.

\latin{quisque},
    decl., \xref[\emph{d}]{140};
    use, \xref[II]{282}.

\latin{quīvīs},
    \xref[8]{142};
    use, \xref[8]{276}.

\latin{quō},
    in cl. of purpose, \xref[2 and \emph{b}]{502}.

\latin{quoad},
    see \latin{dum}.

\latin{quod}, rel., = \latin{id quod},
    \xref[\emph{a}, n.~2]{325};
    \latin{quod sciam}, etc., \xref[1, \emph{f}]{521}.

\latin{quod}-clauses,
    indic., of time included in the reckoning, \xref[and
      \sftn{2}]{550};
    of equivalent action, \xref{551};
    subst. cl., \xref[1 and \sftn{1}]{552};
    \latin{quid quod\dots?} \ibid[\emph{a}];
    \latin{quod}-cl. or respect (\english{as to the fact that}),
        \xref[2]{552};
    of cause or reason, \xref{555};
    subj., of cause or reason in ind. disc., \xref[2, \emph{a}]{535};
    of rejected reason, \ibid[\emph{b}];
    of obligation or propriety, \xref[2]{513}.

\latin{quom},
    earlier form of \latin{cum}, \xref[1]{44}.

\latin{quōminus},
    in subst. vol. cl., \xref[3, \emph{b})]{502}.

\latin{quoniam},
    see \latin{quia}-clauses.

\latin{quoque},
    \xref[2]{302};
    added to \latin{sed} or \latin{vērum}, \xref[4, \emph{b}]{310}.

\latin{quot}, correl.,
    \xref{144};
    meaning, \xref[I]{282}.

\latin{quot} and \latin{quotus}, interrog.,
    \xref[3]{275}.

\latin{quotcumque},
    meaning, \xref[II]{282}.

Quoted expressions,
    gender of, \xref[3]{58}.

Quoted reason,
    subj. cl. of, \xref[2, \emph{a}]{535}.

\latin{quotquot},
    meaning, \xref[II]{282}.

\latin{quotus quisque},
    \xref[2, \emph{c})]{278}.

\bigskip

\prefix{re-}, \prefix{red-}, prefix,
    \xref[2]{24}, \xref[15]{51}, \xref[1, \emph{b}]{218}.

Reason,
    see Cause.

Reciprocal pronouns,
    \xref{265};
    equivalent phrases, \xref{266}.

Recomposition,
    \xref[n.~2]{41}, \xref{50}.

\latin{recūsō},
    w. subj., \xref[3, \emph{b})]{502};
    w. infin., \xref[\emph{d}]{586}.

Reduplication,
    in present, \xref[\emph{B}]{168};
    in perf., \xref[\emph{D}]{173};
    in cpds., \ibid[\emph{a}].

Reference,
    dat. of, \xref{366}–\xref{369};
    gen. of, see Application, \xref{354}.

\latin{rēfert},
    cases w., \xref{345}.

Reflexive,
    passive used as, \xref[3]{288}.

Reflexive pronouns,
    \xref{135}, \xref[\emph{a}, \emph{b}]{260};
    use of \latin{sē} and \latin{suus}, \xref{262}, \xref{264};
    \latin{ipse} as reflex., \xref{263}, \xref[4]{264};
    \latin{inter sē}, etc., as reflex., \xref{266}.

\latin{reiciō},
    quantity of first syll., \xref[1]{30}.

Rejected reason,
    subj. cl. of, \xref[2, \emph{b}]{535}.

Relation, words of,
    w. dat., \xref{362}–\xref{364}.

Relative adverb,
    see Adverb.

Relative clause,
    is either declarative or conditional, \xref[3, \emph{b}]{228};
    = noun, partic., appos., etc., \xref[3]{284};
    position, \xref[10]{624}, \xref[5]{284};
    w. infin., in ind. disc., \xref[1, \emph{b}]{535};
    moods in, see \latin{quī}-cls.

Relative pronoun,
    decl., \xref{140};
    defined, \xref{281};
    meanings, \xref{282}, \xref{283};
    peculiarities in use, \xref{284};
    agreement of, \xref{322}–\xref{326}.

Relative tenses,
    of indic., \xref[1]{467}, \xref[\emph{b}]{477};
    of subj., \xref[3]{470}, \xref[\emph{b}]{477}.

\latin{relinquitur},
    w. vol. \latin{ut}-cl., \xref[3, \emph{c})]{502};
    w. \latin{ut}-cl. of fact, \xref[3, \emph{a})]{521}.

\latin{reliquī},
    meaning, \xref[1, \emph{b}]{279}.

“Remembering,”
    cases w., \xref{350};
    w infin., \xref{589};
    w. pres. infin. of past act, \xref[\emph{b}]{539}.

“Reminding,”
    cases w., \xref{351}.

\latin{reminīscor},
    cases w., \xref{350}.

\latin{Repeated action},
    subj. of, \xref{540};
    see also habitual action, \xref{484},
    and generalizing condition, \xref[\emph{a}]{576}.

“\latin{Repraesentātiō},”
    see Picturesque tenses.

“Representing,”
    verbs of, w. pres. partic., \xref[1]{605};
    w. infin., \ibid[n.]

Request,
    expr. by imper., \xref{496};
    by subj., \xref{530}.

\latin{requiēs},
    decl., \xref[2]{107}.

\latin{rēs},
    decl., \xref{99}, \xref[1]{100};
    see also \latin{ea rēs}.

“Resisting,”
    w. dat., \xref{362}.

Resolve,
    expr. by subj., \xref[1]{501};
    by pres. indic., \xref{571};
    by fut. indic., \xref{572}.

Respect,
    expr. by acc., \xref{388}, \xref{389};
    by abl., \xref{441};
    by supine in \ending{-ū}, \xref[1, 2]{619}.

Restrictive clauses,
    \xref{522}.

Result,
    abl. expressing, \xref[II]{422}.

Result,
    cls. of: obligatory or proper, \xref[4]{513};
    possible, \xref[3]{517};
    ideally certain, \xref[3]{519};
    actual, \xref[2]{521}.

Result produced,
    acc. of, \xref{394}.

\latin{reus},
    w. gen., \xref[\emph{a}]{342};
    \latin{vōtī reus}, \xref{343}.

Rhetoric,
    figures of, \xref{632}.

Rhetorical det. cl.,
    \xref[\emph{a}, n.~3]{550}.

Rhetorical questions,
    \xref{235};
    question of fact, in infin. in ind. disc., \xref[\emph{a}]{591}.

Rhotacism,
    \xref{47}, \xref[n.]{86}

Rhythm,
    \xref{633}.

Rivers,
    gend. of names of, \xref[1]{58}.

\latin{rogō},
    w. two accs., \xref{393};
    w. subj. cl., \xref[2]{530}.

Root,
    \xref[n.~2]{147}. \xref[\sftn{1}]{203}.

Route,
    abl. of, \xref{426}.

\latin{rūs},
    decl., \xref[1]{86};
    place relations w., \xref{93}, \xref[\emph{a}]{449},
        \xref[\emph{b}]{450}, \xref[\emph{a}]{451}.

\bigskip

\phone{s}-stems,
    \xref{85}, \xref{86}, \xref[\emph{a}]{116}.

\latin{sacer},
    comparison, \xref[\emph{a}]{123}.

\latin{saepe},
    comparison, \xref{129}.

\latin{salvē},
    defective, \xref[1]{200}.

\latin{sānē},
    in answers, \xref[1]{232}.

\latin{satin},
    \xref[2, \emph{d}]{236}.

\latin{satis},
    comparison, \xref{129};
    w. dat., \xref[I]{362}.

“Saying,”
    infin. w. verbs of, \xref{589};
    w. pass. of (\latin{dīcor}, etc.), \xref[1]{590};
    subj. w., \xref[3, \emph{a})]{502}.

Scanning,
    \xref{643}.

\latin{sciēns},
    w. force of adv., \xref{245}.

\latin{scītō},
    force of tense, \xref[\emph{c}]{496}.

\suffix{-scō},
    verbs in, \xref[\emph{F}]{168}, \xref[2]{212};
    length of vowel before, \xref{18}.

\latin{sē}, \latin{sēsē},
    see \latin{suī}.

\prefix{sē-}, \prefix{sed-}, prefix,
    \xref[1, \emph{b})]{218}.

Second conjugation,
    \xref{148}, \xref{156}, \xref{167}.

Second declension,
    \xref{69}–\xref{73}.

Second object,
    in acc., \xref{392}, \xref{393};
    in abl. w. \latin{ūtor}, etc., \xref[\emph{a}]{429}.

Second person sing. indef.,
    in generalizing, \xref[2]{504};
    in potential subj., \xref[1]{517};
    in generalizing statements of fact, \xref{542}.

Secondary derivatives,
    \xref{203};
    nouns, \xref{207};
    adj., \xref{209}, \xref{210};
    verbs, \xref{211}, \xref{212}.

“Secondary” tenses,
    \ftn{247}{2}.

\latin{sēcum},
    \xref[\emph{a}]{418}.

\latin{secundum},
    w. acc., \xref{380}.

\latin{sed},
    \xref[4 and \emph{b}]{310};
    \latin{sed enim}, \xref[6, \emph{b}]{311}.

“Seeing,”
    verbs of, w. pres. partic., \xref[1]{605};
    w. infin., \ibid[n.]

Semi-deponents,
    \xref{161};
    voice-meanings, \xref{291}.

Semihiatus,
    \xref{648}.

Semivowels,
    \xref{2}.

\latin{senex},
    decl., \xref[4]{88};
    compar., \xref{122}, \xref[\emph{b}]{123}.

Sentence,
    defined, \xref{220};
    how made up, \xref{221};
    simple, \xref[1]{223};
    compound, \ibid[2];
    coördinate, \ibid;
    complex, \ibid[3];
    four functions, \xref{228}.

Separation,
    abl. of, \xref{405}–\xref{411};
    gen. w.~verbs of, \xref{348}.

“Sequence of tenses,”
    reg., \xref{476};
    exceptions to, \xref{478}–\xref{480}.

“Service,”
    dat. of, see Tendency, \xref{360}.

“Serving,”
    dat. w., \xref[II]{362}.

\latin{sēstertius}, \latin{sēstertium},
    \xref{675}.

\latin{seu},
    see \latin{sīve}.

Shortening of vowels,
    \xref[1, 2]{20}, \xref{26}.

“Should,”
    see Obligation and Natural likelihood.

\latin{sī},
    meaning, \xref[1]{578};
    in ordinary conditions, \xref{579}–\xref{581};
    in loosely attached condition, \xref[2]{582};
    in virtual wish, \ibid[5];
    in ind. question of fact, \xref[2, \emph{b}]{582};
    adversative or concessive, \xref[7]{582}.

\latin{sī minus},
    \xref[3]{578};
    \latin{sī modō}, \xref[6]{582}.

\latin{sī nōn},
    \xref[2, 3]{578}.

\latin{sī quidem},
    = “for” or “since,” \xref[9]{582}.

Sibilants,
    \xref[3]{6}, \xref{12}.

\latin{sīc},
    in answers, \xref[1]{232};
    \latin{sīc ut}, \xref[2, \emph{a}, and ftn.]{521}

Símile,
    \xref[15]{632}.

\latin{similis},
    compar., \xref[2]{120};
    cases w., \xref[\emph{c}, n.]{339}

Simple sentence,
    \xref[1]{233}.

\latin{simul},
    poetic w. abl., \xref[\emph{b}]{418}.

\latin{simul}, \latin{simul atque (ac)},
    w. aor. indic., \xref{557}.

\latin{sīn},
    use, \xref[2, 3]{578};
    see also \latin{sī}.

\latin{sine},
    w. abl., \xref{405}.

Singular,
    nouns used only in, \xref{103};
    wanting, \xref{104};
    of different meaning in pl., \xref{105}.

\latin{sinō},
    w. subj., \xref[2]{531};
    w. infin., \xref{587}.

Situation,
    descr. \latin{cum}-cl. of, in subj., \xref{524};
    w. caus. or advers. idea, \xref{525};
    w. \latin{ubi}, etc., in indic., \xref{558};
    \latin{dum}-cl. of, \xref{559};
    expr. by partic., \xref[2]{604};
    by historical infin., \xref{595}.

Situation,
    tenses of, \xref[1, \emph{a}]{466}.

\latin{sīve} or \latin{seu},
    \xref[3 and \emph{a}]{308};
    correlative, \xref{309}.

Slurring,
    \xref[1]{34}, \xref{38}, \xref{646}.

\suffix{-sō},
    frequentatives in, \xref[1]{212}.

Softened statements, etc.,
    in subj., \xref[1, \emph{b}]{519}.

\latin{soleō},
    semi-deponent, \xref{161}.

\latin{sōlus},
    decl., \xref[\emph{a}]{112};
    gen. of, w. poss. pron., \xref[\emph{b}]{339};
    \latin{sōlus quī}, w. subj., \xref[1, \emph{a} and \sftn{}]{521}

Sonant, or voiced, consonants,
    \xref[1]{8}, \xref{12}.

Sounds,
    general statement, \xref{2}–\xref{8};
    classification of the Latin sounds, \xref{12}.

Space over which,
    \xref[\emph{c}]{426};
    see also Extent.

“Sparing,”
    dat. w., \xref[II]{362}.

“Specification,”
    gen. of, see Application;
    abl. and acc. of, see Respect.

Speech,
    parts of, \xref{221};
    figures of, \xref{631}, \xref{632}.

Spelling,
    variations in, \xref{52}.

Spirants,
    \xref[3]{6}.

Spondaic verse,
    \xref[\emph{b}]{639}.

Spondee,
    \xref{637}.

\latin{sponte},
    defect., \xref[1]{106};
    of manner, \xref[1]{445}.

Stage,
    tenses of the, \xref[1]{466}.

Standard,
    abl. of, \xref{415};
    w. \latin{ex}, \ibid[\emph{a}].

\latin{statuō},
    w. subj., \xref[3,\emph{a})]{502};
    infin., \xref{586}, \xref{589}.

Stem,
    of nouns, \xref[1,n.]{62}, \xref{63}, \xref{202}–\xref{205};
    of verbs, \xref{202};
    the three stems, \xref{147};
    pres. stem of the four conjs., \xref{148};
    union of stem and ends., \xref{152};
    form of tense-stem, mood-stem, etc., \xref{166}–\xref{184}.

\latin{stō}, \english{abide by},
    w. abl., \xref[1]{438}.

Stress,
    \xref{31}, \xref[1, 2]{33}.

\latin{studeō},
    w. dat., \xref[II]{362};
    w. infin., \xref{586}, \xref{587}.

\suffix{su-},
    pronunc. in \latin{suāvis}, \latin{suādeō}, \latin{suēscō},
        \xref{11}.

\latin{suādeō},
    w. dat., \xref[II]{362};
    w. acc., \xref[4]{364};
    w. vol. cl., \xref[3, \emph{a})]{502};
    w. infin., \xref{587}.

\latin{sub} (\latin{subs}),
    form in cpds., \xref[14]{51};
    w. acc., \xref[and \emph{b}]{381};
    w. abl., \xref{433}.

Subject,
    defined, \xref{229};
    omitted, \xref{285};
    indef., \xref{286};
    of finite verb, in nom., \xref{335};
    likewise of historical infin., \xref{595};
    verb agrees w., \xref{328};
    two or more w. one verb, \xref{329};
    subj. of infin. in acc., \xref{398};
    sometimes omitted, \xref{592};
    cl. as subject, \xref{238}.

Subjective genitive,
    \xref{344}.

Subjunctive,
    origin and mood-signs, \xref{175};
    tables of general forces, \xref{462};
    for details, see synopsis, \xref{499}.

Subordinate clause,
    defined, \xref[1, 2, \emph{a}]{224}.

Subsequent action,
    defined, \xref[3 and \sftn{2}]{470}.

Substantives cls.,
    defined, \xref{238};
    used as subject, obj., etc., \ibid, \xref[2,
      \emph{c}]{319}, \xref[1, \emph{a}), \emph{b})]{597};
    subj.:
        vol., \xref[3]{502};
        antic., \xref[2]{507};
        opt., \xref[2]{511};
        of obligation or propriety, \xref[5]{513};
        of natural likelihood, \xref[3]{515};
        potential, \xref[3]{517};
        of ideal certainty, \xref[4]{519};
        of actuality, \xref[3]{521};
        of request, \xref[2]{530};
        of consent or indifference, \xref[2]{531};
    ind. questions of fact, \xref[\emph{c}]{537};
    indic.:
        w. \latin{quod}, \xref[1]{552};
        w. \latin{cum}, \xref{553};
        infin., \xref{585}–\xref{594}.

Substantives,
    defined, \xref[\emph{a}]{221};
    adjs. and partics. used as, \xref{249}, \xref{250}.

\latin{subter},
    w. acc., \xref{382};
    w. abl., \ibid[\emph{a}].

Suffixes,
    primary and secondary, \xref{203}–\xref{210}.

Suggestion,
    in imper., \xref{496};
    in subj., \xref[2]{501}.

\latin{suī},
    decl., \xref{135};
    use, \xref{260}–\xref{264};
    w. gen. of gerundive, \xref{614};
    gen. of, reg. objective, \xref[\emph{a}]{254}.

\latin{sum},
    conj., \xref{153}, \xref{154};
    as copula, \xref[\emph{a}]{230};
    w. dat. of possession, \xref{374};
    \latin{est ut} w. subj., \xref[3, \emph{a})]{521};
    \latin{est} w.~infin., \xref[3]{598}.

\latin{summus}, \english{the top of},
    \xref{244}.

\latin{sunt quī}, moods after,
    \xref[1, \emph{b}]{521}.

\latin{super},
    form in cpds., \xref[1, \emph{a})]{218};
    w. acc., \xref{383};
    w. abl., \xref[and \emph{a}, \emph{b}]{435};
    as adv., \xref[\emph{c}]{303}.

\latin{superior},
    comparison, \xref{123}.

Superlative degree,
    \xref{119};
    in \ending{-errimus}, \ending{-illimus}, \xref[1, 2]{120};
    in \ending{-mus}, \latin{-timus}, etc., \xref[4]{120}, \xref{122},
        \xref{123};
    expr. by \latin{maximē}, \xref{121};
    wanting, \xref[\emph{b}]{123};
    of advs., \xref{128}, \xref{129};
    force of degree, \xref{241}, \xref{300};
    w.~\latin{quisque}, \xref[2, \emph{b})]{278};
    w.~\latin{vel} or \latin{ūnus}, \xref[3, \emph{a}]{241};
    w.~\latin{quam} or \latin{quam possum}, \ibid[4].

Supine, formation,
    \xref{181};
    in \ending{-um}, use, \xref{618};
    in \ending{-ū}, uses, \xref{619}.

\latin{suprā},
    w.~acc., \xref{380}.

Surd, or voiceless, consonants,
    \xref[2]{8}, \xref{12}.

Surprise,
    expr. by subj., \xref{503};
    by fut. indic., \xref{572};
    by infin., \xref{596}.

Suspense,
    in Latin sentence, \xref[III]{625}.

\latin{suus},
    \xref{136};
    uses of, \xref{260}–\xref{264};
    special meanings, \xref[3]{264};
    \latin{suus quisque}, \ibid[2, \emph{a}].

Syllables,
    \xref{13};
    division of, \xref{14}, \xref{15};
    open and closed, \xref[\emph{a}, \emph{b}]{14};
    quantity of, \xref{29}, \xref{30}.

Synaéresis, \ftn{352}{3}.

Synaloépha,
    \ftn{350}{1}.

Synapheía,
    \xref[n.~4]{641}.

Syncope of vowels,
    \xref{43};
    used in poetry, \xref{650}.

Synécdoche,
    \xref[8]{632}.

Sýnesis, \xref[6]{631}.

Sýnizesis,
    \ftn{352}{3}.

Syntax,
    defined, \xref{219}.

Syntax,
    figures of, \xref{631}.

Sýstole,
    \xref[3, \sftn{4}]{652}.

\bigskip

Tacit caus.-advers. cl.,
    \xref[\emph{a}]{569};
    explicit, \xref[\emph{a}]{523}.

\latin{taedet},
    w. gen., \xref[1]{352};
    w. acc., \xref[\emph{a}]{390}.

“Taking away,”
    dat. w. verbs of, \xref{371}.

\latin{tālis},
    \xref{143}, \xref{144};
    meaning, \xref{271};
    w. \latin{quī} and subj., \xref[1, \emph{b}]{521}.

\latin{tam}
    w. \latin{quī} or \latin{quīn},
    \xref[1, \emph{a}]{521};
    \latin{tam} w. \latin{ut} or \latin{quīn}, \xref[2,
      \emph{a}]{521}.

\latin{tamen},
    \xref[6]{310};
    position, \ibid

\latin{tametsī},
    corrective, \xref[7]{310};
    = \english{although}, \xref[8]{582}.

\latin{tamquam},
    w. abl. absolute, \xref[6, \emph{a}]{421};
    w. subj., \xref[3]{504}.

\latin{tandem}, \english{pray},
    in questions, \xref[2, n.]{231}

\latin{tantī},
    gen. of value, \xref[1]{356};
    \latin{tantī ut}, \xref[4]{513}.

\latin{tantum abest ut},
    \xref[3, \emph{a})]{521}.

\latin{tantus},
    pronom. adj., \xref{143}, \xref{144};
    meaning, \xref{271};
    w. subj. \latin{quī}- or \latin{ut}-cl., \xref[1, \emph{a}]{521};
    w.~\latin{quantus} and indic., \xref{550}.

\enclitic{-te},
    encl. particle, \xref[\emph{d}]{134}.

“Teaching,”
    two accs. w., \xref{393}.

\latin{tegō},
    conj., \xref{157}.

“Temporal” \latin{cum}-cl.,
    \xref{524}, \xref{525};
    \latin{ubi}-cl., etc., \xref{557}, \xref{558}.

\latin{tempus est},
    w. subj., \xref[3, \emph{c})]{502};
    w. infin., \xref{585}.

Tendency or purpose,
    dat. of, \xref{360}.

\latin{teneō},
    w. perf. pass. partic., \xref[5]{605}.

Tenses,
    \xref{145};
    stems of, \xref[n.~1]{147}, \xref{166}–\xref{175};
    tense defined, \xref{465};
    tenses of the stage, \xref[1]{466};
    aor. tenses, \ibid[2];
    rel. tenses, \xref[1]{467};
    absolute tenses, \ibid[2];
    tenses of indicative, \xref{468};
    of imperative, \xref{471}, \xref{496};
    of infinitive, \xref{472}, \xref{593};
    of subjunctive, \xref{469}, \xref{470};
    of the participles, \xref{473}, \xref{600}, \xref{601};
    special points:
        combinations (“sequence”) of tenses, \xref{476}, \xref{477};
        less usual combinations (“exceptions to the sequence”),
            \xref{478}, \xref{479};
        mechancial harmony of subj. tenses, \xref{480};
        tenses depending on pres. perf., \xref{481};
        permanent truths depending on past tenses, \xref{482};
        tenses of habitual (“repeated” or “customary”) action and
            attempted (“conative”) action, \xref{484};
        w. \latin{iam diū}, etc., \xref{485};
        of discovery, expr. by imperf., \xref[1]{486};
        by fut., \ibid[2];
        perf. of experience (“gnomic”), \xref{488};
        perf. of state of affairs no longer existing, \xref{489};
        energetic or emphatic perf., \xref{490};
        historical pres. and perf., \xref[1]{491};
        tenses of rapid succession of events, \xref{492};
        epistolary tenses, \xref{493};
        accuracy of use of Latin tenses%
            \versionA{, \xref{494}}%
            \versionB*{, \xref[\emph{a}]{577}}.

\latin{tenus},
    w. gen. or abl., \xref[4]{458};
    position, \ibid

\latin{terrā},
    \english{by land}, \xref[\emph{a}]{426};
    \english{on land}, \xref[\emph{a}]{449}.

Thematic vowel in verbs,
    \ftn{76}{}, ftn.;
    changes in, \xref[1]{152}.

Thesis,
    \xref[and \sftn{5}]{654}.

Things personified,
    as agents, \xref[1, \emph{b}]{406}.

“Thinking,”
    w. infin., \xref{589}.

Third conj.,
    \xref{148}, \xref{157}–\xref{158}, \xref{168}.

Third decl.,
    nouns, \xref{74}–\xref{95};
    adj., \xref{113}–\xref{118}.

Time,
    adjs. denoting, \xref[5]{290};
    time at or within wh., expr. by abl., \xref{439};
    by abl. absolute, \xref[1]{421};
    duration of time, by acc., \xref[II]{387};
    by abl., \xref{440};
    absolute and rel. time, \xref[1, 2]{467}.

Time in Roman reckoning,
    \xref{660}–\xref{671}.

\latin{timeō},
    w. dat. or acc., \xref{367};
    w. subj., \xref[4]{502};
    w. infin., \xref{586}.

Tmesis,
    \xref{659}.

\suffix{-tō} (\suffix{-āre}),
    frequentatives in, \xref[1]{212}.

\latin{tot},
    meaning and uses, \xref[2, \emph{a}]{271}.

\latin{tōtus},
    \xref{112};
    w. abl. noun, \xref[and \emph{a}]{436}.

Towns where, whither, or whence,
    \xref{449}–\xref{451};
    appositives w., \xref{452};
    preps. w., \xref{453}.

\latin{trāns},
    form in cpds., \xref[16]{51};
    w. acc., \xref{380};
    cpds. of, w. acc., \xref{386}.

Transitive,
    see Voice and Verb.

Trees,
    gend. of names, \xref[2]{58}.

\latin{trēs},
    decl., \xref[2]{131}.

Trochee,
    \xref[\emph{b}]{637}.

“Trusting,”
    w. dat., \xref[II]{362}.

\latin{tū},
    decl., \xref{134};
    \latin{tūte}, \xref[\emph{d}]{134}.

\latin{tuī},
    gen. of \latin{tū}, reg. objective, \xref[\emph{a}]{254}.

\latin{tum\ellipsis cum},
    see \latin{cum}.

\ending{-tum}, \ending{-tū},
    supine endings, \xref{181}, \xref[4, 5]{49}.

\ending{-tus},
    perf. pass. partic. in, \xref[4]{179};
    adj. in, \xref[3]{209};
    adv. in, \xref[8]{126}.

Two comparatives,
    adjs., \xref{242};
    advs., \xref{301}.

“Two datives,”
    \xref[\emph{b}]{360}.

Two negatives,
    \xref[2]{298}.

Two objects in acc.,
    \xref{392}, \xref{393};
    one retained w. pass., \xref[\emph{a}]{393};
    two objs. in abl., \xref[\emph{a}]{429}.

\bigskip

\phone{u},
    consonantal may become vowel in poetry, \xref[1]{656};
    vocalic may become consonantal in poetry, \ibid[2].

\latin{ubi}, \english{when}:
    for ordinary uses, see \latin{postquam};
    in cl. of equivalent action, \xref{551}.

\latin{ubi}, \english{where},
    same constrs. as \latin{quī}; see \latin{quī}.

\ending{-ubus},
    dat.-abl. pl., decl.~IV, \xref[1]{97}.

\ending{-uī},
    perfects in, \xref[\emph{B}]{173}.

\latin{ūllus},
    decl., \xref[\emph{a}]{112}, \xref{143};
    use, \xref[7]{276}.

\latin{ulterior},
    comparison, \xref{123}.

\latin{ultimus}, \english{the last to},
    \xref{243}.

\latin{ultrā} (\latin{uls}),
    w. acc., \xref{380}.

\ending{-um},
    gen. plur., in decl.~I, \xref[3]{66};
    decl.~II, \xref[4]{71};
    decl.~III, in \phone{i}-stems, \xref[4]{88};
    decl.~IV, \xref[3]{97};
    in adjs., \xref[1]{118};
    of \latin{ducentī}, etc., \xref[4]{131}.

\latin{unde}, \english{whence},
    same constrs. as \latin{quī};
    see \latin{quī}.

Unthematic verbs,
    \ftn{76}{}, ftn., \xref{170}.

\latin{ūnus},
    decl., \xref[\emph{a}]{112};
    gen. of, w. poss. pron., \xref[\emph{b}]{339};
    \latin{ūnus dē} or \latin{ex}, \xref[\emph{e}]{346};
    \latin{ūnus quī}, w. subj., \xref[1, \emph{a}]{521};
    \latin{unī}, use, \xref[1, \emph{b}]{247}.

\latin{ūnusquisque},
    \xref[7]{142}.

\latin{urbs},
    decl., \xref{90};
    in apposition to names of towns, \xref{452}.

\latin{ūsus est},
    constrs. w., \xref[1, 2, and \emph{c}]{430}.

\latin{ut} (\latin{utī}),
    often merely formal, \ftn{261}{2};
    w.~subj., see especially in cl. of purpose, \xref[2]{502};
    in vol. subst. cl., \ibid[3];
    in cl. of fear, \ibid[4];
    in question or exclamation of surprise, etc., \xref{503};
    in antic. subst. cl.\emend{67}{}{,} \xref[2]{507};
    in opt. subst. cl., \xref[2]{511};
    in cl. of actual result, \xref[2]{521};
    in subst. cl. of actuality, \ibid[3];
    in ind. questions or exclamations, \xref[\emph{d}, 3) and
      \sftn{}]{537};
    w. indic., see synopsis, \xref{543}.

\latin{ut nē},
    \ftn{261}{2}.

\latin{ut prīmum}, \latin{ut semel},
    \xref[\emph{a}]{557}.

\latin{ut sī},
    \english{as if}, w. subj., \xref[3]{504}.

\latin{uter},
    rel., \xref[\emph{d}]{140};
    interrog., \xref[\emph{b}]{141};
    use, \xref[1]{275};
    decl., \xref{112};
    cpds. of, \xref[\emph{a}]{142}.

\latin{utercumque},
    decl., \xref[\emph{d}]{140}.

\latin{uterlibet},
    meaning, \xref[\emph{a}]{142};
    \xref[8]{276}.

\latin{uterque},
    \xref[\emph{a}]{142};
    use, \xref{278};
    meaning in pl., \ibid[\emph{a}, \emph{b}];
    as recipr. pron., \xref{265};
    in agreement or w. gen.\emend{68}{}{,} \xref[\emph{b}]{346}.

\latin{utervīs},
    meaning, \xref[\emph{a}]{142}, \xref[8]{276}.

\latin{utī},
    see \latin{ut}.

\latin{utinam},
    in wishes, \xref[1, and \sftn{1}]{511}.

\latin{ūtor},
    w. abl., \xref{429};
    w. 2d abl., \ibid[\emph{a}];
    w. acc., \ibid[\emph{b}];
    in gerundive constr., \xref[2, n.]{613}

\latin{utpote},
    w. \latin{quī}-cl., \xref[\emph{b}]{523};
    w. \latin{cum}-cl., \xref[\emph{a}]{526}.

\latin{utrum\ellipsis an}, etc.,
    \xref[I and \emph{b}]{234};
    \latin{utrum} suppressed, \ibid[III].

\bigskip

\phone{v},
    may become \phone{u} in poetry, \xref[1]{656}.

Value,
    expr. by gen., \xref{356};
    by abl., \xref{427}.

Variable nouns,
    \xref{107}, \xref{108}.

\enclitic{-ve},
    encl. particle, \xref[n.]{32}, \xref{33};
    see also \latin{vel}.

\prefix{vē-}, inseparable prefix,
    \xref[2]{214}.

\latin{vel} or \enclitic{-ve},
    disjunctive, \xref[2]{308};
    correction, \ibid[3, \emph{a}];
    \latin{vel} w. superlative, \xref[3, \emph{a}]{241}.

\latin{velim} or \latin{vellem},
    in softened statements, \xref[1, \emph{b}]{519};
    in virtual wishes, \ibid[\emph{c}].

\latin{velut} or \latin{velutī} (\latin{sī}),
    w. subj., \xref[3]{504}.

\latin{ventūrus},
    as adj., \xref{248}.

Verbal nouns,
    \xref{146}.

Verbs,
    \emph{Form}:
        voices, moods, etc., \xref{145};
        three stems, \xref{147};
        the conjs., \xref{148};
        principal parts, \xref{150};
        ends., \xref{151};
        union of end. with stem, \xref{152};
        examples of inflection of the four reg. conjs.,
            \xref{155}–\xref{159};
        depons., \xref{160};
        semi-depons., \xref{161};
        periphr. conj., \xref{162};
        peculiarities in conj., \xref{163}–\xref{165};
        formation of the stems, \xref{166}–\xref{184};
        illustrations of the various types (principal parts),
            \xref{185}–\xref{189};
        conj. of the irreg. verbs, \xref{190}–\xref{197};
        defect. verbs, \xref{198}–\xref{200};
        impers. verbs, \xref{201};
        derivation of verbs, \xref{211}–\xref{212};
        composition of verbs, \xref{218};
        list of, p.~\pageref{verbcat}.
    \enskip
    \emph{Syntax}:
        verb defined, \xref{221};
        sometimes omitted, \xref[\emph{a}]{222};
        impers. verbs, \xref{287};
        trans. verbs, \xref{289};
        intrans. verbs, \xref{290}.

\latin{vereor},
    \xref{160};
    w. subj., \xref[4]{502};
    w. infin., \xref{586}.

\latin{vērō},
    \xref[5, \emph{a}, \emph{b}]{310};
    in answers, \xref[1]{232}.

Verse,
    defined, \xref{636}.

Versification,
    \xref{633}–\xref{659}.

\latin{versus},
    w. acc., \xref{380};
    position of, \ibid[\emph{a}].

\latin{vērum},
    \xref[4]{310};
    following \latin{nōn}, \ibid[\emph{b}];
    w. \latin{etiam} or \latin{quoque} added, \ibid

\latin{vēscor},
    w. abl., \xref{429};
    w. acc., \ibid[\emph{b}];
    in gerundive constr., \xref[2, n.]{613}

\latin{vestrī},
    objective gen., \xref[\emph{b}]{134}, \xref[\emph{a}]{254};
    \latin{vestrum}, gen. of the whole, \ibid

\latin{vetō},
    moods w., \xref[\emph{b}]{587};
    \latin{vetor} w. infin., \xref{588}.

\latin{vetus},
    decl., \xref{117}, \xref[1, \emph{a}, 2]{118};
    stem of, \xref[\emph{b}]{117};
    comparison, \xref[1]{120}.

\ending{-vī},
    perfects in, \xref[\emph{A}]{173};
    short forms, \xref{163}.

\latin{vicis},
    defect., \xref[4]{106};
    \latin{vicem}, \xref{388}.

\latin{videō},
    w.~vol. cl., \xref[3, \emph{a})]{502};
    \latin{vidē nē} in prohibitions, \xref[3, \emph{a}, 2)]{501};
    \latin{videō} and \latin{videor} w. infin., \xref{589},
        \xref[1]{590};
    \latin{videō} w. pres. partic. or infin., \xref[1, n.]{605};
    \latin{videor}, w. dat., \xref{370}.

\latin{vir},
    decl., \xref{70};
    poetic gen. \latin{virum}, \xref[4, \emph{c}]{71}.

Virtual wishes,
    expr. by \latin{velim}, etc., w. subj., \xref[1, \emph{c}]{519};
    by condition w. \latin{sī} or \latin{ō sī}, w. subj.,
    \xref[5]{582}.

\latin{vīs},
    decl., \xref{92}.

Vocative,
    \emph{Forms},
        exeptional:
            in \ending{-ī}, of nouns in \ending{-ius}, \xref[2]{71};
            in \ending{-ie}, of adjs. in \ending{-ius},
                \xref[\emph{a}]{110}.
    \enskip
    \emph{Syntax}\emend{69}{,}{:}
        \xref{400};
        position of, \xref[6]{624};
        nom. for, \xref{401}.

Voice,
    \xref{145};
    act., \xref[1]{288};
    pass., \ibid[2];
    pass. used reflexively (“middle voice”), \ibid[3];
    w. acc., \xref[\emph{b}]{390};
    intrans. verbs, pass. of, used impers., \xref[\emph{a}, 1)]{290};
    voice-meanings of depons. and semi-depons., \xref{291}.

Voiced consonants,
    \xref[1]{8}, \xref{12};
    change of voiced mutes, \xref[1]{49}.

Voiceless consonants,
    \xref[2]{8}, \xref{12}.

Volitive subjunctive,
    \xref{500}–\xref{505}.

\latin{volō} and its cpds.,
    \xref{192};
    w. vol. cl., \xref[3, \emph{a})]{502};
    w. infin., \xref{586}, \xref{587};
    w. perf. partic., \xref[3]{605}.

Vowels,
    \xref{2};
    classification, \xref{3};
    pronunc., \xref{9};
    quantity, \xref{16}–\xref{28}, \xref{36};
    weakening in interior syll., \xref{41}, \xref{42};
    syncope, \xref{43};
    changes in final syll., \xref{44};
    contraction, \xref{45};
    vowel-gradation, \xref{46};
    thematic vowel, \ftn{76}{}, ftn., \xref[1]{152};
    final, slurred, \xref[1]{34}, \xref{646}.

\bigskip

“Want,”
    w. gen., \xref{347};
    w. abl., \xref{425}.

Watches of the night,
    \xref[2]{670}.

Way or manner,
    expr. by abl., \xref{445};
    by \latin{ad}, \latin{in}, or \latin{per}, w. acc., \ibid[3, \emph{a}];
    by \latin{dum}-cl., \xref[\emph{a}]{559};
    by partic., \xref[5]{604}.

Weights and Measures,
    \xref{672}–\xref{677}.

Whole,
    idea of, expr. by gen., \xref{346};
    by \latin{dē} or \latin{ex}, w. abl., \ibid[\emph{e}].

Will,
    expr. by vol. subj., \xref{500}.

Winds,
    gend. of names of, \xref[1]{58}.

Wish,
    expr. by opt. subj., \xref[1]{511};
    by \latin{velim}, etc., w. subj., \xref[1, \emph{c}]{519};
    by \latin{sī} or \latin{ō sī} w. subj., \xref[5]{582}.

“Wishing,”
    w. opt. subj., \xref[2]{511};
    w. infin., \xref[and \emph{d}]{586};
    see also \latin{volō}.

Women,
    names of, \xref[5]{678}.

Word-accent,
    in verse, \xref{645}.

Word-formation,
    \xref{202}–\xref{218}.

Word-order:
    normal, \xref{623}, \xref{624};
    rhetorical, \xref{625}–\xref{630};
    parallel order, \xref{628};
    cross order, \ibid

\bigskip

\grapheme{y},
    in borrowed words only, \xref[\emph{a}]{1}.

Year,
    how indicated, \xref{661}.

“Yes,”
    how expressed, \xref[1]{232}.

\bigskip

\grapheme{z},
    in borrowed words only, \xref[\emph{a}]{1};
    pronunc., \xref{11};
    in poetry, \xref[3, \emph{b}]{29}.

Zeúgma,
    \xref[7]{631}.

\end{algindex}

\addtocontents{toc}{\bigskip}

%% This work is licensed under a Creative Commons
%% Attribution-NonCommercial 4.0 International License.
%% http://creativecommons.org/licenses/by-nc/4.0/
%% 
%% David M. Jones, July 2016

\Unnumbered{Index of Abbreviations}
\label{abbreviations}
\markboth{Index of Abbreviations}{Index of Abbreviations}
\markthird{}

\thispagestyle{dropfolio}

\contentsentry{B}{Index of Abbreviations}

% The following abbreviations are used in the text.

\begin{abbreviations}

\bibitem[A. P.]{A. P.}
Horace.
\perseus{phi0893.phi006}{\emph{Ars Poētica}}.

\bibitem[Ac.]{Ac.}
Cicero.
\perseus{phi0474.phi045}{\emph{Academica}}.

\bibitem[Ad.]{Ad.}
Terence.
\perseus{phi0134.phi006}{\emph{Adelphī}}.

\bibitem[Aen.]{Aen.}
Virgil.
\perseus{phi0690.phi003}{\emph{Aenēis}}.

\bibitem[Am.]{Am.}
Cicero.
\perseus{phi0474.phi052}{\emph{Laelius dē Amīcitiā}}.

\bibitem[Amph.]{Amph.}
Plautus.
\perseus{phi0119.phi001}{\emph{Amphitruō}}.

\bibitem[And.]{And.}
Terence.
\perseus{phi0134.phi001}{\emph{Andria}}.

\bibitem[Arch.]{Arch.}
Cicero.
\perseus{phi0474.phi016}{\emph{Prō Archiā Poētā}}.

\bibitem[As.]{As.}
Plautus.
\perseus{phi0119.phi002}{\emph{Asināria}}.

\bibitem[Att.]{Att.}
Cicero.
\perseus{phi0474.phi057}{\emph{Epistolae ad Atticum}}.

\bibitem[Aul.]{Aul.}
Plautus.
\perseus{phi0119.phi003}{\emph{Aululāria}}.

\bibitem[B. C.]{B. C.}
Caesar.
\perseus{phi0448.phi002}{\emph{Dē Bellō Cīvīlī}}.

\bibitem[B. G.]{B. G.}
Caesar.
\perseus{phi0448.phi001}{\emph{Dē Bellō Gallicō}}.

\bibitem[Bacch.]{Bacch.}
Plautus.
\perseus{phi0119.phi004}{\emph{Bacchides}}.

\bibitem[Balb.]{Balb.}
Cicero.
\perseus{phi0474.phi026}{\emph{Prō Balbō}}.

\bibitem[Brut.]{Brut.}
Cicero.
\perseus{phi0474.phi039}{\emph{Brutus}}.

\bibitem[Caecil.]{Caecil.}
Cicero.
\perseus{phi0474.phi004}{\emph{In Q. Caecilium}}.
% \emph{Divinatio against Q. Caecilius}.

\bibitem[Caecin. ap. Fam.]{Caecin. ap. Fam.}
Cicero.
\perseus{phi0474.phi056}{``Letter to Caecina''},
in \emph{Epistulae ad Familares}.

\bibitem[Caecin.]{Caecin.}
Cicero.
\perseus{phi0474.phi008}{\emph{Prō Caecinā}}. %% check

\bibitem[Cael.]{Cael.}
Cicero.
\perseus{phi0474.phi024}{\emph{Prō Caeliō}}.

% check title of letter
\bibitem[Cael., Fam.]{Cael., Fam.}
Cicero.
\perseus{phi0474.phi056}{``M. Caelī Epistulae ad M. Tullium Ciceronem''},
in \emph{Epistulae ad Familares}.

\bibitem[Capt.]{Capt.}
Plautus.
\perseus{phi0119.phi005}{\emph{Captīvī}}.

\bibitem[Carm.]{Carm.}
Horace.
\perseus{phi0893.phi001}{\emph{Carmina}}.

\bibitem[Cat.]{Cat.}
Cicero.
\perseus{phi0474.phi013}{\emph{In Catilinam}}.

\bibitem[Cato Agr.]{Cato Agr.}
Cato the Elder.
\href{http://penelope.uchicago.edu/Thayer/E/Roman/Texts/Cato/De_Agricultura/}
{\emph{Dē Agrī Cultūrā}}.

\bibitem[Catull.]{Catull.}
Catullus.
\perseus{phi0472.phi001}{\emph{Carmina}}.

\bibitem[Clu.]{Clu.}
Cicero.
\perseus{phi0474.phi010}{\emph{Prō Cluentiā}}.

\bibitem[De Or.]{De Or.}
Cicero.
\perseus{phi0474.phi037}{\emph{Dē Oratore}}.

\bibitem[Dei.]{Dei.}
Cicero.
\perseus{phi0474.phi034}{\emph{Prō Rege Deiotarō}}.

\bibitem[Div.]{Div.}
Cicero.
\perseus{phi0474.phi053}{\emph{Dē Divinatione}}.

\bibitem[Dom.]{Dom.}
Cicero.
\perseus{phi0474.phi020}{\emph{Dē Domō suā}}.

\bibitem[Ecl.]{Ecl.}
Virgil.
\perseus{phi0690.phi001}{\emph{Eclogae}}.

\bibitem[Enn. Ann.]{Enn. Ann.}
Ennius.
\href{http://www.thelatinlibrary.com/enn.html}{\emph{Annales}}.

\bibitem[Ep.]{Ep.}
Horace.
\perseus{phi0893.phi005}{\emph{Epistolae}}.

\bibitem[Epod.]{Epod.}
Horace.
\perseus{phi0893.phi003}{\emph{Epodī}}.

\bibitem[Eun.]{Eun.}
Terence.
\perseus{phi0134.phi003}{\emph{Eunūchus}}.

\bibitem[Fam.]{Fam.}
Cicero.
\perseus{phi0474.phi056}{\emph{Epistolae ad Familiārēs}}.

\bibitem[Fin.]{Fin.}
Cicero.
\perseus{phi0474.phi048}{\emph{Dē Finibus Bonōrum et Malōrum}}.

\bibitem[Flacc.]{Flacc.}
Cicero.
\perseus{phi0474.phi017}{\emph{Prō Flaccō}}.

\bibitem[Font.]{Font.}
Cicero.
\perseus{phi0474.phi007}{\emph{Prō Fonteiō}}.

\bibitem[Georg.]{Georg.}
Virgil.
\perseus{phi0690.phi002}{\emph{Geōrgica}}.

\bibitem[Har. Resp.]{Har. Resp.}
Cicero.
\perseus{phi0474.phi021}{\emph{Dē Haruspicum Responsō}}.

\bibitem[Heaut.]{Heaut.}
Terence.
\perseus{phi0134.phi002}{\emph{Heauton Timorumenos}}.

\bibitem[Hec.]{Hec.}
Terence.
\perseus{phi0134.phi005}{\emph{Hecyra}}.

\bibitem[Inv.]{Inv.}
Cicero.
\perseus{phi0474.phi036}{\emph{Dē Inventione}}.

\bibitem[Iuv.]{Iuv.}
Cicero.
\emph{Dē Iure Civilī in artem redigendō}.

\bibitem[Leg. Agr.]{Leg. Agr.}
Cicero.
\perseus{phi0474.phi011}{\emph{Dē Lege Agrariā}}.

\bibitem[Leg.]{Leg.}
Cicero.
\perseus{phi0474.phi044}{\emph{Dē Legibus}}.

\bibitem[Lig.]{Lig.}
Cicero.
\perseus{phi0474.phi033}{\emph{Prō Ligariō}}.

\bibitem[Liv.]{Liv.}
Livy.
\href{http://www.thelatinlibrary.com/liv.html}{\emph{Ab Urbe Conditā}}.

\bibitem[Marc.]{Marc.}
Cicero.
\perseus{phi0474.phi032}{\emph{Prō Marcellō}}.

\bibitem[Mart.]{Mart.}
Martial.
\perseus{phi1294.phi002}{\emph{Epigrammata}}.

\bibitem[Men.]{Men.}
Plautus.
\perseus{phi0119.phi010}{\emph{Menaechmī}}.

\bibitem[Merc.]{Merc.}
Plautus.
\perseus{phi0119.phi011}{\emph{Mercātor}}.

\bibitem[Met.]{Met.}
Cicero.
\emph{Contra Contionem Q. Metellī}.

\bibitem[Mil. Gl.]{Mil. Gl.}
Plautus.
\perseus{phi0119.phi012}{\emph{Mīles Glōriōsus}}.

\bibitem[Mil.]{Mil.}
Cicero.
\perseus{phi0474.phi031}{\emph{Prō Milone}}.

\bibitem[Mur.]{Mur.}
Cicero.
\perseus{phi0474.phi014}{\emph{Prō Murena}}.

\bibitem[N. D.]{N. D.}
Cicero.
\perseus{phi0474.phi050}{\emph{Dē Naturā Deōrum}}.

\bibitem[Nep. Ages.]{Nep. Ages.}
Nepos.
\perseus{phi0588.phi017}{\emph{Agesilaus}}.

\bibitem[Nep. Att.]{Nep. Att.}
Nepos.
\perseus{phi0588.phi025}{\emph{Atticus}}.

\bibitem[Nep. Eum.]{Nep. Eum.}
Nepos.
\perseus{phi0588.phi018}{\emph{Eumenes}}.

\bibitem[Nep. Hann.]{Nep. Hann.}
Nepos.
\perseus{phi0588.phi023}{\emph{Hannibal}}.

\bibitem[Nep. Paus.]{Nep. Paus.}
Nepos.
\perseus{phi0588.phi004}{\emph{Pausanias}}.

\bibitem[Nep. Them.]{Nep. Them.}
Nepos.
\perseus{phi0588.phi00}{\emph{Themistocles}}.

\bibitem[Nep. Thras.]{Nep. Thras.}
Nepos.
\perseus{phi0588.phi008}{\emph{Thrasybulus}}.

\bibitem[Nep. Timol.]{Nep. Timol.}
Nepos.
\perseus{phi0588.phi020}{\emph{Timoleon}}.

\bibitem[Off.]{Off.}
Cicero.
\perseus{phi0474.phi055}{\emph{Dē Officiīs}}.

\bibitem[Or.]{Or.}
Cicero.
\emph{Orator}. %???

\bibitem[Ov. A. A.]{Ov. A. A.}
Ovid.
\perseus{phi0959.phi004}{\emph{Ars Amatoria}}.

\bibitem[Ov. Her.]{Ov. Her.}
Ovid.
\perseus{phi0959.phi002}{\emph{Epistulae (Heroides)}}.

\bibitem[Ov. Met.]{Ov. Met.}
Ovid.
\perseus{phi0959.phi006}{\emph{Metamorphoses}}.

\bibitem[Ov. Pont.]{Ov. Pont.}
Ovid.
\perseus{phi0959.phi009}{\emph{Epistulae ex Pontō}}.

\bibitem[Ov. Trist.]{Ov. Trist.}
Ovid.
\perseus{phi0959.phi008}{\emph{Tristia}}.

\bibitem[Par.]{Par.}
Cicero.
\perseus{phi0474.phi047}{\emph{Paradoxa Stoicorum ad M. Brutum}}.

\bibitem[Pers.]{Pers.}
Plautus.
\perseus{phi0119.phi014}{\emph{Persa}}.

\bibitem[Persius]{Persius}
Persius.
\perseus{phi0969.phi001}{\emph{Saturae}}.

\bibitem[Ph.]{Ph.}
Terence.
\perseus{phi0134.phi004}{\emph{Phormiō}}.

\bibitem[Phil.]{Phil.}
Cicero.
\perseus{phi0474.phi035}{\emph{Philippicae}}.

\bibitem[Pis.]{Pis.}
Cicero.
\perseus{phi0474.phi027}{\emph{In Pisonem}}.

\bibitem[Planc.]{Planc.}
Cicero.
\perseus{phi0474.phi028}{\emph{Prō Planciō}}.

\bibitem[Plin. Ep.]{Plin. Ep.}
Pliny the Younger.
\perseus{phi1318.phi001}{\emph{Epistulae}}.

\bibitem[Plin. N. H.]{Plin. N. H.}
Pliny the Elder.
\perseus{phi0978.phi001}{\emph{Naturalis Historia}}.

\bibitem[Poen.]{Poen.}
Plautus.
\perseus{phi0119.phi015}{\emph{Poenulus}}.

\bibitem[Pomp.]{Pomp.}
Cicero.
\perseus{phi0474.phi009}{\emph{Dē Imperiō Gnaei Pompei}}. %% need latin title
% \perseus{phi0474.phi009}{\emph{On Pompey's Command}}.

\bibitem[Prov. Cons.]{Prov. Cons.}
Cicero.
\perseus{phi0474.phi025}{\emph{Prō Provinciīs Consularibus}}.

\bibitem[Pseud.]{Pseud.}
Plautus.
\perseus{phi0119.phi016}{\emph{Pseudolus}}.

\bibitem[Q. Fr.]{Q. Fr.}
Cicero.
\perseus{phi0474.phi058}{\emph{Epistulae ad Quintum fratrem}}.

\bibitem[Quinct.]{Quinct.}
Cicero.
\perseus{phi0474.phi001}{\emph{Prō Quinctiō}}.

\bibitem[Quintil.]{Quintil.}
Quintilian.
\emph{Institutio Oratoria},
\perseus{phi1002.phi0011}{Book~1},
\perseus{phi1002.phi0012}{Book~2},
\perseus{phi1002.phi0014}{Book~4},
\perseus{phi1002.phi0016}{Book~6},
\perseus{phi1002.phi0017}{Book~7},
\perseus{phi1002.phi0018}{Book~8},
\perseus{phi1002.phi00110}{Book~10},
\perseus{phi1002.phi00111}{Book~11}.

\bibitem[Rab. Perd.]{Rab. Perd.}
Cicero.
\perseus{phi0474.phi012}{\emph{Prō Rabiriō Perduellionis Reō}}.

\bibitem[Rab. Post.]{Rab. Post.}
Cicero.
\perseus{phi0474.phi030}{\emph{Prō Rabiriō Postumō}}.

\bibitem[Rep.]{Rep.}
Cicero.
\perseus{phi0474.phi043}{\emph{Dē Republicā}}.

\bibitem[Rosc. Am.]{Rosc. Am.}
Cicero.
\perseus{phi0474.phi002}{\emph{Prō S. Rosciō Amerinō}}.

\bibitem[Rosc. Com.]{Rosc. Com.}
Cicero.
\perseus{phi0474.phi003}{\emph{Prō Q. Rosciō comoedō}}.

\bibitem[Rud.]{Rud.}
Plautus.
\perseus{phi0119.phi017}{\emph{Rudēns}}.

\bibitem[Sall. Cat.]{Sall. Cat.}
Sallust.
\perseus{phi0631.phi001}{\emph{Catilinae Coniuratio}}.

\bibitem[Sall. Iug.]{Sall. Iug.}
Sallust.
\perseus{phi0631.phi002}{\emph{Bellum Iugurthinum}}.

\bibitem[Sat.]{Sat.}
Horace.
\perseus{phi0893.phi004}{\emph{Sermōnes}} (\emph{Satyres}).

\bibitem[Sen. Med.]{Sen. Med.}
Seneca.
\perseus{phi1017.phi004}{\emph{Medea}}.

\bibitem[Sen.]{Sen.}
Cicero.
\perseus{phi0474.phi051}{\emph{Dē Senectūte}}.

\bibitem[Senat.]{Senat.}
Cicero.
\perseus{phi0474.phi019}{\emph{Ōrātiō post Reditum in Senātū Habita}}.

\bibitem[Sest.]{Sest.}
Cicero.
\perseus{phi0474.phi022}{\emph{Prō Sestiō}}.

\bibitem[Stich.]{Stich.}
Plautus.
\perseus{phi0119.phi018}{\emph{Stichus}}.

\bibitem[Sull.]{Sull.}
Cicero.
\perseus{phi0474.phi015}{\emph{Prō Sullā}}.

\bibitem[Tac. Agric.]{Tac. Agric.}
Tacitus.
\perseus{phi1351.phi001}{\emph{Agricola}}.

\bibitem[Tac. Ann.]{Tac. Ann.}
Tacitus.
\perseus{phi1351.phi005}{\emph{Annales}}.

\bibitem[Tac. Hist.]{Tac. Hist.}
Tacitus.
\perseus{phi1351.phi004}{\emph{Historiae}}.

\bibitem[Trin.]{Trin.}
Plautus.
\perseus{phi0119.phi019}{\emph{Trinummus}}.

\bibitem[Tull.]{Tull.}
Cicero.
\perseus{phi0474.phi006}{\emph{Prō Tulliō}}.

\bibitem[Tusc.]{Tusc.}
Cicero.
\perseus{phi0474.phi049}{\emph{Tusculanae Disputationēs}}.

\bibitem[Varro, Sat. Men.]{Varro, Sat. Men.}
Varro.
\emph{Menippeae}.

\bibitem[Vat.]{Vat.}
Cicero.
\perseus{phi0474.phi023}{\emph{In Vatinium}}.

\bibitem[Verr.]{Verr.}
Cicero.
\perseus{phi0474.phi005}{\emph{In Verrem}}.

\end{abbreviations}

\endinput


%% This work is licensed under a Creative Commons
%% Attribution-NonCommercial 4.0 International License.
%% http://creativecommons.org/licenses/by-nc/4.0/
%% 
%% David M. Jones, July 2016

\Unnumbered{Index of Passages Cited}
\label{passages}
\markboth{Index of Passages Cited}{Index of Passages Cited}
\markthird{}

\contentsentry{B}{Index of Passages Cited}

\begin{autindex}

%\def\subsubitem{}

\indexauthor{Caesar},
  \subitem \emph{dē Bellō Cīvīlī} (B. C.),
    \subsubitem 1,   4,  5, \xref{587};
    \subsubitem 1,  11,  4, \xref{453};
    \subsubitem 1,  16,  4, \xref{380};
    \subsubitem 1,  18,  3, \xref{408};
    \subsubitem 1,  23,  2, \xref{279};
    \subsubitem 1,  30,  5, \xref{364};
    \subsubitem 1,  70,  2, \xref{519};
    \subsubitem 1,  73,  2, \xref{509};
    \subsubitem 1,  74,  7, \xref{425};
    \subsubitem 1,  80,  4, \xref{391};
    \subsubitem 2,   7,  3, \xref{288};
    \subsubitem 2,  11,  2, \xref{557};
    \subsubitem 2,  15,  1, \xref{517};
    \subsubitem 2,  18,  2, \xref{454};
    \subsubitem 2,  32, 14, \xref{390};
    \subsubitem 2,  35,  2, \xref{502};
    \subsubitem 3,   1,  1, \xref{392};
    \subsubitem 3,  19,  5, \xref{384};
    \subsubitem 3,  29,  1, \xref{408};
    \subsubitem 3,  30,  3, \xref{331};
    \subsubitem 3,  32,  4, \xref{425};
    \subsubitem 3,  41,  3, \xref{566};
    \subsubitem 3,  42,  5, \xref{482};
    \subsubitem 3,  61,  1, \xref{386};
    \subsubitem 3,  80,  1, \xref{370};
    \subsubitem 3,  86,  4, \xref{476};
    \subsubitem 3,  89,  4–5, \xref{438};
    \subsubitem 3, 106,  1, \xref{449}.

  \subitem \emph{dē Bellō Gallicō} (B. G.),
    \subsubitem 1,  1,  1, \xref{320}, \xref{436};
    \subsubitem 1,  1,  2, \xref{346}, \xref{405};
    \subsubitem 1,  1,  3, \xref{230}, \xref{307};
    \subsubitem 1,  1,  4, \xref{261}, \xref{362}, \xref{419}, \xref{476}, \xref{624};
    \subsubitem 1,  1,  5, \xref{346};
    \subsubitem 1,  2,  1, \xref{305}, \xref{331}, \xref{354}, \xref{362}, 
		\xref{421}, \xref{629};
    \subsubitem 1,  2,  2, \xref{376}, \xref{429}, \xref{441};
    \subsubitem 1,  2,  3, \xref{224}, \xref{364}, \xref{478};
    \subsubitem 1,  2,  4, \xref{295}, \xref{376}, \xref{521}, \xref{629};
    \subsubitem 1,  2,  5, \xref{423}, \xref{612};
    \subsubitem 1,  3,  1, \xref{291}, \xref{419}, \xref{477};
    \subsubitem 1,  3,  2, \xref{362}, \xref{589};
    \subsubitem 1,  3,  3, \xref{281}, \xref{339}, \xref{624};
    \subsubitem 1,  3,  4, \xref{384};
    \subsubitem 1,  3,  5, \xref{619};
    \subsubitem 1,  3,  7, \xref{353}, \xref{472}, \xref{593};
    \subsubitem 1,  4,  1, \xref{328}, \xref{406}, \xref{502}, \xref{604};
    \subsubitem 1,  4,  2, \xref{227}, \xref{322};
    \subsubitem 1,  4,  3–4, \xref{307};
    \subsubitem 1,  4,  4, \xref{521};
    \subsubitem 1,  5,  1, \xref{271};
    \subsubitem 1,  5,  2, \xref{279}, \xref{441};
    \subsubitem 1,  5,  3, \xref{450}, \xref{627};
    \subsubitem 1,  6,  1, \xref{279}, \xref{288}, \xref{328}, \xref{482}, 
		\xref{517}, \xref{521};
    \subsubitem 1,  6,  3, \xref{405}, \xref{421};
    \subsubitem 1,  6,  4, \xref{305}; % TBD
    \subsubitem 1,  7,  1, \xref{273}, \xref{408}, \xref{453}, \xref{586}, 
		\xref{626};
    \subsubitem 1,  7,  2, \xref{284};
    \subsubitem 1,  7,  3, \xref{288}, \xref{555}, \xref{557},
                           \xref{589}, \xref{627};
    \subsubitem 1,  7,  4, \xref{408};
    \subsubitem 1,  8,  1, \xref{423}, \xref{550}, \xref{567};
    \subsubitem 1,  8,  3, \xref{380};
    \subsubitem 1,  8,  4, \xref{408};
    \subsubitem 1,  9,  1, \xref{328}, \xref{569};
    \subsubitem 1,  9,  1–3, \xref{477};
    \subsubitem 1,  9,  2, \xref{525}, \xref{629};
    \subsubitem 1,  9,  3, \xref{241};
    \subsubitem 1, 10,  3, \xref{307}, \xref{420};
    \subsubitem 1, 10,  4, \xref{439};
    \subsubitem 1, 11,  1, \xref{426};
    \subsubitem 1, 11,  2, \xref{618};
    \subsubitem 1, 11,  3, \xref{295};
    \subsubitem 1, 12,  1, \xref{421};
    \subsubitem 1, 12,  2, \xref{295}, \xref{322};
    \subsubitem 1, 12,  3, \xref{307}, \xref{391};
    \subsubitem 1, 12,  5, \xref{305}, \xref{436};
    \subsubitem 1, 12,  6, \xref{243}, \xref{309}, \xref{445};
    \subsubitem 1, 13,  1, \xref{295}, \xref{612};
    \subsubitem 1, 13,  2, \xref{491}, \xref{521};
    \subsubitem 1, 13,  3, \xref{419};
    \subsubitem 1, 13,  4, \xref{350}, \xref{438};
    \subsubitem 1, 14,  1, \xref{229}, \xref{273}, \xref{321}, \xref{424}, 
		\xref{535};
    \subsubitem 1, 14,  2, \xref{354}, \xref{582};
    \subsubitem 1, 14,  3, \xref{303}, \xref{591};
    \subsubitem 1, 14,  4, \xref{552};
    \subsubitem 1, 14,  5, \xref{444};
    \subsubitem 1, 14,  6, \xref{526}, \xref{536};
    \subsubitem 1, 14,  7, \xref{328};
    \subsubitem 1, 15,  1, \xref{225}, \xref{264};
    \subsubitem 1, 15,  2, \xref{600};
    \subsubitem 1, 15,  3, \xref{416};
    \subsubitem 1, 16,  1, \xref{535}, \xref{595};
    \subsubitem 1, 16,  2, \xref{567};
    \subsubitem 1, 16,  3, \xref{405};
    \subsubitem 1, 16,  5, \xref{509};
    \subsubitem 1, 17,  2, \xref{502};
    \subsubitem 1, 18,  1, \xref{534};
    \subsubitem 1, 18,  2, \xref{223};
    \subsubitem 1, 18,  3, \xref{354}, \xref{427};
    \subsubitem 1, 18,  5, \xref{423};
    \subsubitem 1, 18,  9, \xref{632};
    \subsubitem 1, 19,  1, \xref{421}, \xref{513};
    \subsubitem 1, 19,  2, \xref{502};
    \subsubitem 1, 19,  3, \xref{354}, \xref{587};
    \subsubitem 1, 19,  5, \xref{309};
    \subsubitem 1, 20,  1, \xref{320}, \xref{530};
    \subsubitem 1, 20,  2, \xref{535};
    \subsubitem 1, 20,  4, \xref{472};
    \subsubitem 1, 20,  5, \xref{289}, \xref{354}, \xref{366},
                                       \xref{393}, \xref{491};
    \subsubitem 1, 21,  2, \xref{346};
    \subsubitem 1, 22,  1, \xref{244};
    \subsubitem 1, 22,  2, \xref{377};
    \subsubitem 1, 22,  3, \xref{263}, \xref{477};
    \subsubitem 1, 22,  4, \xref{244}, \xref{250};
    \subsubitem 1, 23,  1, \xref{380}, \xref{453}, \xref{509};
    \subsubitem 1, 23,  3, \xref{535};
    \subsubitem 1, 24,  1, \xref{391}, \xref{502};
    \subsubitem 1, 24,  3, \xref{425}, \xref{487};
    \subsubitem 1, 24,  4, \xref{381};
    \subsubitem 1, 25,  4, \xref{445};
    \subsubitem 1, 26,  1, \xref{290};
    \subsubitem 1, 26,  2, \xref{478};
    \subsubitem 1, 26,  3, \xref{276};
    \subsubitem 1, 26,  4, \xref{329};
    \subsubitem 1, 26,  5, \xref{440}, \xref{604};
    \subsubitem 1, 27,  1, \xref{255};
    \subsubitem 1, 27,  4, \xref{408};
    \subsubitem 1, 28,  3, \xref{307}, \xref{478}, \xref{517};
    \subsubitem 1, 29,  3, \xref{450};
    \subsubitem 1, 31,  2, \xref{368}, \xref{534};
    \subsubitem 1, 31,  7, \xref{502};
    \subsubitem 1, 31, 11, \xref{419};
    \subsubitem 1, 31, 14, \xref{507};
    \subsubitem 1, 32,  4, \xref{416};
    \subsubitem 1, 33,  1, \xref{360};
    \subsubitem 1, 34,  1, \xref{238}, \xref{586};
    \subsubitem 1, 34,  2, \xref{263}, \xref{397}, \xref{581};
    \subsubitem 1, 34,  3, \xref{405};
    \subsubitem 1, 35,  3, \xref{386};
    \subsubitem 1, 35,  4, \xref{339}, \xref{422}, \xref{578};
    \subsubitem 1, 36,  5, \xref{408}, \xref{438};
    \subsubitem 1, 36,  7, \xref{439}, \xref{538};
    \subsubitem 1, 38,  1, \xref{535};
    \subsubitem 1, 39,  2, \xref{339};
    \subsubitem 1, 39,  4, \xref{309};
    \subsubitem 1, 39,  6, \xref{367};
    \subsubitem 1, 39,  7, \xref{362};
    \subsubitem 1, 40,  1…8, \xref{536};
    \subsubitem 1, 40,  1…15, \xref{535};
    \subsubitem 1, 40,  2, \xref{276}, \xref{408};
    \subsubitem 1, 40,  4, \xref{263}, \xref{308}, \xref{513};
    \subsubitem 1, 40,  5, \xref{323}, \xref{590};
    \subsubitem 1, 41,  1, \xref{612};
    \subsubitem 1, 42,  3, \xref{472};
    \subsubitem 1, 42,  4, \xref{361};
    \subsubitem 1, 42,  5, \xref{278}, \xref{391}, \xref{420};
    \subsubitem 1, 42,  6, \xref{363}, \xref{430};
    \subsubitem 1, 43,  1, \xref{384}, \xref{624};
    \subsubitem 1, 43,  2, \xref{446};
    \subsubitem 1, 43,  6–7, \xref{537};
    \subsubitem 1, 43,  8, \xref{521};
    \subsubitem 1, 44,  1…8, \xref{537};
    \subsubitem 1, 44,  2, \xref{267}, \xref{284};
    \subsubitem 1, 44,  3, \xref{307};
    \subsubitem 1, 44,  4, \xref{586};
    \subsubitem 1, 44, 10, \xref{586}, \xref{614};
    \subsubitem 1, 45,  1, \xref{340};
    \subsubitem 1, 46,  1, \xref{377}, \xref{559};
    \subsubitem 1, 46,  4, \xref{366};
    \subsubitem 1, 47,  4, \xref{443}, \xref{585};
    \subsubitem 1, 47,  5, \xref{262};
    \subsubitem 1, 48,  1, \xref{433};
    \subsubitem 1, 48,  2, \xref{366}, \xref{424};
    \subsubitem 1, 49,  1, \xref{284}, \xref{361};
    \subsubitem 1, 51,  1, \xref{278}, \xref{476};
    \subsubitem 1, 51,  3, \xref{604};
    \subsubitem 1, 52,  2, \xref{405};
    \subsubitem 1, 52,  4, \xref{423};
    \subsubitem 1, 53,  1, \xref{561};
    \subsubitem 1, 53,  4, \xref{279}, \xref{451};
    \subsubitem 1, 54,  1, \xref{380};

    \subsubitem 2,  1,  1, \xref{326}, \xref{423};
    \subsubitem 2,  1,  3, \xref{406}, \xref{594};
    \subsubitem 2,  1,  4, \xref{422};
    \subsubitem 2,  2,  1, \xref{390};
    \subsubitem 2,  2,  3, \xref{346}, \xref{365};
    \subsubitem 2,  2,  4, \xref{398};
    \subsubitem 2,  3,  1, \xref{346};
    \subsubitem 2,  3,  3, \xref{446};
    \subsubitem 2,  4,  1, \xref{305}, \xref{386, 387}, \xref{413};
    \subsubitem 2,  4,  7, \xref{302};
    \subsubitem 2,  5,  1, \xref{390}, \xref{398};
    \subsubitem 2,  6,  1, \xref{387};
    \subsubitem 2,  6,  2, \xref{307};
    \subsubitem 2,  7,  1, \xref{429}, \xref{601};
    \subsubitem 2,  7,  2, \xref{270}, \xref{447};
    \subsubitem 2,  7,  3, \xref{380};
    \subsubitem 2,  8,  1, \xref{408};
    \subsubitem 2,  8,  3, \xref{384};
    \subsubitem 2,  9,  1, \xref{310}, \xref{509}, \xref{582};
    \subsubitem 2,  9,  2, \xref{276};
    \subsubitem 2,  9,  4, \xref{502}, \xref{578};
    \subsubitem 2, 10,  4, \xref{589};
    \subsubitem 2, 10,  5, \xref{302}, \xref{362}, \xref{364};
    \subsubitem 2, 11,  1, \xref{408};
    \subsubitem 2, 11,  2, \xref{380};
    \subsubitem 2, 11,  3, \xref{380};
    \subsubitem 2, 11,  4, \xref{387};
    \subsubitem 2, 11,  6, \xref{276}, \xref{381}, \xref{439};
    \subsubitem 2, 12,  2, \xref{421};
    \subsubitem 2, 13,  7, \xref{441};
    \subsubitem 2, 14,  1, \xref{589};
    \subsubitem 2, 14,  2, \xref{262};
    \subsubitem 2, 14,  3, \xref{393};
    \subsubitem 2, 14,  5, \xref{224}, \xref{355};
    \subsubitem 2, 15,  1, \xref{416};
    \subsubitem 2, 15,  5, \xref{364};
    \subsubitem 2, 16,  4, \xref{612};
    \subsubitem 2, 18,  1, \xref{331};
    \subsubitem 2, 18,  2, \xref{414};
    \subsubitem 2, 18,  4, \xref{391};
    \subsubitem 2, 18,  6, \xref{433};
    \subsubitem 2, 18,  8, \xref{426};
    \subsubitem 2, 19,  1, \xref{600}
    \subsubitem 2, 19,  2, \xref{421}, \xref{447};
    \subsubitem 2, 20,  1, \xref{373}, \xref{540};
    \subsubitem 2, 20,  2, \xref{320};
    \subsubitem 2, 20,  3, \xref{521};
    \subsubitem 2, 20,  5, \xref{478}, \xref{521};
    \subsubitem 2, 20,  6, \xref{278};
    \subsubitem 2, 22,  1, \xref{309};
    \subsubitem 2, 23,  1, \xref{244};
    \subsubitem 2, 23,  3, \xref{269};
    \subsubitem 2, 24,  1, \xref{246}, \xref{250}, \xref{265},
                           \xref{309}, \xref{517};
    \subsubitem 2, 24,  1–2, \xref{630};
    \subsubitem 2, 25,  2, \xref{371};
    \subsubitem 2, 25,  3, \xref{278};
    \subsubitem 2, 26,  5, \xref{340};
    \subsubitem 2, 27,  1, \xref{438}, \xref{477}, \xref{521};
    \subsubitem 2, 27,  3, \xref{302};
    \subsubitem 2, 29,  1, \xref{360};
    \subsubitem 2, 29,  3, \xref{433};
    \subsubitem 2, 30,  4, \xref{231}, \xref{443};
    \subsubitem 2, 32,  4, \xref{323}, \xref{380}, \xref{405};
    \subsubitem 2, 34,  3, \xref{362};
    \subsubitem 2, 34,  4, \xref{307}, \xref{325}, \xref{444};
    \subsubitem 3,  1, 3, \xref{531}, \xref{587};
    \subsubitem 3,  2, 1, \xref{482};
    \subsubitem 3,  2, 2, \xref{302};
    \subsubitem 3,  2, 5, \xref{436}, \xref{552};
    \subsubitem 3,  3, 4, \xref{586};
    \subsubitem 3,  4, 1, \xref{595};
    \subsubitem 3,  5, 1, \xref{485};
    \subsubitem 3,  5, 2, \xref{319};
    \subsubitem 3,  6, 1, \xref{223}, \xref{491};
    \subsubitem 3,  6, 2, \xref{613};
    \subsubitem 3,  6, 5, \xref{276};
    \subsubitem 3,  7, 1, \xref{586};
    \subsubitem 3,  7, 2, \xref{361};
    \subsubitem 3,  8, 1, \xref{391};
    \subsubitem 3,  8, 3, \xref{562};
    \subsubitem 3,  9, 3, \xref{380}, \xref{437};
    \subsubitem 3,  9, 6, \xref{532};
    \subsubitem 3,  9, 9, \xref{241};
    \subsubitem 3, 10, 1, \xref{612};
    \subsubitem 3, 10, 2, \xref{362};
    \subsubitem 3, 11, 2, \xref{225};
    \subsubitem 3, 11, 5, \xref{223}, \xref{420};
    \subsubitem 3, 12, 5, \xref{387}, \xref{612};
    \subsubitem 3, 13, 1, \xref{273};
    \subsubitem 3, 13, 3, \xref{349}, \xref{406};
    \subsubitem 3, 14, 4, \xref{421};
    \subsubitem 3, 15, 3, \xref{327}, \xref{477};
    \subsubitem 3, 15, 4, \xref{325};
    \subsubitem 3, 16, 2, \xref{564};
    \subsubitem 3, 16, 6, \xref{364}, \xref{468};
    \subsubitem 3, 16, 7, \xref{507};
    \subsubitem 3, 16, 8, \xref{245};
    \subsubitem 3, 17, 1, \xref{420};
    \subsubitem 3, 17, 3, \xref{273};
    \subsubitem 3, 17, 4, \xref{307};
    \subsubitem 3, 17, 5, \xref{446};
    \subsubitem 3, 17, 6, \xref{563}, \xref{612};
    \subsubitem 3, 18, 3, \xref{407};
    \subsubitem 3, 18, 4, \xref{502};
    \subsubitem 3, 21, 1, \xref{287};
    \subsubitem 3, 21, 3, \xref{588};
    \subsubitem 3, 22, 2, \xref{276};
    \subsubitem 3, 24, 1, \xref{507};
    \subsubitem 3, 24, 2, \xref{391};
    \subsubitem 3, 24, 4, \xref{433};
    \subsubitem 3, 24, 5, \xref{612};
    \subsubitem 3, 25, 1, \xref{612};
    \subsubitem 3, 26, 2, \xref{624};
    \subsubitem 3, 26, 4, \xref{436};
    \subsubitem 3, 27, 2, \xref{355};

    \subsubitem 4,  1,  4, \xref{590};
    \subsubitem 4,  1,  5, \xref{260}, \xref{303}, \xref{449};
    \subsubitem 4,  1,  7, \xref{417};
    \subsubitem 4,  1,  8, \xref{387};
    \subsubitem 4,  1,  9, \xref{431};
    \subsubitem 4,  2,  1, \xref{284};
    \subsubitem 4,  2,  3, \xref{445};
    \subsubitem 4,  2,  5, \xref{276};
    \subsubitem 4,  3,  4, \xref{362};
    \subsubitem 4,  7,  2, \xref{321};
    \subsubitem 4,  7,  3, \xref{307}, \xref{502};
    \subsubitem 4, 10,  3, \xref{310};
    \subsubitem 4, 11,  3, \xref{612};
    \subsubitem 4, 11,  4, \xref{355};
    \subsubitem 4, 12,  4, \xref{413};
    \subsubitem 4, 12,  6, \xref{421}, \xref{550};
    \subsubitem 4, 13,  4, \xref{245};
    \subsubitem 4, 13,  5, \xref{612};
    \subsubitem 4, 14,  2, \xref{612};
    \subsubitem 4, 16,  1, \xref{367}, \xref{385};
    \subsubitem 4, 17, 10, \xref{616};
    \subsubitem 4, 18,  1, \xref{550};
    \subsubitem 4, 18,  4, \xref{550};
    \subsubitem 4, 21,  5, \xref{256}, \xref{593};
    \subsubitem 4, 22,  6, \xref{295};
    \subsubitem 4, 23,  2, \xref{624};
    \subsubitem 4, 23,  4, \xref{507};
    \subsubitem 4, 23,  6, \xref{323}, \xref{436};
    \subsubitem 4, 24,  2, \xref{600};
    \subsubitem 4, 24,  4, \xref{323};
    \subsubitem 4, 25,  3, \xref{284}, \xref{391}, \xref{578};
    \subsubitem 4, 25,  4, \xref{261};
    \subsubitem 4, 25,  5, \xref{266};
    \subsubitem 4, 26,  1, \xref{278}, \xref{279};
    \subsubitem 4, 27,  2, \xref{418};
    \subsubitem 4, 27,  5, \xref{592};
    \subsubitem 4, 28,  1, \xref{550};
    \subsubitem 4, 29,  3, \xref{325};
    \subsubitem 4, 31,  1, \xref{582};
    \subsubitem 4, 32,  5, \xref{376};
    \subsubitem 4, 34,  4, \xref{446};
    \subsubitem 4, 35,  1, \xref{407};
    \subsubitem 4, 35,  3, \xref{426};
    \subsubitem 4, 37,  1, \xref{381};
    \subsubitem 4, 38,  5, \xref{355};
    \subsubitem 5,  7,  3, \xref{436};
    \subsubitem 5,  8,  3, \xref{439};
    \subsubitem 5,  8,  6, \xref{616};
    \subsubitem 5, 12,  2, \xref{612};
    \subsubitem 5, 12,  5, \xref{355};
    \subsubitem 5, 14,  1, \xref{284};
    \subsubitem 5, 15,  5, \xref{285};
    \subsubitem 5, 16,  4, \xref{265};
    \subsubitem 5, 17,  3, \xref{560};
    \subsubitem 5, 19,  3, \xref{292};
    \subsubitem 5, 20,  3, \xref{344};
    \subsubitem 5, 27,  1, \xref{288};
    \subsubitem 5, 28,  3, \xref{276};
    \subsubitem 5, 30,  1, \xref{582};
    \subsubitem 5, 33,  3, \xref{278}, \xref{391};
    \subsubitem 5, 35,  1, \xref{276};
    \subsubitem 5, 35,  7, \xref{355};
    \subsubitem 5, 37,  5, \xref{524}, \xref{600};
    \subsubitem 5, 39,  4, \xref{321};
    \subsubitem 5, 40,  5, \xref{430}, \xref{491};
    \subsubitem 5, 41,  5, \xref{276};
    \subsubitem 5, 44, 13, \xref{277};
    \subsubitem 5, 44, 14, \xref{265};
    \subsubitem 5, 50,  3, \xref{264};
    \subsubitem 5, 52,  2, \xref{278};
    \subsubitem 5, 53,  3, \xref{264};
    \subsubitem 5, 54,  5, \xref{297};
    \subsubitem 5, 55,  3, \xref{436};
    \subsubitem 6,  3,  2, \xref{507};
    \subsubitem 6,  4,  4, \xref{605};
    \subsubitem 6,  7,  2, \xref{566};
    \subsubitem 6,  8,  3, \xref{284};
    \subsubitem 6, 10,  1, \xref{320};
    \subsubitem 6, 11,  1, \xref{266};
    \subsubitem 6, 11,  4, \xref{347};
    \subsubitem 6, 12,  1, \xref{550};
    \subsubitem 6, 12,  2, \xref{476};
    \subsubitem 6, 13,  6, \xref{438};
    \subsubitem 6, 14,  4, \xref{483};
    \subsubitem 6, 16,  3, \xref{628};
    \subsubitem 6, 21,  3, \xref{438};
    \subsubitem 6, 24,  1, \xref{521};
    \subsubitem 6, 27,  1, \xref{631};
    \subsubitem 6, 27,  4, \xref{436};
    \subsubitem 6, 32,  4, \xref{329};
    \subsubitem 6, 32,  6, \xref{307};
    \subsubitem 6, 35,  2, \xref{482};
    \subsubitem 6, 37,  8, \xref{323};
    \subsubitem 6, 41,  4, \xref{344};
    \subsubitem 6, 44,  1, \xref{422};
    \subsubitem 6, 44,  3, \xref{366};
    \subsubitem 7,  3,  3, \xref{244};
    \subsubitem 7, 11,  9, \xref{365}, \xref{386};
    \subsubitem 7, 16,  3, \xref{416};
    \subsubitem 7, 20, 12, \xref{240};
    \subsubitem 7, 28,  1, \xref{408};
    \subsubitem 7, 36,  5, \xref{580};
    \subsubitem 7, 42,  1, \xref{612};
    \subsubitem 7, 45,  4, \xref{562};
    \subsubitem 7, 50,  6, \xref{559};
    \subsubitem 7, 54,  4, \xref{428};
    \subsubitem 7, 60,  1, \xref{381};
    \subsubitem 7, 60,  3, \xref{587};
    \subsubitem 7, 65,  3, \xref{445};
    \subsubitem 7, 66,  6, \xref{502};
    \subsubitem 7, 67,  5, \xref{604};
    \subsubitem 7, 81,  4, \xref{264};
    \subsubitem 7, 83,  8, \xref{624};
    \subsubitem 7, 84,  4, \xref{438};
    \subsubitem 7, 87,  5, \xref{558}; % TBD
    \subsubitem 7, 88,  6, \xref{582}.

\indexspace

\indexauthor{Cato},
  \subitem \emph{dē Agrī Cultūrā} (Cato Agr.),
    \subsubitem Intr. 1, \xref{484}.

\indexspace

\indexauthor{Catullus},
  \subitem  6,   1, \xref{581};
  \subitem 11,  15, \xref{464};
  \subitem 23,  25, \xref{356};
  \subitem 61, 152, \xref{502};
  \subitem 86,   1, \xref{370};
  \subitem 89,   4, \xref{515};
  \subitem 89,   6, \xref{515}.

\indexspace

\indexauthor{Cicero},
  \subitem \emph{Academica} (Ac.),
    \subsubitem 1, 12,  46, \xref{346};
    \subsubitem 2,  2,   5, \xref{521};
    \subsubitem 2,  3,   7, \xref{445};
    \subsubitem 2,  4,  11, \xref{524};
    \subsubitem 2, 29,  92, \xref{481};
    \subsubitem 2, 45, 139, \xref{362};
    \subsubitem 2, 46, 141, \xref{464}.

  \subitem \emph{Brutus} (Brut.),
    \subsubitem 43, 161, \xref{550};
    \subsubitem 48, 180, \xref{522};
    \subsubitem 88, 302, \xref{276}.

  \subitem \emph{in Q. Caecilium} (Caecil.),
    \subsubitem  2,  5, \xref{307};
    \subsubitem 13, 44, \xref{521};
    \subsubitem 15, 48, \xref{417}.

  \subitem \emph{prō Caecinā} (Caecin.),
    \subsubitem  2,   4, \xref{502};
    \subsubitem  3,   7, \xref{424};
    \subsubitem  4,  12, \xref{363};
    \subsubitem  7,  20, \xref{238};
    \subsubitem  9,  23, \xref{511};
    \subsubitem 11,  30, \xref{423};
    \subsubitem 14,  39, \xref{388};
    \subsubitem 17,  49, \xref{415};
    \subsubitem 36, 103, \xref{605};
    \subsubitem 36, 104, \xref{393}.

  \subitem \emph{Caecin. ap. Fam.},
    \subsubitem 6, 712, \xref{292}.

  \subitem \emph{prō Caeliō} (Cael.),
    \subsubitem 6,  14, \xref{464};
    \subsubitem 14, 34, \xref{432}.

    % Check
  \subitem \emph{M. Caelī Epistulae ad M. Tullium Ci\-ce\-ro\-nem}
  (Cael., Fam.),
    \subsubitem 8, 10, 3, \xref{390}.

  \subitem \emph{contra Contionem Q. Metellī} (Met.),
    \subsubitem 7, 61, \xref{652}.

  \subitem \emph{dē Divinatione} (Div.),
    \subsubitem 1,  5,   8, \xref{302};
    \subsubitem 1, 15,  28, \xref{624};
    \subsubitem 1, 31,  67, \xref{436};
    \subsubitem 1, 46, 104, \xref{524};
    \subsubitem 1, 58, 132, \xref{356};
    \subsubitem 2, 18,  42, \xref{582};
    \subsubitem 2, 23,  50, \xref{628};
    \subsubitem 2, 61, 127, \xref{501};
    \subsubitem 2, 68, 140, \xref{446}.

  \subitem \emph{dē Domō suā} (Dom.),
    \subsubitem S. 4,  8, \xref{388};
    \subsubitem S. 9, 22, \xref{566}.

\columnbreak

  \subitem \emph{dē Finibus Bonōrum et Malōrum} (Fin.),
    \subsubitem 1,  4, 11, \xref{339};
    \subsubitem 1,  5, 14, \xref{397};
    \subsubitem 1,  7, 26, \xref{284};
    \subsubitem 1, 14, 47, \xref{438};
    \subsubitem 1, 16, 51, \xref{250};
    \subsubitem 1, 18, 60, \xref{613};
    \subsubitem 2, 13, 43, \xref{597};
    \subsubitem 2, 17, 55, \xref{319};
    \subsubitem 2, 18, 59, \xref{582};
    \subsubitem 2, 26, 82, \xref{521};
    \subsubitem 2, 27, 86, \xref{597};
    \subsubitem 3,  2,  5, \xref{441};
    \subsubitem 3,  2,  9, \xref{501};
    \subsubitem 3,  3, 10, \xref{502};
    \subsubitem 3, 20, 66, \xref{571};
    \subsubitem 3, 22, 73, \xref{291};
    \subsubitem 4, 23, 62, \xref{427};
    \subsubitem 4, 27, 76, \xref{596};
    \subsubitem 5,  1,  3, \xref{350};
    \subsubitem 5, 13, 37, \xref{260};
    \subsubitem 5, 22, 64, \xref{408}.

  \subitem \emph{dē Haruspicum Responsō} (Har. Resp.),
    \subsubitem 10, 20, \xref{519}.

  \subitem \emph{dē Inventione} (Inv.),
    \subsubitem 1, 20,  28, \xref{619};
    \subsubitem 2,  2,   5, \xref{615};
    \subsubitem 2, 51, 154, \xref{377}.

  \subitem \emph{dē Iure Civilī in artem redigendō} (Iuv.),
    \subsubitem  1,  74, \xref{276};
    \subsubitem  2,   3, \xref{396};
    \subsubitem  3,  41, \xref{503};
    \subsubitem  3,  80, \xref{449};
    \subsubitem  4,  98, \xref{519};
    \subsubitem  7, 197, \xref{406};
    \subsubitem 16,  41, \xref{507}.

\columnbreak

  \subitem \emph{dē Lege Agrariā} (Leg. Agr.),
    \subsubitem 1,  7,  22–23, \xref{429}, \xref{551};
    \subsubitem 1,  8,  101, \xref{422};
    \subsubitem 1,  9,  27, \xref{348}, \xref{422};
    \subsubitem 2,  8,  20, \xref{363};
    \subsubitem 2, 28,  76, \xref{452};
    \subsubitem 2, 30,  83, \xref{339};
    \subsubitem 2, 37, 101, \xref{422};
    \subsubitem 3,  3,  12, \xref{393}.

  \subitem \emph{dē Legibus} (Leg.),
    \subsubitem 1, 14, 41, \xref{250};
    \subsubitem 1, 15, 42, \xref{363};
    \subsubitem 1, 19, 52, \xref{612};
    \subsubitem 2,  2,  5, \xref{624};
    \subsubitem 2, 23, 59, \xref{616};
    \subsubitem 3,  1,  2, \xref{563};
    \subsubitem 3, 11, 26, \xref{257}.

  \subitem \emph{dē Naturā Deōrum} (N. D.),
    \subsubitem 1,  5, 10, \xref{267};
    \subsubitem 1,  7, 17, \xref{363};
    \subsubitem 1,  8, 19, \xref{605};
    \subsubitem 1, 18, 47, \xref{274};
    \subsubitem 1, 27, 75, \xref{250};
    \subsubitem 1, 28, 79, \xref{278};
    \subsubitem 2, 23, 59, \xref{432};
    \subsubitem 3,  6, 14, \xref{604};
    \subsubitem 3, 22, 56, \xref{450};
    \subsubitem 3, 34, 84, \xref{354}.

  \subitem \emph{dē Officiīs} (Off.),
    \subsubitem 1, 13,  40, \xref{439}, \xref{535};
    \subsubitem 1, 14,  43, \xref{521};
    \subsubitem 1, 24,  83, \xref{250};
    \subsubitem 1, 30, 105, \xref{391}, \xref{441};
    \subsubitem 1, 38, 137, \xref{264};
    \subsubitem 1, 41, 147, \xref{332};
    \subsubitem 1, 42, 150, \xref{580};
    \subsubitem 2, 20,  70, \xref{586};
    \subsubitem 3,  5,  23, \xref{362};
    \subsubitem 3,  7,  34, \xref{362};
    \subsubitem 3, 15,  64, \xref{582};
    \subsubitem 3, 19,  77, \xref{299};
    \subsubitem 3, 20,  82, \xref{513};
    \subsubitem 3, 25,  94, \xref{582};
    \subsubitem 3, 33, 121, \xref{414}.

  \subitem \emph{dē Oratore} (De Or.),
    \subsubitem 1,  4,  13, \xref{319};
    \subsubitem 1, 33, 150, \xref{319};
    \subsubitem 1, 44, 195, \xref{532};
    \subsubitem 1, 44, 196, \xref{478};
    \subsubitem 1, 58, 246, \xref{438};
    \subsubitem 1, 61, 260, \xref{539};
    \subsubitem 2, 38, 157, \xref{613};
    \subsubitem 2, 42, 178, \xref{494};
    \subsubitem 2, 44, 186, \xref{507};
    \subsubitem 2, 48, 198, \xref{479};
    \subsubitem 2, 58, 235, \xref{604};
    \subsubitem 2, 70, 285, \xref{284};
    \subsubitem 2, 71, 287, \xref{288};
    \subsubitem 3,  2,   6, \xref{418};
    \subsubitem 3, 10,  39, \xref{612}.

  \subitem \emph{dē Republicā} (Rep.),
    \subsubitem 1,  2,  3, \xref{397};
    \subsubitem 1,  3,  6, \xref{297};
    \subsubitem 1, 15, 23, \xref{297};
    \subsubitem 1, 16, 25, \xref{661};
    \subsubitem 2, 10, 18, \xref{521};
    \subsubitem 2, 19, 34, \xref{452};
    \subsubitem 2, 21, 38, \xref{444};
    \subsubitem 2, 31, 55, \xref{297};
    \subsubitem 3, 18, 28, \xref{589};
    \subsubitem 3, 35, 47, \xref{397};
    \subsubitem 6, 23, 25, \xref{464}.

  \subitem \emph{dē Senectūte} (Sen.),
    \subsubitem  1,  1, \xref{535};
    \subsubitem  1,  2, \xref{284};
    \subsubitem  3,  4, \xref{581};
    \subsubitem  3,  8, \xref{566};
    \subsubitem  4, 10, \xref{271}, \xref{550}, \xref{567}, \xref{625};
    \subsubitem  4, 11, \xref{484};
    \subsubitem  6, 18, \xref{507}, \xref{569};
    \subsubitem 10, 31, \xref{231}, \xref{521};
    \subsubitem 10, 33, \xref{276}, \xref{501};
    \subsubitem 11, 35, \xref{600};
    \subsubitem 11, 38, \xref{320}, \xref{521}, \xref{578}, \xref{609}, \xref{632};
    \subsubitem 13, 43, \xref{439};
    \subsubitem 14, 50, \xref{550};
    \subsubitem 15, 56, \xref{582};
    \subsubitem 16, 55, \xref{582};
    \subsubitem 16, 57, \xref{531};
    \subsubitem 17, 59, \xref{550};
    \subsubitem 19, 68, \xref{593};
    \subsubitem 20, 74, \xref{604};
    \subsubitem 21, 77, \xref{396}, \xref{513};
    \subsubitem 22, 79, \xref{239};
    \subsubitem 22, 81, \xref{254};
    \subsubitem 23, 82, \xref{582}.

  \subitem \emph{prō Rege Deiotarō} (Dei.),
    \subsubitem  6, 16, \xref{579};
    \subsubitem 14, 40, \xref{423}.

  \subitem \emph{Epistolae ad Atticum} (Att.),
    \subsubitem  1,  9,  2, \xref{490}, \xref{501};
    \subsubitem  1, 17,  6, \xref{585};
    \subsubitem  2, 22,  6, \xref{597};
    \subsubitem  3, 13,  1, \xref{504};
    \subsubitem  4,  1,  6, \xref{612};
    \subsubitem  4,  3,  4, \xref{439};
    \subsubitem  4,  3,  5, \xref{493};
    \subsubitem  5, 11,  7, \xref{524};
    \subsubitem  6,  1, 14, \xref{423};
    \subsubitem  6,  1, 18, \xref{276};
    \subsubitem  7,  8,  2, \xref{464};
    \subsubitem  7, 11,  4, \xref{608};
    \subsubitem  7, 18,  1, \xref{480};
    \subsubitem  8,  6,  3, \xref{477}, \xref{582};
    \subsubitem  9,  6,  2, \xref{501};
    \subsubitem  9,  6,  6, \xref{502};
    \subsubitem  9, 10,  2, \xref{513};
    \subsubitem  9, 10,  4, \xref{408};
    \subsubitem 11,  9,  3, \xref{511};
    \subsubitem 12, 22,  3, \xref{464};
    \subsubitem 13, 28,  2, \xref{585};
    \subsubitem 13, 29,  3, \xref{569};
    \subsubitem 14, 19,  6, \xref{537};
    \subsubitem 16,  6,  1, \xref{435};
    \subsubitem 16,  8,  2, \xref{234}.

  \subitem \emph{Epistolae ad Familiārēs} (Fam.),
    \subsubitem  1,  7, 2, \xref{552};
    \subsubitem  3,  8, 2, \xref{612};
    \subsubitem  4,  4, 3, \xref{368};
    \subsubitem  4,  6, 1, \xref{521};
    \subsubitem  4,  7, 1, \xref{412};
    \subsubitem  5,  7, 3, \xref{416};
    \subsubitem  6,  6, 9, \xref{248};
    \subsubitem  6,  7, 1, \xref{478};
    \subsubitem  6,  7, 6, \xref{502};
    \subsubitem  7, 13, 1, \xref{531};
    \subsubitem  7, 30, 1, \xref{478};
    \subsubitem  7, 32, 3, \xref{532};
    \subsubitem  8,  1, 1, \xref{362};
    \subsubitem  8,  6, 1, \xref{552};
    \subsubitem  9,  1, 2, \xref{535};
    \subsubitem  9, 16, 4, \xref{517};
    \subsubitem  9, 16, 7, \xref{513};
    \subsubitem 10, 18, 1, \xref{613};
    \subsubitem 12, 10, 3, \xref{581};
    \subsubitem 13, 16, 4, \xref{274};
    \subsubitem 14,  3, 5, \xref{325};
    \subsubitem 14,  5, 1, \xref{329}, \xref{624};
    \subsubitem 14, 12,    \xref{451};
    \subsubitem 15,  2, 1, \xref{377};
    \subsubitem 15, 13, 1, \xref{339}.

  \subitem \emph{Epistulae ad Quintum fratrem} (Q. Fr.),
    \subsubitem 1, 2, 4, 14, \xref{356};
    \subsubitem 1, 2, 5, 15, \xref{355};
    \subsubitem 1, 2, 5, 16, \xref{443}.

  \subitem \emph{in Catilinam} (Cat.),
    \subsubitem 1,  1,  1, \xref{231}, \xref{400}, \xref{624}, \xref{627};
    \subsubitem 1,  1,  2, \xref{222}, \xref{233}, \xref{254}, \xref{273},
		\xref{305}, \xref{335}, \xref{362}, \xref{399};
    \subsubitem 1,  1,  3, \xref{236}, \xref{317}, \xref{600};
    \subsubitem 1,  2,  4, \xref{268}, \xref{335}, \xref{342}, \xref{414},
		\xref{443}, \xref{490}, \xref{625};
    \subsubitem 1,  2,  5, \xref{339}, \xref{524}, \xref{582}, \xref{624};
    \subsubitem 1,  2,  6, \xref{276}, \xref{550}, \xref{625};
    \subsubitem 1,  3,  6, \xref{362}, \xref{496}, \xref{531};
    \subsubitem 1,  3,  7, \xref{270}, \xref{325}, \xref{535}, \xref{593}, 
		\xref{600};
    \subsubitem 1,  3,  8, \xref{632};
    \subsubitem 1,  4,  8, \xref{231}, \xref{276}, \xref{388}, \xref{454}, 
		\xref{545}, \xref{624};
    \subsubitem 1,  4,  9, \xref{284}, \xref{346}, \xref{393}, \xref{545}, 
		\xref{582}, \xref{627};
    \subsubitem 1,  4, 10, \xref{246}, \xref{388};
    \subsubitem 1,  5, 10, \xref{284}, \xref{526}, \xref{529}, \xref{572}, 
		\xref{578}, \xref{627};
    \subsubitem 1,  5, 11, \xref{261}, \xref{439}, \xref{579},
                \xref{624}, \xref{625};
    \subsubitem 1,  5, 12, \xref{397}, \xref{485}, \xref{632};
    \subsubitem 1,  6, 12, \xref{624};
    \subsubitem 1,  6, 13, \xref{367}, \xref{484}, \xref{579};
    \subsubitem 1,  6, 14, \xref{254}, \xref{385};
    \subsubitem 1,  6, 15, \xref{231}, \xref{305}, \xref{368}, \xref{380};
    \subsubitem 1,  6, 16, \xref{624};
    \subsubitem 1,  7, 16, \xref{373};
    \subsubitem 1,  7, 17, \xref{373}, \xref{581}, \xref{626};
    \subsubitem 1,  7, 18, \xref{371}, \xref{478}, \xref{502}, \xref{539}, 
		\xref{578};
    \subsubitem 1,  8, 19, \xref{454}, \xref{552}, \xref{582}, \xref{604}, 
		\xref{632};
    \subsubitem 1,  8, 20, \xref{412}, \xref{587};
    \subsubitem 1,  8,  21, \xref{551};
    \subsubitem 1,  9, 22, \xref{310}, \xref{503}, \xref{521};
    \subsubitem 1,  9, 23, \xref{362};
    \subsubitem 1,  9, 24, \xref{291}, \xref{513};
    \subsubitem 1, 11, 27, \xref{416}, \xref{572}, \xref{587}, \xref{625};
    \subsubitem 1, 11, 28, \xref{310};
    \subsubitem 1, 12, 29, \xref{581}, \xref{597};
    \subsubitem 1, 12, 30, \xref{568};
    \subsubitem 1, 13, 31, \xref{276}, \xref{537};
    \subsubitem 1, 13, 32, \xref{250}, \xref{257}, \xref{264}, 
		\xref{418}, \xref{490}, \xref{501};
    \subsubitem 1, 13, 33, \xref{422};
    \subsubitem 2,  1,  1, \xref{250}, \xref{269}, \xref{285}, \xref{308};
    \subsubitem 2,  1,  2, \xref{604}, \xref{625};
    \subsubitem 2,  2,  3, \xref{521}, \xref{625};
    \subsubitem 2,  2,  4, \xref{372}, \xref{443}, \xref{511};
    \subsubitem 2,  3,  5, \xref{494}, \xref{496}, \xref{519}, \xref{604, 605};
    \subsubitem 2,  3,  6, \xref{267}, \xref{302};
    \subsubitem 2,  4,  6, \xref{366}, \xref{426}, \xref{579};
    \subsubitem 2,  4,  7, \xref{222}, \xref{370}, \xref{387};
    \subsubitem 2,  4,  8, \xref{612};
    \subsubitem 2,  5, 10, \xref{240}, \xref{372}, \xref{578}, \xref{601};
    \subsubitem 2,  5, 11, \xref{276}, \xref{311}, \xref{363}, \xref{449}, 
		\xref{571}, \xref{627};
    \subsubitem 2,  6, 12, \xref{454}, \xref{521}, \xref{581};
    \subsubitem 2,  6, 13, \xref{234}, \xref{537};
    \subsubitem 2,  7, 15, \xref{600};
    \subsubitem 2,  7, 16, \xref{511};
    \subsubitem 2,  8, 17, \xref{231};
    \subsubitem 2,  8, 18, \xref{231}, \xref{434}, \xref{503}, \xref{571};
    \subsubitem 2,  9, 19, \xref{391};
    \subsubitem 2,  9, 20, \xref{310}, \xref{397}, \xref{559};
    \subsubitem 2, 10, 21, \xref{578};
    \subsubitem 2, 10, 23, \xref{496};
    \subsubitem 2, 11, 25, \xref{406};
    \subsubitem 2, 12, 26, \xref{259};
    \subsubitem 2, 12, 27, \xref{363}, \xref{490}, \xref{507}, \xref{556};
    \subsubitem 2, 13, 29, \xref{284};
    \subsubitem 3,  1,     \xref{630};
    \subsubitem 3,  1,  1, \xref{354}, \xref{384};
    \subsubitem 3,  1,  3, \xref{424}, \xref{550};
    \subsubitem 3,  2,  5, \xref{354}, \xref{523};
    \subsubitem 3,  3,  6, \xref{524};
    \subsubitem 3,  3,  7, \xref{298};
    \subsubitem 3,  5, 10, \xref{502}, \xref{523}, \xref{569}, \xref{624};
    \subsubitem 3,  5, 11, \xref{302}, \xref{525}, \xref{536};
    \subsubitem 3,  5, 12, \xref{310}, \xref{323};
    \subsubitem 3,  5, 13, \xref{302}, \xref{564};
    \subsubitem 3,  6, 14, \xref{590}, \xref{623};
    \subsubitem 3,  6, 15, \xref{408}, \xref{535}, \xref{582}, \xref{608};
    \subsubitem 3,  7, 16, \xref{305}, \xref{346}, \xref{374}, \xref{579}, 
		\xref{605};
    \subsubitem 3,  7, 17, \xref{519};
    \subsubitem 3,  8, 19, \xref{436};
    \subsubitem 3,  8, 20, \xref{309};
    \subsubitem 3,  9, 21, \xref{305}, \xref{479};
    \subsubitem 3,  9, 22, \xref{442}, \xref{580};
    \subsubitem 3, 10, 24, \xref{310}, \xref{624};
    \subsubitem 3, 12, 27, \xref{250}, \xref{481};
    \subsubitem 3, 12, 28, \xref{373}, \xref{526};
    \subsubitem 4,  1,  1, \xref{380};
    \subsubitem 4,  1,  2, \xref{391}, \xref{411}, \xref{422}, \xref{550};
    \subsubitem 4,  2,  3, \xref{307}, \xref{367}, \xref{445};
    \subsubitem 4,  2,  4, \xref{377};
    \subsubitem 4,  4,  7, \xref{274}, \xref{429}, \xref{484};
    \subsubitem 4,  5,  9, \xref{345}, \xref{507};
    \subsubitem 4,  5, 10, \xref{274}, \xref{298};
    \subsubitem 4,  6, 11, \xref{275}, \xref{605};
    \subsubitem 4,  6, 12, \xref{310}, \xref{392};
    \subsubitem 4,  6, 13, \xref{240}, \xref{513}, \xref{578};
    \subsubitem 4,  7, 14, \xref{333}, \xref{502};
    \subsubitem 4,  7, 15, \xref{279}, \xref{521}, \xref{624};
    \subsubitem 4,  7, 16, \xref{564};
    \subsubitem 4,  8, 16, \xref{276}, \xref{579};
    \subsubitem 4, 10, 20, \xref{307}, \xref{352}, \xref{571};
    \subsubitem 4, 10, 21, \xref{578}, \xref{624};
    \subsubitem 4, 10, 22, \xref{542};
    \subsubitem 4, 11, 23, \xref{339};
    \subsubitem 4, 11, 24, \xref{271}, \xref{477}.

  \subitem \emph{in Pisonem} (Pis.),
    \subsubitem  6, 12, \xref{407};
    \subsubitem 11, 26, \xref{406};
    \subsubitem 17, 39, \xref{345};
    \subsubitem 28, 68, \xref{517}.

  \subitem \emph{in Vatinium} (Vat.),
    \subsubitem  4, 10, \xref{410};
    \subsubitem 16, 40, \xref{393}.

  \subitem \emph{in Verrem} (Verr.),
    \subsubitem A. Pr. 6, 16, \xref{550};
    \subsubitem A. Pr. 9, 25, \xref{408};
    \subsubitem 1, 17,  46, \xref{450};
    \subsubitem 1, 18,  46, \xref{451};
    \subsubitem 1, 18,  47, \xref{439};
    \subsubitem 1, 27,  70, \xref{582}, \xref{612};
    \subsubitem 1, 33,  83, \xref{364};
    \subsubitem 1, 50, 132, \xref{290};
    \subsubitem 2, 10,  26, \xref{504};
    \subsubitem 2, 20,  49, \xref{513};
    \subsubitem 2, 38,  94, \xref{438};
    \subsubitem 2, 46, 113, \xref{242};
    \subsubitem 2, 67,  16, \xref{587};
    \subsubitem 2, 74, 182, \xref{377};
    \subsubitem 3,  6,  15, \xref{333};
    \subsubitem 3, 11,  28, \xref{343};
    \subsubitem 3, 30,  71, \xref{427};
    \subsubitem 3, 48, 115, \xref{464};
    \subsubitem 3, 54, 126, \xref{310};
    \subsubitem 3, 84, 195, \xref{513};
    \subsubitem 3, 92, 215, \xref{427};
    \subsubitem 4,  6,  11, \xref{292}, \xref{515};
    \subsubitem 4, 27,  62, \xref{406};
    \subsubitem 4, 34,  76, \xref{422};
    \subsubitem 4, 36,  79, \xref{407};
    \subsubitem 4, 40,  87, \xref{582};
    \subsubitem 4, 43,  95, \xref{521};
    \subsubitem 4, 52, 117, \xref{441};
    \subsubitem 4, 55, 123, \xref{276};
    \subsubitem 4, 64, 142, \xref{539};
    \subsubitem 5,  5,  11, \xref{464};
    \subsubitem 5, 36,  93, \xref{422};
    \subsubitem 5, 41, 106, \xref{408};
    \subsubitem 5, 54, 140, \xref{521}.

  \subitem \emph{Laelius dē Amīcitiā} (Am.),
    \subsubitem  1,   1, \xref{586};
    \subsubitem  1,  39, \xref{295};
    \subsubitem  2,  10, \xref{496};
    \subsubitem  3,  11, \xref{532};
    \subsubitem  6,  21, \xref{240};
    \subsubitem  6,  22, \xref{604};
    \subsubitem 13,  47, \xref{632};
    \subsubitem 15,  53, \xref{347};
    \subsubitem 16,  58, \xref{416};
    \subsubitem 17,  63, \xref{390};
    \subsubitem 19,  70, \xref{354};
    \subsubitem 27, 101, \xref{438};
    \subsubitem 27, 103, \xref{298}.

  \subitem \emph{Ōrātiō post Reditum in Senātū Ha\-bi\-ta} (Senat.),
    \subsubitem 8, 19, \xref{342}.

  \subitem \emph{Orator} (Or.),
    \subsubitem 20,  68, \xref{615};
    \subsubitem 44, 151, \xref{539}.

  \subitem \emph{Paradoxa Stoicorum ad M. Brutum} (Par.),
    \subsubitem 6, 3, 51, \xref{585}.

  \subitem \emph{Philippicae} (Phil.),
    \subsubitem  1,  4,   9, \xref{422};
    \subsubitem  1,  4,  11, \xref{396};
    \subsubitem  1,  5,  11, \xref{380};
    \subsubitem  1,  9,  21, \xref{234}, \xref{310};
    \subsubitem  1, 12,  31, \xref{627};
    \subsubitem  1, 14,  35, \xref{579};
    \subsubitem  1, 15,  36, \xref{362};
    \subsubitem  1, 15,  37, \xref{521};
    \subsubitem  2,  1,   1, \xref{416};
    \subsubitem  2,  3,   6, \xref{432};
    \subsubitem  2, 16,  41, \xref{443};
    \subsubitem  2, 19,  47, \xref{439};
    \subsubitem  2, 26,  64, \xref{579};
    \subsubitem  2, 34,  85, \xref{422};
    \subsubitem  2, 37,  95, \xref{513};
    \subsubitem  2, 40, 102, \xref{423};
    \subsubitem  3, 14,  35, \xref{431};
    \subsubitem  4,  2,   6, \xref{452};
    \subsubitem  5,  7,  18, \xref{406};
    \subsubitem  5,  9,  24, \xref{377};
    \subsubitem  5,  9,  25, \xref{238}, \xref{586};
    \subsubitem  6,  3,   6, \xref{464};
    \subsubitem  6,  7,  18, \xref{397};
    \subsubitem  8,  1,   3, \xref{439};
    \subsubitem  8,  3,   8, \xref{397};
    \subsubitem  8,  8,  25, \xref{523};
    \subsubitem  8, 11,  32, \xref{325};
    \subsubitem 11,  2,   6, \xref{436};
    \subsubitem 11,  4,   9, \xref{444};
    \subsubitem 12, 10,  24, \xref{550};
    \subsubitem 14,  5,  13, \xref{582}.

  \subitem \emph{On Pompey's Command} (Pomp.), % need Latin
    \subsubitem  1,  2, \xref{302}, \xref{624};
    \subsubitem  2,  6, \xref{373};
    \subsubitem  3,  7, \xref{436};
    \subsubitem  5, 11, \xref{326}, \xref{421};
    \subsubitem  5, 13, \xref{483};
    \subsubitem  6, 15, \xref{579};
    \subsubitem  7, 18, \xref{582};
    \subsubitem  7, 19, \xref{521};
    \subsubitem  8, 20–21, \xref{483};
    \subsubitem  9, 22, \xref{250}, \xref{624};
    \subsubitem  9, 23, \xref{568};
    \subsubitem 10, 28, \xref{339}, \xref{439};
    \subsubitem 11, 29, \xref{612};
    \subsubitem 11, 32, \xref{444};
    \subsubitem 14, 41, \xref{302}, \xref{478};
    \subsubitem 16, 46, \xref{231};
    \subsubitem 16, 47, \xref{502};
    \subsubitem 17, 50, \xref{582};
    \subsubitem 17, 53, \xref{550};
    \subsubitem 19, 57, \xref{513};
    \subsubitem 19, 58, \xref{346}, \xref{432};
    \subsubitem 20, 59, \xref{412}, \xref{423}, \xref{424};
    \subsubitem 22, 63, \xref{307}, \xref{502};
    \subsubitem 22, 64, \xref{310};
    \subsubitem 22, 65, \xref{619};
    \subsubitem 23, 68, \xref{443}, \xref{501}, \xref{586};
    \subsubitem 24, 69, \xref{433}, \xref{502}.

  \subitem \emph{prō Archiā Poētā} (Arch.),
    \subsubitem  1,  1, \xref{309}, \xref{360};
    \subsubitem  1,  2, \xref{276};
    \subsubitem  3,  4, \xref{302}, \xref{425}, \xref{452}, \xref{624};
    \subsubitem  3,  5, \xref{442};
    \subsubitem  4,  7, \xref{550};
    \subsubitem  5, 10, \xref{445}, \xref{550};
    \subsubitem  6, 14, \xref{347};
    \subsubitem  7, 15, \xref{503}, \xref{582};
    \subsubitem  7, 16, \xref{274};
    \subsubitem  8, 18, \xref{438}, \xref{513};
    \subsubitem  8, 19, \xref{341};
    \subsubitem  9, 19, \xref{479};
    \subsubitem  9, 20, \xref{276};
    \subsubitem 10, 16, \xref{365};
    \subsubitem 10, 26, \xref{396};
    \subsubitem 11, 26, \xref{278}, \xref{609};
    \subsubitem 11, 28, \xref{439}.

  \subitem \emph{prō Balbō} (Balb.),
    \subsubitem 10, 26, \xref{310};
    \subsubitem 12, 20, \xref{585}.

  \subitem \emph{prō Cluentiā} (Clu.),
    \subsubitem  5,  11, \xref{438};
    \subsubitem 16,  46, \xref{441};
    \subsubitem 23,  62, \xref{537};
    \subsubitem 26,  72, \xref{568};
    \subsubitem 29,  80, \xref{580};
    \subsubitem 41, 116, \xref{625};
    \subsubitem 49, 136, \xref{625};
    \subsubitem 51, 141, \xref{388};
    \subsubitem 55, 150, \xref{502};
    \subsubitem 57, 155, \xref{513}.

  \subitem \emph{prō Flaccō} (Flacc.),
    \subsubitem  2,  4, \xref{446};
    \subsubitem 15, 36, \xref{363};
    \subsubitem 33, 83, \xref{342};
    \subsubitem 37, 92, \xref{362}.

  \subitem \emph{prō Fonteiō} (Font.),
    \subsubitem  5, 12, \xref{385};
    \subsubitem 18, 41, \xref{434}, \xref{452}.

  \subitem \emph{prō Ligariō} (Lig.),
    \subsubitem  5, 14, \xref{501};
    \subsubitem  5, 16, \xref{507};
    \subsubitem 12, 34, \xref{507};
    \subsubitem 12, 37, \xref{346}.

  \subitem \emph{prō Marcellō} (Marc.),
    \subsubitem 6, 16, \xref{422};
    \subsubitem 7, 22, \xref{446};
    \subsubitem 8, 24, \xref{346}.

  \subitem \emph{prō Milone} (Mil.),
    \subsubitem  2,   4, \xref{604};
    \subsubitem  2,   5, \xref{425};
    \subsubitem  3,   8, \xref{608};
    \subsubitem  4,  10, \xref{625};
    \subsubitem  4,  11, \xref{350};
    \subsubitem  7,  19, \xref{438};
    \subsubitem 10,  27, \xref{451};
    \subsubitem 10,  28, \xref{519};
    \subsubitem 10,  29, \xref{362}, \xref{519};
    \subsubitem 11,  30, \xref{632};
    \subsubitem 11,  31, \xref{445};
    \subsubitem 12,  31, \xref{502};
    \subsubitem 12,  32, \xref{346};
    \subsubitem 13,  34, \xref{290};
    \subsubitem 13,  35, \xref{274}, \xref{515};
    \subsubitem 14,  37, \xref{478};
    \subsubitem 15,  39, \xref{275};
    \subsubitem 16,  43, \xref{275};
    \subsubitem 18,  47, \xref{423};
    \subsubitem 19,  49, \xref{275}, \xref{430};
    \subsubitem 19,  56, \xref{408};
    \subsubitem 21,  55, \xref{349};
    \subsubitem 21,  56, \xref{345};
    \subsubitem 21,  57, \xref{430};
    \subsubitem 22,  59, \xref{339}, \xref{380};
    \subsubitem 22,  60, \xref{502};
    \subsubitem 23,  61, \xref{450};
    \subsubitem 23,  62, \xref{350};
    \subsubitem 24,  65, \xref{594};
    \subsubitem 26,  69, \xref{521};
    \subsubitem 27,  72, \xref{521};
    \subsubitem 27,  73, \xref{290}, \xref{339};
    \subsubitem 27,  75, \xref{346};
    \subsubitem 28,  78, \xref{421}, \xref{521};
    \subsubitem 29,  78, \xref{301};
    \subsubitem 29,  79, \xref{496};
    \subsubitem 30,  82, \xref{340};
    \subsubitem 33,  91, \xref{339};
    \subsubitem 34,  92, \xref{339}, \xref{352}, \xref{434};
    \subsubitem 34,  93, \xref{511};
    \subsubitem 34,  94, \xref{399};
    \subsubitem 36,  99, \xref{553};
    \subsubitem 37, 101, \xref{513};
    \subsubitem 37, 102, \xref{438}.

  \subitem \emph{prō Murena} (Mur.),
    \subsubitem  1,  1, \xref{307};
    \subsubitem 16, 34, \xref{290}, \xref{450};
    \subsubitem 32, 67, \xref{342}.

  \subitem \emph{prō Planciō} (Planc.),
    \subsubitem 10,  25, \xref{444};
    \subsubitem 15,  37, \xref{301};
    \subsubitem 41,  99, \xref{377};
    \subsubitem 42, 101, \xref{350}.

  \subitem \emph{prō Provinciīs Consularibus} (Prov. Cons.),
    \subsubitem  8, 18, \xref{412}, \xref{511};
    \subsubitem 12, 30, \xref{411}.

  \subitem \emph{prō Quinctiō} (Quinct.),
    \subsubitem  1,  4, \xref{366};
    \subsubitem 11, 39, \xref{540};
    \subsubitem 17, 56, \xref{513};
    \subsubitem 24, 76, \xref{479}.

  \subitem \emph{prō Rabiriō Perduellionis Reō} (Rab. Perd.),
    \subsubitem 4, 12, \xref{343};
    \subsubitem 5, 16, \xref{442}.

  \subitem \emph{prō Rabiriō Postumō} (Rab. Post.),
    \subsubitem 7, 17, \xref{397}.

  \subitem \emph{prō S. Rosciō Amerinō} (Rosc. Am.),
    \subsubitem  7,  20, \xref{450};
    \subsubitem 14,  41, \xref{479};
    \subsubitem 29,  80, \xref{521};
    \subsubitem 29,  81, \xref{449};
    \subsubitem 32,  90, \xref{342};
    \subsubitem 33,  92, \xref{515};
    \subsubitem 33,  94, \xref{302};
    \subsubitem 33, 100, \xref{597};
    \subsubitem 34,  97, \xref{604};
    \subsubitem 46, 132, \xref{439}.

  \subitem \emph{} (Rosc. Com.),
    \subsubitem  7, 20, \xref{441};
    \subsubitem 12, 33, \xref{524};
    \subsubitem 13, 37, \xref{387}.

  \subitem \emph{prō Sestiō} (Sest.),
    \subsubitem 17,  39, \xref{393};
    \subsubitem 29,  62, \xref{519};
    \subsubitem 42,  92, \xref{502};
    \subsubitem 50, 106, \xref{390};
    \subsubitem 66, 138, \xref{605}.

  \subitem \emph{prō Sullā} (Sull.),
    \subsubitem  5,  17, \xref{308};
    \subsubitem 49, 136, \xref{624}.

  \subitem \emph{prō Tulliō} (Tull.),
    \subsubitem 15, 35, \xref{346}.

  \subitem \emph{Tusculanae Disputationēs} (Tusc.),
    \subsubitem 1,  3,   6, \xref{517};
    \subsubitem 1,  4,   7, \xref{419};
    \subsubitem 1, 10,  20, \xref{382};
    \subsubitem 1, 12,  26, \xref{354};
    \subsubitem 1, 27,  67, \xref{275};
    \subsubitem 1, 33,  78, \xref{619};
    \subsubitem 1, 36,  87, \xref{519};
    \subsubitem 1, 41,  99, \xref{585};
    \subsubitem 2,  5,  14, \xref{532};
    \subsubitem 2, 14,  33, \xref{582};
    \subsubitem 2, 23,  56, \xref{288};
    \subsubitem 3, 10,  21, \xref{597};
    \subsubitem 4,  1,   2, \xref{550};
    \subsubitem 4,  7,  14, \xref{248};
    \subsubitem 4, 16,  35, \xref{449};
    \subsubitem 5,  6,  15, \xref{604};
    \subsubitem 5,  7,  20, \xref{509};
    \subsubitem 5, 13,  38, \xref{619};
    \subsubitem 5, 15,  45, \xref{307};
    \subsubitem 5, 16,  48, \xref{250};
    \subsubitem 5, 19,  55, \xref{522};
    \subsubitem 5, 20,  57, \xref{387};
    \subsubitem 5, 38, 111, \xref{597};
    \subsubitem 5, 39, 113, \xref{241}.

\indexspace

\indexauthor{Ennius},
  \subitem \emph{Annales} (Enn. Ann.),
    \subsubitem 414, 4, \xref{507};
    \subsubitem 425, \xref{645}.

\indexspace

\indexauthor{Horace},
  \subitem \emph{Ars Poētica} (A. P.),
    \subsubitem  56, \xref{292};
    \subsubitem 212, \xref{348};
    \subsubitem 467, \xref{363}.

  \subitem \emph{Carmina} (Carm.),
    \subsubitem 1,  8,  1, \xref{631};
    \subsubitem 1, 11,  8, \xref{363};
    \subsubitem 1, 14,  1, \xref{632};
    \subsubitem 1, 17,  3, \xref{366};
    \subsubitem 1, 22,  1, \xref{354}, \xref{411};
    \subsubitem 1, 22, 23, \xref{396};
    \subsubitem 1, 24,  9, \xref{373};
    \subsubitem 1, 34,  2, \xref{632};
    \subsubitem 2,  9, 17, \xref{348};
    \subsubitem 2, 16, 13, \xref{369};
    \subsubitem 2, 19, 17, \xref{632};
    \subsubitem 3, 13, 13, \xref{346};
    \subsubitem 3, 19, 18, \xref{287};
    \subsubitem 3, 29, 52, \xref{364};
    \subsubitem 4,  4, 65, \xref{504}.

  \subitem \emph{Epistolae} (Ep.),
    \subsubitem 1,  1,  50, \xref{396};
    \subsubitem 1,  2,  56, \xref{250};
    \subsubitem 1,  5,  12, \xref{399};
    \subsubitem 1,  6,  37, \xref{319};
    \subsubitem 1, 10,   8, \xref{290};
    \subsubitem 1, 14,  30, \xref{639};
    \subsubitem 1, 16,  20, \xref{412};
    \subsubitem 1, 16,  31, \xref{496};
    \subsubitem 1, 16,  32, \xref{631};
    \subsubitem 1, 19,  22, \xref{579};
    \subsubitem 1, 19,  48, \xref{488};
    \subsubitem 2,  1,  50, \xref{624};
    \subsubitem 2,  2,  17, \xref{354};
    \subsubitem 2,  2, 125, \xref{396}.

  \subitem \emph{Epodī} (Epod.),
    \subsubitem  2, 1, \xref{405};
    \subsubitem 13, 2, \xref{656}.

  \subitem \emph{Sermōnes} (Sat.),
    \subsubitem 1,  1,   7, \xref{308};
    \subsubitem 1,  3,   3, \xref{580};
    \subsubitem 1,  3,  19, \xref{517};
    \subsubitem 1,  9,  23, \xref{394};
    \subsubitem 1, 10,  48, \xref{490};
    \subsubitem 1, 10,  72, \xref{502};
    \subsubitem 2,  1,  29, \xref{320};
    \subsubitem 2,  2,  12, \xref{645};
    \subsubitem 2,  3, 156, \xref{275}, \xref{427};
    \subsubitem 2,  3, 177, \xref{649};
    \subsubitem 2,  3, 295, \xref{231};
    \subsubitem 2,  5,  69, \xref{598};
    \subsubitem 2,  6,   3, \xref{435};
    \subsubitem 2,  6,  44, \xref{275};
    \subsubitem 2,  6,  83, \xref{348};
    \subsubitem 2,  7, 104, \xref{362}.

\indexspace

\indexauthor{Livy},
  \subitem \emph{ab Urbe Conditā},
    \subsubitem  Praef. 13, \xref{422};
    \subsubitem  1,  Praef. 9, \xref{569};
    \subsubitem  1,  1,  5, \xref{326};
    \subsubitem  1,  1,  8, \xref{384};
    \subsubitem  1,  3,  6, \xref{491};
    \subsubitem  1,  5,  5, \xref{405};
    \subsubitem  1,  7,  5, \xref{580};
    \subsubitem  1, 10,  4, \xref{604};
    \subsubitem  1, 14, 11, \xref{507};
    \subsubitem  1, 16,  3, \xref{305};
    \subsubitem  1, 17, 10, \xref{325};
    \subsubitem  1, 20,  4, \xref{366};
    \subsubitem  1, 21,  6, \xref{319};
    \subsubitem  1, 24,  7, \xref{401};
    \subsubitem  1, 26,  7, \xref{496};
    \subsubitem  1, 27,  9, \xref{523};
    \subsubitem  1, 28,  4, \xref{339};
    \subsubitem  1, 29,  4, \xref{326};
    \subsubitem  1, 32,  7, \xref{464};
    \subsubitem  1, 32, 10, \xref{496};
    \subsubitem  1, 32, 13, \xref{540};
    \subsubitem  1, 41,  1, \xref{335};
    \subsubitem  1, 41,  5, \xref{362};
    \subsubitem  1, 41,  6, \xref{433};
    \subsubitem  1, 43,  8, \xref{411};
    \subsubitem  1, 44,  2, \xref{612};
    \subsubitem  1, 46,  1, \xref{305};
    \subsubitem  1, 57,  7, \xref{545};
    \subsubitem  2,  8,  2, \xref{384};
    \subsubitem  2,  8,  3, \xref{364}, \xref{612};
    \subsubitem  2, 12,  5, \xref{582};
    \subsubitem  2, 36,  1, \xref{601};
    \subsubitem  2, 40,  7, \xref{541};
    \subsubitem  3, 55, 14, \xref{428};
    \subsubitem  4,  7, 10, \xref{392};
    \subsubitem  4, 19,  4, \xref{438};
    \subsubitem  4, 30,  4, \xref{619};
    \subsubitem  4, 52,  3, \xref{242};
    \subsubitem  4, 58,  2, \xref{478};
    \subsubitem  5, 33,  5, \xref{507};
    \subsubitem  5, 51,  9, \xref{449};
    \subsubitem  6, 14,  5, \xref{407};
    \subsubitem  8,  7,  6, \xref{502};
    \subsubitem  8, 11, 12, \xref{325};
    \subsubitem  9, 22,  4, \xref{377};
    \subsubitem  9, 30, 10, \xref{421};
    \subsubitem  9, 33,  7, \xref{581};
    \subsubitem 21,  1,  5, \xref{323};
    \subsubitem 21,  2,  6, \xref{608};
    \subsubitem 21,  4,  3, \xref{612};
    \subsubitem 21,  5, 15, \xref{406};
    \subsubitem 21,  6,  2, \xref{606};
    \subsubitem 21,  7, 10, \xref{389};
    \subsubitem 21, 10, 11, \xref{572};
    \subsubitem 21, 12,  4, \xref{629};
    \subsubitem 21, 16,  2, \xref{504};
    \subsubitem 21, 33,  2, \xref{566};
    \subsubitem 21, 35,  7, \xref{626};
    \subsubitem 21, 39,  1, \xref{507};
    \subsubitem 21, 41, 15, \xref{504};
    \subsubitem 21, 43,  2, \xref{571};
    \subsubitem 21, 44,  6, \xref{501};
    \subsubitem 21, 47,  5, \xref{517};
    \subsubitem 21, 50, 11, \xref{323};
    \subsubitem 21, 52,  4, \xref{529};
    \subsubitem 21, 54,  3, \xref{502};
    \subsubitem 21, 56,  9, \xref{426};
    \subsubitem 21, 60,  7, \xref{324};
    \subsubitem 22,  2,  1, \xref{612};
    \subsubitem 22,  2, 11, \xref{279}, \xref{441};
    \subsubitem 22, 10,  2, \xref{519};
    \subsubitem 22, 29,  2, \xref{289};
    \subsubitem 22, 32,  3, \xref{581};
    \subsubitem 22, 36,  1, \xref{519};
    \subsubitem 22, 50,  9, \xref{517};
    \subsubitem 22, 51,  3, \xref{521};
    \subsubitem 22, 54, 10, \xref{578};
    \subsubitem 22, 58,  1, \xref{453};
    \subsubitem 23,  1,  1, \xref{453};
    \subsubitem 24, 40, 17, \xref{426};
    \subsubitem 25, 31,  3, \xref{453};
    \subsubitem 27,  1, 12, \xref{631};
    \subsubitem 32, 12,  6, \xref{582};
    \subsubitem 34,  3,  2, \xref{373};
    \subsubitem 36, 27,  2, \xref{616};
    \subsubitem 36, 34,  6, \xref{541};
    \subsubitem 37, 45,  1, \xref{453};
    \subsubitem 38, 47,  4, \xref{581};
    \subsubitem 39, 22,  6, \xref{612};
    \subsubitem 39, 49,  8, \xref{288};
    \subsubitem 39, 50,  7, \xref{582};
    \subsubitem 40, 15, 14, \xref{505};
    \subsubitem 44, 40,  8, \xref{407};
    \subsubitem 45, 28, 11, \xref{325}.

\indexspace

\indexauthor{Martial},
  \subitem \emph{Epigrammata} (Mart.),
    \subsubitem 5, 38, 6, \xref{582}.

\indexspace

\indexauthor{Nepos},
  \subitem \emph{Agesilaus} (Nep. Ages.),
    \subsubitem 4, 1, \xref{538}.

  \subitem \emph{Atticus} (Nep. Att.),
    \subsubitem 8, 6, \xref{360}.

  \subitem \emph{Eumenes} (Nep. Eum.),
    \subsubitem 1, 1, \xref{415};
    \subsubitem 8, 7, \xref{393}.

  \subitem \emph{Hannibal} (Nep. Hann.),
    \subsubitem 2, 4, \xref{559};
    \subsubitem 7, 4, \xref{449}, \xref{484}.

  \subitem \emph{Pausanias} (Nep. Paus.),
    \subsubitem 2, 3, \xref{530}.

  \subitem \emph{Themistocles} (Nep. Them.),
    \subsubitem 8, 2, \xref{262}.

  \subitem \emph{Thrasybulus} (Nep. Thras.),
    \subsubitem 2, 7, \xref{524}.

  \subitem \emph{Tomoleon} (Nep. Timol.),
    \subsubitem 5, 3, \xref{343}.

\indexspace

\indexauthor{Ovid},
  \subitem \emph{Ars Amatoria} (Ov. A. A.),
    \subsubitem 3, 129, \xref{464}.

  \subitem \emph{Epistulae (Heroides)} (Ov. Her.),
    \subsubitem 10, 77, \xref{511}.

  \subitem \emph{Metamorphoses} (Ov. Met.),
    \subsubitem  3, 654, \xref{485};
    \subsubitem  5, 192, \xref{406};
    \subsubitem  6, 195, \xref{521};
    \subsubitem  8,  76, \xref{519};
    \subsubitem 10, 536, \xref{407};
    \subsubitem 14, 819, \xref{438}.

  \subitem \emph{Epistulae ex Pontō} (Ov. Pont.),
    \subsubitem 2, 7,  9, \xref{406};
    \subsubitem 2, 9, 47–48, \xref{552}.

  \subitem \emph{Tristia} (Ov. Trist.),
    \subsubitem 4, 10, 25–26, \xref{642}.

\indexspace

\indexauthor{Persius},
  \subitem \emph{Saturae} (Persius),
    \subsubitem 6, 38, \xref{598}.

\indexspace

\indexauthor{Plautus},
  \subitem \emph{Amphitruō} (Amph.),
    \subsubitem 240, \xref{507};
    \subsubitem 377, \xref{388};
    \subsubitem 559, \xref{502};
    \subsubitem 642, \xref{553}.

  \subitem \emph{Asināria} (As.),
    \subsubitem 188, \xref{581}.

  \subitem \emph{Aululāria} (Aul.),
    \subsubitem 174, \xref{537};
    \subsubitem 186, \xref{406};
    \subsubitem 196, \xref{362}.

  \subitem \emph{Bacchides} (Bacch.),
    \subsubitem    2, \xref{440};
    \subsubitem  731, \xref{503};
    \subsubitem  989, \emph{a}, \xref{502};
    \subsubitem 1023, \xref{497}.

  \subitem \emph{Captīvī} (Capt.),
    \subsubitem 312, \xref{285};
    \subsubitem 334, \xref{234};
    \subsubitem 628, \xref{232};
    \subsubitem 646, \xref{443};
    \subsubitem 866, \xref{366}.

  \subitem \emph{Menaechmī} (Men.),
    \subsubitem   3, \xref{376};
    \subsubitem 221, \xref{464};
    \subsubitem 295, \xref{511};
    \subsubitem 859, \xref{407}.

  \subitem \emph{Mercātor} (Merc.),
    \subsubitem 356, \xref{289};
    \subsubitem 770, \xref{496}.

  \subitem \emph{Mīles Glōriōsus} (Mil. Gl.),
    \subsubitem  426, \xref{513};
    \subsubitem  646, \xref{388};
    \subsubitem  963, \xref{352};
    \subsubitem 1140, \xref{513};
    \subsubitem 1158, \xref{388}.

  \subitem \emph{Persa} (Pers.),
    \subsubitem 211, \xref{276};
    \subsubitem 835, \xref{408}.

  \subitem \emph{Poenulus} (Poen.),
    \subsubitem 1095, \xref{391}.

  \subitem \emph{Pseudolus} (Pseud.),
    \subsubitem 554, \xref{507}.

  \subitem \emph{Rudēns} (Rud.),
    \subsubitem    5, \xref{326};
    \subsubitem  247, \xref{348};
    \subsubitem  379, \xref{582};
    \subsubitem  397, \xref{388};
    \subsubitem  564, \xref{275};
    \subsubitem  825, \xref{363};
    \subsubitem  833, \xref{502};
    \subsubitem  962, \xref{414};
    \subsubitem 1011, \xref{231};
    \subsubitem 1029, \xref{497};
    \subsubitem 1146, \xref{283}.

  \subitem \emph{Stichus} (Stich.),
    \subsubitem  37, \xref{501};
    \subsubitem 132, \xref{421}.

  \subitem \emph{Trinummus} (Trin.),
    \subsubitem  105, \xref{502};
    \subsubitem  133, \xref{232}, \xref{513};
    \subsubitem  148, \xref{582};
    \subsubitem  321, \xref{360};
    \subsubitem  384, \xref{496};
    \subsubitem  496, \xref{542};
    \subsubitem  515, \xref{490};
    \subsubitem  549, \xref{286};
    \subsubitem  587–588, \xref{513};
    \subsubitem  679, \xref{286};
    \subsubitem  748, \xref{501};
    \subsubitem  925, \xref{355};
    \subsubitem 1017, \xref{355};
    \subsubitem 1136, \xref{501}.

\indexspace

\indexauthor{Pliny the Elder},
  \subitem \emph{Naturalis Historia} (Plin. N. H.),
    \subsubitem  2, 57, \xref{319};
    \subsubitem  5, 14, \xref{380};
    \subsubitem  7, 51, \xref{401};
    \subsubitem 14, 11, \xref{507};
    \subsubitem 31, 59, \xref{612}.

\indexspace

\indexauthor{Pliny the Younger},
  \subitem \emph{Epistulae} (Plin. Ep.),
    \subsubitem  1,  9,  3, \xref{504};
    \subsubitem  1, 10,  3, \xref{511};
    \subsubitem  1, 20,  6, \xref{521};
    \subsubitem  4,  6,  2, \xref{268};
    \subsubitem  4,  8,  1, \xref{535};
    \subsubitem  5,  1,  9, \xref{673};
    \subsubitem  5,  1, 12, \xref{481};
    \subsubitem  6,  2,  2, \xref{362};
    \subsubitem  8,  9,  2, \xref{513};
    \subsubitem  9, 13, 16, \xref{535};
    \subsubitem  9, 24,  1, \xref{551};
    \subsubitem  9, 33,  6, \xref{507};
    \subsubitem 10, 15,     \xref{541};
    \subsubitem 10, 97,     \xref{582}.

\indexspace

\indexauthor{Quintilian},
  \subitem \emph{Institutio Oratoria} (Quintil.),
    \subsubitem  1,  3, 11, \xref{612};
    \subsubitem  1,  5,  7, \xref{351};
    \subsubitem  1,  5, 50, \xref{580};
    \subsubitem  2,  4, 21, \xref{264};
    \subsubitem  4,  5, 13, \xref{579};
    \subsubitem  6, Pr.  4, \xref{339};
    \subsubitem  6,  1, 35, \xref{438};
    \subsubitem  7, 10, 14, \xref{507};
    \subsubitem  8,  6, 44, \xref{632};
    \subsubitem 10,  1, 96, \xref{598};
    \subsubitem 10,  7, 29, \xref{445};
    \subsubitem 11,  2,  1, \xref{612};
    \subsubitem 11,  2, 12, \xref{384}.

\indexspace

\indexauthor{Sallust},
  \subitem \emph{Catilina} (Sall. Cat.),
    \subsubitem  2,  7, \xref{396};
    \subsubitem  2,  9, \xref{438};
    \subsubitem  6,  3, \xref{558};
    \subsubitem  6,  7, \xref{616};
    \subsubitem  7,  3, \xref{291};
    \subsubitem  9,  3, \xref{309};
    \subsubitem 10,  5, \xref{415};
    \subsubitem 11,  8, \xref{505};
    \subsubitem 12,  5, \xref{371};
    \subsubitem 13,  3, \xref{289};
    \subsubitem 18,  3, \xref{342};
    \subsubitem 20,  4, \xref{326};
    \subsubitem 21,  4, \xref{351};
    \subsubitem 31,  7, \xref{430};
    \subsubitem 32,  1, \xref{363};
    \subsubitem 33,  2, \xref{425};
    \subsubitem 43,  1, \xref{331};
    \subsubitem 43,  4, \xref{307};
    \subsubitem 47,  1, \xref{422};
    \subsubitem 47,  2, \xref{353};
    \subsubitem 48,  5, \xref{439};
    \subsubitem 50,  4, \xref{393};
    \subsubitem 58, 15, \xref{431};
    \subsubitem 61,  2, \xref{612}.

  \subitem \emph{Iugurtha} (Sall. Iug.),
    \subsubitem  14, 22, \xref{290};
    \subsubitem  33,  4, \xref{449};
    \subsubitem  41,  2, \xref{354};
    \subsubitem  54,  5, \xref{445};
    \subsubitem  58,  2, \xref{325};
    \subsubitem  59,  3, \xref{581};
    \subsubitem  61,  1, \xref{453};
    \subsubitem  64,  5, \xref{434};
    \subsubitem  68,  1, \xref{323};
    \subsubitem 102,  1, \xref{295};
    \subsubitem 113,  1, \xref{485}.

\indexspace

\indexauthor{Seneca},
  \subitem \emph{Medea} (Sen. Med.),
    \subsubitem 375, \xref{507}.

\indexspace

\indexauthor{Tacitus},
  \subitem \emph{Agricola} (Tac. Agric.),
    \subsubitem 42, \xref{509}.

  \subitem \emph{Annales} (Tac. Ann.),
    \subsubitem  1, 66, \xref{363};
    \subsubitem  2, 64, \xref{361}, \xref{395};
    \subsubitem  2, 71, \xref{579};
    \subsubitem  6,  9, \xref{389};
    \subsubitem 12,  3, \xref{389};
    \subsubitem 14, 14, \xref{540};
    \subsubitem 16, 35, \xref{521};
    \subsubitem 15, 60, \xref{438}.

  \subitem \emph{Historiae} (Tac. Hist.),
    \subsubitem  3, 63, \xref{309}.

\indexspace

\indexauthor{Terence},
  \subitem \emph{Adelphī} (Ad.),
    \subsubitem 104, \xref{329};
    \subsubitem 123, \xref{517};
    \subsubitem 905, \xref{486}.

  \subitem \emph{Andria} (And.),
    \subsubitem  45, \xref{519};
    \subsubitem  64, \xref{362};
    \subsubitem 555, \xref{332};
    \subsubitem 582, \xref{287};
    \subsubitem 792, \xref{515}.

  \subitem \emph{Eunūchus} (Eun.),
    \subsubitem 197, \xref{399};
    \subsubitem 216, \xref{450};
    \subsubitem 252, \xref{545};
    \subsubitem 510, \xref{276};
    \subsubitem 559, \xref{399};
    \subsubitem 728, \xref{329}.

  \subitem \emph{Heauton Timorumenos} (Heaut.),
    \subsubitem  572, \xref{497};
    \subsubitem 1067, \xref{519}.

  \subitem \emph{Hecyra} (Hec.),
    \subsubitem 191, \xref{451};
    \subsubitem 287, \xref{631};
    \subsubitem 369, \xref{555};
    \subsubitem 646, \xref{393};
    \subsubitem 801, \xref{355}.

  \subitem \emph{Phormiō} (Ph.),
    \subsubitem  102, \xref{231};
    \subsubitem  123, \xref{511};
    \subsubitem  137, \xref{423};
    \subsubitem  147, \xref{234};
    \subsubitem  168, \xref{238};
    \subsubitem  188, \xref{287};
    \subsubitem  223, \xref{503};
    \subsubitem  254, \xref{258};
    \subsubitem  275, \xref{258};
    \subsubitem  287, \xref{632};
    \subsubitem  291, \xref{434};
    \subsubitem  388, \xref{504};
    \subsubitem  401, \xref{258};
    \subsubitem  422, \xref{257};
    \subsubitem  449, \xref{519};
    \subsubitem  454, \xref{229};
    \subsubitem  480, \xref{507};
    \subsubitem  486, \xref{496};
    \subsubitem  525, \xref{232};
    \subsubitem  527, \xref{295};
    \subsubitem  553, \xref{582};
    \subsubitem  565, \xref{519};
    \subsubitem  594, \xref{492};
    \subsubitem  677, \xref{530};
    \subsubitem  685–686, \xref{258};
    \subsubitem  713, \xref{257};
    \subsubitem  792, \xref{519};
    \subsubitem  800, \xref{265};
    \subsubitem  801, \xref{486};
    \subsubitem  803, \xref{502};
    \subsubitem  810, \xref{531};
    \subsubitem  813, \xref{232};
    \subsubitem  819, \xref{502};
    \subsubitem  882, \xref{490};
    \subsubitem 1000, \xref{503}.

\indexspace

\indexauthor{Varro},
  \subitem \emph{Menippeae} (Varro, Sat. Men.),
    \subsubitem 333, \xref{507}.

\indexspace

\indexauthor{Virgil},
  \subitem \emph{Aenēis} (Aen.),
    \subsubitem  1,   2, \xref{385};
    \subsubitem  1,   5, \xref{644};
    \subsubitem  1,   6, \xref{651};
    \subsubitem  1,   8, \xref{652};
    \subsubitem  1,  10, \xref{641};
    \subsubitem  1,  12 and 13, \xref{641};
    \subsubitem  1,  13, \xref{639};
    \subsubitem  1,  15, \xref{417};
    \subsubitem  1,  19, \xref{311};
    \subsubitem  1,  20, \xref{507};
    \subsubitem  1,  21, \xref{295}, \xref{631};
    \subsubitem  1,  26, \xref{650};
    \subsubitem  1,  35, \xref{632};
    \subsubitem  1,  37, \xref{596};
    \subsubitem  1,  41, \xref{653};
    \subsubitem  1,  45, \xref{641};
    \subsubitem  1,  46, \xref{641};
    \subsubitem  1,  47, \xref{387};
    \subsubitem  1,  47, \xref{640};
    \subsubitem  1,  55, \xref{632};
    \subsubitem  1,  61, \xref{631};
    \subsubitem  1,  69, \xref{631};
    \subsubitem  1,  71, \xref{374};
    \subsubitem  1,  73, \xref{656};
    \subsubitem  1,  76, \xref{631};
    \subsubitem  1,  79, \xref{598};
    \subsubitem  1,  90, \xref{492};
    \subsubitem  1,  97, \xref{641};
    \subsubitem  1,  99, \xref{423};
    \subsubitem  1, 102, \xref{369};
    \subsubitem  1, 105, \xref{445};
    \subsubitem  1, 105, \xref{644};
    \subsubitem  1, 115, \xref{641};
    \subsubitem  1, 120, \xref{658};
    \subsubitem  1, 123, \xref{445};
    \subsubitem  1, 126, \xref{426};
    \subsubitem  1, 130, \xref{391};
    \subsubitem  1, 135, \xref{632};
    \subsubitem  1, 137, \xref{365};
    \subsubitem  1, 148, \xref{632};
    \subsubitem  1, 174, \xref{371};
    \subsubitem  1, 178, \xref{354};
    \subsubitem  1, 181, \xref{426};
    \subsubitem  1, 192, \xref{507};
    \subsubitem  1, 195, \xref{375};
    \subsubitem  1, 199, \xref{641};
    \subsubitem  1, 202, \xref{641};
    \subsubitem  1, 208, \xref{444};
    \subsubitem  1, 215, \xref{347};
    \subsubitem  1, 224, \xref{632};
    \subsubitem  1, 263, \xref{433};
    \subsubitem  1, 273, \xref{507};
    \subsubitem  1, 283, \xref{507};
    \subsubitem  1, 286, \xref{507};
    \subsubitem  1, 295, \xref{383};
    \subsubitem  1, 300, \xref{426};
    \subsubitem  1, 304, \xref{364};
    \subsubitem  1, 308, \xref{652};
    \subsubitem  1, 320, \xref{389};
    \subsubitem  1, 325, \xref{222};
    \subsubitem  1, 328, \xref{396};
    \subsubitem  1, 330, \xref{283}, \xref{530};
    \subsubitem  1, 332, \xref{641};
    \subsubitem  1, 335, \xref{442};
    \subsubitem  1, 347, \xref{416};
    \subsubitem  1, 357, \xref{598};
    \subsubitem  1, 385, \xref{396};
    \subsubitem  1, 405, \xref{647};
    \subsubitem  1, 422, \xref{357};
    \subsubitem  1, 439, \xref{363};
    \subsubitem  1, 440, \xref{373};
    \subsubitem  1, 458, \xref{363};
    \subsubitem  1, 461, \xref{335};
    \subsubitem  1, 477, \xref{368};
    \subsubitem  1, 478, \xref{654};
    \subsubitem  1, 481, \xref{390}, \xref{601};
    \subsubitem  1, 524, \xref{391};
    \subsubitem  1, 527, \xref{598};
    \subsubitem  1, 535, \xref{410};
    \subsubitem  1, 544, \xref{416};
    \subsubitem  1, 565, \xref{341};
    \subsubitem  1, 573, \xref{326};
    \subsubitem  1, 610, \xref{631};
    \subsubitem  1, 614, \xref{631};
    \subsubitem  1, 617, \xref{647};
    \subsubitem  1, 630, \xref{632};
    \subsubitem  1, 651, \xref{652};
    \subsubitem  1, 654, \xref{366};
    \subsubitem  1, 664, \xref{401};
    \subsubitem  1, 668, \xref{654};
    \subsubitem  1, 689, \xref{310};
    \subsubitem  1, 696, \xref{333};
    \subsubitem  1, 713, \xref{612};
    \subsubitem  1, 737, \xref{421};
    \subsubitem  2,   6, \xref{612};
    \subsubitem  2,  15, \xref{339};
    \subsubitem  2,  16, \xref{656};
    \subsubitem  2,  31, \xref{330}, \xref{331};
    \subsubitem  2,  33, \xref{598};
    \subsubitem  2,  36, \xref{375};
    \subsubitem  2,  38, \xref{655};
    \subsubitem  2,  57, \xref{390};
    \subsubitem  2,  68, \xref{639};
    \subsubitem  2,  74, \xref{413};
    \subsubitem  2,  86, \xref{392};
    \subsubitem  2, 100, \xref{311};
    \subsubitem  2, 114, \xref{606};
    \subsubitem  2, 129, \xref{394};
    \subsubitem  2, 144, \xref{352};
    \subsubitem  2, 160, \xref{433}, \xref{438};
    \subsubitem  2, 180, \xref{552};
    \subsubitem  2, 247, \xref{292};
    \subsubitem  2, 258, \xref{631};
    \subsubitem  2, 323, \xref{566};
    \subsubitem  2, 325, \xref{489};
    \subsubitem  2, 332, \xref{357};
    \subsubitem  2, 333, \xref{632};
    \subsubitem  2, 334, \xref{384};
    \subsubitem  2, 353, \xref{631};
    \subsubitem  2, 361, \xref{517};
    \subsubitem  2, 377, \xref{592};
    \subsubitem  2, 392, \xref{390};
    \subsubitem  2, 408, \xref{607};
    \subsubitem  2, 413, \xref{354}, \xref{608};
    \subsubitem  2, 444, \xref{601};
    \subsubitem  2, 509, \xref{376};
    \subsubitem  2, 547, \xref{572};
    \subsubitem  2, 553, \xref{375}, \xref{407};
    \subsubitem  2, 563, \xref{654};
    \subsubitem  2, 567, \xref{659};
    \subsubitem  2, 643, \xref{377};
    \subsubitem  2, 662, \xref{302};
    \subsubitem  2, 669–670, \xref{572};
    \subsubitem  2, 675, \xref{607};
    \subsubitem  2, 721, \xref{288};
    \subsubitem  2, 730, \xref{363};
    \subsubitem  2, 774, \xref{652};
    \subsubitem  2, 786, \xref{618};
    \subsubitem  3,   1, \xref{657};
    \subsubitem  3,   7, \xref{537};
    \subsubitem  3,  13, \xref{290};
    \subsubitem  3,  39, \xref{503};
    \subsubitem  3,  47, \xref{389};
    \subsubitem  3,  61, \xref{631};
    \subsubitem  3,  63, \xref{361};
    \subsubitem  3,  84, \xref{406};
    \subsubitem  3,  85, \xref{607};
    \subsubitem  3,  91, \xref{654};
    \subsubitem  3, 162, \xref{449};
    \subsubitem  3, 224, \xref{315}, \xref{429};
    \subsubitem  3, 305, \xref{361};
    \subsubitem  3, 317, \xref{410};
    \subsubitem  3, 319, \xref{231}, \xref{339};
    \subsubitem  3, 327, \xref{422};
    \subsubitem  3, 349, \xref{363};
    \subsubitem  3, 464, \xref{654};
    \subsubitem  3, 465, \xref{375};
    \subsubitem  3, 477, \xref{372};
    \subsubitem  3, 533, \xref{406};
    \subsubitem  3, 594, \xref{389};
    \subsubitem  3, 633, \xref{431};
    \subsubitem  3, 658, \xref{639};
    \subsubitem  3, 678, \xref{375};
    \subsubitem  3, 688, \xref{386};
    \subsubitem  4,  24, \xref{571};
    \subsubitem  4,  36, \xref{410};
    \subsubitem  4,  38, \xref{363};
    \subsubitem  4,  73, \xref{363};
    \subsubitem  4,  89, \xref{363};
    \subsubitem  4,  99, \xref{545};
    \subsubitem  4, 110, \xref{582};
    \subsubitem  4, 157, \xref{274};
    \subsubitem  4, 165, \xref{385};
    \subsubitem  4, 223, \xref{496};
    \subsubitem  4, 235, \xref{647};
    \subsubitem  4, 314, \xref{627};
    \subsubitem  4, 320, \xref{624};
    \subsubitem  4, 421, \xref{595};
    \subsubitem  4, 451, \xref{352};
    \subsubitem  4, 457, \xref{406};
    \subsubitem  4, 467, \xref{396};
    \subsubitem  4, 534, \xref{571};
    \subsubitem  4, 545, \xref{572};
    \subsubitem  4, 547, \xref{496};
    \subsubitem  4, 564, \xref{598};
    \subsubitem  4, 565, \xref{598};
    \subsubitem  4, 569, \xref{325};
    \subsubitem  4, 576, \xref{346};
    \subsubitem  5,   6, \xref{252};
    \subsubitem  5,  15, \xref{377};
    \subsubitem  5,  42, \xref{527};
    \subsubitem  5,  98, \xref{446};
    \subsubitem  5, 127, \xref{447};
    \subsubitem  5, 198, \xref{632};
    \subsubitem  5, 202, \xref{449};
    \subsubitem  5, 216, \xref{426};
    \subsubitem  5, 237, \xref{343};
    \subsubitem  5, 260, \xref{598};
    \subsubitem  5, 261, \xref{648};
    \subsubitem  5, 265, \xref{445};
    \subsubitem  5, 285, \xref{389};
    \subsubitem  5, 291, \xref{536};
    \subsubitem  5, 319, \xref{632};
    \subsubitem  5, 325, \xref{491};
    \subsubitem  5, 357, \xref{418};
    \subsubitem  5, 414, \xref{363};
    \subsubitem  5, 434, \xref{377};
    \subsubitem  5, 451, \xref{375};
    \subsubitem  5, 542, \xref{541};
    \subsubitem  5, 559, \xref{349};
    \subsubitem  5, 603, \xref{659};
    \subsubitem  5, 626, \xref{550};
    \subsubitem  5, 662, \xref{632};
    \subsubitem  5, 669, \xref{307};
    \subsubitem  5, 702, \xref{234};
    \subsubitem  5, 716, \xref{325};
    \subsubitem  5, 728, \xref{284};
    \subsubitem  5, 845, \xref{371};
    \subsubitem  6,  31, \xref{504};
    \subsubitem  6,  50, \xref{396};
    \subsubitem  6, 159, \xref{422};
    \subsubitem  6, 165, \xref{598};
    \subsubitem  6, 173, \xref{598};
    \subsubitem  6, 187, \xref{582};
    \subsubitem  6, 196, \xref{363};
    \subsubitem  6, 244, \xref{377};
    \subsubitem  6, 351, \xref{391};
    \subsubitem  6, 358, \xref{581};
    \subsubitem  6, 400, \xref{410};
    \subsubitem  6, 595, \xref{598};
    \subsubitem  6, 621, \xref{427};
    \subsubitem  6, 670, \xref{339};
    \subsubitem  6, 743, \xref{319};
    \subsubitem  6, 779, \xref{537};
    \subsubitem  7,  48, \xref{363};
    \subsubitem  7,  98, \xref{507};
    \subsubitem  7, 145, \xref{507};
    \subsubitem  7, 490, \xref{363};
    \subsubitem  8,  98, \xref{654};
    \subsubitem  8, 363, \xref{652};
    \subsubitem  8, 441, \xref{430};
    \subsubitem  8, 596, \xref{639};
    \subsubitem  8, 643, \xref{632};
    \subsubitem  9,   7, \xref{600};
    \subsubitem  9,  61, \xref{435};
    \subsubitem  9, 115, \xref{507};
    \subsubitem  9, 240, \xref{618};
    \subsubitem  9, 366, \xref{390};
    \subsubitem  9, 427, \xref{328};
    \subsubitem  9, 514, \xref{382};
    \subsubitem  9, 724, \xref{601};
    \subsubitem  9, 794, \xref{396};
    \subsubitem 10,  18, \xref{647};
    \subsubitem 10, 361, \xref{363};
    \subsubitem 10, 566, \xref{247};
    \subsubitem 10, 663, \xref{655};
    \subsubitem 11,  87, \xref{449};
    \subsubitem 11, 126, \xref{352};
    \subsubitem 11, 162, \xref{511};
    \subsubitem 11, 702, \xref{371};
    \subsubitem 12, 409, \xref{375};
    \subsubitem 12, 649, \xref{354}.

  \subitem \emph{Eclogae} (Ecl.),
    \subsubitem  1, 81, \xref{435};
    \subsubitem  2,  1, \xref{391};
    \subsubitem  3,  1, \xref{237}, \xref{275};
    \subsubitem  3,  1, \xref{641};
    \subsubitem  3, 48, \xref{513};
    \subsubitem  7, 23, \xref{654};
    \subsubitem  9, 66, \xref{652};
    \subsubitem 10,  5, \xref{464};
    \subsubitem 10, 69, \xref{652}.

  \subitem \emph{Geōrgica} (Georg.),
    \subsubitem 2, 132, \xref{582}.

\end{autindex}

\endinput


%% This work is licensed under a Creative Commons
%% Attribution-NonCommercial 4.0 International License.
%% http://creativecommons.org/licenses/by-nc/4.0/
%% 
%% David M. Jones, July 2016

\Unnumbered{Emendations to the Text}
\label{emendations}
\markboth{Emendations to the Text}{Emendations to the Text}
\markthird{}

\thispagestyle{dropfolio}

\contentsentry{B}{Emendations to the Text}

The first set of emendations requires little explanation, since they
merely correct minor typos, mostly inconsistencies in usage,
punctuation, or capitalization.  The second is a list of places where
I have altered section or line numbers of cited passages to reconcile
them with the texts in the Perseus Digital Library.

\medskip

\begin{emendations}
0. && \textbf{Original}  & \textbf{Emended} \\[\smallskipamount]
\endhead

70. & \erefpage{70}, l.~13
    & \textsc{gained}
    & \textsc{Gained}
\\

71. & \erefpage{71}, l.~13
    & \textsc{various}
    & \textsc{Various}
\\

73. & \erefpage{73}, l.~7
    & \textsc{noticed}
    & \textsc{Noticed}
\\

1.  & \erefpage{1}, l.~3% p.~\pageref{authors_cited}, l.~3
    & Sallust, 86–34.
    & Sallust, 86–34
\\

74. & \eref[4]{74}, l.~7
    & ;
    & ,
\\

2.  & \eref[\emph{a}]{2}, l.~1
    & lx
    & -lx
\\

3.  & \eref[\emph{b}]{3}, l.~2
    & (-ium)
    & (-ium),
\\

4.  & \eref{4}, l.~10
    & hērō\ending{i}
    & hērō\ending{ī}
\\

5.  & \eref[\emph{a}]{5}, l.~3
    & Supine.
    & Supine
\\

6. & \eref[n.]{6}, l.~5
    & p.~80
    & p.~76 (p.~73 in this edition)
\\

75. & \eref[B.]{75}, l.~8
    & -ui
    & -uī
\\

7. & \xref{197}, l.~−1%\eref{7}
    & \latin{Do}, \emph{give}
    & \latin{Dō}, \emph{give}
\\

8. & \eref[5]{8}, l.~1
    & from
    & form
\\

76. & \eref{76}, l.~??
    & derived
    & Derived
\\

77. & \eref{77}, l.~??
    & derived
    & Derived
\\

80. & \eref{80}, l.~4
    & Adverb of time
    & Adverb of Time
\\

9. & \eref[2]{9}, l.~2
    & Cæsar
    & Caesar
\\

11. & \eref{11}, l.~6
    & does n't
    & doesn't
\\

13. & \eref[1]{13}, l.~3
    & Cæsar
    & Caesar
\\

14. & \eref{14}, l.~7
    & \latin{vel\ellipsis vel\dots}
    & \latin{vel\ellipsis vel\dots},
\\

21. & \eref[ftn, (\emph{b})]{21}, l.~1
    & verb,
    & verb
\\

22. & \eref[ftn, (\emph{b})]{22}, l.~
    & \latin{crēdō}, \latin{dēsum},
    & \latin{crēdō},
\\

81. & \eref[\emph{b}]{81}, l.~2
    & object
    & Direct Object
\\

23. & \eref[\emph{b}]{23}, l.~2
    & means
    & Means
\\

29. & \eref[1]{29}, l.~1
    & \latin{cīvitātībus}
    & \latin{cīvitātibus}
\\

82. & \eref[c]{82}, l.~3
    & ;
    & ,
\\

34. & \eref{34}
    & \emph{purpose}).
    & \emph{purpose}.)
\\

37. & \eref[3\emph{a}, ftn.]{37}, l.~1
    & optimum est:
    & optimum est;
\\

39. & \eref[4]{39}, l.~1
    & \latin{ut} \textbf{or} \latin{ut}
    & \latin{ut} or \latin{ut}
\\

40. & \eref[1]{40}, l.~11
    & ‘That
    & “That
\\

41. & \eref[1]{41}, l.~11
    & is’
    & is”
\\

78. & \eref{78}, l.~??
    & developed
    & Developed
\\

79. & \eref{79}, l.~??
    & developed
    & Developed
\\

83. & \eref{83}, l.~4
    &
    & .
\\

84. & \eref{84}, l.~4
    & .
    &
\\

47. & \eref[ftn.~1]{47}, l.~5
    & \latin{foveō} \latin{invītō}
    & \latin{foveō}, \latin{invītō}
\\

48. & \eref[ftn.~1]{48}, l.~2
    & Participle”
    & Participle,”
\\

85. & \eref{85}, l.~1
    & motion
    & Motion
\\

52. & \eref{52}
    & quantity\tsup{1})
    & quantity\tsup{1}).
\\

53. & \eref{53}, l.~1
    & (marked~∥)
    & (marked~∥),
\\

54. & \eref[2]{54}, l.~3
    & \latin{tertia pars}\footnotemark[1]
    & \latin{tertia pars}
\\

55. & \eref[4\emph{a}]{55}
    & \latin{Caecilianus}
    & \latin{Caeciliānus}
\\

72. & \erefpage{72}, l.~21
    & \textsc{singular indefinite}
    & \textsc{singular person indefinite}
\\

56. & \erefpage{56}, \latin{faciō}, l.~5
    & \textbf{218}, 3,
    & \textbf{218}, 3.
\\

57. & \erefpage{57}, \latin{ferō}, l.~12
    & ob-lātum.
    & ob-lātum;
\\

58. & \erefpage{58}, \latin{facilis}, l.~2
    & -u
    & -ū
\\

59. & \erefpage{59}, \latin{fīdō}
    & w. abl.
    & w. abl.,
\\

60. & \erefpage{60}, Genitive, l.~10
    & \ending{-ūm}
    & \ending{-um}
\\

61. & \erefpage{61}, \latin{īdem}, l.~2
    & b.
    & b
\\

62. & \erefpage{62}, \latin{nisi}, l.~1
    & nisi si
    & nisi sī
\\

63. & \erefpage{63}, Obligation or \dots, l.~2
    & \textbf{513}
    & \textbf{513}.
\\

64. & \erefpage{64}, \latin{post}, l.~2
    & \latin{ante}
    & \latin{ante}.
\\

65. & \erefpage{65}, \latin{postquam}, l.~3
    & \emph{c}.
    & \emph{c}
\\

66. & \erefpage{66}, Predicate, l.~5
    & \textbf{318}–\textbf{332},
    & \textbf{318}–\textbf{332};
\\

67. & \erefpage{67}, \latin{ut}, l.~6
    & subst. cl.
    & subst. cl.,
\\

68. & \erefpage{68}, \latin{uterque}, l.~3
    & w. gen.
    & w. gen.,
\\

69. & \erefpage{69}, Vocative, l.~3
    & \emph{Syntax},
    & \emph{Syntax}:
\\[\bigskipamount]

\M{4}{@{}l}{\large\textbf{Emendations to citations}}\\[\medskipamount]

134. & \eref{134}
     & B.~G. 2,  15, 5
     & B.~G. 2,  14, 5
\\

135. & \eref{135}
     & Cat. 1,  16, 15
     & Cat. 1,  6, 15
\\

136. & \eref{136}
     & B.~G. 2,  30, 4
     & B.~G. 2,  29, 4
\\

137. & \eref{137}
     & B.~G. 3, 9,  9 
     & B.~G. 3, 9,  9–10 
\\

101. & \eref{101}
     & B.~G. 2,  24, 1
     & B.~G. 2,  23, 1
\\

125. & \eref{125}
     & B.~G. 3,  18, 8
     & B.~G. 3,  16, 8
\\

100. & \eref{100}
     & B.~G. 2,  25, 1
     & B.~G. 2,  24, 1
\\

102. & \eref{102}
     & B.~G. 2,  25, 1
     & B.~G. 2,  24, 1
\\

138. & \eref{138}
     & Cat. 3,  13, 27
     & Cat. 3,  12, 27
\\

139. & \eref{139}
     & Ph.  685 
     & Ph.  685–686 
\\

140. & \eref{140}
     & B.~G. 4, 1,  5 
     & B.~G. 4, 1,  4 
\\

141. & \eref{141}
     & B.~G. 2,  15, 2
     & B.~G. 2,  14, 2
\\

142. & \eref{142}
     & B.~G. 1, 15,  2 
     & B.~G. 1, 15,  1 
\\

103. & \eref{103}
     & B.~G. 2,  25, 1
     & B.~G. 2,  24, 1
\\

143. & \eref{143}
     & B.~G. 2,  23, 3
     & B.~G. 2,  22, 3
\\

144. & \eref{144}
     & B.~G. 3,  17, 3 
     & B.~G. 3,  15, 7 
\\

10.  & \eref[1]{10}
     & Eun.  511 
     & Eun.  510 
\\

145. & \eref{145}
     & B.~G. 3,  22, 2
     & B.~G. 3,  20, 2
\\

130. & \eref{130}
     & B.~G. 5, 28,  4 
     & B.~G. 5, 28,  3 
\\

146. & \eref{146}
     & B.~G. 1, 42,  4 
     & B.~G. 1, 42,  5 
\\

147. & \eref{147}
     & B.~G. 2,  25, 3
     & B.~G. 2,  24, 3
\\

117. & \eref{117}
     & B.~G. 2,  21, 6
     & B.~G. 2,  20, 6
\\

148. & \eref{148}
     & B.~G. 1, 3,  4 
     & B.~G. 1, 3,  3 
\\

149. & \eref{149}
     & And.  581 
     & And.  582 
\\

150. & \eref{150}
     & B.~G. 3,  21, 1
     & B.~G. 3,  19, 1
\\

12.  & \eref[\emph{a}]{12}
     & Am. 1,  139 
     & Am. 1,  39 
\\

151. & \eref{151}
     & B.~G. 2,  27, 3
     & B.~G. 2,  26, 3
\\

106. & \eref{106}
     & B.~G. 2, 10,  15 
     & B.~G. 2, 10,  5 
\\

152. & \eref{152}
     & B.~G. 3,  17, 4 
     & B.~G. 3,  15, 8 
\\

119. & \eref{119}
     & B.~G. 2,  35, 4
     & B.~G. 2,  34, 4
\\

153. & \eref{153}
     & B.~G. 1, 4, 3  
     & B.~G. 1, 4, 3 –4 
\\

154. & \eref{154}
     & B.~G. 2,  23, 1
     & B.~G. 2,  22, 1
\\

155. & \eref{155}
     & B.~G. 2,  22, 1
     & B.~G. 2,  21, 1
\\

104. & \eref{104}
     & B.~G. 2,  25, 1
     & B.~G. 2,  24, 1
\\

15.  & \eref[I]{15}
     & Plin.\ N.~H. 2,  139 
     & Plin.\ N.~H. 2,  57 
\\

156. & \eref{156}
     & B.~G. 2,  20, 2 
     & B.~G. 2,  19, 1 
\\

16.  & \eref[1]{16}
     & B.~G. 2,  31, 4 
     & B.~G. 2,  30, 4 
\\

121. & \eref{121}
     & B.~G. 2,  35, 4
     & B.~G. 2,  34, 4
\\

157. & \eref{157}
     & B.~G. 3,  15, 4 
     & B.~G. 3,  14, 13 
\\

158. & \eref{158}
     & B.~G. 3,  \hbox{15, 3} 
     & B.~G. 3,  \hbox{14, 12} 
\\

17.  & \eref[3]{17}
     & Eun.  729 
     & Eun.  728 
\\

18.  & \eref[3]{18}
     & Ad.  103 
     & Ad.  104 
\\

159. & \eref{159}
     & B.~G. 2,  19, 1
     & B.~G. 2,  18, 1
\\

160. & \eref{160}
     & B.~G. 1, 3,  4 
     & B.~G. 1, 3,  3 
\\

161. & \eref{161}
     & B.~G. 2,  26, 5
     & B.~G. 2,  25, 5
\\

162. & \eref{162}
     & B.~G. 5, 20,  3 
     & B.~G. 5, 20,  2 
\\

163. & \eref{163}
     & Lig. 12,  38 
     & Lig. 12,  37 
\\

93.  & \eref{93}
     & B.~G. 1, 3,  8 
     & B.~G. 1, 3,  7 
\\

19.  & \eref{19}
     & Tusc. 1, 12,  27 
     & Tusc. 1, 12,  26 
\\

164. & \eref{164}
     & B.~G. 2,  15, 5
     & B.~G. 2,  14, 5
\\

86.  & \eref{86}
     & B.~G. 3,  29, 2
     & B.~G. 3,  27, 2
\\

20.  & \eref{20}
     & Hec.  799 
     & Hec.  801 
\\

165. & \eref{165}
     & B.~G. 2,  29, 1
     & B.~G. 2,  28, 1
\\

166. & \eref{166}
     & B.~G. 1, 42,  3 
     & B.~G. 1, 42,  4 
\\

118. & \eref{118}
     & B.~G. 2,  35, 3
     & B.~G. 2,  34, 3
\\

167. & \eref{167}
     & B.~G. 1,  19, 3
     & B.~G. 1,  42, 6
\\

122. & \eref{122}
     & B.~G. 3,  18, 6
     & B.~G. 3,  16, 6
\\

168. & \eref{168}
     & B.~G. 2,  16, 5
     & B.~G. 2,  15, 5
\\

91.  & \eref{91}
     & B.~G. 1, 20,  6 
     & B.~G. 1, 20,  5 
\\

169. & \eref{169}
     & Cat. 1,  5, 13
     & Cat. 1,  6, 13
\\

170. & \eref{170}
     & Cat. 1, 6,  16 
     & Cat. 1, 6,  15 
\\

171. & \eref{171}
     & B.~G. 2,  25, 2
     & B.~G. 2,  24, 2
\\

172. & \eref{172}
     & B.~G. 2,  20, 1
     & B.~G. 2,  19, 1
\\

173. & \eref{173}
     & B.~G. 2,  32, 4
     & B.~G. 2,  31, 4
\\

24.  & \eref[\emph{a}]{24}
     & Plin.\ N.~H. 5,  43 
     & Plin.\ N.~H. 5,  14 
\\

25.  & \eref[\emph{a}]{25}
     & Plin.\ N.~H. 5,  43 
     & Plin.\ N.~H. 5,  14 
\\

97.  & \eref{97}
     & B.~G. 1, 24,  5 
     & B.~G. 1, 24,  4 
\\

174. & \eref{174}
     & B.~G. 1, 3,  5 
     & B.~G. 1, 3,  4 
\\

175. & \eref{175}
     & B.~C. 3, 19,  4 
     & B.~C. 3, 19,  5 
\\

98.  & \eref{98}
     & B.~G. 1, 35,  5 
     & B.~G. 1, 35,  3 
\\

108. & \eref{108}
     & B.~G. 2,  19, 4
     & B.~G. 2,  18, 4
\\

176. & \eref{176}
     & B.~G. 1, 42,  4 
     & B.~G. 1, 42,  5 
\\

177. & \eref{177}
     & B.~G. 3,  24, 2
     & B.~G. 3,  22, 2
\\

26.  & \eref[\emph{a}]{26}
     & Hec.  645 
     & Hec.  646 
\\

178. & \eref{178}
     & B.~G. 2,  15, 3
     & B.~G. 2,  14, 3
\\

27.  & \eref[\emph{a}]{27}
     & Eun.  560 
     & Eun.  559 
\\

28.  & \eref{28}
     & Plin.\ N.~H. 7,  117 
     & Plin.\ N.~H. 7,  51 
\\

179. & \eref{179}
     & B.~G. 2,  32, 4
     & B.~G. 2,  31, 4
\\

180. & \eref{180}
     & B.~G. 1, 52,  1 
     & B.~G. 1, 52,  2 
\\

181. & \eref{181}
     & B.~G. 3,  25, 2
     & B.~G. 3,  23, 2
\\

182. & \eref{182}
     & B.~G. 3,  18, 3
     & B.~G. 3,  16, 3
\\

183. & \eref{183}
     & B.~G. 1, 7,  5 
     & B.~G. 1, 7,  4 
\\

113. & \eref{113}
     & B.~G. 2,  19, 2
     & B.~G. 2,  18, 2
\\

90.  & \eref{90}
     & B.~G. 1, 15,  5 
     & B.~G. 1, 15,  3 
\\

184. & \eref{184}
     & B.~G. 2,  16, 1
     & B.~G. 2,  15, 1
\\

185. & \eref{185}
     & B.~G. 3,  17, 1 
     & B.~G. 3,  15, 5 
\\

186. & \eref{186}
     & B.~G. 1, 42,  4 
     & B.~G. 1, 42,  5 
\\

96.  & \eref{96}
     & B.~G. 1, 6,  4 
     & B.~G. 1, 6,  3 
\\

112. & \eref{112}
     & B.~G. 2,  20, 3 
     & B.~G. 2,  19, 2 
\\

187. & \eref{187}
     & Leg.\ Agr. 1, 8,  26 
     & Leg.\ Agr. 1, 8,  101 
\\

188. & \eref{188}
     & B.~G. 1, 2,  4 
     & B.~G. 1, 2,  5 
\\

110. & \eref{110}
     & B.~G. 2,  19, 8
     & B.~G. 2,  18, 8
\\

189. & \eref{189}
     & B.~G. 7,  54, 4 
     & B.~G. 7,  54, 3 
\\

133. & \eref{133}
     & Mil.  18, 49
     & Mil.  19, 49
\\

190. & \eref{190}
     & B.~G. 5, 40,  5 
     & B.~G. 5, 40,  6 
\\

191. & \eref{191}
     & B.~G. 1, 42,  5 
     & B.~G. 1, 42,  6 
\\

109. & \eref{109}
     & B.~G. 2,  19, 6
     & B.~G. 2,  18, 6
\\

129. & \eref{129}
     & B.~G. 3,  26, 4
     & B.~G. 3,  24, 4
\\

192. & \eref{192}
     & B.~G. 2,  29, 3
     & B.~G. 2,  28, 3
\\

193. & \eref{193}
     & Ecl. 1,  80 
     & Ecl. 1,  81 
\\

194. & \eref{194}
     & B.~G. 5, 7,  3 
     & B.~G. 5, 7,  2 
\\

131. & \eref{131}
     & Pomp.  2, 7
     & Pomp.  3, 7
\\

128. & \eref{128}
     & B.~G. 3,  28, 4
     & B.~G. 3,  26, 4
\\

89. & \eref{89}
     & B.~G. 1, 13,  6 
     & B.~G. 1, 13,  4 
\\

195. & \eref{195}
     & B.~G. 2,  27, 1
     & B.~G. 2,  26, 1
\\

30.  & \eref[\emph{b}]{30}
     & Ov.\ Met. 14,  665 
     & Ov.\ Met. 14,  819 
\\

196. & \eref{196}
     & B.~C. 3, 89,  3 
     & B.~C. 3, 89,  4–5 
\\

87.  & \eref{87}
     & B.~G. 1, 10,  5 
     & B.~G. 1, 10,  4 
\\

197. & \eref{197}
     & B.~G. 2,  13, 2 
     & B.~G. 2,  12, 7 
\\

198. & \eref{198}
     & B.~G. 2,  30, 4
     & B.~G. 2,  29, 4
\\

120. & \eref{120}
     & B.~G. 2,  35, 4
     & B.~G. 2,  34, 4
\\

199. & \eref{199}
     & B.~G. 3, 17,  5 
     & B.~G. 3, 17,  9 
\\

32.  & \eref[\emph{a}]{32}
     & Div. 2, 68,  14 
     & Div. 2, 68,  140 
\\

114. & \eref{114}
     & B.~G. 2,  20, 3 
     & B.~G. 2,  19, 2 
\\

200. & \eref{200}
     & Aen. 3,  161 
     & Aen. 3,  162 
\\

33.  & \eref[\emph{a}]{33}
     & Hec.  190 
     & Hec.  191 
\\

123. & \eref{123}
     & B.~G. 3,  18, 6
     & B.~G. 3,  16, 6
\\

201. & \eref{201}
     & B.~G. 1, 3,~ 8 
     & B.~G. 1, 3,~ 7 
\\

202. & \eref{202}
     & B.~G. 2,  27, 1
     & B.~G. 2,  26, 1
\\

115. & \eref{115}
     & B.~G. 2,  21, 5
     & B.~G. 2,  20, 5
\\

203. & \eref{203}
     & B.~G. 1, 28,  4 
     & B.~G. 1, 28,  3 
\\

31. & \eref[\emph{b}]{31}
    & Tac.
    & Tac.\ Ann.
\\

204. & \eref{204}
     & Pomp. 8,  20 
     & Pomp. 8,  20–21 
\\

205. & \eref{205}
     & Cat. 1,  5, 13
     & Cat. 1,  6, 13
\\

206. & \eref{206}
     & Ov.\ Met. 3,  656 
     & Ov.\ Met. 3,  654 
\\

35.  & \eref[1]{35}
     & Ad.  901 
     & Ad.  905 
\\

207. & \eref{207}
     & Aen. 2,  324 
     & Aen. 2,  325 
\\

208. & \eref{208}
     & B.~G. 5, 40,  5 
     & B.~G. 5, 40,  6 
\\

92.  & \eref{92}
     & B.~G. 1, 20,  6 
     & B.~G. 1, 20,  5 
\\

88.  & \eref{88}
     & B.~G. 1, 13,  12 
     & B.~G. 1, 13,  2 
\\

36.  & \eref[1]{36}
     & Rud.  828 
     & Rud.  833 
\\

209. & \eref{209}
     & B.~G. 3,  18, 4
     & B.~G. 3,  16, 4
\\

210. & \eref{210}
     & Sen.\ Med.  378 
     & Sen.\ Med.  375 
\\

211. & \eref{211}
     & B.~G. 1, 31,  15 
     & B.~G. 1, 31,  14 
\\

212. & \eref{212}
     & B.~G. 3,  24, 1
     & B.~G. 3,  22, 1
\\

38.  & \eref[4\emph{a}, n.~1]{38}
     & Plin.\ N.~H. 14,  37 
     & Plin.\ N.~H. 14,  11 
\\

124. & \eref{124}
     & B.~G. 3,  18, 7
     & B.~G. 3,  16, 7
\\

213. & \eref{213}
     & Trin.  588 
     & Trin.  587–588 
\\

214. & \eref{214}
     & And.  791 
     & And.  792 
\\

42.  & \eref[2]{42}
     & Ad.  122 
     & Ad.  123 
\\

105. & \eref{105}
     & B.~G. 2, 25
     & B.~G. 2, 24
\\

215. & \eref{215}
     & Heaut.  1066 
     & Heaut.  1067 
\\

116. & \eref{116}
     & B.~G. 2,  21, 5
     & B.~G. 2,  20, 5
\\

216. & \eref{216}
     & B.~G. 2,  27, 1
     & B.~G. 2,  26, 1
\\

111. & \eref{111}
     & B.~G. 2,~ 21,~3
     & B.~G. 2,~ 20,~3
\\

217. & \eref{217}
     & Sen. 16,  58 
     & Sen. 16,  57 
\\

218. & \eref{218}
     & B.~G. 1, 40,  15 
     & B.~G. 1, 40,  1…15 
\\

219. & \eref{219}
     & B.~G. 1, 40,  8 
     & B.~G. 1, 40,  1…8 
\\

220. & \eref{220}
     & B.~G. 1, 44,  8 
     & B.~G. 1, 44,  1…8 
\\

221. & \eref{221}
     & B.~G. 1, 43,  6 
     & B.~G. 1, 43,  6–7 
\\

222. & \eref{222}
     & B.~G. 2,  20, 1
     & B.~G. 2,  19, 1
\\

223. & \eref{223}
     & Aen. 5,  541 
     & Aen. 5,  542 
\\

224. & \eref{224}
     & B.~G. 4, 12,  6 
     & B.~G. 4, 12,  5 
\\

225. & \eref{225}
     & Leg.\ Agr. 1, 7,  23 
     & Leg.\ Agr. 1, 7,  22–23 
\\

226. & \eref{226}
     & Cat. 1, 8,  2 
     & Cat. 1, 8,  21 
\\

227. & \eref{227}
     & Ov.\ Pont. 2, 9,  49 
     & Ov.\ Pont. 2, 9,  47–48 
\\

43.  & \eref{43}
     & Amph.  644 
     & Amph.  642 
\\

228. & \eref{228}
     & B.~G. 1, 7,  4 
     & B.~G. 1, 7,  3 
\\

44.  & \eref[ftn.~1]{44}
     & Hec.  368 
     & Hec.  369 
\\

126. & \eref{126}
     & B.~G. 3,  19, 6
     & B.~G. 3,  17, 6
\\

229. & \eref{229}
     & B.~G. 3,  16, 2
     & B.~G. 3,  15, 2
\\

45.  & \eref[\emph{a}]{45}
     & Att. 13,  30, 1 
     & Att. 13,  29, 3 
\\

230. & \eref{230}
     & Aen. 2,  668 
     & Aen. 2,  669–670 
\\

231. & \eref{231}
     & Aen. 2,  546 
     & Aen. 2,  547 
\\

232. & \eref{232}
     & B.~G. 2, 9,  4 
     & B.~G. 2, 9,  5 
\\

46.  & \eref[7]{46}
     & Verr. 4, 40,  88 
     & Verr. 4, 40,  87 
\\

233. & \eref{233}
     & B.~G. 3, 1,  3 
     & B.~G. 3, 1,  2 
\\

234. & \eref{234}
     & B.~C. 1, 4,  4 
     & B.~C. 1, 4,  5 
\\

235. & \eref{235}
     & B.~G. 3,  21, 3
     & B.~G. 3,  19, 3
\\

236. & \eref{236}
     & B.~G. 2,  15, 1
     & B.~G. 2,  14, 1
\\

237. & \eref{237}
     & B.~G. 1, 7,  4 
     & B.~G. 1, 7,  3 
\\

94.  & \eref{94}
     & B.~G. 1, 3,  8 
     & B.~G. 1, 3,  7 
\\

238. & \eref{238}
     & B.~G. 5, 37,  5 
     & B.~G. 5, 37,  4–5 
\\

239. & \eref{239}
     & B.~G. 2,  20, 1
     & B.~G. 2,  19, 1
\\

240. & \eref{240}
     & B.~G. 1, 15,  3 
     & B.~G. 1, 15,  2 
\\

241. & \eref{241}
     & B.~G. 1, 51,  2 
     & B.~G. 1, 51,  3 
\\

242. & \eref{242}
     & B.~G. 1, 2,  4 
     & B.~G. 1, 2,  5 
\\

107. & \eref{107}
     & B.~G. 2,  17, 4
     & B.~G. 2,  16, 4
\\

127. & \eref{127}
     & B.~G. 3,  19, 6
     & B.~G. 3,  16, 6
\\

243. & \eref{243}
     & B.~G. 3,  24, 5
     & B.~G. 3,  22, 5
\\

245. & \eref{245}
     & B.~G. 1, 11,  3 
     & B.~G. 1, 11,  2 
\\

244. & \eref{244}
     & Aen. 9,  241 
     & Aen. 9,  240 
\\

246. & \eref{246}
     & B.~G. 1, 3,  6 
     & B.~G. 1, 3,  5 
\\

49.  & \eref[1]{49}
     & Arch.  1,  3, 4
     & Arch.   3, 4
\\

50.  & \eref[1]{50}
     & Pomp. 1,  1,  2
     & Pomp. 1,   2
\\

247. & \eref{247}
     & Cat. 3, 10,  14 
     & Cat. 3, 10,  24 
\\

248. & \eref{248}
     & Cat. 1,  5, 13 
     & Cat. 1,  6, 12 
\\

249. & \eref{249}
     & B.~G. 3,  26, 2
     & B.~G. 3,  24, 2
\\

132. & \eref{132}
     & Mil.  5, 10
     & Mil.  4, 10
\\

95.  & \eref{95}
     & B.~G. 1, 5,  4 
     & B.~G. 1, 5,  3 
\\

250. & \eref{250}
     & B.~G. 6, 16,  3 
     & B.~G. 6, 16,  2 
\\

99.  & \eref{99}
     & B.~G. 2,  25, 1–2
     & B.~G. 2,  24, 1–2
\\

51.  & \eref[8]{51}
     & Hec.  286 
     & Hec.  287 
\\

\end{emendations}

\endinput


%% This work is licensed under a Creative Commons
%% Attribution-NonCommercial 4.0 International License.
%% http://creativecommons.org/licenses/by-nc/4.0/
%% 
%% David M. Jones, July 2016

\Unnumbered{Variations in the Text}
\label{variations}
\markboth{Variations in the Text}{Variations in the Text}
\markthird{}

\thispagestyle{dropfolio}

\contentsentry{B}{Variations in the Text}

All of the variations between versions A and~B of the text are listed
below, but they are described with respect to the wording chosen for
the current edition.  For example, in the first item, it is understood
that the paragraph referred to was added in version~B.

\begin{variations}

\item[p.~\pagelink{vi}, ll.~7–10]

Version~A lacks the paragraph beginning “The views upon.”

\item[p.~\pagelink{vi}, l.~33]

Version~A lacks “proof-reading and suggestions, and also.”

\item[{\xref[3.]{100}, l~2}]

Version~A lacks “Similarly (rarely), \latin{diī} for \latin{diēī}.”

\item[\xref{122}, l.~11]

Version~A has “(\latin{iuvenior} late)” in place of “[\latin{minor
    nātū}]” and “\latin{nātū minimus}” in place of “\latin{minor
  nātū}.”

\item[\xref{122}, l.~12]

Version~A lacks “[\latin{maior nātū}]” and reads “\latin{nātū
  maximus}” in place of “\latin{maximus nātū}.”

\item[{\xref[\emph{a}]{22}, l.~2}]

Version~A lacks “, and the auxiliary \latin{est} (\latin{sunt},
etc.).”

\item[{\xref[\emph{b}]{234}, l.~2}]

Version~A lacks “and \apud{}{1, 308}.”

\item[{\xref[5, \emph{c}]{240}}]

Version~A lacks this paragraph.

\item[{\xref[5, \emph{d}]{240}, ll.~2–3}]

Version~B omits the sentence “Such \emph{intermediate} (or
\emph{semi\hyphenbreak abstract}) nouns are usually classed as Concrete.”

\item[{\xref[3, \emph{a}]{264}}]

Version~A reads
\begin{quote}
Similarly \latin{aliēnus}, \english{belonging to another}, gains the
meaning \english{unfavorable}.  Thus \latin{aliēnō locō}, \english{in
  an unfavorable place}; \apud{B.~G.}{1, 15, 2}.
\end{quote}
Version~B reads
\begin{quote}
Similarly \latin{noster}, \english{our}, may have the meaning
\english{favorable}, and \latin{aliēnus}, \english{belonging to
  another}, the meaning \english{unfavorable}.
\end{quote}

\item[{\xref[3]{274}}]

Version~A reads
\begin{quote}
\latin{Is} or \latin{is quidem}, in combination with various
connectives (\latin{et is}, \latin{atque is}, \latin{isque}, \latin{et
  is quidem}, \latin{nec is}, \latin{neque is}, etc.), is used\ellipsis
\end{quote}

Version~B reads
\begin{quote}
\latin{Is} or \latin{is quidem}, and \latin{ille} or \latin{ille
  quidem}, in combination with various connectives (\latin{et},
\latin{atque}, \latin{nec}, etc.), are used\ellipsis
\end{quote}

\item[{\xref[1]{284}}]

Version~B omits the example beginning “\latin{sunt hūmānissimī}” and
the\linebreak
phrase “(Indefinite Antecedent.)”\ at the end of the second
example.

\item[{\xref[\emph{a}]{284}}]

Version~A lacks this entire subsection.

\item[{\xref[2]{284}}]

Version~B omits the example beginning “\latin{habētis quam}.”

\item[{\xref[3]{284}}]

Version~A reads
\begin{quote}
(in English idiom) \english{the bridge at Geneva};
\end{quote}
in place of Version~B's
\begin{quote}
\english{the bridge \emph{(which was)} at Geneva} (in English idiom,
\english{the bridge at Geneva});
\end{quote}

\item[{\xref[6]{284}}]

Version~B omits “the clause containing the Antecedent.”

\item[{\xref[\emph{c}]{295}}]

Version~A lacks “The poets extend the list.”

\item[{\xref{327}, ll.~1–2}]

Version~A reads
\begin{quote}
The Romans avoided putting an Appositive word directly before a
Relative, preferring to attach it \emph{to the Relative itself}.
\end{quote}

\item[{\xref[\emph{c}]{346}, ll.~4–5}]

Version~B omits the second example (“\latin{quōs omnīs}”);
Version~A lacks the third example (“\latin{reliquīs Gallīs}”).

\item[{\xref[\emph{a}, 2]{352}, l.~2}]

Version~A lacks “(\latin{Miseror} takes the Accusative.)”

\item[{\xref[\emph{c}]{354}, ll.~5, 7}]

Version~B omits the third (“\latin{poenae sēcūrus}”) and fifth
(“\latin{ēreptae virginis īrā}”) examples.

\item[{\xref[\emph{d}]{354}}]

Version~A lacks this entire subsection.

\item[{\xref[\emph{a}]{361}, ll.~1–2}]

Version~A reads
\begin{quote}
\textbf{Later Freer Dative of the Concrete Object for Which.}  The
poets and later writers use the construction of the Concrete Object
more boldly, even attaching it directly to nouns.
\end{quote}

Version~B reads
\begin{quote}
\textbf{Later Freer Dative of the Object for Which.}  The poets and
later writers use the construction of the Object for Which more
boldly, even attaching it directly to nouns.
\end{quote}

\item[{\xref[5]{364}, l.~3}]

Version~A lacks “Similarly \latin{aequō} in poetry.”

\item[{\xref[III]{387}, footnote, l.~2}]

Version~A lacks “The same use appears with \latin{ecquid}, \latin{sī
  quid}, and \latin{nē quid}.”

\item[{\xref[\emph{a}]{388}, note}]

Version~A reads
\begin{quote}
From such combinations arose the free use of \latin{quid} in the sense
of \english{why}, as in \latin{quid tacēs?} \english{why are you
  silent?} \apud{Cat.}{1, 4, 8}.
\end{quote}

Version~B reads
\begin{quote}
Hence arose the use of \latin{quid} in the sense of \english{why}, and
of \latin{quod} in phrases like \latin{quod sī}, \english{but if}
(touching which matter, if).
\end{quote}

\item[{\xref[\emph{a}]{426}, footnote}]

Version~A lacks “\latin{pelagō}.”

Version~B omits “\latin{stagnō}.”

\item[{\xref[\emph{a}]{431}}]

Version~A reads “occasionally” in place of Version~B's “may also.”

\item[{\xref[3, \emph{b}]{438}, l.~1}]

Version~A reads “See~\xref[4, \emph{b}]{406}.”

\item[{\xref[1]{464}, footnote, l.~1}]

Version~A lacks “(in the finite verb).”

\item[{\xref[2]{467}, ll.~1–2}]

Version~A reads
\begin{quote}
An act can be thought of as a whole only if looked at \emph{without}
reference to any particular time.  Hence the aoristic tenses are
\emph{Absolute}.
\end{quote}

Version~B reads
\begin{quote}
An act thought of as a whole (i.e.\ aoristically) may be looked at
either without, or with, reference to a particular time, i.e.\ either
\emph{Absolutely} or \emph{Relatively}.
\end{quote}

In addition, Version~A lacks all of \xref[2, \emph{a}]{467}.

\item[{\xref[4, \emph{a}]{470}, l.~2–3}]

Version~A lacks “(generally).”  Version~B omits “in” three times:
before “Consecutive,” before “Causal-Adversative,” and before
“\latin{quīn}-Clauses.”

\item[{\xref[\emph{b}]{477}}]

Version~A reads
\begin{quote}
The relative tenses of the Indicative all express \emph{situation}.
So do the relative tenses of the Subjunctive, when used with the same
force as the corresponding tenses of the Indicative.  When used with
future force, they may express either the idea of future (or
subsequent) \emph{situation}, or a mere \emph{aoristic} idea for
future (or subsequent) time.
\end{quote}

Version~B reads
\begin{quote}
The relative tenses of the Indicative all express \emph{situation};
the aoristic tenses of the Indicative do not (\xref[2,
  \emph{a}]{467}).

The Subjunctive tenses, when used with relative force, may express
either the idea of situation, or the aoristic idea.  Thus, either a
situation, or an act seen in summary, may be put as relatively future
to a past time.
\end{quote}

\item[{\xref{481}, column~2, l.~2}]

Version~A reads “for the reason that” in place of Version~B's “(I have
written) because.”

\item[{\xref{487}}]

Version~A reads
\begin{quote}
In several verbs the Present Perfect, Past Perfect, and Future Perfect
have come to express a present, past, or future \emph{state}.  Thus
\latin{nōvī}, (\english{have learned}) \english{know},
\latin{nōveram}, \english{knew}, \latin{nōverō}, \english{shall know},
\latin{cognōvī}, \english{know}, \latin{cōnsuēvī}, (\english{have
  formed the habit}) \english{am in the habit}, \latin{meminī},
(\english{have recollected}) \english{remember}, \latin{ōdī},
(\english{have come to dislike}) \english{hate}. Similarly
  \latin{coepī}, \english{begin}.
\end{quote}

Version~B reads
\begin{quote}
In several verbs the Present Perfect, Past Perfect, and Future Perfect
have come to express a present, past, or future \emph{state}.  Thus
\latin{nōvī}, (\english{have learned}) \english{know},
\latin{cōnsuēvī}, \english{am accustomed}, \latin{meminī},
\english{remember}, \latin{ōdī}, \english{hate}, \latin{coepī},
\english{begin}, etc.  Similarly, sometimes, in other verbs.  Thus
\latin{cōnstiterant}, \english{had taken their stand}, = \english{were
  standing}; \apud{B.~G.}{1, 24, 3}.
\end{quote}

\item[{\xref[3, \emph{a})]{502}, footnote~4}]

Version~A reads “these constructions” in place of Version~B's
“substantive clauses” and lacks “(likewise in clauses of purpose).”
Version~B omits “Thus \latin{ut nē sit impūne}, \apud{Mil.}{12, 31}.”

\item[{\xref[1]{507}}]

In the second example (“\latin{nunc est ille diēs}\ellipsis”), Version~B
omits “Similarly, though in indirect discourse, \latin{diem quō
  condant}, \apud{Aen.}{7, 145}.”

Version~B shortens the third example to
\begin{quote}
\latin{nāscētur Troiānus, fāmam quī terminet astrīs},
\english{there will be born a Trojan, who shall \emph{(prophetic, =
    will)} make the stars the boundary of his fame};
\apud{Aen.}{1, 286}.  (A Trojan of what kind?  A Trojan that
shall.\dots\ Cf.\ \latin{quae verteret}, expressing a \emph{past}
Anticipation, \apud{Aen.}{1, 20}.)
\end{quote}

Version~B omits the example beginning “\latin{venient annīs saecula}.”

\item[{\xref[\emph{a}]{507}}]

Version~A has the phrase “almost completely” between “has” and
“driven” instead of after “Subjunctive.”

\item[{\xref[2]{507}, ll.1–2}]

Version~A reads “In \term{Substantive Clauses} with \latin{ut}, after
verbs of \emph{expecting}.”

\item[{\xref[2, \emph{b}]{507}}]

Version~A lacks this subsection.

\item[{\xref[4, \emph{b}]{507}, note}]

Version~A reads
\begin{quote}
Since an event forestalled is one which the main actor tries to make
\emph{impossible}, the Anticipatory Subjunctive of \latin{possum}
(with the Infinitive) is sometimes used in this construction (as in
\apud{B.~G.}{6, 3, 2}, \latin{priusquam convenīre possent}), in place
of the simple verb in the Subjunctive (\latin{priusquam convenīrent}).
\end{quote}

\item[{\xref[4, \emph{d}]{507}, l.~5}]

Version~A lacks “Cf.\ \latin{prius quam ut}, \apud{Lig.}{12, 34}.”

\item[{\xref[1, \emph{c}]{519}, l.~2}]

Version~A reads “Substantive Volitive Clause” in place of Version~B's
“an Infinitive or Volitive Clause.”

\item[{\xref[3, \emph{b}]{521}, l.~2}]

Version~A lacks “, or imply a negative.”

\item[{\xref[\emph{a}]{535}, note~3}]

Version~A lacks note~3 (but see the next item).

\item[{\xref[\emph{b}]{535}}]

Version~A contains the following note here:
\begin{quote}
\begin{note}[Note]

By a natural confusion, \latin{dīcō} is sometimes put in the
Subjunctive in a \latin{quod}-Clause of Reason.
\begin{examples}

\latin{rediit quod sē oblītum nesciō quid dīceret},
\english{he came back, because he said he had forgotten something}
(properly \latin{quod oblītus esset}, \english{because}, as he said,
\english{he had forgotten});
\apud{Off.}{1, 13, 40}.
Similarly \latin{quod exīstimārent}; \apud{B.~G.}{1, 23, 3}.

\end{examples}

\end{note}
\end{quote}

\item[{\xref[\emph{a}]{550}, l.~6}]

Version~A lacks “(See also \xref[\emph{d}]{524}.)”

\item[{\xref{555}, footnote~1, l.~1}]

Version~A lacks “\latin{id maesta est},”.

\item[{\xref{558}, ll.~1–2}]

Version~A lacks “or \latin{simul atque},” but has “(the less common
usage).”

\item[{\xref{577}, ll.~2–3}]

Version~B omits “In what follows, the two kinds will be treated
together.”

\item[{\xref[\emph{a}]{577}}]

Version~A lacks this subsection.

\item[{\xref[\emph{e}]{585}}]

Version~A reads
\begin{quote}
Verbs or phrases of this class suggesting that the action is
\emph{wanted} or \emph{urged} may also take a Volitive Substantive
Clause (\xref[3, \emph{c}]{502}, and lists).
\end{quote}

Version~B reads
\begin{quote}
Verbs or phrases of this class suggesting that the action is
\emph{wanted} or \emph{urged} may also take a Subjunctive Substantive
Clause.
\end{quote}

\item[{\xref[3]{598}}]

Version~A reads
\begin{quote}
The later poets use the Infinitive occasionally as a Substantive with
a Verb, or after certain Prepositions governing the Accusative.
\end{quote}

Version~B reads
\begin{quote}
The later writers, especially the poets, use the Infinitive
occasionally as a mere Substantive depending upon a Verb, or in the
Accusative after certain prepositions.
\end{quote}

\item[{\xref[2]{605}, l.~2}]

Version~A lacks “(and many others in poetry)”.

\item[{\xref[III]{625}}]

Version~A reads “the separation of connected words” in place of
Version~B's “postponement”.

\item[{\xref[11]{623}}]

Version~B omits “till the end of the sentence is reached”.

\item[{\xref[12]{623}}]

Version~A reads
\begin{quote}
\english{if anything} (for \english{any disaster}) \english{should
  happen to the Romans}
\end{quote}

Version~B reads
\begin{quote}
\english{if anything should happen to the Romans} (instead of
\english{if they should be defeated})
\end{quote}

\item[{\xref[14]{623}, ll.~4–5}]

Version~B shortens the parenthetical explanation to
\begin{quote}
(The ship is the state, the billows the civil wars, etc.)
\end{quote}

\item[{\xref[20]{623}}]

Version~B replaces “use of words the sound of which corresponds with
the thing signified” by “matching of sound to sense”.

\item[{\xref[21]{623}}]

Version~A lacks the definition of \term{Figūra Etymologica}.

\item[{\xref[2]{652}}]

Version~A lacks “, especially in the caesura”.

\item[Index, “Absolute tenses”]

Version~A has “\xref[2]{467}, \xref{478}”.

\item[Index, “Deprecated act”]

Version~A has “in subj.\ w.\ \latin{antequam}, etc., \xref[4,
  \emph{d}\)]{507}.”

\item[Index, “\latin{dīcō}”]

Version~A has “\xref[2, \emph{b}, n.]{535}” in place of Version~B's
“\xref[2, \emph{a}, n.~3]{535}”.

\item[Index, “Tenses”]

In the last line, Version~A has “\xref{494}” in place of Version~B's
“\xref[\emph{a}]{577}”.

\end{variations}

\endinput


%% This work is licensed under a Creative Commons
%% Attribution-NonCommercial 4.0 International License.
%% http://creativecommons.org/licenses/by-nc/4.0/
%% 
%% David M. Jones, July 2016

\begin{colophon}

This edition was typeset using \XeTeX\ and \LaTeX\ with
additional macros by the editor.

The primary typeface is Brill, commissioned by Brill Publishers and
designed by John Hudson of Tiro Typeworks.  These fonts are freely
available for non-commercial use at
\url{http://www.brill.com/about/brill-fonts}.

Additional characters for notating verse scansion, especially in the
discussion of prosody in part~V, are taken from from ALPHABETUM
Unicode by Juan-José Marcos.

% The discussion of prosody in part~V required additional characters not
% included in the Brill fonts.  Characters for notating verse scansion
% are taken from ALPHABETUM Unicode by Juan-José Marcos.  The half and
% quarter notes in \xref{637} and~\xref{639} come from Sonata Std by
% Adobe Systems.  Finally, a small number of glyphs were drawn from
% Donald E. Knuth's Computer Modern typefaces.

\end{colophon}

\endinput


\scrollmode
\end{document}
